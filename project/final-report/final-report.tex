\documentclass{article}
    % General document formatting
    \usepackage[margin=0.7in]{geometry}
    \usepackage[parfill]{parskip}
    \usepackage[utf8]{inputenc}

    \usepackage[colorlinks]{hyperref}
    \usepackage{caption}
    \usepackage{subcaption}
    \usepackage{array}
    \usepackage{amsmath}
    \usepackage{cleveref}
    \usepackage{pgf}
    \usepackage{import}

\begin{document}

\title{Final Report on the Topic of Climate Change
for the Cource Methods of Advanced Data Engineering at FAU}
\author{Luca Preibsch}
\date{\today}

% \begin{center}
% Luca Preibsch - \today
% \end{center}

\maketitle
\section*{Introduction}

Climate change is a huge challenge worldide and has gained more attention in recent.
The global cause to stop climate change or at least slow it down often focusses on global warming, like the
\href{https://unfccc.int/process-and-meetings/the-paris-agreement}{paris climate agreement},
which legally binds the 196 parties "to limit the temperatute increase to 1.5°C above pre-industrial levels".

Among various other contributors to global warming, burning fossil fuels for energy consumption stands out as
a primary factor to this day.
When burning fossil fuels, substantial amounts of greenhouse gases are released into the atmosphere,
which increase the so called greenhouse effect, in turn accelerating global warming and climate change.
Facing the amount of greenhouse gases produced by the energy sector, many countries - including countries of
the European Union (EU) - intensified their efforts to transition towards more sustainable energy sources.

This report aims to explore the aforementioned relationship between energy consumption and net greenhouse gas
emissions across countries of the EU, in order to show how they are interconnected.
Furthermore this report is set to discover how this connection is influenced by the share of renewable energy sources
in the energy pool of a country. This might help to indicate, how effective the adoption of renewable energy sources is
in fighting climate change.

The resulting core questions are:
\begin{enumerate}
    \item How does the amount of energy consumed influence the net greenhouse gas emissions of European countries?
    \item And how is this influenced by the share of renewables in total energy?
\end{enumerate}

\section*{Used Data}
This report uses a merged dataset produces by a data pipeline, derived from three data sources,
all originating from \href{https://ec.europa.eu/eurostat}{Eurostat}.
The resulting dataset provides a comprehensive view of net greenhouse gas emissions, primary energy consumption
and the share of renewable energy sources across countries of the EU.

\subsection*{Structure and Meaning of the Dataset}
The dataset is structured as a SQLite database containing three tables, each covering one data source from Eurostat.
Each table includes the following three columns:
\begin{itemize}
    \item \textbf{geo}: The \href{https://www.destatis.de/Europa/EN/Country/Country-Codes.html}{ALPHA-2 country codes} representing each EU country.
    \item \textbf{year}: The year in which the data was recorded.
    \item \textbf{value}: The observed value for the specific metric in each table.
\end{itemize}

\subsubsection*{Net Greenhouse Gas Emissions}
This table contains the amount of greenhouse gases emitted per person in each country for the specified years.
The emissions data is measured in tonnes per capita.
An example of how the data is structured can be seen in table \ref{tab:emissions}.

\subsubsection*{Primary Energy Consumption}
This table shows the energy consumption per person in each country, measured in Tonnes of Oil Equivalent (TOE) per capita.
An example of how the data is structured can be seen in table \ref{tab:consumption}.

\subsubsection*{Share of Energy from Renewable Sources}
This table covers the proportion of energy consumed, which originates from renewable sources as declared by the EU,
highlighting the adoption of renewable energy sources in each country.
The data is measured in percentage of the total consumed energy.
An example of how the data is structured can be seen in table \ref{tab:share}.

\begin{figure}[h!]
    \centering
    \begin{subfigure}[b]{0.3\textwidth}
        \centering
        \begin{tabular}{c c c}
            geo & year & value \\
            \hline\hline
            AT & 1990 & 8.9 \\
            AT & 1991 & 8.4 \\
            AT & 1992 & 8.7 \\
            \dots & \dots & \dots
        \end{tabular}
        \caption{Net Greenhouse Gas Emissions}
        \label{tab:emissions}
    \end{subfigure}
    \hfill
    \begin{subfigure}[b]{0.3\textwidth}
        \centering
        \begin{tabular}{c c c}
            geo & year & value \\
            \hline\hline
            AT & 2000 & 3.43 \\
            AT & 2001 & 3.62 \\
            AT & 2002 & 3.62 \\
            \dots & \dots & \dots
        \end{tabular}
        \caption{Primary Energy Consumption}
        \label{tab:consumption}
    \end{subfigure}
    \hfill
    \begin{subfigure}[b]{0.3\textwidth}
        \centering
        \begin{tabular}{c c c}
            geo & year & value \\
            \hline\hline
            AT & 2004 & 22.553 \\
            AT & 2005 & 24.353 \\
            AT & 2006 & 26.276 \\
            \dots & \dots & \dots
        \end{tabular}
        \caption{Share of Renewable Sources}
        \label{tab:share}
    \end{subfigure}
    \caption{Dataset Table Structure}
\end{figure}

\subsection*{Compliance with Data Licenses}
All data used in this analysis is sourced from Eurostat and is subject to the
\href{https://ec.europa.eu/eurostat/about-us/policies/copyright}{Eurostat copyright notice},
which allows free re-use of data under obligations.

In order to comply, this report attributes Eurostat as the source of all data.
Furthermore, during the processing of the data sources by the pipeline, the source data was cleaned with certain
rows and columns being removed for relevance and consistency.

In detail the following modifications to the source data were performed:
\begin{itemize}
    \item Table Net Greenhouse Gas Emissions: all rows were removed, which did not cover the unit of measure "Total (Tonnes per capita)";
    all rows that covered other source sectors than "Total (excluding memo items, including international aviation)" were removed.
    \item Table Primary Energy Consumption: all rows were removed, that did not cover the unit of measure "Tonnes of oil equivalent (TOE) per capita".
    \item Table Share of Energy from Renewable Sources: all rows were removed, that did not cover the topic "Renewable energy sources".
    \item For all tables all columns except "geo", "TIME\_PERIOD" and "OBS\_VALUE" were removed.
    \item For all tables, the headers for "TIME\_PERIOD" and "OBS\_VALUE" were renamed to "year" and "value" respectively.
    \item For all tables, all rows were removed that did cover countries, which are not currently part of the \href{https://www.destatis.de/Europa/EN/Country/Country-Codes.html}{27 EU member states}.
\end{itemize}

\section*{Analysis of the Data}
% Zwar sind in verschiedenen Tabellen unterschiedlich viele Daten zu unterschiedlich vielen Jahren, aber ist nicht relevant,
% denn überflüssige Jahre werden durch Merge eliminiert
\subsection*{Correlation between Energy Consumption and Emissions}
\subsubsection*{Method Used}
In order to answer the first question considering the correlation between emissions and energy consumption, the data first had to be prepared:
The data on emissions, energy consumption and the share of renewables was extracted from the SQLite database and the datasets were merged on
country codes and years to create a unified dataset for analysis.
The approach then was to create a scatter plot for visualizing the relationship between energy consumption and the
greenhouse gas emissions. The plot includes data points for all countries and all years in order to give an overview
over the EU as a whole. The single dots are colored appropriate to a scale.
This scale was then included in the plot in the form of a colorbar as a legend.
In order to make the plot more readable, linear regression was used to create a linear function describing the data.
Also, to assess the strength of the two datasets correlation, the pearson correlation coefficient ($r$) was calculated.

Since the global plot shadows differences between countries, also plots for all single countries were created in order to get
insights into the workings of the relationship on a national level.

\subsubsection*{Results}
The resulting global plot can be seen in \cref{plt:global_consumption_vs_emissions}
The scatter plot backed by the regression line point to a positive correlation between energy consumption and greenhouse gas emissions.
Furthermore, the $r$-value is calculated to be .64, indicating a moderate to strong positive correlation.

Since the 27 single output plots would't fit the report, two example plots were chosen to highlight the differences, which can be observed between countries.
One example is Luxembourg (\cref{plt:LU_consumption_vs_emissions}) with a very strong correlation coefficient of $r=1$, another example is Austria
(\cref{plt:AT_consumption_vs_emissions}) with a mediocre $r=.56$. Another third example is
Slovenia, for which no positive correlation could be found with $r=-.46$.
Slovenia is an rare exception, since it is the only country with no positive correlation.
All in all, the median $r$-value over all countries is .86, which is even much higher than the correlation coefficient for the EU as a whole.

\begin{figure}
    \centering
    \resizebox{.7\textwidth}{!}{%% Creator: Matplotlib, PGF backend
%%
%% To include the figure in your LaTeX document, write
%%   \input{<filename>.pgf}
%%
%% Make sure the required packages are loaded in your preamble
%%   \usepackage{pgf}
%%
%% Also ensure that all the required font packages are loaded; for instance,
%% the lmodern package is sometimes necessary when using math font.
%%   \usepackage{lmodern}
%%
%% Figures using additional raster images can only be included by \input if
%% they are in the same directory as the main LaTeX file. For loading figures
%% from other directories you can use the `import` package
%%   \usepackage{import}
%%
%% and then include the figures with
%%   \import{<path to file>}{<filename>.pgf}
%%
%% Matplotlib used the following preamble
%%   \def\mathdefault#1{#1}
%%   \everymath=\expandafter{\the\everymath\displaystyle}
%%   
%%   \makeatletter\@ifpackageloaded{underscore}{}{\usepackage[strings]{underscore}}\makeatother
%%
\begingroup%
\makeatletter%
\begin{pgfpicture}%
\pgfpathrectangle{\pgfpointorigin}{\pgfqpoint{12.000000in}{8.000000in}}%
\pgfusepath{use as bounding box, clip}%
\begin{pgfscope}%
\pgfsetbuttcap%
\pgfsetmiterjoin%
\definecolor{currentfill}{rgb}{1.000000,1.000000,1.000000}%
\pgfsetfillcolor{currentfill}%
\pgfsetlinewidth{0.000000pt}%
\definecolor{currentstroke}{rgb}{1.000000,1.000000,1.000000}%
\pgfsetstrokecolor{currentstroke}%
\pgfsetdash{}{0pt}%
\pgfpathmoveto{\pgfqpoint{0.000000in}{0.000000in}}%
\pgfpathlineto{\pgfqpoint{12.000000in}{0.000000in}}%
\pgfpathlineto{\pgfqpoint{12.000000in}{8.000000in}}%
\pgfpathlineto{\pgfqpoint{0.000000in}{8.000000in}}%
\pgfpathlineto{\pgfqpoint{0.000000in}{0.000000in}}%
\pgfpathclose%
\pgfusepath{fill}%
\end{pgfscope}%
\begin{pgfscope}%
\pgfsetbuttcap%
\pgfsetmiterjoin%
\definecolor{currentfill}{rgb}{1.000000,1.000000,1.000000}%
\pgfsetfillcolor{currentfill}%
\pgfsetlinewidth{0.000000pt}%
\definecolor{currentstroke}{rgb}{0.000000,0.000000,0.000000}%
\pgfsetstrokecolor{currentstroke}%
\pgfsetstrokeopacity{0.000000}%
\pgfsetdash{}{0pt}%
\pgfpathmoveto{\pgfqpoint{0.786164in}{0.768110in}}%
\pgfpathlineto{\pgfqpoint{9.637233in}{0.768110in}}%
\pgfpathlineto{\pgfqpoint{9.637233in}{7.850000in}}%
\pgfpathlineto{\pgfqpoint{0.786164in}{7.850000in}}%
\pgfpathlineto{\pgfqpoint{0.786164in}{0.768110in}}%
\pgfpathclose%
\pgfusepath{fill}%
\end{pgfscope}%
\begin{pgfscope}%
\pgfpathrectangle{\pgfqpoint{0.786164in}{0.768110in}}{\pgfqpoint{8.851069in}{7.081890in}}%
\pgfusepath{clip}%
\pgfsetbuttcap%
\pgfsetroundjoin%
\definecolor{currentfill}{rgb}{0.187231,0.414746,0.556547}%
\pgfsetfillcolor{currentfill}%
\pgfsetfillopacity{0.700000}%
\pgfsetlinewidth{0.501875pt}%
\definecolor{currentstroke}{rgb}{1.000000,1.000000,1.000000}%
\pgfsetstrokecolor{currentstroke}%
\pgfsetstrokeopacity{0.700000}%
\pgfsetdash{}{0pt}%
\pgfpathmoveto{\pgfqpoint{3.362208in}{2.924158in}}%
\pgfpathcurveto{\pgfqpoint{3.375230in}{2.924158in}}{\pgfqpoint{3.387721in}{2.929332in}}{\pgfqpoint{3.396930in}{2.938541in}}%
\pgfpathcurveto{\pgfqpoint{3.406138in}{2.947749in}}{\pgfqpoint{3.411312in}{2.960240in}}{\pgfqpoint{3.411312in}{2.973263in}}%
\pgfpathcurveto{\pgfqpoint{3.411312in}{2.986286in}}{\pgfqpoint{3.406138in}{2.998777in}}{\pgfqpoint{3.396930in}{3.007985in}}%
\pgfpathcurveto{\pgfqpoint{3.387721in}{3.017193in}}{\pgfqpoint{3.375230in}{3.022367in}}{\pgfqpoint{3.362208in}{3.022367in}}%
\pgfpathcurveto{\pgfqpoint{3.349185in}{3.022367in}}{\pgfqpoint{3.336694in}{3.017193in}}{\pgfqpoint{3.327485in}{3.007985in}}%
\pgfpathcurveto{\pgfqpoint{3.318277in}{2.998777in}}{\pgfqpoint{3.313103in}{2.986286in}}{\pgfqpoint{3.313103in}{2.973263in}}%
\pgfpathcurveto{\pgfqpoint{3.313103in}{2.960240in}}{\pgfqpoint{3.318277in}{2.947749in}}{\pgfqpoint{3.327485in}{2.938541in}}%
\pgfpathcurveto{\pgfqpoint{3.336694in}{2.929332in}}{\pgfqpoint{3.349185in}{2.924158in}}{\pgfqpoint{3.362208in}{2.924158in}}%
\pgfpathlineto{\pgfqpoint{3.362208in}{2.924158in}}%
\pgfpathclose%
\pgfusepath{stroke,fill}%
\end{pgfscope}%
\begin{pgfscope}%
\pgfpathrectangle{\pgfqpoint{0.786164in}{0.768110in}}{\pgfqpoint{8.851069in}{7.081890in}}%
\pgfusepath{clip}%
\pgfsetbuttcap%
\pgfsetroundjoin%
\definecolor{currentfill}{rgb}{0.175841,0.441290,0.557685}%
\pgfsetfillcolor{currentfill}%
\pgfsetfillopacity{0.700000}%
\pgfsetlinewidth{0.501875pt}%
\definecolor{currentstroke}{rgb}{1.000000,1.000000,1.000000}%
\pgfsetstrokecolor{currentstroke}%
\pgfsetstrokeopacity{0.700000}%
\pgfsetdash{}{0pt}%
\pgfpathmoveto{\pgfqpoint{3.508340in}{3.055548in}}%
\pgfpathcurveto{\pgfqpoint{3.521363in}{3.055548in}}{\pgfqpoint{3.533854in}{3.060722in}}{\pgfqpoint{3.543062in}{3.069930in}}%
\pgfpathcurveto{\pgfqpoint{3.552271in}{3.079138in}}{\pgfqpoint{3.557445in}{3.091630in}}{\pgfqpoint{3.557445in}{3.104652in}}%
\pgfpathcurveto{\pgfqpoint{3.557445in}{3.117675in}}{\pgfqpoint{3.552271in}{3.130166in}}{\pgfqpoint{3.543062in}{3.139374in}}%
\pgfpathcurveto{\pgfqpoint{3.533854in}{3.148583in}}{\pgfqpoint{3.521363in}{3.153757in}}{\pgfqpoint{3.508340in}{3.153757in}}%
\pgfpathcurveto{\pgfqpoint{3.495318in}{3.153757in}}{\pgfqpoint{3.482826in}{3.148583in}}{\pgfqpoint{3.473618in}{3.139374in}}%
\pgfpathcurveto{\pgfqpoint{3.464410in}{3.130166in}}{\pgfqpoint{3.459236in}{3.117675in}}{\pgfqpoint{3.459236in}{3.104652in}}%
\pgfpathcurveto{\pgfqpoint{3.459236in}{3.091630in}}{\pgfqpoint{3.464410in}{3.079138in}}{\pgfqpoint{3.473618in}{3.069930in}}%
\pgfpathcurveto{\pgfqpoint{3.482826in}{3.060722in}}{\pgfqpoint{3.495318in}{3.055548in}}{\pgfqpoint{3.508340in}{3.055548in}}%
\pgfpathlineto{\pgfqpoint{3.508340in}{3.055548in}}%
\pgfpathclose%
\pgfusepath{stroke,fill}%
\end{pgfscope}%
\begin{pgfscope}%
\pgfpathrectangle{\pgfqpoint{0.786164in}{0.768110in}}{\pgfqpoint{8.851069in}{7.081890in}}%
\pgfusepath{clip}%
\pgfsetbuttcap%
\pgfsetroundjoin%
\definecolor{currentfill}{rgb}{0.165117,0.467423,0.558141}%
\pgfsetfillcolor{currentfill}%
\pgfsetfillopacity{0.700000}%
\pgfsetlinewidth{0.501875pt}%
\definecolor{currentstroke}{rgb}{1.000000,1.000000,1.000000}%
\pgfsetstrokecolor{currentstroke}%
\pgfsetstrokeopacity{0.700000}%
\pgfsetdash{}{0pt}%
\pgfpathmoveto{\pgfqpoint{3.480940in}{3.230733in}}%
\pgfpathcurveto{\pgfqpoint{3.493963in}{3.230733in}}{\pgfqpoint{3.506454in}{3.235907in}}{\pgfqpoint{3.515663in}{3.245116in}}%
\pgfpathcurveto{\pgfqpoint{3.524871in}{3.254324in}}{\pgfqpoint{3.530045in}{3.266815in}}{\pgfqpoint{3.530045in}{3.279838in}}%
\pgfpathcurveto{\pgfqpoint{3.530045in}{3.292861in}}{\pgfqpoint{3.524871in}{3.305352in}}{\pgfqpoint{3.515663in}{3.314560in}}%
\pgfpathcurveto{\pgfqpoint{3.506454in}{3.323769in}}{\pgfqpoint{3.493963in}{3.328943in}}{\pgfqpoint{3.480940in}{3.328943in}}%
\pgfpathcurveto{\pgfqpoint{3.467918in}{3.328943in}}{\pgfqpoint{3.455427in}{3.323769in}}{\pgfqpoint{3.446218in}{3.314560in}}%
\pgfpathcurveto{\pgfqpoint{3.437010in}{3.305352in}}{\pgfqpoint{3.431836in}{3.292861in}}{\pgfqpoint{3.431836in}{3.279838in}}%
\pgfpathcurveto{\pgfqpoint{3.431836in}{3.266815in}}{\pgfqpoint{3.437010in}{3.254324in}}{\pgfqpoint{3.446218in}{3.245116in}}%
\pgfpathcurveto{\pgfqpoint{3.455427in}{3.235907in}}{\pgfqpoint{3.467918in}{3.230733in}}{\pgfqpoint{3.480940in}{3.230733in}}%
\pgfpathlineto{\pgfqpoint{3.480940in}{3.230733in}}%
\pgfpathclose%
\pgfusepath{stroke,fill}%
\end{pgfscope}%
\begin{pgfscope}%
\pgfpathrectangle{\pgfqpoint{0.786164in}{0.768110in}}{\pgfqpoint{8.851069in}{7.081890in}}%
\pgfusepath{clip}%
\pgfsetbuttcap%
\pgfsetroundjoin%
\definecolor{currentfill}{rgb}{0.154815,0.493313,0.557840}%
\pgfsetfillcolor{currentfill}%
\pgfsetfillopacity{0.700000}%
\pgfsetlinewidth{0.501875pt}%
\definecolor{currentstroke}{rgb}{1.000000,1.000000,1.000000}%
\pgfsetstrokecolor{currentstroke}%
\pgfsetstrokeopacity{0.700000}%
\pgfsetdash{}{0pt}%
\pgfpathmoveto{\pgfqpoint{3.417007in}{3.252632in}}%
\pgfpathcurveto{\pgfqpoint{3.430030in}{3.252632in}}{\pgfqpoint{3.442521in}{3.257806in}}{\pgfqpoint{3.451730in}{3.267014in}}%
\pgfpathcurveto{\pgfqpoint{3.460938in}{3.276223in}}{\pgfqpoint{3.466112in}{3.288714in}}{\pgfqpoint{3.466112in}{3.301736in}}%
\pgfpathcurveto{\pgfqpoint{3.466112in}{3.314759in}}{\pgfqpoint{3.460938in}{3.327250in}}{\pgfqpoint{3.451730in}{3.336459in}}%
\pgfpathcurveto{\pgfqpoint{3.442521in}{3.345667in}}{\pgfqpoint{3.430030in}{3.350841in}}{\pgfqpoint{3.417007in}{3.350841in}}%
\pgfpathcurveto{\pgfqpoint{3.403985in}{3.350841in}}{\pgfqpoint{3.391494in}{3.345667in}}{\pgfqpoint{3.382285in}{3.336459in}}%
\pgfpathcurveto{\pgfqpoint{3.373077in}{3.327250in}}{\pgfqpoint{3.367903in}{3.314759in}}{\pgfqpoint{3.367903in}{3.301736in}}%
\pgfpathcurveto{\pgfqpoint{3.367903in}{3.288714in}}{\pgfqpoint{3.373077in}{3.276223in}}{\pgfqpoint{3.382285in}{3.267014in}}%
\pgfpathcurveto{\pgfqpoint{3.391494in}{3.257806in}}{\pgfqpoint{3.403985in}{3.252632in}}{\pgfqpoint{3.417007in}{3.252632in}}%
\pgfpathlineto{\pgfqpoint{3.417007in}{3.252632in}}%
\pgfpathclose%
\pgfusepath{stroke,fill}%
\end{pgfscope}%
\begin{pgfscope}%
\pgfpathrectangle{\pgfqpoint{0.786164in}{0.768110in}}{\pgfqpoint{8.851069in}{7.081890in}}%
\pgfusepath{clip}%
\pgfsetbuttcap%
\pgfsetroundjoin%
\definecolor{currentfill}{rgb}{0.150476,0.504369,0.557430}%
\pgfsetfillcolor{currentfill}%
\pgfsetfillopacity{0.700000}%
\pgfsetlinewidth{0.501875pt}%
\definecolor{currentstroke}{rgb}{1.000000,1.000000,1.000000}%
\pgfsetstrokecolor{currentstroke}%
\pgfsetstrokeopacity{0.700000}%
\pgfsetdash{}{0pt}%
\pgfpathmoveto{\pgfqpoint{3.435274in}{3.055548in}}%
\pgfpathcurveto{\pgfqpoint{3.448297in}{3.055548in}}{\pgfqpoint{3.460788in}{3.060722in}}{\pgfqpoint{3.469996in}{3.069930in}}%
\pgfpathcurveto{\pgfqpoint{3.479205in}{3.079138in}}{\pgfqpoint{3.484379in}{3.091630in}}{\pgfqpoint{3.484379in}{3.104652in}}%
\pgfpathcurveto{\pgfqpoint{3.484379in}{3.117675in}}{\pgfqpoint{3.479205in}{3.130166in}}{\pgfqpoint{3.469996in}{3.139374in}}%
\pgfpathcurveto{\pgfqpoint{3.460788in}{3.148583in}}{\pgfqpoint{3.448297in}{3.153757in}}{\pgfqpoint{3.435274in}{3.153757in}}%
\pgfpathcurveto{\pgfqpoint{3.422251in}{3.153757in}}{\pgfqpoint{3.409760in}{3.148583in}}{\pgfqpoint{3.400552in}{3.139374in}}%
\pgfpathcurveto{\pgfqpoint{3.391343in}{3.130166in}}{\pgfqpoint{3.386169in}{3.117675in}}{\pgfqpoint{3.386169in}{3.104652in}}%
\pgfpathcurveto{\pgfqpoint{3.386169in}{3.091630in}}{\pgfqpoint{3.391343in}{3.079138in}}{\pgfqpoint{3.400552in}{3.069930in}}%
\pgfpathcurveto{\pgfqpoint{3.409760in}{3.060722in}}{\pgfqpoint{3.422251in}{3.055548in}}{\pgfqpoint{3.435274in}{3.055548in}}%
\pgfpathlineto{\pgfqpoint{3.435274in}{3.055548in}}%
\pgfpathclose%
\pgfusepath{stroke,fill}%
\end{pgfscope}%
\begin{pgfscope}%
\pgfpathrectangle{\pgfqpoint{0.786164in}{0.768110in}}{\pgfqpoint{8.851069in}{7.081890in}}%
\pgfusepath{clip}%
\pgfsetbuttcap%
\pgfsetroundjoin%
\definecolor{currentfill}{rgb}{0.137770,0.537492,0.554906}%
\pgfsetfillcolor{currentfill}%
\pgfsetfillopacity{0.700000}%
\pgfsetlinewidth{0.501875pt}%
\definecolor{currentstroke}{rgb}{1.000000,1.000000,1.000000}%
\pgfsetstrokecolor{currentstroke}%
\pgfsetstrokeopacity{0.700000}%
\pgfsetdash{}{0pt}%
\pgfpathmoveto{\pgfqpoint{3.225208in}{2.967955in}}%
\pgfpathcurveto{\pgfqpoint{3.238231in}{2.967955in}}{\pgfqpoint{3.250722in}{2.973129in}}{\pgfqpoint{3.259931in}{2.982337in}}%
\pgfpathcurveto{\pgfqpoint{3.269139in}{2.991545in}}{\pgfqpoint{3.274313in}{3.004037in}}{\pgfqpoint{3.274313in}{3.017059in}}%
\pgfpathcurveto{\pgfqpoint{3.274313in}{3.030082in}}{\pgfqpoint{3.269139in}{3.042573in}}{\pgfqpoint{3.259931in}{3.051781in}}%
\pgfpathcurveto{\pgfqpoint{3.250722in}{3.060990in}}{\pgfqpoint{3.238231in}{3.066164in}}{\pgfqpoint{3.225208in}{3.066164in}}%
\pgfpathcurveto{\pgfqpoint{3.212186in}{3.066164in}}{\pgfqpoint{3.199695in}{3.060990in}}{\pgfqpoint{3.190486in}{3.051781in}}%
\pgfpathcurveto{\pgfqpoint{3.181278in}{3.042573in}}{\pgfqpoint{3.176104in}{3.030082in}}{\pgfqpoint{3.176104in}{3.017059in}}%
\pgfpathcurveto{\pgfqpoint{3.176104in}{3.004037in}}{\pgfqpoint{3.181278in}{2.991545in}}{\pgfqpoint{3.190486in}{2.982337in}}%
\pgfpathcurveto{\pgfqpoint{3.199695in}{2.973129in}}{\pgfqpoint{3.212186in}{2.967955in}}{\pgfqpoint{3.225208in}{2.967955in}}%
\pgfpathlineto{\pgfqpoint{3.225208in}{2.967955in}}%
\pgfpathclose%
\pgfusepath{stroke,fill}%
\end{pgfscope}%
\begin{pgfscope}%
\pgfpathrectangle{\pgfqpoint{0.786164in}{0.768110in}}{\pgfqpoint{8.851069in}{7.081890in}}%
\pgfusepath{clip}%
\pgfsetbuttcap%
\pgfsetroundjoin%
\definecolor{currentfill}{rgb}{0.137770,0.537492,0.554906}%
\pgfsetfillcolor{currentfill}%
\pgfsetfillopacity{0.700000}%
\pgfsetlinewidth{0.501875pt}%
\definecolor{currentstroke}{rgb}{1.000000,1.000000,1.000000}%
\pgfsetstrokecolor{currentstroke}%
\pgfsetstrokeopacity{0.700000}%
\pgfsetdash{}{0pt}%
\pgfpathmoveto{\pgfqpoint{3.462674in}{2.792769in}}%
\pgfpathcurveto{\pgfqpoint{3.475696in}{2.792769in}}{\pgfqpoint{3.488188in}{2.797943in}}{\pgfqpoint{3.497396in}{2.807151in}}%
\pgfpathcurveto{\pgfqpoint{3.506604in}{2.816360in}}{\pgfqpoint{3.511778in}{2.828851in}}{\pgfqpoint{3.511778in}{2.841873in}}%
\pgfpathcurveto{\pgfqpoint{3.511778in}{2.854896in}}{\pgfqpoint{3.506604in}{2.867387in}}{\pgfqpoint{3.497396in}{2.876596in}}%
\pgfpathcurveto{\pgfqpoint{3.488188in}{2.885804in}}{\pgfqpoint{3.475696in}{2.890978in}}{\pgfqpoint{3.462674in}{2.890978in}}%
\pgfpathcurveto{\pgfqpoint{3.449651in}{2.890978in}}{\pgfqpoint{3.437160in}{2.885804in}}{\pgfqpoint{3.427952in}{2.876596in}}%
\pgfpathcurveto{\pgfqpoint{3.418743in}{2.867387in}}{\pgfqpoint{3.413569in}{2.854896in}}{\pgfqpoint{3.413569in}{2.841873in}}%
\pgfpathcurveto{\pgfqpoint{3.413569in}{2.828851in}}{\pgfqpoint{3.418743in}{2.816360in}}{\pgfqpoint{3.427952in}{2.807151in}}%
\pgfpathcurveto{\pgfqpoint{3.437160in}{2.797943in}}{\pgfqpoint{3.449651in}{2.792769in}}{\pgfqpoint{3.462674in}{2.792769in}}%
\pgfpathlineto{\pgfqpoint{3.462674in}{2.792769in}}%
\pgfpathclose%
\pgfusepath{stroke,fill}%
\end{pgfscope}%
\begin{pgfscope}%
\pgfpathrectangle{\pgfqpoint{0.786164in}{0.768110in}}{\pgfqpoint{8.851069in}{7.081890in}}%
\pgfusepath{clip}%
\pgfsetbuttcap%
\pgfsetroundjoin%
\definecolor{currentfill}{rgb}{0.135066,0.544853,0.554029}%
\pgfsetfillcolor{currentfill}%
\pgfsetfillopacity{0.700000}%
\pgfsetlinewidth{0.501875pt}%
\definecolor{currentstroke}{rgb}{1.000000,1.000000,1.000000}%
\pgfsetstrokecolor{currentstroke}%
\pgfsetstrokeopacity{0.700000}%
\pgfsetdash{}{0pt}%
\pgfpathmoveto{\pgfqpoint{3.353074in}{2.836565in}}%
\pgfpathcurveto{\pgfqpoint{3.366097in}{2.836565in}}{\pgfqpoint{3.378588in}{2.841739in}}{\pgfqpoint{3.387797in}{2.850948in}}%
\pgfpathcurveto{\pgfqpoint{3.397005in}{2.860156in}}{\pgfqpoint{3.402179in}{2.872647in}}{\pgfqpoint{3.402179in}{2.885670in}}%
\pgfpathcurveto{\pgfqpoint{3.402179in}{2.898693in}}{\pgfqpoint{3.397005in}{2.911184in}}{\pgfqpoint{3.387797in}{2.920392in}}%
\pgfpathcurveto{\pgfqpoint{3.378588in}{2.929601in}}{\pgfqpoint{3.366097in}{2.934774in}}{\pgfqpoint{3.353074in}{2.934774in}}%
\pgfpathcurveto{\pgfqpoint{3.340052in}{2.934774in}}{\pgfqpoint{3.327561in}{2.929601in}}{\pgfqpoint{3.318352in}{2.920392in}}%
\pgfpathcurveto{\pgfqpoint{3.309144in}{2.911184in}}{\pgfqpoint{3.303970in}{2.898693in}}{\pgfqpoint{3.303970in}{2.885670in}}%
\pgfpathcurveto{\pgfqpoint{3.303970in}{2.872647in}}{\pgfqpoint{3.309144in}{2.860156in}}{\pgfqpoint{3.318352in}{2.850948in}}%
\pgfpathcurveto{\pgfqpoint{3.327561in}{2.841739in}}{\pgfqpoint{3.340052in}{2.836565in}}{\pgfqpoint{3.353074in}{2.836565in}}%
\pgfpathlineto{\pgfqpoint{3.353074in}{2.836565in}}%
\pgfpathclose%
\pgfusepath{stroke,fill}%
\end{pgfscope}%
\begin{pgfscope}%
\pgfpathrectangle{\pgfqpoint{0.786164in}{0.768110in}}{\pgfqpoint{8.851069in}{7.081890in}}%
\pgfusepath{clip}%
\pgfsetbuttcap%
\pgfsetroundjoin%
\definecolor{currentfill}{rgb}{0.129933,0.559582,0.551864}%
\pgfsetfillcolor{currentfill}%
\pgfsetfillopacity{0.700000}%
\pgfsetlinewidth{0.501875pt}%
\definecolor{currentstroke}{rgb}{1.000000,1.000000,1.000000}%
\pgfsetstrokecolor{currentstroke}%
\pgfsetstrokeopacity{0.700000}%
\pgfsetdash{}{0pt}%
\pgfpathmoveto{\pgfqpoint{3.307408in}{3.011751in}}%
\pgfpathcurveto{\pgfqpoint{3.320431in}{3.011751in}}{\pgfqpoint{3.332922in}{3.016925in}}{\pgfqpoint{3.342130in}{3.026134in}}%
\pgfpathcurveto{\pgfqpoint{3.351339in}{3.035342in}}{\pgfqpoint{3.356513in}{3.047833in}}{\pgfqpoint{3.356513in}{3.060856in}}%
\pgfpathcurveto{\pgfqpoint{3.356513in}{3.073878in}}{\pgfqpoint{3.351339in}{3.086370in}}{\pgfqpoint{3.342130in}{3.095578in}}%
\pgfpathcurveto{\pgfqpoint{3.332922in}{3.104786in}}{\pgfqpoint{3.320431in}{3.109960in}}{\pgfqpoint{3.307408in}{3.109960in}}%
\pgfpathcurveto{\pgfqpoint{3.294385in}{3.109960in}}{\pgfqpoint{3.281894in}{3.104786in}}{\pgfqpoint{3.272686in}{3.095578in}}%
\pgfpathcurveto{\pgfqpoint{3.263477in}{3.086370in}}{\pgfqpoint{3.258303in}{3.073878in}}{\pgfqpoint{3.258303in}{3.060856in}}%
\pgfpathcurveto{\pgfqpoint{3.258303in}{3.047833in}}{\pgfqpoint{3.263477in}{3.035342in}}{\pgfqpoint{3.272686in}{3.026134in}}%
\pgfpathcurveto{\pgfqpoint{3.281894in}{3.016925in}}{\pgfqpoint{3.294385in}{3.011751in}}{\pgfqpoint{3.307408in}{3.011751in}}%
\pgfpathlineto{\pgfqpoint{3.307408in}{3.011751in}}%
\pgfpathclose%
\pgfusepath{stroke,fill}%
\end{pgfscope}%
\begin{pgfscope}%
\pgfpathrectangle{\pgfqpoint{0.786164in}{0.768110in}}{\pgfqpoint{8.851069in}{7.081890in}}%
\pgfusepath{clip}%
\pgfsetbuttcap%
\pgfsetroundjoin%
\definecolor{currentfill}{rgb}{0.129933,0.559582,0.551864}%
\pgfsetfillcolor{currentfill}%
\pgfsetfillopacity{0.700000}%
\pgfsetlinewidth{0.501875pt}%
\definecolor{currentstroke}{rgb}{1.000000,1.000000,1.000000}%
\pgfsetstrokecolor{currentstroke}%
\pgfsetstrokeopacity{0.700000}%
\pgfsetdash{}{0pt}%
\pgfpathmoveto{\pgfqpoint{3.325674in}{2.989853in}}%
\pgfpathcurveto{\pgfqpoint{3.338697in}{2.989853in}}{\pgfqpoint{3.351188in}{2.995027in}}{\pgfqpoint{3.360397in}{3.004235in}}%
\pgfpathcurveto{\pgfqpoint{3.369605in}{3.013444in}}{\pgfqpoint{3.374779in}{3.025935in}}{\pgfqpoint{3.374779in}{3.038958in}}%
\pgfpathcurveto{\pgfqpoint{3.374779in}{3.051980in}}{\pgfqpoint{3.369605in}{3.064471in}}{\pgfqpoint{3.360397in}{3.073680in}}%
\pgfpathcurveto{\pgfqpoint{3.351188in}{3.082888in}}{\pgfqpoint{3.338697in}{3.088062in}}{\pgfqpoint{3.325674in}{3.088062in}}%
\pgfpathcurveto{\pgfqpoint{3.312652in}{3.088062in}}{\pgfqpoint{3.300161in}{3.082888in}}{\pgfqpoint{3.290952in}{3.073680in}}%
\pgfpathcurveto{\pgfqpoint{3.281744in}{3.064471in}}{\pgfqpoint{3.276570in}{3.051980in}}{\pgfqpoint{3.276570in}{3.038958in}}%
\pgfpathcurveto{\pgfqpoint{3.276570in}{3.025935in}}{\pgfqpoint{3.281744in}{3.013444in}}{\pgfqpoint{3.290952in}{3.004235in}}%
\pgfpathcurveto{\pgfqpoint{3.300161in}{2.995027in}}{\pgfqpoint{3.312652in}{2.989853in}}{\pgfqpoint{3.325674in}{2.989853in}}%
\pgfpathlineto{\pgfqpoint{3.325674in}{2.989853in}}%
\pgfpathclose%
\pgfusepath{stroke,fill}%
\end{pgfscope}%
\begin{pgfscope}%
\pgfpathrectangle{\pgfqpoint{0.786164in}{0.768110in}}{\pgfqpoint{8.851069in}{7.081890in}}%
\pgfusepath{clip}%
\pgfsetbuttcap%
\pgfsetroundjoin%
\definecolor{currentfill}{rgb}{0.126453,0.570633,0.549841}%
\pgfsetfillcolor{currentfill}%
\pgfsetfillopacity{0.700000}%
\pgfsetlinewidth{0.501875pt}%
\definecolor{currentstroke}{rgb}{1.000000,1.000000,1.000000}%
\pgfsetstrokecolor{currentstroke}%
\pgfsetstrokeopacity{0.700000}%
\pgfsetdash{}{0pt}%
\pgfpathmoveto{\pgfqpoint{3.161275in}{2.858463in}}%
\pgfpathcurveto{\pgfqpoint{3.174298in}{2.858463in}}{\pgfqpoint{3.186789in}{2.863637in}}{\pgfqpoint{3.195998in}{2.872846in}}%
\pgfpathcurveto{\pgfqpoint{3.205206in}{2.882054in}}{\pgfqpoint{3.210380in}{2.894545in}}{\pgfqpoint{3.210380in}{2.907568in}}%
\pgfpathcurveto{\pgfqpoint{3.210380in}{2.920591in}}{\pgfqpoint{3.205206in}{2.933082in}}{\pgfqpoint{3.195998in}{2.942290in}}%
\pgfpathcurveto{\pgfqpoint{3.186789in}{2.951499in}}{\pgfqpoint{3.174298in}{2.956673in}}{\pgfqpoint{3.161275in}{2.956673in}}%
\pgfpathcurveto{\pgfqpoint{3.148253in}{2.956673in}}{\pgfqpoint{3.135762in}{2.951499in}}{\pgfqpoint{3.126553in}{2.942290in}}%
\pgfpathcurveto{\pgfqpoint{3.117345in}{2.933082in}}{\pgfqpoint{3.112171in}{2.920591in}}{\pgfqpoint{3.112171in}{2.907568in}}%
\pgfpathcurveto{\pgfqpoint{3.112171in}{2.894545in}}{\pgfqpoint{3.117345in}{2.882054in}}{\pgfqpoint{3.126553in}{2.872846in}}%
\pgfpathcurveto{\pgfqpoint{3.135762in}{2.863637in}}{\pgfqpoint{3.148253in}{2.858463in}}{\pgfqpoint{3.161275in}{2.858463in}}%
\pgfpathlineto{\pgfqpoint{3.161275in}{2.858463in}}%
\pgfpathclose%
\pgfusepath{stroke,fill}%
\end{pgfscope}%
\begin{pgfscope}%
\pgfpathrectangle{\pgfqpoint{0.786164in}{0.768110in}}{\pgfqpoint{8.851069in}{7.081890in}}%
\pgfusepath{clip}%
\pgfsetbuttcap%
\pgfsetroundjoin%
\definecolor{currentfill}{rgb}{0.126453,0.570633,0.549841}%
\pgfsetfillcolor{currentfill}%
\pgfsetfillopacity{0.700000}%
\pgfsetlinewidth{0.501875pt}%
\definecolor{currentstroke}{rgb}{1.000000,1.000000,1.000000}%
\pgfsetstrokecolor{currentstroke}%
\pgfsetstrokeopacity{0.700000}%
\pgfsetdash{}{0pt}%
\pgfpathmoveto{\pgfqpoint{3.216075in}{2.924158in}}%
\pgfpathcurveto{\pgfqpoint{3.229098in}{2.924158in}}{\pgfqpoint{3.241589in}{2.929332in}}{\pgfqpoint{3.250797in}{2.938541in}}%
\pgfpathcurveto{\pgfqpoint{3.260006in}{2.947749in}}{\pgfqpoint{3.265180in}{2.960240in}}{\pgfqpoint{3.265180in}{2.973263in}}%
\pgfpathcurveto{\pgfqpoint{3.265180in}{2.986286in}}{\pgfqpoint{3.260006in}{2.998777in}}{\pgfqpoint{3.250797in}{3.007985in}}%
\pgfpathcurveto{\pgfqpoint{3.241589in}{3.017193in}}{\pgfqpoint{3.229098in}{3.022367in}}{\pgfqpoint{3.216075in}{3.022367in}}%
\pgfpathcurveto{\pgfqpoint{3.203052in}{3.022367in}}{\pgfqpoint{3.190561in}{3.017193in}}{\pgfqpoint{3.181353in}{3.007985in}}%
\pgfpathcurveto{\pgfqpoint{3.172144in}{2.998777in}}{\pgfqpoint{3.166970in}{2.986286in}}{\pgfqpoint{3.166970in}{2.973263in}}%
\pgfpathcurveto{\pgfqpoint{3.166970in}{2.960240in}}{\pgfqpoint{3.172144in}{2.947749in}}{\pgfqpoint{3.181353in}{2.938541in}}%
\pgfpathcurveto{\pgfqpoint{3.190561in}{2.929332in}}{\pgfqpoint{3.203052in}{2.924158in}}{\pgfqpoint{3.216075in}{2.924158in}}%
\pgfpathlineto{\pgfqpoint{3.216075in}{2.924158in}}%
\pgfpathclose%
\pgfusepath{stroke,fill}%
\end{pgfscope}%
\begin{pgfscope}%
\pgfpathrectangle{\pgfqpoint{0.786164in}{0.768110in}}{\pgfqpoint{8.851069in}{7.081890in}}%
\pgfusepath{clip}%
\pgfsetbuttcap%
\pgfsetroundjoin%
\definecolor{currentfill}{rgb}{0.126453,0.570633,0.549841}%
\pgfsetfillcolor{currentfill}%
\pgfsetfillopacity{0.700000}%
\pgfsetlinewidth{0.501875pt}%
\definecolor{currentstroke}{rgb}{1.000000,1.000000,1.000000}%
\pgfsetstrokecolor{currentstroke}%
\pgfsetstrokeopacity{0.700000}%
\pgfsetdash{}{0pt}%
\pgfpathmoveto{\pgfqpoint{3.225208in}{2.902260in}}%
\pgfpathcurveto{\pgfqpoint{3.238231in}{2.902260in}}{\pgfqpoint{3.250722in}{2.907434in}}{\pgfqpoint{3.259931in}{2.916642in}}%
\pgfpathcurveto{\pgfqpoint{3.269139in}{2.925851in}}{\pgfqpoint{3.274313in}{2.938342in}}{\pgfqpoint{3.274313in}{2.951365in}}%
\pgfpathcurveto{\pgfqpoint{3.274313in}{2.964387in}}{\pgfqpoint{3.269139in}{2.976878in}}{\pgfqpoint{3.259931in}{2.986087in}}%
\pgfpathcurveto{\pgfqpoint{3.250722in}{2.995295in}}{\pgfqpoint{3.238231in}{3.000469in}}{\pgfqpoint{3.225208in}{3.000469in}}%
\pgfpathcurveto{\pgfqpoint{3.212186in}{3.000469in}}{\pgfqpoint{3.199695in}{2.995295in}}{\pgfqpoint{3.190486in}{2.986087in}}%
\pgfpathcurveto{\pgfqpoint{3.181278in}{2.976878in}}{\pgfqpoint{3.176104in}{2.964387in}}{\pgfqpoint{3.176104in}{2.951365in}}%
\pgfpathcurveto{\pgfqpoint{3.176104in}{2.938342in}}{\pgfqpoint{3.181278in}{2.925851in}}{\pgfqpoint{3.190486in}{2.916642in}}%
\pgfpathcurveto{\pgfqpoint{3.199695in}{2.907434in}}{\pgfqpoint{3.212186in}{2.902260in}}{\pgfqpoint{3.225208in}{2.902260in}}%
\pgfpathlineto{\pgfqpoint{3.225208in}{2.902260in}}%
\pgfpathclose%
\pgfusepath{stroke,fill}%
\end{pgfscope}%
\begin{pgfscope}%
\pgfpathrectangle{\pgfqpoint{0.786164in}{0.768110in}}{\pgfqpoint{8.851069in}{7.081890in}}%
\pgfusepath{clip}%
\pgfsetbuttcap%
\pgfsetroundjoin%
\definecolor{currentfill}{rgb}{0.127568,0.566949,0.550556}%
\pgfsetfillcolor{currentfill}%
\pgfsetfillopacity{0.700000}%
\pgfsetlinewidth{0.501875pt}%
\definecolor{currentstroke}{rgb}{1.000000,1.000000,1.000000}%
\pgfsetstrokecolor{currentstroke}%
\pgfsetstrokeopacity{0.700000}%
\pgfsetdash{}{0pt}%
\pgfpathmoveto{\pgfqpoint{3.280008in}{3.055548in}}%
\pgfpathcurveto{\pgfqpoint{3.293031in}{3.055548in}}{\pgfqpoint{3.305522in}{3.060722in}}{\pgfqpoint{3.314730in}{3.069930in}}%
\pgfpathcurveto{\pgfqpoint{3.323939in}{3.079138in}}{\pgfqpoint{3.329113in}{3.091630in}}{\pgfqpoint{3.329113in}{3.104652in}}%
\pgfpathcurveto{\pgfqpoint{3.329113in}{3.117675in}}{\pgfqpoint{3.323939in}{3.130166in}}{\pgfqpoint{3.314730in}{3.139374in}}%
\pgfpathcurveto{\pgfqpoint{3.305522in}{3.148583in}}{\pgfqpoint{3.293031in}{3.153757in}}{\pgfqpoint{3.280008in}{3.153757in}}%
\pgfpathcurveto{\pgfqpoint{3.266985in}{3.153757in}}{\pgfqpoint{3.254494in}{3.148583in}}{\pgfqpoint{3.245286in}{3.139374in}}%
\pgfpathcurveto{\pgfqpoint{3.236077in}{3.130166in}}{\pgfqpoint{3.230903in}{3.117675in}}{\pgfqpoint{3.230903in}{3.104652in}}%
\pgfpathcurveto{\pgfqpoint{3.230903in}{3.091630in}}{\pgfqpoint{3.236077in}{3.079138in}}{\pgfqpoint{3.245286in}{3.069930in}}%
\pgfpathcurveto{\pgfqpoint{3.254494in}{3.060722in}}{\pgfqpoint{3.266985in}{3.055548in}}{\pgfqpoint{3.280008in}{3.055548in}}%
\pgfpathlineto{\pgfqpoint{3.280008in}{3.055548in}}%
\pgfpathclose%
\pgfusepath{stroke,fill}%
\end{pgfscope}%
\begin{pgfscope}%
\pgfpathrectangle{\pgfqpoint{0.786164in}{0.768110in}}{\pgfqpoint{8.851069in}{7.081890in}}%
\pgfusepath{clip}%
\pgfsetbuttcap%
\pgfsetroundjoin%
\definecolor{currentfill}{rgb}{0.125394,0.574318,0.549086}%
\pgfsetfillcolor{currentfill}%
\pgfsetfillopacity{0.700000}%
\pgfsetlinewidth{0.501875pt}%
\definecolor{currentstroke}{rgb}{1.000000,1.000000,1.000000}%
\pgfsetstrokecolor{currentstroke}%
\pgfsetstrokeopacity{0.700000}%
\pgfsetdash{}{0pt}%
\pgfpathmoveto{\pgfqpoint{3.161275in}{3.165039in}}%
\pgfpathcurveto{\pgfqpoint{3.174298in}{3.165039in}}{\pgfqpoint{3.186789in}{3.170213in}}{\pgfqpoint{3.195998in}{3.179421in}}%
\pgfpathcurveto{\pgfqpoint{3.205206in}{3.188630in}}{\pgfqpoint{3.210380in}{3.201121in}}{\pgfqpoint{3.210380in}{3.214143in}}%
\pgfpathcurveto{\pgfqpoint{3.210380in}{3.227166in}}{\pgfqpoint{3.205206in}{3.239657in}}{\pgfqpoint{3.195998in}{3.248866in}}%
\pgfpathcurveto{\pgfqpoint{3.186789in}{3.258074in}}{\pgfqpoint{3.174298in}{3.263248in}}{\pgfqpoint{3.161275in}{3.263248in}}%
\pgfpathcurveto{\pgfqpoint{3.148253in}{3.263248in}}{\pgfqpoint{3.135762in}{3.258074in}}{\pgfqpoint{3.126553in}{3.248866in}}%
\pgfpathcurveto{\pgfqpoint{3.117345in}{3.239657in}}{\pgfqpoint{3.112171in}{3.227166in}}{\pgfqpoint{3.112171in}{3.214143in}}%
\pgfpathcurveto{\pgfqpoint{3.112171in}{3.201121in}}{\pgfqpoint{3.117345in}{3.188630in}}{\pgfqpoint{3.126553in}{3.179421in}}%
\pgfpathcurveto{\pgfqpoint{3.135762in}{3.170213in}}{\pgfqpoint{3.148253in}{3.165039in}}{\pgfqpoint{3.161275in}{3.165039in}}%
\pgfpathlineto{\pgfqpoint{3.161275in}{3.165039in}}%
\pgfpathclose%
\pgfusepath{stroke,fill}%
\end{pgfscope}%
\begin{pgfscope}%
\pgfpathrectangle{\pgfqpoint{0.786164in}{0.768110in}}{\pgfqpoint{8.851069in}{7.081890in}}%
\pgfusepath{clip}%
\pgfsetbuttcap%
\pgfsetroundjoin%
\definecolor{currentfill}{rgb}{0.125394,0.574318,0.549086}%
\pgfsetfillcolor{currentfill}%
\pgfsetfillopacity{0.700000}%
\pgfsetlinewidth{0.501875pt}%
\definecolor{currentstroke}{rgb}{1.000000,1.000000,1.000000}%
\pgfsetstrokecolor{currentstroke}%
\pgfsetstrokeopacity{0.700000}%
\pgfsetdash{}{0pt}%
\pgfpathmoveto{\pgfqpoint{3.188675in}{3.121242in}}%
\pgfpathcurveto{\pgfqpoint{3.201698in}{3.121242in}}{\pgfqpoint{3.214189in}{3.126416in}}{\pgfqpoint{3.223397in}{3.135625in}}%
\pgfpathcurveto{\pgfqpoint{3.232606in}{3.144833in}}{\pgfqpoint{3.237780in}{3.157324in}}{\pgfqpoint{3.237780in}{3.170347in}}%
\pgfpathcurveto{\pgfqpoint{3.237780in}{3.183370in}}{\pgfqpoint{3.232606in}{3.195861in}}{\pgfqpoint{3.223397in}{3.205069in}}%
\pgfpathcurveto{\pgfqpoint{3.214189in}{3.214278in}}{\pgfqpoint{3.201698in}{3.219452in}}{\pgfqpoint{3.188675in}{3.219452in}}%
\pgfpathcurveto{\pgfqpoint{3.175652in}{3.219452in}}{\pgfqpoint{3.163161in}{3.214278in}}{\pgfqpoint{3.153953in}{3.205069in}}%
\pgfpathcurveto{\pgfqpoint{3.144744in}{3.195861in}}{\pgfqpoint{3.139571in}{3.183370in}}{\pgfqpoint{3.139571in}{3.170347in}}%
\pgfpathcurveto{\pgfqpoint{3.139571in}{3.157324in}}{\pgfqpoint{3.144744in}{3.144833in}}{\pgfqpoint{3.153953in}{3.135625in}}%
\pgfpathcurveto{\pgfqpoint{3.163161in}{3.126416in}}{\pgfqpoint{3.175652in}{3.121242in}}{\pgfqpoint{3.188675in}{3.121242in}}%
\pgfpathlineto{\pgfqpoint{3.188675in}{3.121242in}}%
\pgfpathclose%
\pgfusepath{stroke,fill}%
\end{pgfscope}%
\begin{pgfscope}%
\pgfpathrectangle{\pgfqpoint{0.786164in}{0.768110in}}{\pgfqpoint{8.851069in}{7.081890in}}%
\pgfusepath{clip}%
\pgfsetbuttcap%
\pgfsetroundjoin%
\definecolor{currentfill}{rgb}{0.119483,0.614817,0.537692}%
\pgfsetfillcolor{currentfill}%
\pgfsetfillopacity{0.700000}%
\pgfsetlinewidth{0.501875pt}%
\definecolor{currentstroke}{rgb}{1.000000,1.000000,1.000000}%
\pgfsetstrokecolor{currentstroke}%
\pgfsetstrokeopacity{0.700000}%
\pgfsetdash{}{0pt}%
\pgfpathmoveto{\pgfqpoint{2.932943in}{2.727074in}}%
\pgfpathcurveto{\pgfqpoint{2.945966in}{2.727074in}}{\pgfqpoint{2.958457in}{2.732248in}}{\pgfqpoint{2.967665in}{2.741456in}}%
\pgfpathcurveto{\pgfqpoint{2.976874in}{2.750665in}}{\pgfqpoint{2.982048in}{2.763156in}}{\pgfqpoint{2.982048in}{2.776179in}}%
\pgfpathcurveto{\pgfqpoint{2.982048in}{2.789201in}}{\pgfqpoint{2.976874in}{2.801692in}}{\pgfqpoint{2.967665in}{2.810901in}}%
\pgfpathcurveto{\pgfqpoint{2.958457in}{2.820109in}}{\pgfqpoint{2.945966in}{2.825283in}}{\pgfqpoint{2.932943in}{2.825283in}}%
\pgfpathcurveto{\pgfqpoint{2.919920in}{2.825283in}}{\pgfqpoint{2.907429in}{2.820109in}}{\pgfqpoint{2.898221in}{2.810901in}}%
\pgfpathcurveto{\pgfqpoint{2.889012in}{2.801692in}}{\pgfqpoint{2.883838in}{2.789201in}}{\pgfqpoint{2.883838in}{2.776179in}}%
\pgfpathcurveto{\pgfqpoint{2.883838in}{2.763156in}}{\pgfqpoint{2.889012in}{2.750665in}}{\pgfqpoint{2.898221in}{2.741456in}}%
\pgfpathcurveto{\pgfqpoint{2.907429in}{2.732248in}}{\pgfqpoint{2.919920in}{2.727074in}}{\pgfqpoint{2.932943in}{2.727074in}}%
\pgfpathlineto{\pgfqpoint{2.932943in}{2.727074in}}%
\pgfpathclose%
\pgfusepath{stroke,fill}%
\end{pgfscope}%
\begin{pgfscope}%
\pgfpathrectangle{\pgfqpoint{0.786164in}{0.768110in}}{\pgfqpoint{8.851069in}{7.081890in}}%
\pgfusepath{clip}%
\pgfsetbuttcap%
\pgfsetroundjoin%
\definecolor{currentfill}{rgb}{0.122606,0.585371,0.546557}%
\pgfsetfillcolor{currentfill}%
\pgfsetfillopacity{0.700000}%
\pgfsetlinewidth{0.501875pt}%
\definecolor{currentstroke}{rgb}{1.000000,1.000000,1.000000}%
\pgfsetstrokecolor{currentstroke}%
\pgfsetstrokeopacity{0.700000}%
\pgfsetdash{}{0pt}%
\pgfpathmoveto{\pgfqpoint{3.097342in}{2.661379in}}%
\pgfpathcurveto{\pgfqpoint{3.110365in}{2.661379in}}{\pgfqpoint{3.122856in}{2.666553in}}{\pgfqpoint{3.132065in}{2.675762in}}%
\pgfpathcurveto{\pgfqpoint{3.141273in}{2.684970in}}{\pgfqpoint{3.146447in}{2.697461in}}{\pgfqpoint{3.146447in}{2.710484in}}%
\pgfpathcurveto{\pgfqpoint{3.146447in}{2.723507in}}{\pgfqpoint{3.141273in}{2.735998in}}{\pgfqpoint{3.132065in}{2.745206in}}%
\pgfpathcurveto{\pgfqpoint{3.122856in}{2.754415in}}{\pgfqpoint{3.110365in}{2.759589in}}{\pgfqpoint{3.097342in}{2.759589in}}%
\pgfpathcurveto{\pgfqpoint{3.084320in}{2.759589in}}{\pgfqpoint{3.071829in}{2.754415in}}{\pgfqpoint{3.062620in}{2.745206in}}%
\pgfpathcurveto{\pgfqpoint{3.053412in}{2.735998in}}{\pgfqpoint{3.048238in}{2.723507in}}{\pgfqpoint{3.048238in}{2.710484in}}%
\pgfpathcurveto{\pgfqpoint{3.048238in}{2.697461in}}{\pgfqpoint{3.053412in}{2.684970in}}{\pgfqpoint{3.062620in}{2.675762in}}%
\pgfpathcurveto{\pgfqpoint{3.071829in}{2.666553in}}{\pgfqpoint{3.084320in}{2.661379in}}{\pgfqpoint{3.097342in}{2.661379in}}%
\pgfpathlineto{\pgfqpoint{3.097342in}{2.661379in}}%
\pgfpathclose%
\pgfusepath{stroke,fill}%
\end{pgfscope}%
\begin{pgfscope}%
\pgfpathrectangle{\pgfqpoint{0.786164in}{0.768110in}}{\pgfqpoint{8.851069in}{7.081890in}}%
\pgfusepath{clip}%
\pgfsetbuttcap%
\pgfsetroundjoin%
\definecolor{currentfill}{rgb}{0.125394,0.574318,0.549086}%
\pgfsetfillcolor{currentfill}%
\pgfsetfillopacity{0.700000}%
\pgfsetlinewidth{0.501875pt}%
\definecolor{currentstroke}{rgb}{1.000000,1.000000,1.000000}%
\pgfsetstrokecolor{currentstroke}%
\pgfsetstrokeopacity{0.700000}%
\pgfsetdash{}{0pt}%
\pgfpathmoveto{\pgfqpoint{2.923810in}{2.727074in}}%
\pgfpathcurveto{\pgfqpoint{2.936833in}{2.727074in}}{\pgfqpoint{2.949324in}{2.732248in}}{\pgfqpoint{2.958532in}{2.741456in}}%
\pgfpathcurveto{\pgfqpoint{2.967740in}{2.750665in}}{\pgfqpoint{2.972914in}{2.763156in}}{\pgfqpoint{2.972914in}{2.776179in}}%
\pgfpathcurveto{\pgfqpoint{2.972914in}{2.789201in}}{\pgfqpoint{2.967740in}{2.801692in}}{\pgfqpoint{2.958532in}{2.810901in}}%
\pgfpathcurveto{\pgfqpoint{2.949324in}{2.820109in}}{\pgfqpoint{2.936833in}{2.825283in}}{\pgfqpoint{2.923810in}{2.825283in}}%
\pgfpathcurveto{\pgfqpoint{2.910787in}{2.825283in}}{\pgfqpoint{2.898296in}{2.820109in}}{\pgfqpoint{2.889088in}{2.810901in}}%
\pgfpathcurveto{\pgfqpoint{2.879879in}{2.801692in}}{\pgfqpoint{2.874705in}{2.789201in}}{\pgfqpoint{2.874705in}{2.776179in}}%
\pgfpathcurveto{\pgfqpoint{2.874705in}{2.763156in}}{\pgfqpoint{2.879879in}{2.750665in}}{\pgfqpoint{2.889088in}{2.741456in}}%
\pgfpathcurveto{\pgfqpoint{2.898296in}{2.732248in}}{\pgfqpoint{2.910787in}{2.727074in}}{\pgfqpoint{2.923810in}{2.727074in}}%
\pgfpathlineto{\pgfqpoint{2.923810in}{2.727074in}}%
\pgfpathclose%
\pgfusepath{stroke,fill}%
\end{pgfscope}%
\begin{pgfscope}%
\pgfpathrectangle{\pgfqpoint{0.786164in}{0.768110in}}{\pgfqpoint{8.851069in}{7.081890in}}%
\pgfusepath{clip}%
\pgfsetbuttcap%
\pgfsetroundjoin%
\definecolor{currentfill}{rgb}{0.276022,0.044167,0.370164}%
\pgfsetfillcolor{currentfill}%
\pgfsetfillopacity{0.700000}%
\pgfsetlinewidth{0.501875pt}%
\definecolor{currentstroke}{rgb}{1.000000,1.000000,1.000000}%
\pgfsetstrokecolor{currentstroke}%
\pgfsetstrokeopacity{0.700000}%
\pgfsetdash{}{0pt}%
\pgfpathmoveto{\pgfqpoint{4.467335in}{4.194256in}}%
\pgfpathcurveto{\pgfqpoint{4.480358in}{4.194256in}}{\pgfqpoint{4.492849in}{4.199430in}}{\pgfqpoint{4.502058in}{4.208638in}}%
\pgfpathcurveto{\pgfqpoint{4.511266in}{4.217847in}}{\pgfqpoint{4.516440in}{4.230338in}}{\pgfqpoint{4.516440in}{4.243360in}}%
\pgfpathcurveto{\pgfqpoint{4.516440in}{4.256383in}}{\pgfqpoint{4.511266in}{4.268874in}}{\pgfqpoint{4.502058in}{4.278083in}}%
\pgfpathcurveto{\pgfqpoint{4.492849in}{4.287291in}}{\pgfqpoint{4.480358in}{4.292465in}}{\pgfqpoint{4.467335in}{4.292465in}}%
\pgfpathcurveto{\pgfqpoint{4.454313in}{4.292465in}}{\pgfqpoint{4.441822in}{4.287291in}}{\pgfqpoint{4.432613in}{4.278083in}}%
\pgfpathcurveto{\pgfqpoint{4.423405in}{4.268874in}}{\pgfqpoint{4.418231in}{4.256383in}}{\pgfqpoint{4.418231in}{4.243360in}}%
\pgfpathcurveto{\pgfqpoint{4.418231in}{4.230338in}}{\pgfqpoint{4.423405in}{4.217847in}}{\pgfqpoint{4.432613in}{4.208638in}}%
\pgfpathcurveto{\pgfqpoint{4.441822in}{4.199430in}}{\pgfqpoint{4.454313in}{4.194256in}}{\pgfqpoint{4.467335in}{4.194256in}}%
\pgfpathlineto{\pgfqpoint{4.467335in}{4.194256in}}%
\pgfpathclose%
\pgfusepath{stroke,fill}%
\end{pgfscope}%
\begin{pgfscope}%
\pgfpathrectangle{\pgfqpoint{0.786164in}{0.768110in}}{\pgfqpoint{8.851069in}{7.081890in}}%
\pgfusepath{clip}%
\pgfsetbuttcap%
\pgfsetroundjoin%
\definecolor{currentfill}{rgb}{0.277018,0.050344,0.375715}%
\pgfsetfillcolor{currentfill}%
\pgfsetfillopacity{0.700000}%
\pgfsetlinewidth{0.501875pt}%
\definecolor{currentstroke}{rgb}{1.000000,1.000000,1.000000}%
\pgfsetstrokecolor{currentstroke}%
\pgfsetstrokeopacity{0.700000}%
\pgfsetdash{}{0pt}%
\pgfpathmoveto{\pgfqpoint{4.376003in}{4.106663in}}%
\pgfpathcurveto{\pgfqpoint{4.389025in}{4.106663in}}{\pgfqpoint{4.401516in}{4.111837in}}{\pgfqpoint{4.410725in}{4.121045in}}%
\pgfpathcurveto{\pgfqpoint{4.419933in}{4.130254in}}{\pgfqpoint{4.425107in}{4.142745in}}{\pgfqpoint{4.425107in}{4.155768in}}%
\pgfpathcurveto{\pgfqpoint{4.425107in}{4.168790in}}{\pgfqpoint{4.419933in}{4.181281in}}{\pgfqpoint{4.410725in}{4.190490in}}%
\pgfpathcurveto{\pgfqpoint{4.401516in}{4.199698in}}{\pgfqpoint{4.389025in}{4.204872in}}{\pgfqpoint{4.376003in}{4.204872in}}%
\pgfpathcurveto{\pgfqpoint{4.362980in}{4.204872in}}{\pgfqpoint{4.350489in}{4.199698in}}{\pgfqpoint{4.341280in}{4.190490in}}%
\pgfpathcurveto{\pgfqpoint{4.332072in}{4.181281in}}{\pgfqpoint{4.326898in}{4.168790in}}{\pgfqpoint{4.326898in}{4.155768in}}%
\pgfpathcurveto{\pgfqpoint{4.326898in}{4.142745in}}{\pgfqpoint{4.332072in}{4.130254in}}{\pgfqpoint{4.341280in}{4.121045in}}%
\pgfpathcurveto{\pgfqpoint{4.350489in}{4.111837in}}{\pgfqpoint{4.362980in}{4.106663in}}{\pgfqpoint{4.376003in}{4.106663in}}%
\pgfpathlineto{\pgfqpoint{4.376003in}{4.106663in}}%
\pgfpathclose%
\pgfusepath{stroke,fill}%
\end{pgfscope}%
\begin{pgfscope}%
\pgfpathrectangle{\pgfqpoint{0.786164in}{0.768110in}}{\pgfqpoint{8.851069in}{7.081890in}}%
\pgfusepath{clip}%
\pgfsetbuttcap%
\pgfsetroundjoin%
\definecolor{currentfill}{rgb}{0.277941,0.056324,0.381191}%
\pgfsetfillcolor{currentfill}%
\pgfsetfillopacity{0.700000}%
\pgfsetlinewidth{0.501875pt}%
\definecolor{currentstroke}{rgb}{1.000000,1.000000,1.000000}%
\pgfsetstrokecolor{currentstroke}%
\pgfsetstrokeopacity{0.700000}%
\pgfsetdash{}{0pt}%
\pgfpathmoveto{\pgfqpoint{4.330336in}{4.019070in}}%
\pgfpathcurveto{\pgfqpoint{4.343359in}{4.019070in}}{\pgfqpoint{4.355850in}{4.024244in}}{\pgfqpoint{4.365058in}{4.033452in}}%
\pgfpathcurveto{\pgfqpoint{4.374267in}{4.042661in}}{\pgfqpoint{4.379441in}{4.055152in}}{\pgfqpoint{4.379441in}{4.068175in}}%
\pgfpathcurveto{\pgfqpoint{4.379441in}{4.081197in}}{\pgfqpoint{4.374267in}{4.093688in}}{\pgfqpoint{4.365058in}{4.102897in}}%
\pgfpathcurveto{\pgfqpoint{4.355850in}{4.112105in}}{\pgfqpoint{4.343359in}{4.117279in}}{\pgfqpoint{4.330336in}{4.117279in}}%
\pgfpathcurveto{\pgfqpoint{4.317313in}{4.117279in}}{\pgfqpoint{4.304822in}{4.112105in}}{\pgfqpoint{4.295614in}{4.102897in}}%
\pgfpathcurveto{\pgfqpoint{4.286405in}{4.093688in}}{\pgfqpoint{4.281231in}{4.081197in}}{\pgfqpoint{4.281231in}{4.068175in}}%
\pgfpathcurveto{\pgfqpoint{4.281231in}{4.055152in}}{\pgfqpoint{4.286405in}{4.042661in}}{\pgfqpoint{4.295614in}{4.033452in}}%
\pgfpathcurveto{\pgfqpoint{4.304822in}{4.024244in}}{\pgfqpoint{4.317313in}{4.019070in}}{\pgfqpoint{4.330336in}{4.019070in}}%
\pgfpathlineto{\pgfqpoint{4.330336in}{4.019070in}}%
\pgfpathclose%
\pgfusepath{stroke,fill}%
\end{pgfscope}%
\begin{pgfscope}%
\pgfpathrectangle{\pgfqpoint{0.786164in}{0.768110in}}{\pgfqpoint{8.851069in}{7.081890in}}%
\pgfusepath{clip}%
\pgfsetbuttcap%
\pgfsetroundjoin%
\definecolor{currentfill}{rgb}{0.279566,0.067836,0.391917}%
\pgfsetfillcolor{currentfill}%
\pgfsetfillopacity{0.700000}%
\pgfsetlinewidth{0.501875pt}%
\definecolor{currentstroke}{rgb}{1.000000,1.000000,1.000000}%
\pgfsetstrokecolor{currentstroke}%
\pgfsetstrokeopacity{0.700000}%
\pgfsetdash{}{0pt}%
\pgfpathmoveto{\pgfqpoint{4.202470in}{3.931477in}}%
\pgfpathcurveto{\pgfqpoint{4.215493in}{3.931477in}}{\pgfqpoint{4.227984in}{3.936651in}}{\pgfqpoint{4.237192in}{3.945859in}}%
\pgfpathcurveto{\pgfqpoint{4.246401in}{3.955068in}}{\pgfqpoint{4.251575in}{3.967559in}}{\pgfqpoint{4.251575in}{3.980582in}}%
\pgfpathcurveto{\pgfqpoint{4.251575in}{3.993604in}}{\pgfqpoint{4.246401in}{4.006095in}}{\pgfqpoint{4.237192in}{4.015304in}}%
\pgfpathcurveto{\pgfqpoint{4.227984in}{4.024512in}}{\pgfqpoint{4.215493in}{4.029686in}}{\pgfqpoint{4.202470in}{4.029686in}}%
\pgfpathcurveto{\pgfqpoint{4.189447in}{4.029686in}}{\pgfqpoint{4.176956in}{4.024512in}}{\pgfqpoint{4.167748in}{4.015304in}}%
\pgfpathcurveto{\pgfqpoint{4.158539in}{4.006095in}}{\pgfqpoint{4.153365in}{3.993604in}}{\pgfqpoint{4.153365in}{3.980582in}}%
\pgfpathcurveto{\pgfqpoint{4.153365in}{3.967559in}}{\pgfqpoint{4.158539in}{3.955068in}}{\pgfqpoint{4.167748in}{3.945859in}}%
\pgfpathcurveto{\pgfqpoint{4.176956in}{3.936651in}}{\pgfqpoint{4.189447in}{3.931477in}}{\pgfqpoint{4.202470in}{3.931477in}}%
\pgfpathlineto{\pgfqpoint{4.202470in}{3.931477in}}%
\pgfpathclose%
\pgfusepath{stroke,fill}%
\end{pgfscope}%
\begin{pgfscope}%
\pgfpathrectangle{\pgfqpoint{0.786164in}{0.768110in}}{\pgfqpoint{8.851069in}{7.081890in}}%
\pgfusepath{clip}%
\pgfsetbuttcap%
\pgfsetroundjoin%
\definecolor{currentfill}{rgb}{0.280894,0.078907,0.402329}%
\pgfsetfillcolor{currentfill}%
\pgfsetfillopacity{0.700000}%
\pgfsetlinewidth{0.501875pt}%
\definecolor{currentstroke}{rgb}{1.000000,1.000000,1.000000}%
\pgfsetstrokecolor{currentstroke}%
\pgfsetstrokeopacity{0.700000}%
\pgfsetdash{}{0pt}%
\pgfpathmoveto{\pgfqpoint{4.239003in}{3.931477in}}%
\pgfpathcurveto{\pgfqpoint{4.252026in}{3.931477in}}{\pgfqpoint{4.264517in}{3.936651in}}{\pgfqpoint{4.273725in}{3.945859in}}%
\pgfpathcurveto{\pgfqpoint{4.282934in}{3.955068in}}{\pgfqpoint{4.288108in}{3.967559in}}{\pgfqpoint{4.288108in}{3.980582in}}%
\pgfpathcurveto{\pgfqpoint{4.288108in}{3.993604in}}{\pgfqpoint{4.282934in}{4.006095in}}{\pgfqpoint{4.273725in}{4.015304in}}%
\pgfpathcurveto{\pgfqpoint{4.264517in}{4.024512in}}{\pgfqpoint{4.252026in}{4.029686in}}{\pgfqpoint{4.239003in}{4.029686in}}%
\pgfpathcurveto{\pgfqpoint{4.225981in}{4.029686in}}{\pgfqpoint{4.213489in}{4.024512in}}{\pgfqpoint{4.204281in}{4.015304in}}%
\pgfpathcurveto{\pgfqpoint{4.195073in}{4.006095in}}{\pgfqpoint{4.189899in}{3.993604in}}{\pgfqpoint{4.189899in}{3.980582in}}%
\pgfpathcurveto{\pgfqpoint{4.189899in}{3.967559in}}{\pgfqpoint{4.195073in}{3.955068in}}{\pgfqpoint{4.204281in}{3.945859in}}%
\pgfpathcurveto{\pgfqpoint{4.213489in}{3.936651in}}{\pgfqpoint{4.225981in}{3.931477in}}{\pgfqpoint{4.239003in}{3.931477in}}%
\pgfpathlineto{\pgfqpoint{4.239003in}{3.931477in}}%
\pgfpathclose%
\pgfusepath{stroke,fill}%
\end{pgfscope}%
\begin{pgfscope}%
\pgfpathrectangle{\pgfqpoint{0.786164in}{0.768110in}}{\pgfqpoint{8.851069in}{7.081890in}}%
\pgfusepath{clip}%
\pgfsetbuttcap%
\pgfsetroundjoin%
\definecolor{currentfill}{rgb}{0.282910,0.105393,0.426902}%
\pgfsetfillcolor{currentfill}%
\pgfsetfillopacity{0.700000}%
\pgfsetlinewidth{0.501875pt}%
\definecolor{currentstroke}{rgb}{1.000000,1.000000,1.000000}%
\pgfsetstrokecolor{currentstroke}%
\pgfsetstrokeopacity{0.700000}%
\pgfsetdash{}{0pt}%
\pgfpathmoveto{\pgfqpoint{4.111137in}{3.646800in}}%
\pgfpathcurveto{\pgfqpoint{4.124160in}{3.646800in}}{\pgfqpoint{4.136651in}{3.651974in}}{\pgfqpoint{4.145859in}{3.661182in}}%
\pgfpathcurveto{\pgfqpoint{4.155068in}{3.670391in}}{\pgfqpoint{4.160242in}{3.682882in}}{\pgfqpoint{4.160242in}{3.695905in}}%
\pgfpathcurveto{\pgfqpoint{4.160242in}{3.708927in}}{\pgfqpoint{4.155068in}{3.721418in}}{\pgfqpoint{4.145859in}{3.730627in}}%
\pgfpathcurveto{\pgfqpoint{4.136651in}{3.739835in}}{\pgfqpoint{4.124160in}{3.745009in}}{\pgfqpoint{4.111137in}{3.745009in}}%
\pgfpathcurveto{\pgfqpoint{4.098114in}{3.745009in}}{\pgfqpoint{4.085623in}{3.739835in}}{\pgfqpoint{4.076415in}{3.730627in}}%
\pgfpathcurveto{\pgfqpoint{4.067207in}{3.721418in}}{\pgfqpoint{4.062033in}{3.708927in}}{\pgfqpoint{4.062033in}{3.695905in}}%
\pgfpathcurveto{\pgfqpoint{4.062033in}{3.682882in}}{\pgfqpoint{4.067207in}{3.670391in}}{\pgfqpoint{4.076415in}{3.661182in}}%
\pgfpathcurveto{\pgfqpoint{4.085623in}{3.651974in}}{\pgfqpoint{4.098114in}{3.646800in}}{\pgfqpoint{4.111137in}{3.646800in}}%
\pgfpathlineto{\pgfqpoint{4.111137in}{3.646800in}}%
\pgfpathclose%
\pgfusepath{stroke,fill}%
\end{pgfscope}%
\begin{pgfscope}%
\pgfpathrectangle{\pgfqpoint{0.786164in}{0.768110in}}{\pgfqpoint{8.851069in}{7.081890in}}%
\pgfusepath{clip}%
\pgfsetbuttcap%
\pgfsetroundjoin%
\definecolor{currentfill}{rgb}{0.283187,0.125848,0.444960}%
\pgfsetfillcolor{currentfill}%
\pgfsetfillopacity{0.700000}%
\pgfsetlinewidth{0.501875pt}%
\definecolor{currentstroke}{rgb}{1.000000,1.000000,1.000000}%
\pgfsetstrokecolor{currentstroke}%
\pgfsetstrokeopacity{0.700000}%
\pgfsetdash{}{0pt}%
\pgfpathmoveto{\pgfqpoint{4.348603in}{3.778189in}}%
\pgfpathcurveto{\pgfqpoint{4.361625in}{3.778189in}}{\pgfqpoint{4.374116in}{3.783363in}}{\pgfqpoint{4.383325in}{3.792572in}}%
\pgfpathcurveto{\pgfqpoint{4.392533in}{3.801780in}}{\pgfqpoint{4.397707in}{3.814271in}}{\pgfqpoint{4.397707in}{3.827294in}}%
\pgfpathcurveto{\pgfqpoint{4.397707in}{3.840317in}}{\pgfqpoint{4.392533in}{3.852808in}}{\pgfqpoint{4.383325in}{3.862016in}}%
\pgfpathcurveto{\pgfqpoint{4.374116in}{3.871225in}}{\pgfqpoint{4.361625in}{3.876399in}}{\pgfqpoint{4.348603in}{3.876399in}}%
\pgfpathcurveto{\pgfqpoint{4.335580in}{3.876399in}}{\pgfqpoint{4.323089in}{3.871225in}}{\pgfqpoint{4.313880in}{3.862016in}}%
\pgfpathcurveto{\pgfqpoint{4.304672in}{3.852808in}}{\pgfqpoint{4.299498in}{3.840317in}}{\pgfqpoint{4.299498in}{3.827294in}}%
\pgfpathcurveto{\pgfqpoint{4.299498in}{3.814271in}}{\pgfqpoint{4.304672in}{3.801780in}}{\pgfqpoint{4.313880in}{3.792572in}}%
\pgfpathcurveto{\pgfqpoint{4.323089in}{3.783363in}}{\pgfqpoint{4.335580in}{3.778189in}}{\pgfqpoint{4.348603in}{3.778189in}}%
\pgfpathlineto{\pgfqpoint{4.348603in}{3.778189in}}%
\pgfpathclose%
\pgfusepath{stroke,fill}%
\end{pgfscope}%
\begin{pgfscope}%
\pgfpathrectangle{\pgfqpoint{0.786164in}{0.768110in}}{\pgfqpoint{8.851069in}{7.081890in}}%
\pgfusepath{clip}%
\pgfsetbuttcap%
\pgfsetroundjoin%
\definecolor{currentfill}{rgb}{0.282884,0.135920,0.453427}%
\pgfsetfillcolor{currentfill}%
\pgfsetfillopacity{0.700000}%
\pgfsetlinewidth{0.501875pt}%
\definecolor{currentstroke}{rgb}{1.000000,1.000000,1.000000}%
\pgfsetstrokecolor{currentstroke}%
\pgfsetstrokeopacity{0.700000}%
\pgfsetdash{}{0pt}%
\pgfpathmoveto{\pgfqpoint{3.965005in}{3.537309in}}%
\pgfpathcurveto{\pgfqpoint{3.978027in}{3.537309in}}{\pgfqpoint{3.990518in}{3.542483in}}{\pgfqpoint{3.999727in}{3.551691in}}%
\pgfpathcurveto{\pgfqpoint{4.008935in}{3.560900in}}{\pgfqpoint{4.014109in}{3.573391in}}{\pgfqpoint{4.014109in}{3.586413in}}%
\pgfpathcurveto{\pgfqpoint{4.014109in}{3.599436in}}{\pgfqpoint{4.008935in}{3.611927in}}{\pgfqpoint{3.999727in}{3.621136in}}%
\pgfpathcurveto{\pgfqpoint{3.990518in}{3.630344in}}{\pgfqpoint{3.978027in}{3.635518in}}{\pgfqpoint{3.965005in}{3.635518in}}%
\pgfpathcurveto{\pgfqpoint{3.951982in}{3.635518in}}{\pgfqpoint{3.939491in}{3.630344in}}{\pgfqpoint{3.930282in}{3.621136in}}%
\pgfpathcurveto{\pgfqpoint{3.921074in}{3.611927in}}{\pgfqpoint{3.915900in}{3.599436in}}{\pgfqpoint{3.915900in}{3.586413in}}%
\pgfpathcurveto{\pgfqpoint{3.915900in}{3.573391in}}{\pgfqpoint{3.921074in}{3.560900in}}{\pgfqpoint{3.930282in}{3.551691in}}%
\pgfpathcurveto{\pgfqpoint{3.939491in}{3.542483in}}{\pgfqpoint{3.951982in}{3.537309in}}{\pgfqpoint{3.965005in}{3.537309in}}%
\pgfpathlineto{\pgfqpoint{3.965005in}{3.537309in}}%
\pgfpathclose%
\pgfusepath{stroke,fill}%
\end{pgfscope}%
\begin{pgfscope}%
\pgfpathrectangle{\pgfqpoint{0.786164in}{0.768110in}}{\pgfqpoint{8.851069in}{7.081890in}}%
\pgfusepath{clip}%
\pgfsetbuttcap%
\pgfsetroundjoin%
\definecolor{currentfill}{rgb}{0.281887,0.150881,0.465405}%
\pgfsetfillcolor{currentfill}%
\pgfsetfillopacity{0.700000}%
\pgfsetlinewidth{0.501875pt}%
\definecolor{currentstroke}{rgb}{1.000000,1.000000,1.000000}%
\pgfsetstrokecolor{currentstroke}%
\pgfsetstrokeopacity{0.700000}%
\pgfsetdash{}{0pt}%
\pgfpathmoveto{\pgfqpoint{3.745806in}{3.471614in}}%
\pgfpathcurveto{\pgfqpoint{3.758828in}{3.471614in}}{\pgfqpoint{3.771319in}{3.476788in}}{\pgfqpoint{3.780528in}{3.485996in}}%
\pgfpathcurveto{\pgfqpoint{3.789736in}{3.495205in}}{\pgfqpoint{3.794910in}{3.507696in}}{\pgfqpoint{3.794910in}{3.520719in}}%
\pgfpathcurveto{\pgfqpoint{3.794910in}{3.533741in}}{\pgfqpoint{3.789736in}{3.546232in}}{\pgfqpoint{3.780528in}{3.555441in}}%
\pgfpathcurveto{\pgfqpoint{3.771319in}{3.564649in}}{\pgfqpoint{3.758828in}{3.569823in}}{\pgfqpoint{3.745806in}{3.569823in}}%
\pgfpathcurveto{\pgfqpoint{3.732783in}{3.569823in}}{\pgfqpoint{3.720292in}{3.564649in}}{\pgfqpoint{3.711083in}{3.555441in}}%
\pgfpathcurveto{\pgfqpoint{3.701875in}{3.546232in}}{\pgfqpoint{3.696701in}{3.533741in}}{\pgfqpoint{3.696701in}{3.520719in}}%
\pgfpathcurveto{\pgfqpoint{3.696701in}{3.507696in}}{\pgfqpoint{3.701875in}{3.495205in}}{\pgfqpoint{3.711083in}{3.485996in}}%
\pgfpathcurveto{\pgfqpoint{3.720292in}{3.476788in}}{\pgfqpoint{3.732783in}{3.471614in}}{\pgfqpoint{3.745806in}{3.471614in}}%
\pgfpathlineto{\pgfqpoint{3.745806in}{3.471614in}}%
\pgfpathclose%
\pgfusepath{stroke,fill}%
\end{pgfscope}%
\begin{pgfscope}%
\pgfpathrectangle{\pgfqpoint{0.786164in}{0.768110in}}{\pgfqpoint{8.851069in}{7.081890in}}%
\pgfusepath{clip}%
\pgfsetbuttcap%
\pgfsetroundjoin%
\definecolor{currentfill}{rgb}{0.280868,0.160771,0.472899}%
\pgfsetfillcolor{currentfill}%
\pgfsetfillopacity{0.700000}%
\pgfsetlinewidth{0.501875pt}%
\definecolor{currentstroke}{rgb}{1.000000,1.000000,1.000000}%
\pgfsetstrokecolor{currentstroke}%
\pgfsetstrokeopacity{0.700000}%
\pgfsetdash{}{0pt}%
\pgfpathmoveto{\pgfqpoint{3.855405in}{3.449716in}}%
\pgfpathcurveto{\pgfqpoint{3.868428in}{3.449716in}}{\pgfqpoint{3.880919in}{3.454890in}}{\pgfqpoint{3.890127in}{3.464098in}}%
\pgfpathcurveto{\pgfqpoint{3.899336in}{3.473307in}}{\pgfqpoint{3.904510in}{3.485798in}}{\pgfqpoint{3.904510in}{3.498820in}}%
\pgfpathcurveto{\pgfqpoint{3.904510in}{3.511843in}}{\pgfqpoint{3.899336in}{3.524334in}}{\pgfqpoint{3.890127in}{3.533543in}}%
\pgfpathcurveto{\pgfqpoint{3.880919in}{3.542751in}}{\pgfqpoint{3.868428in}{3.547925in}}{\pgfqpoint{3.855405in}{3.547925in}}%
\pgfpathcurveto{\pgfqpoint{3.842382in}{3.547925in}}{\pgfqpoint{3.829891in}{3.542751in}}{\pgfqpoint{3.820683in}{3.533543in}}%
\pgfpathcurveto{\pgfqpoint{3.811474in}{3.524334in}}{\pgfqpoint{3.806301in}{3.511843in}}{\pgfqpoint{3.806301in}{3.498820in}}%
\pgfpathcurveto{\pgfqpoint{3.806301in}{3.485798in}}{\pgfqpoint{3.811474in}{3.473307in}}{\pgfqpoint{3.820683in}{3.464098in}}%
\pgfpathcurveto{\pgfqpoint{3.829891in}{3.454890in}}{\pgfqpoint{3.842382in}{3.449716in}}{\pgfqpoint{3.855405in}{3.449716in}}%
\pgfpathlineto{\pgfqpoint{3.855405in}{3.449716in}}%
\pgfpathclose%
\pgfusepath{stroke,fill}%
\end{pgfscope}%
\begin{pgfscope}%
\pgfpathrectangle{\pgfqpoint{0.786164in}{0.768110in}}{\pgfqpoint{8.851069in}{7.081890in}}%
\pgfusepath{clip}%
\pgfsetbuttcap%
\pgfsetroundjoin%
\definecolor{currentfill}{rgb}{0.280255,0.165693,0.476498}%
\pgfsetfillcolor{currentfill}%
\pgfsetfillopacity{0.700000}%
\pgfsetlinewidth{0.501875pt}%
\definecolor{currentstroke}{rgb}{1.000000,1.000000,1.000000}%
\pgfsetstrokecolor{currentstroke}%
\pgfsetstrokeopacity{0.700000}%
\pgfsetdash{}{0pt}%
\pgfpathmoveto{\pgfqpoint{3.563140in}{3.318326in}}%
\pgfpathcurveto{\pgfqpoint{3.576163in}{3.318326in}}{\pgfqpoint{3.588654in}{3.323500in}}{\pgfqpoint{3.597862in}{3.332709in}}%
\pgfpathcurveto{\pgfqpoint{3.607071in}{3.341917in}}{\pgfqpoint{3.612245in}{3.354408in}}{\pgfqpoint{3.612245in}{3.367431in}}%
\pgfpathcurveto{\pgfqpoint{3.612245in}{3.380454in}}{\pgfqpoint{3.607071in}{3.392945in}}{\pgfqpoint{3.597862in}{3.402153in}}%
\pgfpathcurveto{\pgfqpoint{3.588654in}{3.411362in}}{\pgfqpoint{3.576163in}{3.416536in}}{\pgfqpoint{3.563140in}{3.416536in}}%
\pgfpathcurveto{\pgfqpoint{3.550117in}{3.416536in}}{\pgfqpoint{3.537626in}{3.411362in}}{\pgfqpoint{3.528418in}{3.402153in}}%
\pgfpathcurveto{\pgfqpoint{3.519209in}{3.392945in}}{\pgfqpoint{3.514035in}{3.380454in}}{\pgfqpoint{3.514035in}{3.367431in}}%
\pgfpathcurveto{\pgfqpoint{3.514035in}{3.354408in}}{\pgfqpoint{3.519209in}{3.341917in}}{\pgfqpoint{3.528418in}{3.332709in}}%
\pgfpathcurveto{\pgfqpoint{3.537626in}{3.323500in}}{\pgfqpoint{3.550117in}{3.318326in}}{\pgfqpoint{3.563140in}{3.318326in}}%
\pgfpathlineto{\pgfqpoint{3.563140in}{3.318326in}}%
\pgfpathclose%
\pgfusepath{stroke,fill}%
\end{pgfscope}%
\begin{pgfscope}%
\pgfpathrectangle{\pgfqpoint{0.786164in}{0.768110in}}{\pgfqpoint{8.851069in}{7.081890in}}%
\pgfusepath{clip}%
\pgfsetbuttcap%
\pgfsetroundjoin%
\definecolor{currentfill}{rgb}{0.280255,0.165693,0.476498}%
\pgfsetfillcolor{currentfill}%
\pgfsetfillopacity{0.700000}%
\pgfsetlinewidth{0.501875pt}%
\definecolor{currentstroke}{rgb}{1.000000,1.000000,1.000000}%
\pgfsetstrokecolor{currentstroke}%
\pgfsetstrokeopacity{0.700000}%
\pgfsetdash{}{0pt}%
\pgfpathmoveto{\pgfqpoint{3.572273in}{3.405919in}}%
\pgfpathcurveto{\pgfqpoint{3.585296in}{3.405919in}}{\pgfqpoint{3.597787in}{3.411093in}}{\pgfqpoint{3.606995in}{3.420302in}}%
\pgfpathcurveto{\pgfqpoint{3.616204in}{3.429510in}}{\pgfqpoint{3.621378in}{3.442001in}}{\pgfqpoint{3.621378in}{3.455024in}}%
\pgfpathcurveto{\pgfqpoint{3.621378in}{3.468047in}}{\pgfqpoint{3.616204in}{3.480538in}}{\pgfqpoint{3.606995in}{3.489746in}}%
\pgfpathcurveto{\pgfqpoint{3.597787in}{3.498955in}}{\pgfqpoint{3.585296in}{3.504129in}}{\pgfqpoint{3.572273in}{3.504129in}}%
\pgfpathcurveto{\pgfqpoint{3.559251in}{3.504129in}}{\pgfqpoint{3.546759in}{3.498955in}}{\pgfqpoint{3.537551in}{3.489746in}}%
\pgfpathcurveto{\pgfqpoint{3.528343in}{3.480538in}}{\pgfqpoint{3.523169in}{3.468047in}}{\pgfqpoint{3.523169in}{3.455024in}}%
\pgfpathcurveto{\pgfqpoint{3.523169in}{3.442001in}}{\pgfqpoint{3.528343in}{3.429510in}}{\pgfqpoint{3.537551in}{3.420302in}}%
\pgfpathcurveto{\pgfqpoint{3.546759in}{3.411093in}}{\pgfqpoint{3.559251in}{3.405919in}}{\pgfqpoint{3.572273in}{3.405919in}}%
\pgfpathlineto{\pgfqpoint{3.572273in}{3.405919in}}%
\pgfpathclose%
\pgfusepath{stroke,fill}%
\end{pgfscope}%
\begin{pgfscope}%
\pgfpathrectangle{\pgfqpoint{0.786164in}{0.768110in}}{\pgfqpoint{8.851069in}{7.081890in}}%
\pgfusepath{clip}%
\pgfsetbuttcap%
\pgfsetroundjoin%
\definecolor{currentfill}{rgb}{0.278012,0.180367,0.486697}%
\pgfsetfillcolor{currentfill}%
\pgfsetfillopacity{0.700000}%
\pgfsetlinewidth{0.501875pt}%
\definecolor{currentstroke}{rgb}{1.000000,1.000000,1.000000}%
\pgfsetstrokecolor{currentstroke}%
\pgfsetstrokeopacity{0.700000}%
\pgfsetdash{}{0pt}%
\pgfpathmoveto{\pgfqpoint{3.782339in}{3.362123in}}%
\pgfpathcurveto{\pgfqpoint{3.795362in}{3.362123in}}{\pgfqpoint{3.807853in}{3.367297in}}{\pgfqpoint{3.817061in}{3.376505in}}%
\pgfpathcurveto{\pgfqpoint{3.826270in}{3.385714in}}{\pgfqpoint{3.831443in}{3.398205in}}{\pgfqpoint{3.831443in}{3.411228in}}%
\pgfpathcurveto{\pgfqpoint{3.831443in}{3.424250in}}{\pgfqpoint{3.826270in}{3.436741in}}{\pgfqpoint{3.817061in}{3.445950in}}%
\pgfpathcurveto{\pgfqpoint{3.807853in}{3.455158in}}{\pgfqpoint{3.795362in}{3.460332in}}{\pgfqpoint{3.782339in}{3.460332in}}%
\pgfpathcurveto{\pgfqpoint{3.769316in}{3.460332in}}{\pgfqpoint{3.756825in}{3.455158in}}{\pgfqpoint{3.747617in}{3.445950in}}%
\pgfpathcurveto{\pgfqpoint{3.738408in}{3.436741in}}{\pgfqpoint{3.733234in}{3.424250in}}{\pgfqpoint{3.733234in}{3.411228in}}%
\pgfpathcurveto{\pgfqpoint{3.733234in}{3.398205in}}{\pgfqpoint{3.738408in}{3.385714in}}{\pgfqpoint{3.747617in}{3.376505in}}%
\pgfpathcurveto{\pgfqpoint{3.756825in}{3.367297in}}{\pgfqpoint{3.769316in}{3.362123in}}{\pgfqpoint{3.782339in}{3.362123in}}%
\pgfpathlineto{\pgfqpoint{3.782339in}{3.362123in}}%
\pgfpathclose%
\pgfusepath{stroke,fill}%
\end{pgfscope}%
\begin{pgfscope}%
\pgfpathrectangle{\pgfqpoint{0.786164in}{0.768110in}}{\pgfqpoint{8.851069in}{7.081890in}}%
\pgfusepath{clip}%
\pgfsetbuttcap%
\pgfsetroundjoin%
\definecolor{currentfill}{rgb}{0.276194,0.190074,0.493001}%
\pgfsetfillcolor{currentfill}%
\pgfsetfillopacity{0.700000}%
\pgfsetlinewidth{0.501875pt}%
\definecolor{currentstroke}{rgb}{1.000000,1.000000,1.000000}%
\pgfsetstrokecolor{currentstroke}%
\pgfsetstrokeopacity{0.700000}%
\pgfsetdash{}{0pt}%
\pgfpathmoveto{\pgfqpoint{3.764072in}{3.362123in}}%
\pgfpathcurveto{\pgfqpoint{3.777095in}{3.362123in}}{\pgfqpoint{3.789586in}{3.367297in}}{\pgfqpoint{3.798794in}{3.376505in}}%
\pgfpathcurveto{\pgfqpoint{3.808003in}{3.385714in}}{\pgfqpoint{3.813177in}{3.398205in}}{\pgfqpoint{3.813177in}{3.411228in}}%
\pgfpathcurveto{\pgfqpoint{3.813177in}{3.424250in}}{\pgfqpoint{3.808003in}{3.436741in}}{\pgfqpoint{3.798794in}{3.445950in}}%
\pgfpathcurveto{\pgfqpoint{3.789586in}{3.455158in}}{\pgfqpoint{3.777095in}{3.460332in}}{\pgfqpoint{3.764072in}{3.460332in}}%
\pgfpathcurveto{\pgfqpoint{3.751050in}{3.460332in}}{\pgfqpoint{3.738558in}{3.455158in}}{\pgfqpoint{3.729350in}{3.445950in}}%
\pgfpathcurveto{\pgfqpoint{3.720142in}{3.436741in}}{\pgfqpoint{3.714968in}{3.424250in}}{\pgfqpoint{3.714968in}{3.411228in}}%
\pgfpathcurveto{\pgfqpoint{3.714968in}{3.398205in}}{\pgfqpoint{3.720142in}{3.385714in}}{\pgfqpoint{3.729350in}{3.376505in}}%
\pgfpathcurveto{\pgfqpoint{3.738558in}{3.367297in}}{\pgfqpoint{3.751050in}{3.362123in}}{\pgfqpoint{3.764072in}{3.362123in}}%
\pgfpathlineto{\pgfqpoint{3.764072in}{3.362123in}}%
\pgfpathclose%
\pgfusepath{stroke,fill}%
\end{pgfscope}%
\begin{pgfscope}%
\pgfpathrectangle{\pgfqpoint{0.786164in}{0.768110in}}{\pgfqpoint{8.851069in}{7.081890in}}%
\pgfusepath{clip}%
\pgfsetbuttcap%
\pgfsetroundjoin%
\definecolor{currentfill}{rgb}{0.275191,0.194905,0.496005}%
\pgfsetfillcolor{currentfill}%
\pgfsetfillopacity{0.700000}%
\pgfsetlinewidth{0.501875pt}%
\definecolor{currentstroke}{rgb}{1.000000,1.000000,1.000000}%
\pgfsetstrokecolor{currentstroke}%
\pgfsetstrokeopacity{0.700000}%
\pgfsetdash{}{0pt}%
\pgfpathmoveto{\pgfqpoint{3.590540in}{3.362123in}}%
\pgfpathcurveto{\pgfqpoint{3.603563in}{3.362123in}}{\pgfqpoint{3.616054in}{3.367297in}}{\pgfqpoint{3.625262in}{3.376505in}}%
\pgfpathcurveto{\pgfqpoint{3.634470in}{3.385714in}}{\pgfqpoint{3.639644in}{3.398205in}}{\pgfqpoint{3.639644in}{3.411228in}}%
\pgfpathcurveto{\pgfqpoint{3.639644in}{3.424250in}}{\pgfqpoint{3.634470in}{3.436741in}}{\pgfqpoint{3.625262in}{3.445950in}}%
\pgfpathcurveto{\pgfqpoint{3.616054in}{3.455158in}}{\pgfqpoint{3.603563in}{3.460332in}}{\pgfqpoint{3.590540in}{3.460332in}}%
\pgfpathcurveto{\pgfqpoint{3.577517in}{3.460332in}}{\pgfqpoint{3.565026in}{3.455158in}}{\pgfqpoint{3.555818in}{3.445950in}}%
\pgfpathcurveto{\pgfqpoint{3.546609in}{3.436741in}}{\pgfqpoint{3.541435in}{3.424250in}}{\pgfqpoint{3.541435in}{3.411228in}}%
\pgfpathcurveto{\pgfqpoint{3.541435in}{3.398205in}}{\pgfqpoint{3.546609in}{3.385714in}}{\pgfqpoint{3.555818in}{3.376505in}}%
\pgfpathcurveto{\pgfqpoint{3.565026in}{3.367297in}}{\pgfqpoint{3.577517in}{3.362123in}}{\pgfqpoint{3.590540in}{3.362123in}}%
\pgfpathlineto{\pgfqpoint{3.590540in}{3.362123in}}%
\pgfpathclose%
\pgfusepath{stroke,fill}%
\end{pgfscope}%
\begin{pgfscope}%
\pgfpathrectangle{\pgfqpoint{0.786164in}{0.768110in}}{\pgfqpoint{8.851069in}{7.081890in}}%
\pgfusepath{clip}%
\pgfsetbuttcap%
\pgfsetroundjoin%
\definecolor{currentfill}{rgb}{0.273006,0.204520,0.501721}%
\pgfsetfillcolor{currentfill}%
\pgfsetfillopacity{0.700000}%
\pgfsetlinewidth{0.501875pt}%
\definecolor{currentstroke}{rgb}{1.000000,1.000000,1.000000}%
\pgfsetstrokecolor{currentstroke}%
\pgfsetstrokeopacity{0.700000}%
\pgfsetdash{}{0pt}%
\pgfpathmoveto{\pgfqpoint{3.718406in}{3.340225in}}%
\pgfpathcurveto{\pgfqpoint{3.731429in}{3.340225in}}{\pgfqpoint{3.743920in}{3.345399in}}{\pgfqpoint{3.753128in}{3.354607in}}%
\pgfpathcurveto{\pgfqpoint{3.762336in}{3.363816in}}{\pgfqpoint{3.767510in}{3.376307in}}{\pgfqpoint{3.767510in}{3.389329in}}%
\pgfpathcurveto{\pgfqpoint{3.767510in}{3.402352in}}{\pgfqpoint{3.762336in}{3.414843in}}{\pgfqpoint{3.753128in}{3.424052in}}%
\pgfpathcurveto{\pgfqpoint{3.743920in}{3.433260in}}{\pgfqpoint{3.731429in}{3.438434in}}{\pgfqpoint{3.718406in}{3.438434in}}%
\pgfpathcurveto{\pgfqpoint{3.705383in}{3.438434in}}{\pgfqpoint{3.692892in}{3.433260in}}{\pgfqpoint{3.683684in}{3.424052in}}%
\pgfpathcurveto{\pgfqpoint{3.674475in}{3.414843in}}{\pgfqpoint{3.669301in}{3.402352in}}{\pgfqpoint{3.669301in}{3.389329in}}%
\pgfpathcurveto{\pgfqpoint{3.669301in}{3.376307in}}{\pgfqpoint{3.674475in}{3.363816in}}{\pgfqpoint{3.683684in}{3.354607in}}%
\pgfpathcurveto{\pgfqpoint{3.692892in}{3.345399in}}{\pgfqpoint{3.705383in}{3.340225in}}{\pgfqpoint{3.718406in}{3.340225in}}%
\pgfpathlineto{\pgfqpoint{3.718406in}{3.340225in}}%
\pgfpathclose%
\pgfusepath{stroke,fill}%
\end{pgfscope}%
\begin{pgfscope}%
\pgfpathrectangle{\pgfqpoint{0.786164in}{0.768110in}}{\pgfqpoint{8.851069in}{7.081890in}}%
\pgfusepath{clip}%
\pgfsetbuttcap%
\pgfsetroundjoin%
\definecolor{currentfill}{rgb}{0.255645,0.260703,0.528312}%
\pgfsetfillcolor{currentfill}%
\pgfsetfillopacity{0.700000}%
\pgfsetlinewidth{0.501875pt}%
\definecolor{currentstroke}{rgb}{1.000000,1.000000,1.000000}%
\pgfsetstrokecolor{currentstroke}%
\pgfsetstrokeopacity{0.700000}%
\pgfsetdash{}{0pt}%
\pgfpathmoveto{\pgfqpoint{3.343941in}{3.121242in}}%
\pgfpathcurveto{\pgfqpoint{3.356964in}{3.121242in}}{\pgfqpoint{3.369455in}{3.126416in}}{\pgfqpoint{3.378663in}{3.135625in}}%
\pgfpathcurveto{\pgfqpoint{3.387872in}{3.144833in}}{\pgfqpoint{3.393046in}{3.157324in}}{\pgfqpoint{3.393046in}{3.170347in}}%
\pgfpathcurveto{\pgfqpoint{3.393046in}{3.183370in}}{\pgfqpoint{3.387872in}{3.195861in}}{\pgfqpoint{3.378663in}{3.205069in}}%
\pgfpathcurveto{\pgfqpoint{3.369455in}{3.214278in}}{\pgfqpoint{3.356964in}{3.219452in}}{\pgfqpoint{3.343941in}{3.219452in}}%
\pgfpathcurveto{\pgfqpoint{3.330918in}{3.219452in}}{\pgfqpoint{3.318427in}{3.214278in}}{\pgfqpoint{3.309219in}{3.205069in}}%
\pgfpathcurveto{\pgfqpoint{3.300010in}{3.195861in}}{\pgfqpoint{3.294836in}{3.183370in}}{\pgfqpoint{3.294836in}{3.170347in}}%
\pgfpathcurveto{\pgfqpoint{3.294836in}{3.157324in}}{\pgfqpoint{3.300010in}{3.144833in}}{\pgfqpoint{3.309219in}{3.135625in}}%
\pgfpathcurveto{\pgfqpoint{3.318427in}{3.126416in}}{\pgfqpoint{3.330918in}{3.121242in}}{\pgfqpoint{3.343941in}{3.121242in}}%
\pgfpathlineto{\pgfqpoint{3.343941in}{3.121242in}}%
\pgfpathclose%
\pgfusepath{stroke,fill}%
\end{pgfscope}%
\begin{pgfscope}%
\pgfpathrectangle{\pgfqpoint{0.786164in}{0.768110in}}{\pgfqpoint{8.851069in}{7.081890in}}%
\pgfusepath{clip}%
\pgfsetbuttcap%
\pgfsetroundjoin%
\definecolor{currentfill}{rgb}{0.255645,0.260703,0.528312}%
\pgfsetfillcolor{currentfill}%
\pgfsetfillopacity{0.700000}%
\pgfsetlinewidth{0.501875pt}%
\definecolor{currentstroke}{rgb}{1.000000,1.000000,1.000000}%
\pgfsetstrokecolor{currentstroke}%
\pgfsetstrokeopacity{0.700000}%
\pgfsetdash{}{0pt}%
\pgfpathmoveto{\pgfqpoint{3.718406in}{3.186937in}}%
\pgfpathcurveto{\pgfqpoint{3.731429in}{3.186937in}}{\pgfqpoint{3.743920in}{3.192111in}}{\pgfqpoint{3.753128in}{3.201319in}}%
\pgfpathcurveto{\pgfqpoint{3.762336in}{3.210528in}}{\pgfqpoint{3.767510in}{3.223019in}}{\pgfqpoint{3.767510in}{3.236042in}}%
\pgfpathcurveto{\pgfqpoint{3.767510in}{3.249064in}}{\pgfqpoint{3.762336in}{3.261555in}}{\pgfqpoint{3.753128in}{3.270764in}}%
\pgfpathcurveto{\pgfqpoint{3.743920in}{3.279972in}}{\pgfqpoint{3.731429in}{3.285146in}}{\pgfqpoint{3.718406in}{3.285146in}}%
\pgfpathcurveto{\pgfqpoint{3.705383in}{3.285146in}}{\pgfqpoint{3.692892in}{3.279972in}}{\pgfqpoint{3.683684in}{3.270764in}}%
\pgfpathcurveto{\pgfqpoint{3.674475in}{3.261555in}}{\pgfqpoint{3.669301in}{3.249064in}}{\pgfqpoint{3.669301in}{3.236042in}}%
\pgfpathcurveto{\pgfqpoint{3.669301in}{3.223019in}}{\pgfqpoint{3.674475in}{3.210528in}}{\pgfqpoint{3.683684in}{3.201319in}}%
\pgfpathcurveto{\pgfqpoint{3.692892in}{3.192111in}}{\pgfqpoint{3.705383in}{3.186937in}}{\pgfqpoint{3.718406in}{3.186937in}}%
\pgfpathlineto{\pgfqpoint{3.718406in}{3.186937in}}%
\pgfpathclose%
\pgfusepath{stroke,fill}%
\end{pgfscope}%
\begin{pgfscope}%
\pgfpathrectangle{\pgfqpoint{0.786164in}{0.768110in}}{\pgfqpoint{8.851069in}{7.081890in}}%
\pgfusepath{clip}%
\pgfsetbuttcap%
\pgfsetroundjoin%
\definecolor{currentfill}{rgb}{0.250425,0.274290,0.533103}%
\pgfsetfillcolor{currentfill}%
\pgfsetfillopacity{0.700000}%
\pgfsetlinewidth{0.501875pt}%
\definecolor{currentstroke}{rgb}{1.000000,1.000000,1.000000}%
\pgfsetstrokecolor{currentstroke}%
\pgfsetstrokeopacity{0.700000}%
\pgfsetdash{}{0pt}%
\pgfpathmoveto{\pgfqpoint{3.407874in}{3.055548in}}%
\pgfpathcurveto{\pgfqpoint{3.420897in}{3.055548in}}{\pgfqpoint{3.433388in}{3.060722in}}{\pgfqpoint{3.442596in}{3.069930in}}%
\pgfpathcurveto{\pgfqpoint{3.451805in}{3.079138in}}{\pgfqpoint{3.456979in}{3.091630in}}{\pgfqpoint{3.456979in}{3.104652in}}%
\pgfpathcurveto{\pgfqpoint{3.456979in}{3.117675in}}{\pgfqpoint{3.451805in}{3.130166in}}{\pgfqpoint{3.442596in}{3.139374in}}%
\pgfpathcurveto{\pgfqpoint{3.433388in}{3.148583in}}{\pgfqpoint{3.420897in}{3.153757in}}{\pgfqpoint{3.407874in}{3.153757in}}%
\pgfpathcurveto{\pgfqpoint{3.394851in}{3.153757in}}{\pgfqpoint{3.382360in}{3.148583in}}{\pgfqpoint{3.373152in}{3.139374in}}%
\pgfpathcurveto{\pgfqpoint{3.363943in}{3.130166in}}{\pgfqpoint{3.358769in}{3.117675in}}{\pgfqpoint{3.358769in}{3.104652in}}%
\pgfpathcurveto{\pgfqpoint{3.358769in}{3.091630in}}{\pgfqpoint{3.363943in}{3.079138in}}{\pgfqpoint{3.373152in}{3.069930in}}%
\pgfpathcurveto{\pgfqpoint{3.382360in}{3.060722in}}{\pgfqpoint{3.394851in}{3.055548in}}{\pgfqpoint{3.407874in}{3.055548in}}%
\pgfpathlineto{\pgfqpoint{3.407874in}{3.055548in}}%
\pgfpathclose%
\pgfusepath{stroke,fill}%
\end{pgfscope}%
\begin{pgfscope}%
\pgfpathrectangle{\pgfqpoint{0.786164in}{0.768110in}}{\pgfqpoint{8.851069in}{7.081890in}}%
\pgfusepath{clip}%
\pgfsetbuttcap%
\pgfsetroundjoin%
\definecolor{currentfill}{rgb}{0.276194,0.190074,0.493001}%
\pgfsetfillcolor{currentfill}%
\pgfsetfillopacity{0.700000}%
\pgfsetlinewidth{0.501875pt}%
\definecolor{currentstroke}{rgb}{1.000000,1.000000,1.000000}%
\pgfsetstrokecolor{currentstroke}%
\pgfsetstrokeopacity{0.700000}%
\pgfsetdash{}{0pt}%
\pgfpathmoveto{\pgfqpoint{2.028748in}{2.354804in}}%
\pgfpathcurveto{\pgfqpoint{2.041770in}{2.354804in}}{\pgfqpoint{2.054261in}{2.359978in}}{\pgfqpoint{2.063470in}{2.369186in}}%
\pgfpathcurveto{\pgfqpoint{2.072678in}{2.378395in}}{\pgfqpoint{2.077852in}{2.390886in}}{\pgfqpoint{2.077852in}{2.403909in}}%
\pgfpathcurveto{\pgfqpoint{2.077852in}{2.416931in}}{\pgfqpoint{2.072678in}{2.429422in}}{\pgfqpoint{2.063470in}{2.438631in}}%
\pgfpathcurveto{\pgfqpoint{2.054261in}{2.447839in}}{\pgfqpoint{2.041770in}{2.453013in}}{\pgfqpoint{2.028748in}{2.453013in}}%
\pgfpathcurveto{\pgfqpoint{2.015725in}{2.453013in}}{\pgfqpoint{2.003234in}{2.447839in}}{\pgfqpoint{1.994025in}{2.438631in}}%
\pgfpathcurveto{\pgfqpoint{1.984817in}{2.429422in}}{\pgfqpoint{1.979643in}{2.416931in}}{\pgfqpoint{1.979643in}{2.403909in}}%
\pgfpathcurveto{\pgfqpoint{1.979643in}{2.390886in}}{\pgfqpoint{1.984817in}{2.378395in}}{\pgfqpoint{1.994025in}{2.369186in}}%
\pgfpathcurveto{\pgfqpoint{2.003234in}{2.359978in}}{\pgfqpoint{2.015725in}{2.354804in}}{\pgfqpoint{2.028748in}{2.354804in}}%
\pgfpathlineto{\pgfqpoint{2.028748in}{2.354804in}}%
\pgfpathclose%
\pgfusepath{stroke,fill}%
\end{pgfscope}%
\begin{pgfscope}%
\pgfpathrectangle{\pgfqpoint{0.786164in}{0.768110in}}{\pgfqpoint{8.851069in}{7.081890in}}%
\pgfusepath{clip}%
\pgfsetbuttcap%
\pgfsetroundjoin%
\definecolor{currentfill}{rgb}{0.276194,0.190074,0.493001}%
\pgfsetfillcolor{currentfill}%
\pgfsetfillopacity{0.700000}%
\pgfsetlinewidth{0.501875pt}%
\definecolor{currentstroke}{rgb}{1.000000,1.000000,1.000000}%
\pgfsetstrokecolor{currentstroke}%
\pgfsetstrokeopacity{0.700000}%
\pgfsetdash{}{0pt}%
\pgfpathmoveto{\pgfqpoint{2.165747in}{2.354804in}}%
\pgfpathcurveto{\pgfqpoint{2.178770in}{2.354804in}}{\pgfqpoint{2.191261in}{2.359978in}}{\pgfqpoint{2.200469in}{2.369186in}}%
\pgfpathcurveto{\pgfqpoint{2.209678in}{2.378395in}}{\pgfqpoint{2.214852in}{2.390886in}}{\pgfqpoint{2.214852in}{2.403909in}}%
\pgfpathcurveto{\pgfqpoint{2.214852in}{2.416931in}}{\pgfqpoint{2.209678in}{2.429422in}}{\pgfqpoint{2.200469in}{2.438631in}}%
\pgfpathcurveto{\pgfqpoint{2.191261in}{2.447839in}}{\pgfqpoint{2.178770in}{2.453013in}}{\pgfqpoint{2.165747in}{2.453013in}}%
\pgfpathcurveto{\pgfqpoint{2.152724in}{2.453013in}}{\pgfqpoint{2.140233in}{2.447839in}}{\pgfqpoint{2.131025in}{2.438631in}}%
\pgfpathcurveto{\pgfqpoint{2.121816in}{2.429422in}}{\pgfqpoint{2.116642in}{2.416931in}}{\pgfqpoint{2.116642in}{2.403909in}}%
\pgfpathcurveto{\pgfqpoint{2.116642in}{2.390886in}}{\pgfqpoint{2.121816in}{2.378395in}}{\pgfqpoint{2.131025in}{2.369186in}}%
\pgfpathcurveto{\pgfqpoint{2.140233in}{2.359978in}}{\pgfqpoint{2.152724in}{2.354804in}}{\pgfqpoint{2.165747in}{2.354804in}}%
\pgfpathlineto{\pgfqpoint{2.165747in}{2.354804in}}%
\pgfpathclose%
\pgfusepath{stroke,fill}%
\end{pgfscope}%
\begin{pgfscope}%
\pgfpathrectangle{\pgfqpoint{0.786164in}{0.768110in}}{\pgfqpoint{8.851069in}{7.081890in}}%
\pgfusepath{clip}%
\pgfsetbuttcap%
\pgfsetroundjoin%
\definecolor{currentfill}{rgb}{0.275191,0.194905,0.496005}%
\pgfsetfillcolor{currentfill}%
\pgfsetfillopacity{0.700000}%
\pgfsetlinewidth{0.501875pt}%
\definecolor{currentstroke}{rgb}{1.000000,1.000000,1.000000}%
\pgfsetstrokecolor{currentstroke}%
\pgfsetstrokeopacity{0.700000}%
\pgfsetdash{}{0pt}%
\pgfpathmoveto{\pgfqpoint{2.257080in}{2.486193in}}%
\pgfpathcurveto{\pgfqpoint{2.270103in}{2.486193in}}{\pgfqpoint{2.282594in}{2.491367in}}{\pgfqpoint{2.291802in}{2.500576in}}%
\pgfpathcurveto{\pgfqpoint{2.301010in}{2.509784in}}{\pgfqpoint{2.306184in}{2.522275in}}{\pgfqpoint{2.306184in}{2.535298in}}%
\pgfpathcurveto{\pgfqpoint{2.306184in}{2.548321in}}{\pgfqpoint{2.301010in}{2.560812in}}{\pgfqpoint{2.291802in}{2.570020in}}%
\pgfpathcurveto{\pgfqpoint{2.282594in}{2.579229in}}{\pgfqpoint{2.270103in}{2.584403in}}{\pgfqpoint{2.257080in}{2.584403in}}%
\pgfpathcurveto{\pgfqpoint{2.244057in}{2.584403in}}{\pgfqpoint{2.231566in}{2.579229in}}{\pgfqpoint{2.222358in}{2.570020in}}%
\pgfpathcurveto{\pgfqpoint{2.213149in}{2.560812in}}{\pgfqpoint{2.207975in}{2.548321in}}{\pgfqpoint{2.207975in}{2.535298in}}%
\pgfpathcurveto{\pgfqpoint{2.207975in}{2.522275in}}{\pgfqpoint{2.213149in}{2.509784in}}{\pgfqpoint{2.222358in}{2.500576in}}%
\pgfpathcurveto{\pgfqpoint{2.231566in}{2.491367in}}{\pgfqpoint{2.244057in}{2.486193in}}{\pgfqpoint{2.257080in}{2.486193in}}%
\pgfpathlineto{\pgfqpoint{2.257080in}{2.486193in}}%
\pgfpathclose%
\pgfusepath{stroke,fill}%
\end{pgfscope}%
\begin{pgfscope}%
\pgfpathrectangle{\pgfqpoint{0.786164in}{0.768110in}}{\pgfqpoint{8.851069in}{7.081890in}}%
\pgfusepath{clip}%
\pgfsetbuttcap%
\pgfsetroundjoin%
\definecolor{currentfill}{rgb}{0.277134,0.185228,0.489898}%
\pgfsetfillcolor{currentfill}%
\pgfsetfillopacity{0.700000}%
\pgfsetlinewidth{0.501875pt}%
\definecolor{currentstroke}{rgb}{1.000000,1.000000,1.000000}%
\pgfsetstrokecolor{currentstroke}%
\pgfsetstrokeopacity{0.700000}%
\pgfsetdash{}{0pt}%
\pgfpathmoveto{\pgfqpoint{2.238813in}{2.573786in}}%
\pgfpathcurveto{\pgfqpoint{2.251836in}{2.573786in}}{\pgfqpoint{2.264327in}{2.578960in}}{\pgfqpoint{2.273535in}{2.588169in}}%
\pgfpathcurveto{\pgfqpoint{2.282744in}{2.597377in}}{\pgfqpoint{2.287918in}{2.609868in}}{\pgfqpoint{2.287918in}{2.622891in}}%
\pgfpathcurveto{\pgfqpoint{2.287918in}{2.635914in}}{\pgfqpoint{2.282744in}{2.648405in}}{\pgfqpoint{2.273535in}{2.657613in}}%
\pgfpathcurveto{\pgfqpoint{2.264327in}{2.666822in}}{\pgfqpoint{2.251836in}{2.671996in}}{\pgfqpoint{2.238813in}{2.671996in}}%
\pgfpathcurveto{\pgfqpoint{2.225791in}{2.671996in}}{\pgfqpoint{2.213299in}{2.666822in}}{\pgfqpoint{2.204091in}{2.657613in}}%
\pgfpathcurveto{\pgfqpoint{2.194883in}{2.648405in}}{\pgfqpoint{2.189709in}{2.635914in}}{\pgfqpoint{2.189709in}{2.622891in}}%
\pgfpathcurveto{\pgfqpoint{2.189709in}{2.609868in}}{\pgfqpoint{2.194883in}{2.597377in}}{\pgfqpoint{2.204091in}{2.588169in}}%
\pgfpathcurveto{\pgfqpoint{2.213299in}{2.578960in}}{\pgfqpoint{2.225791in}{2.573786in}}{\pgfqpoint{2.238813in}{2.573786in}}%
\pgfpathlineto{\pgfqpoint{2.238813in}{2.573786in}}%
\pgfpathclose%
\pgfusepath{stroke,fill}%
\end{pgfscope}%
\begin{pgfscope}%
\pgfpathrectangle{\pgfqpoint{0.786164in}{0.768110in}}{\pgfqpoint{8.851069in}{7.081890in}}%
\pgfusepath{clip}%
\pgfsetbuttcap%
\pgfsetroundjoin%
\definecolor{currentfill}{rgb}{0.271828,0.209303,0.504434}%
\pgfsetfillcolor{currentfill}%
\pgfsetfillopacity{0.700000}%
\pgfsetlinewidth{0.501875pt}%
\definecolor{currentstroke}{rgb}{1.000000,1.000000,1.000000}%
\pgfsetstrokecolor{currentstroke}%
\pgfsetstrokeopacity{0.700000}%
\pgfsetdash{}{0pt}%
\pgfpathmoveto{\pgfqpoint{2.193147in}{2.595685in}}%
\pgfpathcurveto{\pgfqpoint{2.206170in}{2.595685in}}{\pgfqpoint{2.218661in}{2.600859in}}{\pgfqpoint{2.227869in}{2.610067in}}%
\pgfpathcurveto{\pgfqpoint{2.237077in}{2.619275in}}{\pgfqpoint{2.242251in}{2.631767in}}{\pgfqpoint{2.242251in}{2.644789in}}%
\pgfpathcurveto{\pgfqpoint{2.242251in}{2.657812in}}{\pgfqpoint{2.237077in}{2.670303in}}{\pgfqpoint{2.227869in}{2.679511in}}%
\pgfpathcurveto{\pgfqpoint{2.218661in}{2.688720in}}{\pgfqpoint{2.206170in}{2.693894in}}{\pgfqpoint{2.193147in}{2.693894in}}%
\pgfpathcurveto{\pgfqpoint{2.180124in}{2.693894in}}{\pgfqpoint{2.167633in}{2.688720in}}{\pgfqpoint{2.158425in}{2.679511in}}%
\pgfpathcurveto{\pgfqpoint{2.149216in}{2.670303in}}{\pgfqpoint{2.144042in}{2.657812in}}{\pgfqpoint{2.144042in}{2.644789in}}%
\pgfpathcurveto{\pgfqpoint{2.144042in}{2.631767in}}{\pgfqpoint{2.149216in}{2.619275in}}{\pgfqpoint{2.158425in}{2.610067in}}%
\pgfpathcurveto{\pgfqpoint{2.167633in}{2.600859in}}{\pgfqpoint{2.180124in}{2.595685in}}{\pgfqpoint{2.193147in}{2.595685in}}%
\pgfpathlineto{\pgfqpoint{2.193147in}{2.595685in}}%
\pgfpathclose%
\pgfusepath{stroke,fill}%
\end{pgfscope}%
\begin{pgfscope}%
\pgfpathrectangle{\pgfqpoint{0.786164in}{0.768110in}}{\pgfqpoint{8.851069in}{7.081890in}}%
\pgfusepath{clip}%
\pgfsetbuttcap%
\pgfsetroundjoin%
\definecolor{currentfill}{rgb}{0.262138,0.242286,0.520837}%
\pgfsetfillcolor{currentfill}%
\pgfsetfillopacity{0.700000}%
\pgfsetlinewidth{0.501875pt}%
\definecolor{currentstroke}{rgb}{1.000000,1.000000,1.000000}%
\pgfsetstrokecolor{currentstroke}%
\pgfsetstrokeopacity{0.700000}%
\pgfsetdash{}{0pt}%
\pgfpathmoveto{\pgfqpoint{1.946548in}{2.332906in}}%
\pgfpathcurveto{\pgfqpoint{1.959571in}{2.332906in}}{\pgfqpoint{1.972062in}{2.338080in}}{\pgfqpoint{1.981270in}{2.347288in}}%
\pgfpathcurveto{\pgfqpoint{1.990479in}{2.356497in}}{\pgfqpoint{1.995653in}{2.368988in}}{\pgfqpoint{1.995653in}{2.382010in}}%
\pgfpathcurveto{\pgfqpoint{1.995653in}{2.395033in}}{\pgfqpoint{1.990479in}{2.407524in}}{\pgfqpoint{1.981270in}{2.416733in}}%
\pgfpathcurveto{\pgfqpoint{1.972062in}{2.425941in}}{\pgfqpoint{1.959571in}{2.431115in}}{\pgfqpoint{1.946548in}{2.431115in}}%
\pgfpathcurveto{\pgfqpoint{1.933525in}{2.431115in}}{\pgfqpoint{1.921034in}{2.425941in}}{\pgfqpoint{1.911826in}{2.416733in}}%
\pgfpathcurveto{\pgfqpoint{1.902617in}{2.407524in}}{\pgfqpoint{1.897443in}{2.395033in}}{\pgfqpoint{1.897443in}{2.382010in}}%
\pgfpathcurveto{\pgfqpoint{1.897443in}{2.368988in}}{\pgfqpoint{1.902617in}{2.356497in}}{\pgfqpoint{1.911826in}{2.347288in}}%
\pgfpathcurveto{\pgfqpoint{1.921034in}{2.338080in}}{\pgfqpoint{1.933525in}{2.332906in}}{\pgfqpoint{1.946548in}{2.332906in}}%
\pgfpathlineto{\pgfqpoint{1.946548in}{2.332906in}}%
\pgfpathclose%
\pgfusepath{stroke,fill}%
\end{pgfscope}%
\begin{pgfscope}%
\pgfpathrectangle{\pgfqpoint{0.786164in}{0.768110in}}{\pgfqpoint{8.851069in}{7.081890in}}%
\pgfusepath{clip}%
\pgfsetbuttcap%
\pgfsetroundjoin%
\definecolor{currentfill}{rgb}{0.250425,0.274290,0.533103}%
\pgfsetfillcolor{currentfill}%
\pgfsetfillopacity{0.700000}%
\pgfsetlinewidth{0.501875pt}%
\definecolor{currentstroke}{rgb}{1.000000,1.000000,1.000000}%
\pgfsetstrokecolor{currentstroke}%
\pgfsetstrokeopacity{0.700000}%
\pgfsetdash{}{0pt}%
\pgfpathmoveto{\pgfqpoint{2.019614in}{2.442397in}}%
\pgfpathcurveto{\pgfqpoint{2.032637in}{2.442397in}}{\pgfqpoint{2.045128in}{2.447571in}}{\pgfqpoint{2.054337in}{2.456779in}}%
\pgfpathcurveto{\pgfqpoint{2.063545in}{2.465988in}}{\pgfqpoint{2.068719in}{2.478479in}}{\pgfqpoint{2.068719in}{2.491502in}}%
\pgfpathcurveto{\pgfqpoint{2.068719in}{2.504524in}}{\pgfqpoint{2.063545in}{2.517015in}}{\pgfqpoint{2.054337in}{2.526224in}}%
\pgfpathcurveto{\pgfqpoint{2.045128in}{2.535432in}}{\pgfqpoint{2.032637in}{2.540606in}}{\pgfqpoint{2.019614in}{2.540606in}}%
\pgfpathcurveto{\pgfqpoint{2.006592in}{2.540606in}}{\pgfqpoint{1.994101in}{2.535432in}}{\pgfqpoint{1.984892in}{2.526224in}}%
\pgfpathcurveto{\pgfqpoint{1.975684in}{2.517015in}}{\pgfqpoint{1.970510in}{2.504524in}}{\pgfqpoint{1.970510in}{2.491502in}}%
\pgfpathcurveto{\pgfqpoint{1.970510in}{2.478479in}}{\pgfqpoint{1.975684in}{2.465988in}}{\pgfqpoint{1.984892in}{2.456779in}}%
\pgfpathcurveto{\pgfqpoint{1.994101in}{2.447571in}}{\pgfqpoint{2.006592in}{2.442397in}}{\pgfqpoint{2.019614in}{2.442397in}}%
\pgfpathlineto{\pgfqpoint{2.019614in}{2.442397in}}%
\pgfpathclose%
\pgfusepath{stroke,fill}%
\end{pgfscope}%
\begin{pgfscope}%
\pgfpathrectangle{\pgfqpoint{0.786164in}{0.768110in}}{\pgfqpoint{8.851069in}{7.081890in}}%
\pgfusepath{clip}%
\pgfsetbuttcap%
\pgfsetroundjoin%
\definecolor{currentfill}{rgb}{0.248629,0.278775,0.534556}%
\pgfsetfillcolor{currentfill}%
\pgfsetfillopacity{0.700000}%
\pgfsetlinewidth{0.501875pt}%
\definecolor{currentstroke}{rgb}{1.000000,1.000000,1.000000}%
\pgfsetstrokecolor{currentstroke}%
\pgfsetstrokeopacity{0.700000}%
\pgfsetdash{}{0pt}%
\pgfpathmoveto{\pgfqpoint{2.184014in}{2.705176in}}%
\pgfpathcurveto{\pgfqpoint{2.197036in}{2.705176in}}{\pgfqpoint{2.209527in}{2.710350in}}{\pgfqpoint{2.218736in}{2.719558in}}%
\pgfpathcurveto{\pgfqpoint{2.227944in}{2.728767in}}{\pgfqpoint{2.233118in}{2.741258in}}{\pgfqpoint{2.233118in}{2.754280in}}%
\pgfpathcurveto{\pgfqpoint{2.233118in}{2.767303in}}{\pgfqpoint{2.227944in}{2.779794in}}{\pgfqpoint{2.218736in}{2.789003in}}%
\pgfpathcurveto{\pgfqpoint{2.209527in}{2.798211in}}{\pgfqpoint{2.197036in}{2.803385in}}{\pgfqpoint{2.184014in}{2.803385in}}%
\pgfpathcurveto{\pgfqpoint{2.170991in}{2.803385in}}{\pgfqpoint{2.158500in}{2.798211in}}{\pgfqpoint{2.149291in}{2.789003in}}%
\pgfpathcurveto{\pgfqpoint{2.140083in}{2.779794in}}{\pgfqpoint{2.134909in}{2.767303in}}{\pgfqpoint{2.134909in}{2.754280in}}%
\pgfpathcurveto{\pgfqpoint{2.134909in}{2.741258in}}{\pgfqpoint{2.140083in}{2.728767in}}{\pgfqpoint{2.149291in}{2.719558in}}%
\pgfpathcurveto{\pgfqpoint{2.158500in}{2.710350in}}{\pgfqpoint{2.170991in}{2.705176in}}{\pgfqpoint{2.184014in}{2.705176in}}%
\pgfpathlineto{\pgfqpoint{2.184014in}{2.705176in}}%
\pgfpathclose%
\pgfusepath{stroke,fill}%
\end{pgfscope}%
\begin{pgfscope}%
\pgfpathrectangle{\pgfqpoint{0.786164in}{0.768110in}}{\pgfqpoint{8.851069in}{7.081890in}}%
\pgfusepath{clip}%
\pgfsetbuttcap%
\pgfsetroundjoin%
\definecolor{currentfill}{rgb}{0.235526,0.309527,0.542944}%
\pgfsetfillcolor{currentfill}%
\pgfsetfillopacity{0.700000}%
\pgfsetlinewidth{0.501875pt}%
\definecolor{currentstroke}{rgb}{1.000000,1.000000,1.000000}%
\pgfsetstrokecolor{currentstroke}%
\pgfsetstrokeopacity{0.700000}%
\pgfsetdash{}{0pt}%
\pgfpathmoveto{\pgfqpoint{2.101814in}{2.573786in}}%
\pgfpathcurveto{\pgfqpoint{2.114837in}{2.573786in}}{\pgfqpoint{2.127328in}{2.578960in}}{\pgfqpoint{2.136536in}{2.588169in}}%
\pgfpathcurveto{\pgfqpoint{2.145745in}{2.597377in}}{\pgfqpoint{2.150919in}{2.609868in}}{\pgfqpoint{2.150919in}{2.622891in}}%
\pgfpathcurveto{\pgfqpoint{2.150919in}{2.635914in}}{\pgfqpoint{2.145745in}{2.648405in}}{\pgfqpoint{2.136536in}{2.657613in}}%
\pgfpathcurveto{\pgfqpoint{2.127328in}{2.666822in}}{\pgfqpoint{2.114837in}{2.671996in}}{\pgfqpoint{2.101814in}{2.671996in}}%
\pgfpathcurveto{\pgfqpoint{2.088791in}{2.671996in}}{\pgfqpoint{2.076300in}{2.666822in}}{\pgfqpoint{2.067092in}{2.657613in}}%
\pgfpathcurveto{\pgfqpoint{2.057883in}{2.648405in}}{\pgfqpoint{2.052709in}{2.635914in}}{\pgfqpoint{2.052709in}{2.622891in}}%
\pgfpathcurveto{\pgfqpoint{2.052709in}{2.609868in}}{\pgfqpoint{2.057883in}{2.597377in}}{\pgfqpoint{2.067092in}{2.588169in}}%
\pgfpathcurveto{\pgfqpoint{2.076300in}{2.578960in}}{\pgfqpoint{2.088791in}{2.573786in}}{\pgfqpoint{2.101814in}{2.573786in}}%
\pgfpathlineto{\pgfqpoint{2.101814in}{2.573786in}}%
\pgfpathclose%
\pgfusepath{stroke,fill}%
\end{pgfscope}%
\begin{pgfscope}%
\pgfpathrectangle{\pgfqpoint{0.786164in}{0.768110in}}{\pgfqpoint{8.851069in}{7.081890in}}%
\pgfusepath{clip}%
\pgfsetbuttcap%
\pgfsetroundjoin%
\definecolor{currentfill}{rgb}{0.212395,0.359683,0.551710}%
\pgfsetfillcolor{currentfill}%
\pgfsetfillopacity{0.700000}%
\pgfsetlinewidth{0.501875pt}%
\definecolor{currentstroke}{rgb}{1.000000,1.000000,1.000000}%
\pgfsetstrokecolor{currentstroke}%
\pgfsetstrokeopacity{0.700000}%
\pgfsetdash{}{0pt}%
\pgfpathmoveto{\pgfqpoint{1.946548in}{2.464295in}}%
\pgfpathcurveto{\pgfqpoint{1.959571in}{2.464295in}}{\pgfqpoint{1.972062in}{2.469469in}}{\pgfqpoint{1.981270in}{2.478678in}}%
\pgfpathcurveto{\pgfqpoint{1.990479in}{2.487886in}}{\pgfqpoint{1.995653in}{2.500377in}}{\pgfqpoint{1.995653in}{2.513400in}}%
\pgfpathcurveto{\pgfqpoint{1.995653in}{2.526423in}}{\pgfqpoint{1.990479in}{2.538914in}}{\pgfqpoint{1.981270in}{2.548122in}}%
\pgfpathcurveto{\pgfqpoint{1.972062in}{2.557331in}}{\pgfqpoint{1.959571in}{2.562504in}}{\pgfqpoint{1.946548in}{2.562504in}}%
\pgfpathcurveto{\pgfqpoint{1.933525in}{2.562504in}}{\pgfqpoint{1.921034in}{2.557331in}}{\pgfqpoint{1.911826in}{2.548122in}}%
\pgfpathcurveto{\pgfqpoint{1.902617in}{2.538914in}}{\pgfqpoint{1.897443in}{2.526423in}}{\pgfqpoint{1.897443in}{2.513400in}}%
\pgfpathcurveto{\pgfqpoint{1.897443in}{2.500377in}}{\pgfqpoint{1.902617in}{2.487886in}}{\pgfqpoint{1.911826in}{2.478678in}}%
\pgfpathcurveto{\pgfqpoint{1.921034in}{2.469469in}}{\pgfqpoint{1.933525in}{2.464295in}}{\pgfqpoint{1.946548in}{2.464295in}}%
\pgfpathlineto{\pgfqpoint{1.946548in}{2.464295in}}%
\pgfpathclose%
\pgfusepath{stroke,fill}%
\end{pgfscope}%
\begin{pgfscope}%
\pgfpathrectangle{\pgfqpoint{0.786164in}{0.768110in}}{\pgfqpoint{8.851069in}{7.081890in}}%
\pgfusepath{clip}%
\pgfsetbuttcap%
\pgfsetroundjoin%
\definecolor{currentfill}{rgb}{0.220057,0.343307,0.549413}%
\pgfsetfillcolor{currentfill}%
\pgfsetfillopacity{0.700000}%
\pgfsetlinewidth{0.501875pt}%
\definecolor{currentstroke}{rgb}{1.000000,1.000000,1.000000}%
\pgfsetstrokecolor{currentstroke}%
\pgfsetstrokeopacity{0.700000}%
\pgfsetdash{}{0pt}%
\pgfpathmoveto{\pgfqpoint{2.056148in}{2.508092in}}%
\pgfpathcurveto{\pgfqpoint{2.069170in}{2.508092in}}{\pgfqpoint{2.081661in}{2.513266in}}{\pgfqpoint{2.090870in}{2.522474in}}%
\pgfpathcurveto{\pgfqpoint{2.100078in}{2.531683in}}{\pgfqpoint{2.105252in}{2.544174in}}{\pgfqpoint{2.105252in}{2.557196in}}%
\pgfpathcurveto{\pgfqpoint{2.105252in}{2.570219in}}{\pgfqpoint{2.100078in}{2.582710in}}{\pgfqpoint{2.090870in}{2.591919in}}%
\pgfpathcurveto{\pgfqpoint{2.081661in}{2.601127in}}{\pgfqpoint{2.069170in}{2.606301in}}{\pgfqpoint{2.056148in}{2.606301in}}%
\pgfpathcurveto{\pgfqpoint{2.043125in}{2.606301in}}{\pgfqpoint{2.030634in}{2.601127in}}{\pgfqpoint{2.021425in}{2.591919in}}%
\pgfpathcurveto{\pgfqpoint{2.012217in}{2.582710in}}{\pgfqpoint{2.007043in}{2.570219in}}{\pgfqpoint{2.007043in}{2.557196in}}%
\pgfpathcurveto{\pgfqpoint{2.007043in}{2.544174in}}{\pgfqpoint{2.012217in}{2.531683in}}{\pgfqpoint{2.021425in}{2.522474in}}%
\pgfpathcurveto{\pgfqpoint{2.030634in}{2.513266in}}{\pgfqpoint{2.043125in}{2.508092in}}{\pgfqpoint{2.056148in}{2.508092in}}%
\pgfpathlineto{\pgfqpoint{2.056148in}{2.508092in}}%
\pgfpathclose%
\pgfusepath{stroke,fill}%
\end{pgfscope}%
\begin{pgfscope}%
\pgfpathrectangle{\pgfqpoint{0.786164in}{0.768110in}}{\pgfqpoint{8.851069in}{7.081890in}}%
\pgfusepath{clip}%
\pgfsetbuttcap%
\pgfsetroundjoin%
\definecolor{currentfill}{rgb}{0.218130,0.347432,0.550038}%
\pgfsetfillcolor{currentfill}%
\pgfsetfillopacity{0.700000}%
\pgfsetlinewidth{0.501875pt}%
\definecolor{currentstroke}{rgb}{1.000000,1.000000,1.000000}%
\pgfsetstrokecolor{currentstroke}%
\pgfsetstrokeopacity{0.700000}%
\pgfsetdash{}{0pt}%
\pgfpathmoveto{\pgfqpoint{2.156614in}{2.639481in}}%
\pgfpathcurveto{\pgfqpoint{2.169636in}{2.639481in}}{\pgfqpoint{2.182127in}{2.644655in}}{\pgfqpoint{2.191336in}{2.653864in}}%
\pgfpathcurveto{\pgfqpoint{2.200544in}{2.663072in}}{\pgfqpoint{2.205718in}{2.675563in}}{\pgfqpoint{2.205718in}{2.688586in}}%
\pgfpathcurveto{\pgfqpoint{2.205718in}{2.701608in}}{\pgfqpoint{2.200544in}{2.714100in}}{\pgfqpoint{2.191336in}{2.723308in}}%
\pgfpathcurveto{\pgfqpoint{2.182127in}{2.732516in}}{\pgfqpoint{2.169636in}{2.737690in}}{\pgfqpoint{2.156614in}{2.737690in}}%
\pgfpathcurveto{\pgfqpoint{2.143591in}{2.737690in}}{\pgfqpoint{2.131100in}{2.732516in}}{\pgfqpoint{2.121891in}{2.723308in}}%
\pgfpathcurveto{\pgfqpoint{2.112683in}{2.714100in}}{\pgfqpoint{2.107509in}{2.701608in}}{\pgfqpoint{2.107509in}{2.688586in}}%
\pgfpathcurveto{\pgfqpoint{2.107509in}{2.675563in}}{\pgfqpoint{2.112683in}{2.663072in}}{\pgfqpoint{2.121891in}{2.653864in}}%
\pgfpathcurveto{\pgfqpoint{2.131100in}{2.644655in}}{\pgfqpoint{2.143591in}{2.639481in}}{\pgfqpoint{2.156614in}{2.639481in}}%
\pgfpathlineto{\pgfqpoint{2.156614in}{2.639481in}}%
\pgfpathclose%
\pgfusepath{stroke,fill}%
\end{pgfscope}%
\begin{pgfscope}%
\pgfpathrectangle{\pgfqpoint{0.786164in}{0.768110in}}{\pgfqpoint{8.851069in}{7.081890in}}%
\pgfusepath{clip}%
\pgfsetbuttcap%
\pgfsetroundjoin%
\definecolor{currentfill}{rgb}{0.214298,0.355619,0.551184}%
\pgfsetfillcolor{currentfill}%
\pgfsetfillopacity{0.700000}%
\pgfsetlinewidth{0.501875pt}%
\definecolor{currentstroke}{rgb}{1.000000,1.000000,1.000000}%
\pgfsetstrokecolor{currentstroke}%
\pgfsetstrokeopacity{0.700000}%
\pgfsetdash{}{0pt}%
\pgfpathmoveto{\pgfqpoint{2.138347in}{2.508092in}}%
\pgfpathcurveto{\pgfqpoint{2.151370in}{2.508092in}}{\pgfqpoint{2.163861in}{2.513266in}}{\pgfqpoint{2.173069in}{2.522474in}}%
\pgfpathcurveto{\pgfqpoint{2.182278in}{2.531683in}}{\pgfqpoint{2.187452in}{2.544174in}}{\pgfqpoint{2.187452in}{2.557196in}}%
\pgfpathcurveto{\pgfqpoint{2.187452in}{2.570219in}}{\pgfqpoint{2.182278in}{2.582710in}}{\pgfqpoint{2.173069in}{2.591919in}}%
\pgfpathcurveto{\pgfqpoint{2.163861in}{2.601127in}}{\pgfqpoint{2.151370in}{2.606301in}}{\pgfqpoint{2.138347in}{2.606301in}}%
\pgfpathcurveto{\pgfqpoint{2.125324in}{2.606301in}}{\pgfqpoint{2.112833in}{2.601127in}}{\pgfqpoint{2.103625in}{2.591919in}}%
\pgfpathcurveto{\pgfqpoint{2.094416in}{2.582710in}}{\pgfqpoint{2.089242in}{2.570219in}}{\pgfqpoint{2.089242in}{2.557196in}}%
\pgfpathcurveto{\pgfqpoint{2.089242in}{2.544174in}}{\pgfqpoint{2.094416in}{2.531683in}}{\pgfqpoint{2.103625in}{2.522474in}}%
\pgfpathcurveto{\pgfqpoint{2.112833in}{2.513266in}}{\pgfqpoint{2.125324in}{2.508092in}}{\pgfqpoint{2.138347in}{2.508092in}}%
\pgfpathlineto{\pgfqpoint{2.138347in}{2.508092in}}%
\pgfpathclose%
\pgfusepath{stroke,fill}%
\end{pgfscope}%
\begin{pgfscope}%
\pgfpathrectangle{\pgfqpoint{0.786164in}{0.768110in}}{\pgfqpoint{8.851069in}{7.081890in}}%
\pgfusepath{clip}%
\pgfsetbuttcap%
\pgfsetroundjoin%
\definecolor{currentfill}{rgb}{0.214298,0.355619,0.551184}%
\pgfsetfillcolor{currentfill}%
\pgfsetfillopacity{0.700000}%
\pgfsetlinewidth{0.501875pt}%
\definecolor{currentstroke}{rgb}{1.000000,1.000000,1.000000}%
\pgfsetstrokecolor{currentstroke}%
\pgfsetstrokeopacity{0.700000}%
\pgfsetdash{}{0pt}%
\pgfpathmoveto{\pgfqpoint{2.238813in}{2.595685in}}%
\pgfpathcurveto{\pgfqpoint{2.251836in}{2.595685in}}{\pgfqpoint{2.264327in}{2.600859in}}{\pgfqpoint{2.273535in}{2.610067in}}%
\pgfpathcurveto{\pgfqpoint{2.282744in}{2.619275in}}{\pgfqpoint{2.287918in}{2.631767in}}{\pgfqpoint{2.287918in}{2.644789in}}%
\pgfpathcurveto{\pgfqpoint{2.287918in}{2.657812in}}{\pgfqpoint{2.282744in}{2.670303in}}{\pgfqpoint{2.273535in}{2.679511in}}%
\pgfpathcurveto{\pgfqpoint{2.264327in}{2.688720in}}{\pgfqpoint{2.251836in}{2.693894in}}{\pgfqpoint{2.238813in}{2.693894in}}%
\pgfpathcurveto{\pgfqpoint{2.225791in}{2.693894in}}{\pgfqpoint{2.213299in}{2.688720in}}{\pgfqpoint{2.204091in}{2.679511in}}%
\pgfpathcurveto{\pgfqpoint{2.194883in}{2.670303in}}{\pgfqpoint{2.189709in}{2.657812in}}{\pgfqpoint{2.189709in}{2.644789in}}%
\pgfpathcurveto{\pgfqpoint{2.189709in}{2.631767in}}{\pgfqpoint{2.194883in}{2.619275in}}{\pgfqpoint{2.204091in}{2.610067in}}%
\pgfpathcurveto{\pgfqpoint{2.213299in}{2.600859in}}{\pgfqpoint{2.225791in}{2.595685in}}{\pgfqpoint{2.238813in}{2.595685in}}%
\pgfpathlineto{\pgfqpoint{2.238813in}{2.595685in}}%
\pgfpathclose%
\pgfusepath{stroke,fill}%
\end{pgfscope}%
\begin{pgfscope}%
\pgfpathrectangle{\pgfqpoint{0.786164in}{0.768110in}}{\pgfqpoint{8.851069in}{7.081890in}}%
\pgfusepath{clip}%
\pgfsetbuttcap%
\pgfsetroundjoin%
\definecolor{currentfill}{rgb}{0.201239,0.383670,0.554294}%
\pgfsetfillcolor{currentfill}%
\pgfsetfillopacity{0.700000}%
\pgfsetlinewidth{0.501875pt}%
\definecolor{currentstroke}{rgb}{1.000000,1.000000,1.000000}%
\pgfsetstrokecolor{currentstroke}%
\pgfsetstrokeopacity{0.700000}%
\pgfsetdash{}{0pt}%
\pgfpathmoveto{\pgfqpoint{2.247947in}{2.464295in}}%
\pgfpathcurveto{\pgfqpoint{2.260969in}{2.464295in}}{\pgfqpoint{2.273460in}{2.469469in}}{\pgfqpoint{2.282669in}{2.478678in}}%
\pgfpathcurveto{\pgfqpoint{2.291877in}{2.487886in}}{\pgfqpoint{2.297051in}{2.500377in}}{\pgfqpoint{2.297051in}{2.513400in}}%
\pgfpathcurveto{\pgfqpoint{2.297051in}{2.526423in}}{\pgfqpoint{2.291877in}{2.538914in}}{\pgfqpoint{2.282669in}{2.548122in}}%
\pgfpathcurveto{\pgfqpoint{2.273460in}{2.557331in}}{\pgfqpoint{2.260969in}{2.562504in}}{\pgfqpoint{2.247947in}{2.562504in}}%
\pgfpathcurveto{\pgfqpoint{2.234924in}{2.562504in}}{\pgfqpoint{2.222433in}{2.557331in}}{\pgfqpoint{2.213224in}{2.548122in}}%
\pgfpathcurveto{\pgfqpoint{2.204016in}{2.538914in}}{\pgfqpoint{2.198842in}{2.526423in}}{\pgfqpoint{2.198842in}{2.513400in}}%
\pgfpathcurveto{\pgfqpoint{2.198842in}{2.500377in}}{\pgfqpoint{2.204016in}{2.487886in}}{\pgfqpoint{2.213224in}{2.478678in}}%
\pgfpathcurveto{\pgfqpoint{2.222433in}{2.469469in}}{\pgfqpoint{2.234924in}{2.464295in}}{\pgfqpoint{2.247947in}{2.464295in}}%
\pgfpathlineto{\pgfqpoint{2.247947in}{2.464295in}}%
\pgfpathclose%
\pgfusepath{stroke,fill}%
\end{pgfscope}%
\begin{pgfscope}%
\pgfpathrectangle{\pgfqpoint{0.786164in}{0.768110in}}{\pgfqpoint{8.851069in}{7.081890in}}%
\pgfusepath{clip}%
\pgfsetbuttcap%
\pgfsetroundjoin%
\definecolor{currentfill}{rgb}{0.194100,0.399323,0.555565}%
\pgfsetfillcolor{currentfill}%
\pgfsetfillopacity{0.700000}%
\pgfsetlinewidth{0.501875pt}%
\definecolor{currentstroke}{rgb}{1.000000,1.000000,1.000000}%
\pgfsetstrokecolor{currentstroke}%
\pgfsetstrokeopacity{0.700000}%
\pgfsetdash{}{0pt}%
\pgfpathmoveto{\pgfqpoint{2.238813in}{2.442397in}}%
\pgfpathcurveto{\pgfqpoint{2.251836in}{2.442397in}}{\pgfqpoint{2.264327in}{2.447571in}}{\pgfqpoint{2.273535in}{2.456779in}}%
\pgfpathcurveto{\pgfqpoint{2.282744in}{2.465988in}}{\pgfqpoint{2.287918in}{2.478479in}}{\pgfqpoint{2.287918in}{2.491502in}}%
\pgfpathcurveto{\pgfqpoint{2.287918in}{2.504524in}}{\pgfqpoint{2.282744in}{2.517015in}}{\pgfqpoint{2.273535in}{2.526224in}}%
\pgfpathcurveto{\pgfqpoint{2.264327in}{2.535432in}}{\pgfqpoint{2.251836in}{2.540606in}}{\pgfqpoint{2.238813in}{2.540606in}}%
\pgfpathcurveto{\pgfqpoint{2.225791in}{2.540606in}}{\pgfqpoint{2.213299in}{2.535432in}}{\pgfqpoint{2.204091in}{2.526224in}}%
\pgfpathcurveto{\pgfqpoint{2.194883in}{2.517015in}}{\pgfqpoint{2.189709in}{2.504524in}}{\pgfqpoint{2.189709in}{2.491502in}}%
\pgfpathcurveto{\pgfqpoint{2.189709in}{2.478479in}}{\pgfqpoint{2.194883in}{2.465988in}}{\pgfqpoint{2.204091in}{2.456779in}}%
\pgfpathcurveto{\pgfqpoint{2.213299in}{2.447571in}}{\pgfqpoint{2.225791in}{2.442397in}}{\pgfqpoint{2.238813in}{2.442397in}}%
\pgfpathlineto{\pgfqpoint{2.238813in}{2.442397in}}%
\pgfpathclose%
\pgfusepath{stroke,fill}%
\end{pgfscope}%
\begin{pgfscope}%
\pgfpathrectangle{\pgfqpoint{0.786164in}{0.768110in}}{\pgfqpoint{8.851069in}{7.081890in}}%
\pgfusepath{clip}%
\pgfsetbuttcap%
\pgfsetroundjoin%
\definecolor{currentfill}{rgb}{0.182256,0.426184,0.557120}%
\pgfsetfillcolor{currentfill}%
\pgfsetfillopacity{0.700000}%
\pgfsetlinewidth{0.501875pt}%
\definecolor{currentstroke}{rgb}{1.000000,1.000000,1.000000}%
\pgfsetstrokecolor{currentstroke}%
\pgfsetstrokeopacity{0.700000}%
\pgfsetdash{}{0pt}%
\pgfpathmoveto{\pgfqpoint{2.120081in}{2.245313in}}%
\pgfpathcurveto{\pgfqpoint{2.133103in}{2.245313in}}{\pgfqpoint{2.145594in}{2.250487in}}{\pgfqpoint{2.154803in}{2.259695in}}%
\pgfpathcurveto{\pgfqpoint{2.164011in}{2.268904in}}{\pgfqpoint{2.169185in}{2.281395in}}{\pgfqpoint{2.169185in}{2.294417in}}%
\pgfpathcurveto{\pgfqpoint{2.169185in}{2.307440in}}{\pgfqpoint{2.164011in}{2.319931in}}{\pgfqpoint{2.154803in}{2.329140in}}%
\pgfpathcurveto{\pgfqpoint{2.145594in}{2.338348in}}{\pgfqpoint{2.133103in}{2.343522in}}{\pgfqpoint{2.120081in}{2.343522in}}%
\pgfpathcurveto{\pgfqpoint{2.107058in}{2.343522in}}{\pgfqpoint{2.094567in}{2.338348in}}{\pgfqpoint{2.085358in}{2.329140in}}%
\pgfpathcurveto{\pgfqpoint{2.076150in}{2.319931in}}{\pgfqpoint{2.070976in}{2.307440in}}{\pgfqpoint{2.070976in}{2.294417in}}%
\pgfpathcurveto{\pgfqpoint{2.070976in}{2.281395in}}{\pgfqpoint{2.076150in}{2.268904in}}{\pgfqpoint{2.085358in}{2.259695in}}%
\pgfpathcurveto{\pgfqpoint{2.094567in}{2.250487in}}{\pgfqpoint{2.107058in}{2.245313in}}{\pgfqpoint{2.120081in}{2.245313in}}%
\pgfpathlineto{\pgfqpoint{2.120081in}{2.245313in}}%
\pgfpathclose%
\pgfusepath{stroke,fill}%
\end{pgfscope}%
\begin{pgfscope}%
\pgfpathrectangle{\pgfqpoint{0.786164in}{0.768110in}}{\pgfqpoint{8.851069in}{7.081890in}}%
\pgfusepath{clip}%
\pgfsetbuttcap%
\pgfsetroundjoin%
\definecolor{currentfill}{rgb}{0.208623,0.367752,0.552675}%
\pgfsetfillcolor{currentfill}%
\pgfsetfillopacity{0.700000}%
\pgfsetlinewidth{0.501875pt}%
\definecolor{currentstroke}{rgb}{1.000000,1.000000,1.000000}%
\pgfsetstrokecolor{currentstroke}%
\pgfsetstrokeopacity{0.700000}%
\pgfsetdash{}{0pt}%
\pgfpathmoveto{\pgfqpoint{2.339279in}{2.442397in}}%
\pgfpathcurveto{\pgfqpoint{2.352302in}{2.442397in}}{\pgfqpoint{2.364793in}{2.447571in}}{\pgfqpoint{2.374002in}{2.456779in}}%
\pgfpathcurveto{\pgfqpoint{2.383210in}{2.465988in}}{\pgfqpoint{2.388384in}{2.478479in}}{\pgfqpoint{2.388384in}{2.491502in}}%
\pgfpathcurveto{\pgfqpoint{2.388384in}{2.504524in}}{\pgfqpoint{2.383210in}{2.517015in}}{\pgfqpoint{2.374002in}{2.526224in}}%
\pgfpathcurveto{\pgfqpoint{2.364793in}{2.535432in}}{\pgfqpoint{2.352302in}{2.540606in}}{\pgfqpoint{2.339279in}{2.540606in}}%
\pgfpathcurveto{\pgfqpoint{2.326257in}{2.540606in}}{\pgfqpoint{2.313766in}{2.535432in}}{\pgfqpoint{2.304557in}{2.526224in}}%
\pgfpathcurveto{\pgfqpoint{2.295349in}{2.517015in}}{\pgfqpoint{2.290175in}{2.504524in}}{\pgfqpoint{2.290175in}{2.491502in}}%
\pgfpathcurveto{\pgfqpoint{2.290175in}{2.478479in}}{\pgfqpoint{2.295349in}{2.465988in}}{\pgfqpoint{2.304557in}{2.456779in}}%
\pgfpathcurveto{\pgfqpoint{2.313766in}{2.447571in}}{\pgfqpoint{2.326257in}{2.442397in}}{\pgfqpoint{2.339279in}{2.442397in}}%
\pgfpathlineto{\pgfqpoint{2.339279in}{2.442397in}}%
\pgfpathclose%
\pgfusepath{stroke,fill}%
\end{pgfscope}%
\begin{pgfscope}%
\pgfpathrectangle{\pgfqpoint{0.786164in}{0.768110in}}{\pgfqpoint{8.851069in}{7.081890in}}%
\pgfusepath{clip}%
\pgfsetbuttcap%
\pgfsetroundjoin%
\definecolor{currentfill}{rgb}{0.212395,0.359683,0.551710}%
\pgfsetfillcolor{currentfill}%
\pgfsetfillopacity{0.700000}%
\pgfsetlinewidth{0.501875pt}%
\definecolor{currentstroke}{rgb}{1.000000,1.000000,1.000000}%
\pgfsetstrokecolor{currentstroke}%
\pgfsetstrokeopacity{0.700000}%
\pgfsetdash{}{0pt}%
\pgfpathmoveto{\pgfqpoint{2.476279in}{2.661379in}}%
\pgfpathcurveto{\pgfqpoint{2.489301in}{2.661379in}}{\pgfqpoint{2.501793in}{2.666553in}}{\pgfqpoint{2.511001in}{2.675762in}}%
\pgfpathcurveto{\pgfqpoint{2.520209in}{2.684970in}}{\pgfqpoint{2.525383in}{2.697461in}}{\pgfqpoint{2.525383in}{2.710484in}}%
\pgfpathcurveto{\pgfqpoint{2.525383in}{2.723507in}}{\pgfqpoint{2.520209in}{2.735998in}}{\pgfqpoint{2.511001in}{2.745206in}}%
\pgfpathcurveto{\pgfqpoint{2.501793in}{2.754415in}}{\pgfqpoint{2.489301in}{2.759589in}}{\pgfqpoint{2.476279in}{2.759589in}}%
\pgfpathcurveto{\pgfqpoint{2.463256in}{2.759589in}}{\pgfqpoint{2.450765in}{2.754415in}}{\pgfqpoint{2.441557in}{2.745206in}}%
\pgfpathcurveto{\pgfqpoint{2.432348in}{2.735998in}}{\pgfqpoint{2.427174in}{2.723507in}}{\pgfqpoint{2.427174in}{2.710484in}}%
\pgfpathcurveto{\pgfqpoint{2.427174in}{2.697461in}}{\pgfqpoint{2.432348in}{2.684970in}}{\pgfqpoint{2.441557in}{2.675762in}}%
\pgfpathcurveto{\pgfqpoint{2.450765in}{2.666553in}}{\pgfqpoint{2.463256in}{2.661379in}}{\pgfqpoint{2.476279in}{2.661379in}}%
\pgfpathlineto{\pgfqpoint{2.476279in}{2.661379in}}%
\pgfpathclose%
\pgfusepath{stroke,fill}%
\end{pgfscope}%
\begin{pgfscope}%
\pgfpathrectangle{\pgfqpoint{0.786164in}{0.768110in}}{\pgfqpoint{8.851069in}{7.081890in}}%
\pgfusepath{clip}%
\pgfsetbuttcap%
\pgfsetroundjoin%
\definecolor{currentfill}{rgb}{0.279566,0.067836,0.391917}%
\pgfsetfillcolor{currentfill}%
\pgfsetfillopacity{0.700000}%
\pgfsetlinewidth{0.501875pt}%
\definecolor{currentstroke}{rgb}{1.000000,1.000000,1.000000}%
\pgfsetstrokecolor{currentstroke}%
\pgfsetstrokeopacity{0.700000}%
\pgfsetdash{}{0pt}%
\pgfpathmoveto{\pgfqpoint{2.932943in}{3.975274in}}%
\pgfpathcurveto{\pgfqpoint{2.945966in}{3.975274in}}{\pgfqpoint{2.958457in}{3.980447in}}{\pgfqpoint{2.967665in}{3.989656in}}%
\pgfpathcurveto{\pgfqpoint{2.976874in}{3.998864in}}{\pgfqpoint{2.982048in}{4.011355in}}{\pgfqpoint{2.982048in}{4.024378in}}%
\pgfpathcurveto{\pgfqpoint{2.982048in}{4.037401in}}{\pgfqpoint{2.976874in}{4.049892in}}{\pgfqpoint{2.967665in}{4.059100in}}%
\pgfpathcurveto{\pgfqpoint{2.958457in}{4.068309in}}{\pgfqpoint{2.945966in}{4.073483in}}{\pgfqpoint{2.932943in}{4.073483in}}%
\pgfpathcurveto{\pgfqpoint{2.919920in}{4.073483in}}{\pgfqpoint{2.907429in}{4.068309in}}{\pgfqpoint{2.898221in}{4.059100in}}%
\pgfpathcurveto{\pgfqpoint{2.889012in}{4.049892in}}{\pgfqpoint{2.883838in}{4.037401in}}{\pgfqpoint{2.883838in}{4.024378in}}%
\pgfpathcurveto{\pgfqpoint{2.883838in}{4.011355in}}{\pgfqpoint{2.889012in}{3.998864in}}{\pgfqpoint{2.898221in}{3.989656in}}%
\pgfpathcurveto{\pgfqpoint{2.907429in}{3.980447in}}{\pgfqpoint{2.919920in}{3.975274in}}{\pgfqpoint{2.932943in}{3.975274in}}%
\pgfpathlineto{\pgfqpoint{2.932943in}{3.975274in}}%
\pgfpathclose%
\pgfusepath{stroke,fill}%
\end{pgfscope}%
\begin{pgfscope}%
\pgfpathrectangle{\pgfqpoint{0.786164in}{0.768110in}}{\pgfqpoint{8.851069in}{7.081890in}}%
\pgfusepath{clip}%
\pgfsetbuttcap%
\pgfsetroundjoin%
\definecolor{currentfill}{rgb}{0.279566,0.067836,0.391917}%
\pgfsetfillcolor{currentfill}%
\pgfsetfillopacity{0.700000}%
\pgfsetlinewidth{0.501875pt}%
\definecolor{currentstroke}{rgb}{1.000000,1.000000,1.000000}%
\pgfsetstrokecolor{currentstroke}%
\pgfsetstrokeopacity{0.700000}%
\pgfsetdash{}{0pt}%
\pgfpathmoveto{\pgfqpoint{2.942076in}{3.931477in}}%
\pgfpathcurveto{\pgfqpoint{2.955099in}{3.931477in}}{\pgfqpoint{2.967590in}{3.936651in}}{\pgfqpoint{2.976799in}{3.945859in}}%
\pgfpathcurveto{\pgfqpoint{2.986007in}{3.955068in}}{\pgfqpoint{2.991181in}{3.967559in}}{\pgfqpoint{2.991181in}{3.980582in}}%
\pgfpathcurveto{\pgfqpoint{2.991181in}{3.993604in}}{\pgfqpoint{2.986007in}{4.006095in}}{\pgfqpoint{2.976799in}{4.015304in}}%
\pgfpathcurveto{\pgfqpoint{2.967590in}{4.024512in}}{\pgfqpoint{2.955099in}{4.029686in}}{\pgfqpoint{2.942076in}{4.029686in}}%
\pgfpathcurveto{\pgfqpoint{2.929054in}{4.029686in}}{\pgfqpoint{2.916563in}{4.024512in}}{\pgfqpoint{2.907354in}{4.015304in}}%
\pgfpathcurveto{\pgfqpoint{2.898146in}{4.006095in}}{\pgfqpoint{2.892972in}{3.993604in}}{\pgfqpoint{2.892972in}{3.980582in}}%
\pgfpathcurveto{\pgfqpoint{2.892972in}{3.967559in}}{\pgfqpoint{2.898146in}{3.955068in}}{\pgfqpoint{2.907354in}{3.945859in}}%
\pgfpathcurveto{\pgfqpoint{2.916563in}{3.936651in}}{\pgfqpoint{2.929054in}{3.931477in}}{\pgfqpoint{2.942076in}{3.931477in}}%
\pgfpathlineto{\pgfqpoint{2.942076in}{3.931477in}}%
\pgfpathclose%
\pgfusepath{stroke,fill}%
\end{pgfscope}%
\begin{pgfscope}%
\pgfpathrectangle{\pgfqpoint{0.786164in}{0.768110in}}{\pgfqpoint{8.851069in}{7.081890in}}%
\pgfusepath{clip}%
\pgfsetbuttcap%
\pgfsetroundjoin%
\definecolor{currentfill}{rgb}{0.280267,0.073417,0.397163}%
\pgfsetfillcolor{currentfill}%
\pgfsetfillopacity{0.700000}%
\pgfsetlinewidth{0.501875pt}%
\definecolor{currentstroke}{rgb}{1.000000,1.000000,1.000000}%
\pgfsetstrokecolor{currentstroke}%
\pgfsetstrokeopacity{0.700000}%
\pgfsetdash{}{0pt}%
\pgfpathmoveto{\pgfqpoint{2.996876in}{3.975274in}}%
\pgfpathcurveto{\pgfqpoint{3.009899in}{3.975274in}}{\pgfqpoint{3.022390in}{3.980447in}}{\pgfqpoint{3.031598in}{3.989656in}}%
\pgfpathcurveto{\pgfqpoint{3.040807in}{3.998864in}}{\pgfqpoint{3.045981in}{4.011355in}}{\pgfqpoint{3.045981in}{4.024378in}}%
\pgfpathcurveto{\pgfqpoint{3.045981in}{4.037401in}}{\pgfqpoint{3.040807in}{4.049892in}}{\pgfqpoint{3.031598in}{4.059100in}}%
\pgfpathcurveto{\pgfqpoint{3.022390in}{4.068309in}}{\pgfqpoint{3.009899in}{4.073483in}}{\pgfqpoint{2.996876in}{4.073483in}}%
\pgfpathcurveto{\pgfqpoint{2.983853in}{4.073483in}}{\pgfqpoint{2.971362in}{4.068309in}}{\pgfqpoint{2.962154in}{4.059100in}}%
\pgfpathcurveto{\pgfqpoint{2.952945in}{4.049892in}}{\pgfqpoint{2.947771in}{4.037401in}}{\pgfqpoint{2.947771in}{4.024378in}}%
\pgfpathcurveto{\pgfqpoint{2.947771in}{4.011355in}}{\pgfqpoint{2.952945in}{3.998864in}}{\pgfqpoint{2.962154in}{3.989656in}}%
\pgfpathcurveto{\pgfqpoint{2.971362in}{3.980447in}}{\pgfqpoint{2.983853in}{3.975274in}}{\pgfqpoint{2.996876in}{3.975274in}}%
\pgfpathlineto{\pgfqpoint{2.996876in}{3.975274in}}%
\pgfpathclose%
\pgfusepath{stroke,fill}%
\end{pgfscope}%
\begin{pgfscope}%
\pgfpathrectangle{\pgfqpoint{0.786164in}{0.768110in}}{\pgfqpoint{8.851069in}{7.081890in}}%
\pgfusepath{clip}%
\pgfsetbuttcap%
\pgfsetroundjoin%
\definecolor{currentfill}{rgb}{0.281924,0.089666,0.412415}%
\pgfsetfillcolor{currentfill}%
\pgfsetfillopacity{0.700000}%
\pgfsetlinewidth{0.501875pt}%
\definecolor{currentstroke}{rgb}{1.000000,1.000000,1.000000}%
\pgfsetstrokecolor{currentstroke}%
\pgfsetstrokeopacity{0.700000}%
\pgfsetdash{}{0pt}%
\pgfpathmoveto{\pgfqpoint{3.088209in}{4.019070in}}%
\pgfpathcurveto{\pgfqpoint{3.101232in}{4.019070in}}{\pgfqpoint{3.113723in}{4.024244in}}{\pgfqpoint{3.122931in}{4.033452in}}%
\pgfpathcurveto{\pgfqpoint{3.132140in}{4.042661in}}{\pgfqpoint{3.137314in}{4.055152in}}{\pgfqpoint{3.137314in}{4.068175in}}%
\pgfpathcurveto{\pgfqpoint{3.137314in}{4.081197in}}{\pgfqpoint{3.132140in}{4.093688in}}{\pgfqpoint{3.122931in}{4.102897in}}%
\pgfpathcurveto{\pgfqpoint{3.113723in}{4.112105in}}{\pgfqpoint{3.101232in}{4.117279in}}{\pgfqpoint{3.088209in}{4.117279in}}%
\pgfpathcurveto{\pgfqpoint{3.075186in}{4.117279in}}{\pgfqpoint{3.062695in}{4.112105in}}{\pgfqpoint{3.053487in}{4.102897in}}%
\pgfpathcurveto{\pgfqpoint{3.044278in}{4.093688in}}{\pgfqpoint{3.039104in}{4.081197in}}{\pgfqpoint{3.039104in}{4.068175in}}%
\pgfpathcurveto{\pgfqpoint{3.039104in}{4.055152in}}{\pgfqpoint{3.044278in}{4.042661in}}{\pgfqpoint{3.053487in}{4.033452in}}%
\pgfpathcurveto{\pgfqpoint{3.062695in}{4.024244in}}{\pgfqpoint{3.075186in}{4.019070in}}{\pgfqpoint{3.088209in}{4.019070in}}%
\pgfpathlineto{\pgfqpoint{3.088209in}{4.019070in}}%
\pgfpathclose%
\pgfusepath{stroke,fill}%
\end{pgfscope}%
\begin{pgfscope}%
\pgfpathrectangle{\pgfqpoint{0.786164in}{0.768110in}}{\pgfqpoint{8.851069in}{7.081890in}}%
\pgfusepath{clip}%
\pgfsetbuttcap%
\pgfsetroundjoin%
\definecolor{currentfill}{rgb}{0.283091,0.110553,0.431554}%
\pgfsetfillcolor{currentfill}%
\pgfsetfillopacity{0.700000}%
\pgfsetlinewidth{0.501875pt}%
\definecolor{currentstroke}{rgb}{1.000000,1.000000,1.000000}%
\pgfsetstrokecolor{currentstroke}%
\pgfsetstrokeopacity{0.700000}%
\pgfsetdash{}{0pt}%
\pgfpathmoveto{\pgfqpoint{3.179542in}{3.975274in}}%
\pgfpathcurveto{\pgfqpoint{3.192565in}{3.975274in}}{\pgfqpoint{3.205056in}{3.980447in}}{\pgfqpoint{3.214264in}{3.989656in}}%
\pgfpathcurveto{\pgfqpoint{3.223473in}{3.998864in}}{\pgfqpoint{3.228647in}{4.011355in}}{\pgfqpoint{3.228647in}{4.024378in}}%
\pgfpathcurveto{\pgfqpoint{3.228647in}{4.037401in}}{\pgfqpoint{3.223473in}{4.049892in}}{\pgfqpoint{3.214264in}{4.059100in}}%
\pgfpathcurveto{\pgfqpoint{3.205056in}{4.068309in}}{\pgfqpoint{3.192565in}{4.073483in}}{\pgfqpoint{3.179542in}{4.073483in}}%
\pgfpathcurveto{\pgfqpoint{3.166519in}{4.073483in}}{\pgfqpoint{3.154028in}{4.068309in}}{\pgfqpoint{3.144820in}{4.059100in}}%
\pgfpathcurveto{\pgfqpoint{3.135611in}{4.049892in}}{\pgfqpoint{3.130437in}{4.037401in}}{\pgfqpoint{3.130437in}{4.024378in}}%
\pgfpathcurveto{\pgfqpoint{3.130437in}{4.011355in}}{\pgfqpoint{3.135611in}{3.998864in}}{\pgfqpoint{3.144820in}{3.989656in}}%
\pgfpathcurveto{\pgfqpoint{3.154028in}{3.980447in}}{\pgfqpoint{3.166519in}{3.975274in}}{\pgfqpoint{3.179542in}{3.975274in}}%
\pgfpathlineto{\pgfqpoint{3.179542in}{3.975274in}}%
\pgfpathclose%
\pgfusepath{stroke,fill}%
\end{pgfscope}%
\begin{pgfscope}%
\pgfpathrectangle{\pgfqpoint{0.786164in}{0.768110in}}{\pgfqpoint{8.851069in}{7.081890in}}%
\pgfusepath{clip}%
\pgfsetbuttcap%
\pgfsetroundjoin%
\definecolor{currentfill}{rgb}{0.283187,0.125848,0.444960}%
\pgfsetfillcolor{currentfill}%
\pgfsetfillopacity{0.700000}%
\pgfsetlinewidth{0.501875pt}%
\definecolor{currentstroke}{rgb}{1.000000,1.000000,1.000000}%
\pgfsetstrokecolor{currentstroke}%
\pgfsetstrokeopacity{0.700000}%
\pgfsetdash{}{0pt}%
\pgfpathmoveto{\pgfqpoint{3.006009in}{3.821986in}}%
\pgfpathcurveto{\pgfqpoint{3.019032in}{3.821986in}}{\pgfqpoint{3.031523in}{3.827160in}}{\pgfqpoint{3.040732in}{3.836368in}}%
\pgfpathcurveto{\pgfqpoint{3.049940in}{3.845577in}}{\pgfqpoint{3.055114in}{3.858068in}}{\pgfqpoint{3.055114in}{3.871090in}}%
\pgfpathcurveto{\pgfqpoint{3.055114in}{3.884113in}}{\pgfqpoint{3.049940in}{3.896604in}}{\pgfqpoint{3.040732in}{3.905813in}}%
\pgfpathcurveto{\pgfqpoint{3.031523in}{3.915021in}}{\pgfqpoint{3.019032in}{3.920195in}}{\pgfqpoint{3.006009in}{3.920195in}}%
\pgfpathcurveto{\pgfqpoint{2.992987in}{3.920195in}}{\pgfqpoint{2.980496in}{3.915021in}}{\pgfqpoint{2.971287in}{3.905813in}}%
\pgfpathcurveto{\pgfqpoint{2.962079in}{3.896604in}}{\pgfqpoint{2.956905in}{3.884113in}}{\pgfqpoint{2.956905in}{3.871090in}}%
\pgfpathcurveto{\pgfqpoint{2.956905in}{3.858068in}}{\pgfqpoint{2.962079in}{3.845577in}}{\pgfqpoint{2.971287in}{3.836368in}}%
\pgfpathcurveto{\pgfqpoint{2.980496in}{3.827160in}}{\pgfqpoint{2.992987in}{3.821986in}}{\pgfqpoint{3.006009in}{3.821986in}}%
\pgfpathlineto{\pgfqpoint{3.006009in}{3.821986in}}%
\pgfpathclose%
\pgfusepath{stroke,fill}%
\end{pgfscope}%
\begin{pgfscope}%
\pgfpathrectangle{\pgfqpoint{0.786164in}{0.768110in}}{\pgfqpoint{8.851069in}{7.081890in}}%
\pgfusepath{clip}%
\pgfsetbuttcap%
\pgfsetroundjoin%
\definecolor{currentfill}{rgb}{0.283072,0.130895,0.449241}%
\pgfsetfillcolor{currentfill}%
\pgfsetfillopacity{0.700000}%
\pgfsetlinewidth{0.501875pt}%
\definecolor{currentstroke}{rgb}{1.000000,1.000000,1.000000}%
\pgfsetstrokecolor{currentstroke}%
\pgfsetstrokeopacity{0.700000}%
\pgfsetdash{}{0pt}%
\pgfpathmoveto{\pgfqpoint{2.814210in}{3.668698in}}%
\pgfpathcurveto{\pgfqpoint{2.827233in}{3.668698in}}{\pgfqpoint{2.839724in}{3.673872in}}{\pgfqpoint{2.848933in}{3.683081in}}%
\pgfpathcurveto{\pgfqpoint{2.858141in}{3.692289in}}{\pgfqpoint{2.863315in}{3.704780in}}{\pgfqpoint{2.863315in}{3.717803in}}%
\pgfpathcurveto{\pgfqpoint{2.863315in}{3.730826in}}{\pgfqpoint{2.858141in}{3.743317in}}{\pgfqpoint{2.848933in}{3.752525in}}%
\pgfpathcurveto{\pgfqpoint{2.839724in}{3.761733in}}{\pgfqpoint{2.827233in}{3.766907in}}{\pgfqpoint{2.814210in}{3.766907in}}%
\pgfpathcurveto{\pgfqpoint{2.801188in}{3.766907in}}{\pgfqpoint{2.788697in}{3.761733in}}{\pgfqpoint{2.779488in}{3.752525in}}%
\pgfpathcurveto{\pgfqpoint{2.770280in}{3.743317in}}{\pgfqpoint{2.765106in}{3.730826in}}{\pgfqpoint{2.765106in}{3.717803in}}%
\pgfpathcurveto{\pgfqpoint{2.765106in}{3.704780in}}{\pgfqpoint{2.770280in}{3.692289in}}{\pgfqpoint{2.779488in}{3.683081in}}%
\pgfpathcurveto{\pgfqpoint{2.788697in}{3.673872in}}{\pgfqpoint{2.801188in}{3.668698in}}{\pgfqpoint{2.814210in}{3.668698in}}%
\pgfpathlineto{\pgfqpoint{2.814210in}{3.668698in}}%
\pgfpathclose%
\pgfusepath{stroke,fill}%
\end{pgfscope}%
\begin{pgfscope}%
\pgfpathrectangle{\pgfqpoint{0.786164in}{0.768110in}}{\pgfqpoint{8.851069in}{7.081890in}}%
\pgfusepath{clip}%
\pgfsetbuttcap%
\pgfsetroundjoin%
\definecolor{currentfill}{rgb}{0.283072,0.130895,0.449241}%
\pgfsetfillcolor{currentfill}%
\pgfsetfillopacity{0.700000}%
\pgfsetlinewidth{0.501875pt}%
\definecolor{currentstroke}{rgb}{1.000000,1.000000,1.000000}%
\pgfsetstrokecolor{currentstroke}%
\pgfsetstrokeopacity{0.700000}%
\pgfsetdash{}{0pt}%
\pgfpathmoveto{\pgfqpoint{2.713744in}{3.515411in}}%
\pgfpathcurveto{\pgfqpoint{2.726767in}{3.515411in}}{\pgfqpoint{2.739258in}{3.520585in}}{\pgfqpoint{2.748466in}{3.529793in}}%
\pgfpathcurveto{\pgfqpoint{2.757675in}{3.539001in}}{\pgfqpoint{2.762849in}{3.551492in}}{\pgfqpoint{2.762849in}{3.564515in}}%
\pgfpathcurveto{\pgfqpoint{2.762849in}{3.577538in}}{\pgfqpoint{2.757675in}{3.590029in}}{\pgfqpoint{2.748466in}{3.599237in}}%
\pgfpathcurveto{\pgfqpoint{2.739258in}{3.608446in}}{\pgfqpoint{2.726767in}{3.613620in}}{\pgfqpoint{2.713744in}{3.613620in}}%
\pgfpathcurveto{\pgfqpoint{2.700722in}{3.613620in}}{\pgfqpoint{2.688230in}{3.608446in}}{\pgfqpoint{2.679022in}{3.599237in}}%
\pgfpathcurveto{\pgfqpoint{2.669814in}{3.590029in}}{\pgfqpoint{2.664640in}{3.577538in}}{\pgfqpoint{2.664640in}{3.564515in}}%
\pgfpathcurveto{\pgfqpoint{2.664640in}{3.551492in}}{\pgfqpoint{2.669814in}{3.539001in}}{\pgfqpoint{2.679022in}{3.529793in}}%
\pgfpathcurveto{\pgfqpoint{2.688230in}{3.520585in}}{\pgfqpoint{2.700722in}{3.515411in}}{\pgfqpoint{2.713744in}{3.515411in}}%
\pgfpathlineto{\pgfqpoint{2.713744in}{3.515411in}}%
\pgfpathclose%
\pgfusepath{stroke,fill}%
\end{pgfscope}%
\begin{pgfscope}%
\pgfpathrectangle{\pgfqpoint{0.786164in}{0.768110in}}{\pgfqpoint{8.851069in}{7.081890in}}%
\pgfusepath{clip}%
\pgfsetbuttcap%
\pgfsetroundjoin%
\definecolor{currentfill}{rgb}{0.281887,0.150881,0.465405}%
\pgfsetfillcolor{currentfill}%
\pgfsetfillopacity{0.700000}%
\pgfsetlinewidth{0.501875pt}%
\definecolor{currentstroke}{rgb}{1.000000,1.000000,1.000000}%
\pgfsetstrokecolor{currentstroke}%
\pgfsetstrokeopacity{0.700000}%
\pgfsetdash{}{0pt}%
\pgfpathmoveto{\pgfqpoint{2.512812in}{3.340225in}}%
\pgfpathcurveto{\pgfqpoint{2.525835in}{3.340225in}}{\pgfqpoint{2.538326in}{3.345399in}}{\pgfqpoint{2.547534in}{3.354607in}}%
\pgfpathcurveto{\pgfqpoint{2.556743in}{3.363816in}}{\pgfqpoint{2.561917in}{3.376307in}}{\pgfqpoint{2.561917in}{3.389329in}}%
\pgfpathcurveto{\pgfqpoint{2.561917in}{3.402352in}}{\pgfqpoint{2.556743in}{3.414843in}}{\pgfqpoint{2.547534in}{3.424052in}}%
\pgfpathcurveto{\pgfqpoint{2.538326in}{3.433260in}}{\pgfqpoint{2.525835in}{3.438434in}}{\pgfqpoint{2.512812in}{3.438434in}}%
\pgfpathcurveto{\pgfqpoint{2.499789in}{3.438434in}}{\pgfqpoint{2.487298in}{3.433260in}}{\pgfqpoint{2.478090in}{3.424052in}}%
\pgfpathcurveto{\pgfqpoint{2.468881in}{3.414843in}}{\pgfqpoint{2.463707in}{3.402352in}}{\pgfqpoint{2.463707in}{3.389329in}}%
\pgfpathcurveto{\pgfqpoint{2.463707in}{3.376307in}}{\pgfqpoint{2.468881in}{3.363816in}}{\pgfqpoint{2.478090in}{3.354607in}}%
\pgfpathcurveto{\pgfqpoint{2.487298in}{3.345399in}}{\pgfqpoint{2.499789in}{3.340225in}}{\pgfqpoint{2.512812in}{3.340225in}}%
\pgfpathlineto{\pgfqpoint{2.512812in}{3.340225in}}%
\pgfpathclose%
\pgfusepath{stroke,fill}%
\end{pgfscope}%
\begin{pgfscope}%
\pgfpathrectangle{\pgfqpoint{0.786164in}{0.768110in}}{\pgfqpoint{8.851069in}{7.081890in}}%
\pgfusepath{clip}%
\pgfsetbuttcap%
\pgfsetroundjoin%
\definecolor{currentfill}{rgb}{0.278826,0.175490,0.483397}%
\pgfsetfillcolor{currentfill}%
\pgfsetfillopacity{0.700000}%
\pgfsetlinewidth{0.501875pt}%
\definecolor{currentstroke}{rgb}{1.000000,1.000000,1.000000}%
\pgfsetstrokecolor{currentstroke}%
\pgfsetstrokeopacity{0.700000}%
\pgfsetdash{}{0pt}%
\pgfpathmoveto{\pgfqpoint{2.184014in}{3.165039in}}%
\pgfpathcurveto{\pgfqpoint{2.197036in}{3.165039in}}{\pgfqpoint{2.209527in}{3.170213in}}{\pgfqpoint{2.218736in}{3.179421in}}%
\pgfpathcurveto{\pgfqpoint{2.227944in}{3.188630in}}{\pgfqpoint{2.233118in}{3.201121in}}{\pgfqpoint{2.233118in}{3.214143in}}%
\pgfpathcurveto{\pgfqpoint{2.233118in}{3.227166in}}{\pgfqpoint{2.227944in}{3.239657in}}{\pgfqpoint{2.218736in}{3.248866in}}%
\pgfpathcurveto{\pgfqpoint{2.209527in}{3.258074in}}{\pgfqpoint{2.197036in}{3.263248in}}{\pgfqpoint{2.184014in}{3.263248in}}%
\pgfpathcurveto{\pgfqpoint{2.170991in}{3.263248in}}{\pgfqpoint{2.158500in}{3.258074in}}{\pgfqpoint{2.149291in}{3.248866in}}%
\pgfpathcurveto{\pgfqpoint{2.140083in}{3.239657in}}{\pgfqpoint{2.134909in}{3.227166in}}{\pgfqpoint{2.134909in}{3.214143in}}%
\pgfpathcurveto{\pgfqpoint{2.134909in}{3.201121in}}{\pgfqpoint{2.140083in}{3.188630in}}{\pgfqpoint{2.149291in}{3.179421in}}%
\pgfpathcurveto{\pgfqpoint{2.158500in}{3.170213in}}{\pgfqpoint{2.170991in}{3.165039in}}{\pgfqpoint{2.184014in}{3.165039in}}%
\pgfpathlineto{\pgfqpoint{2.184014in}{3.165039in}}%
\pgfpathclose%
\pgfusepath{stroke,fill}%
\end{pgfscope}%
\begin{pgfscope}%
\pgfpathrectangle{\pgfqpoint{0.786164in}{0.768110in}}{\pgfqpoint{8.851069in}{7.081890in}}%
\pgfusepath{clip}%
\pgfsetbuttcap%
\pgfsetroundjoin%
\definecolor{currentfill}{rgb}{0.276194,0.190074,0.493001}%
\pgfsetfillcolor{currentfill}%
\pgfsetfillopacity{0.700000}%
\pgfsetlinewidth{0.501875pt}%
\definecolor{currentstroke}{rgb}{1.000000,1.000000,1.000000}%
\pgfsetstrokecolor{currentstroke}%
\pgfsetstrokeopacity{0.700000}%
\pgfsetdash{}{0pt}%
\pgfpathmoveto{\pgfqpoint{2.247947in}{3.274530in}}%
\pgfpathcurveto{\pgfqpoint{2.260969in}{3.274530in}}{\pgfqpoint{2.273460in}{3.279704in}}{\pgfqpoint{2.282669in}{3.288912in}}%
\pgfpathcurveto{\pgfqpoint{2.291877in}{3.298121in}}{\pgfqpoint{2.297051in}{3.310612in}}{\pgfqpoint{2.297051in}{3.323635in}}%
\pgfpathcurveto{\pgfqpoint{2.297051in}{3.336657in}}{\pgfqpoint{2.291877in}{3.349148in}}{\pgfqpoint{2.282669in}{3.358357in}}%
\pgfpathcurveto{\pgfqpoint{2.273460in}{3.367565in}}{\pgfqpoint{2.260969in}{3.372739in}}{\pgfqpoint{2.247947in}{3.372739in}}%
\pgfpathcurveto{\pgfqpoint{2.234924in}{3.372739in}}{\pgfqpoint{2.222433in}{3.367565in}}{\pgfqpoint{2.213224in}{3.358357in}}%
\pgfpathcurveto{\pgfqpoint{2.204016in}{3.349148in}}{\pgfqpoint{2.198842in}{3.336657in}}{\pgfqpoint{2.198842in}{3.323635in}}%
\pgfpathcurveto{\pgfqpoint{2.198842in}{3.310612in}}{\pgfqpoint{2.204016in}{3.298121in}}{\pgfqpoint{2.213224in}{3.288912in}}%
\pgfpathcurveto{\pgfqpoint{2.222433in}{3.279704in}}{\pgfqpoint{2.234924in}{3.274530in}}{\pgfqpoint{2.247947in}{3.274530in}}%
\pgfpathlineto{\pgfqpoint{2.247947in}{3.274530in}}%
\pgfpathclose%
\pgfusepath{stroke,fill}%
\end{pgfscope}%
\begin{pgfscope}%
\pgfpathrectangle{\pgfqpoint{0.786164in}{0.768110in}}{\pgfqpoint{8.851069in}{7.081890in}}%
\pgfusepath{clip}%
\pgfsetbuttcap%
\pgfsetroundjoin%
\definecolor{currentfill}{rgb}{0.273006,0.204520,0.501721}%
\pgfsetfillcolor{currentfill}%
\pgfsetfillopacity{0.700000}%
\pgfsetlinewidth{0.501875pt}%
\definecolor{currentstroke}{rgb}{1.000000,1.000000,1.000000}%
\pgfsetstrokecolor{currentstroke}%
\pgfsetstrokeopacity{0.700000}%
\pgfsetdash{}{0pt}%
\pgfpathmoveto{\pgfqpoint{2.321013in}{3.296428in}}%
\pgfpathcurveto{\pgfqpoint{2.334036in}{3.296428in}}{\pgfqpoint{2.346527in}{3.301602in}}{\pgfqpoint{2.355735in}{3.310811in}}%
\pgfpathcurveto{\pgfqpoint{2.364944in}{3.320019in}}{\pgfqpoint{2.370117in}{3.332510in}}{\pgfqpoint{2.370117in}{3.345533in}}%
\pgfpathcurveto{\pgfqpoint{2.370117in}{3.358556in}}{\pgfqpoint{2.364944in}{3.371047in}}{\pgfqpoint{2.355735in}{3.380255in}}%
\pgfpathcurveto{\pgfqpoint{2.346527in}{3.389463in}}{\pgfqpoint{2.334036in}{3.394637in}}{\pgfqpoint{2.321013in}{3.394637in}}%
\pgfpathcurveto{\pgfqpoint{2.307990in}{3.394637in}}{\pgfqpoint{2.295499in}{3.389463in}}{\pgfqpoint{2.286291in}{3.380255in}}%
\pgfpathcurveto{\pgfqpoint{2.277082in}{3.371047in}}{\pgfqpoint{2.271908in}{3.358556in}}{\pgfqpoint{2.271908in}{3.345533in}}%
\pgfpathcurveto{\pgfqpoint{2.271908in}{3.332510in}}{\pgfqpoint{2.277082in}{3.320019in}}{\pgfqpoint{2.286291in}{3.310811in}}%
\pgfpathcurveto{\pgfqpoint{2.295499in}{3.301602in}}{\pgfqpoint{2.307990in}{3.296428in}}{\pgfqpoint{2.321013in}{3.296428in}}%
\pgfpathlineto{\pgfqpoint{2.321013in}{3.296428in}}%
\pgfpathclose%
\pgfusepath{stroke,fill}%
\end{pgfscope}%
\begin{pgfscope}%
\pgfpathrectangle{\pgfqpoint{0.786164in}{0.768110in}}{\pgfqpoint{8.851069in}{7.081890in}}%
\pgfusepath{clip}%
\pgfsetbuttcap%
\pgfsetroundjoin%
\definecolor{currentfill}{rgb}{0.274128,0.199721,0.498911}%
\pgfsetfillcolor{currentfill}%
\pgfsetfillopacity{0.700000}%
\pgfsetlinewidth{0.501875pt}%
\definecolor{currentstroke}{rgb}{1.000000,1.000000,1.000000}%
\pgfsetstrokecolor{currentstroke}%
\pgfsetstrokeopacity{0.700000}%
\pgfsetdash{}{0pt}%
\pgfpathmoveto{\pgfqpoint{2.467145in}{3.449716in}}%
\pgfpathcurveto{\pgfqpoint{2.480168in}{3.449716in}}{\pgfqpoint{2.492659in}{3.454890in}}{\pgfqpoint{2.501868in}{3.464098in}}%
\pgfpathcurveto{\pgfqpoint{2.511076in}{3.473307in}}{\pgfqpoint{2.516250in}{3.485798in}}{\pgfqpoint{2.516250in}{3.498820in}}%
\pgfpathcurveto{\pgfqpoint{2.516250in}{3.511843in}}{\pgfqpoint{2.511076in}{3.524334in}}{\pgfqpoint{2.501868in}{3.533543in}}%
\pgfpathcurveto{\pgfqpoint{2.492659in}{3.542751in}}{\pgfqpoint{2.480168in}{3.547925in}}{\pgfqpoint{2.467145in}{3.547925in}}%
\pgfpathcurveto{\pgfqpoint{2.454123in}{3.547925in}}{\pgfqpoint{2.441632in}{3.542751in}}{\pgfqpoint{2.432423in}{3.533543in}}%
\pgfpathcurveto{\pgfqpoint{2.423215in}{3.524334in}}{\pgfqpoint{2.418041in}{3.511843in}}{\pgfqpoint{2.418041in}{3.498820in}}%
\pgfpathcurveto{\pgfqpoint{2.418041in}{3.485798in}}{\pgfqpoint{2.423215in}{3.473307in}}{\pgfqpoint{2.432423in}{3.464098in}}%
\pgfpathcurveto{\pgfqpoint{2.441632in}{3.454890in}}{\pgfqpoint{2.454123in}{3.449716in}}{\pgfqpoint{2.467145in}{3.449716in}}%
\pgfpathlineto{\pgfqpoint{2.467145in}{3.449716in}}%
\pgfpathclose%
\pgfusepath{stroke,fill}%
\end{pgfscope}%
\begin{pgfscope}%
\pgfpathrectangle{\pgfqpoint{0.786164in}{0.768110in}}{\pgfqpoint{8.851069in}{7.081890in}}%
\pgfusepath{clip}%
\pgfsetbuttcap%
\pgfsetroundjoin%
\definecolor{currentfill}{rgb}{0.270595,0.214069,0.507052}%
\pgfsetfillcolor{currentfill}%
\pgfsetfillopacity{0.700000}%
\pgfsetlinewidth{0.501875pt}%
\definecolor{currentstroke}{rgb}{1.000000,1.000000,1.000000}%
\pgfsetstrokecolor{currentstroke}%
\pgfsetstrokeopacity{0.700000}%
\pgfsetdash{}{0pt}%
\pgfpathmoveto{\pgfqpoint{2.558478in}{3.471614in}}%
\pgfpathcurveto{\pgfqpoint{2.571501in}{3.471614in}}{\pgfqpoint{2.583992in}{3.476788in}}{\pgfqpoint{2.593201in}{3.485996in}}%
\pgfpathcurveto{\pgfqpoint{2.602409in}{3.495205in}}{\pgfqpoint{2.607583in}{3.507696in}}{\pgfqpoint{2.607583in}{3.520719in}}%
\pgfpathcurveto{\pgfqpoint{2.607583in}{3.533741in}}{\pgfqpoint{2.602409in}{3.546232in}}{\pgfqpoint{2.593201in}{3.555441in}}%
\pgfpathcurveto{\pgfqpoint{2.583992in}{3.564649in}}{\pgfqpoint{2.571501in}{3.569823in}}{\pgfqpoint{2.558478in}{3.569823in}}%
\pgfpathcurveto{\pgfqpoint{2.545456in}{3.569823in}}{\pgfqpoint{2.532965in}{3.564649in}}{\pgfqpoint{2.523756in}{3.555441in}}%
\pgfpathcurveto{\pgfqpoint{2.514548in}{3.546232in}}{\pgfqpoint{2.509374in}{3.533741in}}{\pgfqpoint{2.509374in}{3.520719in}}%
\pgfpathcurveto{\pgfqpoint{2.509374in}{3.507696in}}{\pgfqpoint{2.514548in}{3.495205in}}{\pgfqpoint{2.523756in}{3.485996in}}%
\pgfpathcurveto{\pgfqpoint{2.532965in}{3.476788in}}{\pgfqpoint{2.545456in}{3.471614in}}{\pgfqpoint{2.558478in}{3.471614in}}%
\pgfpathlineto{\pgfqpoint{2.558478in}{3.471614in}}%
\pgfpathclose%
\pgfusepath{stroke,fill}%
\end{pgfscope}%
\begin{pgfscope}%
\pgfpathrectangle{\pgfqpoint{0.786164in}{0.768110in}}{\pgfqpoint{8.851069in}{7.081890in}}%
\pgfusepath{clip}%
\pgfsetbuttcap%
\pgfsetroundjoin%
\definecolor{currentfill}{rgb}{0.250425,0.274290,0.533103}%
\pgfsetfillcolor{currentfill}%
\pgfsetfillopacity{0.700000}%
\pgfsetlinewidth{0.501875pt}%
\definecolor{currentstroke}{rgb}{1.000000,1.000000,1.000000}%
\pgfsetstrokecolor{currentstroke}%
\pgfsetstrokeopacity{0.700000}%
\pgfsetdash{}{0pt}%
\pgfpathmoveto{\pgfqpoint{2.549345in}{3.427818in}}%
\pgfpathcurveto{\pgfqpoint{2.562368in}{3.427818in}}{\pgfqpoint{2.574859in}{3.432992in}}{\pgfqpoint{2.584067in}{3.442200in}}%
\pgfpathcurveto{\pgfqpoint{2.593276in}{3.451408in}}{\pgfqpoint{2.598450in}{3.463900in}}{\pgfqpoint{2.598450in}{3.476922in}}%
\pgfpathcurveto{\pgfqpoint{2.598450in}{3.489945in}}{\pgfqpoint{2.593276in}{3.502436in}}{\pgfqpoint{2.584067in}{3.511644in}}%
\pgfpathcurveto{\pgfqpoint{2.574859in}{3.520853in}}{\pgfqpoint{2.562368in}{3.526027in}}{\pgfqpoint{2.549345in}{3.526027in}}%
\pgfpathcurveto{\pgfqpoint{2.536322in}{3.526027in}}{\pgfqpoint{2.523831in}{3.520853in}}{\pgfqpoint{2.514623in}{3.511644in}}%
\pgfpathcurveto{\pgfqpoint{2.505414in}{3.502436in}}{\pgfqpoint{2.500240in}{3.489945in}}{\pgfqpoint{2.500240in}{3.476922in}}%
\pgfpathcurveto{\pgfqpoint{2.500240in}{3.463900in}}{\pgfqpoint{2.505414in}{3.451408in}}{\pgfqpoint{2.514623in}{3.442200in}}%
\pgfpathcurveto{\pgfqpoint{2.523831in}{3.432992in}}{\pgfqpoint{2.536322in}{3.427818in}}{\pgfqpoint{2.549345in}{3.427818in}}%
\pgfpathlineto{\pgfqpoint{2.549345in}{3.427818in}}%
\pgfpathclose%
\pgfusepath{stroke,fill}%
\end{pgfscope}%
\begin{pgfscope}%
\pgfpathrectangle{\pgfqpoint{0.786164in}{0.768110in}}{\pgfqpoint{8.851069in}{7.081890in}}%
\pgfusepath{clip}%
\pgfsetbuttcap%
\pgfsetroundjoin%
\definecolor{currentfill}{rgb}{0.250425,0.274290,0.533103}%
\pgfsetfillcolor{currentfill}%
\pgfsetfillopacity{0.700000}%
\pgfsetlinewidth{0.501875pt}%
\definecolor{currentstroke}{rgb}{1.000000,1.000000,1.000000}%
\pgfsetstrokecolor{currentstroke}%
\pgfsetstrokeopacity{0.700000}%
\pgfsetdash{}{0pt}%
\pgfpathmoveto{\pgfqpoint{2.503679in}{3.405919in}}%
\pgfpathcurveto{\pgfqpoint{2.516701in}{3.405919in}}{\pgfqpoint{2.529192in}{3.411093in}}{\pgfqpoint{2.538401in}{3.420302in}}%
\pgfpathcurveto{\pgfqpoint{2.547609in}{3.429510in}}{\pgfqpoint{2.552783in}{3.442001in}}{\pgfqpoint{2.552783in}{3.455024in}}%
\pgfpathcurveto{\pgfqpoint{2.552783in}{3.468047in}}{\pgfqpoint{2.547609in}{3.480538in}}{\pgfqpoint{2.538401in}{3.489746in}}%
\pgfpathcurveto{\pgfqpoint{2.529192in}{3.498955in}}{\pgfqpoint{2.516701in}{3.504129in}}{\pgfqpoint{2.503679in}{3.504129in}}%
\pgfpathcurveto{\pgfqpoint{2.490656in}{3.504129in}}{\pgfqpoint{2.478165in}{3.498955in}}{\pgfqpoint{2.468956in}{3.489746in}}%
\pgfpathcurveto{\pgfqpoint{2.459748in}{3.480538in}}{\pgfqpoint{2.454574in}{3.468047in}}{\pgfqpoint{2.454574in}{3.455024in}}%
\pgfpathcurveto{\pgfqpoint{2.454574in}{3.442001in}}{\pgfqpoint{2.459748in}{3.429510in}}{\pgfqpoint{2.468956in}{3.420302in}}%
\pgfpathcurveto{\pgfqpoint{2.478165in}{3.411093in}}{\pgfqpoint{2.490656in}{3.405919in}}{\pgfqpoint{2.503679in}{3.405919in}}%
\pgfpathlineto{\pgfqpoint{2.503679in}{3.405919in}}%
\pgfpathclose%
\pgfusepath{stroke,fill}%
\end{pgfscope}%
\begin{pgfscope}%
\pgfpathrectangle{\pgfqpoint{0.786164in}{0.768110in}}{\pgfqpoint{8.851069in}{7.081890in}}%
\pgfusepath{clip}%
\pgfsetbuttcap%
\pgfsetroundjoin%
\definecolor{currentfill}{rgb}{0.227802,0.326594,0.546532}%
\pgfsetfillcolor{currentfill}%
\pgfsetfillopacity{0.700000}%
\pgfsetlinewidth{0.501875pt}%
\definecolor{currentstroke}{rgb}{1.000000,1.000000,1.000000}%
\pgfsetstrokecolor{currentstroke}%
\pgfsetstrokeopacity{0.700000}%
\pgfsetdash{}{0pt}%
\pgfpathmoveto{\pgfqpoint{2.129214in}{3.121242in}}%
\pgfpathcurveto{\pgfqpoint{2.142237in}{3.121242in}}{\pgfqpoint{2.154728in}{3.126416in}}{\pgfqpoint{2.163936in}{3.135625in}}%
\pgfpathcurveto{\pgfqpoint{2.173144in}{3.144833in}}{\pgfqpoint{2.178318in}{3.157324in}}{\pgfqpoint{2.178318in}{3.170347in}}%
\pgfpathcurveto{\pgfqpoint{2.178318in}{3.183370in}}{\pgfqpoint{2.173144in}{3.195861in}}{\pgfqpoint{2.163936in}{3.205069in}}%
\pgfpathcurveto{\pgfqpoint{2.154728in}{3.214278in}}{\pgfqpoint{2.142237in}{3.219452in}}{\pgfqpoint{2.129214in}{3.219452in}}%
\pgfpathcurveto{\pgfqpoint{2.116191in}{3.219452in}}{\pgfqpoint{2.103700in}{3.214278in}}{\pgfqpoint{2.094492in}{3.205069in}}%
\pgfpathcurveto{\pgfqpoint{2.085283in}{3.195861in}}{\pgfqpoint{2.080109in}{3.183370in}}{\pgfqpoint{2.080109in}{3.170347in}}%
\pgfpathcurveto{\pgfqpoint{2.080109in}{3.157324in}}{\pgfqpoint{2.085283in}{3.144833in}}{\pgfqpoint{2.094492in}{3.135625in}}%
\pgfpathcurveto{\pgfqpoint{2.103700in}{3.126416in}}{\pgfqpoint{2.116191in}{3.121242in}}{\pgfqpoint{2.129214in}{3.121242in}}%
\pgfpathlineto{\pgfqpoint{2.129214in}{3.121242in}}%
\pgfpathclose%
\pgfusepath{stroke,fill}%
\end{pgfscope}%
\begin{pgfscope}%
\pgfpathrectangle{\pgfqpoint{0.786164in}{0.768110in}}{\pgfqpoint{8.851069in}{7.081890in}}%
\pgfusepath{clip}%
\pgfsetbuttcap%
\pgfsetroundjoin%
\definecolor{currentfill}{rgb}{0.212395,0.359683,0.551710}%
\pgfsetfillcolor{currentfill}%
\pgfsetfillopacity{0.700000}%
\pgfsetlinewidth{0.501875pt}%
\definecolor{currentstroke}{rgb}{1.000000,1.000000,1.000000}%
\pgfsetstrokecolor{currentstroke}%
\pgfsetstrokeopacity{0.700000}%
\pgfsetdash{}{0pt}%
\pgfpathmoveto{\pgfqpoint{2.220547in}{3.230733in}}%
\pgfpathcurveto{\pgfqpoint{2.233569in}{3.230733in}}{\pgfqpoint{2.246060in}{3.235907in}}{\pgfqpoint{2.255269in}{3.245116in}}%
\pgfpathcurveto{\pgfqpoint{2.264477in}{3.254324in}}{\pgfqpoint{2.269651in}{3.266815in}}{\pgfqpoint{2.269651in}{3.279838in}}%
\pgfpathcurveto{\pgfqpoint{2.269651in}{3.292861in}}{\pgfqpoint{2.264477in}{3.305352in}}{\pgfqpoint{2.255269in}{3.314560in}}%
\pgfpathcurveto{\pgfqpoint{2.246060in}{3.323769in}}{\pgfqpoint{2.233569in}{3.328943in}}{\pgfqpoint{2.220547in}{3.328943in}}%
\pgfpathcurveto{\pgfqpoint{2.207524in}{3.328943in}}{\pgfqpoint{2.195033in}{3.323769in}}{\pgfqpoint{2.185824in}{3.314560in}}%
\pgfpathcurveto{\pgfqpoint{2.176616in}{3.305352in}}{\pgfqpoint{2.171442in}{3.292861in}}{\pgfqpoint{2.171442in}{3.279838in}}%
\pgfpathcurveto{\pgfqpoint{2.171442in}{3.266815in}}{\pgfqpoint{2.176616in}{3.254324in}}{\pgfqpoint{2.185824in}{3.245116in}}%
\pgfpathcurveto{\pgfqpoint{2.195033in}{3.235907in}}{\pgfqpoint{2.207524in}{3.230733in}}{\pgfqpoint{2.220547in}{3.230733in}}%
\pgfpathlineto{\pgfqpoint{2.220547in}{3.230733in}}%
\pgfpathclose%
\pgfusepath{stroke,fill}%
\end{pgfscope}%
\begin{pgfscope}%
\pgfpathrectangle{\pgfqpoint{0.786164in}{0.768110in}}{\pgfqpoint{8.851069in}{7.081890in}}%
\pgfusepath{clip}%
\pgfsetbuttcap%
\pgfsetroundjoin%
\definecolor{currentfill}{rgb}{0.208623,0.367752,0.552675}%
\pgfsetfillcolor{currentfill}%
\pgfsetfillopacity{0.700000}%
\pgfsetlinewidth{0.501875pt}%
\definecolor{currentstroke}{rgb}{1.000000,1.000000,1.000000}%
\pgfsetstrokecolor{currentstroke}%
\pgfsetstrokeopacity{0.700000}%
\pgfsetdash{}{0pt}%
\pgfpathmoveto{\pgfqpoint{2.357546in}{3.252632in}}%
\pgfpathcurveto{\pgfqpoint{2.370569in}{3.252632in}}{\pgfqpoint{2.383060in}{3.257806in}}{\pgfqpoint{2.392268in}{3.267014in}}%
\pgfpathcurveto{\pgfqpoint{2.401477in}{3.276223in}}{\pgfqpoint{2.406651in}{3.288714in}}{\pgfqpoint{2.406651in}{3.301736in}}%
\pgfpathcurveto{\pgfqpoint{2.406651in}{3.314759in}}{\pgfqpoint{2.401477in}{3.327250in}}{\pgfqpoint{2.392268in}{3.336459in}}%
\pgfpathcurveto{\pgfqpoint{2.383060in}{3.345667in}}{\pgfqpoint{2.370569in}{3.350841in}}{\pgfqpoint{2.357546in}{3.350841in}}%
\pgfpathcurveto{\pgfqpoint{2.344523in}{3.350841in}}{\pgfqpoint{2.332032in}{3.345667in}}{\pgfqpoint{2.322824in}{3.336459in}}%
\pgfpathcurveto{\pgfqpoint{2.313615in}{3.327250in}}{\pgfqpoint{2.308441in}{3.314759in}}{\pgfqpoint{2.308441in}{3.301736in}}%
\pgfpathcurveto{\pgfqpoint{2.308441in}{3.288714in}}{\pgfqpoint{2.313615in}{3.276223in}}{\pgfqpoint{2.322824in}{3.267014in}}%
\pgfpathcurveto{\pgfqpoint{2.332032in}{3.257806in}}{\pgfqpoint{2.344523in}{3.252632in}}{\pgfqpoint{2.357546in}{3.252632in}}%
\pgfpathlineto{\pgfqpoint{2.357546in}{3.252632in}}%
\pgfpathclose%
\pgfusepath{stroke,fill}%
\end{pgfscope}%
\begin{pgfscope}%
\pgfpathrectangle{\pgfqpoint{0.786164in}{0.768110in}}{\pgfqpoint{8.851069in}{7.081890in}}%
\pgfusepath{clip}%
\pgfsetbuttcap%
\pgfsetroundjoin%
\definecolor{currentfill}{rgb}{0.282623,0.140926,0.457517}%
\pgfsetfillcolor{currentfill}%
\pgfsetfillopacity{0.700000}%
\pgfsetlinewidth{0.501875pt}%
\definecolor{currentstroke}{rgb}{1.000000,1.000000,1.000000}%
\pgfsetstrokecolor{currentstroke}%
\pgfsetstrokeopacity{0.700000}%
\pgfsetdash{}{0pt}%
\pgfpathmoveto{\pgfqpoint{3.727539in}{4.128561in}}%
\pgfpathcurveto{\pgfqpoint{3.740562in}{4.128561in}}{\pgfqpoint{3.753053in}{4.133735in}}{\pgfqpoint{3.762261in}{4.142944in}}%
\pgfpathcurveto{\pgfqpoint{3.771470in}{4.152152in}}{\pgfqpoint{3.776644in}{4.164643in}}{\pgfqpoint{3.776644in}{4.177666in}}%
\pgfpathcurveto{\pgfqpoint{3.776644in}{4.190688in}}{\pgfqpoint{3.771470in}{4.203180in}}{\pgfqpoint{3.762261in}{4.212388in}}%
\pgfpathcurveto{\pgfqpoint{3.753053in}{4.221596in}}{\pgfqpoint{3.740562in}{4.226770in}}{\pgfqpoint{3.727539in}{4.226770in}}%
\pgfpathcurveto{\pgfqpoint{3.714516in}{4.226770in}}{\pgfqpoint{3.702025in}{4.221596in}}{\pgfqpoint{3.692817in}{4.212388in}}%
\pgfpathcurveto{\pgfqpoint{3.683608in}{4.203180in}}{\pgfqpoint{3.678434in}{4.190688in}}{\pgfqpoint{3.678434in}{4.177666in}}%
\pgfpathcurveto{\pgfqpoint{3.678434in}{4.164643in}}{\pgfqpoint{3.683608in}{4.152152in}}{\pgfqpoint{3.692817in}{4.142944in}}%
\pgfpathcurveto{\pgfqpoint{3.702025in}{4.133735in}}{\pgfqpoint{3.714516in}{4.128561in}}{\pgfqpoint{3.727539in}{4.128561in}}%
\pgfpathlineto{\pgfqpoint{3.727539in}{4.128561in}}%
\pgfpathclose%
\pgfusepath{stroke,fill}%
\end{pgfscope}%
\begin{pgfscope}%
\pgfpathrectangle{\pgfqpoint{0.786164in}{0.768110in}}{\pgfqpoint{8.851069in}{7.081890in}}%
\pgfusepath{clip}%
\pgfsetbuttcap%
\pgfsetroundjoin%
\definecolor{currentfill}{rgb}{0.281887,0.150881,0.465405}%
\pgfsetfillcolor{currentfill}%
\pgfsetfillopacity{0.700000}%
\pgfsetlinewidth{0.501875pt}%
\definecolor{currentstroke}{rgb}{1.000000,1.000000,1.000000}%
\pgfsetstrokecolor{currentstroke}%
\pgfsetstrokeopacity{0.700000}%
\pgfsetdash{}{0pt}%
\pgfpathmoveto{\pgfqpoint{3.672739in}{4.084765in}}%
\pgfpathcurveto{\pgfqpoint{3.685762in}{4.084765in}}{\pgfqpoint{3.698253in}{4.089939in}}{\pgfqpoint{3.707462in}{4.099147in}}%
\pgfpathcurveto{\pgfqpoint{3.716670in}{4.108356in}}{\pgfqpoint{3.721844in}{4.120847in}}{\pgfqpoint{3.721844in}{4.133869in}}%
\pgfpathcurveto{\pgfqpoint{3.721844in}{4.146892in}}{\pgfqpoint{3.716670in}{4.159383in}}{\pgfqpoint{3.707462in}{4.168592in}}%
\pgfpathcurveto{\pgfqpoint{3.698253in}{4.177800in}}{\pgfqpoint{3.685762in}{4.182974in}}{\pgfqpoint{3.672739in}{4.182974in}}%
\pgfpathcurveto{\pgfqpoint{3.659717in}{4.182974in}}{\pgfqpoint{3.647226in}{4.177800in}}{\pgfqpoint{3.638017in}{4.168592in}}%
\pgfpathcurveto{\pgfqpoint{3.628809in}{4.159383in}}{\pgfqpoint{3.623635in}{4.146892in}}{\pgfqpoint{3.623635in}{4.133869in}}%
\pgfpathcurveto{\pgfqpoint{3.623635in}{4.120847in}}{\pgfqpoint{3.628809in}{4.108356in}}{\pgfqpoint{3.638017in}{4.099147in}}%
\pgfpathcurveto{\pgfqpoint{3.647226in}{4.089939in}}{\pgfqpoint{3.659717in}{4.084765in}}{\pgfqpoint{3.672739in}{4.084765in}}%
\pgfpathlineto{\pgfqpoint{3.672739in}{4.084765in}}%
\pgfpathclose%
\pgfusepath{stroke,fill}%
\end{pgfscope}%
\begin{pgfscope}%
\pgfpathrectangle{\pgfqpoint{0.786164in}{0.768110in}}{\pgfqpoint{8.851069in}{7.081890in}}%
\pgfusepath{clip}%
\pgfsetbuttcap%
\pgfsetroundjoin%
\definecolor{currentfill}{rgb}{0.281412,0.155834,0.469201}%
\pgfsetfillcolor{currentfill}%
\pgfsetfillopacity{0.700000}%
\pgfsetlinewidth{0.501875pt}%
\definecolor{currentstroke}{rgb}{1.000000,1.000000,1.000000}%
\pgfsetstrokecolor{currentstroke}%
\pgfsetstrokeopacity{0.700000}%
\pgfsetdash{}{0pt}%
\pgfpathmoveto{\pgfqpoint{3.754939in}{4.128561in}}%
\pgfpathcurveto{\pgfqpoint{3.767962in}{4.128561in}}{\pgfqpoint{3.780453in}{4.133735in}}{\pgfqpoint{3.789661in}{4.142944in}}%
\pgfpathcurveto{\pgfqpoint{3.798870in}{4.152152in}}{\pgfqpoint{3.804044in}{4.164643in}}{\pgfqpoint{3.804044in}{4.177666in}}%
\pgfpathcurveto{\pgfqpoint{3.804044in}{4.190688in}}{\pgfqpoint{3.798870in}{4.203180in}}{\pgfqpoint{3.789661in}{4.212388in}}%
\pgfpathcurveto{\pgfqpoint{3.780453in}{4.221596in}}{\pgfqpoint{3.767962in}{4.226770in}}{\pgfqpoint{3.754939in}{4.226770in}}%
\pgfpathcurveto{\pgfqpoint{3.741916in}{4.226770in}}{\pgfqpoint{3.729425in}{4.221596in}}{\pgfqpoint{3.720217in}{4.212388in}}%
\pgfpathcurveto{\pgfqpoint{3.711008in}{4.203180in}}{\pgfqpoint{3.705834in}{4.190688in}}{\pgfqpoint{3.705834in}{4.177666in}}%
\pgfpathcurveto{\pgfqpoint{3.705834in}{4.164643in}}{\pgfqpoint{3.711008in}{4.152152in}}{\pgfqpoint{3.720217in}{4.142944in}}%
\pgfpathcurveto{\pgfqpoint{3.729425in}{4.133735in}}{\pgfqpoint{3.741916in}{4.128561in}}{\pgfqpoint{3.754939in}{4.128561in}}%
\pgfpathlineto{\pgfqpoint{3.754939in}{4.128561in}}%
\pgfpathclose%
\pgfusepath{stroke,fill}%
\end{pgfscope}%
\begin{pgfscope}%
\pgfpathrectangle{\pgfqpoint{0.786164in}{0.768110in}}{\pgfqpoint{8.851069in}{7.081890in}}%
\pgfusepath{clip}%
\pgfsetbuttcap%
\pgfsetroundjoin%
\definecolor{currentfill}{rgb}{0.280255,0.165693,0.476498}%
\pgfsetfillcolor{currentfill}%
\pgfsetfillopacity{0.700000}%
\pgfsetlinewidth{0.501875pt}%
\definecolor{currentstroke}{rgb}{1.000000,1.000000,1.000000}%
\pgfsetstrokecolor{currentstroke}%
\pgfsetstrokeopacity{0.700000}%
\pgfsetdash{}{0pt}%
\pgfpathmoveto{\pgfqpoint{3.745806in}{4.150459in}}%
\pgfpathcurveto{\pgfqpoint{3.758828in}{4.150459in}}{\pgfqpoint{3.771319in}{4.155633in}}{\pgfqpoint{3.780528in}{4.164842in}}%
\pgfpathcurveto{\pgfqpoint{3.789736in}{4.174050in}}{\pgfqpoint{3.794910in}{4.186541in}}{\pgfqpoint{3.794910in}{4.199564in}}%
\pgfpathcurveto{\pgfqpoint{3.794910in}{4.212587in}}{\pgfqpoint{3.789736in}{4.225078in}}{\pgfqpoint{3.780528in}{4.234286in}}%
\pgfpathcurveto{\pgfqpoint{3.771319in}{4.243495in}}{\pgfqpoint{3.758828in}{4.248669in}}{\pgfqpoint{3.745806in}{4.248669in}}%
\pgfpathcurveto{\pgfqpoint{3.732783in}{4.248669in}}{\pgfqpoint{3.720292in}{4.243495in}}{\pgfqpoint{3.711083in}{4.234286in}}%
\pgfpathcurveto{\pgfqpoint{3.701875in}{4.225078in}}{\pgfqpoint{3.696701in}{4.212587in}}{\pgfqpoint{3.696701in}{4.199564in}}%
\pgfpathcurveto{\pgfqpoint{3.696701in}{4.186541in}}{\pgfqpoint{3.701875in}{4.174050in}}{\pgfqpoint{3.711083in}{4.164842in}}%
\pgfpathcurveto{\pgfqpoint{3.720292in}{4.155633in}}{\pgfqpoint{3.732783in}{4.150459in}}{\pgfqpoint{3.745806in}{4.150459in}}%
\pgfpathlineto{\pgfqpoint{3.745806in}{4.150459in}}%
\pgfpathclose%
\pgfusepath{stroke,fill}%
\end{pgfscope}%
\begin{pgfscope}%
\pgfpathrectangle{\pgfqpoint{0.786164in}{0.768110in}}{\pgfqpoint{8.851069in}{7.081890in}}%
\pgfusepath{clip}%
\pgfsetbuttcap%
\pgfsetroundjoin%
\definecolor{currentfill}{rgb}{0.278012,0.180367,0.486697}%
\pgfsetfillcolor{currentfill}%
\pgfsetfillopacity{0.700000}%
\pgfsetlinewidth{0.501875pt}%
\definecolor{currentstroke}{rgb}{1.000000,1.000000,1.000000}%
\pgfsetstrokecolor{currentstroke}%
\pgfsetstrokeopacity{0.700000}%
\pgfsetdash{}{0pt}%
\pgfpathmoveto{\pgfqpoint{3.608806in}{3.997172in}}%
\pgfpathcurveto{\pgfqpoint{3.621829in}{3.997172in}}{\pgfqpoint{3.634320in}{4.002346in}}{\pgfqpoint{3.643529in}{4.011554in}}%
\pgfpathcurveto{\pgfqpoint{3.652737in}{4.020763in}}{\pgfqpoint{3.657911in}{4.033254in}}{\pgfqpoint{3.657911in}{4.046276in}}%
\pgfpathcurveto{\pgfqpoint{3.657911in}{4.059299in}}{\pgfqpoint{3.652737in}{4.071790in}}{\pgfqpoint{3.643529in}{4.080999in}}%
\pgfpathcurveto{\pgfqpoint{3.634320in}{4.090207in}}{\pgfqpoint{3.621829in}{4.095381in}}{\pgfqpoint{3.608806in}{4.095381in}}%
\pgfpathcurveto{\pgfqpoint{3.595784in}{4.095381in}}{\pgfqpoint{3.583293in}{4.090207in}}{\pgfqpoint{3.574084in}{4.080999in}}%
\pgfpathcurveto{\pgfqpoint{3.564876in}{4.071790in}}{\pgfqpoint{3.559702in}{4.059299in}}{\pgfqpoint{3.559702in}{4.046276in}}%
\pgfpathcurveto{\pgfqpoint{3.559702in}{4.033254in}}{\pgfqpoint{3.564876in}{4.020763in}}{\pgfqpoint{3.574084in}{4.011554in}}%
\pgfpathcurveto{\pgfqpoint{3.583293in}{4.002346in}}{\pgfqpoint{3.595784in}{3.997172in}}{\pgfqpoint{3.608806in}{3.997172in}}%
\pgfpathlineto{\pgfqpoint{3.608806in}{3.997172in}}%
\pgfpathclose%
\pgfusepath{stroke,fill}%
\end{pgfscope}%
\begin{pgfscope}%
\pgfpathrectangle{\pgfqpoint{0.786164in}{0.768110in}}{\pgfqpoint{8.851069in}{7.081890in}}%
\pgfusepath{clip}%
\pgfsetbuttcap%
\pgfsetroundjoin%
\definecolor{currentfill}{rgb}{0.273006,0.204520,0.501721}%
\pgfsetfillcolor{currentfill}%
\pgfsetfillopacity{0.700000}%
\pgfsetlinewidth{0.501875pt}%
\definecolor{currentstroke}{rgb}{1.000000,1.000000,1.000000}%
\pgfsetstrokecolor{currentstroke}%
\pgfsetstrokeopacity{0.700000}%
\pgfsetdash{}{0pt}%
\pgfpathmoveto{\pgfqpoint{3.389607in}{3.778189in}}%
\pgfpathcurveto{\pgfqpoint{3.402630in}{3.778189in}}{\pgfqpoint{3.415121in}{3.783363in}}{\pgfqpoint{3.424330in}{3.792572in}}%
\pgfpathcurveto{\pgfqpoint{3.433538in}{3.801780in}}{\pgfqpoint{3.438712in}{3.814271in}}{\pgfqpoint{3.438712in}{3.827294in}}%
\pgfpathcurveto{\pgfqpoint{3.438712in}{3.840317in}}{\pgfqpoint{3.433538in}{3.852808in}}{\pgfqpoint{3.424330in}{3.862016in}}%
\pgfpathcurveto{\pgfqpoint{3.415121in}{3.871225in}}{\pgfqpoint{3.402630in}{3.876399in}}{\pgfqpoint{3.389607in}{3.876399in}}%
\pgfpathcurveto{\pgfqpoint{3.376585in}{3.876399in}}{\pgfqpoint{3.364094in}{3.871225in}}{\pgfqpoint{3.354885in}{3.862016in}}%
\pgfpathcurveto{\pgfqpoint{3.345677in}{3.852808in}}{\pgfqpoint{3.340503in}{3.840317in}}{\pgfqpoint{3.340503in}{3.827294in}}%
\pgfpathcurveto{\pgfqpoint{3.340503in}{3.814271in}}{\pgfqpoint{3.345677in}{3.801780in}}{\pgfqpoint{3.354885in}{3.792572in}}%
\pgfpathcurveto{\pgfqpoint{3.364094in}{3.783363in}}{\pgfqpoint{3.376585in}{3.778189in}}{\pgfqpoint{3.389607in}{3.778189in}}%
\pgfpathlineto{\pgfqpoint{3.389607in}{3.778189in}}%
\pgfpathclose%
\pgfusepath{stroke,fill}%
\end{pgfscope}%
\begin{pgfscope}%
\pgfpathrectangle{\pgfqpoint{0.786164in}{0.768110in}}{\pgfqpoint{8.851069in}{7.081890in}}%
\pgfusepath{clip}%
\pgfsetbuttcap%
\pgfsetroundjoin%
\definecolor{currentfill}{rgb}{0.270595,0.214069,0.507052}%
\pgfsetfillcolor{currentfill}%
\pgfsetfillopacity{0.700000}%
\pgfsetlinewidth{0.501875pt}%
\definecolor{currentstroke}{rgb}{1.000000,1.000000,1.000000}%
\pgfsetstrokecolor{currentstroke}%
\pgfsetstrokeopacity{0.700000}%
\pgfsetdash{}{0pt}%
\pgfpathmoveto{\pgfqpoint{3.581407in}{3.843884in}}%
\pgfpathcurveto{\pgfqpoint{3.594429in}{3.843884in}}{\pgfqpoint{3.606920in}{3.849058in}}{\pgfqpoint{3.616129in}{3.858266in}}%
\pgfpathcurveto{\pgfqpoint{3.625337in}{3.867475in}}{\pgfqpoint{3.630511in}{3.879966in}}{\pgfqpoint{3.630511in}{3.892989in}}%
\pgfpathcurveto{\pgfqpoint{3.630511in}{3.906011in}}{\pgfqpoint{3.625337in}{3.918503in}}{\pgfqpoint{3.616129in}{3.927711in}}%
\pgfpathcurveto{\pgfqpoint{3.606920in}{3.936919in}}{\pgfqpoint{3.594429in}{3.942093in}}{\pgfqpoint{3.581407in}{3.942093in}}%
\pgfpathcurveto{\pgfqpoint{3.568384in}{3.942093in}}{\pgfqpoint{3.555893in}{3.936919in}}{\pgfqpoint{3.546684in}{3.927711in}}%
\pgfpathcurveto{\pgfqpoint{3.537476in}{3.918503in}}{\pgfqpoint{3.532302in}{3.906011in}}{\pgfqpoint{3.532302in}{3.892989in}}%
\pgfpathcurveto{\pgfqpoint{3.532302in}{3.879966in}}{\pgfqpoint{3.537476in}{3.867475in}}{\pgfqpoint{3.546684in}{3.858266in}}%
\pgfpathcurveto{\pgfqpoint{3.555893in}{3.849058in}}{\pgfqpoint{3.568384in}{3.843884in}}{\pgfqpoint{3.581407in}{3.843884in}}%
\pgfpathlineto{\pgfqpoint{3.581407in}{3.843884in}}%
\pgfpathclose%
\pgfusepath{stroke,fill}%
\end{pgfscope}%
\begin{pgfscope}%
\pgfpathrectangle{\pgfqpoint{0.786164in}{0.768110in}}{\pgfqpoint{8.851069in}{7.081890in}}%
\pgfusepath{clip}%
\pgfsetbuttcap%
\pgfsetroundjoin%
\definecolor{currentfill}{rgb}{0.267968,0.223549,0.512008}%
\pgfsetfillcolor{currentfill}%
\pgfsetfillopacity{0.700000}%
\pgfsetlinewidth{0.501875pt}%
\definecolor{currentstroke}{rgb}{1.000000,1.000000,1.000000}%
\pgfsetstrokecolor{currentstroke}%
\pgfsetstrokeopacity{0.700000}%
\pgfsetdash{}{0pt}%
\pgfpathmoveto{\pgfqpoint{3.426141in}{3.778189in}}%
\pgfpathcurveto{\pgfqpoint{3.439163in}{3.778189in}}{\pgfqpoint{3.451654in}{3.783363in}}{\pgfqpoint{3.460863in}{3.792572in}}%
\pgfpathcurveto{\pgfqpoint{3.470071in}{3.801780in}}{\pgfqpoint{3.475245in}{3.814271in}}{\pgfqpoint{3.475245in}{3.827294in}}%
\pgfpathcurveto{\pgfqpoint{3.475245in}{3.840317in}}{\pgfqpoint{3.470071in}{3.852808in}}{\pgfqpoint{3.460863in}{3.862016in}}%
\pgfpathcurveto{\pgfqpoint{3.451654in}{3.871225in}}{\pgfqpoint{3.439163in}{3.876399in}}{\pgfqpoint{3.426141in}{3.876399in}}%
\pgfpathcurveto{\pgfqpoint{3.413118in}{3.876399in}}{\pgfqpoint{3.400627in}{3.871225in}}{\pgfqpoint{3.391418in}{3.862016in}}%
\pgfpathcurveto{\pgfqpoint{3.382210in}{3.852808in}}{\pgfqpoint{3.377036in}{3.840317in}}{\pgfqpoint{3.377036in}{3.827294in}}%
\pgfpathcurveto{\pgfqpoint{3.377036in}{3.814271in}}{\pgfqpoint{3.382210in}{3.801780in}}{\pgfqpoint{3.391418in}{3.792572in}}%
\pgfpathcurveto{\pgfqpoint{3.400627in}{3.783363in}}{\pgfqpoint{3.413118in}{3.778189in}}{\pgfqpoint{3.426141in}{3.778189in}}%
\pgfpathlineto{\pgfqpoint{3.426141in}{3.778189in}}%
\pgfpathclose%
\pgfusepath{stroke,fill}%
\end{pgfscope}%
\begin{pgfscope}%
\pgfpathrectangle{\pgfqpoint{0.786164in}{0.768110in}}{\pgfqpoint{8.851069in}{7.081890in}}%
\pgfusepath{clip}%
\pgfsetbuttcap%
\pgfsetroundjoin%
\definecolor{currentfill}{rgb}{0.257322,0.256130,0.526563}%
\pgfsetfillcolor{currentfill}%
\pgfsetfillopacity{0.700000}%
\pgfsetlinewidth{0.501875pt}%
\definecolor{currentstroke}{rgb}{1.000000,1.000000,1.000000}%
\pgfsetstrokecolor{currentstroke}%
\pgfsetstrokeopacity{0.700000}%
\pgfsetdash{}{0pt}%
\pgfpathmoveto{\pgfqpoint{3.380474in}{3.690596in}}%
\pgfpathcurveto{\pgfqpoint{3.393497in}{3.690596in}}{\pgfqpoint{3.405988in}{3.695770in}}{\pgfqpoint{3.415196in}{3.704979in}}%
\pgfpathcurveto{\pgfqpoint{3.424405in}{3.714187in}}{\pgfqpoint{3.429579in}{3.726678in}}{\pgfqpoint{3.429579in}{3.739701in}}%
\pgfpathcurveto{\pgfqpoint{3.429579in}{3.752724in}}{\pgfqpoint{3.424405in}{3.765215in}}{\pgfqpoint{3.415196in}{3.774423in}}%
\pgfpathcurveto{\pgfqpoint{3.405988in}{3.783632in}}{\pgfqpoint{3.393497in}{3.788806in}}{\pgfqpoint{3.380474in}{3.788806in}}%
\pgfpathcurveto{\pgfqpoint{3.367451in}{3.788806in}}{\pgfqpoint{3.354960in}{3.783632in}}{\pgfqpoint{3.345752in}{3.774423in}}%
\pgfpathcurveto{\pgfqpoint{3.336544in}{3.765215in}}{\pgfqpoint{3.331370in}{3.752724in}}{\pgfqpoint{3.331370in}{3.739701in}}%
\pgfpathcurveto{\pgfqpoint{3.331370in}{3.726678in}}{\pgfqpoint{3.336544in}{3.714187in}}{\pgfqpoint{3.345752in}{3.704979in}}%
\pgfpathcurveto{\pgfqpoint{3.354960in}{3.695770in}}{\pgfqpoint{3.367451in}{3.690596in}}{\pgfqpoint{3.380474in}{3.690596in}}%
\pgfpathlineto{\pgfqpoint{3.380474in}{3.690596in}}%
\pgfpathclose%
\pgfusepath{stroke,fill}%
\end{pgfscope}%
\begin{pgfscope}%
\pgfpathrectangle{\pgfqpoint{0.786164in}{0.768110in}}{\pgfqpoint{8.851069in}{7.081890in}}%
\pgfusepath{clip}%
\pgfsetbuttcap%
\pgfsetroundjoin%
\definecolor{currentfill}{rgb}{0.250425,0.274290,0.533103}%
\pgfsetfillcolor{currentfill}%
\pgfsetfillopacity{0.700000}%
\pgfsetlinewidth{0.501875pt}%
\definecolor{currentstroke}{rgb}{1.000000,1.000000,1.000000}%
\pgfsetstrokecolor{currentstroke}%
\pgfsetstrokeopacity{0.700000}%
\pgfsetdash{}{0pt}%
\pgfpathmoveto{\pgfqpoint{3.407874in}{3.603003in}}%
\pgfpathcurveto{\pgfqpoint{3.420897in}{3.603003in}}{\pgfqpoint{3.433388in}{3.608177in}}{\pgfqpoint{3.442596in}{3.617386in}}%
\pgfpathcurveto{\pgfqpoint{3.451805in}{3.626594in}}{\pgfqpoint{3.456979in}{3.639085in}}{\pgfqpoint{3.456979in}{3.652108in}}%
\pgfpathcurveto{\pgfqpoint{3.456979in}{3.665131in}}{\pgfqpoint{3.451805in}{3.677622in}}{\pgfqpoint{3.442596in}{3.686830in}}%
\pgfpathcurveto{\pgfqpoint{3.433388in}{3.696039in}}{\pgfqpoint{3.420897in}{3.701213in}}{\pgfqpoint{3.407874in}{3.701213in}}%
\pgfpathcurveto{\pgfqpoint{3.394851in}{3.701213in}}{\pgfqpoint{3.382360in}{3.696039in}}{\pgfqpoint{3.373152in}{3.686830in}}%
\pgfpathcurveto{\pgfqpoint{3.363943in}{3.677622in}}{\pgfqpoint{3.358769in}{3.665131in}}{\pgfqpoint{3.358769in}{3.652108in}}%
\pgfpathcurveto{\pgfqpoint{3.358769in}{3.639085in}}{\pgfqpoint{3.363943in}{3.626594in}}{\pgfqpoint{3.373152in}{3.617386in}}%
\pgfpathcurveto{\pgfqpoint{3.382360in}{3.608177in}}{\pgfqpoint{3.394851in}{3.603003in}}{\pgfqpoint{3.407874in}{3.603003in}}%
\pgfpathlineto{\pgfqpoint{3.407874in}{3.603003in}}%
\pgfpathclose%
\pgfusepath{stroke,fill}%
\end{pgfscope}%
\begin{pgfscope}%
\pgfpathrectangle{\pgfqpoint{0.786164in}{0.768110in}}{\pgfqpoint{8.851069in}{7.081890in}}%
\pgfusepath{clip}%
\pgfsetbuttcap%
\pgfsetroundjoin%
\definecolor{currentfill}{rgb}{0.241237,0.296485,0.539709}%
\pgfsetfillcolor{currentfill}%
\pgfsetfillopacity{0.700000}%
\pgfsetlinewidth{0.501875pt}%
\definecolor{currentstroke}{rgb}{1.000000,1.000000,1.000000}%
\pgfsetstrokecolor{currentstroke}%
\pgfsetstrokeopacity{0.700000}%
\pgfsetdash{}{0pt}%
\pgfpathmoveto{\pgfqpoint{3.252608in}{3.537309in}}%
\pgfpathcurveto{\pgfqpoint{3.265631in}{3.537309in}}{\pgfqpoint{3.278122in}{3.542483in}}{\pgfqpoint{3.287330in}{3.551691in}}%
\pgfpathcurveto{\pgfqpoint{3.296539in}{3.560900in}}{\pgfqpoint{3.301713in}{3.573391in}}{\pgfqpoint{3.301713in}{3.586413in}}%
\pgfpathcurveto{\pgfqpoint{3.301713in}{3.599436in}}{\pgfqpoint{3.296539in}{3.611927in}}{\pgfqpoint{3.287330in}{3.621136in}}%
\pgfpathcurveto{\pgfqpoint{3.278122in}{3.630344in}}{\pgfqpoint{3.265631in}{3.635518in}}{\pgfqpoint{3.252608in}{3.635518in}}%
\pgfpathcurveto{\pgfqpoint{3.239585in}{3.635518in}}{\pgfqpoint{3.227094in}{3.630344in}}{\pgfqpoint{3.217886in}{3.621136in}}%
\pgfpathcurveto{\pgfqpoint{3.208678in}{3.611927in}}{\pgfqpoint{3.203504in}{3.599436in}}{\pgfqpoint{3.203504in}{3.586413in}}%
\pgfpathcurveto{\pgfqpoint{3.203504in}{3.573391in}}{\pgfqpoint{3.208678in}{3.560900in}}{\pgfqpoint{3.217886in}{3.551691in}}%
\pgfpathcurveto{\pgfqpoint{3.227094in}{3.542483in}}{\pgfqpoint{3.239585in}{3.537309in}}{\pgfqpoint{3.252608in}{3.537309in}}%
\pgfpathlineto{\pgfqpoint{3.252608in}{3.537309in}}%
\pgfpathclose%
\pgfusepath{stroke,fill}%
\end{pgfscope}%
\begin{pgfscope}%
\pgfpathrectangle{\pgfqpoint{0.786164in}{0.768110in}}{\pgfqpoint{8.851069in}{7.081890in}}%
\pgfusepath{clip}%
\pgfsetbuttcap%
\pgfsetroundjoin%
\definecolor{currentfill}{rgb}{0.241237,0.296485,0.539709}%
\pgfsetfillcolor{currentfill}%
\pgfsetfillopacity{0.700000}%
\pgfsetlinewidth{0.501875pt}%
\definecolor{currentstroke}{rgb}{1.000000,1.000000,1.000000}%
\pgfsetstrokecolor{currentstroke}%
\pgfsetstrokeopacity{0.700000}%
\pgfsetdash{}{0pt}%
\pgfpathmoveto{\pgfqpoint{3.289141in}{3.581105in}}%
\pgfpathcurveto{\pgfqpoint{3.302164in}{3.581105in}}{\pgfqpoint{3.314655in}{3.586279in}}{\pgfqpoint{3.323864in}{3.595488in}}%
\pgfpathcurveto{\pgfqpoint{3.333072in}{3.604696in}}{\pgfqpoint{3.338246in}{3.617187in}}{\pgfqpoint{3.338246in}{3.630210in}}%
\pgfpathcurveto{\pgfqpoint{3.338246in}{3.643233in}}{\pgfqpoint{3.333072in}{3.655724in}}{\pgfqpoint{3.323864in}{3.664932in}}%
\pgfpathcurveto{\pgfqpoint{3.314655in}{3.674141in}}{\pgfqpoint{3.302164in}{3.679315in}}{\pgfqpoint{3.289141in}{3.679315in}}%
\pgfpathcurveto{\pgfqpoint{3.276119in}{3.679315in}}{\pgfqpoint{3.263628in}{3.674141in}}{\pgfqpoint{3.254419in}{3.664932in}}%
\pgfpathcurveto{\pgfqpoint{3.245211in}{3.655724in}}{\pgfqpoint{3.240037in}{3.643233in}}{\pgfqpoint{3.240037in}{3.630210in}}%
\pgfpathcurveto{\pgfqpoint{3.240037in}{3.617187in}}{\pgfqpoint{3.245211in}{3.604696in}}{\pgfqpoint{3.254419in}{3.595488in}}%
\pgfpathcurveto{\pgfqpoint{3.263628in}{3.586279in}}{\pgfqpoint{3.276119in}{3.581105in}}{\pgfqpoint{3.289141in}{3.581105in}}%
\pgfpathlineto{\pgfqpoint{3.289141in}{3.581105in}}%
\pgfpathclose%
\pgfusepath{stroke,fill}%
\end{pgfscope}%
\begin{pgfscope}%
\pgfpathrectangle{\pgfqpoint{0.786164in}{0.768110in}}{\pgfqpoint{8.851069in}{7.081890in}}%
\pgfusepath{clip}%
\pgfsetbuttcap%
\pgfsetroundjoin%
\definecolor{currentfill}{rgb}{0.243113,0.292092,0.538516}%
\pgfsetfillcolor{currentfill}%
\pgfsetfillopacity{0.700000}%
\pgfsetlinewidth{0.501875pt}%
\definecolor{currentstroke}{rgb}{1.000000,1.000000,1.000000}%
\pgfsetstrokecolor{currentstroke}%
\pgfsetstrokeopacity{0.700000}%
\pgfsetdash{}{0pt}%
\pgfpathmoveto{\pgfqpoint{3.307408in}{3.624902in}}%
\pgfpathcurveto{\pgfqpoint{3.320431in}{3.624902in}}{\pgfqpoint{3.332922in}{3.630076in}}{\pgfqpoint{3.342130in}{3.639284in}}%
\pgfpathcurveto{\pgfqpoint{3.351339in}{3.648493in}}{\pgfqpoint{3.356513in}{3.660984in}}{\pgfqpoint{3.356513in}{3.674006in}}%
\pgfpathcurveto{\pgfqpoint{3.356513in}{3.687029in}}{\pgfqpoint{3.351339in}{3.699520in}}{\pgfqpoint{3.342130in}{3.708729in}}%
\pgfpathcurveto{\pgfqpoint{3.332922in}{3.717937in}}{\pgfqpoint{3.320431in}{3.723111in}}{\pgfqpoint{3.307408in}{3.723111in}}%
\pgfpathcurveto{\pgfqpoint{3.294385in}{3.723111in}}{\pgfqpoint{3.281894in}{3.717937in}}{\pgfqpoint{3.272686in}{3.708729in}}%
\pgfpathcurveto{\pgfqpoint{3.263477in}{3.699520in}}{\pgfqpoint{3.258303in}{3.687029in}}{\pgfqpoint{3.258303in}{3.674006in}}%
\pgfpathcurveto{\pgfqpoint{3.258303in}{3.660984in}}{\pgfqpoint{3.263477in}{3.648493in}}{\pgfqpoint{3.272686in}{3.639284in}}%
\pgfpathcurveto{\pgfqpoint{3.281894in}{3.630076in}}{\pgfqpoint{3.294385in}{3.624902in}}{\pgfqpoint{3.307408in}{3.624902in}}%
\pgfpathlineto{\pgfqpoint{3.307408in}{3.624902in}}%
\pgfpathclose%
\pgfusepath{stroke,fill}%
\end{pgfscope}%
\begin{pgfscope}%
\pgfpathrectangle{\pgfqpoint{0.786164in}{0.768110in}}{\pgfqpoint{8.851069in}{7.081890in}}%
\pgfusepath{clip}%
\pgfsetbuttcap%
\pgfsetroundjoin%
\definecolor{currentfill}{rgb}{0.243113,0.292092,0.538516}%
\pgfsetfillcolor{currentfill}%
\pgfsetfillopacity{0.700000}%
\pgfsetlinewidth{0.501875pt}%
\definecolor{currentstroke}{rgb}{1.000000,1.000000,1.000000}%
\pgfsetstrokecolor{currentstroke}%
\pgfsetstrokeopacity{0.700000}%
\pgfsetdash{}{0pt}%
\pgfpathmoveto{\pgfqpoint{3.353074in}{3.668698in}}%
\pgfpathcurveto{\pgfqpoint{3.366097in}{3.668698in}}{\pgfqpoint{3.378588in}{3.673872in}}{\pgfqpoint{3.387797in}{3.683081in}}%
\pgfpathcurveto{\pgfqpoint{3.397005in}{3.692289in}}{\pgfqpoint{3.402179in}{3.704780in}}{\pgfqpoint{3.402179in}{3.717803in}}%
\pgfpathcurveto{\pgfqpoint{3.402179in}{3.730826in}}{\pgfqpoint{3.397005in}{3.743317in}}{\pgfqpoint{3.387797in}{3.752525in}}%
\pgfpathcurveto{\pgfqpoint{3.378588in}{3.761733in}}{\pgfqpoint{3.366097in}{3.766907in}}{\pgfqpoint{3.353074in}{3.766907in}}%
\pgfpathcurveto{\pgfqpoint{3.340052in}{3.766907in}}{\pgfqpoint{3.327561in}{3.761733in}}{\pgfqpoint{3.318352in}{3.752525in}}%
\pgfpathcurveto{\pgfqpoint{3.309144in}{3.743317in}}{\pgfqpoint{3.303970in}{3.730826in}}{\pgfqpoint{3.303970in}{3.717803in}}%
\pgfpathcurveto{\pgfqpoint{3.303970in}{3.704780in}}{\pgfqpoint{3.309144in}{3.692289in}}{\pgfqpoint{3.318352in}{3.683081in}}%
\pgfpathcurveto{\pgfqpoint{3.327561in}{3.673872in}}{\pgfqpoint{3.340052in}{3.668698in}}{\pgfqpoint{3.353074in}{3.668698in}}%
\pgfpathlineto{\pgfqpoint{3.353074in}{3.668698in}}%
\pgfpathclose%
\pgfusepath{stroke,fill}%
\end{pgfscope}%
\begin{pgfscope}%
\pgfpathrectangle{\pgfqpoint{0.786164in}{0.768110in}}{\pgfqpoint{8.851069in}{7.081890in}}%
\pgfusepath{clip}%
\pgfsetbuttcap%
\pgfsetroundjoin%
\definecolor{currentfill}{rgb}{0.241237,0.296485,0.539709}%
\pgfsetfillcolor{currentfill}%
\pgfsetfillopacity{0.700000}%
\pgfsetlinewidth{0.501875pt}%
\definecolor{currentstroke}{rgb}{1.000000,1.000000,1.000000}%
\pgfsetstrokecolor{currentstroke}%
\pgfsetstrokeopacity{0.700000}%
\pgfsetdash{}{0pt}%
\pgfpathmoveto{\pgfqpoint{3.353074in}{3.734393in}}%
\pgfpathcurveto{\pgfqpoint{3.366097in}{3.734393in}}{\pgfqpoint{3.378588in}{3.739567in}}{\pgfqpoint{3.387797in}{3.748775in}}%
\pgfpathcurveto{\pgfqpoint{3.397005in}{3.757984in}}{\pgfqpoint{3.402179in}{3.770475in}}{\pgfqpoint{3.402179in}{3.783498in}}%
\pgfpathcurveto{\pgfqpoint{3.402179in}{3.796520in}}{\pgfqpoint{3.397005in}{3.809011in}}{\pgfqpoint{3.387797in}{3.818220in}}%
\pgfpathcurveto{\pgfqpoint{3.378588in}{3.827428in}}{\pgfqpoint{3.366097in}{3.832602in}}{\pgfqpoint{3.353074in}{3.832602in}}%
\pgfpathcurveto{\pgfqpoint{3.340052in}{3.832602in}}{\pgfqpoint{3.327561in}{3.827428in}}{\pgfqpoint{3.318352in}{3.818220in}}%
\pgfpathcurveto{\pgfqpoint{3.309144in}{3.809011in}}{\pgfqpoint{3.303970in}{3.796520in}}{\pgfqpoint{3.303970in}{3.783498in}}%
\pgfpathcurveto{\pgfqpoint{3.303970in}{3.770475in}}{\pgfqpoint{3.309144in}{3.757984in}}{\pgfqpoint{3.318352in}{3.748775in}}%
\pgfpathcurveto{\pgfqpoint{3.327561in}{3.739567in}}{\pgfqpoint{3.340052in}{3.734393in}}{\pgfqpoint{3.353074in}{3.734393in}}%
\pgfpathlineto{\pgfqpoint{3.353074in}{3.734393in}}%
\pgfpathclose%
\pgfusepath{stroke,fill}%
\end{pgfscope}%
\begin{pgfscope}%
\pgfpathrectangle{\pgfqpoint{0.786164in}{0.768110in}}{\pgfqpoint{8.851069in}{7.081890in}}%
\pgfusepath{clip}%
\pgfsetbuttcap%
\pgfsetroundjoin%
\definecolor{currentfill}{rgb}{0.233603,0.313828,0.543914}%
\pgfsetfillcolor{currentfill}%
\pgfsetfillopacity{0.700000}%
\pgfsetlinewidth{0.501875pt}%
\definecolor{currentstroke}{rgb}{1.000000,1.000000,1.000000}%
\pgfsetstrokecolor{currentstroke}%
\pgfsetstrokeopacity{0.700000}%
\pgfsetdash{}{0pt}%
\pgfpathmoveto{\pgfqpoint{3.270875in}{3.734393in}}%
\pgfpathcurveto{\pgfqpoint{3.283897in}{3.734393in}}{\pgfqpoint{3.296389in}{3.739567in}}{\pgfqpoint{3.305597in}{3.748775in}}%
\pgfpathcurveto{\pgfqpoint{3.314805in}{3.757984in}}{\pgfqpoint{3.319979in}{3.770475in}}{\pgfqpoint{3.319979in}{3.783498in}}%
\pgfpathcurveto{\pgfqpoint{3.319979in}{3.796520in}}{\pgfqpoint{3.314805in}{3.809011in}}{\pgfqpoint{3.305597in}{3.818220in}}%
\pgfpathcurveto{\pgfqpoint{3.296389in}{3.827428in}}{\pgfqpoint{3.283897in}{3.832602in}}{\pgfqpoint{3.270875in}{3.832602in}}%
\pgfpathcurveto{\pgfqpoint{3.257852in}{3.832602in}}{\pgfqpoint{3.245361in}{3.827428in}}{\pgfqpoint{3.236153in}{3.818220in}}%
\pgfpathcurveto{\pgfqpoint{3.226944in}{3.809011in}}{\pgfqpoint{3.221770in}{3.796520in}}{\pgfqpoint{3.221770in}{3.783498in}}%
\pgfpathcurveto{\pgfqpoint{3.221770in}{3.770475in}}{\pgfqpoint{3.226944in}{3.757984in}}{\pgfqpoint{3.236153in}{3.748775in}}%
\pgfpathcurveto{\pgfqpoint{3.245361in}{3.739567in}}{\pgfqpoint{3.257852in}{3.734393in}}{\pgfqpoint{3.270875in}{3.734393in}}%
\pgfpathlineto{\pgfqpoint{3.270875in}{3.734393in}}%
\pgfpathclose%
\pgfusepath{stroke,fill}%
\end{pgfscope}%
\begin{pgfscope}%
\pgfpathrectangle{\pgfqpoint{0.786164in}{0.768110in}}{\pgfqpoint{8.851069in}{7.081890in}}%
\pgfusepath{clip}%
\pgfsetbuttcap%
\pgfsetroundjoin%
\definecolor{currentfill}{rgb}{0.225863,0.330805,0.547314}%
\pgfsetfillcolor{currentfill}%
\pgfsetfillopacity{0.700000}%
\pgfsetlinewidth{0.501875pt}%
\definecolor{currentstroke}{rgb}{1.000000,1.000000,1.000000}%
\pgfsetstrokecolor{currentstroke}%
\pgfsetstrokeopacity{0.700000}%
\pgfsetdash{}{0pt}%
\pgfpathmoveto{\pgfqpoint{3.079076in}{3.559207in}}%
\pgfpathcurveto{\pgfqpoint{3.092098in}{3.559207in}}{\pgfqpoint{3.104589in}{3.564381in}}{\pgfqpoint{3.113798in}{3.573589in}}%
\pgfpathcurveto{\pgfqpoint{3.123006in}{3.582798in}}{\pgfqpoint{3.128180in}{3.595289in}}{\pgfqpoint{3.128180in}{3.608312in}}%
\pgfpathcurveto{\pgfqpoint{3.128180in}{3.621334in}}{\pgfqpoint{3.123006in}{3.633825in}}{\pgfqpoint{3.113798in}{3.643034in}}%
\pgfpathcurveto{\pgfqpoint{3.104589in}{3.652242in}}{\pgfqpoint{3.092098in}{3.657416in}}{\pgfqpoint{3.079076in}{3.657416in}}%
\pgfpathcurveto{\pgfqpoint{3.066053in}{3.657416in}}{\pgfqpoint{3.053562in}{3.652242in}}{\pgfqpoint{3.044353in}{3.643034in}}%
\pgfpathcurveto{\pgfqpoint{3.035145in}{3.633825in}}{\pgfqpoint{3.029971in}{3.621334in}}{\pgfqpoint{3.029971in}{3.608312in}}%
\pgfpathcurveto{\pgfqpoint{3.029971in}{3.595289in}}{\pgfqpoint{3.035145in}{3.582798in}}{\pgfqpoint{3.044353in}{3.573589in}}%
\pgfpathcurveto{\pgfqpoint{3.053562in}{3.564381in}}{\pgfqpoint{3.066053in}{3.559207in}}{\pgfqpoint{3.079076in}{3.559207in}}%
\pgfpathlineto{\pgfqpoint{3.079076in}{3.559207in}}%
\pgfpathclose%
\pgfusepath{stroke,fill}%
\end{pgfscope}%
\begin{pgfscope}%
\pgfpathrectangle{\pgfqpoint{0.786164in}{0.768110in}}{\pgfqpoint{8.851069in}{7.081890in}}%
\pgfusepath{clip}%
\pgfsetbuttcap%
\pgfsetroundjoin%
\definecolor{currentfill}{rgb}{0.221989,0.339161,0.548752}%
\pgfsetfillcolor{currentfill}%
\pgfsetfillopacity{0.700000}%
\pgfsetlinewidth{0.501875pt}%
\definecolor{currentstroke}{rgb}{1.000000,1.000000,1.000000}%
\pgfsetstrokecolor{currentstroke}%
\pgfsetstrokeopacity{0.700000}%
\pgfsetdash{}{0pt}%
\pgfpathmoveto{\pgfqpoint{3.307408in}{3.646800in}}%
\pgfpathcurveto{\pgfqpoint{3.320431in}{3.646800in}}{\pgfqpoint{3.332922in}{3.651974in}}{\pgfqpoint{3.342130in}{3.661182in}}%
\pgfpathcurveto{\pgfqpoint{3.351339in}{3.670391in}}{\pgfqpoint{3.356513in}{3.682882in}}{\pgfqpoint{3.356513in}{3.695905in}}%
\pgfpathcurveto{\pgfqpoint{3.356513in}{3.708927in}}{\pgfqpoint{3.351339in}{3.721418in}}{\pgfqpoint{3.342130in}{3.730627in}}%
\pgfpathcurveto{\pgfqpoint{3.332922in}{3.739835in}}{\pgfqpoint{3.320431in}{3.745009in}}{\pgfqpoint{3.307408in}{3.745009in}}%
\pgfpathcurveto{\pgfqpoint{3.294385in}{3.745009in}}{\pgfqpoint{3.281894in}{3.739835in}}{\pgfqpoint{3.272686in}{3.730627in}}%
\pgfpathcurveto{\pgfqpoint{3.263477in}{3.721418in}}{\pgfqpoint{3.258303in}{3.708927in}}{\pgfqpoint{3.258303in}{3.695905in}}%
\pgfpathcurveto{\pgfqpoint{3.258303in}{3.682882in}}{\pgfqpoint{3.263477in}{3.670391in}}{\pgfqpoint{3.272686in}{3.661182in}}%
\pgfpathcurveto{\pgfqpoint{3.281894in}{3.651974in}}{\pgfqpoint{3.294385in}{3.646800in}}{\pgfqpoint{3.307408in}{3.646800in}}%
\pgfpathlineto{\pgfqpoint{3.307408in}{3.646800in}}%
\pgfpathclose%
\pgfusepath{stroke,fill}%
\end{pgfscope}%
\begin{pgfscope}%
\pgfpathrectangle{\pgfqpoint{0.786164in}{0.768110in}}{\pgfqpoint{8.851069in}{7.081890in}}%
\pgfusepath{clip}%
\pgfsetbuttcap%
\pgfsetroundjoin%
\definecolor{currentfill}{rgb}{0.218130,0.347432,0.550038}%
\pgfsetfillcolor{currentfill}%
\pgfsetfillopacity{0.700000}%
\pgfsetlinewidth{0.501875pt}%
\definecolor{currentstroke}{rgb}{1.000000,1.000000,1.000000}%
\pgfsetstrokecolor{currentstroke}%
\pgfsetstrokeopacity{0.700000}%
\pgfsetdash{}{0pt}%
\pgfpathmoveto{\pgfqpoint{3.179542in}{3.515411in}}%
\pgfpathcurveto{\pgfqpoint{3.192565in}{3.515411in}}{\pgfqpoint{3.205056in}{3.520585in}}{\pgfqpoint{3.214264in}{3.529793in}}%
\pgfpathcurveto{\pgfqpoint{3.223473in}{3.539001in}}{\pgfqpoint{3.228647in}{3.551492in}}{\pgfqpoint{3.228647in}{3.564515in}}%
\pgfpathcurveto{\pgfqpoint{3.228647in}{3.577538in}}{\pgfqpoint{3.223473in}{3.590029in}}{\pgfqpoint{3.214264in}{3.599237in}}%
\pgfpathcurveto{\pgfqpoint{3.205056in}{3.608446in}}{\pgfqpoint{3.192565in}{3.613620in}}{\pgfqpoint{3.179542in}{3.613620in}}%
\pgfpathcurveto{\pgfqpoint{3.166519in}{3.613620in}}{\pgfqpoint{3.154028in}{3.608446in}}{\pgfqpoint{3.144820in}{3.599237in}}%
\pgfpathcurveto{\pgfqpoint{3.135611in}{3.590029in}}{\pgfqpoint{3.130437in}{3.577538in}}{\pgfqpoint{3.130437in}{3.564515in}}%
\pgfpathcurveto{\pgfqpoint{3.130437in}{3.551492in}}{\pgfqpoint{3.135611in}{3.539001in}}{\pgfqpoint{3.144820in}{3.529793in}}%
\pgfpathcurveto{\pgfqpoint{3.154028in}{3.520585in}}{\pgfqpoint{3.166519in}{3.515411in}}{\pgfqpoint{3.179542in}{3.515411in}}%
\pgfpathlineto{\pgfqpoint{3.179542in}{3.515411in}}%
\pgfpathclose%
\pgfusepath{stroke,fill}%
\end{pgfscope}%
\begin{pgfscope}%
\pgfpathrectangle{\pgfqpoint{0.786164in}{0.768110in}}{\pgfqpoint{8.851069in}{7.081890in}}%
\pgfusepath{clip}%
\pgfsetbuttcap%
\pgfsetroundjoin%
\definecolor{currentfill}{rgb}{0.283072,0.130895,0.449241}%
\pgfsetfillcolor{currentfill}%
\pgfsetfillopacity{0.700000}%
\pgfsetlinewidth{0.501875pt}%
\definecolor{currentstroke}{rgb}{1.000000,1.000000,1.000000}%
\pgfsetstrokecolor{currentstroke}%
\pgfsetstrokeopacity{0.700000}%
\pgfsetdash{}{0pt}%
\pgfpathmoveto{\pgfqpoint{3.453540in}{3.756291in}}%
\pgfpathcurveto{\pgfqpoint{3.466563in}{3.756291in}}{\pgfqpoint{3.479054in}{3.761465in}}{\pgfqpoint{3.488263in}{3.770674in}}%
\pgfpathcurveto{\pgfqpoint{3.497471in}{3.779882in}}{\pgfqpoint{3.502645in}{3.792373in}}{\pgfqpoint{3.502645in}{3.805396in}}%
\pgfpathcurveto{\pgfqpoint{3.502645in}{3.818418in}}{\pgfqpoint{3.497471in}{3.830910in}}{\pgfqpoint{3.488263in}{3.840118in}}%
\pgfpathcurveto{\pgfqpoint{3.479054in}{3.849326in}}{\pgfqpoint{3.466563in}{3.854500in}}{\pgfqpoint{3.453540in}{3.854500in}}%
\pgfpathcurveto{\pgfqpoint{3.440518in}{3.854500in}}{\pgfqpoint{3.428027in}{3.849326in}}{\pgfqpoint{3.418818in}{3.840118in}}%
\pgfpathcurveto{\pgfqpoint{3.409610in}{3.830910in}}{\pgfqpoint{3.404436in}{3.818418in}}{\pgfqpoint{3.404436in}{3.805396in}}%
\pgfpathcurveto{\pgfqpoint{3.404436in}{3.792373in}}{\pgfqpoint{3.409610in}{3.779882in}}{\pgfqpoint{3.418818in}{3.770674in}}%
\pgfpathcurveto{\pgfqpoint{3.428027in}{3.761465in}}{\pgfqpoint{3.440518in}{3.756291in}}{\pgfqpoint{3.453540in}{3.756291in}}%
\pgfpathlineto{\pgfqpoint{3.453540in}{3.756291in}}%
\pgfpathclose%
\pgfusepath{stroke,fill}%
\end{pgfscope}%
\begin{pgfscope}%
\pgfpathrectangle{\pgfqpoint{0.786164in}{0.768110in}}{\pgfqpoint{8.851069in}{7.081890in}}%
\pgfusepath{clip}%
\pgfsetbuttcap%
\pgfsetroundjoin%
\definecolor{currentfill}{rgb}{0.281887,0.150881,0.465405}%
\pgfsetfillcolor{currentfill}%
\pgfsetfillopacity{0.700000}%
\pgfsetlinewidth{0.501875pt}%
\definecolor{currentstroke}{rgb}{1.000000,1.000000,1.000000}%
\pgfsetstrokecolor{currentstroke}%
\pgfsetstrokeopacity{0.700000}%
\pgfsetdash{}{0pt}%
\pgfpathmoveto{\pgfqpoint{3.435274in}{3.734393in}}%
\pgfpathcurveto{\pgfqpoint{3.448297in}{3.734393in}}{\pgfqpoint{3.460788in}{3.739567in}}{\pgfqpoint{3.469996in}{3.748775in}}%
\pgfpathcurveto{\pgfqpoint{3.479205in}{3.757984in}}{\pgfqpoint{3.484379in}{3.770475in}}{\pgfqpoint{3.484379in}{3.783498in}}%
\pgfpathcurveto{\pgfqpoint{3.484379in}{3.796520in}}{\pgfqpoint{3.479205in}{3.809011in}}{\pgfqpoint{3.469996in}{3.818220in}}%
\pgfpathcurveto{\pgfqpoint{3.460788in}{3.827428in}}{\pgfqpoint{3.448297in}{3.832602in}}{\pgfqpoint{3.435274in}{3.832602in}}%
\pgfpathcurveto{\pgfqpoint{3.422251in}{3.832602in}}{\pgfqpoint{3.409760in}{3.827428in}}{\pgfqpoint{3.400552in}{3.818220in}}%
\pgfpathcurveto{\pgfqpoint{3.391343in}{3.809011in}}{\pgfqpoint{3.386169in}{3.796520in}}{\pgfqpoint{3.386169in}{3.783498in}}%
\pgfpathcurveto{\pgfqpoint{3.386169in}{3.770475in}}{\pgfqpoint{3.391343in}{3.757984in}}{\pgfqpoint{3.400552in}{3.748775in}}%
\pgfpathcurveto{\pgfqpoint{3.409760in}{3.739567in}}{\pgfqpoint{3.422251in}{3.734393in}}{\pgfqpoint{3.435274in}{3.734393in}}%
\pgfpathlineto{\pgfqpoint{3.435274in}{3.734393in}}%
\pgfpathclose%
\pgfusepath{stroke,fill}%
\end{pgfscope}%
\begin{pgfscope}%
\pgfpathrectangle{\pgfqpoint{0.786164in}{0.768110in}}{\pgfqpoint{8.851069in}{7.081890in}}%
\pgfusepath{clip}%
\pgfsetbuttcap%
\pgfsetroundjoin%
\definecolor{currentfill}{rgb}{0.278826,0.175490,0.483397}%
\pgfsetfillcolor{currentfill}%
\pgfsetfillopacity{0.700000}%
\pgfsetlinewidth{0.501875pt}%
\definecolor{currentstroke}{rgb}{1.000000,1.000000,1.000000}%
\pgfsetstrokecolor{currentstroke}%
\pgfsetstrokeopacity{0.700000}%
\pgfsetdash{}{0pt}%
\pgfpathmoveto{\pgfqpoint{3.563140in}{3.756291in}}%
\pgfpathcurveto{\pgfqpoint{3.576163in}{3.756291in}}{\pgfqpoint{3.588654in}{3.761465in}}{\pgfqpoint{3.597862in}{3.770674in}}%
\pgfpathcurveto{\pgfqpoint{3.607071in}{3.779882in}}{\pgfqpoint{3.612245in}{3.792373in}}{\pgfqpoint{3.612245in}{3.805396in}}%
\pgfpathcurveto{\pgfqpoint{3.612245in}{3.818418in}}{\pgfqpoint{3.607071in}{3.830910in}}{\pgfqpoint{3.597862in}{3.840118in}}%
\pgfpathcurveto{\pgfqpoint{3.588654in}{3.849326in}}{\pgfqpoint{3.576163in}{3.854500in}}{\pgfqpoint{3.563140in}{3.854500in}}%
\pgfpathcurveto{\pgfqpoint{3.550117in}{3.854500in}}{\pgfqpoint{3.537626in}{3.849326in}}{\pgfqpoint{3.528418in}{3.840118in}}%
\pgfpathcurveto{\pgfqpoint{3.519209in}{3.830910in}}{\pgfqpoint{3.514035in}{3.818418in}}{\pgfqpoint{3.514035in}{3.805396in}}%
\pgfpathcurveto{\pgfqpoint{3.514035in}{3.792373in}}{\pgfqpoint{3.519209in}{3.779882in}}{\pgfqpoint{3.528418in}{3.770674in}}%
\pgfpathcurveto{\pgfqpoint{3.537626in}{3.761465in}}{\pgfqpoint{3.550117in}{3.756291in}}{\pgfqpoint{3.563140in}{3.756291in}}%
\pgfpathlineto{\pgfqpoint{3.563140in}{3.756291in}}%
\pgfpathclose%
\pgfusepath{stroke,fill}%
\end{pgfscope}%
\begin{pgfscope}%
\pgfpathrectangle{\pgfqpoint{0.786164in}{0.768110in}}{\pgfqpoint{8.851069in}{7.081890in}}%
\pgfusepath{clip}%
\pgfsetbuttcap%
\pgfsetroundjoin%
\definecolor{currentfill}{rgb}{0.273006,0.204520,0.501721}%
\pgfsetfillcolor{currentfill}%
\pgfsetfillopacity{0.700000}%
\pgfsetlinewidth{0.501875pt}%
\definecolor{currentstroke}{rgb}{1.000000,1.000000,1.000000}%
\pgfsetstrokecolor{currentstroke}%
\pgfsetstrokeopacity{0.700000}%
\pgfsetdash{}{0pt}%
\pgfpathmoveto{\pgfqpoint{3.380474in}{3.668698in}}%
\pgfpathcurveto{\pgfqpoint{3.393497in}{3.668698in}}{\pgfqpoint{3.405988in}{3.673872in}}{\pgfqpoint{3.415196in}{3.683081in}}%
\pgfpathcurveto{\pgfqpoint{3.424405in}{3.692289in}}{\pgfqpoint{3.429579in}{3.704780in}}{\pgfqpoint{3.429579in}{3.717803in}}%
\pgfpathcurveto{\pgfqpoint{3.429579in}{3.730826in}}{\pgfqpoint{3.424405in}{3.743317in}}{\pgfqpoint{3.415196in}{3.752525in}}%
\pgfpathcurveto{\pgfqpoint{3.405988in}{3.761733in}}{\pgfqpoint{3.393497in}{3.766907in}}{\pgfqpoint{3.380474in}{3.766907in}}%
\pgfpathcurveto{\pgfqpoint{3.367451in}{3.766907in}}{\pgfqpoint{3.354960in}{3.761733in}}{\pgfqpoint{3.345752in}{3.752525in}}%
\pgfpathcurveto{\pgfqpoint{3.336544in}{3.743317in}}{\pgfqpoint{3.331370in}{3.730826in}}{\pgfqpoint{3.331370in}{3.717803in}}%
\pgfpathcurveto{\pgfqpoint{3.331370in}{3.704780in}}{\pgfqpoint{3.336544in}{3.692289in}}{\pgfqpoint{3.345752in}{3.683081in}}%
\pgfpathcurveto{\pgfqpoint{3.354960in}{3.673872in}}{\pgfqpoint{3.367451in}{3.668698in}}{\pgfqpoint{3.380474in}{3.668698in}}%
\pgfpathlineto{\pgfqpoint{3.380474in}{3.668698in}}%
\pgfpathclose%
\pgfusepath{stroke,fill}%
\end{pgfscope}%
\begin{pgfscope}%
\pgfpathrectangle{\pgfqpoint{0.786164in}{0.768110in}}{\pgfqpoint{8.851069in}{7.081890in}}%
\pgfusepath{clip}%
\pgfsetbuttcap%
\pgfsetroundjoin%
\definecolor{currentfill}{rgb}{0.273006,0.204520,0.501721}%
\pgfsetfillcolor{currentfill}%
\pgfsetfillopacity{0.700000}%
\pgfsetlinewidth{0.501875pt}%
\definecolor{currentstroke}{rgb}{1.000000,1.000000,1.000000}%
\pgfsetstrokecolor{currentstroke}%
\pgfsetstrokeopacity{0.700000}%
\pgfsetdash{}{0pt}%
\pgfpathmoveto{\pgfqpoint{3.444407in}{3.646800in}}%
\pgfpathcurveto{\pgfqpoint{3.457430in}{3.646800in}}{\pgfqpoint{3.469921in}{3.651974in}}{\pgfqpoint{3.479129in}{3.661182in}}%
\pgfpathcurveto{\pgfqpoint{3.488338in}{3.670391in}}{\pgfqpoint{3.493512in}{3.682882in}}{\pgfqpoint{3.493512in}{3.695905in}}%
\pgfpathcurveto{\pgfqpoint{3.493512in}{3.708927in}}{\pgfqpoint{3.488338in}{3.721418in}}{\pgfqpoint{3.479129in}{3.730627in}}%
\pgfpathcurveto{\pgfqpoint{3.469921in}{3.739835in}}{\pgfqpoint{3.457430in}{3.745009in}}{\pgfqpoint{3.444407in}{3.745009in}}%
\pgfpathcurveto{\pgfqpoint{3.431385in}{3.745009in}}{\pgfqpoint{3.418893in}{3.739835in}}{\pgfqpoint{3.409685in}{3.730627in}}%
\pgfpathcurveto{\pgfqpoint{3.400477in}{3.721418in}}{\pgfqpoint{3.395303in}{3.708927in}}{\pgfqpoint{3.395303in}{3.695905in}}%
\pgfpathcurveto{\pgfqpoint{3.395303in}{3.682882in}}{\pgfqpoint{3.400477in}{3.670391in}}{\pgfqpoint{3.409685in}{3.661182in}}%
\pgfpathcurveto{\pgfqpoint{3.418893in}{3.651974in}}{\pgfqpoint{3.431385in}{3.646800in}}{\pgfqpoint{3.444407in}{3.646800in}}%
\pgfpathlineto{\pgfqpoint{3.444407in}{3.646800in}}%
\pgfpathclose%
\pgfusepath{stroke,fill}%
\end{pgfscope}%
\begin{pgfscope}%
\pgfpathrectangle{\pgfqpoint{0.786164in}{0.768110in}}{\pgfqpoint{8.851069in}{7.081890in}}%
\pgfusepath{clip}%
\pgfsetbuttcap%
\pgfsetroundjoin%
\definecolor{currentfill}{rgb}{0.269308,0.218818,0.509577}%
\pgfsetfillcolor{currentfill}%
\pgfsetfillopacity{0.700000}%
\pgfsetlinewidth{0.501875pt}%
\definecolor{currentstroke}{rgb}{1.000000,1.000000,1.000000}%
\pgfsetstrokecolor{currentstroke}%
\pgfsetstrokeopacity{0.700000}%
\pgfsetdash{}{0pt}%
\pgfpathmoveto{\pgfqpoint{3.216075in}{3.471614in}}%
\pgfpathcurveto{\pgfqpoint{3.229098in}{3.471614in}}{\pgfqpoint{3.241589in}{3.476788in}}{\pgfqpoint{3.250797in}{3.485996in}}%
\pgfpathcurveto{\pgfqpoint{3.260006in}{3.495205in}}{\pgfqpoint{3.265180in}{3.507696in}}{\pgfqpoint{3.265180in}{3.520719in}}%
\pgfpathcurveto{\pgfqpoint{3.265180in}{3.533741in}}{\pgfqpoint{3.260006in}{3.546232in}}{\pgfqpoint{3.250797in}{3.555441in}}%
\pgfpathcurveto{\pgfqpoint{3.241589in}{3.564649in}}{\pgfqpoint{3.229098in}{3.569823in}}{\pgfqpoint{3.216075in}{3.569823in}}%
\pgfpathcurveto{\pgfqpoint{3.203052in}{3.569823in}}{\pgfqpoint{3.190561in}{3.564649in}}{\pgfqpoint{3.181353in}{3.555441in}}%
\pgfpathcurveto{\pgfqpoint{3.172144in}{3.546232in}}{\pgfqpoint{3.166970in}{3.533741in}}{\pgfqpoint{3.166970in}{3.520719in}}%
\pgfpathcurveto{\pgfqpoint{3.166970in}{3.507696in}}{\pgfqpoint{3.172144in}{3.495205in}}{\pgfqpoint{3.181353in}{3.485996in}}%
\pgfpathcurveto{\pgfqpoint{3.190561in}{3.476788in}}{\pgfqpoint{3.203052in}{3.471614in}}{\pgfqpoint{3.216075in}{3.471614in}}%
\pgfpathlineto{\pgfqpoint{3.216075in}{3.471614in}}%
\pgfpathclose%
\pgfusepath{stroke,fill}%
\end{pgfscope}%
\begin{pgfscope}%
\pgfpathrectangle{\pgfqpoint{0.786164in}{0.768110in}}{\pgfqpoint{8.851069in}{7.081890in}}%
\pgfusepath{clip}%
\pgfsetbuttcap%
\pgfsetroundjoin%
\definecolor{currentfill}{rgb}{0.265145,0.232956,0.516599}%
\pgfsetfillcolor{currentfill}%
\pgfsetfillopacity{0.700000}%
\pgfsetlinewidth{0.501875pt}%
\definecolor{currentstroke}{rgb}{1.000000,1.000000,1.000000}%
\pgfsetstrokecolor{currentstroke}%
\pgfsetstrokeopacity{0.700000}%
\pgfsetdash{}{0pt}%
\pgfpathmoveto{\pgfqpoint{3.389607in}{3.559207in}}%
\pgfpathcurveto{\pgfqpoint{3.402630in}{3.559207in}}{\pgfqpoint{3.415121in}{3.564381in}}{\pgfqpoint{3.424330in}{3.573589in}}%
\pgfpathcurveto{\pgfqpoint{3.433538in}{3.582798in}}{\pgfqpoint{3.438712in}{3.595289in}}{\pgfqpoint{3.438712in}{3.608312in}}%
\pgfpathcurveto{\pgfqpoint{3.438712in}{3.621334in}}{\pgfqpoint{3.433538in}{3.633825in}}{\pgfqpoint{3.424330in}{3.643034in}}%
\pgfpathcurveto{\pgfqpoint{3.415121in}{3.652242in}}{\pgfqpoint{3.402630in}{3.657416in}}{\pgfqpoint{3.389607in}{3.657416in}}%
\pgfpathcurveto{\pgfqpoint{3.376585in}{3.657416in}}{\pgfqpoint{3.364094in}{3.652242in}}{\pgfqpoint{3.354885in}{3.643034in}}%
\pgfpathcurveto{\pgfqpoint{3.345677in}{3.633825in}}{\pgfqpoint{3.340503in}{3.621334in}}{\pgfqpoint{3.340503in}{3.608312in}}%
\pgfpathcurveto{\pgfqpoint{3.340503in}{3.595289in}}{\pgfqpoint{3.345677in}{3.582798in}}{\pgfqpoint{3.354885in}{3.573589in}}%
\pgfpathcurveto{\pgfqpoint{3.364094in}{3.564381in}}{\pgfqpoint{3.376585in}{3.559207in}}{\pgfqpoint{3.389607in}{3.559207in}}%
\pgfpathlineto{\pgfqpoint{3.389607in}{3.559207in}}%
\pgfpathclose%
\pgfusepath{stroke,fill}%
\end{pgfscope}%
\begin{pgfscope}%
\pgfpathrectangle{\pgfqpoint{0.786164in}{0.768110in}}{\pgfqpoint{8.851069in}{7.081890in}}%
\pgfusepath{clip}%
\pgfsetbuttcap%
\pgfsetroundjoin%
\definecolor{currentfill}{rgb}{0.258965,0.251537,0.524736}%
\pgfsetfillcolor{currentfill}%
\pgfsetfillopacity{0.700000}%
\pgfsetlinewidth{0.501875pt}%
\definecolor{currentstroke}{rgb}{1.000000,1.000000,1.000000}%
\pgfsetstrokecolor{currentstroke}%
\pgfsetstrokeopacity{0.700000}%
\pgfsetdash{}{0pt}%
\pgfpathmoveto{\pgfqpoint{3.261741in}{3.537309in}}%
\pgfpathcurveto{\pgfqpoint{3.274764in}{3.537309in}}{\pgfqpoint{3.287255in}{3.542483in}}{\pgfqpoint{3.296464in}{3.551691in}}%
\pgfpathcurveto{\pgfqpoint{3.305672in}{3.560900in}}{\pgfqpoint{3.310846in}{3.573391in}}{\pgfqpoint{3.310846in}{3.586413in}}%
\pgfpathcurveto{\pgfqpoint{3.310846in}{3.599436in}}{\pgfqpoint{3.305672in}{3.611927in}}{\pgfqpoint{3.296464in}{3.621136in}}%
\pgfpathcurveto{\pgfqpoint{3.287255in}{3.630344in}}{\pgfqpoint{3.274764in}{3.635518in}}{\pgfqpoint{3.261741in}{3.635518in}}%
\pgfpathcurveto{\pgfqpoint{3.248719in}{3.635518in}}{\pgfqpoint{3.236228in}{3.630344in}}{\pgfqpoint{3.227019in}{3.621136in}}%
\pgfpathcurveto{\pgfqpoint{3.217811in}{3.611927in}}{\pgfqpoint{3.212637in}{3.599436in}}{\pgfqpoint{3.212637in}{3.586413in}}%
\pgfpathcurveto{\pgfqpoint{3.212637in}{3.573391in}}{\pgfqpoint{3.217811in}{3.560900in}}{\pgfqpoint{3.227019in}{3.551691in}}%
\pgfpathcurveto{\pgfqpoint{3.236228in}{3.542483in}}{\pgfqpoint{3.248719in}{3.537309in}}{\pgfqpoint{3.261741in}{3.537309in}}%
\pgfpathlineto{\pgfqpoint{3.261741in}{3.537309in}}%
\pgfpathclose%
\pgfusepath{stroke,fill}%
\end{pgfscope}%
\begin{pgfscope}%
\pgfpathrectangle{\pgfqpoint{0.786164in}{0.768110in}}{\pgfqpoint{8.851069in}{7.081890in}}%
\pgfusepath{clip}%
\pgfsetbuttcap%
\pgfsetroundjoin%
\definecolor{currentfill}{rgb}{0.252194,0.269783,0.531579}%
\pgfsetfillcolor{currentfill}%
\pgfsetfillopacity{0.700000}%
\pgfsetlinewidth{0.501875pt}%
\definecolor{currentstroke}{rgb}{1.000000,1.000000,1.000000}%
\pgfsetstrokecolor{currentstroke}%
\pgfsetstrokeopacity{0.700000}%
\pgfsetdash{}{0pt}%
\pgfpathmoveto{\pgfqpoint{3.289141in}{3.537309in}}%
\pgfpathcurveto{\pgfqpoint{3.302164in}{3.537309in}}{\pgfqpoint{3.314655in}{3.542483in}}{\pgfqpoint{3.323864in}{3.551691in}}%
\pgfpathcurveto{\pgfqpoint{3.333072in}{3.560900in}}{\pgfqpoint{3.338246in}{3.573391in}}{\pgfqpoint{3.338246in}{3.586413in}}%
\pgfpathcurveto{\pgfqpoint{3.338246in}{3.599436in}}{\pgfqpoint{3.333072in}{3.611927in}}{\pgfqpoint{3.323864in}{3.621136in}}%
\pgfpathcurveto{\pgfqpoint{3.314655in}{3.630344in}}{\pgfqpoint{3.302164in}{3.635518in}}{\pgfqpoint{3.289141in}{3.635518in}}%
\pgfpathcurveto{\pgfqpoint{3.276119in}{3.635518in}}{\pgfqpoint{3.263628in}{3.630344in}}{\pgfqpoint{3.254419in}{3.621136in}}%
\pgfpathcurveto{\pgfqpoint{3.245211in}{3.611927in}}{\pgfqpoint{3.240037in}{3.599436in}}{\pgfqpoint{3.240037in}{3.586413in}}%
\pgfpathcurveto{\pgfqpoint{3.240037in}{3.573391in}}{\pgfqpoint{3.245211in}{3.560900in}}{\pgfqpoint{3.254419in}{3.551691in}}%
\pgfpathcurveto{\pgfqpoint{3.263628in}{3.542483in}}{\pgfqpoint{3.276119in}{3.537309in}}{\pgfqpoint{3.289141in}{3.537309in}}%
\pgfpathlineto{\pgfqpoint{3.289141in}{3.537309in}}%
\pgfpathclose%
\pgfusepath{stroke,fill}%
\end{pgfscope}%
\begin{pgfscope}%
\pgfpathrectangle{\pgfqpoint{0.786164in}{0.768110in}}{\pgfqpoint{8.851069in}{7.081890in}}%
\pgfusepath{clip}%
\pgfsetbuttcap%
\pgfsetroundjoin%
\definecolor{currentfill}{rgb}{0.250425,0.274290,0.533103}%
\pgfsetfillcolor{currentfill}%
\pgfsetfillopacity{0.700000}%
\pgfsetlinewidth{0.501875pt}%
\definecolor{currentstroke}{rgb}{1.000000,1.000000,1.000000}%
\pgfsetstrokecolor{currentstroke}%
\pgfsetstrokeopacity{0.700000}%
\pgfsetdash{}{0pt}%
\pgfpathmoveto{\pgfqpoint{3.362208in}{3.581105in}}%
\pgfpathcurveto{\pgfqpoint{3.375230in}{3.581105in}}{\pgfqpoint{3.387721in}{3.586279in}}{\pgfqpoint{3.396930in}{3.595488in}}%
\pgfpathcurveto{\pgfqpoint{3.406138in}{3.604696in}}{\pgfqpoint{3.411312in}{3.617187in}}{\pgfqpoint{3.411312in}{3.630210in}}%
\pgfpathcurveto{\pgfqpoint{3.411312in}{3.643233in}}{\pgfqpoint{3.406138in}{3.655724in}}{\pgfqpoint{3.396930in}{3.664932in}}%
\pgfpathcurveto{\pgfqpoint{3.387721in}{3.674141in}}{\pgfqpoint{3.375230in}{3.679315in}}{\pgfqpoint{3.362208in}{3.679315in}}%
\pgfpathcurveto{\pgfqpoint{3.349185in}{3.679315in}}{\pgfqpoint{3.336694in}{3.674141in}}{\pgfqpoint{3.327485in}{3.664932in}}%
\pgfpathcurveto{\pgfqpoint{3.318277in}{3.655724in}}{\pgfqpoint{3.313103in}{3.643233in}}{\pgfqpoint{3.313103in}{3.630210in}}%
\pgfpathcurveto{\pgfqpoint{3.313103in}{3.617187in}}{\pgfqpoint{3.318277in}{3.604696in}}{\pgfqpoint{3.327485in}{3.595488in}}%
\pgfpathcurveto{\pgfqpoint{3.336694in}{3.586279in}}{\pgfqpoint{3.349185in}{3.581105in}}{\pgfqpoint{3.362208in}{3.581105in}}%
\pgfpathlineto{\pgfqpoint{3.362208in}{3.581105in}}%
\pgfpathclose%
\pgfusepath{stroke,fill}%
\end{pgfscope}%
\begin{pgfscope}%
\pgfpathrectangle{\pgfqpoint{0.786164in}{0.768110in}}{\pgfqpoint{8.851069in}{7.081890in}}%
\pgfusepath{clip}%
\pgfsetbuttcap%
\pgfsetroundjoin%
\definecolor{currentfill}{rgb}{0.246811,0.283237,0.535941}%
\pgfsetfillcolor{currentfill}%
\pgfsetfillopacity{0.700000}%
\pgfsetlinewidth{0.501875pt}%
\definecolor{currentstroke}{rgb}{1.000000,1.000000,1.000000}%
\pgfsetstrokecolor{currentstroke}%
\pgfsetstrokeopacity{0.700000}%
\pgfsetdash{}{0pt}%
\pgfpathmoveto{\pgfqpoint{3.188675in}{3.471614in}}%
\pgfpathcurveto{\pgfqpoint{3.201698in}{3.471614in}}{\pgfqpoint{3.214189in}{3.476788in}}{\pgfqpoint{3.223397in}{3.485996in}}%
\pgfpathcurveto{\pgfqpoint{3.232606in}{3.495205in}}{\pgfqpoint{3.237780in}{3.507696in}}{\pgfqpoint{3.237780in}{3.520719in}}%
\pgfpathcurveto{\pgfqpoint{3.237780in}{3.533741in}}{\pgfqpoint{3.232606in}{3.546232in}}{\pgfqpoint{3.223397in}{3.555441in}}%
\pgfpathcurveto{\pgfqpoint{3.214189in}{3.564649in}}{\pgfqpoint{3.201698in}{3.569823in}}{\pgfqpoint{3.188675in}{3.569823in}}%
\pgfpathcurveto{\pgfqpoint{3.175652in}{3.569823in}}{\pgfqpoint{3.163161in}{3.564649in}}{\pgfqpoint{3.153953in}{3.555441in}}%
\pgfpathcurveto{\pgfqpoint{3.144744in}{3.546232in}}{\pgfqpoint{3.139571in}{3.533741in}}{\pgfqpoint{3.139571in}{3.520719in}}%
\pgfpathcurveto{\pgfqpoint{3.139571in}{3.507696in}}{\pgfqpoint{3.144744in}{3.495205in}}{\pgfqpoint{3.153953in}{3.485996in}}%
\pgfpathcurveto{\pgfqpoint{3.163161in}{3.476788in}}{\pgfqpoint{3.175652in}{3.471614in}}{\pgfqpoint{3.188675in}{3.471614in}}%
\pgfpathlineto{\pgfqpoint{3.188675in}{3.471614in}}%
\pgfpathclose%
\pgfusepath{stroke,fill}%
\end{pgfscope}%
\begin{pgfscope}%
\pgfpathrectangle{\pgfqpoint{0.786164in}{0.768110in}}{\pgfqpoint{8.851069in}{7.081890in}}%
\pgfusepath{clip}%
\pgfsetbuttcap%
\pgfsetroundjoin%
\definecolor{currentfill}{rgb}{0.243113,0.292092,0.538516}%
\pgfsetfillcolor{currentfill}%
\pgfsetfillopacity{0.700000}%
\pgfsetlinewidth{0.501875pt}%
\definecolor{currentstroke}{rgb}{1.000000,1.000000,1.000000}%
\pgfsetstrokecolor{currentstroke}%
\pgfsetstrokeopacity{0.700000}%
\pgfsetdash{}{0pt}%
\pgfpathmoveto{\pgfqpoint{3.179542in}{3.471614in}}%
\pgfpathcurveto{\pgfqpoint{3.192565in}{3.471614in}}{\pgfqpoint{3.205056in}{3.476788in}}{\pgfqpoint{3.214264in}{3.485996in}}%
\pgfpathcurveto{\pgfqpoint{3.223473in}{3.495205in}}{\pgfqpoint{3.228647in}{3.507696in}}{\pgfqpoint{3.228647in}{3.520719in}}%
\pgfpathcurveto{\pgfqpoint{3.228647in}{3.533741in}}{\pgfqpoint{3.223473in}{3.546232in}}{\pgfqpoint{3.214264in}{3.555441in}}%
\pgfpathcurveto{\pgfqpoint{3.205056in}{3.564649in}}{\pgfqpoint{3.192565in}{3.569823in}}{\pgfqpoint{3.179542in}{3.569823in}}%
\pgfpathcurveto{\pgfqpoint{3.166519in}{3.569823in}}{\pgfqpoint{3.154028in}{3.564649in}}{\pgfqpoint{3.144820in}{3.555441in}}%
\pgfpathcurveto{\pgfqpoint{3.135611in}{3.546232in}}{\pgfqpoint{3.130437in}{3.533741in}}{\pgfqpoint{3.130437in}{3.520719in}}%
\pgfpathcurveto{\pgfqpoint{3.130437in}{3.507696in}}{\pgfqpoint{3.135611in}{3.495205in}}{\pgfqpoint{3.144820in}{3.485996in}}%
\pgfpathcurveto{\pgfqpoint{3.154028in}{3.476788in}}{\pgfqpoint{3.166519in}{3.471614in}}{\pgfqpoint{3.179542in}{3.471614in}}%
\pgfpathlineto{\pgfqpoint{3.179542in}{3.471614in}}%
\pgfpathclose%
\pgfusepath{stroke,fill}%
\end{pgfscope}%
\begin{pgfscope}%
\pgfpathrectangle{\pgfqpoint{0.786164in}{0.768110in}}{\pgfqpoint{8.851069in}{7.081890in}}%
\pgfusepath{clip}%
\pgfsetbuttcap%
\pgfsetroundjoin%
\definecolor{currentfill}{rgb}{0.243113,0.292092,0.538516}%
\pgfsetfillcolor{currentfill}%
\pgfsetfillopacity{0.700000}%
\pgfsetlinewidth{0.501875pt}%
\definecolor{currentstroke}{rgb}{1.000000,1.000000,1.000000}%
\pgfsetstrokecolor{currentstroke}%
\pgfsetstrokeopacity{0.700000}%
\pgfsetdash{}{0pt}%
\pgfpathmoveto{\pgfqpoint{3.170409in}{3.449716in}}%
\pgfpathcurveto{\pgfqpoint{3.183431in}{3.449716in}}{\pgfqpoint{3.195922in}{3.454890in}}{\pgfqpoint{3.205131in}{3.464098in}}%
\pgfpathcurveto{\pgfqpoint{3.214339in}{3.473307in}}{\pgfqpoint{3.219513in}{3.485798in}}{\pgfqpoint{3.219513in}{3.498820in}}%
\pgfpathcurveto{\pgfqpoint{3.219513in}{3.511843in}}{\pgfqpoint{3.214339in}{3.524334in}}{\pgfqpoint{3.205131in}{3.533543in}}%
\pgfpathcurveto{\pgfqpoint{3.195922in}{3.542751in}}{\pgfqpoint{3.183431in}{3.547925in}}{\pgfqpoint{3.170409in}{3.547925in}}%
\pgfpathcurveto{\pgfqpoint{3.157386in}{3.547925in}}{\pgfqpoint{3.144895in}{3.542751in}}{\pgfqpoint{3.135686in}{3.533543in}}%
\pgfpathcurveto{\pgfqpoint{3.126478in}{3.524334in}}{\pgfqpoint{3.121304in}{3.511843in}}{\pgfqpoint{3.121304in}{3.498820in}}%
\pgfpathcurveto{\pgfqpoint{3.121304in}{3.485798in}}{\pgfqpoint{3.126478in}{3.473307in}}{\pgfqpoint{3.135686in}{3.464098in}}%
\pgfpathcurveto{\pgfqpoint{3.144895in}{3.454890in}}{\pgfqpoint{3.157386in}{3.449716in}}{\pgfqpoint{3.170409in}{3.449716in}}%
\pgfpathlineto{\pgfqpoint{3.170409in}{3.449716in}}%
\pgfpathclose%
\pgfusepath{stroke,fill}%
\end{pgfscope}%
\begin{pgfscope}%
\pgfpathrectangle{\pgfqpoint{0.786164in}{0.768110in}}{\pgfqpoint{8.851069in}{7.081890in}}%
\pgfusepath{clip}%
\pgfsetbuttcap%
\pgfsetroundjoin%
\definecolor{currentfill}{rgb}{0.239346,0.300855,0.540844}%
\pgfsetfillcolor{currentfill}%
\pgfsetfillopacity{0.700000}%
\pgfsetlinewidth{0.501875pt}%
\definecolor{currentstroke}{rgb}{1.000000,1.000000,1.000000}%
\pgfsetstrokecolor{currentstroke}%
\pgfsetstrokeopacity{0.700000}%
\pgfsetdash{}{0pt}%
\pgfpathmoveto{\pgfqpoint{3.170409in}{3.405919in}}%
\pgfpathcurveto{\pgfqpoint{3.183431in}{3.405919in}}{\pgfqpoint{3.195922in}{3.411093in}}{\pgfqpoint{3.205131in}{3.420302in}}%
\pgfpathcurveto{\pgfqpoint{3.214339in}{3.429510in}}{\pgfqpoint{3.219513in}{3.442001in}}{\pgfqpoint{3.219513in}{3.455024in}}%
\pgfpathcurveto{\pgfqpoint{3.219513in}{3.468047in}}{\pgfqpoint{3.214339in}{3.480538in}}{\pgfqpoint{3.205131in}{3.489746in}}%
\pgfpathcurveto{\pgfqpoint{3.195922in}{3.498955in}}{\pgfqpoint{3.183431in}{3.504129in}}{\pgfqpoint{3.170409in}{3.504129in}}%
\pgfpathcurveto{\pgfqpoint{3.157386in}{3.504129in}}{\pgfqpoint{3.144895in}{3.498955in}}{\pgfqpoint{3.135686in}{3.489746in}}%
\pgfpathcurveto{\pgfqpoint{3.126478in}{3.480538in}}{\pgfqpoint{3.121304in}{3.468047in}}{\pgfqpoint{3.121304in}{3.455024in}}%
\pgfpathcurveto{\pgfqpoint{3.121304in}{3.442001in}}{\pgfqpoint{3.126478in}{3.429510in}}{\pgfqpoint{3.135686in}{3.420302in}}%
\pgfpathcurveto{\pgfqpoint{3.144895in}{3.411093in}}{\pgfqpoint{3.157386in}{3.405919in}}{\pgfqpoint{3.170409in}{3.405919in}}%
\pgfpathlineto{\pgfqpoint{3.170409in}{3.405919in}}%
\pgfpathclose%
\pgfusepath{stroke,fill}%
\end{pgfscope}%
\begin{pgfscope}%
\pgfpathrectangle{\pgfqpoint{0.786164in}{0.768110in}}{\pgfqpoint{8.851069in}{7.081890in}}%
\pgfusepath{clip}%
\pgfsetbuttcap%
\pgfsetroundjoin%
\definecolor{currentfill}{rgb}{0.229739,0.322361,0.545706}%
\pgfsetfillcolor{currentfill}%
\pgfsetfillopacity{0.700000}%
\pgfsetlinewidth{0.501875pt}%
\definecolor{currentstroke}{rgb}{1.000000,1.000000,1.000000}%
\pgfsetstrokecolor{currentstroke}%
\pgfsetstrokeopacity{0.700000}%
\pgfsetdash{}{0pt}%
\pgfpathmoveto{\pgfqpoint{3.088209in}{3.340225in}}%
\pgfpathcurveto{\pgfqpoint{3.101232in}{3.340225in}}{\pgfqpoint{3.113723in}{3.345399in}}{\pgfqpoint{3.122931in}{3.354607in}}%
\pgfpathcurveto{\pgfqpoint{3.132140in}{3.363816in}}{\pgfqpoint{3.137314in}{3.376307in}}{\pgfqpoint{3.137314in}{3.389329in}}%
\pgfpathcurveto{\pgfqpoint{3.137314in}{3.402352in}}{\pgfqpoint{3.132140in}{3.414843in}}{\pgfqpoint{3.122931in}{3.424052in}}%
\pgfpathcurveto{\pgfqpoint{3.113723in}{3.433260in}}{\pgfqpoint{3.101232in}{3.438434in}}{\pgfqpoint{3.088209in}{3.438434in}}%
\pgfpathcurveto{\pgfqpoint{3.075186in}{3.438434in}}{\pgfqpoint{3.062695in}{3.433260in}}{\pgfqpoint{3.053487in}{3.424052in}}%
\pgfpathcurveto{\pgfqpoint{3.044278in}{3.414843in}}{\pgfqpoint{3.039104in}{3.402352in}}{\pgfqpoint{3.039104in}{3.389329in}}%
\pgfpathcurveto{\pgfqpoint{3.039104in}{3.376307in}}{\pgfqpoint{3.044278in}{3.363816in}}{\pgfqpoint{3.053487in}{3.354607in}}%
\pgfpathcurveto{\pgfqpoint{3.062695in}{3.345399in}}{\pgfqpoint{3.075186in}{3.340225in}}{\pgfqpoint{3.088209in}{3.340225in}}%
\pgfpathlineto{\pgfqpoint{3.088209in}{3.340225in}}%
\pgfpathclose%
\pgfusepath{stroke,fill}%
\end{pgfscope}%
\begin{pgfscope}%
\pgfpathrectangle{\pgfqpoint{0.786164in}{0.768110in}}{\pgfqpoint{8.851069in}{7.081890in}}%
\pgfusepath{clip}%
\pgfsetbuttcap%
\pgfsetroundjoin%
\definecolor{currentfill}{rgb}{0.225863,0.330805,0.547314}%
\pgfsetfillcolor{currentfill}%
\pgfsetfillopacity{0.700000}%
\pgfsetlinewidth{0.501875pt}%
\definecolor{currentstroke}{rgb}{1.000000,1.000000,1.000000}%
\pgfsetstrokecolor{currentstroke}%
\pgfsetstrokeopacity{0.700000}%
\pgfsetdash{}{0pt}%
\pgfpathmoveto{\pgfqpoint{3.006009in}{3.186937in}}%
\pgfpathcurveto{\pgfqpoint{3.019032in}{3.186937in}}{\pgfqpoint{3.031523in}{3.192111in}}{\pgfqpoint{3.040732in}{3.201319in}}%
\pgfpathcurveto{\pgfqpoint{3.049940in}{3.210528in}}{\pgfqpoint{3.055114in}{3.223019in}}{\pgfqpoint{3.055114in}{3.236042in}}%
\pgfpathcurveto{\pgfqpoint{3.055114in}{3.249064in}}{\pgfqpoint{3.049940in}{3.261555in}}{\pgfqpoint{3.040732in}{3.270764in}}%
\pgfpathcurveto{\pgfqpoint{3.031523in}{3.279972in}}{\pgfqpoint{3.019032in}{3.285146in}}{\pgfqpoint{3.006009in}{3.285146in}}%
\pgfpathcurveto{\pgfqpoint{2.992987in}{3.285146in}}{\pgfqpoint{2.980496in}{3.279972in}}{\pgfqpoint{2.971287in}{3.270764in}}%
\pgfpathcurveto{\pgfqpoint{2.962079in}{3.261555in}}{\pgfqpoint{2.956905in}{3.249064in}}{\pgfqpoint{2.956905in}{3.236042in}}%
\pgfpathcurveto{\pgfqpoint{2.956905in}{3.223019in}}{\pgfqpoint{2.962079in}{3.210528in}}{\pgfqpoint{2.971287in}{3.201319in}}%
\pgfpathcurveto{\pgfqpoint{2.980496in}{3.192111in}}{\pgfqpoint{2.992987in}{3.186937in}}{\pgfqpoint{3.006009in}{3.186937in}}%
\pgfpathlineto{\pgfqpoint{3.006009in}{3.186937in}}%
\pgfpathclose%
\pgfusepath{stroke,fill}%
\end{pgfscope}%
\begin{pgfscope}%
\pgfpathrectangle{\pgfqpoint{0.786164in}{0.768110in}}{\pgfqpoint{8.851069in}{7.081890in}}%
\pgfusepath{clip}%
\pgfsetbuttcap%
\pgfsetroundjoin%
\definecolor{currentfill}{rgb}{0.212395,0.359683,0.551710}%
\pgfsetfillcolor{currentfill}%
\pgfsetfillopacity{0.700000}%
\pgfsetlinewidth{0.501875pt}%
\definecolor{currentstroke}{rgb}{1.000000,1.000000,1.000000}%
\pgfsetstrokecolor{currentstroke}%
\pgfsetstrokeopacity{0.700000}%
\pgfsetdash{}{0pt}%
\pgfpathmoveto{\pgfqpoint{2.750277in}{2.989853in}}%
\pgfpathcurveto{\pgfqpoint{2.763300in}{2.989853in}}{\pgfqpoint{2.775791in}{2.995027in}}{\pgfqpoint{2.785000in}{3.004235in}}%
\pgfpathcurveto{\pgfqpoint{2.794208in}{3.013444in}}{\pgfqpoint{2.799382in}{3.025935in}}{\pgfqpoint{2.799382in}{3.038958in}}%
\pgfpathcurveto{\pgfqpoint{2.799382in}{3.051980in}}{\pgfqpoint{2.794208in}{3.064471in}}{\pgfqpoint{2.785000in}{3.073680in}}%
\pgfpathcurveto{\pgfqpoint{2.775791in}{3.082888in}}{\pgfqpoint{2.763300in}{3.088062in}}{\pgfqpoint{2.750277in}{3.088062in}}%
\pgfpathcurveto{\pgfqpoint{2.737255in}{3.088062in}}{\pgfqpoint{2.724764in}{3.082888in}}{\pgfqpoint{2.715555in}{3.073680in}}%
\pgfpathcurveto{\pgfqpoint{2.706347in}{3.064471in}}{\pgfqpoint{2.701173in}{3.051980in}}{\pgfqpoint{2.701173in}{3.038958in}}%
\pgfpathcurveto{\pgfqpoint{2.701173in}{3.025935in}}{\pgfqpoint{2.706347in}{3.013444in}}{\pgfqpoint{2.715555in}{3.004235in}}%
\pgfpathcurveto{\pgfqpoint{2.724764in}{2.995027in}}{\pgfqpoint{2.737255in}{2.989853in}}{\pgfqpoint{2.750277in}{2.989853in}}%
\pgfpathlineto{\pgfqpoint{2.750277in}{2.989853in}}%
\pgfpathclose%
\pgfusepath{stroke,fill}%
\end{pgfscope}%
\begin{pgfscope}%
\pgfpathrectangle{\pgfqpoint{0.786164in}{0.768110in}}{\pgfqpoint{8.851069in}{7.081890in}}%
\pgfusepath{clip}%
\pgfsetbuttcap%
\pgfsetroundjoin%
\definecolor{currentfill}{rgb}{0.210503,0.363727,0.552206}%
\pgfsetfillcolor{currentfill}%
\pgfsetfillopacity{0.700000}%
\pgfsetlinewidth{0.501875pt}%
\definecolor{currentstroke}{rgb}{1.000000,1.000000,1.000000}%
\pgfsetstrokecolor{currentstroke}%
\pgfsetstrokeopacity{0.700000}%
\pgfsetdash{}{0pt}%
\pgfpathmoveto{\pgfqpoint{2.850744in}{3.077446in}}%
\pgfpathcurveto{\pgfqpoint{2.863766in}{3.077446in}}{\pgfqpoint{2.876257in}{3.082620in}}{\pgfqpoint{2.885466in}{3.091828in}}%
\pgfpathcurveto{\pgfqpoint{2.894674in}{3.101037in}}{\pgfqpoint{2.899848in}{3.113528in}}{\pgfqpoint{2.899848in}{3.126550in}}%
\pgfpathcurveto{\pgfqpoint{2.899848in}{3.139573in}}{\pgfqpoint{2.894674in}{3.152064in}}{\pgfqpoint{2.885466in}{3.161273in}}%
\pgfpathcurveto{\pgfqpoint{2.876257in}{3.170481in}}{\pgfqpoint{2.863766in}{3.175655in}}{\pgfqpoint{2.850744in}{3.175655in}}%
\pgfpathcurveto{\pgfqpoint{2.837721in}{3.175655in}}{\pgfqpoint{2.825230in}{3.170481in}}{\pgfqpoint{2.816021in}{3.161273in}}%
\pgfpathcurveto{\pgfqpoint{2.806813in}{3.152064in}}{\pgfqpoint{2.801639in}{3.139573in}}{\pgfqpoint{2.801639in}{3.126550in}}%
\pgfpathcurveto{\pgfqpoint{2.801639in}{3.113528in}}{\pgfqpoint{2.806813in}{3.101037in}}{\pgfqpoint{2.816021in}{3.091828in}}%
\pgfpathcurveto{\pgfqpoint{2.825230in}{3.082620in}}{\pgfqpoint{2.837721in}{3.077446in}}{\pgfqpoint{2.850744in}{3.077446in}}%
\pgfpathlineto{\pgfqpoint{2.850744in}{3.077446in}}%
\pgfpathclose%
\pgfusepath{stroke,fill}%
\end{pgfscope}%
\begin{pgfscope}%
\pgfpathrectangle{\pgfqpoint{0.786164in}{0.768110in}}{\pgfqpoint{8.851069in}{7.081890in}}%
\pgfusepath{clip}%
\pgfsetbuttcap%
\pgfsetroundjoin%
\definecolor{currentfill}{rgb}{0.199430,0.387607,0.554642}%
\pgfsetfillcolor{currentfill}%
\pgfsetfillopacity{0.700000}%
\pgfsetlinewidth{0.501875pt}%
\definecolor{currentstroke}{rgb}{1.000000,1.000000,1.000000}%
\pgfsetstrokecolor{currentstroke}%
\pgfsetstrokeopacity{0.700000}%
\pgfsetdash{}{0pt}%
\pgfpathmoveto{\pgfqpoint{2.704611in}{3.055548in}}%
\pgfpathcurveto{\pgfqpoint{2.717634in}{3.055548in}}{\pgfqpoint{2.730125in}{3.060722in}}{\pgfqpoint{2.739333in}{3.069930in}}%
\pgfpathcurveto{\pgfqpoint{2.748542in}{3.079138in}}{\pgfqpoint{2.753716in}{3.091630in}}{\pgfqpoint{2.753716in}{3.104652in}}%
\pgfpathcurveto{\pgfqpoint{2.753716in}{3.117675in}}{\pgfqpoint{2.748542in}{3.130166in}}{\pgfqpoint{2.739333in}{3.139374in}}%
\pgfpathcurveto{\pgfqpoint{2.730125in}{3.148583in}}{\pgfqpoint{2.717634in}{3.153757in}}{\pgfqpoint{2.704611in}{3.153757in}}%
\pgfpathcurveto{\pgfqpoint{2.691588in}{3.153757in}}{\pgfqpoint{2.679097in}{3.148583in}}{\pgfqpoint{2.669889in}{3.139374in}}%
\pgfpathcurveto{\pgfqpoint{2.660680in}{3.130166in}}{\pgfqpoint{2.655506in}{3.117675in}}{\pgfqpoint{2.655506in}{3.104652in}}%
\pgfpathcurveto{\pgfqpoint{2.655506in}{3.091630in}}{\pgfqpoint{2.660680in}{3.079138in}}{\pgfqpoint{2.669889in}{3.069930in}}%
\pgfpathcurveto{\pgfqpoint{2.679097in}{3.060722in}}{\pgfqpoint{2.691588in}{3.055548in}}{\pgfqpoint{2.704611in}{3.055548in}}%
\pgfpathlineto{\pgfqpoint{2.704611in}{3.055548in}}%
\pgfpathclose%
\pgfusepath{stroke,fill}%
\end{pgfscope}%
\begin{pgfscope}%
\pgfpathrectangle{\pgfqpoint{0.786164in}{0.768110in}}{\pgfqpoint{8.851069in}{7.081890in}}%
\pgfusepath{clip}%
\pgfsetbuttcap%
\pgfsetroundjoin%
\definecolor{currentfill}{rgb}{0.243113,0.292092,0.538516}%
\pgfsetfillcolor{currentfill}%
\pgfsetfillopacity{0.700000}%
\pgfsetlinewidth{0.501875pt}%
\definecolor{currentstroke}{rgb}{1.000000,1.000000,1.000000}%
\pgfsetstrokecolor{currentstroke}%
\pgfsetstrokeopacity{0.700000}%
\pgfsetdash{}{0pt}%
\pgfpathmoveto{\pgfqpoint{3.243475in}{4.259951in}}%
\pgfpathcurveto{\pgfqpoint{3.256498in}{4.259951in}}{\pgfqpoint{3.268989in}{4.265125in}}{\pgfqpoint{3.278197in}{4.274333in}}%
\pgfpathcurveto{\pgfqpoint{3.287406in}{4.283541in}}{\pgfqpoint{3.292580in}{4.296033in}}{\pgfqpoint{3.292580in}{4.309055in}}%
\pgfpathcurveto{\pgfqpoint{3.292580in}{4.322078in}}{\pgfqpoint{3.287406in}{4.334569in}}{\pgfqpoint{3.278197in}{4.343777in}}%
\pgfpathcurveto{\pgfqpoint{3.268989in}{4.352986in}}{\pgfqpoint{3.256498in}{4.358160in}}{\pgfqpoint{3.243475in}{4.358160in}}%
\pgfpathcurveto{\pgfqpoint{3.230452in}{4.358160in}}{\pgfqpoint{3.217961in}{4.352986in}}{\pgfqpoint{3.208753in}{4.343777in}}%
\pgfpathcurveto{\pgfqpoint{3.199544in}{4.334569in}}{\pgfqpoint{3.194370in}{4.322078in}}{\pgfqpoint{3.194370in}{4.309055in}}%
\pgfpathcurveto{\pgfqpoint{3.194370in}{4.296033in}}{\pgfqpoint{3.199544in}{4.283541in}}{\pgfqpoint{3.208753in}{4.274333in}}%
\pgfpathcurveto{\pgfqpoint{3.217961in}{4.265125in}}{\pgfqpoint{3.230452in}{4.259951in}}{\pgfqpoint{3.243475in}{4.259951in}}%
\pgfpathlineto{\pgfqpoint{3.243475in}{4.259951in}}%
\pgfpathclose%
\pgfusepath{stroke,fill}%
\end{pgfscope}%
\begin{pgfscope}%
\pgfpathrectangle{\pgfqpoint{0.786164in}{0.768110in}}{\pgfqpoint{8.851069in}{7.081890in}}%
\pgfusepath{clip}%
\pgfsetbuttcap%
\pgfsetroundjoin%
\definecolor{currentfill}{rgb}{0.235526,0.309527,0.542944}%
\pgfsetfillcolor{currentfill}%
\pgfsetfillopacity{0.700000}%
\pgfsetlinewidth{0.501875pt}%
\definecolor{currentstroke}{rgb}{1.000000,1.000000,1.000000}%
\pgfsetstrokecolor{currentstroke}%
\pgfsetstrokeopacity{0.700000}%
\pgfsetdash{}{0pt}%
\pgfpathmoveto{\pgfqpoint{3.152142in}{4.062866in}}%
\pgfpathcurveto{\pgfqpoint{3.165165in}{4.062866in}}{\pgfqpoint{3.177656in}{4.068040in}}{\pgfqpoint{3.186864in}{4.077249in}}%
\pgfpathcurveto{\pgfqpoint{3.196073in}{4.086457in}}{\pgfqpoint{3.201247in}{4.098948in}}{\pgfqpoint{3.201247in}{4.111971in}}%
\pgfpathcurveto{\pgfqpoint{3.201247in}{4.124994in}}{\pgfqpoint{3.196073in}{4.137485in}}{\pgfqpoint{3.186864in}{4.146693in}}%
\pgfpathcurveto{\pgfqpoint{3.177656in}{4.155902in}}{\pgfqpoint{3.165165in}{4.161076in}}{\pgfqpoint{3.152142in}{4.161076in}}%
\pgfpathcurveto{\pgfqpoint{3.139119in}{4.161076in}}{\pgfqpoint{3.126628in}{4.155902in}}{\pgfqpoint{3.117420in}{4.146693in}}%
\pgfpathcurveto{\pgfqpoint{3.108211in}{4.137485in}}{\pgfqpoint{3.103037in}{4.124994in}}{\pgfqpoint{3.103037in}{4.111971in}}%
\pgfpathcurveto{\pgfqpoint{3.103037in}{4.098948in}}{\pgfqpoint{3.108211in}{4.086457in}}{\pgfqpoint{3.117420in}{4.077249in}}%
\pgfpathcurveto{\pgfqpoint{3.126628in}{4.068040in}}{\pgfqpoint{3.139119in}{4.062866in}}{\pgfqpoint{3.152142in}{4.062866in}}%
\pgfpathlineto{\pgfqpoint{3.152142in}{4.062866in}}%
\pgfpathclose%
\pgfusepath{stroke,fill}%
\end{pgfscope}%
\begin{pgfscope}%
\pgfpathrectangle{\pgfqpoint{0.786164in}{0.768110in}}{\pgfqpoint{8.851069in}{7.081890in}}%
\pgfusepath{clip}%
\pgfsetbuttcap%
\pgfsetroundjoin%
\definecolor{currentfill}{rgb}{0.231674,0.318106,0.544834}%
\pgfsetfillcolor{currentfill}%
\pgfsetfillopacity{0.700000}%
\pgfsetlinewidth{0.501875pt}%
\definecolor{currentstroke}{rgb}{1.000000,1.000000,1.000000}%
\pgfsetstrokecolor{currentstroke}%
\pgfsetstrokeopacity{0.700000}%
\pgfsetdash{}{0pt}%
\pgfpathmoveto{\pgfqpoint{3.371341in}{4.391340in}}%
\pgfpathcurveto{\pgfqpoint{3.384364in}{4.391340in}}{\pgfqpoint{3.396855in}{4.396514in}}{\pgfqpoint{3.406063in}{4.405722in}}%
\pgfpathcurveto{\pgfqpoint{3.415272in}{4.414931in}}{\pgfqpoint{3.420446in}{4.427422in}}{\pgfqpoint{3.420446in}{4.440445in}}%
\pgfpathcurveto{\pgfqpoint{3.420446in}{4.453467in}}{\pgfqpoint{3.415272in}{4.465958in}}{\pgfqpoint{3.406063in}{4.475167in}}%
\pgfpathcurveto{\pgfqpoint{3.396855in}{4.484375in}}{\pgfqpoint{3.384364in}{4.489549in}}{\pgfqpoint{3.371341in}{4.489549in}}%
\pgfpathcurveto{\pgfqpoint{3.358318in}{4.489549in}}{\pgfqpoint{3.345827in}{4.484375in}}{\pgfqpoint{3.336619in}{4.475167in}}%
\pgfpathcurveto{\pgfqpoint{3.327410in}{4.465958in}}{\pgfqpoint{3.322236in}{4.453467in}}{\pgfqpoint{3.322236in}{4.440445in}}%
\pgfpathcurveto{\pgfqpoint{3.322236in}{4.427422in}}{\pgfqpoint{3.327410in}{4.414931in}}{\pgfqpoint{3.336619in}{4.405722in}}%
\pgfpathcurveto{\pgfqpoint{3.345827in}{4.396514in}}{\pgfqpoint{3.358318in}{4.391340in}}{\pgfqpoint{3.371341in}{4.391340in}}%
\pgfpathlineto{\pgfqpoint{3.371341in}{4.391340in}}%
\pgfpathclose%
\pgfusepath{stroke,fill}%
\end{pgfscope}%
\begin{pgfscope}%
\pgfpathrectangle{\pgfqpoint{0.786164in}{0.768110in}}{\pgfqpoint{8.851069in}{7.081890in}}%
\pgfusepath{clip}%
\pgfsetbuttcap%
\pgfsetroundjoin%
\definecolor{currentfill}{rgb}{0.221989,0.339161,0.548752}%
\pgfsetfillcolor{currentfill}%
\pgfsetfillopacity{0.700000}%
\pgfsetlinewidth{0.501875pt}%
\definecolor{currentstroke}{rgb}{1.000000,1.000000,1.000000}%
\pgfsetstrokecolor{currentstroke}%
\pgfsetstrokeopacity{0.700000}%
\pgfsetdash{}{0pt}%
\pgfpathmoveto{\pgfqpoint{3.280008in}{4.194256in}}%
\pgfpathcurveto{\pgfqpoint{3.293031in}{4.194256in}}{\pgfqpoint{3.305522in}{4.199430in}}{\pgfqpoint{3.314730in}{4.208638in}}%
\pgfpathcurveto{\pgfqpoint{3.323939in}{4.217847in}}{\pgfqpoint{3.329113in}{4.230338in}}{\pgfqpoint{3.329113in}{4.243360in}}%
\pgfpathcurveto{\pgfqpoint{3.329113in}{4.256383in}}{\pgfqpoint{3.323939in}{4.268874in}}{\pgfqpoint{3.314730in}{4.278083in}}%
\pgfpathcurveto{\pgfqpoint{3.305522in}{4.287291in}}{\pgfqpoint{3.293031in}{4.292465in}}{\pgfqpoint{3.280008in}{4.292465in}}%
\pgfpathcurveto{\pgfqpoint{3.266985in}{4.292465in}}{\pgfqpoint{3.254494in}{4.287291in}}{\pgfqpoint{3.245286in}{4.278083in}}%
\pgfpathcurveto{\pgfqpoint{3.236077in}{4.268874in}}{\pgfqpoint{3.230903in}{4.256383in}}{\pgfqpoint{3.230903in}{4.243360in}}%
\pgfpathcurveto{\pgfqpoint{3.230903in}{4.230338in}}{\pgfqpoint{3.236077in}{4.217847in}}{\pgfqpoint{3.245286in}{4.208638in}}%
\pgfpathcurveto{\pgfqpoint{3.254494in}{4.199430in}}{\pgfqpoint{3.266985in}{4.194256in}}{\pgfqpoint{3.280008in}{4.194256in}}%
\pgfpathlineto{\pgfqpoint{3.280008in}{4.194256in}}%
\pgfpathclose%
\pgfusepath{stroke,fill}%
\end{pgfscope}%
\begin{pgfscope}%
\pgfpathrectangle{\pgfqpoint{0.786164in}{0.768110in}}{\pgfqpoint{8.851069in}{7.081890in}}%
\pgfusepath{clip}%
\pgfsetbuttcap%
\pgfsetroundjoin%
\definecolor{currentfill}{rgb}{0.216210,0.351535,0.550627}%
\pgfsetfillcolor{currentfill}%
\pgfsetfillopacity{0.700000}%
\pgfsetlinewidth{0.501875pt}%
\definecolor{currentstroke}{rgb}{1.000000,1.000000,1.000000}%
\pgfsetstrokecolor{currentstroke}%
\pgfsetstrokeopacity{0.700000}%
\pgfsetdash{}{0pt}%
\pgfpathmoveto{\pgfqpoint{3.170409in}{3.975274in}}%
\pgfpathcurveto{\pgfqpoint{3.183431in}{3.975274in}}{\pgfqpoint{3.195922in}{3.980447in}}{\pgfqpoint{3.205131in}{3.989656in}}%
\pgfpathcurveto{\pgfqpoint{3.214339in}{3.998864in}}{\pgfqpoint{3.219513in}{4.011355in}}{\pgfqpoint{3.219513in}{4.024378in}}%
\pgfpathcurveto{\pgfqpoint{3.219513in}{4.037401in}}{\pgfqpoint{3.214339in}{4.049892in}}{\pgfqpoint{3.205131in}{4.059100in}}%
\pgfpathcurveto{\pgfqpoint{3.195922in}{4.068309in}}{\pgfqpoint{3.183431in}{4.073483in}}{\pgfqpoint{3.170409in}{4.073483in}}%
\pgfpathcurveto{\pgfqpoint{3.157386in}{4.073483in}}{\pgfqpoint{3.144895in}{4.068309in}}{\pgfqpoint{3.135686in}{4.059100in}}%
\pgfpathcurveto{\pgfqpoint{3.126478in}{4.049892in}}{\pgfqpoint{3.121304in}{4.037401in}}{\pgfqpoint{3.121304in}{4.024378in}}%
\pgfpathcurveto{\pgfqpoint{3.121304in}{4.011355in}}{\pgfqpoint{3.126478in}{3.998864in}}{\pgfqpoint{3.135686in}{3.989656in}}%
\pgfpathcurveto{\pgfqpoint{3.144895in}{3.980447in}}{\pgfqpoint{3.157386in}{3.975274in}}{\pgfqpoint{3.170409in}{3.975274in}}%
\pgfpathlineto{\pgfqpoint{3.170409in}{3.975274in}}%
\pgfpathclose%
\pgfusepath{stroke,fill}%
\end{pgfscope}%
\begin{pgfscope}%
\pgfpathrectangle{\pgfqpoint{0.786164in}{0.768110in}}{\pgfqpoint{8.851069in}{7.081890in}}%
\pgfusepath{clip}%
\pgfsetbuttcap%
\pgfsetroundjoin%
\definecolor{currentfill}{rgb}{0.204903,0.375746,0.553533}%
\pgfsetfillcolor{currentfill}%
\pgfsetfillopacity{0.700000}%
\pgfsetlinewidth{0.501875pt}%
\definecolor{currentstroke}{rgb}{1.000000,1.000000,1.000000}%
\pgfsetstrokecolor{currentstroke}%
\pgfsetstrokeopacity{0.700000}%
\pgfsetdash{}{0pt}%
\pgfpathmoveto{\pgfqpoint{3.024276in}{3.800088in}}%
\pgfpathcurveto{\pgfqpoint{3.037299in}{3.800088in}}{\pgfqpoint{3.049790in}{3.805262in}}{\pgfqpoint{3.058998in}{3.814470in}}%
\pgfpathcurveto{\pgfqpoint{3.068207in}{3.823678in}}{\pgfqpoint{3.073381in}{3.836170in}}{\pgfqpoint{3.073381in}{3.849192in}}%
\pgfpathcurveto{\pgfqpoint{3.073381in}{3.862215in}}{\pgfqpoint{3.068207in}{3.874706in}}{\pgfqpoint{3.058998in}{3.883914in}}%
\pgfpathcurveto{\pgfqpoint{3.049790in}{3.893123in}}{\pgfqpoint{3.037299in}{3.898297in}}{\pgfqpoint{3.024276in}{3.898297in}}%
\pgfpathcurveto{\pgfqpoint{3.011253in}{3.898297in}}{\pgfqpoint{2.998762in}{3.893123in}}{\pgfqpoint{2.989554in}{3.883914in}}%
\pgfpathcurveto{\pgfqpoint{2.980345in}{3.874706in}}{\pgfqpoint{2.975171in}{3.862215in}}{\pgfqpoint{2.975171in}{3.849192in}}%
\pgfpathcurveto{\pgfqpoint{2.975171in}{3.836170in}}{\pgfqpoint{2.980345in}{3.823678in}}{\pgfqpoint{2.989554in}{3.814470in}}%
\pgfpathcurveto{\pgfqpoint{2.998762in}{3.805262in}}{\pgfqpoint{3.011253in}{3.800088in}}{\pgfqpoint{3.024276in}{3.800088in}}%
\pgfpathlineto{\pgfqpoint{3.024276in}{3.800088in}}%
\pgfpathclose%
\pgfusepath{stroke,fill}%
\end{pgfscope}%
\begin{pgfscope}%
\pgfpathrectangle{\pgfqpoint{0.786164in}{0.768110in}}{\pgfqpoint{8.851069in}{7.081890in}}%
\pgfusepath{clip}%
\pgfsetbuttcap%
\pgfsetroundjoin%
\definecolor{currentfill}{rgb}{0.192357,0.403199,0.555836}%
\pgfsetfillcolor{currentfill}%
\pgfsetfillopacity{0.700000}%
\pgfsetlinewidth{0.501875pt}%
\definecolor{currentstroke}{rgb}{1.000000,1.000000,1.000000}%
\pgfsetstrokecolor{currentstroke}%
\pgfsetstrokeopacity{0.700000}%
\pgfsetdash{}{0pt}%
\pgfpathmoveto{\pgfqpoint{3.152142in}{3.756291in}}%
\pgfpathcurveto{\pgfqpoint{3.165165in}{3.756291in}}{\pgfqpoint{3.177656in}{3.761465in}}{\pgfqpoint{3.186864in}{3.770674in}}%
\pgfpathcurveto{\pgfqpoint{3.196073in}{3.779882in}}{\pgfqpoint{3.201247in}{3.792373in}}{\pgfqpoint{3.201247in}{3.805396in}}%
\pgfpathcurveto{\pgfqpoint{3.201247in}{3.818418in}}{\pgfqpoint{3.196073in}{3.830910in}}{\pgfqpoint{3.186864in}{3.840118in}}%
\pgfpathcurveto{\pgfqpoint{3.177656in}{3.849326in}}{\pgfqpoint{3.165165in}{3.854500in}}{\pgfqpoint{3.152142in}{3.854500in}}%
\pgfpathcurveto{\pgfqpoint{3.139119in}{3.854500in}}{\pgfqpoint{3.126628in}{3.849326in}}{\pgfqpoint{3.117420in}{3.840118in}}%
\pgfpathcurveto{\pgfqpoint{3.108211in}{3.830910in}}{\pgfqpoint{3.103037in}{3.818418in}}{\pgfqpoint{3.103037in}{3.805396in}}%
\pgfpathcurveto{\pgfqpoint{3.103037in}{3.792373in}}{\pgfqpoint{3.108211in}{3.779882in}}{\pgfqpoint{3.117420in}{3.770674in}}%
\pgfpathcurveto{\pgfqpoint{3.126628in}{3.761465in}}{\pgfqpoint{3.139119in}{3.756291in}}{\pgfqpoint{3.152142in}{3.756291in}}%
\pgfpathlineto{\pgfqpoint{3.152142in}{3.756291in}}%
\pgfpathclose%
\pgfusepath{stroke,fill}%
\end{pgfscope}%
\begin{pgfscope}%
\pgfpathrectangle{\pgfqpoint{0.786164in}{0.768110in}}{\pgfqpoint{8.851069in}{7.081890in}}%
\pgfusepath{clip}%
\pgfsetbuttcap%
\pgfsetroundjoin%
\definecolor{currentfill}{rgb}{0.182256,0.426184,0.557120}%
\pgfsetfillcolor{currentfill}%
\pgfsetfillopacity{0.700000}%
\pgfsetlinewidth{0.501875pt}%
\definecolor{currentstroke}{rgb}{1.000000,1.000000,1.000000}%
\pgfsetstrokecolor{currentstroke}%
\pgfsetstrokeopacity{0.700000}%
\pgfsetdash{}{0pt}%
\pgfpathmoveto{\pgfqpoint{2.905543in}{3.515411in}}%
\pgfpathcurveto{\pgfqpoint{2.918566in}{3.515411in}}{\pgfqpoint{2.931057in}{3.520585in}}{\pgfqpoint{2.940265in}{3.529793in}}%
\pgfpathcurveto{\pgfqpoint{2.949474in}{3.539001in}}{\pgfqpoint{2.954648in}{3.551492in}}{\pgfqpoint{2.954648in}{3.564515in}}%
\pgfpathcurveto{\pgfqpoint{2.954648in}{3.577538in}}{\pgfqpoint{2.949474in}{3.590029in}}{\pgfqpoint{2.940265in}{3.599237in}}%
\pgfpathcurveto{\pgfqpoint{2.931057in}{3.608446in}}{\pgfqpoint{2.918566in}{3.613620in}}{\pgfqpoint{2.905543in}{3.613620in}}%
\pgfpathcurveto{\pgfqpoint{2.892521in}{3.613620in}}{\pgfqpoint{2.880029in}{3.608446in}}{\pgfqpoint{2.870821in}{3.599237in}}%
\pgfpathcurveto{\pgfqpoint{2.861613in}{3.590029in}}{\pgfqpoint{2.856439in}{3.577538in}}{\pgfqpoint{2.856439in}{3.564515in}}%
\pgfpathcurveto{\pgfqpoint{2.856439in}{3.551492in}}{\pgfqpoint{2.861613in}{3.539001in}}{\pgfqpoint{2.870821in}{3.529793in}}%
\pgfpathcurveto{\pgfqpoint{2.880029in}{3.520585in}}{\pgfqpoint{2.892521in}{3.515411in}}{\pgfqpoint{2.905543in}{3.515411in}}%
\pgfpathlineto{\pgfqpoint{2.905543in}{3.515411in}}%
\pgfpathclose%
\pgfusepath{stroke,fill}%
\end{pgfscope}%
\begin{pgfscope}%
\pgfpathrectangle{\pgfqpoint{0.786164in}{0.768110in}}{\pgfqpoint{8.851069in}{7.081890in}}%
\pgfusepath{clip}%
\pgfsetbuttcap%
\pgfsetroundjoin%
\definecolor{currentfill}{rgb}{0.169646,0.456262,0.558030}%
\pgfsetfillcolor{currentfill}%
\pgfsetfillopacity{0.700000}%
\pgfsetlinewidth{0.501875pt}%
\definecolor{currentstroke}{rgb}{1.000000,1.000000,1.000000}%
\pgfsetstrokecolor{currentstroke}%
\pgfsetstrokeopacity{0.700000}%
\pgfsetdash{}{0pt}%
\pgfpathmoveto{\pgfqpoint{2.768544in}{3.296428in}}%
\pgfpathcurveto{\pgfqpoint{2.781567in}{3.296428in}}{\pgfqpoint{2.794058in}{3.301602in}}{\pgfqpoint{2.803266in}{3.310811in}}%
\pgfpathcurveto{\pgfqpoint{2.812475in}{3.320019in}}{\pgfqpoint{2.817649in}{3.332510in}}{\pgfqpoint{2.817649in}{3.345533in}}%
\pgfpathcurveto{\pgfqpoint{2.817649in}{3.358556in}}{\pgfqpoint{2.812475in}{3.371047in}}{\pgfqpoint{2.803266in}{3.380255in}}%
\pgfpathcurveto{\pgfqpoint{2.794058in}{3.389463in}}{\pgfqpoint{2.781567in}{3.394637in}}{\pgfqpoint{2.768544in}{3.394637in}}%
\pgfpathcurveto{\pgfqpoint{2.755521in}{3.394637in}}{\pgfqpoint{2.743030in}{3.389463in}}{\pgfqpoint{2.733822in}{3.380255in}}%
\pgfpathcurveto{\pgfqpoint{2.724613in}{3.371047in}}{\pgfqpoint{2.719439in}{3.358556in}}{\pgfqpoint{2.719439in}{3.345533in}}%
\pgfpathcurveto{\pgfqpoint{2.719439in}{3.332510in}}{\pgfqpoint{2.724613in}{3.320019in}}{\pgfqpoint{2.733822in}{3.310811in}}%
\pgfpathcurveto{\pgfqpoint{2.743030in}{3.301602in}}{\pgfqpoint{2.755521in}{3.296428in}}{\pgfqpoint{2.768544in}{3.296428in}}%
\pgfpathlineto{\pgfqpoint{2.768544in}{3.296428in}}%
\pgfpathclose%
\pgfusepath{stroke,fill}%
\end{pgfscope}%
\begin{pgfscope}%
\pgfpathrectangle{\pgfqpoint{0.786164in}{0.768110in}}{\pgfqpoint{8.851069in}{7.081890in}}%
\pgfusepath{clip}%
\pgfsetbuttcap%
\pgfsetroundjoin%
\definecolor{currentfill}{rgb}{0.159194,0.482237,0.558073}%
\pgfsetfillcolor{currentfill}%
\pgfsetfillopacity{0.700000}%
\pgfsetlinewidth{0.501875pt}%
\definecolor{currentstroke}{rgb}{1.000000,1.000000,1.000000}%
\pgfsetstrokecolor{currentstroke}%
\pgfsetstrokeopacity{0.700000}%
\pgfsetdash{}{0pt}%
\pgfpathmoveto{\pgfqpoint{2.768544in}{3.340225in}}%
\pgfpathcurveto{\pgfqpoint{2.781567in}{3.340225in}}{\pgfqpoint{2.794058in}{3.345399in}}{\pgfqpoint{2.803266in}{3.354607in}}%
\pgfpathcurveto{\pgfqpoint{2.812475in}{3.363816in}}{\pgfqpoint{2.817649in}{3.376307in}}{\pgfqpoint{2.817649in}{3.389329in}}%
\pgfpathcurveto{\pgfqpoint{2.817649in}{3.402352in}}{\pgfqpoint{2.812475in}{3.414843in}}{\pgfqpoint{2.803266in}{3.424052in}}%
\pgfpathcurveto{\pgfqpoint{2.794058in}{3.433260in}}{\pgfqpoint{2.781567in}{3.438434in}}{\pgfqpoint{2.768544in}{3.438434in}}%
\pgfpathcurveto{\pgfqpoint{2.755521in}{3.438434in}}{\pgfqpoint{2.743030in}{3.433260in}}{\pgfqpoint{2.733822in}{3.424052in}}%
\pgfpathcurveto{\pgfqpoint{2.724613in}{3.414843in}}{\pgfqpoint{2.719439in}{3.402352in}}{\pgfqpoint{2.719439in}{3.389329in}}%
\pgfpathcurveto{\pgfqpoint{2.719439in}{3.376307in}}{\pgfqpoint{2.724613in}{3.363816in}}{\pgfqpoint{2.733822in}{3.354607in}}%
\pgfpathcurveto{\pgfqpoint{2.743030in}{3.345399in}}{\pgfqpoint{2.755521in}{3.340225in}}{\pgfqpoint{2.768544in}{3.340225in}}%
\pgfpathlineto{\pgfqpoint{2.768544in}{3.340225in}}%
\pgfpathclose%
\pgfusepath{stroke,fill}%
\end{pgfscope}%
\begin{pgfscope}%
\pgfpathrectangle{\pgfqpoint{0.786164in}{0.768110in}}{\pgfqpoint{8.851069in}{7.081890in}}%
\pgfusepath{clip}%
\pgfsetbuttcap%
\pgfsetroundjoin%
\definecolor{currentfill}{rgb}{0.147607,0.511733,0.557049}%
\pgfsetfillcolor{currentfill}%
\pgfsetfillopacity{0.700000}%
\pgfsetlinewidth{0.501875pt}%
\definecolor{currentstroke}{rgb}{1.000000,1.000000,1.000000}%
\pgfsetstrokecolor{currentstroke}%
\pgfsetstrokeopacity{0.700000}%
\pgfsetdash{}{0pt}%
\pgfpathmoveto{\pgfqpoint{2.613278in}{3.186937in}}%
\pgfpathcurveto{\pgfqpoint{2.626301in}{3.186937in}}{\pgfqpoint{2.638792in}{3.192111in}}{\pgfqpoint{2.648000in}{3.201319in}}%
\pgfpathcurveto{\pgfqpoint{2.657209in}{3.210528in}}{\pgfqpoint{2.662383in}{3.223019in}}{\pgfqpoint{2.662383in}{3.236042in}}%
\pgfpathcurveto{\pgfqpoint{2.662383in}{3.249064in}}{\pgfqpoint{2.657209in}{3.261555in}}{\pgfqpoint{2.648000in}{3.270764in}}%
\pgfpathcurveto{\pgfqpoint{2.638792in}{3.279972in}}{\pgfqpoint{2.626301in}{3.285146in}}{\pgfqpoint{2.613278in}{3.285146in}}%
\pgfpathcurveto{\pgfqpoint{2.600255in}{3.285146in}}{\pgfqpoint{2.587764in}{3.279972in}}{\pgfqpoint{2.578556in}{3.270764in}}%
\pgfpathcurveto{\pgfqpoint{2.569347in}{3.261555in}}{\pgfqpoint{2.564173in}{3.249064in}}{\pgfqpoint{2.564173in}{3.236042in}}%
\pgfpathcurveto{\pgfqpoint{2.564173in}{3.223019in}}{\pgfqpoint{2.569347in}{3.210528in}}{\pgfqpoint{2.578556in}{3.201319in}}%
\pgfpathcurveto{\pgfqpoint{2.587764in}{3.192111in}}{\pgfqpoint{2.600255in}{3.186937in}}{\pgfqpoint{2.613278in}{3.186937in}}%
\pgfpathlineto{\pgfqpoint{2.613278in}{3.186937in}}%
\pgfpathclose%
\pgfusepath{stroke,fill}%
\end{pgfscope}%
\begin{pgfscope}%
\pgfpathrectangle{\pgfqpoint{0.786164in}{0.768110in}}{\pgfqpoint{8.851069in}{7.081890in}}%
\pgfusepath{clip}%
\pgfsetbuttcap%
\pgfsetroundjoin%
\definecolor{currentfill}{rgb}{0.141935,0.526453,0.555991}%
\pgfsetfillcolor{currentfill}%
\pgfsetfillopacity{0.700000}%
\pgfsetlinewidth{0.501875pt}%
\definecolor{currentstroke}{rgb}{1.000000,1.000000,1.000000}%
\pgfsetstrokecolor{currentstroke}%
\pgfsetstrokeopacity{0.700000}%
\pgfsetdash{}{0pt}%
\pgfpathmoveto{\pgfqpoint{2.576745in}{3.033649in}}%
\pgfpathcurveto{\pgfqpoint{2.589768in}{3.033649in}}{\pgfqpoint{2.602259in}{3.038823in}}{\pgfqpoint{2.611467in}{3.048032in}}%
\pgfpathcurveto{\pgfqpoint{2.620676in}{3.057240in}}{\pgfqpoint{2.625850in}{3.069731in}}{\pgfqpoint{2.625850in}{3.082754in}}%
\pgfpathcurveto{\pgfqpoint{2.625850in}{3.095777in}}{\pgfqpoint{2.620676in}{3.108268in}}{\pgfqpoint{2.611467in}{3.117476in}}%
\pgfpathcurveto{\pgfqpoint{2.602259in}{3.126685in}}{\pgfqpoint{2.589768in}{3.131859in}}{\pgfqpoint{2.576745in}{3.131859in}}%
\pgfpathcurveto{\pgfqpoint{2.563722in}{3.131859in}}{\pgfqpoint{2.551231in}{3.126685in}}{\pgfqpoint{2.542023in}{3.117476in}}%
\pgfpathcurveto{\pgfqpoint{2.532814in}{3.108268in}}{\pgfqpoint{2.527640in}{3.095777in}}{\pgfqpoint{2.527640in}{3.082754in}}%
\pgfpathcurveto{\pgfqpoint{2.527640in}{3.069731in}}{\pgfqpoint{2.532814in}{3.057240in}}{\pgfqpoint{2.542023in}{3.048032in}}%
\pgfpathcurveto{\pgfqpoint{2.551231in}{3.038823in}}{\pgfqpoint{2.563722in}{3.033649in}}{\pgfqpoint{2.576745in}{3.033649in}}%
\pgfpathlineto{\pgfqpoint{2.576745in}{3.033649in}}%
\pgfpathclose%
\pgfusepath{stroke,fill}%
\end{pgfscope}%
\begin{pgfscope}%
\pgfpathrectangle{\pgfqpoint{0.786164in}{0.768110in}}{\pgfqpoint{8.851069in}{7.081890in}}%
\pgfusepath{clip}%
\pgfsetbuttcap%
\pgfsetroundjoin%
\definecolor{currentfill}{rgb}{0.135066,0.544853,0.554029}%
\pgfsetfillcolor{currentfill}%
\pgfsetfillopacity{0.700000}%
\pgfsetlinewidth{0.501875pt}%
\definecolor{currentstroke}{rgb}{1.000000,1.000000,1.000000}%
\pgfsetstrokecolor{currentstroke}%
\pgfsetstrokeopacity{0.700000}%
\pgfsetdash{}{0pt}%
\pgfpathmoveto{\pgfqpoint{2.631545in}{3.143141in}}%
\pgfpathcurveto{\pgfqpoint{2.644567in}{3.143141in}}{\pgfqpoint{2.657058in}{3.148314in}}{\pgfqpoint{2.666267in}{3.157523in}}%
\pgfpathcurveto{\pgfqpoint{2.675475in}{3.166731in}}{\pgfqpoint{2.680649in}{3.179222in}}{\pgfqpoint{2.680649in}{3.192245in}}%
\pgfpathcurveto{\pgfqpoint{2.680649in}{3.205268in}}{\pgfqpoint{2.675475in}{3.217759in}}{\pgfqpoint{2.666267in}{3.226967in}}%
\pgfpathcurveto{\pgfqpoint{2.657058in}{3.236176in}}{\pgfqpoint{2.644567in}{3.241350in}}{\pgfqpoint{2.631545in}{3.241350in}}%
\pgfpathcurveto{\pgfqpoint{2.618522in}{3.241350in}}{\pgfqpoint{2.606031in}{3.236176in}}{\pgfqpoint{2.596822in}{3.226967in}}%
\pgfpathcurveto{\pgfqpoint{2.587614in}{3.217759in}}{\pgfqpoint{2.582440in}{3.205268in}}{\pgfqpoint{2.582440in}{3.192245in}}%
\pgfpathcurveto{\pgfqpoint{2.582440in}{3.179222in}}{\pgfqpoint{2.587614in}{3.166731in}}{\pgfqpoint{2.596822in}{3.157523in}}%
\pgfpathcurveto{\pgfqpoint{2.606031in}{3.148314in}}{\pgfqpoint{2.618522in}{3.143141in}}{\pgfqpoint{2.631545in}{3.143141in}}%
\pgfpathlineto{\pgfqpoint{2.631545in}{3.143141in}}%
\pgfpathclose%
\pgfusepath{stroke,fill}%
\end{pgfscope}%
\begin{pgfscope}%
\pgfpathrectangle{\pgfqpoint{0.786164in}{0.768110in}}{\pgfqpoint{8.851069in}{7.081890in}}%
\pgfusepath{clip}%
\pgfsetbuttcap%
\pgfsetroundjoin%
\definecolor{currentfill}{rgb}{0.122606,0.585371,0.546557}%
\pgfsetfillcolor{currentfill}%
\pgfsetfillopacity{0.700000}%
\pgfsetlinewidth{0.501875pt}%
\definecolor{currentstroke}{rgb}{1.000000,1.000000,1.000000}%
\pgfsetstrokecolor{currentstroke}%
\pgfsetstrokeopacity{0.700000}%
\pgfsetdash{}{0pt}%
\pgfpathmoveto{\pgfqpoint{2.631545in}{3.033649in}}%
\pgfpathcurveto{\pgfqpoint{2.644567in}{3.033649in}}{\pgfqpoint{2.657058in}{3.038823in}}{\pgfqpoint{2.666267in}{3.048032in}}%
\pgfpathcurveto{\pgfqpoint{2.675475in}{3.057240in}}{\pgfqpoint{2.680649in}{3.069731in}}{\pgfqpoint{2.680649in}{3.082754in}}%
\pgfpathcurveto{\pgfqpoint{2.680649in}{3.095777in}}{\pgfqpoint{2.675475in}{3.108268in}}{\pgfqpoint{2.666267in}{3.117476in}}%
\pgfpathcurveto{\pgfqpoint{2.657058in}{3.126685in}}{\pgfqpoint{2.644567in}{3.131859in}}{\pgfqpoint{2.631545in}{3.131859in}}%
\pgfpathcurveto{\pgfqpoint{2.618522in}{3.131859in}}{\pgfqpoint{2.606031in}{3.126685in}}{\pgfqpoint{2.596822in}{3.117476in}}%
\pgfpathcurveto{\pgfqpoint{2.587614in}{3.108268in}}{\pgfqpoint{2.582440in}{3.095777in}}{\pgfqpoint{2.582440in}{3.082754in}}%
\pgfpathcurveto{\pgfqpoint{2.582440in}{3.069731in}}{\pgfqpoint{2.587614in}{3.057240in}}{\pgfqpoint{2.596822in}{3.048032in}}%
\pgfpathcurveto{\pgfqpoint{2.606031in}{3.038823in}}{\pgfqpoint{2.618522in}{3.033649in}}{\pgfqpoint{2.631545in}{3.033649in}}%
\pgfpathlineto{\pgfqpoint{2.631545in}{3.033649in}}%
\pgfpathclose%
\pgfusepath{stroke,fill}%
\end{pgfscope}%
\begin{pgfscope}%
\pgfpathrectangle{\pgfqpoint{0.786164in}{0.768110in}}{\pgfqpoint{8.851069in}{7.081890in}}%
\pgfusepath{clip}%
\pgfsetbuttcap%
\pgfsetroundjoin%
\definecolor{currentfill}{rgb}{0.120565,0.596422,0.543611}%
\pgfsetfillcolor{currentfill}%
\pgfsetfillopacity{0.700000}%
\pgfsetlinewidth{0.501875pt}%
\definecolor{currentstroke}{rgb}{1.000000,1.000000,1.000000}%
\pgfsetstrokecolor{currentstroke}%
\pgfsetstrokeopacity{0.700000}%
\pgfsetdash{}{0pt}%
\pgfpathmoveto{\pgfqpoint{2.622411in}{3.077446in}}%
\pgfpathcurveto{\pgfqpoint{2.635434in}{3.077446in}}{\pgfqpoint{2.647925in}{3.082620in}}{\pgfqpoint{2.657134in}{3.091828in}}%
\pgfpathcurveto{\pgfqpoint{2.666342in}{3.101037in}}{\pgfqpoint{2.671516in}{3.113528in}}{\pgfqpoint{2.671516in}{3.126550in}}%
\pgfpathcurveto{\pgfqpoint{2.671516in}{3.139573in}}{\pgfqpoint{2.666342in}{3.152064in}}{\pgfqpoint{2.657134in}{3.161273in}}%
\pgfpathcurveto{\pgfqpoint{2.647925in}{3.170481in}}{\pgfqpoint{2.635434in}{3.175655in}}{\pgfqpoint{2.622411in}{3.175655in}}%
\pgfpathcurveto{\pgfqpoint{2.609389in}{3.175655in}}{\pgfqpoint{2.596898in}{3.170481in}}{\pgfqpoint{2.587689in}{3.161273in}}%
\pgfpathcurveto{\pgfqpoint{2.578481in}{3.152064in}}{\pgfqpoint{2.573307in}{3.139573in}}{\pgfqpoint{2.573307in}{3.126550in}}%
\pgfpathcurveto{\pgfqpoint{2.573307in}{3.113528in}}{\pgfqpoint{2.578481in}{3.101037in}}{\pgfqpoint{2.587689in}{3.091828in}}%
\pgfpathcurveto{\pgfqpoint{2.596898in}{3.082620in}}{\pgfqpoint{2.609389in}{3.077446in}}{\pgfqpoint{2.622411in}{3.077446in}}%
\pgfpathlineto{\pgfqpoint{2.622411in}{3.077446in}}%
\pgfpathclose%
\pgfusepath{stroke,fill}%
\end{pgfscope}%
\begin{pgfscope}%
\pgfpathrectangle{\pgfqpoint{0.786164in}{0.768110in}}{\pgfqpoint{8.851069in}{7.081890in}}%
\pgfusepath{clip}%
\pgfsetbuttcap%
\pgfsetroundjoin%
\definecolor{currentfill}{rgb}{0.120081,0.622161,0.534946}%
\pgfsetfillcolor{currentfill}%
\pgfsetfillopacity{0.700000}%
\pgfsetlinewidth{0.501875pt}%
\definecolor{currentstroke}{rgb}{1.000000,1.000000,1.000000}%
\pgfsetstrokecolor{currentstroke}%
\pgfsetstrokeopacity{0.700000}%
\pgfsetdash{}{0pt}%
\pgfpathmoveto{\pgfqpoint{2.521945in}{2.902260in}}%
\pgfpathcurveto{\pgfqpoint{2.534968in}{2.902260in}}{\pgfqpoint{2.547459in}{2.907434in}}{\pgfqpoint{2.556667in}{2.916642in}}%
\pgfpathcurveto{\pgfqpoint{2.565876in}{2.925851in}}{\pgfqpoint{2.571050in}{2.938342in}}{\pgfqpoint{2.571050in}{2.951365in}}%
\pgfpathcurveto{\pgfqpoint{2.571050in}{2.964387in}}{\pgfqpoint{2.565876in}{2.976878in}}{\pgfqpoint{2.556667in}{2.986087in}}%
\pgfpathcurveto{\pgfqpoint{2.547459in}{2.995295in}}{\pgfqpoint{2.534968in}{3.000469in}}{\pgfqpoint{2.521945in}{3.000469in}}%
\pgfpathcurveto{\pgfqpoint{2.508922in}{3.000469in}}{\pgfqpoint{2.496431in}{2.995295in}}{\pgfqpoint{2.487223in}{2.986087in}}%
\pgfpathcurveto{\pgfqpoint{2.478015in}{2.976878in}}{\pgfqpoint{2.472841in}{2.964387in}}{\pgfqpoint{2.472841in}{2.951365in}}%
\pgfpathcurveto{\pgfqpoint{2.472841in}{2.938342in}}{\pgfqpoint{2.478015in}{2.925851in}}{\pgfqpoint{2.487223in}{2.916642in}}%
\pgfpathcurveto{\pgfqpoint{2.496431in}{2.907434in}}{\pgfqpoint{2.508922in}{2.902260in}}{\pgfqpoint{2.521945in}{2.902260in}}%
\pgfpathlineto{\pgfqpoint{2.521945in}{2.902260in}}%
\pgfpathclose%
\pgfusepath{stroke,fill}%
\end{pgfscope}%
\begin{pgfscope}%
\pgfpathrectangle{\pgfqpoint{0.786164in}{0.768110in}}{\pgfqpoint{8.851069in}{7.081890in}}%
\pgfusepath{clip}%
\pgfsetbuttcap%
\pgfsetroundjoin%
\definecolor{currentfill}{rgb}{0.135066,0.544853,0.554029}%
\pgfsetfillcolor{currentfill}%
\pgfsetfillopacity{0.700000}%
\pgfsetlinewidth{0.501875pt}%
\definecolor{currentstroke}{rgb}{1.000000,1.000000,1.000000}%
\pgfsetstrokecolor{currentstroke}%
\pgfsetstrokeopacity{0.700000}%
\pgfsetdash{}{0pt}%
\pgfpathmoveto{\pgfqpoint{2.293613in}{2.705176in}}%
\pgfpathcurveto{\pgfqpoint{2.306636in}{2.705176in}}{\pgfqpoint{2.319127in}{2.710350in}}{\pgfqpoint{2.328335in}{2.719558in}}%
\pgfpathcurveto{\pgfqpoint{2.337544in}{2.728767in}}{\pgfqpoint{2.342718in}{2.741258in}}{\pgfqpoint{2.342718in}{2.754280in}}%
\pgfpathcurveto{\pgfqpoint{2.342718in}{2.767303in}}{\pgfqpoint{2.337544in}{2.779794in}}{\pgfqpoint{2.328335in}{2.789003in}}%
\pgfpathcurveto{\pgfqpoint{2.319127in}{2.798211in}}{\pgfqpoint{2.306636in}{2.803385in}}{\pgfqpoint{2.293613in}{2.803385in}}%
\pgfpathcurveto{\pgfqpoint{2.280590in}{2.803385in}}{\pgfqpoint{2.268099in}{2.798211in}}{\pgfqpoint{2.258891in}{2.789003in}}%
\pgfpathcurveto{\pgfqpoint{2.249682in}{2.779794in}}{\pgfqpoint{2.244508in}{2.767303in}}{\pgfqpoint{2.244508in}{2.754280in}}%
\pgfpathcurveto{\pgfqpoint{2.244508in}{2.741258in}}{\pgfqpoint{2.249682in}{2.728767in}}{\pgfqpoint{2.258891in}{2.719558in}}%
\pgfpathcurveto{\pgfqpoint{2.268099in}{2.710350in}}{\pgfqpoint{2.280590in}{2.705176in}}{\pgfqpoint{2.293613in}{2.705176in}}%
\pgfpathlineto{\pgfqpoint{2.293613in}{2.705176in}}%
\pgfpathclose%
\pgfusepath{stroke,fill}%
\end{pgfscope}%
\begin{pgfscope}%
\pgfpathrectangle{\pgfqpoint{0.786164in}{0.768110in}}{\pgfqpoint{8.851069in}{7.081890in}}%
\pgfusepath{clip}%
\pgfsetbuttcap%
\pgfsetroundjoin%
\definecolor{currentfill}{rgb}{0.150148,0.676631,0.506589}%
\pgfsetfillcolor{currentfill}%
\pgfsetfillopacity{0.700000}%
\pgfsetlinewidth{0.501875pt}%
\definecolor{currentstroke}{rgb}{1.000000,1.000000,1.000000}%
\pgfsetstrokecolor{currentstroke}%
\pgfsetstrokeopacity{0.700000}%
\pgfsetdash{}{0pt}%
\pgfpathmoveto{\pgfqpoint{2.421479in}{2.705176in}}%
\pgfpathcurveto{\pgfqpoint{2.434502in}{2.705176in}}{\pgfqpoint{2.446993in}{2.710350in}}{\pgfqpoint{2.456201in}{2.719558in}}%
\pgfpathcurveto{\pgfqpoint{2.465410in}{2.728767in}}{\pgfqpoint{2.470584in}{2.741258in}}{\pgfqpoint{2.470584in}{2.754280in}}%
\pgfpathcurveto{\pgfqpoint{2.470584in}{2.767303in}}{\pgfqpoint{2.465410in}{2.779794in}}{\pgfqpoint{2.456201in}{2.789003in}}%
\pgfpathcurveto{\pgfqpoint{2.446993in}{2.798211in}}{\pgfqpoint{2.434502in}{2.803385in}}{\pgfqpoint{2.421479in}{2.803385in}}%
\pgfpathcurveto{\pgfqpoint{2.408456in}{2.803385in}}{\pgfqpoint{2.395965in}{2.798211in}}{\pgfqpoint{2.386757in}{2.789003in}}%
\pgfpathcurveto{\pgfqpoint{2.377548in}{2.779794in}}{\pgfqpoint{2.372374in}{2.767303in}}{\pgfqpoint{2.372374in}{2.754280in}}%
\pgfpathcurveto{\pgfqpoint{2.372374in}{2.741258in}}{\pgfqpoint{2.377548in}{2.728767in}}{\pgfqpoint{2.386757in}{2.719558in}}%
\pgfpathcurveto{\pgfqpoint{2.395965in}{2.710350in}}{\pgfqpoint{2.408456in}{2.705176in}}{\pgfqpoint{2.421479in}{2.705176in}}%
\pgfpathlineto{\pgfqpoint{2.421479in}{2.705176in}}%
\pgfpathclose%
\pgfusepath{stroke,fill}%
\end{pgfscope}%
\begin{pgfscope}%
\pgfpathrectangle{\pgfqpoint{0.786164in}{0.768110in}}{\pgfqpoint{8.851069in}{7.081890in}}%
\pgfusepath{clip}%
\pgfsetbuttcap%
\pgfsetroundjoin%
\definecolor{currentfill}{rgb}{0.162016,0.687316,0.499129}%
\pgfsetfillcolor{currentfill}%
\pgfsetfillopacity{0.700000}%
\pgfsetlinewidth{0.501875pt}%
\definecolor{currentstroke}{rgb}{1.000000,1.000000,1.000000}%
\pgfsetstrokecolor{currentstroke}%
\pgfsetstrokeopacity{0.700000}%
\pgfsetdash{}{0pt}%
\pgfpathmoveto{\pgfqpoint{2.348413in}{2.639481in}}%
\pgfpathcurveto{\pgfqpoint{2.361435in}{2.639481in}}{\pgfqpoint{2.373926in}{2.644655in}}{\pgfqpoint{2.383135in}{2.653864in}}%
\pgfpathcurveto{\pgfqpoint{2.392343in}{2.663072in}}{\pgfqpoint{2.397517in}{2.675563in}}{\pgfqpoint{2.397517in}{2.688586in}}%
\pgfpathcurveto{\pgfqpoint{2.397517in}{2.701608in}}{\pgfqpoint{2.392343in}{2.714100in}}{\pgfqpoint{2.383135in}{2.723308in}}%
\pgfpathcurveto{\pgfqpoint{2.373926in}{2.732516in}}{\pgfqpoint{2.361435in}{2.737690in}}{\pgfqpoint{2.348413in}{2.737690in}}%
\pgfpathcurveto{\pgfqpoint{2.335390in}{2.737690in}}{\pgfqpoint{2.322899in}{2.732516in}}{\pgfqpoint{2.313690in}{2.723308in}}%
\pgfpathcurveto{\pgfqpoint{2.304482in}{2.714100in}}{\pgfqpoint{2.299308in}{2.701608in}}{\pgfqpoint{2.299308in}{2.688586in}}%
\pgfpathcurveto{\pgfqpoint{2.299308in}{2.675563in}}{\pgfqpoint{2.304482in}{2.663072in}}{\pgfqpoint{2.313690in}{2.653864in}}%
\pgfpathcurveto{\pgfqpoint{2.322899in}{2.644655in}}{\pgfqpoint{2.335390in}{2.639481in}}{\pgfqpoint{2.348413in}{2.639481in}}%
\pgfpathlineto{\pgfqpoint{2.348413in}{2.639481in}}%
\pgfpathclose%
\pgfusepath{stroke,fill}%
\end{pgfscope}%
\begin{pgfscope}%
\pgfpathrectangle{\pgfqpoint{0.786164in}{0.768110in}}{\pgfqpoint{8.851069in}{7.081890in}}%
\pgfusepath{clip}%
\pgfsetbuttcap%
\pgfsetroundjoin%
\definecolor{currentfill}{rgb}{0.216210,0.351535,0.550627}%
\pgfsetfillcolor{currentfill}%
\pgfsetfillopacity{0.700000}%
\pgfsetlinewidth{0.501875pt}%
\definecolor{currentstroke}{rgb}{1.000000,1.000000,1.000000}%
\pgfsetstrokecolor{currentstroke}%
\pgfsetstrokeopacity{0.700000}%
\pgfsetdash{}{0pt}%
\pgfpathmoveto{\pgfqpoint{3.362208in}{3.887681in}}%
\pgfpathcurveto{\pgfqpoint{3.375230in}{3.887681in}}{\pgfqpoint{3.387721in}{3.892855in}}{\pgfqpoint{3.396930in}{3.902063in}}%
\pgfpathcurveto{\pgfqpoint{3.406138in}{3.911271in}}{\pgfqpoint{3.411312in}{3.923762in}}{\pgfqpoint{3.411312in}{3.936785in}}%
\pgfpathcurveto{\pgfqpoint{3.411312in}{3.949808in}}{\pgfqpoint{3.406138in}{3.962299in}}{\pgfqpoint{3.396930in}{3.971507in}}%
\pgfpathcurveto{\pgfqpoint{3.387721in}{3.980716in}}{\pgfqpoint{3.375230in}{3.985890in}}{\pgfqpoint{3.362208in}{3.985890in}}%
\pgfpathcurveto{\pgfqpoint{3.349185in}{3.985890in}}{\pgfqpoint{3.336694in}{3.980716in}}{\pgfqpoint{3.327485in}{3.971507in}}%
\pgfpathcurveto{\pgfqpoint{3.318277in}{3.962299in}}{\pgfqpoint{3.313103in}{3.949808in}}{\pgfqpoint{3.313103in}{3.936785in}}%
\pgfpathcurveto{\pgfqpoint{3.313103in}{3.923762in}}{\pgfqpoint{3.318277in}{3.911271in}}{\pgfqpoint{3.327485in}{3.902063in}}%
\pgfpathcurveto{\pgfqpoint{3.336694in}{3.892855in}}{\pgfqpoint{3.349185in}{3.887681in}}{\pgfqpoint{3.362208in}{3.887681in}}%
\pgfpathlineto{\pgfqpoint{3.362208in}{3.887681in}}%
\pgfpathclose%
\pgfusepath{stroke,fill}%
\end{pgfscope}%
\begin{pgfscope}%
\pgfpathrectangle{\pgfqpoint{0.786164in}{0.768110in}}{\pgfqpoint{8.851069in}{7.081890in}}%
\pgfusepath{clip}%
\pgfsetbuttcap%
\pgfsetroundjoin%
\definecolor{currentfill}{rgb}{0.223925,0.334994,0.548053}%
\pgfsetfillcolor{currentfill}%
\pgfsetfillopacity{0.700000}%
\pgfsetlinewidth{0.501875pt}%
\definecolor{currentstroke}{rgb}{1.000000,1.000000,1.000000}%
\pgfsetstrokecolor{currentstroke}%
\pgfsetstrokeopacity{0.700000}%
\pgfsetdash{}{0pt}%
\pgfpathmoveto{\pgfqpoint{3.435274in}{3.668698in}}%
\pgfpathcurveto{\pgfqpoint{3.448297in}{3.668698in}}{\pgfqpoint{3.460788in}{3.673872in}}{\pgfqpoint{3.469996in}{3.683081in}}%
\pgfpathcurveto{\pgfqpoint{3.479205in}{3.692289in}}{\pgfqpoint{3.484379in}{3.704780in}}{\pgfqpoint{3.484379in}{3.717803in}}%
\pgfpathcurveto{\pgfqpoint{3.484379in}{3.730826in}}{\pgfqpoint{3.479205in}{3.743317in}}{\pgfqpoint{3.469996in}{3.752525in}}%
\pgfpathcurveto{\pgfqpoint{3.460788in}{3.761733in}}{\pgfqpoint{3.448297in}{3.766907in}}{\pgfqpoint{3.435274in}{3.766907in}}%
\pgfpathcurveto{\pgfqpoint{3.422251in}{3.766907in}}{\pgfqpoint{3.409760in}{3.761733in}}{\pgfqpoint{3.400552in}{3.752525in}}%
\pgfpathcurveto{\pgfqpoint{3.391343in}{3.743317in}}{\pgfqpoint{3.386169in}{3.730826in}}{\pgfqpoint{3.386169in}{3.717803in}}%
\pgfpathcurveto{\pgfqpoint{3.386169in}{3.704780in}}{\pgfqpoint{3.391343in}{3.692289in}}{\pgfqpoint{3.400552in}{3.683081in}}%
\pgfpathcurveto{\pgfqpoint{3.409760in}{3.673872in}}{\pgfqpoint{3.422251in}{3.668698in}}{\pgfqpoint{3.435274in}{3.668698in}}%
\pgfpathlineto{\pgfqpoint{3.435274in}{3.668698in}}%
\pgfpathclose%
\pgfusepath{stroke,fill}%
\end{pgfscope}%
\begin{pgfscope}%
\pgfpathrectangle{\pgfqpoint{0.786164in}{0.768110in}}{\pgfqpoint{8.851069in}{7.081890in}}%
\pgfusepath{clip}%
\pgfsetbuttcap%
\pgfsetroundjoin%
\definecolor{currentfill}{rgb}{0.235526,0.309527,0.542944}%
\pgfsetfillcolor{currentfill}%
\pgfsetfillopacity{0.700000}%
\pgfsetlinewidth{0.501875pt}%
\definecolor{currentstroke}{rgb}{1.000000,1.000000,1.000000}%
\pgfsetstrokecolor{currentstroke}%
\pgfsetstrokeopacity{0.700000}%
\pgfsetdash{}{0pt}%
\pgfpathmoveto{\pgfqpoint{3.417007in}{3.405919in}}%
\pgfpathcurveto{\pgfqpoint{3.430030in}{3.405919in}}{\pgfqpoint{3.442521in}{3.411093in}}{\pgfqpoint{3.451730in}{3.420302in}}%
\pgfpathcurveto{\pgfqpoint{3.460938in}{3.429510in}}{\pgfqpoint{3.466112in}{3.442001in}}{\pgfqpoint{3.466112in}{3.455024in}}%
\pgfpathcurveto{\pgfqpoint{3.466112in}{3.468047in}}{\pgfqpoint{3.460938in}{3.480538in}}{\pgfqpoint{3.451730in}{3.489746in}}%
\pgfpathcurveto{\pgfqpoint{3.442521in}{3.498955in}}{\pgfqpoint{3.430030in}{3.504129in}}{\pgfqpoint{3.417007in}{3.504129in}}%
\pgfpathcurveto{\pgfqpoint{3.403985in}{3.504129in}}{\pgfqpoint{3.391494in}{3.498955in}}{\pgfqpoint{3.382285in}{3.489746in}}%
\pgfpathcurveto{\pgfqpoint{3.373077in}{3.480538in}}{\pgfqpoint{3.367903in}{3.468047in}}{\pgfqpoint{3.367903in}{3.455024in}}%
\pgfpathcurveto{\pgfqpoint{3.367903in}{3.442001in}}{\pgfqpoint{3.373077in}{3.429510in}}{\pgfqpoint{3.382285in}{3.420302in}}%
\pgfpathcurveto{\pgfqpoint{3.391494in}{3.411093in}}{\pgfqpoint{3.403985in}{3.405919in}}{\pgfqpoint{3.417007in}{3.405919in}}%
\pgfpathlineto{\pgfqpoint{3.417007in}{3.405919in}}%
\pgfpathclose%
\pgfusepath{stroke,fill}%
\end{pgfscope}%
\begin{pgfscope}%
\pgfpathrectangle{\pgfqpoint{0.786164in}{0.768110in}}{\pgfqpoint{8.851069in}{7.081890in}}%
\pgfusepath{clip}%
\pgfsetbuttcap%
\pgfsetroundjoin%
\definecolor{currentfill}{rgb}{0.225863,0.330805,0.547314}%
\pgfsetfillcolor{currentfill}%
\pgfsetfillopacity{0.700000}%
\pgfsetlinewidth{0.501875pt}%
\definecolor{currentstroke}{rgb}{1.000000,1.000000,1.000000}%
\pgfsetstrokecolor{currentstroke}%
\pgfsetstrokeopacity{0.700000}%
\pgfsetdash{}{0pt}%
\pgfpathmoveto{\pgfqpoint{4.065471in}{3.778189in}}%
\pgfpathcurveto{\pgfqpoint{4.078493in}{3.778189in}}{\pgfqpoint{4.090985in}{3.783363in}}{\pgfqpoint{4.100193in}{3.792572in}}%
\pgfpathcurveto{\pgfqpoint{4.109401in}{3.801780in}}{\pgfqpoint{4.114575in}{3.814271in}}{\pgfqpoint{4.114575in}{3.827294in}}%
\pgfpathcurveto{\pgfqpoint{4.114575in}{3.840317in}}{\pgfqpoint{4.109401in}{3.852808in}}{\pgfqpoint{4.100193in}{3.862016in}}%
\pgfpathcurveto{\pgfqpoint{4.090985in}{3.871225in}}{\pgfqpoint{4.078493in}{3.876399in}}{\pgfqpoint{4.065471in}{3.876399in}}%
\pgfpathcurveto{\pgfqpoint{4.052448in}{3.876399in}}{\pgfqpoint{4.039957in}{3.871225in}}{\pgfqpoint{4.030749in}{3.862016in}}%
\pgfpathcurveto{\pgfqpoint{4.021540in}{3.852808in}}{\pgfqpoint{4.016366in}{3.840317in}}{\pgfqpoint{4.016366in}{3.827294in}}%
\pgfpathcurveto{\pgfqpoint{4.016366in}{3.814271in}}{\pgfqpoint{4.021540in}{3.801780in}}{\pgfqpoint{4.030749in}{3.792572in}}%
\pgfpathcurveto{\pgfqpoint{4.039957in}{3.783363in}}{\pgfqpoint{4.052448in}{3.778189in}}{\pgfqpoint{4.065471in}{3.778189in}}%
\pgfpathlineto{\pgfqpoint{4.065471in}{3.778189in}}%
\pgfpathclose%
\pgfusepath{stroke,fill}%
\end{pgfscope}%
\begin{pgfscope}%
\pgfpathrectangle{\pgfqpoint{0.786164in}{0.768110in}}{\pgfqpoint{8.851069in}{7.081890in}}%
\pgfusepath{clip}%
\pgfsetbuttcap%
\pgfsetroundjoin%
\definecolor{currentfill}{rgb}{0.214298,0.355619,0.551184}%
\pgfsetfillcolor{currentfill}%
\pgfsetfillopacity{0.700000}%
\pgfsetlinewidth{0.501875pt}%
\definecolor{currentstroke}{rgb}{1.000000,1.000000,1.000000}%
\pgfsetstrokecolor{currentstroke}%
\pgfsetstrokeopacity{0.700000}%
\pgfsetdash{}{0pt}%
\pgfpathmoveto{\pgfqpoint{3.544873in}{3.405919in}}%
\pgfpathcurveto{\pgfqpoint{3.557896in}{3.405919in}}{\pgfqpoint{3.570387in}{3.411093in}}{\pgfqpoint{3.579596in}{3.420302in}}%
\pgfpathcurveto{\pgfqpoint{3.588804in}{3.429510in}}{\pgfqpoint{3.593978in}{3.442001in}}{\pgfqpoint{3.593978in}{3.455024in}}%
\pgfpathcurveto{\pgfqpoint{3.593978in}{3.468047in}}{\pgfqpoint{3.588804in}{3.480538in}}{\pgfqpoint{3.579596in}{3.489746in}}%
\pgfpathcurveto{\pgfqpoint{3.570387in}{3.498955in}}{\pgfqpoint{3.557896in}{3.504129in}}{\pgfqpoint{3.544873in}{3.504129in}}%
\pgfpathcurveto{\pgfqpoint{3.531851in}{3.504129in}}{\pgfqpoint{3.519360in}{3.498955in}}{\pgfqpoint{3.510151in}{3.489746in}}%
\pgfpathcurveto{\pgfqpoint{3.500943in}{3.480538in}}{\pgfqpoint{3.495769in}{3.468047in}}{\pgfqpoint{3.495769in}{3.455024in}}%
\pgfpathcurveto{\pgfqpoint{3.495769in}{3.442001in}}{\pgfqpoint{3.500943in}{3.429510in}}{\pgfqpoint{3.510151in}{3.420302in}}%
\pgfpathcurveto{\pgfqpoint{3.519360in}{3.411093in}}{\pgfqpoint{3.531851in}{3.405919in}}{\pgfqpoint{3.544873in}{3.405919in}}%
\pgfpathlineto{\pgfqpoint{3.544873in}{3.405919in}}%
\pgfpathclose%
\pgfusepath{stroke,fill}%
\end{pgfscope}%
\begin{pgfscope}%
\pgfpathrectangle{\pgfqpoint{0.786164in}{0.768110in}}{\pgfqpoint{8.851069in}{7.081890in}}%
\pgfusepath{clip}%
\pgfsetbuttcap%
\pgfsetroundjoin%
\definecolor{currentfill}{rgb}{0.185556,0.418570,0.556753}%
\pgfsetfillcolor{currentfill}%
\pgfsetfillopacity{0.700000}%
\pgfsetlinewidth{0.501875pt}%
\definecolor{currentstroke}{rgb}{1.000000,1.000000,1.000000}%
\pgfsetstrokecolor{currentstroke}%
\pgfsetstrokeopacity{0.700000}%
\pgfsetdash{}{0pt}%
\pgfpathmoveto{\pgfqpoint{2.832477in}{2.705176in}}%
\pgfpathcurveto{\pgfqpoint{2.845500in}{2.705176in}}{\pgfqpoint{2.857991in}{2.710350in}}{\pgfqpoint{2.867199in}{2.719558in}}%
\pgfpathcurveto{\pgfqpoint{2.876408in}{2.728767in}}{\pgfqpoint{2.881582in}{2.741258in}}{\pgfqpoint{2.881582in}{2.754280in}}%
\pgfpathcurveto{\pgfqpoint{2.881582in}{2.767303in}}{\pgfqpoint{2.876408in}{2.779794in}}{\pgfqpoint{2.867199in}{2.789003in}}%
\pgfpathcurveto{\pgfqpoint{2.857991in}{2.798211in}}{\pgfqpoint{2.845500in}{2.803385in}}{\pgfqpoint{2.832477in}{2.803385in}}%
\pgfpathcurveto{\pgfqpoint{2.819454in}{2.803385in}}{\pgfqpoint{2.806963in}{2.798211in}}{\pgfqpoint{2.797755in}{2.789003in}}%
\pgfpathcurveto{\pgfqpoint{2.788546in}{2.779794in}}{\pgfqpoint{2.783372in}{2.767303in}}{\pgfqpoint{2.783372in}{2.754280in}}%
\pgfpathcurveto{\pgfqpoint{2.783372in}{2.741258in}}{\pgfqpoint{2.788546in}{2.728767in}}{\pgfqpoint{2.797755in}{2.719558in}}%
\pgfpathcurveto{\pgfqpoint{2.806963in}{2.710350in}}{\pgfqpoint{2.819454in}{2.705176in}}{\pgfqpoint{2.832477in}{2.705176in}}%
\pgfpathlineto{\pgfqpoint{2.832477in}{2.705176in}}%
\pgfpathclose%
\pgfusepath{stroke,fill}%
\end{pgfscope}%
\begin{pgfscope}%
\pgfpathrectangle{\pgfqpoint{0.786164in}{0.768110in}}{\pgfqpoint{8.851069in}{7.081890in}}%
\pgfusepath{clip}%
\pgfsetbuttcap%
\pgfsetroundjoin%
\definecolor{currentfill}{rgb}{0.174274,0.445044,0.557792}%
\pgfsetfillcolor{currentfill}%
\pgfsetfillopacity{0.700000}%
\pgfsetlinewidth{0.501875pt}%
\definecolor{currentstroke}{rgb}{1.000000,1.000000,1.000000}%
\pgfsetstrokecolor{currentstroke}%
\pgfsetstrokeopacity{0.700000}%
\pgfsetdash{}{0pt}%
\pgfpathmoveto{\pgfqpoint{3.873672in}{3.734393in}}%
\pgfpathcurveto{\pgfqpoint{3.886694in}{3.734393in}}{\pgfqpoint{3.899185in}{3.739567in}}{\pgfqpoint{3.908394in}{3.748775in}}%
\pgfpathcurveto{\pgfqpoint{3.917602in}{3.757984in}}{\pgfqpoint{3.922776in}{3.770475in}}{\pgfqpoint{3.922776in}{3.783498in}}%
\pgfpathcurveto{\pgfqpoint{3.922776in}{3.796520in}}{\pgfqpoint{3.917602in}{3.809011in}}{\pgfqpoint{3.908394in}{3.818220in}}%
\pgfpathcurveto{\pgfqpoint{3.899185in}{3.827428in}}{\pgfqpoint{3.886694in}{3.832602in}}{\pgfqpoint{3.873672in}{3.832602in}}%
\pgfpathcurveto{\pgfqpoint{3.860649in}{3.832602in}}{\pgfqpoint{3.848158in}{3.827428in}}{\pgfqpoint{3.838949in}{3.818220in}}%
\pgfpathcurveto{\pgfqpoint{3.829741in}{3.809011in}}{\pgfqpoint{3.824567in}{3.796520in}}{\pgfqpoint{3.824567in}{3.783498in}}%
\pgfpathcurveto{\pgfqpoint{3.824567in}{3.770475in}}{\pgfqpoint{3.829741in}{3.757984in}}{\pgfqpoint{3.838949in}{3.748775in}}%
\pgfpathcurveto{\pgfqpoint{3.848158in}{3.739567in}}{\pgfqpoint{3.860649in}{3.734393in}}{\pgfqpoint{3.873672in}{3.734393in}}%
\pgfpathlineto{\pgfqpoint{3.873672in}{3.734393in}}%
\pgfpathclose%
\pgfusepath{stroke,fill}%
\end{pgfscope}%
\begin{pgfscope}%
\pgfpathrectangle{\pgfqpoint{0.786164in}{0.768110in}}{\pgfqpoint{8.851069in}{7.081890in}}%
\pgfusepath{clip}%
\pgfsetbuttcap%
\pgfsetroundjoin%
\definecolor{currentfill}{rgb}{0.169646,0.456262,0.558030}%
\pgfsetfillcolor{currentfill}%
\pgfsetfillopacity{0.700000}%
\pgfsetlinewidth{0.501875pt}%
\definecolor{currentstroke}{rgb}{1.000000,1.000000,1.000000}%
\pgfsetstrokecolor{currentstroke}%
\pgfsetstrokeopacity{0.700000}%
\pgfsetdash{}{0pt}%
\pgfpathmoveto{\pgfqpoint{3.791472in}{3.997172in}}%
\pgfpathcurveto{\pgfqpoint{3.804495in}{3.997172in}}{\pgfqpoint{3.816986in}{4.002346in}}{\pgfqpoint{3.826194in}{4.011554in}}%
\pgfpathcurveto{\pgfqpoint{3.835403in}{4.020763in}}{\pgfqpoint{3.840577in}{4.033254in}}{\pgfqpoint{3.840577in}{4.046276in}}%
\pgfpathcurveto{\pgfqpoint{3.840577in}{4.059299in}}{\pgfqpoint{3.835403in}{4.071790in}}{\pgfqpoint{3.826194in}{4.080999in}}%
\pgfpathcurveto{\pgfqpoint{3.816986in}{4.090207in}}{\pgfqpoint{3.804495in}{4.095381in}}{\pgfqpoint{3.791472in}{4.095381in}}%
\pgfpathcurveto{\pgfqpoint{3.778449in}{4.095381in}}{\pgfqpoint{3.765958in}{4.090207in}}{\pgfqpoint{3.756750in}{4.080999in}}%
\pgfpathcurveto{\pgfqpoint{3.747541in}{4.071790in}}{\pgfqpoint{3.742367in}{4.059299in}}{\pgfqpoint{3.742367in}{4.046276in}}%
\pgfpathcurveto{\pgfqpoint{3.742367in}{4.033254in}}{\pgfqpoint{3.747541in}{4.020763in}}{\pgfqpoint{3.756750in}{4.011554in}}%
\pgfpathcurveto{\pgfqpoint{3.765958in}{4.002346in}}{\pgfqpoint{3.778449in}{3.997172in}}{\pgfqpoint{3.791472in}{3.997172in}}%
\pgfpathlineto{\pgfqpoint{3.791472in}{3.997172in}}%
\pgfpathclose%
\pgfusepath{stroke,fill}%
\end{pgfscope}%
\begin{pgfscope}%
\pgfpathrectangle{\pgfqpoint{0.786164in}{0.768110in}}{\pgfqpoint{8.851069in}{7.081890in}}%
\pgfusepath{clip}%
\pgfsetbuttcap%
\pgfsetroundjoin%
\definecolor{currentfill}{rgb}{0.169646,0.456262,0.558030}%
\pgfsetfillcolor{currentfill}%
\pgfsetfillopacity{0.700000}%
\pgfsetlinewidth{0.501875pt}%
\definecolor{currentstroke}{rgb}{1.000000,1.000000,1.000000}%
\pgfsetstrokecolor{currentstroke}%
\pgfsetstrokeopacity{0.700000}%
\pgfsetdash{}{0pt}%
\pgfpathmoveto{\pgfqpoint{3.462674in}{4.084765in}}%
\pgfpathcurveto{\pgfqpoint{3.475696in}{4.084765in}}{\pgfqpoint{3.488188in}{4.089939in}}{\pgfqpoint{3.497396in}{4.099147in}}%
\pgfpathcurveto{\pgfqpoint{3.506604in}{4.108356in}}{\pgfqpoint{3.511778in}{4.120847in}}{\pgfqpoint{3.511778in}{4.133869in}}%
\pgfpathcurveto{\pgfqpoint{3.511778in}{4.146892in}}{\pgfqpoint{3.506604in}{4.159383in}}{\pgfqpoint{3.497396in}{4.168592in}}%
\pgfpathcurveto{\pgfqpoint{3.488188in}{4.177800in}}{\pgfqpoint{3.475696in}{4.182974in}}{\pgfqpoint{3.462674in}{4.182974in}}%
\pgfpathcurveto{\pgfqpoint{3.449651in}{4.182974in}}{\pgfqpoint{3.437160in}{4.177800in}}{\pgfqpoint{3.427952in}{4.168592in}}%
\pgfpathcurveto{\pgfqpoint{3.418743in}{4.159383in}}{\pgfqpoint{3.413569in}{4.146892in}}{\pgfqpoint{3.413569in}{4.133869in}}%
\pgfpathcurveto{\pgfqpoint{3.413569in}{4.120847in}}{\pgfqpoint{3.418743in}{4.108356in}}{\pgfqpoint{3.427952in}{4.099147in}}%
\pgfpathcurveto{\pgfqpoint{3.437160in}{4.089939in}}{\pgfqpoint{3.449651in}{4.084765in}}{\pgfqpoint{3.462674in}{4.084765in}}%
\pgfpathlineto{\pgfqpoint{3.462674in}{4.084765in}}%
\pgfpathclose%
\pgfusepath{stroke,fill}%
\end{pgfscope}%
\begin{pgfscope}%
\pgfpathrectangle{\pgfqpoint{0.786164in}{0.768110in}}{\pgfqpoint{8.851069in}{7.081890in}}%
\pgfusepath{clip}%
\pgfsetbuttcap%
\pgfsetroundjoin%
\definecolor{currentfill}{rgb}{0.169646,0.456262,0.558030}%
\pgfsetfillcolor{currentfill}%
\pgfsetfillopacity{0.700000}%
\pgfsetlinewidth{0.501875pt}%
\definecolor{currentstroke}{rgb}{1.000000,1.000000,1.000000}%
\pgfsetstrokecolor{currentstroke}%
\pgfsetstrokeopacity{0.700000}%
\pgfsetdash{}{0pt}%
\pgfpathmoveto{\pgfqpoint{3.837139in}{4.588424in}}%
\pgfpathcurveto{\pgfqpoint{3.850161in}{4.588424in}}{\pgfqpoint{3.862652in}{4.593598in}}{\pgfqpoint{3.871861in}{4.602807in}}%
\pgfpathcurveto{\pgfqpoint{3.881069in}{4.612015in}}{\pgfqpoint{3.886243in}{4.624506in}}{\pgfqpoint{3.886243in}{4.637529in}}%
\pgfpathcurveto{\pgfqpoint{3.886243in}{4.650551in}}{\pgfqpoint{3.881069in}{4.663043in}}{\pgfqpoint{3.871861in}{4.672251in}}%
\pgfpathcurveto{\pgfqpoint{3.862652in}{4.681459in}}{\pgfqpoint{3.850161in}{4.686633in}}{\pgfqpoint{3.837139in}{4.686633in}}%
\pgfpathcurveto{\pgfqpoint{3.824116in}{4.686633in}}{\pgfqpoint{3.811625in}{4.681459in}}{\pgfqpoint{3.802416in}{4.672251in}}%
\pgfpathcurveto{\pgfqpoint{3.793208in}{4.663043in}}{\pgfqpoint{3.788034in}{4.650551in}}{\pgfqpoint{3.788034in}{4.637529in}}%
\pgfpathcurveto{\pgfqpoint{3.788034in}{4.624506in}}{\pgfqpoint{3.793208in}{4.612015in}}{\pgfqpoint{3.802416in}{4.602807in}}%
\pgfpathcurveto{\pgfqpoint{3.811625in}{4.593598in}}{\pgfqpoint{3.824116in}{4.588424in}}{\pgfqpoint{3.837139in}{4.588424in}}%
\pgfpathlineto{\pgfqpoint{3.837139in}{4.588424in}}%
\pgfpathclose%
\pgfusepath{stroke,fill}%
\end{pgfscope}%
\begin{pgfscope}%
\pgfpathrectangle{\pgfqpoint{0.786164in}{0.768110in}}{\pgfqpoint{8.851069in}{7.081890in}}%
\pgfusepath{clip}%
\pgfsetbuttcap%
\pgfsetroundjoin%
\definecolor{currentfill}{rgb}{0.165117,0.467423,0.558141}%
\pgfsetfillcolor{currentfill}%
\pgfsetfillopacity{0.700000}%
\pgfsetlinewidth{0.501875pt}%
\definecolor{currentstroke}{rgb}{1.000000,1.000000,1.000000}%
\pgfsetstrokecolor{currentstroke}%
\pgfsetstrokeopacity{0.700000}%
\pgfsetdash{}{0pt}%
\pgfpathmoveto{\pgfqpoint{3.681873in}{4.588424in}}%
\pgfpathcurveto{\pgfqpoint{3.694895in}{4.588424in}}{\pgfqpoint{3.707386in}{4.593598in}}{\pgfqpoint{3.716595in}{4.602807in}}%
\pgfpathcurveto{\pgfqpoint{3.725803in}{4.612015in}}{\pgfqpoint{3.730977in}{4.624506in}}{\pgfqpoint{3.730977in}{4.637529in}}%
\pgfpathcurveto{\pgfqpoint{3.730977in}{4.650551in}}{\pgfqpoint{3.725803in}{4.663043in}}{\pgfqpoint{3.716595in}{4.672251in}}%
\pgfpathcurveto{\pgfqpoint{3.707386in}{4.681459in}}{\pgfqpoint{3.694895in}{4.686633in}}{\pgfqpoint{3.681873in}{4.686633in}}%
\pgfpathcurveto{\pgfqpoint{3.668850in}{4.686633in}}{\pgfqpoint{3.656359in}{4.681459in}}{\pgfqpoint{3.647150in}{4.672251in}}%
\pgfpathcurveto{\pgfqpoint{3.637942in}{4.663043in}}{\pgfqpoint{3.632768in}{4.650551in}}{\pgfqpoint{3.632768in}{4.637529in}}%
\pgfpathcurveto{\pgfqpoint{3.632768in}{4.624506in}}{\pgfqpoint{3.637942in}{4.612015in}}{\pgfqpoint{3.647150in}{4.602807in}}%
\pgfpathcurveto{\pgfqpoint{3.656359in}{4.593598in}}{\pgfqpoint{3.668850in}{4.588424in}}{\pgfqpoint{3.681873in}{4.588424in}}%
\pgfpathlineto{\pgfqpoint{3.681873in}{4.588424in}}%
\pgfpathclose%
\pgfusepath{stroke,fill}%
\end{pgfscope}%
\begin{pgfscope}%
\pgfpathrectangle{\pgfqpoint{0.786164in}{0.768110in}}{\pgfqpoint{8.851069in}{7.081890in}}%
\pgfusepath{clip}%
\pgfsetbuttcap%
\pgfsetroundjoin%
\definecolor{currentfill}{rgb}{0.149039,0.508051,0.557250}%
\pgfsetfillcolor{currentfill}%
\pgfsetfillopacity{0.700000}%
\pgfsetlinewidth{0.501875pt}%
\definecolor{currentstroke}{rgb}{1.000000,1.000000,1.000000}%
\pgfsetstrokecolor{currentstroke}%
\pgfsetstrokeopacity{0.700000}%
\pgfsetdash{}{0pt}%
\pgfpathmoveto{\pgfqpoint{3.179542in}{4.172358in}}%
\pgfpathcurveto{\pgfqpoint{3.192565in}{4.172358in}}{\pgfqpoint{3.205056in}{4.177532in}}{\pgfqpoint{3.214264in}{4.186740in}}%
\pgfpathcurveto{\pgfqpoint{3.223473in}{4.195948in}}{\pgfqpoint{3.228647in}{4.208440in}}{\pgfqpoint{3.228647in}{4.221462in}}%
\pgfpathcurveto{\pgfqpoint{3.228647in}{4.234485in}}{\pgfqpoint{3.223473in}{4.246976in}}{\pgfqpoint{3.214264in}{4.256184in}}%
\pgfpathcurveto{\pgfqpoint{3.205056in}{4.265393in}}{\pgfqpoint{3.192565in}{4.270567in}}{\pgfqpoint{3.179542in}{4.270567in}}%
\pgfpathcurveto{\pgfqpoint{3.166519in}{4.270567in}}{\pgfqpoint{3.154028in}{4.265393in}}{\pgfqpoint{3.144820in}{4.256184in}}%
\pgfpathcurveto{\pgfqpoint{3.135611in}{4.246976in}}{\pgfqpoint{3.130437in}{4.234485in}}{\pgfqpoint{3.130437in}{4.221462in}}%
\pgfpathcurveto{\pgfqpoint{3.130437in}{4.208440in}}{\pgfqpoint{3.135611in}{4.195948in}}{\pgfqpoint{3.144820in}{4.186740in}}%
\pgfpathcurveto{\pgfqpoint{3.154028in}{4.177532in}}{\pgfqpoint{3.166519in}{4.172358in}}{\pgfqpoint{3.179542in}{4.172358in}}%
\pgfpathlineto{\pgfqpoint{3.179542in}{4.172358in}}%
\pgfpathclose%
\pgfusepath{stroke,fill}%
\end{pgfscope}%
\begin{pgfscope}%
\pgfpathrectangle{\pgfqpoint{0.786164in}{0.768110in}}{\pgfqpoint{8.851069in}{7.081890in}}%
\pgfusepath{clip}%
\pgfsetbuttcap%
\pgfsetroundjoin%
\definecolor{currentfill}{rgb}{0.147607,0.511733,0.557049}%
\pgfsetfillcolor{currentfill}%
\pgfsetfillopacity{0.700000}%
\pgfsetlinewidth{0.501875pt}%
\definecolor{currentstroke}{rgb}{1.000000,1.000000,1.000000}%
\pgfsetstrokecolor{currentstroke}%
\pgfsetstrokeopacity{0.700000}%
\pgfsetdash{}{0pt}%
\pgfpathmoveto{\pgfqpoint{4.010671in}{4.610322in}}%
\pgfpathcurveto{\pgfqpoint{4.023694in}{4.610322in}}{\pgfqpoint{4.036185in}{4.615496in}}{\pgfqpoint{4.045393in}{4.624705in}}%
\pgfpathcurveto{\pgfqpoint{4.054602in}{4.633913in}}{\pgfqpoint{4.059776in}{4.646404in}}{\pgfqpoint{4.059776in}{4.659427in}}%
\pgfpathcurveto{\pgfqpoint{4.059776in}{4.672450in}}{\pgfqpoint{4.054602in}{4.684941in}}{\pgfqpoint{4.045393in}{4.694149in}}%
\pgfpathcurveto{\pgfqpoint{4.036185in}{4.703358in}}{\pgfqpoint{4.023694in}{4.708532in}}{\pgfqpoint{4.010671in}{4.708532in}}%
\pgfpathcurveto{\pgfqpoint{3.997648in}{4.708532in}}{\pgfqpoint{3.985157in}{4.703358in}}{\pgfqpoint{3.975949in}{4.694149in}}%
\pgfpathcurveto{\pgfqpoint{3.966740in}{4.684941in}}{\pgfqpoint{3.961566in}{4.672450in}}{\pgfqpoint{3.961566in}{4.659427in}}%
\pgfpathcurveto{\pgfqpoint{3.961566in}{4.646404in}}{\pgfqpoint{3.966740in}{4.633913in}}{\pgfqpoint{3.975949in}{4.624705in}}%
\pgfpathcurveto{\pgfqpoint{3.985157in}{4.615496in}}{\pgfqpoint{3.997648in}{4.610322in}}{\pgfqpoint{4.010671in}{4.610322in}}%
\pgfpathlineto{\pgfqpoint{4.010671in}{4.610322in}}%
\pgfpathclose%
\pgfusepath{stroke,fill}%
\end{pgfscope}%
\begin{pgfscope}%
\pgfpathrectangle{\pgfqpoint{0.786164in}{0.768110in}}{\pgfqpoint{8.851069in}{7.081890in}}%
\pgfusepath{clip}%
\pgfsetbuttcap%
\pgfsetroundjoin%
\definecolor{currentfill}{rgb}{0.146180,0.515413,0.556823}%
\pgfsetfillcolor{currentfill}%
\pgfsetfillopacity{0.700000}%
\pgfsetlinewidth{0.501875pt}%
\definecolor{currentstroke}{rgb}{1.000000,1.000000,1.000000}%
\pgfsetstrokecolor{currentstroke}%
\pgfsetstrokeopacity{0.700000}%
\pgfsetdash{}{0pt}%
\pgfpathmoveto{\pgfqpoint{3.882805in}{4.785508in}}%
\pgfpathcurveto{\pgfqpoint{3.895828in}{4.785508in}}{\pgfqpoint{3.908319in}{4.790682in}}{\pgfqpoint{3.917527in}{4.799891in}}%
\pgfpathcurveto{\pgfqpoint{3.926736in}{4.809099in}}{\pgfqpoint{3.931910in}{4.821590in}}{\pgfqpoint{3.931910in}{4.834613in}}%
\pgfpathcurveto{\pgfqpoint{3.931910in}{4.847636in}}{\pgfqpoint{3.926736in}{4.860127in}}{\pgfqpoint{3.917527in}{4.869335in}}%
\pgfpathcurveto{\pgfqpoint{3.908319in}{4.878544in}}{\pgfqpoint{3.895828in}{4.883718in}}{\pgfqpoint{3.882805in}{4.883718in}}%
\pgfpathcurveto{\pgfqpoint{3.869782in}{4.883718in}}{\pgfqpoint{3.857291in}{4.878544in}}{\pgfqpoint{3.848083in}{4.869335in}}%
\pgfpathcurveto{\pgfqpoint{3.838874in}{4.860127in}}{\pgfqpoint{3.833700in}{4.847636in}}{\pgfqpoint{3.833700in}{4.834613in}}%
\pgfpathcurveto{\pgfqpoint{3.833700in}{4.821590in}}{\pgfqpoint{3.838874in}{4.809099in}}{\pgfqpoint{3.848083in}{4.799891in}}%
\pgfpathcurveto{\pgfqpoint{3.857291in}{4.790682in}}{\pgfqpoint{3.869782in}{4.785508in}}{\pgfqpoint{3.882805in}{4.785508in}}%
\pgfpathlineto{\pgfqpoint{3.882805in}{4.785508in}}%
\pgfpathclose%
\pgfusepath{stroke,fill}%
\end{pgfscope}%
\begin{pgfscope}%
\pgfpathrectangle{\pgfqpoint{0.786164in}{0.768110in}}{\pgfqpoint{8.851069in}{7.081890in}}%
\pgfusepath{clip}%
\pgfsetbuttcap%
\pgfsetroundjoin%
\definecolor{currentfill}{rgb}{0.143343,0.522773,0.556295}%
\pgfsetfillcolor{currentfill}%
\pgfsetfillopacity{0.700000}%
\pgfsetlinewidth{0.501875pt}%
\definecolor{currentstroke}{rgb}{1.000000,1.000000,1.000000}%
\pgfsetstrokecolor{currentstroke}%
\pgfsetstrokeopacity{0.700000}%
\pgfsetdash{}{0pt}%
\pgfpathmoveto{\pgfqpoint{3.745806in}{4.938796in}}%
\pgfpathcurveto{\pgfqpoint{3.758828in}{4.938796in}}{\pgfqpoint{3.771319in}{4.943970in}}{\pgfqpoint{3.780528in}{4.953178in}}%
\pgfpathcurveto{\pgfqpoint{3.789736in}{4.962387in}}{\pgfqpoint{3.794910in}{4.974878in}}{\pgfqpoint{3.794910in}{4.987901in}}%
\pgfpathcurveto{\pgfqpoint{3.794910in}{5.000923in}}{\pgfqpoint{3.789736in}{5.013414in}}{\pgfqpoint{3.780528in}{5.022623in}}%
\pgfpathcurveto{\pgfqpoint{3.771319in}{5.031831in}}{\pgfqpoint{3.758828in}{5.037005in}}{\pgfqpoint{3.745806in}{5.037005in}}%
\pgfpathcurveto{\pgfqpoint{3.732783in}{5.037005in}}{\pgfqpoint{3.720292in}{5.031831in}}{\pgfqpoint{3.711083in}{5.022623in}}%
\pgfpathcurveto{\pgfqpoint{3.701875in}{5.013414in}}{\pgfqpoint{3.696701in}{5.000923in}}{\pgfqpoint{3.696701in}{4.987901in}}%
\pgfpathcurveto{\pgfqpoint{3.696701in}{4.974878in}}{\pgfqpoint{3.701875in}{4.962387in}}{\pgfqpoint{3.711083in}{4.953178in}}%
\pgfpathcurveto{\pgfqpoint{3.720292in}{4.943970in}}{\pgfqpoint{3.732783in}{4.938796in}}{\pgfqpoint{3.745806in}{4.938796in}}%
\pgfpathlineto{\pgfqpoint{3.745806in}{4.938796in}}%
\pgfpathclose%
\pgfusepath{stroke,fill}%
\end{pgfscope}%
\begin{pgfscope}%
\pgfpathrectangle{\pgfqpoint{0.786164in}{0.768110in}}{\pgfqpoint{8.851069in}{7.081890in}}%
\pgfusepath{clip}%
\pgfsetbuttcap%
\pgfsetroundjoin%
\definecolor{currentfill}{rgb}{0.135066,0.544853,0.554029}%
\pgfsetfillcolor{currentfill}%
\pgfsetfillopacity{0.700000}%
\pgfsetlinewidth{0.501875pt}%
\definecolor{currentstroke}{rgb}{1.000000,1.000000,1.000000}%
\pgfsetstrokecolor{currentstroke}%
\pgfsetstrokeopacity{0.700000}%
\pgfsetdash{}{0pt}%
\pgfpathmoveto{\pgfqpoint{3.170409in}{3.997172in}}%
\pgfpathcurveto{\pgfqpoint{3.183431in}{3.997172in}}{\pgfqpoint{3.195922in}{4.002346in}}{\pgfqpoint{3.205131in}{4.011554in}}%
\pgfpathcurveto{\pgfqpoint{3.214339in}{4.020763in}}{\pgfqpoint{3.219513in}{4.033254in}}{\pgfqpoint{3.219513in}{4.046276in}}%
\pgfpathcurveto{\pgfqpoint{3.219513in}{4.059299in}}{\pgfqpoint{3.214339in}{4.071790in}}{\pgfqpoint{3.205131in}{4.080999in}}%
\pgfpathcurveto{\pgfqpoint{3.195922in}{4.090207in}}{\pgfqpoint{3.183431in}{4.095381in}}{\pgfqpoint{3.170409in}{4.095381in}}%
\pgfpathcurveto{\pgfqpoint{3.157386in}{4.095381in}}{\pgfqpoint{3.144895in}{4.090207in}}{\pgfqpoint{3.135686in}{4.080999in}}%
\pgfpathcurveto{\pgfqpoint{3.126478in}{4.071790in}}{\pgfqpoint{3.121304in}{4.059299in}}{\pgfqpoint{3.121304in}{4.046276in}}%
\pgfpathcurveto{\pgfqpoint{3.121304in}{4.033254in}}{\pgfqpoint{3.126478in}{4.020763in}}{\pgfqpoint{3.135686in}{4.011554in}}%
\pgfpathcurveto{\pgfqpoint{3.144895in}{4.002346in}}{\pgfqpoint{3.157386in}{3.997172in}}{\pgfqpoint{3.170409in}{3.997172in}}%
\pgfpathlineto{\pgfqpoint{3.170409in}{3.997172in}}%
\pgfpathclose%
\pgfusepath{stroke,fill}%
\end{pgfscope}%
\begin{pgfscope}%
\pgfpathrectangle{\pgfqpoint{0.786164in}{0.768110in}}{\pgfqpoint{8.851069in}{7.081890in}}%
\pgfusepath{clip}%
\pgfsetbuttcap%
\pgfsetroundjoin%
\definecolor{currentfill}{rgb}{0.143343,0.522773,0.556295}%
\pgfsetfillcolor{currentfill}%
\pgfsetfillopacity{0.700000}%
\pgfsetlinewidth{0.501875pt}%
\definecolor{currentstroke}{rgb}{1.000000,1.000000,1.000000}%
\pgfsetstrokecolor{currentstroke}%
\pgfsetstrokeopacity{0.700000}%
\pgfsetdash{}{0pt}%
\pgfpathmoveto{\pgfqpoint{2.841610in}{3.099344in}}%
\pgfpathcurveto{\pgfqpoint{2.854633in}{3.099344in}}{\pgfqpoint{2.867124in}{3.104518in}}{\pgfqpoint{2.876332in}{3.113726in}}%
\pgfpathcurveto{\pgfqpoint{2.885541in}{3.122935in}}{\pgfqpoint{2.890715in}{3.135426in}}{\pgfqpoint{2.890715in}{3.148449in}}%
\pgfpathcurveto{\pgfqpoint{2.890715in}{3.161471in}}{\pgfqpoint{2.885541in}{3.173962in}}{\pgfqpoint{2.876332in}{3.183171in}}%
\pgfpathcurveto{\pgfqpoint{2.867124in}{3.192379in}}{\pgfqpoint{2.854633in}{3.197553in}}{\pgfqpoint{2.841610in}{3.197553in}}%
\pgfpathcurveto{\pgfqpoint{2.828588in}{3.197553in}}{\pgfqpoint{2.816096in}{3.192379in}}{\pgfqpoint{2.806888in}{3.183171in}}%
\pgfpathcurveto{\pgfqpoint{2.797680in}{3.173962in}}{\pgfqpoint{2.792506in}{3.161471in}}{\pgfqpoint{2.792506in}{3.148449in}}%
\pgfpathcurveto{\pgfqpoint{2.792506in}{3.135426in}}{\pgfqpoint{2.797680in}{3.122935in}}{\pgfqpoint{2.806888in}{3.113726in}}%
\pgfpathcurveto{\pgfqpoint{2.816096in}{3.104518in}}{\pgfqpoint{2.828588in}{3.099344in}}{\pgfqpoint{2.841610in}{3.099344in}}%
\pgfpathlineto{\pgfqpoint{2.841610in}{3.099344in}}%
\pgfpathclose%
\pgfusepath{stroke,fill}%
\end{pgfscope}%
\begin{pgfscope}%
\pgfpathrectangle{\pgfqpoint{0.786164in}{0.768110in}}{\pgfqpoint{8.851069in}{7.081890in}}%
\pgfusepath{clip}%
\pgfsetbuttcap%
\pgfsetroundjoin%
\definecolor{currentfill}{rgb}{0.121380,0.629492,0.531973}%
\pgfsetfillcolor{currentfill}%
\pgfsetfillopacity{0.700000}%
\pgfsetlinewidth{0.501875pt}%
\definecolor{currentstroke}{rgb}{1.000000,1.000000,1.000000}%
\pgfsetstrokecolor{currentstroke}%
\pgfsetstrokeopacity{0.700000}%
\pgfsetdash{}{0pt}%
\pgfpathmoveto{\pgfqpoint{2.923810in}{3.252632in}}%
\pgfpathcurveto{\pgfqpoint{2.936833in}{3.252632in}}{\pgfqpoint{2.949324in}{3.257806in}}{\pgfqpoint{2.958532in}{3.267014in}}%
\pgfpathcurveto{\pgfqpoint{2.967740in}{3.276223in}}{\pgfqpoint{2.972914in}{3.288714in}}{\pgfqpoint{2.972914in}{3.301736in}}%
\pgfpathcurveto{\pgfqpoint{2.972914in}{3.314759in}}{\pgfqpoint{2.967740in}{3.327250in}}{\pgfqpoint{2.958532in}{3.336459in}}%
\pgfpathcurveto{\pgfqpoint{2.949324in}{3.345667in}}{\pgfqpoint{2.936833in}{3.350841in}}{\pgfqpoint{2.923810in}{3.350841in}}%
\pgfpathcurveto{\pgfqpoint{2.910787in}{3.350841in}}{\pgfqpoint{2.898296in}{3.345667in}}{\pgfqpoint{2.889088in}{3.336459in}}%
\pgfpathcurveto{\pgfqpoint{2.879879in}{3.327250in}}{\pgfqpoint{2.874705in}{3.314759in}}{\pgfqpoint{2.874705in}{3.301736in}}%
\pgfpathcurveto{\pgfqpoint{2.874705in}{3.288714in}}{\pgfqpoint{2.879879in}{3.276223in}}{\pgfqpoint{2.889088in}{3.267014in}}%
\pgfpathcurveto{\pgfqpoint{2.898296in}{3.257806in}}{\pgfqpoint{2.910787in}{3.252632in}}{\pgfqpoint{2.923810in}{3.252632in}}%
\pgfpathlineto{\pgfqpoint{2.923810in}{3.252632in}}%
\pgfpathclose%
\pgfusepath{stroke,fill}%
\end{pgfscope}%
\begin{pgfscope}%
\pgfpathrectangle{\pgfqpoint{0.786164in}{0.768110in}}{\pgfqpoint{8.851069in}{7.081890in}}%
\pgfusepath{clip}%
\pgfsetbuttcap%
\pgfsetroundjoin%
\definecolor{currentfill}{rgb}{0.126326,0.644107,0.525311}%
\pgfsetfillcolor{currentfill}%
\pgfsetfillopacity{0.700000}%
\pgfsetlinewidth{0.501875pt}%
\definecolor{currentstroke}{rgb}{1.000000,1.000000,1.000000}%
\pgfsetstrokecolor{currentstroke}%
\pgfsetstrokeopacity{0.700000}%
\pgfsetdash{}{0pt}%
\pgfpathmoveto{\pgfqpoint{3.069942in}{3.362123in}}%
\pgfpathcurveto{\pgfqpoint{3.082965in}{3.362123in}}{\pgfqpoint{3.095456in}{3.367297in}}{\pgfqpoint{3.104665in}{3.376505in}}%
\pgfpathcurveto{\pgfqpoint{3.113873in}{3.385714in}}{\pgfqpoint{3.119047in}{3.398205in}}{\pgfqpoint{3.119047in}{3.411228in}}%
\pgfpathcurveto{\pgfqpoint{3.119047in}{3.424250in}}{\pgfqpoint{3.113873in}{3.436741in}}{\pgfqpoint{3.104665in}{3.445950in}}%
\pgfpathcurveto{\pgfqpoint{3.095456in}{3.455158in}}{\pgfqpoint{3.082965in}{3.460332in}}{\pgfqpoint{3.069942in}{3.460332in}}%
\pgfpathcurveto{\pgfqpoint{3.056920in}{3.460332in}}{\pgfqpoint{3.044429in}{3.455158in}}{\pgfqpoint{3.035220in}{3.445950in}}%
\pgfpathcurveto{\pgfqpoint{3.026012in}{3.436741in}}{\pgfqpoint{3.020838in}{3.424250in}}{\pgfqpoint{3.020838in}{3.411228in}}%
\pgfpathcurveto{\pgfqpoint{3.020838in}{3.398205in}}{\pgfqpoint{3.026012in}{3.385714in}}{\pgfqpoint{3.035220in}{3.376505in}}%
\pgfpathcurveto{\pgfqpoint{3.044429in}{3.367297in}}{\pgfqpoint{3.056920in}{3.362123in}}{\pgfqpoint{3.069942in}{3.362123in}}%
\pgfpathlineto{\pgfqpoint{3.069942in}{3.362123in}}%
\pgfpathclose%
\pgfusepath{stroke,fill}%
\end{pgfscope}%
\begin{pgfscope}%
\pgfpathrectangle{\pgfqpoint{0.786164in}{0.768110in}}{\pgfqpoint{8.851069in}{7.081890in}}%
\pgfusepath{clip}%
\pgfsetbuttcap%
\pgfsetroundjoin%
\definecolor{currentfill}{rgb}{0.281887,0.150881,0.465405}%
\pgfsetfillcolor{currentfill}%
\pgfsetfillopacity{0.700000}%
\pgfsetlinewidth{0.501875pt}%
\definecolor{currentstroke}{rgb}{1.000000,1.000000,1.000000}%
\pgfsetstrokecolor{currentstroke}%
\pgfsetstrokeopacity{0.700000}%
\pgfsetdash{}{0pt}%
\pgfpathmoveto{\pgfqpoint{2.348413in}{3.668698in}}%
\pgfpathcurveto{\pgfqpoint{2.361435in}{3.668698in}}{\pgfqpoint{2.373926in}{3.673872in}}{\pgfqpoint{2.383135in}{3.683081in}}%
\pgfpathcurveto{\pgfqpoint{2.392343in}{3.692289in}}{\pgfqpoint{2.397517in}{3.704780in}}{\pgfqpoint{2.397517in}{3.717803in}}%
\pgfpathcurveto{\pgfqpoint{2.397517in}{3.730826in}}{\pgfqpoint{2.392343in}{3.743317in}}{\pgfqpoint{2.383135in}{3.752525in}}%
\pgfpathcurveto{\pgfqpoint{2.373926in}{3.761733in}}{\pgfqpoint{2.361435in}{3.766907in}}{\pgfqpoint{2.348413in}{3.766907in}}%
\pgfpathcurveto{\pgfqpoint{2.335390in}{3.766907in}}{\pgfqpoint{2.322899in}{3.761733in}}{\pgfqpoint{2.313690in}{3.752525in}}%
\pgfpathcurveto{\pgfqpoint{2.304482in}{3.743317in}}{\pgfqpoint{2.299308in}{3.730826in}}{\pgfqpoint{2.299308in}{3.717803in}}%
\pgfpathcurveto{\pgfqpoint{2.299308in}{3.704780in}}{\pgfqpoint{2.304482in}{3.692289in}}{\pgfqpoint{2.313690in}{3.683081in}}%
\pgfpathcurveto{\pgfqpoint{2.322899in}{3.673872in}}{\pgfqpoint{2.335390in}{3.668698in}}{\pgfqpoint{2.348413in}{3.668698in}}%
\pgfpathlineto{\pgfqpoint{2.348413in}{3.668698in}}%
\pgfpathclose%
\pgfusepath{stroke,fill}%
\end{pgfscope}%
\begin{pgfscope}%
\pgfpathrectangle{\pgfqpoint{0.786164in}{0.768110in}}{\pgfqpoint{8.851069in}{7.081890in}}%
\pgfusepath{clip}%
\pgfsetbuttcap%
\pgfsetroundjoin%
\definecolor{currentfill}{rgb}{0.281887,0.150881,0.465405}%
\pgfsetfillcolor{currentfill}%
\pgfsetfillopacity{0.700000}%
\pgfsetlinewidth{0.501875pt}%
\definecolor{currentstroke}{rgb}{1.000000,1.000000,1.000000}%
\pgfsetstrokecolor{currentstroke}%
\pgfsetstrokeopacity{0.700000}%
\pgfsetdash{}{0pt}%
\pgfpathmoveto{\pgfqpoint{2.394079in}{3.734393in}}%
\pgfpathcurveto{\pgfqpoint{2.407102in}{3.734393in}}{\pgfqpoint{2.419593in}{3.739567in}}{\pgfqpoint{2.428801in}{3.748775in}}%
\pgfpathcurveto{\pgfqpoint{2.438010in}{3.757984in}}{\pgfqpoint{2.443184in}{3.770475in}}{\pgfqpoint{2.443184in}{3.783498in}}%
\pgfpathcurveto{\pgfqpoint{2.443184in}{3.796520in}}{\pgfqpoint{2.438010in}{3.809011in}}{\pgfqpoint{2.428801in}{3.818220in}}%
\pgfpathcurveto{\pgfqpoint{2.419593in}{3.827428in}}{\pgfqpoint{2.407102in}{3.832602in}}{\pgfqpoint{2.394079in}{3.832602in}}%
\pgfpathcurveto{\pgfqpoint{2.381056in}{3.832602in}}{\pgfqpoint{2.368565in}{3.827428in}}{\pgfqpoint{2.359357in}{3.818220in}}%
\pgfpathcurveto{\pgfqpoint{2.350148in}{3.809011in}}{\pgfqpoint{2.344975in}{3.796520in}}{\pgfqpoint{2.344975in}{3.783498in}}%
\pgfpathcurveto{\pgfqpoint{2.344975in}{3.770475in}}{\pgfqpoint{2.350148in}{3.757984in}}{\pgfqpoint{2.359357in}{3.748775in}}%
\pgfpathcurveto{\pgfqpoint{2.368565in}{3.739567in}}{\pgfqpoint{2.381056in}{3.734393in}}{\pgfqpoint{2.394079in}{3.734393in}}%
\pgfpathlineto{\pgfqpoint{2.394079in}{3.734393in}}%
\pgfpathclose%
\pgfusepath{stroke,fill}%
\end{pgfscope}%
\begin{pgfscope}%
\pgfpathrectangle{\pgfqpoint{0.786164in}{0.768110in}}{\pgfqpoint{8.851069in}{7.081890in}}%
\pgfusepath{clip}%
\pgfsetbuttcap%
\pgfsetroundjoin%
\definecolor{currentfill}{rgb}{0.281412,0.155834,0.469201}%
\pgfsetfillcolor{currentfill}%
\pgfsetfillopacity{0.700000}%
\pgfsetlinewidth{0.501875pt}%
\definecolor{currentstroke}{rgb}{1.000000,1.000000,1.000000}%
\pgfsetstrokecolor{currentstroke}%
\pgfsetstrokeopacity{0.700000}%
\pgfsetdash{}{0pt}%
\pgfpathmoveto{\pgfqpoint{2.384946in}{3.646800in}}%
\pgfpathcurveto{\pgfqpoint{2.397969in}{3.646800in}}{\pgfqpoint{2.410460in}{3.651974in}}{\pgfqpoint{2.419668in}{3.661182in}}%
\pgfpathcurveto{\pgfqpoint{2.428877in}{3.670391in}}{\pgfqpoint{2.434050in}{3.682882in}}{\pgfqpoint{2.434050in}{3.695905in}}%
\pgfpathcurveto{\pgfqpoint{2.434050in}{3.708927in}}{\pgfqpoint{2.428877in}{3.721418in}}{\pgfqpoint{2.419668in}{3.730627in}}%
\pgfpathcurveto{\pgfqpoint{2.410460in}{3.739835in}}{\pgfqpoint{2.397969in}{3.745009in}}{\pgfqpoint{2.384946in}{3.745009in}}%
\pgfpathcurveto{\pgfqpoint{2.371923in}{3.745009in}}{\pgfqpoint{2.359432in}{3.739835in}}{\pgfqpoint{2.350224in}{3.730627in}}%
\pgfpathcurveto{\pgfqpoint{2.341015in}{3.721418in}}{\pgfqpoint{2.335841in}{3.708927in}}{\pgfqpoint{2.335841in}{3.695905in}}%
\pgfpathcurveto{\pgfqpoint{2.335841in}{3.682882in}}{\pgfqpoint{2.341015in}{3.670391in}}{\pgfqpoint{2.350224in}{3.661182in}}%
\pgfpathcurveto{\pgfqpoint{2.359432in}{3.651974in}}{\pgfqpoint{2.371923in}{3.646800in}}{\pgfqpoint{2.384946in}{3.646800in}}%
\pgfpathlineto{\pgfqpoint{2.384946in}{3.646800in}}%
\pgfpathclose%
\pgfusepath{stroke,fill}%
\end{pgfscope}%
\begin{pgfscope}%
\pgfpathrectangle{\pgfqpoint{0.786164in}{0.768110in}}{\pgfqpoint{8.851069in}{7.081890in}}%
\pgfusepath{clip}%
\pgfsetbuttcap%
\pgfsetroundjoin%
\definecolor{currentfill}{rgb}{0.279574,0.170599,0.479997}%
\pgfsetfillcolor{currentfill}%
\pgfsetfillopacity{0.700000}%
\pgfsetlinewidth{0.501875pt}%
\definecolor{currentstroke}{rgb}{1.000000,1.000000,1.000000}%
\pgfsetstrokecolor{currentstroke}%
\pgfsetstrokeopacity{0.700000}%
\pgfsetdash{}{0pt}%
\pgfpathmoveto{\pgfqpoint{2.375813in}{3.734393in}}%
\pgfpathcurveto{\pgfqpoint{2.388835in}{3.734393in}}{\pgfqpoint{2.401326in}{3.739567in}}{\pgfqpoint{2.410535in}{3.748775in}}%
\pgfpathcurveto{\pgfqpoint{2.419743in}{3.757984in}}{\pgfqpoint{2.424917in}{3.770475in}}{\pgfqpoint{2.424917in}{3.783498in}}%
\pgfpathcurveto{\pgfqpoint{2.424917in}{3.796520in}}{\pgfqpoint{2.419743in}{3.809011in}}{\pgfqpoint{2.410535in}{3.818220in}}%
\pgfpathcurveto{\pgfqpoint{2.401326in}{3.827428in}}{\pgfqpoint{2.388835in}{3.832602in}}{\pgfqpoint{2.375813in}{3.832602in}}%
\pgfpathcurveto{\pgfqpoint{2.362790in}{3.832602in}}{\pgfqpoint{2.350299in}{3.827428in}}{\pgfqpoint{2.341090in}{3.818220in}}%
\pgfpathcurveto{\pgfqpoint{2.331882in}{3.809011in}}{\pgfqpoint{2.326708in}{3.796520in}}{\pgfqpoint{2.326708in}{3.783498in}}%
\pgfpathcurveto{\pgfqpoint{2.326708in}{3.770475in}}{\pgfqpoint{2.331882in}{3.757984in}}{\pgfqpoint{2.341090in}{3.748775in}}%
\pgfpathcurveto{\pgfqpoint{2.350299in}{3.739567in}}{\pgfqpoint{2.362790in}{3.734393in}}{\pgfqpoint{2.375813in}{3.734393in}}%
\pgfpathlineto{\pgfqpoint{2.375813in}{3.734393in}}%
\pgfpathclose%
\pgfusepath{stroke,fill}%
\end{pgfscope}%
\begin{pgfscope}%
\pgfpathrectangle{\pgfqpoint{0.786164in}{0.768110in}}{\pgfqpoint{8.851069in}{7.081890in}}%
\pgfusepath{clip}%
\pgfsetbuttcap%
\pgfsetroundjoin%
\definecolor{currentfill}{rgb}{0.279574,0.170599,0.479997}%
\pgfsetfillcolor{currentfill}%
\pgfsetfillopacity{0.700000}%
\pgfsetlinewidth{0.501875pt}%
\definecolor{currentstroke}{rgb}{1.000000,1.000000,1.000000}%
\pgfsetstrokecolor{currentstroke}%
\pgfsetstrokeopacity{0.700000}%
\pgfsetdash{}{0pt}%
\pgfpathmoveto{\pgfqpoint{2.384946in}{3.624902in}}%
\pgfpathcurveto{\pgfqpoint{2.397969in}{3.624902in}}{\pgfqpoint{2.410460in}{3.630076in}}{\pgfqpoint{2.419668in}{3.639284in}}%
\pgfpathcurveto{\pgfqpoint{2.428877in}{3.648493in}}{\pgfqpoint{2.434050in}{3.660984in}}{\pgfqpoint{2.434050in}{3.674006in}}%
\pgfpathcurveto{\pgfqpoint{2.434050in}{3.687029in}}{\pgfqpoint{2.428877in}{3.699520in}}{\pgfqpoint{2.419668in}{3.708729in}}%
\pgfpathcurveto{\pgfqpoint{2.410460in}{3.717937in}}{\pgfqpoint{2.397969in}{3.723111in}}{\pgfqpoint{2.384946in}{3.723111in}}%
\pgfpathcurveto{\pgfqpoint{2.371923in}{3.723111in}}{\pgfqpoint{2.359432in}{3.717937in}}{\pgfqpoint{2.350224in}{3.708729in}}%
\pgfpathcurveto{\pgfqpoint{2.341015in}{3.699520in}}{\pgfqpoint{2.335841in}{3.687029in}}{\pgfqpoint{2.335841in}{3.674006in}}%
\pgfpathcurveto{\pgfqpoint{2.335841in}{3.660984in}}{\pgfqpoint{2.341015in}{3.648493in}}{\pgfqpoint{2.350224in}{3.639284in}}%
\pgfpathcurveto{\pgfqpoint{2.359432in}{3.630076in}}{\pgfqpoint{2.371923in}{3.624902in}}{\pgfqpoint{2.384946in}{3.624902in}}%
\pgfpathlineto{\pgfqpoint{2.384946in}{3.624902in}}%
\pgfpathclose%
\pgfusepath{stroke,fill}%
\end{pgfscope}%
\begin{pgfscope}%
\pgfpathrectangle{\pgfqpoint{0.786164in}{0.768110in}}{\pgfqpoint{8.851069in}{7.081890in}}%
\pgfusepath{clip}%
\pgfsetbuttcap%
\pgfsetroundjoin%
\definecolor{currentfill}{rgb}{0.278012,0.180367,0.486697}%
\pgfsetfillcolor{currentfill}%
\pgfsetfillopacity{0.700000}%
\pgfsetlinewidth{0.501875pt}%
\definecolor{currentstroke}{rgb}{1.000000,1.000000,1.000000}%
\pgfsetstrokecolor{currentstroke}%
\pgfsetstrokeopacity{0.700000}%
\pgfsetdash{}{0pt}%
\pgfpathmoveto{\pgfqpoint{2.293613in}{3.471614in}}%
\pgfpathcurveto{\pgfqpoint{2.306636in}{3.471614in}}{\pgfqpoint{2.319127in}{3.476788in}}{\pgfqpoint{2.328335in}{3.485996in}}%
\pgfpathcurveto{\pgfqpoint{2.337544in}{3.495205in}}{\pgfqpoint{2.342718in}{3.507696in}}{\pgfqpoint{2.342718in}{3.520719in}}%
\pgfpathcurveto{\pgfqpoint{2.342718in}{3.533741in}}{\pgfqpoint{2.337544in}{3.546232in}}{\pgfqpoint{2.328335in}{3.555441in}}%
\pgfpathcurveto{\pgfqpoint{2.319127in}{3.564649in}}{\pgfqpoint{2.306636in}{3.569823in}}{\pgfqpoint{2.293613in}{3.569823in}}%
\pgfpathcurveto{\pgfqpoint{2.280590in}{3.569823in}}{\pgfqpoint{2.268099in}{3.564649in}}{\pgfqpoint{2.258891in}{3.555441in}}%
\pgfpathcurveto{\pgfqpoint{2.249682in}{3.546232in}}{\pgfqpoint{2.244508in}{3.533741in}}{\pgfqpoint{2.244508in}{3.520719in}}%
\pgfpathcurveto{\pgfqpoint{2.244508in}{3.507696in}}{\pgfqpoint{2.249682in}{3.495205in}}{\pgfqpoint{2.258891in}{3.485996in}}%
\pgfpathcurveto{\pgfqpoint{2.268099in}{3.476788in}}{\pgfqpoint{2.280590in}{3.471614in}}{\pgfqpoint{2.293613in}{3.471614in}}%
\pgfpathlineto{\pgfqpoint{2.293613in}{3.471614in}}%
\pgfpathclose%
\pgfusepath{stroke,fill}%
\end{pgfscope}%
\begin{pgfscope}%
\pgfpathrectangle{\pgfqpoint{0.786164in}{0.768110in}}{\pgfqpoint{8.851069in}{7.081890in}}%
\pgfusepath{clip}%
\pgfsetbuttcap%
\pgfsetroundjoin%
\definecolor{currentfill}{rgb}{0.273006,0.204520,0.501721}%
\pgfsetfillcolor{currentfill}%
\pgfsetfillopacity{0.700000}%
\pgfsetlinewidth{0.501875pt}%
\definecolor{currentstroke}{rgb}{1.000000,1.000000,1.000000}%
\pgfsetstrokecolor{currentstroke}%
\pgfsetstrokeopacity{0.700000}%
\pgfsetdash{}{0pt}%
\pgfpathmoveto{\pgfqpoint{2.110947in}{3.340225in}}%
\pgfpathcurveto{\pgfqpoint{2.123970in}{3.340225in}}{\pgfqpoint{2.136461in}{3.345399in}}{\pgfqpoint{2.145669in}{3.354607in}}%
\pgfpathcurveto{\pgfqpoint{2.154878in}{3.363816in}}{\pgfqpoint{2.160052in}{3.376307in}}{\pgfqpoint{2.160052in}{3.389329in}}%
\pgfpathcurveto{\pgfqpoint{2.160052in}{3.402352in}}{\pgfqpoint{2.154878in}{3.414843in}}{\pgfqpoint{2.145669in}{3.424052in}}%
\pgfpathcurveto{\pgfqpoint{2.136461in}{3.433260in}}{\pgfqpoint{2.123970in}{3.438434in}}{\pgfqpoint{2.110947in}{3.438434in}}%
\pgfpathcurveto{\pgfqpoint{2.097925in}{3.438434in}}{\pgfqpoint{2.085433in}{3.433260in}}{\pgfqpoint{2.076225in}{3.424052in}}%
\pgfpathcurveto{\pgfqpoint{2.067017in}{3.414843in}}{\pgfqpoint{2.061843in}{3.402352in}}{\pgfqpoint{2.061843in}{3.389329in}}%
\pgfpathcurveto{\pgfqpoint{2.061843in}{3.376307in}}{\pgfqpoint{2.067017in}{3.363816in}}{\pgfqpoint{2.076225in}{3.354607in}}%
\pgfpathcurveto{\pgfqpoint{2.085433in}{3.345399in}}{\pgfqpoint{2.097925in}{3.340225in}}{\pgfqpoint{2.110947in}{3.340225in}}%
\pgfpathlineto{\pgfqpoint{2.110947in}{3.340225in}}%
\pgfpathclose%
\pgfusepath{stroke,fill}%
\end{pgfscope}%
\begin{pgfscope}%
\pgfpathrectangle{\pgfqpoint{0.786164in}{0.768110in}}{\pgfqpoint{8.851069in}{7.081890in}}%
\pgfusepath{clip}%
\pgfsetbuttcap%
\pgfsetroundjoin%
\definecolor{currentfill}{rgb}{0.267968,0.223549,0.512008}%
\pgfsetfillcolor{currentfill}%
\pgfsetfillopacity{0.700000}%
\pgfsetlinewidth{0.501875pt}%
\definecolor{currentstroke}{rgb}{1.000000,1.000000,1.000000}%
\pgfsetstrokecolor{currentstroke}%
\pgfsetstrokeopacity{0.700000}%
\pgfsetdash{}{0pt}%
\pgfpathmoveto{\pgfqpoint{2.065281in}{3.296428in}}%
\pgfpathcurveto{\pgfqpoint{2.078304in}{3.296428in}}{\pgfqpoint{2.090795in}{3.301602in}}{\pgfqpoint{2.100003in}{3.310811in}}%
\pgfpathcurveto{\pgfqpoint{2.109211in}{3.320019in}}{\pgfqpoint{2.114385in}{3.332510in}}{\pgfqpoint{2.114385in}{3.345533in}}%
\pgfpathcurveto{\pgfqpoint{2.114385in}{3.358556in}}{\pgfqpoint{2.109211in}{3.371047in}}{\pgfqpoint{2.100003in}{3.380255in}}%
\pgfpathcurveto{\pgfqpoint{2.090795in}{3.389463in}}{\pgfqpoint{2.078304in}{3.394637in}}{\pgfqpoint{2.065281in}{3.394637in}}%
\pgfpathcurveto{\pgfqpoint{2.052258in}{3.394637in}}{\pgfqpoint{2.039767in}{3.389463in}}{\pgfqpoint{2.030559in}{3.380255in}}%
\pgfpathcurveto{\pgfqpoint{2.021350in}{3.371047in}}{\pgfqpoint{2.016176in}{3.358556in}}{\pgfqpoint{2.016176in}{3.345533in}}%
\pgfpathcurveto{\pgfqpoint{2.016176in}{3.332510in}}{\pgfqpoint{2.021350in}{3.320019in}}{\pgfqpoint{2.030559in}{3.310811in}}%
\pgfpathcurveto{\pgfqpoint{2.039767in}{3.301602in}}{\pgfqpoint{2.052258in}{3.296428in}}{\pgfqpoint{2.065281in}{3.296428in}}%
\pgfpathlineto{\pgfqpoint{2.065281in}{3.296428in}}%
\pgfpathclose%
\pgfusepath{stroke,fill}%
\end{pgfscope}%
\begin{pgfscope}%
\pgfpathrectangle{\pgfqpoint{0.786164in}{0.768110in}}{\pgfqpoint{8.851069in}{7.081890in}}%
\pgfusepath{clip}%
\pgfsetbuttcap%
\pgfsetroundjoin%
\definecolor{currentfill}{rgb}{0.252194,0.269783,0.531579}%
\pgfsetfillcolor{currentfill}%
\pgfsetfillopacity{0.700000}%
\pgfsetlinewidth{0.501875pt}%
\definecolor{currentstroke}{rgb}{1.000000,1.000000,1.000000}%
\pgfsetstrokecolor{currentstroke}%
\pgfsetstrokeopacity{0.700000}%
\pgfsetdash{}{0pt}%
\pgfpathmoveto{\pgfqpoint{2.065281in}{3.230733in}}%
\pgfpathcurveto{\pgfqpoint{2.078304in}{3.230733in}}{\pgfqpoint{2.090795in}{3.235907in}}{\pgfqpoint{2.100003in}{3.245116in}}%
\pgfpathcurveto{\pgfqpoint{2.109211in}{3.254324in}}{\pgfqpoint{2.114385in}{3.266815in}}{\pgfqpoint{2.114385in}{3.279838in}}%
\pgfpathcurveto{\pgfqpoint{2.114385in}{3.292861in}}{\pgfqpoint{2.109211in}{3.305352in}}{\pgfqpoint{2.100003in}{3.314560in}}%
\pgfpathcurveto{\pgfqpoint{2.090795in}{3.323769in}}{\pgfqpoint{2.078304in}{3.328943in}}{\pgfqpoint{2.065281in}{3.328943in}}%
\pgfpathcurveto{\pgfqpoint{2.052258in}{3.328943in}}{\pgfqpoint{2.039767in}{3.323769in}}{\pgfqpoint{2.030559in}{3.314560in}}%
\pgfpathcurveto{\pgfqpoint{2.021350in}{3.305352in}}{\pgfqpoint{2.016176in}{3.292861in}}{\pgfqpoint{2.016176in}{3.279838in}}%
\pgfpathcurveto{\pgfqpoint{2.016176in}{3.266815in}}{\pgfqpoint{2.021350in}{3.254324in}}{\pgfqpoint{2.030559in}{3.245116in}}%
\pgfpathcurveto{\pgfqpoint{2.039767in}{3.235907in}}{\pgfqpoint{2.052258in}{3.230733in}}{\pgfqpoint{2.065281in}{3.230733in}}%
\pgfpathlineto{\pgfqpoint{2.065281in}{3.230733in}}%
\pgfpathclose%
\pgfusepath{stroke,fill}%
\end{pgfscope}%
\begin{pgfscope}%
\pgfpathrectangle{\pgfqpoint{0.786164in}{0.768110in}}{\pgfqpoint{8.851069in}{7.081890in}}%
\pgfusepath{clip}%
\pgfsetbuttcap%
\pgfsetroundjoin%
\definecolor{currentfill}{rgb}{0.239346,0.300855,0.540844}%
\pgfsetfillcolor{currentfill}%
\pgfsetfillopacity{0.700000}%
\pgfsetlinewidth{0.501875pt}%
\definecolor{currentstroke}{rgb}{1.000000,1.000000,1.000000}%
\pgfsetstrokecolor{currentstroke}%
\pgfsetstrokeopacity{0.700000}%
\pgfsetdash{}{0pt}%
\pgfpathmoveto{\pgfqpoint{1.827815in}{3.099344in}}%
\pgfpathcurveto{\pgfqpoint{1.840838in}{3.099344in}}{\pgfqpoint{1.853329in}{3.104518in}}{\pgfqpoint{1.862538in}{3.113726in}}%
\pgfpathcurveto{\pgfqpoint{1.871746in}{3.122935in}}{\pgfqpoint{1.876920in}{3.135426in}}{\pgfqpoint{1.876920in}{3.148449in}}%
\pgfpathcurveto{\pgfqpoint{1.876920in}{3.161471in}}{\pgfqpoint{1.871746in}{3.173962in}}{\pgfqpoint{1.862538in}{3.183171in}}%
\pgfpathcurveto{\pgfqpoint{1.853329in}{3.192379in}}{\pgfqpoint{1.840838in}{3.197553in}}{\pgfqpoint{1.827815in}{3.197553in}}%
\pgfpathcurveto{\pgfqpoint{1.814793in}{3.197553in}}{\pgfqpoint{1.802302in}{3.192379in}}{\pgfqpoint{1.793093in}{3.183171in}}%
\pgfpathcurveto{\pgfqpoint{1.783885in}{3.173962in}}{\pgfqpoint{1.778711in}{3.161471in}}{\pgfqpoint{1.778711in}{3.148449in}}%
\pgfpathcurveto{\pgfqpoint{1.778711in}{3.135426in}}{\pgfqpoint{1.783885in}{3.122935in}}{\pgfqpoint{1.793093in}{3.113726in}}%
\pgfpathcurveto{\pgfqpoint{1.802302in}{3.104518in}}{\pgfqpoint{1.814793in}{3.099344in}}{\pgfqpoint{1.827815in}{3.099344in}}%
\pgfpathlineto{\pgfqpoint{1.827815in}{3.099344in}}%
\pgfpathclose%
\pgfusepath{stroke,fill}%
\end{pgfscope}%
\begin{pgfscope}%
\pgfpathrectangle{\pgfqpoint{0.786164in}{0.768110in}}{\pgfqpoint{8.851069in}{7.081890in}}%
\pgfusepath{clip}%
\pgfsetbuttcap%
\pgfsetroundjoin%
\definecolor{currentfill}{rgb}{0.237441,0.305202,0.541921}%
\pgfsetfillcolor{currentfill}%
\pgfsetfillopacity{0.700000}%
\pgfsetlinewidth{0.501875pt}%
\definecolor{currentstroke}{rgb}{1.000000,1.000000,1.000000}%
\pgfsetstrokecolor{currentstroke}%
\pgfsetstrokeopacity{0.700000}%
\pgfsetdash{}{0pt}%
\pgfpathmoveto{\pgfqpoint{1.827815in}{3.077446in}}%
\pgfpathcurveto{\pgfqpoint{1.840838in}{3.077446in}}{\pgfqpoint{1.853329in}{3.082620in}}{\pgfqpoint{1.862538in}{3.091828in}}%
\pgfpathcurveto{\pgfqpoint{1.871746in}{3.101037in}}{\pgfqpoint{1.876920in}{3.113528in}}{\pgfqpoint{1.876920in}{3.126550in}}%
\pgfpathcurveto{\pgfqpoint{1.876920in}{3.139573in}}{\pgfqpoint{1.871746in}{3.152064in}}{\pgfqpoint{1.862538in}{3.161273in}}%
\pgfpathcurveto{\pgfqpoint{1.853329in}{3.170481in}}{\pgfqpoint{1.840838in}{3.175655in}}{\pgfqpoint{1.827815in}{3.175655in}}%
\pgfpathcurveto{\pgfqpoint{1.814793in}{3.175655in}}{\pgfqpoint{1.802302in}{3.170481in}}{\pgfqpoint{1.793093in}{3.161273in}}%
\pgfpathcurveto{\pgfqpoint{1.783885in}{3.152064in}}{\pgfqpoint{1.778711in}{3.139573in}}{\pgfqpoint{1.778711in}{3.126550in}}%
\pgfpathcurveto{\pgfqpoint{1.778711in}{3.113528in}}{\pgfqpoint{1.783885in}{3.101037in}}{\pgfqpoint{1.793093in}{3.091828in}}%
\pgfpathcurveto{\pgfqpoint{1.802302in}{3.082620in}}{\pgfqpoint{1.814793in}{3.077446in}}{\pgfqpoint{1.827815in}{3.077446in}}%
\pgfpathlineto{\pgfqpoint{1.827815in}{3.077446in}}%
\pgfpathclose%
\pgfusepath{stroke,fill}%
\end{pgfscope}%
\begin{pgfscope}%
\pgfpathrectangle{\pgfqpoint{0.786164in}{0.768110in}}{\pgfqpoint{8.851069in}{7.081890in}}%
\pgfusepath{clip}%
\pgfsetbuttcap%
\pgfsetroundjoin%
\definecolor{currentfill}{rgb}{0.237441,0.305202,0.541921}%
\pgfsetfillcolor{currentfill}%
\pgfsetfillopacity{0.700000}%
\pgfsetlinewidth{0.501875pt}%
\definecolor{currentstroke}{rgb}{1.000000,1.000000,1.000000}%
\pgfsetstrokecolor{currentstroke}%
\pgfsetstrokeopacity{0.700000}%
\pgfsetdash{}{0pt}%
\pgfpathmoveto{\pgfqpoint{1.846082in}{2.946056in}}%
\pgfpathcurveto{\pgfqpoint{1.859105in}{2.946056in}}{\pgfqpoint{1.871596in}{2.951230in}}{\pgfqpoint{1.880804in}{2.960439in}}%
\pgfpathcurveto{\pgfqpoint{1.890013in}{2.969647in}}{\pgfqpoint{1.895187in}{2.982138in}}{\pgfqpoint{1.895187in}{2.995161in}}%
\pgfpathcurveto{\pgfqpoint{1.895187in}{3.008184in}}{\pgfqpoint{1.890013in}{3.020675in}}{\pgfqpoint{1.880804in}{3.029883in}}%
\pgfpathcurveto{\pgfqpoint{1.871596in}{3.039092in}}{\pgfqpoint{1.859105in}{3.044266in}}{\pgfqpoint{1.846082in}{3.044266in}}%
\pgfpathcurveto{\pgfqpoint{1.833059in}{3.044266in}}{\pgfqpoint{1.820568in}{3.039092in}}{\pgfqpoint{1.811360in}{3.029883in}}%
\pgfpathcurveto{\pgfqpoint{1.802151in}{3.020675in}}{\pgfqpoint{1.796977in}{3.008184in}}{\pgfqpoint{1.796977in}{2.995161in}}%
\pgfpathcurveto{\pgfqpoint{1.796977in}{2.982138in}}{\pgfqpoint{1.802151in}{2.969647in}}{\pgfqpoint{1.811360in}{2.960439in}}%
\pgfpathcurveto{\pgfqpoint{1.820568in}{2.951230in}}{\pgfqpoint{1.833059in}{2.946056in}}{\pgfqpoint{1.846082in}{2.946056in}}%
\pgfpathlineto{\pgfqpoint{1.846082in}{2.946056in}}%
\pgfpathclose%
\pgfusepath{stroke,fill}%
\end{pgfscope}%
\begin{pgfscope}%
\pgfpathrectangle{\pgfqpoint{0.786164in}{0.768110in}}{\pgfqpoint{8.851069in}{7.081890in}}%
\pgfusepath{clip}%
\pgfsetbuttcap%
\pgfsetroundjoin%
\definecolor{currentfill}{rgb}{0.239346,0.300855,0.540844}%
\pgfsetfillcolor{currentfill}%
\pgfsetfillopacity{0.700000}%
\pgfsetlinewidth{0.501875pt}%
\definecolor{currentstroke}{rgb}{1.000000,1.000000,1.000000}%
\pgfsetstrokecolor{currentstroke}%
\pgfsetstrokeopacity{0.700000}%
\pgfsetdash{}{0pt}%
\pgfpathmoveto{\pgfqpoint{1.827815in}{2.880362in}}%
\pgfpathcurveto{\pgfqpoint{1.840838in}{2.880362in}}{\pgfqpoint{1.853329in}{2.885536in}}{\pgfqpoint{1.862538in}{2.894744in}}%
\pgfpathcurveto{\pgfqpoint{1.871746in}{2.903953in}}{\pgfqpoint{1.876920in}{2.916444in}}{\pgfqpoint{1.876920in}{2.929466in}}%
\pgfpathcurveto{\pgfqpoint{1.876920in}{2.942489in}}{\pgfqpoint{1.871746in}{2.954980in}}{\pgfqpoint{1.862538in}{2.964189in}}%
\pgfpathcurveto{\pgfqpoint{1.853329in}{2.973397in}}{\pgfqpoint{1.840838in}{2.978571in}}{\pgfqpoint{1.827815in}{2.978571in}}%
\pgfpathcurveto{\pgfqpoint{1.814793in}{2.978571in}}{\pgfqpoint{1.802302in}{2.973397in}}{\pgfqpoint{1.793093in}{2.964189in}}%
\pgfpathcurveto{\pgfqpoint{1.783885in}{2.954980in}}{\pgfqpoint{1.778711in}{2.942489in}}{\pgfqpoint{1.778711in}{2.929466in}}%
\pgfpathcurveto{\pgfqpoint{1.778711in}{2.916444in}}{\pgfqpoint{1.783885in}{2.903953in}}{\pgfqpoint{1.793093in}{2.894744in}}%
\pgfpathcurveto{\pgfqpoint{1.802302in}{2.885536in}}{\pgfqpoint{1.814793in}{2.880362in}}{\pgfqpoint{1.827815in}{2.880362in}}%
\pgfpathlineto{\pgfqpoint{1.827815in}{2.880362in}}%
\pgfpathclose%
\pgfusepath{stroke,fill}%
\end{pgfscope}%
\begin{pgfscope}%
\pgfpathrectangle{\pgfqpoint{0.786164in}{0.768110in}}{\pgfqpoint{8.851069in}{7.081890in}}%
\pgfusepath{clip}%
\pgfsetbuttcap%
\pgfsetroundjoin%
\definecolor{currentfill}{rgb}{0.225863,0.330805,0.547314}%
\pgfsetfillcolor{currentfill}%
\pgfsetfillopacity{0.700000}%
\pgfsetlinewidth{0.501875pt}%
\definecolor{currentstroke}{rgb}{1.000000,1.000000,1.000000}%
\pgfsetstrokecolor{currentstroke}%
\pgfsetstrokeopacity{0.700000}%
\pgfsetdash{}{0pt}%
\pgfpathmoveto{\pgfqpoint{1.846082in}{2.967955in}}%
\pgfpathcurveto{\pgfqpoint{1.859105in}{2.967955in}}{\pgfqpoint{1.871596in}{2.973129in}}{\pgfqpoint{1.880804in}{2.982337in}}%
\pgfpathcurveto{\pgfqpoint{1.890013in}{2.991545in}}{\pgfqpoint{1.895187in}{3.004037in}}{\pgfqpoint{1.895187in}{3.017059in}}%
\pgfpathcurveto{\pgfqpoint{1.895187in}{3.030082in}}{\pgfqpoint{1.890013in}{3.042573in}}{\pgfqpoint{1.880804in}{3.051781in}}%
\pgfpathcurveto{\pgfqpoint{1.871596in}{3.060990in}}{\pgfqpoint{1.859105in}{3.066164in}}{\pgfqpoint{1.846082in}{3.066164in}}%
\pgfpathcurveto{\pgfqpoint{1.833059in}{3.066164in}}{\pgfqpoint{1.820568in}{3.060990in}}{\pgfqpoint{1.811360in}{3.051781in}}%
\pgfpathcurveto{\pgfqpoint{1.802151in}{3.042573in}}{\pgfqpoint{1.796977in}{3.030082in}}{\pgfqpoint{1.796977in}{3.017059in}}%
\pgfpathcurveto{\pgfqpoint{1.796977in}{3.004037in}}{\pgfqpoint{1.802151in}{2.991545in}}{\pgfqpoint{1.811360in}{2.982337in}}%
\pgfpathcurveto{\pgfqpoint{1.820568in}{2.973129in}}{\pgfqpoint{1.833059in}{2.967955in}}{\pgfqpoint{1.846082in}{2.967955in}}%
\pgfpathlineto{\pgfqpoint{1.846082in}{2.967955in}}%
\pgfpathclose%
\pgfusepath{stroke,fill}%
\end{pgfscope}%
\begin{pgfscope}%
\pgfpathrectangle{\pgfqpoint{0.786164in}{0.768110in}}{\pgfqpoint{8.851069in}{7.081890in}}%
\pgfusepath{clip}%
\pgfsetbuttcap%
\pgfsetroundjoin%
\definecolor{currentfill}{rgb}{0.220057,0.343307,0.549413}%
\pgfsetfillcolor{currentfill}%
\pgfsetfillopacity{0.700000}%
\pgfsetlinewidth{0.501875pt}%
\definecolor{currentstroke}{rgb}{1.000000,1.000000,1.000000}%
\pgfsetstrokecolor{currentstroke}%
\pgfsetstrokeopacity{0.700000}%
\pgfsetdash{}{0pt}%
\pgfpathmoveto{\pgfqpoint{1.800415in}{2.902260in}}%
\pgfpathcurveto{\pgfqpoint{1.813438in}{2.902260in}}{\pgfqpoint{1.825929in}{2.907434in}}{\pgfqpoint{1.835138in}{2.916642in}}%
\pgfpathcurveto{\pgfqpoint{1.844346in}{2.925851in}}{\pgfqpoint{1.849520in}{2.938342in}}{\pgfqpoint{1.849520in}{2.951365in}}%
\pgfpathcurveto{\pgfqpoint{1.849520in}{2.964387in}}{\pgfqpoint{1.844346in}{2.976878in}}{\pgfqpoint{1.835138in}{2.986087in}}%
\pgfpathcurveto{\pgfqpoint{1.825929in}{2.995295in}}{\pgfqpoint{1.813438in}{3.000469in}}{\pgfqpoint{1.800415in}{3.000469in}}%
\pgfpathcurveto{\pgfqpoint{1.787393in}{3.000469in}}{\pgfqpoint{1.774902in}{2.995295in}}{\pgfqpoint{1.765693in}{2.986087in}}%
\pgfpathcurveto{\pgfqpoint{1.756485in}{2.976878in}}{\pgfqpoint{1.751311in}{2.964387in}}{\pgfqpoint{1.751311in}{2.951365in}}%
\pgfpathcurveto{\pgfqpoint{1.751311in}{2.938342in}}{\pgfqpoint{1.756485in}{2.925851in}}{\pgfqpoint{1.765693in}{2.916642in}}%
\pgfpathcurveto{\pgfqpoint{1.774902in}{2.907434in}}{\pgfqpoint{1.787393in}{2.902260in}}{\pgfqpoint{1.800415in}{2.902260in}}%
\pgfpathlineto{\pgfqpoint{1.800415in}{2.902260in}}%
\pgfpathclose%
\pgfusepath{stroke,fill}%
\end{pgfscope}%
\begin{pgfscope}%
\pgfpathrectangle{\pgfqpoint{0.786164in}{0.768110in}}{\pgfqpoint{8.851069in}{7.081890in}}%
\pgfusepath{clip}%
\pgfsetbuttcap%
\pgfsetroundjoin%
\definecolor{currentfill}{rgb}{0.208623,0.367752,0.552675}%
\pgfsetfillcolor{currentfill}%
\pgfsetfillopacity{0.700000}%
\pgfsetlinewidth{0.501875pt}%
\definecolor{currentstroke}{rgb}{1.000000,1.000000,1.000000}%
\pgfsetstrokecolor{currentstroke}%
\pgfsetstrokeopacity{0.700000}%
\pgfsetdash{}{0pt}%
\pgfpathmoveto{\pgfqpoint{1.773016in}{2.748972in}}%
\pgfpathcurveto{\pgfqpoint{1.786038in}{2.748972in}}{\pgfqpoint{1.798529in}{2.754146in}}{\pgfqpoint{1.807738in}{2.763355in}}%
\pgfpathcurveto{\pgfqpoint{1.816946in}{2.772563in}}{\pgfqpoint{1.822120in}{2.785054in}}{\pgfqpoint{1.822120in}{2.798077in}}%
\pgfpathcurveto{\pgfqpoint{1.822120in}{2.811100in}}{\pgfqpoint{1.816946in}{2.823591in}}{\pgfqpoint{1.807738in}{2.832799in}}%
\pgfpathcurveto{\pgfqpoint{1.798529in}{2.842008in}}{\pgfqpoint{1.786038in}{2.847182in}}{\pgfqpoint{1.773016in}{2.847182in}}%
\pgfpathcurveto{\pgfqpoint{1.759993in}{2.847182in}}{\pgfqpoint{1.747502in}{2.842008in}}{\pgfqpoint{1.738293in}{2.832799in}}%
\pgfpathcurveto{\pgfqpoint{1.729085in}{2.823591in}}{\pgfqpoint{1.723911in}{2.811100in}}{\pgfqpoint{1.723911in}{2.798077in}}%
\pgfpathcurveto{\pgfqpoint{1.723911in}{2.785054in}}{\pgfqpoint{1.729085in}{2.772563in}}{\pgfqpoint{1.738293in}{2.763355in}}%
\pgfpathcurveto{\pgfqpoint{1.747502in}{2.754146in}}{\pgfqpoint{1.759993in}{2.748972in}}{\pgfqpoint{1.773016in}{2.748972in}}%
\pgfpathlineto{\pgfqpoint{1.773016in}{2.748972in}}%
\pgfpathclose%
\pgfusepath{stroke,fill}%
\end{pgfscope}%
\begin{pgfscope}%
\pgfpathrectangle{\pgfqpoint{0.786164in}{0.768110in}}{\pgfqpoint{8.851069in}{7.081890in}}%
\pgfusepath{clip}%
\pgfsetbuttcap%
\pgfsetroundjoin%
\definecolor{currentfill}{rgb}{0.192357,0.403199,0.555836}%
\pgfsetfillcolor{currentfill}%
\pgfsetfillopacity{0.700000}%
\pgfsetlinewidth{0.501875pt}%
\definecolor{currentstroke}{rgb}{1.000000,1.000000,1.000000}%
\pgfsetstrokecolor{currentstroke}%
\pgfsetstrokeopacity{0.700000}%
\pgfsetdash{}{0pt}%
\pgfpathmoveto{\pgfqpoint{1.517284in}{2.486193in}}%
\pgfpathcurveto{\pgfqpoint{1.530306in}{2.486193in}}{\pgfqpoint{1.542797in}{2.491367in}}{\pgfqpoint{1.552006in}{2.500576in}}%
\pgfpathcurveto{\pgfqpoint{1.561214in}{2.509784in}}{\pgfqpoint{1.566388in}{2.522275in}}{\pgfqpoint{1.566388in}{2.535298in}}%
\pgfpathcurveto{\pgfqpoint{1.566388in}{2.548321in}}{\pgfqpoint{1.561214in}{2.560812in}}{\pgfqpoint{1.552006in}{2.570020in}}%
\pgfpathcurveto{\pgfqpoint{1.542797in}{2.579229in}}{\pgfqpoint{1.530306in}{2.584403in}}{\pgfqpoint{1.517284in}{2.584403in}}%
\pgfpathcurveto{\pgfqpoint{1.504261in}{2.584403in}}{\pgfqpoint{1.491770in}{2.579229in}}{\pgfqpoint{1.482561in}{2.570020in}}%
\pgfpathcurveto{\pgfqpoint{1.473353in}{2.560812in}}{\pgfqpoint{1.468179in}{2.548321in}}{\pgfqpoint{1.468179in}{2.535298in}}%
\pgfpathcurveto{\pgfqpoint{1.468179in}{2.522275in}}{\pgfqpoint{1.473353in}{2.509784in}}{\pgfqpoint{1.482561in}{2.500576in}}%
\pgfpathcurveto{\pgfqpoint{1.491770in}{2.491367in}}{\pgfqpoint{1.504261in}{2.486193in}}{\pgfqpoint{1.517284in}{2.486193in}}%
\pgfpathlineto{\pgfqpoint{1.517284in}{2.486193in}}%
\pgfpathclose%
\pgfusepath{stroke,fill}%
\end{pgfscope}%
\begin{pgfscope}%
\pgfpathrectangle{\pgfqpoint{0.786164in}{0.768110in}}{\pgfqpoint{8.851069in}{7.081890in}}%
\pgfusepath{clip}%
\pgfsetbuttcap%
\pgfsetroundjoin%
\definecolor{currentfill}{rgb}{0.190631,0.407061,0.556089}%
\pgfsetfillcolor{currentfill}%
\pgfsetfillopacity{0.700000}%
\pgfsetlinewidth{0.501875pt}%
\definecolor{currentstroke}{rgb}{1.000000,1.000000,1.000000}%
\pgfsetstrokecolor{currentstroke}%
\pgfsetstrokeopacity{0.700000}%
\pgfsetdash{}{0pt}%
\pgfpathmoveto{\pgfqpoint{1.626883in}{2.573786in}}%
\pgfpathcurveto{\pgfqpoint{1.639906in}{2.573786in}}{\pgfqpoint{1.652397in}{2.578960in}}{\pgfqpoint{1.661605in}{2.588169in}}%
\pgfpathcurveto{\pgfqpoint{1.670814in}{2.597377in}}{\pgfqpoint{1.675988in}{2.609868in}}{\pgfqpoint{1.675988in}{2.622891in}}%
\pgfpathcurveto{\pgfqpoint{1.675988in}{2.635914in}}{\pgfqpoint{1.670814in}{2.648405in}}{\pgfqpoint{1.661605in}{2.657613in}}%
\pgfpathcurveto{\pgfqpoint{1.652397in}{2.666822in}}{\pgfqpoint{1.639906in}{2.671996in}}{\pgfqpoint{1.626883in}{2.671996in}}%
\pgfpathcurveto{\pgfqpoint{1.613860in}{2.671996in}}{\pgfqpoint{1.601369in}{2.666822in}}{\pgfqpoint{1.592161in}{2.657613in}}%
\pgfpathcurveto{\pgfqpoint{1.582952in}{2.648405in}}{\pgfqpoint{1.577778in}{2.635914in}}{\pgfqpoint{1.577778in}{2.622891in}}%
\pgfpathcurveto{\pgfqpoint{1.577778in}{2.609868in}}{\pgfqpoint{1.582952in}{2.597377in}}{\pgfqpoint{1.592161in}{2.588169in}}%
\pgfpathcurveto{\pgfqpoint{1.601369in}{2.578960in}}{\pgfqpoint{1.613860in}{2.573786in}}{\pgfqpoint{1.626883in}{2.573786in}}%
\pgfpathlineto{\pgfqpoint{1.626883in}{2.573786in}}%
\pgfpathclose%
\pgfusepath{stroke,fill}%
\end{pgfscope}%
\begin{pgfscope}%
\pgfpathrectangle{\pgfqpoint{0.786164in}{0.768110in}}{\pgfqpoint{8.851069in}{7.081890in}}%
\pgfusepath{clip}%
\pgfsetbuttcap%
\pgfsetroundjoin%
\definecolor{currentfill}{rgb}{0.187231,0.414746,0.556547}%
\pgfsetfillcolor{currentfill}%
\pgfsetfillopacity{0.700000}%
\pgfsetlinewidth{0.501875pt}%
\definecolor{currentstroke}{rgb}{1.000000,1.000000,1.000000}%
\pgfsetstrokecolor{currentstroke}%
\pgfsetstrokeopacity{0.700000}%
\pgfsetdash{}{0pt}%
\pgfpathmoveto{\pgfqpoint{1.699949in}{2.639481in}}%
\pgfpathcurveto{\pgfqpoint{1.712972in}{2.639481in}}{\pgfqpoint{1.725463in}{2.644655in}}{\pgfqpoint{1.734672in}{2.653864in}}%
\pgfpathcurveto{\pgfqpoint{1.743880in}{2.663072in}}{\pgfqpoint{1.749054in}{2.675563in}}{\pgfqpoint{1.749054in}{2.688586in}}%
\pgfpathcurveto{\pgfqpoint{1.749054in}{2.701608in}}{\pgfqpoint{1.743880in}{2.714100in}}{\pgfqpoint{1.734672in}{2.723308in}}%
\pgfpathcurveto{\pgfqpoint{1.725463in}{2.732516in}}{\pgfqpoint{1.712972in}{2.737690in}}{\pgfqpoint{1.699949in}{2.737690in}}%
\pgfpathcurveto{\pgfqpoint{1.686927in}{2.737690in}}{\pgfqpoint{1.674436in}{2.732516in}}{\pgfqpoint{1.665227in}{2.723308in}}%
\pgfpathcurveto{\pgfqpoint{1.656019in}{2.714100in}}{\pgfqpoint{1.650845in}{2.701608in}}{\pgfqpoint{1.650845in}{2.688586in}}%
\pgfpathcurveto{\pgfqpoint{1.650845in}{2.675563in}}{\pgfqpoint{1.656019in}{2.663072in}}{\pgfqpoint{1.665227in}{2.653864in}}%
\pgfpathcurveto{\pgfqpoint{1.674436in}{2.644655in}}{\pgfqpoint{1.686927in}{2.639481in}}{\pgfqpoint{1.699949in}{2.639481in}}%
\pgfpathlineto{\pgfqpoint{1.699949in}{2.639481in}}%
\pgfpathclose%
\pgfusepath{stroke,fill}%
\end{pgfscope}%
\begin{pgfscope}%
\pgfpathrectangle{\pgfqpoint{0.786164in}{0.768110in}}{\pgfqpoint{8.851069in}{7.081890in}}%
\pgfusepath{clip}%
\pgfsetbuttcap%
\pgfsetroundjoin%
\definecolor{currentfill}{rgb}{0.278826,0.175490,0.483397}%
\pgfsetfillcolor{currentfill}%
\pgfsetfillopacity{0.700000}%
\pgfsetlinewidth{0.501875pt}%
\definecolor{currentstroke}{rgb}{1.000000,1.000000,1.000000}%
\pgfsetstrokecolor{currentstroke}%
\pgfsetstrokeopacity{0.700000}%
\pgfsetdash{}{0pt}%
\pgfpathmoveto{\pgfqpoint{2.695478in}{3.011751in}}%
\pgfpathcurveto{\pgfqpoint{2.708500in}{3.011751in}}{\pgfqpoint{2.720991in}{3.016925in}}{\pgfqpoint{2.730200in}{3.026134in}}%
\pgfpathcurveto{\pgfqpoint{2.739408in}{3.035342in}}{\pgfqpoint{2.744582in}{3.047833in}}{\pgfqpoint{2.744582in}{3.060856in}}%
\pgfpathcurveto{\pgfqpoint{2.744582in}{3.073878in}}{\pgfqpoint{2.739408in}{3.086370in}}{\pgfqpoint{2.730200in}{3.095578in}}%
\pgfpathcurveto{\pgfqpoint{2.720991in}{3.104786in}}{\pgfqpoint{2.708500in}{3.109960in}}{\pgfqpoint{2.695478in}{3.109960in}}%
\pgfpathcurveto{\pgfqpoint{2.682455in}{3.109960in}}{\pgfqpoint{2.669964in}{3.104786in}}{\pgfqpoint{2.660755in}{3.095578in}}%
\pgfpathcurveto{\pgfqpoint{2.651547in}{3.086370in}}{\pgfqpoint{2.646373in}{3.073878in}}{\pgfqpoint{2.646373in}{3.060856in}}%
\pgfpathcurveto{\pgfqpoint{2.646373in}{3.047833in}}{\pgfqpoint{2.651547in}{3.035342in}}{\pgfqpoint{2.660755in}{3.026134in}}%
\pgfpathcurveto{\pgfqpoint{2.669964in}{3.016925in}}{\pgfqpoint{2.682455in}{3.011751in}}{\pgfqpoint{2.695478in}{3.011751in}}%
\pgfpathlineto{\pgfqpoint{2.695478in}{3.011751in}}%
\pgfpathclose%
\pgfusepath{stroke,fill}%
\end{pgfscope}%
\begin{pgfscope}%
\pgfpathrectangle{\pgfqpoint{0.786164in}{0.768110in}}{\pgfqpoint{8.851069in}{7.081890in}}%
\pgfusepath{clip}%
\pgfsetbuttcap%
\pgfsetroundjoin%
\definecolor{currentfill}{rgb}{0.278826,0.175490,0.483397}%
\pgfsetfillcolor{currentfill}%
\pgfsetfillopacity{0.700000}%
\pgfsetlinewidth{0.501875pt}%
\definecolor{currentstroke}{rgb}{1.000000,1.000000,1.000000}%
\pgfsetstrokecolor{currentstroke}%
\pgfsetstrokeopacity{0.700000}%
\pgfsetdash{}{0pt}%
\pgfpathmoveto{\pgfqpoint{2.722877in}{3.033649in}}%
\pgfpathcurveto{\pgfqpoint{2.735900in}{3.033649in}}{\pgfqpoint{2.748391in}{3.038823in}}{\pgfqpoint{2.757600in}{3.048032in}}%
\pgfpathcurveto{\pgfqpoint{2.766808in}{3.057240in}}{\pgfqpoint{2.771982in}{3.069731in}}{\pgfqpoint{2.771982in}{3.082754in}}%
\pgfpathcurveto{\pgfqpoint{2.771982in}{3.095777in}}{\pgfqpoint{2.766808in}{3.108268in}}{\pgfqpoint{2.757600in}{3.117476in}}%
\pgfpathcurveto{\pgfqpoint{2.748391in}{3.126685in}}{\pgfqpoint{2.735900in}{3.131859in}}{\pgfqpoint{2.722877in}{3.131859in}}%
\pgfpathcurveto{\pgfqpoint{2.709855in}{3.131859in}}{\pgfqpoint{2.697364in}{3.126685in}}{\pgfqpoint{2.688155in}{3.117476in}}%
\pgfpathcurveto{\pgfqpoint{2.678947in}{3.108268in}}{\pgfqpoint{2.673773in}{3.095777in}}{\pgfqpoint{2.673773in}{3.082754in}}%
\pgfpathcurveto{\pgfqpoint{2.673773in}{3.069731in}}{\pgfqpoint{2.678947in}{3.057240in}}{\pgfqpoint{2.688155in}{3.048032in}}%
\pgfpathcurveto{\pgfqpoint{2.697364in}{3.038823in}}{\pgfqpoint{2.709855in}{3.033649in}}{\pgfqpoint{2.722877in}{3.033649in}}%
\pgfpathlineto{\pgfqpoint{2.722877in}{3.033649in}}%
\pgfpathclose%
\pgfusepath{stroke,fill}%
\end{pgfscope}%
\begin{pgfscope}%
\pgfpathrectangle{\pgfqpoint{0.786164in}{0.768110in}}{\pgfqpoint{8.851069in}{7.081890in}}%
\pgfusepath{clip}%
\pgfsetbuttcap%
\pgfsetroundjoin%
\definecolor{currentfill}{rgb}{0.276194,0.190074,0.493001}%
\pgfsetfillcolor{currentfill}%
\pgfsetfillopacity{0.700000}%
\pgfsetlinewidth{0.501875pt}%
\definecolor{currentstroke}{rgb}{1.000000,1.000000,1.000000}%
\pgfsetstrokecolor{currentstroke}%
\pgfsetstrokeopacity{0.700000}%
\pgfsetdash{}{0pt}%
\pgfpathmoveto{\pgfqpoint{2.686344in}{2.967955in}}%
\pgfpathcurveto{\pgfqpoint{2.699367in}{2.967955in}}{\pgfqpoint{2.711858in}{2.973129in}}{\pgfqpoint{2.721067in}{2.982337in}}%
\pgfpathcurveto{\pgfqpoint{2.730275in}{2.991545in}}{\pgfqpoint{2.735449in}{3.004037in}}{\pgfqpoint{2.735449in}{3.017059in}}%
\pgfpathcurveto{\pgfqpoint{2.735449in}{3.030082in}}{\pgfqpoint{2.730275in}{3.042573in}}{\pgfqpoint{2.721067in}{3.051781in}}%
\pgfpathcurveto{\pgfqpoint{2.711858in}{3.060990in}}{\pgfqpoint{2.699367in}{3.066164in}}{\pgfqpoint{2.686344in}{3.066164in}}%
\pgfpathcurveto{\pgfqpoint{2.673322in}{3.066164in}}{\pgfqpoint{2.660831in}{3.060990in}}{\pgfqpoint{2.651622in}{3.051781in}}%
\pgfpathcurveto{\pgfqpoint{2.642414in}{3.042573in}}{\pgfqpoint{2.637240in}{3.030082in}}{\pgfqpoint{2.637240in}{3.017059in}}%
\pgfpathcurveto{\pgfqpoint{2.637240in}{3.004037in}}{\pgfqpoint{2.642414in}{2.991545in}}{\pgfqpoint{2.651622in}{2.982337in}}%
\pgfpathcurveto{\pgfqpoint{2.660831in}{2.973129in}}{\pgfqpoint{2.673322in}{2.967955in}}{\pgfqpoint{2.686344in}{2.967955in}}%
\pgfpathlineto{\pgfqpoint{2.686344in}{2.967955in}}%
\pgfpathclose%
\pgfusepath{stroke,fill}%
\end{pgfscope}%
\begin{pgfscope}%
\pgfpathrectangle{\pgfqpoint{0.786164in}{0.768110in}}{\pgfqpoint{8.851069in}{7.081890in}}%
\pgfusepath{clip}%
\pgfsetbuttcap%
\pgfsetroundjoin%
\definecolor{currentfill}{rgb}{0.274128,0.199721,0.498911}%
\pgfsetfillcolor{currentfill}%
\pgfsetfillopacity{0.700000}%
\pgfsetlinewidth{0.501875pt}%
\definecolor{currentstroke}{rgb}{1.000000,1.000000,1.000000}%
\pgfsetstrokecolor{currentstroke}%
\pgfsetstrokeopacity{0.700000}%
\pgfsetdash{}{0pt}%
\pgfpathmoveto{\pgfqpoint{2.677211in}{3.011751in}}%
\pgfpathcurveto{\pgfqpoint{2.690234in}{3.011751in}}{\pgfqpoint{2.702725in}{3.016925in}}{\pgfqpoint{2.711933in}{3.026134in}}%
\pgfpathcurveto{\pgfqpoint{2.721142in}{3.035342in}}{\pgfqpoint{2.726316in}{3.047833in}}{\pgfqpoint{2.726316in}{3.060856in}}%
\pgfpathcurveto{\pgfqpoint{2.726316in}{3.073878in}}{\pgfqpoint{2.721142in}{3.086370in}}{\pgfqpoint{2.711933in}{3.095578in}}%
\pgfpathcurveto{\pgfqpoint{2.702725in}{3.104786in}}{\pgfqpoint{2.690234in}{3.109960in}}{\pgfqpoint{2.677211in}{3.109960in}}%
\pgfpathcurveto{\pgfqpoint{2.664188in}{3.109960in}}{\pgfqpoint{2.651697in}{3.104786in}}{\pgfqpoint{2.642489in}{3.095578in}}%
\pgfpathcurveto{\pgfqpoint{2.633280in}{3.086370in}}{\pgfqpoint{2.628106in}{3.073878in}}{\pgfqpoint{2.628106in}{3.060856in}}%
\pgfpathcurveto{\pgfqpoint{2.628106in}{3.047833in}}{\pgfqpoint{2.633280in}{3.035342in}}{\pgfqpoint{2.642489in}{3.026134in}}%
\pgfpathcurveto{\pgfqpoint{2.651697in}{3.016925in}}{\pgfqpoint{2.664188in}{3.011751in}}{\pgfqpoint{2.677211in}{3.011751in}}%
\pgfpathlineto{\pgfqpoint{2.677211in}{3.011751in}}%
\pgfpathclose%
\pgfusepath{stroke,fill}%
\end{pgfscope}%
\begin{pgfscope}%
\pgfpathrectangle{\pgfqpoint{0.786164in}{0.768110in}}{\pgfqpoint{8.851069in}{7.081890in}}%
\pgfusepath{clip}%
\pgfsetbuttcap%
\pgfsetroundjoin%
\definecolor{currentfill}{rgb}{0.269308,0.218818,0.509577}%
\pgfsetfillcolor{currentfill}%
\pgfsetfillopacity{0.700000}%
\pgfsetlinewidth{0.501875pt}%
\definecolor{currentstroke}{rgb}{1.000000,1.000000,1.000000}%
\pgfsetstrokecolor{currentstroke}%
\pgfsetstrokeopacity{0.700000}%
\pgfsetdash{}{0pt}%
\pgfpathmoveto{\pgfqpoint{2.531078in}{2.814667in}}%
\pgfpathcurveto{\pgfqpoint{2.544101in}{2.814667in}}{\pgfqpoint{2.556592in}{2.819841in}}{\pgfqpoint{2.565801in}{2.829049in}}%
\pgfpathcurveto{\pgfqpoint{2.575009in}{2.838258in}}{\pgfqpoint{2.580183in}{2.850749in}}{\pgfqpoint{2.580183in}{2.863772in}}%
\pgfpathcurveto{\pgfqpoint{2.580183in}{2.876794in}}{\pgfqpoint{2.575009in}{2.889285in}}{\pgfqpoint{2.565801in}{2.898494in}}%
\pgfpathcurveto{\pgfqpoint{2.556592in}{2.907702in}}{\pgfqpoint{2.544101in}{2.912876in}}{\pgfqpoint{2.531078in}{2.912876in}}%
\pgfpathcurveto{\pgfqpoint{2.518056in}{2.912876in}}{\pgfqpoint{2.505565in}{2.907702in}}{\pgfqpoint{2.496356in}{2.898494in}}%
\pgfpathcurveto{\pgfqpoint{2.487148in}{2.889285in}}{\pgfqpoint{2.481974in}{2.876794in}}{\pgfqpoint{2.481974in}{2.863772in}}%
\pgfpathcurveto{\pgfqpoint{2.481974in}{2.850749in}}{\pgfqpoint{2.487148in}{2.838258in}}{\pgfqpoint{2.496356in}{2.829049in}}%
\pgfpathcurveto{\pgfqpoint{2.505565in}{2.819841in}}{\pgfqpoint{2.518056in}{2.814667in}}{\pgfqpoint{2.531078in}{2.814667in}}%
\pgfpathlineto{\pgfqpoint{2.531078in}{2.814667in}}%
\pgfpathclose%
\pgfusepath{stroke,fill}%
\end{pgfscope}%
\begin{pgfscope}%
\pgfpathrectangle{\pgfqpoint{0.786164in}{0.768110in}}{\pgfqpoint{8.851069in}{7.081890in}}%
\pgfusepath{clip}%
\pgfsetbuttcap%
\pgfsetroundjoin%
\definecolor{currentfill}{rgb}{0.257322,0.256130,0.526563}%
\pgfsetfillcolor{currentfill}%
\pgfsetfillopacity{0.700000}%
\pgfsetlinewidth{0.501875pt}%
\definecolor{currentstroke}{rgb}{1.000000,1.000000,1.000000}%
\pgfsetstrokecolor{currentstroke}%
\pgfsetstrokeopacity{0.700000}%
\pgfsetdash{}{0pt}%
\pgfpathmoveto{\pgfqpoint{2.293613in}{2.617583in}}%
\pgfpathcurveto{\pgfqpoint{2.306636in}{2.617583in}}{\pgfqpoint{2.319127in}{2.622757in}}{\pgfqpoint{2.328335in}{2.631965in}}%
\pgfpathcurveto{\pgfqpoint{2.337544in}{2.641174in}}{\pgfqpoint{2.342718in}{2.653665in}}{\pgfqpoint{2.342718in}{2.666687in}}%
\pgfpathcurveto{\pgfqpoint{2.342718in}{2.679710in}}{\pgfqpoint{2.337544in}{2.692201in}}{\pgfqpoint{2.328335in}{2.701410in}}%
\pgfpathcurveto{\pgfqpoint{2.319127in}{2.710618in}}{\pgfqpoint{2.306636in}{2.715792in}}{\pgfqpoint{2.293613in}{2.715792in}}%
\pgfpathcurveto{\pgfqpoint{2.280590in}{2.715792in}}{\pgfqpoint{2.268099in}{2.710618in}}{\pgfqpoint{2.258891in}{2.701410in}}%
\pgfpathcurveto{\pgfqpoint{2.249682in}{2.692201in}}{\pgfqpoint{2.244508in}{2.679710in}}{\pgfqpoint{2.244508in}{2.666687in}}%
\pgfpathcurveto{\pgfqpoint{2.244508in}{2.653665in}}{\pgfqpoint{2.249682in}{2.641174in}}{\pgfqpoint{2.258891in}{2.631965in}}%
\pgfpathcurveto{\pgfqpoint{2.268099in}{2.622757in}}{\pgfqpoint{2.280590in}{2.617583in}}{\pgfqpoint{2.293613in}{2.617583in}}%
\pgfpathlineto{\pgfqpoint{2.293613in}{2.617583in}}%
\pgfpathclose%
\pgfusepath{stroke,fill}%
\end{pgfscope}%
\begin{pgfscope}%
\pgfpathrectangle{\pgfqpoint{0.786164in}{0.768110in}}{\pgfqpoint{8.851069in}{7.081890in}}%
\pgfusepath{clip}%
\pgfsetbuttcap%
\pgfsetroundjoin%
\definecolor{currentfill}{rgb}{0.250425,0.274290,0.533103}%
\pgfsetfillcolor{currentfill}%
\pgfsetfillopacity{0.700000}%
\pgfsetlinewidth{0.501875pt}%
\definecolor{currentstroke}{rgb}{1.000000,1.000000,1.000000}%
\pgfsetstrokecolor{currentstroke}%
\pgfsetstrokeopacity{0.700000}%
\pgfsetdash{}{0pt}%
\pgfpathmoveto{\pgfqpoint{2.284480in}{2.529990in}}%
\pgfpathcurveto{\pgfqpoint{2.297502in}{2.529990in}}{\pgfqpoint{2.309993in}{2.535164in}}{\pgfqpoint{2.319202in}{2.544372in}}%
\pgfpathcurveto{\pgfqpoint{2.328410in}{2.553581in}}{\pgfqpoint{2.333584in}{2.566072in}}{\pgfqpoint{2.333584in}{2.579095in}}%
\pgfpathcurveto{\pgfqpoint{2.333584in}{2.592117in}}{\pgfqpoint{2.328410in}{2.604608in}}{\pgfqpoint{2.319202in}{2.613817in}}%
\pgfpathcurveto{\pgfqpoint{2.309993in}{2.623025in}}{\pgfqpoint{2.297502in}{2.628199in}}{\pgfqpoint{2.284480in}{2.628199in}}%
\pgfpathcurveto{\pgfqpoint{2.271457in}{2.628199in}}{\pgfqpoint{2.258966in}{2.623025in}}{\pgfqpoint{2.249757in}{2.613817in}}%
\pgfpathcurveto{\pgfqpoint{2.240549in}{2.604608in}}{\pgfqpoint{2.235375in}{2.592117in}}{\pgfqpoint{2.235375in}{2.579095in}}%
\pgfpathcurveto{\pgfqpoint{2.235375in}{2.566072in}}{\pgfqpoint{2.240549in}{2.553581in}}{\pgfqpoint{2.249757in}{2.544372in}}%
\pgfpathcurveto{\pgfqpoint{2.258966in}{2.535164in}}{\pgfqpoint{2.271457in}{2.529990in}}{\pgfqpoint{2.284480in}{2.529990in}}%
\pgfpathlineto{\pgfqpoint{2.284480in}{2.529990in}}%
\pgfpathclose%
\pgfusepath{stroke,fill}%
\end{pgfscope}%
\begin{pgfscope}%
\pgfpathrectangle{\pgfqpoint{0.786164in}{0.768110in}}{\pgfqpoint{8.851069in}{7.081890in}}%
\pgfusepath{clip}%
\pgfsetbuttcap%
\pgfsetroundjoin%
\definecolor{currentfill}{rgb}{0.255645,0.260703,0.528312}%
\pgfsetfillcolor{currentfill}%
\pgfsetfillopacity{0.700000}%
\pgfsetlinewidth{0.501875pt}%
\definecolor{currentstroke}{rgb}{1.000000,1.000000,1.000000}%
\pgfsetstrokecolor{currentstroke}%
\pgfsetstrokeopacity{0.700000}%
\pgfsetdash{}{0pt}%
\pgfpathmoveto{\pgfqpoint{2.266213in}{2.529990in}}%
\pgfpathcurveto{\pgfqpoint{2.279236in}{2.529990in}}{\pgfqpoint{2.291727in}{2.535164in}}{\pgfqpoint{2.300935in}{2.544372in}}%
\pgfpathcurveto{\pgfqpoint{2.310144in}{2.553581in}}{\pgfqpoint{2.315318in}{2.566072in}}{\pgfqpoint{2.315318in}{2.579095in}}%
\pgfpathcurveto{\pgfqpoint{2.315318in}{2.592117in}}{\pgfqpoint{2.310144in}{2.604608in}}{\pgfqpoint{2.300935in}{2.613817in}}%
\pgfpathcurveto{\pgfqpoint{2.291727in}{2.623025in}}{\pgfqpoint{2.279236in}{2.628199in}}{\pgfqpoint{2.266213in}{2.628199in}}%
\pgfpathcurveto{\pgfqpoint{2.253190in}{2.628199in}}{\pgfqpoint{2.240699in}{2.623025in}}{\pgfqpoint{2.231491in}{2.613817in}}%
\pgfpathcurveto{\pgfqpoint{2.222282in}{2.604608in}}{\pgfqpoint{2.217108in}{2.592117in}}{\pgfqpoint{2.217108in}{2.579095in}}%
\pgfpathcurveto{\pgfqpoint{2.217108in}{2.566072in}}{\pgfqpoint{2.222282in}{2.553581in}}{\pgfqpoint{2.231491in}{2.544372in}}%
\pgfpathcurveto{\pgfqpoint{2.240699in}{2.535164in}}{\pgfqpoint{2.253190in}{2.529990in}}{\pgfqpoint{2.266213in}{2.529990in}}%
\pgfpathlineto{\pgfqpoint{2.266213in}{2.529990in}}%
\pgfpathclose%
\pgfusepath{stroke,fill}%
\end{pgfscope}%
\begin{pgfscope}%
\pgfpathrectangle{\pgfqpoint{0.786164in}{0.768110in}}{\pgfqpoint{8.851069in}{7.081890in}}%
\pgfusepath{clip}%
\pgfsetbuttcap%
\pgfsetroundjoin%
\definecolor{currentfill}{rgb}{0.248629,0.278775,0.534556}%
\pgfsetfillcolor{currentfill}%
\pgfsetfillopacity{0.700000}%
\pgfsetlinewidth{0.501875pt}%
\definecolor{currentstroke}{rgb}{1.000000,1.000000,1.000000}%
\pgfsetstrokecolor{currentstroke}%
\pgfsetstrokeopacity{0.700000}%
\pgfsetdash{}{0pt}%
\pgfpathmoveto{\pgfqpoint{2.275346in}{2.508092in}}%
\pgfpathcurveto{\pgfqpoint{2.288369in}{2.508092in}}{\pgfqpoint{2.300860in}{2.513266in}}{\pgfqpoint{2.310069in}{2.522474in}}%
\pgfpathcurveto{\pgfqpoint{2.319277in}{2.531683in}}{\pgfqpoint{2.324451in}{2.544174in}}{\pgfqpoint{2.324451in}{2.557196in}}%
\pgfpathcurveto{\pgfqpoint{2.324451in}{2.570219in}}{\pgfqpoint{2.319277in}{2.582710in}}{\pgfqpoint{2.310069in}{2.591919in}}%
\pgfpathcurveto{\pgfqpoint{2.300860in}{2.601127in}}{\pgfqpoint{2.288369in}{2.606301in}}{\pgfqpoint{2.275346in}{2.606301in}}%
\pgfpathcurveto{\pgfqpoint{2.262324in}{2.606301in}}{\pgfqpoint{2.249833in}{2.601127in}}{\pgfqpoint{2.240624in}{2.591919in}}%
\pgfpathcurveto{\pgfqpoint{2.231416in}{2.582710in}}{\pgfqpoint{2.226242in}{2.570219in}}{\pgfqpoint{2.226242in}{2.557196in}}%
\pgfpathcurveto{\pgfqpoint{2.226242in}{2.544174in}}{\pgfqpoint{2.231416in}{2.531683in}}{\pgfqpoint{2.240624in}{2.522474in}}%
\pgfpathcurveto{\pgfqpoint{2.249833in}{2.513266in}}{\pgfqpoint{2.262324in}{2.508092in}}{\pgfqpoint{2.275346in}{2.508092in}}%
\pgfpathlineto{\pgfqpoint{2.275346in}{2.508092in}}%
\pgfpathclose%
\pgfusepath{stroke,fill}%
\end{pgfscope}%
\begin{pgfscope}%
\pgfpathrectangle{\pgfqpoint{0.786164in}{0.768110in}}{\pgfqpoint{8.851069in}{7.081890in}}%
\pgfusepath{clip}%
\pgfsetbuttcap%
\pgfsetroundjoin%
\definecolor{currentfill}{rgb}{0.241237,0.296485,0.539709}%
\pgfsetfillcolor{currentfill}%
\pgfsetfillopacity{0.700000}%
\pgfsetlinewidth{0.501875pt}%
\definecolor{currentstroke}{rgb}{1.000000,1.000000,1.000000}%
\pgfsetstrokecolor{currentstroke}%
\pgfsetstrokeopacity{0.700000}%
\pgfsetdash{}{0pt}%
\pgfpathmoveto{\pgfqpoint{2.138347in}{2.398600in}}%
\pgfpathcurveto{\pgfqpoint{2.151370in}{2.398600in}}{\pgfqpoint{2.163861in}{2.403774in}}{\pgfqpoint{2.173069in}{2.412983in}}%
\pgfpathcurveto{\pgfqpoint{2.182278in}{2.422191in}}{\pgfqpoint{2.187452in}{2.434682in}}{\pgfqpoint{2.187452in}{2.447705in}}%
\pgfpathcurveto{\pgfqpoint{2.187452in}{2.460728in}}{\pgfqpoint{2.182278in}{2.473219in}}{\pgfqpoint{2.173069in}{2.482427in}}%
\pgfpathcurveto{\pgfqpoint{2.163861in}{2.491636in}}{\pgfqpoint{2.151370in}{2.496810in}}{\pgfqpoint{2.138347in}{2.496810in}}%
\pgfpathcurveto{\pgfqpoint{2.125324in}{2.496810in}}{\pgfqpoint{2.112833in}{2.491636in}}{\pgfqpoint{2.103625in}{2.482427in}}%
\pgfpathcurveto{\pgfqpoint{2.094416in}{2.473219in}}{\pgfqpoint{2.089242in}{2.460728in}}{\pgfqpoint{2.089242in}{2.447705in}}%
\pgfpathcurveto{\pgfqpoint{2.089242in}{2.434682in}}{\pgfqpoint{2.094416in}{2.422191in}}{\pgfqpoint{2.103625in}{2.412983in}}%
\pgfpathcurveto{\pgfqpoint{2.112833in}{2.403774in}}{\pgfqpoint{2.125324in}{2.398600in}}{\pgfqpoint{2.138347in}{2.398600in}}%
\pgfpathlineto{\pgfqpoint{2.138347in}{2.398600in}}%
\pgfpathclose%
\pgfusepath{stroke,fill}%
\end{pgfscope}%
\begin{pgfscope}%
\pgfpathrectangle{\pgfqpoint{0.786164in}{0.768110in}}{\pgfqpoint{8.851069in}{7.081890in}}%
\pgfusepath{clip}%
\pgfsetbuttcap%
\pgfsetroundjoin%
\definecolor{currentfill}{rgb}{0.235526,0.309527,0.542944}%
\pgfsetfillcolor{currentfill}%
\pgfsetfillopacity{0.700000}%
\pgfsetlinewidth{0.501875pt}%
\definecolor{currentstroke}{rgb}{1.000000,1.000000,1.000000}%
\pgfsetstrokecolor{currentstroke}%
\pgfsetstrokeopacity{0.700000}%
\pgfsetdash{}{0pt}%
\pgfpathmoveto{\pgfqpoint{2.110947in}{2.398600in}}%
\pgfpathcurveto{\pgfqpoint{2.123970in}{2.398600in}}{\pgfqpoint{2.136461in}{2.403774in}}{\pgfqpoint{2.145669in}{2.412983in}}%
\pgfpathcurveto{\pgfqpoint{2.154878in}{2.422191in}}{\pgfqpoint{2.160052in}{2.434682in}}{\pgfqpoint{2.160052in}{2.447705in}}%
\pgfpathcurveto{\pgfqpoint{2.160052in}{2.460728in}}{\pgfqpoint{2.154878in}{2.473219in}}{\pgfqpoint{2.145669in}{2.482427in}}%
\pgfpathcurveto{\pgfqpoint{2.136461in}{2.491636in}}{\pgfqpoint{2.123970in}{2.496810in}}{\pgfqpoint{2.110947in}{2.496810in}}%
\pgfpathcurveto{\pgfqpoint{2.097925in}{2.496810in}}{\pgfqpoint{2.085433in}{2.491636in}}{\pgfqpoint{2.076225in}{2.482427in}}%
\pgfpathcurveto{\pgfqpoint{2.067017in}{2.473219in}}{\pgfqpoint{2.061843in}{2.460728in}}{\pgfqpoint{2.061843in}{2.447705in}}%
\pgfpathcurveto{\pgfqpoint{2.061843in}{2.434682in}}{\pgfqpoint{2.067017in}{2.422191in}}{\pgfqpoint{2.076225in}{2.412983in}}%
\pgfpathcurveto{\pgfqpoint{2.085433in}{2.403774in}}{\pgfqpoint{2.097925in}{2.398600in}}{\pgfqpoint{2.110947in}{2.398600in}}%
\pgfpathlineto{\pgfqpoint{2.110947in}{2.398600in}}%
\pgfpathclose%
\pgfusepath{stroke,fill}%
\end{pgfscope}%
\begin{pgfscope}%
\pgfpathrectangle{\pgfqpoint{0.786164in}{0.768110in}}{\pgfqpoint{8.851069in}{7.081890in}}%
\pgfusepath{clip}%
\pgfsetbuttcap%
\pgfsetroundjoin%
\definecolor{currentfill}{rgb}{0.233603,0.313828,0.543914}%
\pgfsetfillcolor{currentfill}%
\pgfsetfillopacity{0.700000}%
\pgfsetlinewidth{0.501875pt}%
\definecolor{currentstroke}{rgb}{1.000000,1.000000,1.000000}%
\pgfsetstrokecolor{currentstroke}%
\pgfsetstrokeopacity{0.700000}%
\pgfsetdash{}{0pt}%
\pgfpathmoveto{\pgfqpoint{2.193147in}{2.442397in}}%
\pgfpathcurveto{\pgfqpoint{2.206170in}{2.442397in}}{\pgfqpoint{2.218661in}{2.447571in}}{\pgfqpoint{2.227869in}{2.456779in}}%
\pgfpathcurveto{\pgfqpoint{2.237077in}{2.465988in}}{\pgfqpoint{2.242251in}{2.478479in}}{\pgfqpoint{2.242251in}{2.491502in}}%
\pgfpathcurveto{\pgfqpoint{2.242251in}{2.504524in}}{\pgfqpoint{2.237077in}{2.517015in}}{\pgfqpoint{2.227869in}{2.526224in}}%
\pgfpathcurveto{\pgfqpoint{2.218661in}{2.535432in}}{\pgfqpoint{2.206170in}{2.540606in}}{\pgfqpoint{2.193147in}{2.540606in}}%
\pgfpathcurveto{\pgfqpoint{2.180124in}{2.540606in}}{\pgfqpoint{2.167633in}{2.535432in}}{\pgfqpoint{2.158425in}{2.526224in}}%
\pgfpathcurveto{\pgfqpoint{2.149216in}{2.517015in}}{\pgfqpoint{2.144042in}{2.504524in}}{\pgfqpoint{2.144042in}{2.491502in}}%
\pgfpathcurveto{\pgfqpoint{2.144042in}{2.478479in}}{\pgfqpoint{2.149216in}{2.465988in}}{\pgfqpoint{2.158425in}{2.456779in}}%
\pgfpathcurveto{\pgfqpoint{2.167633in}{2.447571in}}{\pgfqpoint{2.180124in}{2.442397in}}{\pgfqpoint{2.193147in}{2.442397in}}%
\pgfpathlineto{\pgfqpoint{2.193147in}{2.442397in}}%
\pgfpathclose%
\pgfusepath{stroke,fill}%
\end{pgfscope}%
\begin{pgfscope}%
\pgfpathrectangle{\pgfqpoint{0.786164in}{0.768110in}}{\pgfqpoint{8.851069in}{7.081890in}}%
\pgfusepath{clip}%
\pgfsetbuttcap%
\pgfsetroundjoin%
\definecolor{currentfill}{rgb}{0.227802,0.326594,0.546532}%
\pgfsetfillcolor{currentfill}%
\pgfsetfillopacity{0.700000}%
\pgfsetlinewidth{0.501875pt}%
\definecolor{currentstroke}{rgb}{1.000000,1.000000,1.000000}%
\pgfsetstrokecolor{currentstroke}%
\pgfsetstrokeopacity{0.700000}%
\pgfsetdash{}{0pt}%
\pgfpathmoveto{\pgfqpoint{2.202280in}{2.376702in}}%
\pgfpathcurveto{\pgfqpoint{2.215303in}{2.376702in}}{\pgfqpoint{2.227794in}{2.381876in}}{\pgfqpoint{2.237002in}{2.391085in}}%
\pgfpathcurveto{\pgfqpoint{2.246211in}{2.400293in}}{\pgfqpoint{2.251385in}{2.412784in}}{\pgfqpoint{2.251385in}{2.425807in}}%
\pgfpathcurveto{\pgfqpoint{2.251385in}{2.438830in}}{\pgfqpoint{2.246211in}{2.451321in}}{\pgfqpoint{2.237002in}{2.460529in}}%
\pgfpathcurveto{\pgfqpoint{2.227794in}{2.469738in}}{\pgfqpoint{2.215303in}{2.474912in}}{\pgfqpoint{2.202280in}{2.474912in}}%
\pgfpathcurveto{\pgfqpoint{2.189257in}{2.474912in}}{\pgfqpoint{2.176766in}{2.469738in}}{\pgfqpoint{2.167558in}{2.460529in}}%
\pgfpathcurveto{\pgfqpoint{2.158349in}{2.451321in}}{\pgfqpoint{2.153175in}{2.438830in}}{\pgfqpoint{2.153175in}{2.425807in}}%
\pgfpathcurveto{\pgfqpoint{2.153175in}{2.412784in}}{\pgfqpoint{2.158349in}{2.400293in}}{\pgfqpoint{2.167558in}{2.391085in}}%
\pgfpathcurveto{\pgfqpoint{2.176766in}{2.381876in}}{\pgfqpoint{2.189257in}{2.376702in}}{\pgfqpoint{2.202280in}{2.376702in}}%
\pgfpathlineto{\pgfqpoint{2.202280in}{2.376702in}}%
\pgfpathclose%
\pgfusepath{stroke,fill}%
\end{pgfscope}%
\begin{pgfscope}%
\pgfpathrectangle{\pgfqpoint{0.786164in}{0.768110in}}{\pgfqpoint{8.851069in}{7.081890in}}%
\pgfusepath{clip}%
\pgfsetbuttcap%
\pgfsetroundjoin%
\definecolor{currentfill}{rgb}{0.225863,0.330805,0.547314}%
\pgfsetfillcolor{currentfill}%
\pgfsetfillopacity{0.700000}%
\pgfsetlinewidth{0.501875pt}%
\definecolor{currentstroke}{rgb}{1.000000,1.000000,1.000000}%
\pgfsetstrokecolor{currentstroke}%
\pgfsetstrokeopacity{0.700000}%
\pgfsetdash{}{0pt}%
\pgfpathmoveto{\pgfqpoint{2.321013in}{2.442397in}}%
\pgfpathcurveto{\pgfqpoint{2.334036in}{2.442397in}}{\pgfqpoint{2.346527in}{2.447571in}}{\pgfqpoint{2.355735in}{2.456779in}}%
\pgfpathcurveto{\pgfqpoint{2.364944in}{2.465988in}}{\pgfqpoint{2.370117in}{2.478479in}}{\pgfqpoint{2.370117in}{2.491502in}}%
\pgfpathcurveto{\pgfqpoint{2.370117in}{2.504524in}}{\pgfqpoint{2.364944in}{2.517015in}}{\pgfqpoint{2.355735in}{2.526224in}}%
\pgfpathcurveto{\pgfqpoint{2.346527in}{2.535432in}}{\pgfqpoint{2.334036in}{2.540606in}}{\pgfqpoint{2.321013in}{2.540606in}}%
\pgfpathcurveto{\pgfqpoint{2.307990in}{2.540606in}}{\pgfqpoint{2.295499in}{2.535432in}}{\pgfqpoint{2.286291in}{2.526224in}}%
\pgfpathcurveto{\pgfqpoint{2.277082in}{2.517015in}}{\pgfqpoint{2.271908in}{2.504524in}}{\pgfqpoint{2.271908in}{2.491502in}}%
\pgfpathcurveto{\pgfqpoint{2.271908in}{2.478479in}}{\pgfqpoint{2.277082in}{2.465988in}}{\pgfqpoint{2.286291in}{2.456779in}}%
\pgfpathcurveto{\pgfqpoint{2.295499in}{2.447571in}}{\pgfqpoint{2.307990in}{2.442397in}}{\pgfqpoint{2.321013in}{2.442397in}}%
\pgfpathlineto{\pgfqpoint{2.321013in}{2.442397in}}%
\pgfpathclose%
\pgfusepath{stroke,fill}%
\end{pgfscope}%
\begin{pgfscope}%
\pgfpathrectangle{\pgfqpoint{0.786164in}{0.768110in}}{\pgfqpoint{8.851069in}{7.081890in}}%
\pgfusepath{clip}%
\pgfsetbuttcap%
\pgfsetroundjoin%
\definecolor{currentfill}{rgb}{0.227802,0.326594,0.546532}%
\pgfsetfillcolor{currentfill}%
\pgfsetfillopacity{0.700000}%
\pgfsetlinewidth{0.501875pt}%
\definecolor{currentstroke}{rgb}{1.000000,1.000000,1.000000}%
\pgfsetstrokecolor{currentstroke}%
\pgfsetstrokeopacity{0.700000}%
\pgfsetdash{}{0pt}%
\pgfpathmoveto{\pgfqpoint{2.302746in}{2.420499in}}%
\pgfpathcurveto{\pgfqpoint{2.315769in}{2.420499in}}{\pgfqpoint{2.328260in}{2.425673in}}{\pgfqpoint{2.337468in}{2.434881in}}%
\pgfpathcurveto{\pgfqpoint{2.346677in}{2.444090in}}{\pgfqpoint{2.351851in}{2.456581in}}{\pgfqpoint{2.351851in}{2.469603in}}%
\pgfpathcurveto{\pgfqpoint{2.351851in}{2.482626in}}{\pgfqpoint{2.346677in}{2.495117in}}{\pgfqpoint{2.337468in}{2.504326in}}%
\pgfpathcurveto{\pgfqpoint{2.328260in}{2.513534in}}{\pgfqpoint{2.315769in}{2.518708in}}{\pgfqpoint{2.302746in}{2.518708in}}%
\pgfpathcurveto{\pgfqpoint{2.289724in}{2.518708in}}{\pgfqpoint{2.277232in}{2.513534in}}{\pgfqpoint{2.268024in}{2.504326in}}%
\pgfpathcurveto{\pgfqpoint{2.258816in}{2.495117in}}{\pgfqpoint{2.253642in}{2.482626in}}{\pgfqpoint{2.253642in}{2.469603in}}%
\pgfpathcurveto{\pgfqpoint{2.253642in}{2.456581in}}{\pgfqpoint{2.258816in}{2.444090in}}{\pgfqpoint{2.268024in}{2.434881in}}%
\pgfpathcurveto{\pgfqpoint{2.277232in}{2.425673in}}{\pgfqpoint{2.289724in}{2.420499in}}{\pgfqpoint{2.302746in}{2.420499in}}%
\pgfpathlineto{\pgfqpoint{2.302746in}{2.420499in}}%
\pgfpathclose%
\pgfusepath{stroke,fill}%
\end{pgfscope}%
\begin{pgfscope}%
\pgfpathrectangle{\pgfqpoint{0.786164in}{0.768110in}}{\pgfqpoint{8.851069in}{7.081890in}}%
\pgfusepath{clip}%
\pgfsetbuttcap%
\pgfsetroundjoin%
\definecolor{currentfill}{rgb}{0.221989,0.339161,0.548752}%
\pgfsetfillcolor{currentfill}%
\pgfsetfillopacity{0.700000}%
\pgfsetlinewidth{0.501875pt}%
\definecolor{currentstroke}{rgb}{1.000000,1.000000,1.000000}%
\pgfsetstrokecolor{currentstroke}%
\pgfsetstrokeopacity{0.700000}%
\pgfsetdash{}{0pt}%
\pgfpathmoveto{\pgfqpoint{2.211413in}{2.332906in}}%
\pgfpathcurveto{\pgfqpoint{2.224436in}{2.332906in}}{\pgfqpoint{2.236927in}{2.338080in}}{\pgfqpoint{2.246136in}{2.347288in}}%
\pgfpathcurveto{\pgfqpoint{2.255344in}{2.356497in}}{\pgfqpoint{2.260518in}{2.368988in}}{\pgfqpoint{2.260518in}{2.382010in}}%
\pgfpathcurveto{\pgfqpoint{2.260518in}{2.395033in}}{\pgfqpoint{2.255344in}{2.407524in}}{\pgfqpoint{2.246136in}{2.416733in}}%
\pgfpathcurveto{\pgfqpoint{2.236927in}{2.425941in}}{\pgfqpoint{2.224436in}{2.431115in}}{\pgfqpoint{2.211413in}{2.431115in}}%
\pgfpathcurveto{\pgfqpoint{2.198391in}{2.431115in}}{\pgfqpoint{2.185900in}{2.425941in}}{\pgfqpoint{2.176691in}{2.416733in}}%
\pgfpathcurveto{\pgfqpoint{2.167483in}{2.407524in}}{\pgfqpoint{2.162309in}{2.395033in}}{\pgfqpoint{2.162309in}{2.382010in}}%
\pgfpathcurveto{\pgfqpoint{2.162309in}{2.368988in}}{\pgfqpoint{2.167483in}{2.356497in}}{\pgfqpoint{2.176691in}{2.347288in}}%
\pgfpathcurveto{\pgfqpoint{2.185900in}{2.338080in}}{\pgfqpoint{2.198391in}{2.332906in}}{\pgfqpoint{2.211413in}{2.332906in}}%
\pgfpathlineto{\pgfqpoint{2.211413in}{2.332906in}}%
\pgfpathclose%
\pgfusepath{stroke,fill}%
\end{pgfscope}%
\begin{pgfscope}%
\pgfpathrectangle{\pgfqpoint{0.786164in}{0.768110in}}{\pgfqpoint{8.851069in}{7.081890in}}%
\pgfusepath{clip}%
\pgfsetbuttcap%
\pgfsetroundjoin%
\definecolor{currentfill}{rgb}{0.195860,0.395433,0.555276}%
\pgfsetfillcolor{currentfill}%
\pgfsetfillopacity{0.700000}%
\pgfsetlinewidth{0.501875pt}%
\definecolor{currentstroke}{rgb}{1.000000,1.000000,1.000000}%
\pgfsetstrokecolor{currentstroke}%
\pgfsetstrokeopacity{0.700000}%
\pgfsetdash{}{0pt}%
\pgfpathmoveto{\pgfqpoint{1.900882in}{2.092025in}}%
\pgfpathcurveto{\pgfqpoint{1.913904in}{2.092025in}}{\pgfqpoint{1.926395in}{2.097199in}}{\pgfqpoint{1.935604in}{2.106408in}}%
\pgfpathcurveto{\pgfqpoint{1.944812in}{2.115616in}}{\pgfqpoint{1.949986in}{2.128107in}}{\pgfqpoint{1.949986in}{2.141130in}}%
\pgfpathcurveto{\pgfqpoint{1.949986in}{2.154153in}}{\pgfqpoint{1.944812in}{2.166644in}}{\pgfqpoint{1.935604in}{2.175852in}}%
\pgfpathcurveto{\pgfqpoint{1.926395in}{2.185060in}}{\pgfqpoint{1.913904in}{2.190234in}}{\pgfqpoint{1.900882in}{2.190234in}}%
\pgfpathcurveto{\pgfqpoint{1.887859in}{2.190234in}}{\pgfqpoint{1.875368in}{2.185060in}}{\pgfqpoint{1.866159in}{2.175852in}}%
\pgfpathcurveto{\pgfqpoint{1.856951in}{2.166644in}}{\pgfqpoint{1.851777in}{2.154153in}}{\pgfqpoint{1.851777in}{2.141130in}}%
\pgfpathcurveto{\pgfqpoint{1.851777in}{2.128107in}}{\pgfqpoint{1.856951in}{2.115616in}}{\pgfqpoint{1.866159in}{2.106408in}}%
\pgfpathcurveto{\pgfqpoint{1.875368in}{2.097199in}}{\pgfqpoint{1.887859in}{2.092025in}}{\pgfqpoint{1.900882in}{2.092025in}}%
\pgfpathlineto{\pgfqpoint{1.900882in}{2.092025in}}%
\pgfpathclose%
\pgfusepath{stroke,fill}%
\end{pgfscope}%
\begin{pgfscope}%
\pgfpathrectangle{\pgfqpoint{0.786164in}{0.768110in}}{\pgfqpoint{8.851069in}{7.081890in}}%
\pgfusepath{clip}%
\pgfsetbuttcap%
\pgfsetroundjoin%
\definecolor{currentfill}{rgb}{0.199430,0.387607,0.554642}%
\pgfsetfillcolor{currentfill}%
\pgfsetfillopacity{0.700000}%
\pgfsetlinewidth{0.501875pt}%
\definecolor{currentstroke}{rgb}{1.000000,1.000000,1.000000}%
\pgfsetstrokecolor{currentstroke}%
\pgfsetstrokeopacity{0.700000}%
\pgfsetdash{}{0pt}%
\pgfpathmoveto{\pgfqpoint{2.019614in}{2.179618in}}%
\pgfpathcurveto{\pgfqpoint{2.032637in}{2.179618in}}{\pgfqpoint{2.045128in}{2.184792in}}{\pgfqpoint{2.054337in}{2.194001in}}%
\pgfpathcurveto{\pgfqpoint{2.063545in}{2.203209in}}{\pgfqpoint{2.068719in}{2.215700in}}{\pgfqpoint{2.068719in}{2.228723in}}%
\pgfpathcurveto{\pgfqpoint{2.068719in}{2.241745in}}{\pgfqpoint{2.063545in}{2.254237in}}{\pgfqpoint{2.054337in}{2.263445in}}%
\pgfpathcurveto{\pgfqpoint{2.045128in}{2.272653in}}{\pgfqpoint{2.032637in}{2.277827in}}{\pgfqpoint{2.019614in}{2.277827in}}%
\pgfpathcurveto{\pgfqpoint{2.006592in}{2.277827in}}{\pgfqpoint{1.994101in}{2.272653in}}{\pgfqpoint{1.984892in}{2.263445in}}%
\pgfpathcurveto{\pgfqpoint{1.975684in}{2.254237in}}{\pgfqpoint{1.970510in}{2.241745in}}{\pgfqpoint{1.970510in}{2.228723in}}%
\pgfpathcurveto{\pgfqpoint{1.970510in}{2.215700in}}{\pgfqpoint{1.975684in}{2.203209in}}{\pgfqpoint{1.984892in}{2.194001in}}%
\pgfpathcurveto{\pgfqpoint{1.994101in}{2.184792in}}{\pgfqpoint{2.006592in}{2.179618in}}{\pgfqpoint{2.019614in}{2.179618in}}%
\pgfpathlineto{\pgfqpoint{2.019614in}{2.179618in}}%
\pgfpathclose%
\pgfusepath{stroke,fill}%
\end{pgfscope}%
\begin{pgfscope}%
\pgfpathrectangle{\pgfqpoint{0.786164in}{0.768110in}}{\pgfqpoint{8.851069in}{7.081890in}}%
\pgfusepath{clip}%
\pgfsetbuttcap%
\pgfsetroundjoin%
\definecolor{currentfill}{rgb}{0.190631,0.407061,0.556089}%
\pgfsetfillcolor{currentfill}%
\pgfsetfillopacity{0.700000}%
\pgfsetlinewidth{0.501875pt}%
\definecolor{currentstroke}{rgb}{1.000000,1.000000,1.000000}%
\pgfsetstrokecolor{currentstroke}%
\pgfsetstrokeopacity{0.700000}%
\pgfsetdash{}{0pt}%
\pgfpathmoveto{\pgfqpoint{2.037881in}{2.223415in}}%
\pgfpathcurveto{\pgfqpoint{2.050904in}{2.223415in}}{\pgfqpoint{2.063395in}{2.228589in}}{\pgfqpoint{2.072603in}{2.237797in}}%
\pgfpathcurveto{\pgfqpoint{2.081812in}{2.247005in}}{\pgfqpoint{2.086986in}{2.259497in}}{\pgfqpoint{2.086986in}{2.272519in}}%
\pgfpathcurveto{\pgfqpoint{2.086986in}{2.285542in}}{\pgfqpoint{2.081812in}{2.298033in}}{\pgfqpoint{2.072603in}{2.307241in}}%
\pgfpathcurveto{\pgfqpoint{2.063395in}{2.316450in}}{\pgfqpoint{2.050904in}{2.321624in}}{\pgfqpoint{2.037881in}{2.321624in}}%
\pgfpathcurveto{\pgfqpoint{2.024858in}{2.321624in}}{\pgfqpoint{2.012367in}{2.316450in}}{\pgfqpoint{2.003159in}{2.307241in}}%
\pgfpathcurveto{\pgfqpoint{1.993950in}{2.298033in}}{\pgfqpoint{1.988776in}{2.285542in}}{\pgfqpoint{1.988776in}{2.272519in}}%
\pgfpathcurveto{\pgfqpoint{1.988776in}{2.259497in}}{\pgfqpoint{1.993950in}{2.247005in}}{\pgfqpoint{2.003159in}{2.237797in}}%
\pgfpathcurveto{\pgfqpoint{2.012367in}{2.228589in}}{\pgfqpoint{2.024858in}{2.223415in}}{\pgfqpoint{2.037881in}{2.223415in}}%
\pgfpathlineto{\pgfqpoint{2.037881in}{2.223415in}}%
\pgfpathclose%
\pgfusepath{stroke,fill}%
\end{pgfscope}%
\begin{pgfscope}%
\pgfpathrectangle{\pgfqpoint{0.786164in}{0.768110in}}{\pgfqpoint{8.851069in}{7.081890in}}%
\pgfusepath{clip}%
\pgfsetbuttcap%
\pgfsetroundjoin%
\definecolor{currentfill}{rgb}{0.147607,0.511733,0.557049}%
\pgfsetfillcolor{currentfill}%
\pgfsetfillopacity{0.700000}%
\pgfsetlinewidth{0.501875pt}%
\definecolor{currentstroke}{rgb}{1.000000,1.000000,1.000000}%
\pgfsetstrokecolor{currentstroke}%
\pgfsetstrokeopacity{0.700000}%
\pgfsetdash{}{0pt}%
\pgfpathmoveto{\pgfqpoint{6.257460in}{3.449716in}}%
\pgfpathcurveto{\pgfqpoint{6.270482in}{3.449716in}}{\pgfqpoint{6.282974in}{3.454890in}}{\pgfqpoint{6.292182in}{3.464098in}}%
\pgfpathcurveto{\pgfqpoint{6.301390in}{3.473307in}}{\pgfqpoint{6.306564in}{3.485798in}}{\pgfqpoint{6.306564in}{3.498820in}}%
\pgfpathcurveto{\pgfqpoint{6.306564in}{3.511843in}}{\pgfqpoint{6.301390in}{3.524334in}}{\pgfqpoint{6.292182in}{3.533543in}}%
\pgfpathcurveto{\pgfqpoint{6.282974in}{3.542751in}}{\pgfqpoint{6.270482in}{3.547925in}}{\pgfqpoint{6.257460in}{3.547925in}}%
\pgfpathcurveto{\pgfqpoint{6.244437in}{3.547925in}}{\pgfqpoint{6.231946in}{3.542751in}}{\pgfqpoint{6.222738in}{3.533543in}}%
\pgfpathcurveto{\pgfqpoint{6.213529in}{3.524334in}}{\pgfqpoint{6.208355in}{3.511843in}}{\pgfqpoint{6.208355in}{3.498820in}}%
\pgfpathcurveto{\pgfqpoint{6.208355in}{3.485798in}}{\pgfqpoint{6.213529in}{3.473307in}}{\pgfqpoint{6.222738in}{3.464098in}}%
\pgfpathcurveto{\pgfqpoint{6.231946in}{3.454890in}}{\pgfqpoint{6.244437in}{3.449716in}}{\pgfqpoint{6.257460in}{3.449716in}}%
\pgfpathlineto{\pgfqpoint{6.257460in}{3.449716in}}%
\pgfpathclose%
\pgfusepath{stroke,fill}%
\end{pgfscope}%
\begin{pgfscope}%
\pgfpathrectangle{\pgfqpoint{0.786164in}{0.768110in}}{\pgfqpoint{8.851069in}{7.081890in}}%
\pgfusepath{clip}%
\pgfsetbuttcap%
\pgfsetroundjoin%
\definecolor{currentfill}{rgb}{0.150476,0.504369,0.557430}%
\pgfsetfillcolor{currentfill}%
\pgfsetfillopacity{0.700000}%
\pgfsetlinewidth{0.501875pt}%
\definecolor{currentstroke}{rgb}{1.000000,1.000000,1.000000}%
\pgfsetstrokecolor{currentstroke}%
\pgfsetstrokeopacity{0.700000}%
\pgfsetdash{}{0pt}%
\pgfpathmoveto{\pgfqpoint{5.718596in}{2.946056in}}%
\pgfpathcurveto{\pgfqpoint{5.731618in}{2.946056in}}{\pgfqpoint{5.744110in}{2.951230in}}{\pgfqpoint{5.753318in}{2.960439in}}%
\pgfpathcurveto{\pgfqpoint{5.762526in}{2.969647in}}{\pgfqpoint{5.767700in}{2.982138in}}{\pgfqpoint{5.767700in}{2.995161in}}%
\pgfpathcurveto{\pgfqpoint{5.767700in}{3.008184in}}{\pgfqpoint{5.762526in}{3.020675in}}{\pgfqpoint{5.753318in}{3.029883in}}%
\pgfpathcurveto{\pgfqpoint{5.744110in}{3.039092in}}{\pgfqpoint{5.731618in}{3.044266in}}{\pgfqpoint{5.718596in}{3.044266in}}%
\pgfpathcurveto{\pgfqpoint{5.705573in}{3.044266in}}{\pgfqpoint{5.693082in}{3.039092in}}{\pgfqpoint{5.683874in}{3.029883in}}%
\pgfpathcurveto{\pgfqpoint{5.674665in}{3.020675in}}{\pgfqpoint{5.669491in}{3.008184in}}{\pgfqpoint{5.669491in}{2.995161in}}%
\pgfpathcurveto{\pgfqpoint{5.669491in}{2.982138in}}{\pgfqpoint{5.674665in}{2.969647in}}{\pgfqpoint{5.683874in}{2.960439in}}%
\pgfpathcurveto{\pgfqpoint{5.693082in}{2.951230in}}{\pgfqpoint{5.705573in}{2.946056in}}{\pgfqpoint{5.718596in}{2.946056in}}%
\pgfpathlineto{\pgfqpoint{5.718596in}{2.946056in}}%
\pgfpathclose%
\pgfusepath{stroke,fill}%
\end{pgfscope}%
\begin{pgfscope}%
\pgfpathrectangle{\pgfqpoint{0.786164in}{0.768110in}}{\pgfqpoint{8.851069in}{7.081890in}}%
\pgfusepath{clip}%
\pgfsetbuttcap%
\pgfsetroundjoin%
\definecolor{currentfill}{rgb}{0.143343,0.522773,0.556295}%
\pgfsetfillcolor{currentfill}%
\pgfsetfillopacity{0.700000}%
\pgfsetlinewidth{0.501875pt}%
\definecolor{currentstroke}{rgb}{1.000000,1.000000,1.000000}%
\pgfsetstrokecolor{currentstroke}%
\pgfsetstrokeopacity{0.700000}%
\pgfsetdash{}{0pt}%
\pgfpathmoveto{\pgfqpoint{6.230060in}{3.252632in}}%
\pgfpathcurveto{\pgfqpoint{6.243083in}{3.252632in}}{\pgfqpoint{6.255574in}{3.257806in}}{\pgfqpoint{6.264782in}{3.267014in}}%
\pgfpathcurveto{\pgfqpoint{6.273991in}{3.276223in}}{\pgfqpoint{6.279165in}{3.288714in}}{\pgfqpoint{6.279165in}{3.301736in}}%
\pgfpathcurveto{\pgfqpoint{6.279165in}{3.314759in}}{\pgfqpoint{6.273991in}{3.327250in}}{\pgfqpoint{6.264782in}{3.336459in}}%
\pgfpathcurveto{\pgfqpoint{6.255574in}{3.345667in}}{\pgfqpoint{6.243083in}{3.350841in}}{\pgfqpoint{6.230060in}{3.350841in}}%
\pgfpathcurveto{\pgfqpoint{6.217037in}{3.350841in}}{\pgfqpoint{6.204546in}{3.345667in}}{\pgfqpoint{6.195338in}{3.336459in}}%
\pgfpathcurveto{\pgfqpoint{6.186129in}{3.327250in}}{\pgfqpoint{6.180955in}{3.314759in}}{\pgfqpoint{6.180955in}{3.301736in}}%
\pgfpathcurveto{\pgfqpoint{6.180955in}{3.288714in}}{\pgfqpoint{6.186129in}{3.276223in}}{\pgfqpoint{6.195338in}{3.267014in}}%
\pgfpathcurveto{\pgfqpoint{6.204546in}{3.257806in}}{\pgfqpoint{6.217037in}{3.252632in}}{\pgfqpoint{6.230060in}{3.252632in}}%
\pgfpathlineto{\pgfqpoint{6.230060in}{3.252632in}}%
\pgfpathclose%
\pgfusepath{stroke,fill}%
\end{pgfscope}%
\begin{pgfscope}%
\pgfpathrectangle{\pgfqpoint{0.786164in}{0.768110in}}{\pgfqpoint{8.851069in}{7.081890in}}%
\pgfusepath{clip}%
\pgfsetbuttcap%
\pgfsetroundjoin%
\definecolor{currentfill}{rgb}{0.146180,0.515413,0.556823}%
\pgfsetfillcolor{currentfill}%
\pgfsetfillopacity{0.700000}%
\pgfsetlinewidth{0.501875pt}%
\definecolor{currentstroke}{rgb}{1.000000,1.000000,1.000000}%
\pgfsetstrokecolor{currentstroke}%
\pgfsetstrokeopacity{0.700000}%
\pgfsetdash{}{0pt}%
\pgfpathmoveto{\pgfqpoint{6.093061in}{3.581105in}}%
\pgfpathcurveto{\pgfqpoint{6.106083in}{3.581105in}}{\pgfqpoint{6.118574in}{3.586279in}}{\pgfqpoint{6.127783in}{3.595488in}}%
\pgfpathcurveto{\pgfqpoint{6.136991in}{3.604696in}}{\pgfqpoint{6.142165in}{3.617187in}}{\pgfqpoint{6.142165in}{3.630210in}}%
\pgfpathcurveto{\pgfqpoint{6.142165in}{3.643233in}}{\pgfqpoint{6.136991in}{3.655724in}}{\pgfqpoint{6.127783in}{3.664932in}}%
\pgfpathcurveto{\pgfqpoint{6.118574in}{3.674141in}}{\pgfqpoint{6.106083in}{3.679315in}}{\pgfqpoint{6.093061in}{3.679315in}}%
\pgfpathcurveto{\pgfqpoint{6.080038in}{3.679315in}}{\pgfqpoint{6.067547in}{3.674141in}}{\pgfqpoint{6.058338in}{3.664932in}}%
\pgfpathcurveto{\pgfqpoint{6.049130in}{3.655724in}}{\pgfqpoint{6.043956in}{3.643233in}}{\pgfqpoint{6.043956in}{3.630210in}}%
\pgfpathcurveto{\pgfqpoint{6.043956in}{3.617187in}}{\pgfqpoint{6.049130in}{3.604696in}}{\pgfqpoint{6.058338in}{3.595488in}}%
\pgfpathcurveto{\pgfqpoint{6.067547in}{3.586279in}}{\pgfqpoint{6.080038in}{3.581105in}}{\pgfqpoint{6.093061in}{3.581105in}}%
\pgfpathlineto{\pgfqpoint{6.093061in}{3.581105in}}%
\pgfpathclose%
\pgfusepath{stroke,fill}%
\end{pgfscope}%
\begin{pgfscope}%
\pgfpathrectangle{\pgfqpoint{0.786164in}{0.768110in}}{\pgfqpoint{8.851069in}{7.081890in}}%
\pgfusepath{clip}%
\pgfsetbuttcap%
\pgfsetroundjoin%
\definecolor{currentfill}{rgb}{0.137770,0.537492,0.554906}%
\pgfsetfillcolor{currentfill}%
\pgfsetfillopacity{0.700000}%
\pgfsetlinewidth{0.501875pt}%
\definecolor{currentstroke}{rgb}{1.000000,1.000000,1.000000}%
\pgfsetstrokecolor{currentstroke}%
\pgfsetstrokeopacity{0.700000}%
\pgfsetdash{}{0pt}%
\pgfpathmoveto{\pgfqpoint{5.809929in}{3.143141in}}%
\pgfpathcurveto{\pgfqpoint{5.822951in}{3.143141in}}{\pgfqpoint{5.835442in}{3.148314in}}{\pgfqpoint{5.844651in}{3.157523in}}%
\pgfpathcurveto{\pgfqpoint{5.853859in}{3.166731in}}{\pgfqpoint{5.859033in}{3.179222in}}{\pgfqpoint{5.859033in}{3.192245in}}%
\pgfpathcurveto{\pgfqpoint{5.859033in}{3.205268in}}{\pgfqpoint{5.853859in}{3.217759in}}{\pgfqpoint{5.844651in}{3.226967in}}%
\pgfpathcurveto{\pgfqpoint{5.835442in}{3.236176in}}{\pgfqpoint{5.822951in}{3.241350in}}{\pgfqpoint{5.809929in}{3.241350in}}%
\pgfpathcurveto{\pgfqpoint{5.796906in}{3.241350in}}{\pgfqpoint{5.784415in}{3.236176in}}{\pgfqpoint{5.775206in}{3.226967in}}%
\pgfpathcurveto{\pgfqpoint{5.765998in}{3.217759in}}{\pgfqpoint{5.760824in}{3.205268in}}{\pgfqpoint{5.760824in}{3.192245in}}%
\pgfpathcurveto{\pgfqpoint{5.760824in}{3.179222in}}{\pgfqpoint{5.765998in}{3.166731in}}{\pgfqpoint{5.775206in}{3.157523in}}%
\pgfpathcurveto{\pgfqpoint{5.784415in}{3.148314in}}{\pgfqpoint{5.796906in}{3.143141in}}{\pgfqpoint{5.809929in}{3.143141in}}%
\pgfpathlineto{\pgfqpoint{5.809929in}{3.143141in}}%
\pgfpathclose%
\pgfusepath{stroke,fill}%
\end{pgfscope}%
\begin{pgfscope}%
\pgfpathrectangle{\pgfqpoint{0.786164in}{0.768110in}}{\pgfqpoint{8.851069in}{7.081890in}}%
\pgfusepath{clip}%
\pgfsetbuttcap%
\pgfsetroundjoin%
\definecolor{currentfill}{rgb}{0.137770,0.537492,0.554906}%
\pgfsetfillcolor{currentfill}%
\pgfsetfillopacity{0.700000}%
\pgfsetlinewidth{0.501875pt}%
\definecolor{currentstroke}{rgb}{1.000000,1.000000,1.000000}%
\pgfsetstrokecolor{currentstroke}%
\pgfsetstrokeopacity{0.700000}%
\pgfsetdash{}{0pt}%
\pgfpathmoveto{\pgfqpoint{5.408064in}{2.464295in}}%
\pgfpathcurveto{\pgfqpoint{5.421087in}{2.464295in}}{\pgfqpoint{5.433578in}{2.469469in}}{\pgfqpoint{5.442786in}{2.478678in}}%
\pgfpathcurveto{\pgfqpoint{5.451995in}{2.487886in}}{\pgfqpoint{5.457169in}{2.500377in}}{\pgfqpoint{5.457169in}{2.513400in}}%
\pgfpathcurveto{\pgfqpoint{5.457169in}{2.526423in}}{\pgfqpoint{5.451995in}{2.538914in}}{\pgfqpoint{5.442786in}{2.548122in}}%
\pgfpathcurveto{\pgfqpoint{5.433578in}{2.557331in}}{\pgfqpoint{5.421087in}{2.562504in}}{\pgfqpoint{5.408064in}{2.562504in}}%
\pgfpathcurveto{\pgfqpoint{5.395041in}{2.562504in}}{\pgfqpoint{5.382550in}{2.557331in}}{\pgfqpoint{5.373342in}{2.548122in}}%
\pgfpathcurveto{\pgfqpoint{5.364133in}{2.538914in}}{\pgfqpoint{5.358959in}{2.526423in}}{\pgfqpoint{5.358959in}{2.513400in}}%
\pgfpathcurveto{\pgfqpoint{5.358959in}{2.500377in}}{\pgfqpoint{5.364133in}{2.487886in}}{\pgfqpoint{5.373342in}{2.478678in}}%
\pgfpathcurveto{\pgfqpoint{5.382550in}{2.469469in}}{\pgfqpoint{5.395041in}{2.464295in}}{\pgfqpoint{5.408064in}{2.464295in}}%
\pgfpathlineto{\pgfqpoint{5.408064in}{2.464295in}}%
\pgfpathclose%
\pgfusepath{stroke,fill}%
\end{pgfscope}%
\begin{pgfscope}%
\pgfpathrectangle{\pgfqpoint{0.786164in}{0.768110in}}{\pgfqpoint{8.851069in}{7.081890in}}%
\pgfusepath{clip}%
\pgfsetbuttcap%
\pgfsetroundjoin%
\definecolor{currentfill}{rgb}{0.132444,0.552216,0.553018}%
\pgfsetfillcolor{currentfill}%
\pgfsetfillopacity{0.700000}%
\pgfsetlinewidth{0.501875pt}%
\definecolor{currentstroke}{rgb}{1.000000,1.000000,1.000000}%
\pgfsetstrokecolor{currentstroke}%
\pgfsetstrokeopacity{0.700000}%
\pgfsetdash{}{0pt}%
\pgfpathmoveto{\pgfqpoint{5.910395in}{3.252632in}}%
\pgfpathcurveto{\pgfqpoint{5.923418in}{3.252632in}}{\pgfqpoint{5.935909in}{3.257806in}}{\pgfqpoint{5.945117in}{3.267014in}}%
\pgfpathcurveto{\pgfqpoint{5.954325in}{3.276223in}}{\pgfqpoint{5.959499in}{3.288714in}}{\pgfqpoint{5.959499in}{3.301736in}}%
\pgfpathcurveto{\pgfqpoint{5.959499in}{3.314759in}}{\pgfqpoint{5.954325in}{3.327250in}}{\pgfqpoint{5.945117in}{3.336459in}}%
\pgfpathcurveto{\pgfqpoint{5.935909in}{3.345667in}}{\pgfqpoint{5.923418in}{3.350841in}}{\pgfqpoint{5.910395in}{3.350841in}}%
\pgfpathcurveto{\pgfqpoint{5.897372in}{3.350841in}}{\pgfqpoint{5.884881in}{3.345667in}}{\pgfqpoint{5.875673in}{3.336459in}}%
\pgfpathcurveto{\pgfqpoint{5.866464in}{3.327250in}}{\pgfqpoint{5.861290in}{3.314759in}}{\pgfqpoint{5.861290in}{3.301736in}}%
\pgfpathcurveto{\pgfqpoint{5.861290in}{3.288714in}}{\pgfqpoint{5.866464in}{3.276223in}}{\pgfqpoint{5.875673in}{3.267014in}}%
\pgfpathcurveto{\pgfqpoint{5.884881in}{3.257806in}}{\pgfqpoint{5.897372in}{3.252632in}}{\pgfqpoint{5.910395in}{3.252632in}}%
\pgfpathlineto{\pgfqpoint{5.910395in}{3.252632in}}%
\pgfpathclose%
\pgfusepath{stroke,fill}%
\end{pgfscope}%
\begin{pgfscope}%
\pgfpathrectangle{\pgfqpoint{0.786164in}{0.768110in}}{\pgfqpoint{8.851069in}{7.081890in}}%
\pgfusepath{clip}%
\pgfsetbuttcap%
\pgfsetroundjoin%
\definecolor{currentfill}{rgb}{0.131172,0.555899,0.552459}%
\pgfsetfillcolor{currentfill}%
\pgfsetfillopacity{0.700000}%
\pgfsetlinewidth{0.501875pt}%
\definecolor{currentstroke}{rgb}{1.000000,1.000000,1.000000}%
\pgfsetstrokecolor{currentstroke}%
\pgfsetstrokeopacity{0.700000}%
\pgfsetdash{}{0pt}%
\pgfpathmoveto{\pgfqpoint{5.672929in}{2.967955in}}%
\pgfpathcurveto{\pgfqpoint{5.685952in}{2.967955in}}{\pgfqpoint{5.698443in}{2.973129in}}{\pgfqpoint{5.707652in}{2.982337in}}%
\pgfpathcurveto{\pgfqpoint{5.716860in}{2.991545in}}{\pgfqpoint{5.722034in}{3.004037in}}{\pgfqpoint{5.722034in}{3.017059in}}%
\pgfpathcurveto{\pgfqpoint{5.722034in}{3.030082in}}{\pgfqpoint{5.716860in}{3.042573in}}{\pgfqpoint{5.707652in}{3.051781in}}%
\pgfpathcurveto{\pgfqpoint{5.698443in}{3.060990in}}{\pgfqpoint{5.685952in}{3.066164in}}{\pgfqpoint{5.672929in}{3.066164in}}%
\pgfpathcurveto{\pgfqpoint{5.659907in}{3.066164in}}{\pgfqpoint{5.647416in}{3.060990in}}{\pgfqpoint{5.638207in}{3.051781in}}%
\pgfpathcurveto{\pgfqpoint{5.628999in}{3.042573in}}{\pgfqpoint{5.623825in}{3.030082in}}{\pgfqpoint{5.623825in}{3.017059in}}%
\pgfpathcurveto{\pgfqpoint{5.623825in}{3.004037in}}{\pgfqpoint{5.628999in}{2.991545in}}{\pgfqpoint{5.638207in}{2.982337in}}%
\pgfpathcurveto{\pgfqpoint{5.647416in}{2.973129in}}{\pgfqpoint{5.659907in}{2.967955in}}{\pgfqpoint{5.672929in}{2.967955in}}%
\pgfpathlineto{\pgfqpoint{5.672929in}{2.967955in}}%
\pgfpathclose%
\pgfusepath{stroke,fill}%
\end{pgfscope}%
\begin{pgfscope}%
\pgfpathrectangle{\pgfqpoint{0.786164in}{0.768110in}}{\pgfqpoint{8.851069in}{7.081890in}}%
\pgfusepath{clip}%
\pgfsetbuttcap%
\pgfsetroundjoin%
\definecolor{currentfill}{rgb}{0.123463,0.581687,0.547445}%
\pgfsetfillcolor{currentfill}%
\pgfsetfillopacity{0.700000}%
\pgfsetlinewidth{0.501875pt}%
\definecolor{currentstroke}{rgb}{1.000000,1.000000,1.000000}%
\pgfsetstrokecolor{currentstroke}%
\pgfsetstrokeopacity{0.700000}%
\pgfsetdash{}{0pt}%
\pgfpathmoveto{\pgfqpoint{5.435464in}{2.661379in}}%
\pgfpathcurveto{\pgfqpoint{5.448487in}{2.661379in}}{\pgfqpoint{5.460978in}{2.666553in}}{\pgfqpoint{5.470186in}{2.675762in}}%
\pgfpathcurveto{\pgfqpoint{5.479395in}{2.684970in}}{\pgfqpoint{5.484569in}{2.697461in}}{\pgfqpoint{5.484569in}{2.710484in}}%
\pgfpathcurveto{\pgfqpoint{5.484569in}{2.723507in}}{\pgfqpoint{5.479395in}{2.735998in}}{\pgfqpoint{5.470186in}{2.745206in}}%
\pgfpathcurveto{\pgfqpoint{5.460978in}{2.754415in}}{\pgfqpoint{5.448487in}{2.759589in}}{\pgfqpoint{5.435464in}{2.759589in}}%
\pgfpathcurveto{\pgfqpoint{5.422441in}{2.759589in}}{\pgfqpoint{5.409950in}{2.754415in}}{\pgfqpoint{5.400742in}{2.745206in}}%
\pgfpathcurveto{\pgfqpoint{5.391533in}{2.735998in}}{\pgfqpoint{5.386359in}{2.723507in}}{\pgfqpoint{5.386359in}{2.710484in}}%
\pgfpathcurveto{\pgfqpoint{5.386359in}{2.697461in}}{\pgfqpoint{5.391533in}{2.684970in}}{\pgfqpoint{5.400742in}{2.675762in}}%
\pgfpathcurveto{\pgfqpoint{5.409950in}{2.666553in}}{\pgfqpoint{5.422441in}{2.661379in}}{\pgfqpoint{5.435464in}{2.661379in}}%
\pgfpathlineto{\pgfqpoint{5.435464in}{2.661379in}}%
\pgfpathclose%
\pgfusepath{stroke,fill}%
\end{pgfscope}%
\begin{pgfscope}%
\pgfpathrectangle{\pgfqpoint{0.786164in}{0.768110in}}{\pgfqpoint{8.851069in}{7.081890in}}%
\pgfusepath{clip}%
\pgfsetbuttcap%
\pgfsetroundjoin%
\definecolor{currentfill}{rgb}{0.119483,0.614817,0.537692}%
\pgfsetfillcolor{currentfill}%
\pgfsetfillopacity{0.700000}%
\pgfsetlinewidth{0.501875pt}%
\definecolor{currentstroke}{rgb}{1.000000,1.000000,1.000000}%
\pgfsetstrokecolor{currentstroke}%
\pgfsetstrokeopacity{0.700000}%
\pgfsetdash{}{0pt}%
\pgfpathmoveto{\pgfqpoint{5.243665in}{2.946056in}}%
\pgfpathcurveto{\pgfqpoint{5.256688in}{2.946056in}}{\pgfqpoint{5.269179in}{2.951230in}}{\pgfqpoint{5.278387in}{2.960439in}}%
\pgfpathcurveto{\pgfqpoint{5.287596in}{2.969647in}}{\pgfqpoint{5.292769in}{2.982138in}}{\pgfqpoint{5.292769in}{2.995161in}}%
\pgfpathcurveto{\pgfqpoint{5.292769in}{3.008184in}}{\pgfqpoint{5.287596in}{3.020675in}}{\pgfqpoint{5.278387in}{3.029883in}}%
\pgfpathcurveto{\pgfqpoint{5.269179in}{3.039092in}}{\pgfqpoint{5.256688in}{3.044266in}}{\pgfqpoint{5.243665in}{3.044266in}}%
\pgfpathcurveto{\pgfqpoint{5.230642in}{3.044266in}}{\pgfqpoint{5.218151in}{3.039092in}}{\pgfqpoint{5.208943in}{3.029883in}}%
\pgfpathcurveto{\pgfqpoint{5.199734in}{3.020675in}}{\pgfqpoint{5.194560in}{3.008184in}}{\pgfqpoint{5.194560in}{2.995161in}}%
\pgfpathcurveto{\pgfqpoint{5.194560in}{2.982138in}}{\pgfqpoint{5.199734in}{2.969647in}}{\pgfqpoint{5.208943in}{2.960439in}}%
\pgfpathcurveto{\pgfqpoint{5.218151in}{2.951230in}}{\pgfqpoint{5.230642in}{2.946056in}}{\pgfqpoint{5.243665in}{2.946056in}}%
\pgfpathlineto{\pgfqpoint{5.243665in}{2.946056in}}%
\pgfpathclose%
\pgfusepath{stroke,fill}%
\end{pgfscope}%
\begin{pgfscope}%
\pgfpathrectangle{\pgfqpoint{0.786164in}{0.768110in}}{\pgfqpoint{8.851069in}{7.081890in}}%
\pgfusepath{clip}%
\pgfsetbuttcap%
\pgfsetroundjoin%
\definecolor{currentfill}{rgb}{0.126326,0.644107,0.525311}%
\pgfsetfillcolor{currentfill}%
\pgfsetfillopacity{0.700000}%
\pgfsetlinewidth{0.501875pt}%
\definecolor{currentstroke}{rgb}{1.000000,1.000000,1.000000}%
\pgfsetstrokecolor{currentstroke}%
\pgfsetstrokeopacity{0.700000}%
\pgfsetdash{}{0pt}%
\pgfpathmoveto{\pgfqpoint{5.334998in}{2.727074in}}%
\pgfpathcurveto{\pgfqpoint{5.348020in}{2.727074in}}{\pgfqpoint{5.360511in}{2.732248in}}{\pgfqpoint{5.369720in}{2.741456in}}%
\pgfpathcurveto{\pgfqpoint{5.378928in}{2.750665in}}{\pgfqpoint{5.384102in}{2.763156in}}{\pgfqpoint{5.384102in}{2.776179in}}%
\pgfpathcurveto{\pgfqpoint{5.384102in}{2.789201in}}{\pgfqpoint{5.378928in}{2.801692in}}{\pgfqpoint{5.369720in}{2.810901in}}%
\pgfpathcurveto{\pgfqpoint{5.360511in}{2.820109in}}{\pgfqpoint{5.348020in}{2.825283in}}{\pgfqpoint{5.334998in}{2.825283in}}%
\pgfpathcurveto{\pgfqpoint{5.321975in}{2.825283in}}{\pgfqpoint{5.309484in}{2.820109in}}{\pgfqpoint{5.300275in}{2.810901in}}%
\pgfpathcurveto{\pgfqpoint{5.291067in}{2.801692in}}{\pgfqpoint{5.285893in}{2.789201in}}{\pgfqpoint{5.285893in}{2.776179in}}%
\pgfpathcurveto{\pgfqpoint{5.285893in}{2.763156in}}{\pgfqpoint{5.291067in}{2.750665in}}{\pgfqpoint{5.300275in}{2.741456in}}%
\pgfpathcurveto{\pgfqpoint{5.309484in}{2.732248in}}{\pgfqpoint{5.321975in}{2.727074in}}{\pgfqpoint{5.334998in}{2.727074in}}%
\pgfpathlineto{\pgfqpoint{5.334998in}{2.727074in}}%
\pgfpathclose%
\pgfusepath{stroke,fill}%
\end{pgfscope}%
\begin{pgfscope}%
\pgfpathrectangle{\pgfqpoint{0.786164in}{0.768110in}}{\pgfqpoint{8.851069in}{7.081890in}}%
\pgfusepath{clip}%
\pgfsetbuttcap%
\pgfsetroundjoin%
\definecolor{currentfill}{rgb}{0.130067,0.651384,0.521608}%
\pgfsetfillcolor{currentfill}%
\pgfsetfillopacity{0.700000}%
\pgfsetlinewidth{0.501875pt}%
\definecolor{currentstroke}{rgb}{1.000000,1.000000,1.000000}%
\pgfsetstrokecolor{currentstroke}%
\pgfsetstrokeopacity{0.700000}%
\pgfsetdash{}{0pt}%
\pgfpathmoveto{\pgfqpoint{5.070132in}{2.770871in}}%
\pgfpathcurveto{\pgfqpoint{5.083155in}{2.770871in}}{\pgfqpoint{5.095646in}{2.776044in}}{\pgfqpoint{5.104855in}{2.785253in}}%
\pgfpathcurveto{\pgfqpoint{5.114063in}{2.794461in}}{\pgfqpoint{5.119237in}{2.806952in}}{\pgfqpoint{5.119237in}{2.819975in}}%
\pgfpathcurveto{\pgfqpoint{5.119237in}{2.832998in}}{\pgfqpoint{5.114063in}{2.845489in}}{\pgfqpoint{5.104855in}{2.854697in}}%
\pgfpathcurveto{\pgfqpoint{5.095646in}{2.863906in}}{\pgfqpoint{5.083155in}{2.869080in}}{\pgfqpoint{5.070132in}{2.869080in}}%
\pgfpathcurveto{\pgfqpoint{5.057110in}{2.869080in}}{\pgfqpoint{5.044619in}{2.863906in}}{\pgfqpoint{5.035410in}{2.854697in}}%
\pgfpathcurveto{\pgfqpoint{5.026202in}{2.845489in}}{\pgfqpoint{5.021028in}{2.832998in}}{\pgfqpoint{5.021028in}{2.819975in}}%
\pgfpathcurveto{\pgfqpoint{5.021028in}{2.806952in}}{\pgfqpoint{5.026202in}{2.794461in}}{\pgfqpoint{5.035410in}{2.785253in}}%
\pgfpathcurveto{\pgfqpoint{5.044619in}{2.776044in}}{\pgfqpoint{5.057110in}{2.770871in}}{\pgfqpoint{5.070132in}{2.770871in}}%
\pgfpathlineto{\pgfqpoint{5.070132in}{2.770871in}}%
\pgfpathclose%
\pgfusepath{stroke,fill}%
\end{pgfscope}%
\begin{pgfscope}%
\pgfpathrectangle{\pgfqpoint{0.786164in}{0.768110in}}{\pgfqpoint{8.851069in}{7.081890in}}%
\pgfusepath{clip}%
\pgfsetbuttcap%
\pgfsetroundjoin%
\definecolor{currentfill}{rgb}{0.128087,0.647749,0.523491}%
\pgfsetfillcolor{currentfill}%
\pgfsetfillopacity{0.700000}%
\pgfsetlinewidth{0.501875pt}%
\definecolor{currentstroke}{rgb}{1.000000,1.000000,1.000000}%
\pgfsetstrokecolor{currentstroke}%
\pgfsetstrokeopacity{0.700000}%
\pgfsetdash{}{0pt}%
\pgfpathmoveto{\pgfqpoint{5.234532in}{3.033649in}}%
\pgfpathcurveto{\pgfqpoint{5.247554in}{3.033649in}}{\pgfqpoint{5.260045in}{3.038823in}}{\pgfqpoint{5.269254in}{3.048032in}}%
\pgfpathcurveto{\pgfqpoint{5.278462in}{3.057240in}}{\pgfqpoint{5.283636in}{3.069731in}}{\pgfqpoint{5.283636in}{3.082754in}}%
\pgfpathcurveto{\pgfqpoint{5.283636in}{3.095777in}}{\pgfqpoint{5.278462in}{3.108268in}}{\pgfqpoint{5.269254in}{3.117476in}}%
\pgfpathcurveto{\pgfqpoint{5.260045in}{3.126685in}}{\pgfqpoint{5.247554in}{3.131859in}}{\pgfqpoint{5.234532in}{3.131859in}}%
\pgfpathcurveto{\pgfqpoint{5.221509in}{3.131859in}}{\pgfqpoint{5.209018in}{3.126685in}}{\pgfqpoint{5.199809in}{3.117476in}}%
\pgfpathcurveto{\pgfqpoint{5.190601in}{3.108268in}}{\pgfqpoint{5.185427in}{3.095777in}}{\pgfqpoint{5.185427in}{3.082754in}}%
\pgfpathcurveto{\pgfqpoint{5.185427in}{3.069731in}}{\pgfqpoint{5.190601in}{3.057240in}}{\pgfqpoint{5.199809in}{3.048032in}}%
\pgfpathcurveto{\pgfqpoint{5.209018in}{3.038823in}}{\pgfqpoint{5.221509in}{3.033649in}}{\pgfqpoint{5.234532in}{3.033649in}}%
\pgfpathlineto{\pgfqpoint{5.234532in}{3.033649in}}%
\pgfpathclose%
\pgfusepath{stroke,fill}%
\end{pgfscope}%
\begin{pgfscope}%
\pgfpathrectangle{\pgfqpoint{0.786164in}{0.768110in}}{\pgfqpoint{8.851069in}{7.081890in}}%
\pgfusepath{clip}%
\pgfsetbuttcap%
\pgfsetroundjoin%
\definecolor{currentfill}{rgb}{0.150148,0.676631,0.506589}%
\pgfsetfillcolor{currentfill}%
\pgfsetfillopacity{0.700000}%
\pgfsetlinewidth{0.501875pt}%
\definecolor{currentstroke}{rgb}{1.000000,1.000000,1.000000}%
\pgfsetstrokecolor{currentstroke}%
\pgfsetstrokeopacity{0.700000}%
\pgfsetdash{}{0pt}%
\pgfpathmoveto{\pgfqpoint{5.207132in}{3.011751in}}%
\pgfpathcurveto{\pgfqpoint{5.220154in}{3.011751in}}{\pgfqpoint{5.232645in}{3.016925in}}{\pgfqpoint{5.241854in}{3.026134in}}%
\pgfpathcurveto{\pgfqpoint{5.251062in}{3.035342in}}{\pgfqpoint{5.256236in}{3.047833in}}{\pgfqpoint{5.256236in}{3.060856in}}%
\pgfpathcurveto{\pgfqpoint{5.256236in}{3.073878in}}{\pgfqpoint{5.251062in}{3.086370in}}{\pgfqpoint{5.241854in}{3.095578in}}%
\pgfpathcurveto{\pgfqpoint{5.232645in}{3.104786in}}{\pgfqpoint{5.220154in}{3.109960in}}{\pgfqpoint{5.207132in}{3.109960in}}%
\pgfpathcurveto{\pgfqpoint{5.194109in}{3.109960in}}{\pgfqpoint{5.181618in}{3.104786in}}{\pgfqpoint{5.172409in}{3.095578in}}%
\pgfpathcurveto{\pgfqpoint{5.163201in}{3.086370in}}{\pgfqpoint{5.158027in}{3.073878in}}{\pgfqpoint{5.158027in}{3.060856in}}%
\pgfpathcurveto{\pgfqpoint{5.158027in}{3.047833in}}{\pgfqpoint{5.163201in}{3.035342in}}{\pgfqpoint{5.172409in}{3.026134in}}%
\pgfpathcurveto{\pgfqpoint{5.181618in}{3.016925in}}{\pgfqpoint{5.194109in}{3.011751in}}{\pgfqpoint{5.207132in}{3.011751in}}%
\pgfpathlineto{\pgfqpoint{5.207132in}{3.011751in}}%
\pgfpathclose%
\pgfusepath{stroke,fill}%
\end{pgfscope}%
\begin{pgfscope}%
\pgfpathrectangle{\pgfqpoint{0.786164in}{0.768110in}}{\pgfqpoint{8.851069in}{7.081890in}}%
\pgfusepath{clip}%
\pgfsetbuttcap%
\pgfsetroundjoin%
\definecolor{currentfill}{rgb}{0.153894,0.680203,0.504172}%
\pgfsetfillcolor{currentfill}%
\pgfsetfillopacity{0.700000}%
\pgfsetlinewidth{0.501875pt}%
\definecolor{currentstroke}{rgb}{1.000000,1.000000,1.000000}%
\pgfsetstrokecolor{currentstroke}%
\pgfsetstrokeopacity{0.700000}%
\pgfsetdash{}{0pt}%
\pgfpathmoveto{\pgfqpoint{5.307598in}{3.427818in}}%
\pgfpathcurveto{\pgfqpoint{5.320621in}{3.427818in}}{\pgfqpoint{5.333112in}{3.432992in}}{\pgfqpoint{5.342320in}{3.442200in}}%
\pgfpathcurveto{\pgfqpoint{5.351529in}{3.451408in}}{\pgfqpoint{5.356702in}{3.463900in}}{\pgfqpoint{5.356702in}{3.476922in}}%
\pgfpathcurveto{\pgfqpoint{5.356702in}{3.489945in}}{\pgfqpoint{5.351529in}{3.502436in}}{\pgfqpoint{5.342320in}{3.511644in}}%
\pgfpathcurveto{\pgfqpoint{5.333112in}{3.520853in}}{\pgfqpoint{5.320621in}{3.526027in}}{\pgfqpoint{5.307598in}{3.526027in}}%
\pgfpathcurveto{\pgfqpoint{5.294575in}{3.526027in}}{\pgfqpoint{5.282084in}{3.520853in}}{\pgfqpoint{5.272876in}{3.511644in}}%
\pgfpathcurveto{\pgfqpoint{5.263667in}{3.502436in}}{\pgfqpoint{5.258493in}{3.489945in}}{\pgfqpoint{5.258493in}{3.476922in}}%
\pgfpathcurveto{\pgfqpoint{5.258493in}{3.463900in}}{\pgfqpoint{5.263667in}{3.451408in}}{\pgfqpoint{5.272876in}{3.442200in}}%
\pgfpathcurveto{\pgfqpoint{5.282084in}{3.432992in}}{\pgfqpoint{5.294575in}{3.427818in}}{\pgfqpoint{5.307598in}{3.427818in}}%
\pgfpathlineto{\pgfqpoint{5.307598in}{3.427818in}}%
\pgfpathclose%
\pgfusepath{stroke,fill}%
\end{pgfscope}%
\begin{pgfscope}%
\pgfpathrectangle{\pgfqpoint{0.786164in}{0.768110in}}{\pgfqpoint{8.851069in}{7.081890in}}%
\pgfusepath{clip}%
\pgfsetbuttcap%
\pgfsetroundjoin%
\definecolor{currentfill}{rgb}{0.180653,0.701402,0.488189}%
\pgfsetfillcolor{currentfill}%
\pgfsetfillopacity{0.700000}%
\pgfsetlinewidth{0.501875pt}%
\definecolor{currentstroke}{rgb}{1.000000,1.000000,1.000000}%
\pgfsetstrokecolor{currentstroke}%
\pgfsetstrokeopacity{0.700000}%
\pgfsetdash{}{0pt}%
\pgfpathmoveto{\pgfqpoint{5.179732in}{3.077446in}}%
\pgfpathcurveto{\pgfqpoint{5.192755in}{3.077446in}}{\pgfqpoint{5.205246in}{3.082620in}}{\pgfqpoint{5.214454in}{3.091828in}}%
\pgfpathcurveto{\pgfqpoint{5.223662in}{3.101037in}}{\pgfqpoint{5.228836in}{3.113528in}}{\pgfqpoint{5.228836in}{3.126550in}}%
\pgfpathcurveto{\pgfqpoint{5.228836in}{3.139573in}}{\pgfqpoint{5.223662in}{3.152064in}}{\pgfqpoint{5.214454in}{3.161273in}}%
\pgfpathcurveto{\pgfqpoint{5.205246in}{3.170481in}}{\pgfqpoint{5.192755in}{3.175655in}}{\pgfqpoint{5.179732in}{3.175655in}}%
\pgfpathcurveto{\pgfqpoint{5.166709in}{3.175655in}}{\pgfqpoint{5.154218in}{3.170481in}}{\pgfqpoint{5.145010in}{3.161273in}}%
\pgfpathcurveto{\pgfqpoint{5.135801in}{3.152064in}}{\pgfqpoint{5.130627in}{3.139573in}}{\pgfqpoint{5.130627in}{3.126550in}}%
\pgfpathcurveto{\pgfqpoint{5.130627in}{3.113528in}}{\pgfqpoint{5.135801in}{3.101037in}}{\pgfqpoint{5.145010in}{3.091828in}}%
\pgfpathcurveto{\pgfqpoint{5.154218in}{3.082620in}}{\pgfqpoint{5.166709in}{3.077446in}}{\pgfqpoint{5.179732in}{3.077446in}}%
\pgfpathlineto{\pgfqpoint{5.179732in}{3.077446in}}%
\pgfpathclose%
\pgfusepath{stroke,fill}%
\end{pgfscope}%
\begin{pgfscope}%
\pgfpathrectangle{\pgfqpoint{0.786164in}{0.768110in}}{\pgfqpoint{8.851069in}{7.081890in}}%
\pgfusepath{clip}%
\pgfsetbuttcap%
\pgfsetroundjoin%
\definecolor{currentfill}{rgb}{0.208030,0.718701,0.472873}%
\pgfsetfillcolor{currentfill}%
\pgfsetfillopacity{0.700000}%
\pgfsetlinewidth{0.501875pt}%
\definecolor{currentstroke}{rgb}{1.000000,1.000000,1.000000}%
\pgfsetstrokecolor{currentstroke}%
\pgfsetstrokeopacity{0.700000}%
\pgfsetdash{}{0pt}%
\pgfpathmoveto{\pgfqpoint{4.814400in}{2.727074in}}%
\pgfpathcurveto{\pgfqpoint{4.827423in}{2.727074in}}{\pgfqpoint{4.839914in}{2.732248in}}{\pgfqpoint{4.849123in}{2.741456in}}%
\pgfpathcurveto{\pgfqpoint{4.858331in}{2.750665in}}{\pgfqpoint{4.863505in}{2.763156in}}{\pgfqpoint{4.863505in}{2.776179in}}%
\pgfpathcurveto{\pgfqpoint{4.863505in}{2.789201in}}{\pgfqpoint{4.858331in}{2.801692in}}{\pgfqpoint{4.849123in}{2.810901in}}%
\pgfpathcurveto{\pgfqpoint{4.839914in}{2.820109in}}{\pgfqpoint{4.827423in}{2.825283in}}{\pgfqpoint{4.814400in}{2.825283in}}%
\pgfpathcurveto{\pgfqpoint{4.801378in}{2.825283in}}{\pgfqpoint{4.788887in}{2.820109in}}{\pgfqpoint{4.779678in}{2.810901in}}%
\pgfpathcurveto{\pgfqpoint{4.770470in}{2.801692in}}{\pgfqpoint{4.765296in}{2.789201in}}{\pgfqpoint{4.765296in}{2.776179in}}%
\pgfpathcurveto{\pgfqpoint{4.765296in}{2.763156in}}{\pgfqpoint{4.770470in}{2.750665in}}{\pgfqpoint{4.779678in}{2.741456in}}%
\pgfpathcurveto{\pgfqpoint{4.788887in}{2.732248in}}{\pgfqpoint{4.801378in}{2.727074in}}{\pgfqpoint{4.814400in}{2.727074in}}%
\pgfpathlineto{\pgfqpoint{4.814400in}{2.727074in}}%
\pgfpathclose%
\pgfusepath{stroke,fill}%
\end{pgfscope}%
\begin{pgfscope}%
\pgfpathrectangle{\pgfqpoint{0.786164in}{0.768110in}}{\pgfqpoint{8.851069in}{7.081890in}}%
\pgfusepath{clip}%
\pgfsetbuttcap%
\pgfsetroundjoin%
\definecolor{currentfill}{rgb}{0.185783,0.704891,0.485273}%
\pgfsetfillcolor{currentfill}%
\pgfsetfillopacity{0.700000}%
\pgfsetlinewidth{0.501875pt}%
\definecolor{currentstroke}{rgb}{1.000000,1.000000,1.000000}%
\pgfsetstrokecolor{currentstroke}%
\pgfsetstrokeopacity{0.700000}%
\pgfsetdash{}{0pt}%
\pgfpathmoveto{\pgfqpoint{5.060999in}{3.077446in}}%
\pgfpathcurveto{\pgfqpoint{5.074022in}{3.077446in}}{\pgfqpoint{5.086513in}{3.082620in}}{\pgfqpoint{5.095721in}{3.091828in}}%
\pgfpathcurveto{\pgfqpoint{5.104930in}{3.101037in}}{\pgfqpoint{5.110104in}{3.113528in}}{\pgfqpoint{5.110104in}{3.126550in}}%
\pgfpathcurveto{\pgfqpoint{5.110104in}{3.139573in}}{\pgfqpoint{5.104930in}{3.152064in}}{\pgfqpoint{5.095721in}{3.161273in}}%
\pgfpathcurveto{\pgfqpoint{5.086513in}{3.170481in}}{\pgfqpoint{5.074022in}{3.175655in}}{\pgfqpoint{5.060999in}{3.175655in}}%
\pgfpathcurveto{\pgfqpoint{5.047976in}{3.175655in}}{\pgfqpoint{5.035485in}{3.170481in}}{\pgfqpoint{5.026277in}{3.161273in}}%
\pgfpathcurveto{\pgfqpoint{5.017068in}{3.152064in}}{\pgfqpoint{5.011894in}{3.139573in}}{\pgfqpoint{5.011894in}{3.126550in}}%
\pgfpathcurveto{\pgfqpoint{5.011894in}{3.113528in}}{\pgfqpoint{5.017068in}{3.101037in}}{\pgfqpoint{5.026277in}{3.091828in}}%
\pgfpathcurveto{\pgfqpoint{5.035485in}{3.082620in}}{\pgfqpoint{5.047976in}{3.077446in}}{\pgfqpoint{5.060999in}{3.077446in}}%
\pgfpathlineto{\pgfqpoint{5.060999in}{3.077446in}}%
\pgfpathclose%
\pgfusepath{stroke,fill}%
\end{pgfscope}%
\begin{pgfscope}%
\pgfpathrectangle{\pgfqpoint{0.786164in}{0.768110in}}{\pgfqpoint{8.851069in}{7.081890in}}%
\pgfusepath{clip}%
\pgfsetbuttcap%
\pgfsetroundjoin%
\definecolor{currentfill}{rgb}{0.311925,0.767822,0.415586}%
\pgfsetfillcolor{currentfill}%
\pgfsetfillopacity{0.700000}%
\pgfsetlinewidth{0.501875pt}%
\definecolor{currentstroke}{rgb}{1.000000,1.000000,1.000000}%
\pgfsetstrokecolor{currentstroke}%
\pgfsetstrokeopacity{0.700000}%
\pgfsetdash{}{0pt}%
\pgfpathmoveto{\pgfqpoint{4.832667in}{3.055548in}}%
\pgfpathcurveto{\pgfqpoint{4.845690in}{3.055548in}}{\pgfqpoint{4.858181in}{3.060722in}}{\pgfqpoint{4.867389in}{3.069930in}}%
\pgfpathcurveto{\pgfqpoint{4.876598in}{3.079138in}}{\pgfqpoint{4.881772in}{3.091630in}}{\pgfqpoint{4.881772in}{3.104652in}}%
\pgfpathcurveto{\pgfqpoint{4.881772in}{3.117675in}}{\pgfqpoint{4.876598in}{3.130166in}}{\pgfqpoint{4.867389in}{3.139374in}}%
\pgfpathcurveto{\pgfqpoint{4.858181in}{3.148583in}}{\pgfqpoint{4.845690in}{3.153757in}}{\pgfqpoint{4.832667in}{3.153757in}}%
\pgfpathcurveto{\pgfqpoint{4.819644in}{3.153757in}}{\pgfqpoint{4.807153in}{3.148583in}}{\pgfqpoint{4.797945in}{3.139374in}}%
\pgfpathcurveto{\pgfqpoint{4.788736in}{3.130166in}}{\pgfqpoint{4.783562in}{3.117675in}}{\pgfqpoint{4.783562in}{3.104652in}}%
\pgfpathcurveto{\pgfqpoint{4.783562in}{3.091630in}}{\pgfqpoint{4.788736in}{3.079138in}}{\pgfqpoint{4.797945in}{3.069930in}}%
\pgfpathcurveto{\pgfqpoint{4.807153in}{3.060722in}}{\pgfqpoint{4.819644in}{3.055548in}}{\pgfqpoint{4.832667in}{3.055548in}}%
\pgfpathlineto{\pgfqpoint{4.832667in}{3.055548in}}%
\pgfpathclose%
\pgfusepath{stroke,fill}%
\end{pgfscope}%
\begin{pgfscope}%
\pgfpathrectangle{\pgfqpoint{0.786164in}{0.768110in}}{\pgfqpoint{8.851069in}{7.081890in}}%
\pgfusepath{clip}%
\pgfsetbuttcap%
\pgfsetroundjoin%
\definecolor{currentfill}{rgb}{0.276194,0.190074,0.493001}%
\pgfsetfillcolor{currentfill}%
\pgfsetfillopacity{0.700000}%
\pgfsetlinewidth{0.501875pt}%
\definecolor{currentstroke}{rgb}{1.000000,1.000000,1.000000}%
\pgfsetstrokecolor{currentstroke}%
\pgfsetstrokeopacity{0.700000}%
\pgfsetdash{}{0pt}%
\pgfpathmoveto{\pgfqpoint{3.654473in}{2.814667in}}%
\pgfpathcurveto{\pgfqpoint{3.667496in}{2.814667in}}{\pgfqpoint{3.679987in}{2.819841in}}{\pgfqpoint{3.689195in}{2.829049in}}%
\pgfpathcurveto{\pgfqpoint{3.698403in}{2.838258in}}{\pgfqpoint{3.703577in}{2.850749in}}{\pgfqpoint{3.703577in}{2.863772in}}%
\pgfpathcurveto{\pgfqpoint{3.703577in}{2.876794in}}{\pgfqpoint{3.698403in}{2.889285in}}{\pgfqpoint{3.689195in}{2.898494in}}%
\pgfpathcurveto{\pgfqpoint{3.679987in}{2.907702in}}{\pgfqpoint{3.667496in}{2.912876in}}{\pgfqpoint{3.654473in}{2.912876in}}%
\pgfpathcurveto{\pgfqpoint{3.641450in}{2.912876in}}{\pgfqpoint{3.628959in}{2.907702in}}{\pgfqpoint{3.619751in}{2.898494in}}%
\pgfpathcurveto{\pgfqpoint{3.610542in}{2.889285in}}{\pgfqpoint{3.605368in}{2.876794in}}{\pgfqpoint{3.605368in}{2.863772in}}%
\pgfpathcurveto{\pgfqpoint{3.605368in}{2.850749in}}{\pgfqpoint{3.610542in}{2.838258in}}{\pgfqpoint{3.619751in}{2.829049in}}%
\pgfpathcurveto{\pgfqpoint{3.628959in}{2.819841in}}{\pgfqpoint{3.641450in}{2.814667in}}{\pgfqpoint{3.654473in}{2.814667in}}%
\pgfpathlineto{\pgfqpoint{3.654473in}{2.814667in}}%
\pgfpathclose%
\pgfusepath{stroke,fill}%
\end{pgfscope}%
\begin{pgfscope}%
\pgfpathrectangle{\pgfqpoint{0.786164in}{0.768110in}}{\pgfqpoint{8.851069in}{7.081890in}}%
\pgfusepath{clip}%
\pgfsetbuttcap%
\pgfsetroundjoin%
\definecolor{currentfill}{rgb}{0.276194,0.190074,0.493001}%
\pgfsetfillcolor{currentfill}%
\pgfsetfillopacity{0.700000}%
\pgfsetlinewidth{0.501875pt}%
\definecolor{currentstroke}{rgb}{1.000000,1.000000,1.000000}%
\pgfsetstrokecolor{currentstroke}%
\pgfsetstrokeopacity{0.700000}%
\pgfsetdash{}{0pt}%
\pgfpathmoveto{\pgfqpoint{3.654473in}{2.792769in}}%
\pgfpathcurveto{\pgfqpoint{3.667496in}{2.792769in}}{\pgfqpoint{3.679987in}{2.797943in}}{\pgfqpoint{3.689195in}{2.807151in}}%
\pgfpathcurveto{\pgfqpoint{3.698403in}{2.816360in}}{\pgfqpoint{3.703577in}{2.828851in}}{\pgfqpoint{3.703577in}{2.841873in}}%
\pgfpathcurveto{\pgfqpoint{3.703577in}{2.854896in}}{\pgfqpoint{3.698403in}{2.867387in}}{\pgfqpoint{3.689195in}{2.876596in}}%
\pgfpathcurveto{\pgfqpoint{3.679987in}{2.885804in}}{\pgfqpoint{3.667496in}{2.890978in}}{\pgfqpoint{3.654473in}{2.890978in}}%
\pgfpathcurveto{\pgfqpoint{3.641450in}{2.890978in}}{\pgfqpoint{3.628959in}{2.885804in}}{\pgfqpoint{3.619751in}{2.876596in}}%
\pgfpathcurveto{\pgfqpoint{3.610542in}{2.867387in}}{\pgfqpoint{3.605368in}{2.854896in}}{\pgfqpoint{3.605368in}{2.841873in}}%
\pgfpathcurveto{\pgfqpoint{3.605368in}{2.828851in}}{\pgfqpoint{3.610542in}{2.816360in}}{\pgfqpoint{3.619751in}{2.807151in}}%
\pgfpathcurveto{\pgfqpoint{3.628959in}{2.797943in}}{\pgfqpoint{3.641450in}{2.792769in}}{\pgfqpoint{3.654473in}{2.792769in}}%
\pgfpathlineto{\pgfqpoint{3.654473in}{2.792769in}}%
\pgfpathclose%
\pgfusepath{stroke,fill}%
\end{pgfscope}%
\begin{pgfscope}%
\pgfpathrectangle{\pgfqpoint{0.786164in}{0.768110in}}{\pgfqpoint{8.851069in}{7.081890in}}%
\pgfusepath{clip}%
\pgfsetbuttcap%
\pgfsetroundjoin%
\definecolor{currentfill}{rgb}{0.277134,0.185228,0.489898}%
\pgfsetfillcolor{currentfill}%
\pgfsetfillopacity{0.700000}%
\pgfsetlinewidth{0.501875pt}%
\definecolor{currentstroke}{rgb}{1.000000,1.000000,1.000000}%
\pgfsetstrokecolor{currentstroke}%
\pgfsetstrokeopacity{0.700000}%
\pgfsetdash{}{0pt}%
\pgfpathmoveto{\pgfqpoint{3.563140in}{2.748972in}}%
\pgfpathcurveto{\pgfqpoint{3.576163in}{2.748972in}}{\pgfqpoint{3.588654in}{2.754146in}}{\pgfqpoint{3.597862in}{2.763355in}}%
\pgfpathcurveto{\pgfqpoint{3.607071in}{2.772563in}}{\pgfqpoint{3.612245in}{2.785054in}}{\pgfqpoint{3.612245in}{2.798077in}}%
\pgfpathcurveto{\pgfqpoint{3.612245in}{2.811100in}}{\pgfqpoint{3.607071in}{2.823591in}}{\pgfqpoint{3.597862in}{2.832799in}}%
\pgfpathcurveto{\pgfqpoint{3.588654in}{2.842008in}}{\pgfqpoint{3.576163in}{2.847182in}}{\pgfqpoint{3.563140in}{2.847182in}}%
\pgfpathcurveto{\pgfqpoint{3.550117in}{2.847182in}}{\pgfqpoint{3.537626in}{2.842008in}}{\pgfqpoint{3.528418in}{2.832799in}}%
\pgfpathcurveto{\pgfqpoint{3.519209in}{2.823591in}}{\pgfqpoint{3.514035in}{2.811100in}}{\pgfqpoint{3.514035in}{2.798077in}}%
\pgfpathcurveto{\pgfqpoint{3.514035in}{2.785054in}}{\pgfqpoint{3.519209in}{2.772563in}}{\pgfqpoint{3.528418in}{2.763355in}}%
\pgfpathcurveto{\pgfqpoint{3.537626in}{2.754146in}}{\pgfqpoint{3.550117in}{2.748972in}}{\pgfqpoint{3.563140in}{2.748972in}}%
\pgfpathlineto{\pgfqpoint{3.563140in}{2.748972in}}%
\pgfpathclose%
\pgfusepath{stroke,fill}%
\end{pgfscope}%
\begin{pgfscope}%
\pgfpathrectangle{\pgfqpoint{0.786164in}{0.768110in}}{\pgfqpoint{8.851069in}{7.081890in}}%
\pgfusepath{clip}%
\pgfsetbuttcap%
\pgfsetroundjoin%
\definecolor{currentfill}{rgb}{0.275191,0.194905,0.496005}%
\pgfsetfillcolor{currentfill}%
\pgfsetfillopacity{0.700000}%
\pgfsetlinewidth{0.501875pt}%
\definecolor{currentstroke}{rgb}{1.000000,1.000000,1.000000}%
\pgfsetstrokecolor{currentstroke}%
\pgfsetstrokeopacity{0.700000}%
\pgfsetdash{}{0pt}%
\pgfpathmoveto{\pgfqpoint{3.490074in}{2.727074in}}%
\pgfpathcurveto{\pgfqpoint{3.503096in}{2.727074in}}{\pgfqpoint{3.515587in}{2.732248in}}{\pgfqpoint{3.524796in}{2.741456in}}%
\pgfpathcurveto{\pgfqpoint{3.534004in}{2.750665in}}{\pgfqpoint{3.539178in}{2.763156in}}{\pgfqpoint{3.539178in}{2.776179in}}%
\pgfpathcurveto{\pgfqpoint{3.539178in}{2.789201in}}{\pgfqpoint{3.534004in}{2.801692in}}{\pgfqpoint{3.524796in}{2.810901in}}%
\pgfpathcurveto{\pgfqpoint{3.515587in}{2.820109in}}{\pgfqpoint{3.503096in}{2.825283in}}{\pgfqpoint{3.490074in}{2.825283in}}%
\pgfpathcurveto{\pgfqpoint{3.477051in}{2.825283in}}{\pgfqpoint{3.464560in}{2.820109in}}{\pgfqpoint{3.455351in}{2.810901in}}%
\pgfpathcurveto{\pgfqpoint{3.446143in}{2.801692in}}{\pgfqpoint{3.440969in}{2.789201in}}{\pgfqpoint{3.440969in}{2.776179in}}%
\pgfpathcurveto{\pgfqpoint{3.440969in}{2.763156in}}{\pgfqpoint{3.446143in}{2.750665in}}{\pgfqpoint{3.455351in}{2.741456in}}%
\pgfpathcurveto{\pgfqpoint{3.464560in}{2.732248in}}{\pgfqpoint{3.477051in}{2.727074in}}{\pgfqpoint{3.490074in}{2.727074in}}%
\pgfpathlineto{\pgfqpoint{3.490074in}{2.727074in}}%
\pgfpathclose%
\pgfusepath{stroke,fill}%
\end{pgfscope}%
\begin{pgfscope}%
\pgfpathrectangle{\pgfqpoint{0.786164in}{0.768110in}}{\pgfqpoint{8.851069in}{7.081890in}}%
\pgfusepath{clip}%
\pgfsetbuttcap%
\pgfsetroundjoin%
\definecolor{currentfill}{rgb}{0.266580,0.228262,0.514349}%
\pgfsetfillcolor{currentfill}%
\pgfsetfillopacity{0.700000}%
\pgfsetlinewidth{0.501875pt}%
\definecolor{currentstroke}{rgb}{1.000000,1.000000,1.000000}%
\pgfsetstrokecolor{currentstroke}%
\pgfsetstrokeopacity{0.700000}%
\pgfsetdash{}{0pt}%
\pgfpathmoveto{\pgfqpoint{3.508340in}{2.705176in}}%
\pgfpathcurveto{\pgfqpoint{3.521363in}{2.705176in}}{\pgfqpoint{3.533854in}{2.710350in}}{\pgfqpoint{3.543062in}{2.719558in}}%
\pgfpathcurveto{\pgfqpoint{3.552271in}{2.728767in}}{\pgfqpoint{3.557445in}{2.741258in}}{\pgfqpoint{3.557445in}{2.754280in}}%
\pgfpathcurveto{\pgfqpoint{3.557445in}{2.767303in}}{\pgfqpoint{3.552271in}{2.779794in}}{\pgfqpoint{3.543062in}{2.789003in}}%
\pgfpathcurveto{\pgfqpoint{3.533854in}{2.798211in}}{\pgfqpoint{3.521363in}{2.803385in}}{\pgfqpoint{3.508340in}{2.803385in}}%
\pgfpathcurveto{\pgfqpoint{3.495318in}{2.803385in}}{\pgfqpoint{3.482826in}{2.798211in}}{\pgfqpoint{3.473618in}{2.789003in}}%
\pgfpathcurveto{\pgfqpoint{3.464410in}{2.779794in}}{\pgfqpoint{3.459236in}{2.767303in}}{\pgfqpoint{3.459236in}{2.754280in}}%
\pgfpathcurveto{\pgfqpoint{3.459236in}{2.741258in}}{\pgfqpoint{3.464410in}{2.728767in}}{\pgfqpoint{3.473618in}{2.719558in}}%
\pgfpathcurveto{\pgfqpoint{3.482826in}{2.710350in}}{\pgfqpoint{3.495318in}{2.705176in}}{\pgfqpoint{3.508340in}{2.705176in}}%
\pgfpathlineto{\pgfqpoint{3.508340in}{2.705176in}}%
\pgfpathclose%
\pgfusepath{stroke,fill}%
\end{pgfscope}%
\begin{pgfscope}%
\pgfpathrectangle{\pgfqpoint{0.786164in}{0.768110in}}{\pgfqpoint{8.851069in}{7.081890in}}%
\pgfusepath{clip}%
\pgfsetbuttcap%
\pgfsetroundjoin%
\definecolor{currentfill}{rgb}{0.260571,0.246922,0.522828}%
\pgfsetfillcolor{currentfill}%
\pgfsetfillopacity{0.700000}%
\pgfsetlinewidth{0.501875pt}%
\definecolor{currentstroke}{rgb}{1.000000,1.000000,1.000000}%
\pgfsetstrokecolor{currentstroke}%
\pgfsetstrokeopacity{0.700000}%
\pgfsetdash{}{0pt}%
\pgfpathmoveto{\pgfqpoint{3.362208in}{2.661379in}}%
\pgfpathcurveto{\pgfqpoint{3.375230in}{2.661379in}}{\pgfqpoint{3.387721in}{2.666553in}}{\pgfqpoint{3.396930in}{2.675762in}}%
\pgfpathcurveto{\pgfqpoint{3.406138in}{2.684970in}}{\pgfqpoint{3.411312in}{2.697461in}}{\pgfqpoint{3.411312in}{2.710484in}}%
\pgfpathcurveto{\pgfqpoint{3.411312in}{2.723507in}}{\pgfqpoint{3.406138in}{2.735998in}}{\pgfqpoint{3.396930in}{2.745206in}}%
\pgfpathcurveto{\pgfqpoint{3.387721in}{2.754415in}}{\pgfqpoint{3.375230in}{2.759589in}}{\pgfqpoint{3.362208in}{2.759589in}}%
\pgfpathcurveto{\pgfqpoint{3.349185in}{2.759589in}}{\pgfqpoint{3.336694in}{2.754415in}}{\pgfqpoint{3.327485in}{2.745206in}}%
\pgfpathcurveto{\pgfqpoint{3.318277in}{2.735998in}}{\pgfqpoint{3.313103in}{2.723507in}}{\pgfqpoint{3.313103in}{2.710484in}}%
\pgfpathcurveto{\pgfqpoint{3.313103in}{2.697461in}}{\pgfqpoint{3.318277in}{2.684970in}}{\pgfqpoint{3.327485in}{2.675762in}}%
\pgfpathcurveto{\pgfqpoint{3.336694in}{2.666553in}}{\pgfqpoint{3.349185in}{2.661379in}}{\pgfqpoint{3.362208in}{2.661379in}}%
\pgfpathlineto{\pgfqpoint{3.362208in}{2.661379in}}%
\pgfpathclose%
\pgfusepath{stroke,fill}%
\end{pgfscope}%
\begin{pgfscope}%
\pgfpathrectangle{\pgfqpoint{0.786164in}{0.768110in}}{\pgfqpoint{8.851069in}{7.081890in}}%
\pgfusepath{clip}%
\pgfsetbuttcap%
\pgfsetroundjoin%
\definecolor{currentfill}{rgb}{0.258965,0.251537,0.524736}%
\pgfsetfillcolor{currentfill}%
\pgfsetfillopacity{0.700000}%
\pgfsetlinewidth{0.501875pt}%
\definecolor{currentstroke}{rgb}{1.000000,1.000000,1.000000}%
\pgfsetstrokecolor{currentstroke}%
\pgfsetstrokeopacity{0.700000}%
\pgfsetdash{}{0pt}%
\pgfpathmoveto{\pgfqpoint{3.462674in}{2.661379in}}%
\pgfpathcurveto{\pgfqpoint{3.475696in}{2.661379in}}{\pgfqpoint{3.488188in}{2.666553in}}{\pgfqpoint{3.497396in}{2.675762in}}%
\pgfpathcurveto{\pgfqpoint{3.506604in}{2.684970in}}{\pgfqpoint{3.511778in}{2.697461in}}{\pgfqpoint{3.511778in}{2.710484in}}%
\pgfpathcurveto{\pgfqpoint{3.511778in}{2.723507in}}{\pgfqpoint{3.506604in}{2.735998in}}{\pgfqpoint{3.497396in}{2.745206in}}%
\pgfpathcurveto{\pgfqpoint{3.488188in}{2.754415in}}{\pgfqpoint{3.475696in}{2.759589in}}{\pgfqpoint{3.462674in}{2.759589in}}%
\pgfpathcurveto{\pgfqpoint{3.449651in}{2.759589in}}{\pgfqpoint{3.437160in}{2.754415in}}{\pgfqpoint{3.427952in}{2.745206in}}%
\pgfpathcurveto{\pgfqpoint{3.418743in}{2.735998in}}{\pgfqpoint{3.413569in}{2.723507in}}{\pgfqpoint{3.413569in}{2.710484in}}%
\pgfpathcurveto{\pgfqpoint{3.413569in}{2.697461in}}{\pgfqpoint{3.418743in}{2.684970in}}{\pgfqpoint{3.427952in}{2.675762in}}%
\pgfpathcurveto{\pgfqpoint{3.437160in}{2.666553in}}{\pgfqpoint{3.449651in}{2.661379in}}{\pgfqpoint{3.462674in}{2.661379in}}%
\pgfpathlineto{\pgfqpoint{3.462674in}{2.661379in}}%
\pgfpathclose%
\pgfusepath{stroke,fill}%
\end{pgfscope}%
\begin{pgfscope}%
\pgfpathrectangle{\pgfqpoint{0.786164in}{0.768110in}}{\pgfqpoint{8.851069in}{7.081890in}}%
\pgfusepath{clip}%
\pgfsetbuttcap%
\pgfsetroundjoin%
\definecolor{currentfill}{rgb}{0.269308,0.218818,0.509577}%
\pgfsetfillcolor{currentfill}%
\pgfsetfillopacity{0.700000}%
\pgfsetlinewidth{0.501875pt}%
\definecolor{currentstroke}{rgb}{1.000000,1.000000,1.000000}%
\pgfsetstrokecolor{currentstroke}%
\pgfsetstrokeopacity{0.700000}%
\pgfsetdash{}{0pt}%
\pgfpathmoveto{\pgfqpoint{3.362208in}{2.573786in}}%
\pgfpathcurveto{\pgfqpoint{3.375230in}{2.573786in}}{\pgfqpoint{3.387721in}{2.578960in}}{\pgfqpoint{3.396930in}{2.588169in}}%
\pgfpathcurveto{\pgfqpoint{3.406138in}{2.597377in}}{\pgfqpoint{3.411312in}{2.609868in}}{\pgfqpoint{3.411312in}{2.622891in}}%
\pgfpathcurveto{\pgfqpoint{3.411312in}{2.635914in}}{\pgfqpoint{3.406138in}{2.648405in}}{\pgfqpoint{3.396930in}{2.657613in}}%
\pgfpathcurveto{\pgfqpoint{3.387721in}{2.666822in}}{\pgfqpoint{3.375230in}{2.671996in}}{\pgfqpoint{3.362208in}{2.671996in}}%
\pgfpathcurveto{\pgfqpoint{3.349185in}{2.671996in}}{\pgfqpoint{3.336694in}{2.666822in}}{\pgfqpoint{3.327485in}{2.657613in}}%
\pgfpathcurveto{\pgfqpoint{3.318277in}{2.648405in}}{\pgfqpoint{3.313103in}{2.635914in}}{\pgfqpoint{3.313103in}{2.622891in}}%
\pgfpathcurveto{\pgfqpoint{3.313103in}{2.609868in}}{\pgfqpoint{3.318277in}{2.597377in}}{\pgfqpoint{3.327485in}{2.588169in}}%
\pgfpathcurveto{\pgfqpoint{3.336694in}{2.578960in}}{\pgfqpoint{3.349185in}{2.573786in}}{\pgfqpoint{3.362208in}{2.573786in}}%
\pgfpathlineto{\pgfqpoint{3.362208in}{2.573786in}}%
\pgfpathclose%
\pgfusepath{stroke,fill}%
\end{pgfscope}%
\begin{pgfscope}%
\pgfpathrectangle{\pgfqpoint{0.786164in}{0.768110in}}{\pgfqpoint{8.851069in}{7.081890in}}%
\pgfusepath{clip}%
\pgfsetbuttcap%
\pgfsetroundjoin%
\definecolor{currentfill}{rgb}{0.253935,0.265254,0.529983}%
\pgfsetfillcolor{currentfill}%
\pgfsetfillopacity{0.700000}%
\pgfsetlinewidth{0.501875pt}%
\definecolor{currentstroke}{rgb}{1.000000,1.000000,1.000000}%
\pgfsetstrokecolor{currentstroke}%
\pgfsetstrokeopacity{0.700000}%
\pgfsetdash{}{0pt}%
\pgfpathmoveto{\pgfqpoint{3.353074in}{2.551888in}}%
\pgfpathcurveto{\pgfqpoint{3.366097in}{2.551888in}}{\pgfqpoint{3.378588in}{2.557062in}}{\pgfqpoint{3.387797in}{2.566271in}}%
\pgfpathcurveto{\pgfqpoint{3.397005in}{2.575479in}}{\pgfqpoint{3.402179in}{2.587970in}}{\pgfqpoint{3.402179in}{2.600993in}}%
\pgfpathcurveto{\pgfqpoint{3.402179in}{2.614015in}}{\pgfqpoint{3.397005in}{2.626507in}}{\pgfqpoint{3.387797in}{2.635715in}}%
\pgfpathcurveto{\pgfqpoint{3.378588in}{2.644923in}}{\pgfqpoint{3.366097in}{2.650097in}}{\pgfqpoint{3.353074in}{2.650097in}}%
\pgfpathcurveto{\pgfqpoint{3.340052in}{2.650097in}}{\pgfqpoint{3.327561in}{2.644923in}}{\pgfqpoint{3.318352in}{2.635715in}}%
\pgfpathcurveto{\pgfqpoint{3.309144in}{2.626507in}}{\pgfqpoint{3.303970in}{2.614015in}}{\pgfqpoint{3.303970in}{2.600993in}}%
\pgfpathcurveto{\pgfqpoint{3.303970in}{2.587970in}}{\pgfqpoint{3.309144in}{2.575479in}}{\pgfqpoint{3.318352in}{2.566271in}}%
\pgfpathcurveto{\pgfqpoint{3.327561in}{2.557062in}}{\pgfqpoint{3.340052in}{2.551888in}}{\pgfqpoint{3.353074in}{2.551888in}}%
\pgfpathlineto{\pgfqpoint{3.353074in}{2.551888in}}%
\pgfpathclose%
\pgfusepath{stroke,fill}%
\end{pgfscope}%
\begin{pgfscope}%
\pgfpathrectangle{\pgfqpoint{0.786164in}{0.768110in}}{\pgfqpoint{8.851069in}{7.081890in}}%
\pgfusepath{clip}%
\pgfsetbuttcap%
\pgfsetroundjoin%
\definecolor{currentfill}{rgb}{0.250425,0.274290,0.533103}%
\pgfsetfillcolor{currentfill}%
\pgfsetfillopacity{0.700000}%
\pgfsetlinewidth{0.501875pt}%
\definecolor{currentstroke}{rgb}{1.000000,1.000000,1.000000}%
\pgfsetstrokecolor{currentstroke}%
\pgfsetstrokeopacity{0.700000}%
\pgfsetdash{}{0pt}%
\pgfpathmoveto{\pgfqpoint{3.353074in}{2.529990in}}%
\pgfpathcurveto{\pgfqpoint{3.366097in}{2.529990in}}{\pgfqpoint{3.378588in}{2.535164in}}{\pgfqpoint{3.387797in}{2.544372in}}%
\pgfpathcurveto{\pgfqpoint{3.397005in}{2.553581in}}{\pgfqpoint{3.402179in}{2.566072in}}{\pgfqpoint{3.402179in}{2.579095in}}%
\pgfpathcurveto{\pgfqpoint{3.402179in}{2.592117in}}{\pgfqpoint{3.397005in}{2.604608in}}{\pgfqpoint{3.387797in}{2.613817in}}%
\pgfpathcurveto{\pgfqpoint{3.378588in}{2.623025in}}{\pgfqpoint{3.366097in}{2.628199in}}{\pgfqpoint{3.353074in}{2.628199in}}%
\pgfpathcurveto{\pgfqpoint{3.340052in}{2.628199in}}{\pgfqpoint{3.327561in}{2.623025in}}{\pgfqpoint{3.318352in}{2.613817in}}%
\pgfpathcurveto{\pgfqpoint{3.309144in}{2.604608in}}{\pgfqpoint{3.303970in}{2.592117in}}{\pgfqpoint{3.303970in}{2.579095in}}%
\pgfpathcurveto{\pgfqpoint{3.303970in}{2.566072in}}{\pgfqpoint{3.309144in}{2.553581in}}{\pgfqpoint{3.318352in}{2.544372in}}%
\pgfpathcurveto{\pgfqpoint{3.327561in}{2.535164in}}{\pgfqpoint{3.340052in}{2.529990in}}{\pgfqpoint{3.353074in}{2.529990in}}%
\pgfpathlineto{\pgfqpoint{3.353074in}{2.529990in}}%
\pgfpathclose%
\pgfusepath{stroke,fill}%
\end{pgfscope}%
\begin{pgfscope}%
\pgfpathrectangle{\pgfqpoint{0.786164in}{0.768110in}}{\pgfqpoint{8.851069in}{7.081890in}}%
\pgfusepath{clip}%
\pgfsetbuttcap%
\pgfsetroundjoin%
\definecolor{currentfill}{rgb}{0.246811,0.283237,0.535941}%
\pgfsetfillcolor{currentfill}%
\pgfsetfillopacity{0.700000}%
\pgfsetlinewidth{0.501875pt}%
\definecolor{currentstroke}{rgb}{1.000000,1.000000,1.000000}%
\pgfsetstrokecolor{currentstroke}%
\pgfsetstrokeopacity{0.700000}%
\pgfsetdash{}{0pt}%
\pgfpathmoveto{\pgfqpoint{3.179542in}{2.420499in}}%
\pgfpathcurveto{\pgfqpoint{3.192565in}{2.420499in}}{\pgfqpoint{3.205056in}{2.425673in}}{\pgfqpoint{3.214264in}{2.434881in}}%
\pgfpathcurveto{\pgfqpoint{3.223473in}{2.444090in}}{\pgfqpoint{3.228647in}{2.456581in}}{\pgfqpoint{3.228647in}{2.469603in}}%
\pgfpathcurveto{\pgfqpoint{3.228647in}{2.482626in}}{\pgfqpoint{3.223473in}{2.495117in}}{\pgfqpoint{3.214264in}{2.504326in}}%
\pgfpathcurveto{\pgfqpoint{3.205056in}{2.513534in}}{\pgfqpoint{3.192565in}{2.518708in}}{\pgfqpoint{3.179542in}{2.518708in}}%
\pgfpathcurveto{\pgfqpoint{3.166519in}{2.518708in}}{\pgfqpoint{3.154028in}{2.513534in}}{\pgfqpoint{3.144820in}{2.504326in}}%
\pgfpathcurveto{\pgfqpoint{3.135611in}{2.495117in}}{\pgfqpoint{3.130437in}{2.482626in}}{\pgfqpoint{3.130437in}{2.469603in}}%
\pgfpathcurveto{\pgfqpoint{3.130437in}{2.456581in}}{\pgfqpoint{3.135611in}{2.444090in}}{\pgfqpoint{3.144820in}{2.434881in}}%
\pgfpathcurveto{\pgfqpoint{3.154028in}{2.425673in}}{\pgfqpoint{3.166519in}{2.420499in}}{\pgfqpoint{3.179542in}{2.420499in}}%
\pgfpathlineto{\pgfqpoint{3.179542in}{2.420499in}}%
\pgfpathclose%
\pgfusepath{stroke,fill}%
\end{pgfscope}%
\begin{pgfscope}%
\pgfpathrectangle{\pgfqpoint{0.786164in}{0.768110in}}{\pgfqpoint{8.851069in}{7.081890in}}%
\pgfusepath{clip}%
\pgfsetbuttcap%
\pgfsetroundjoin%
\definecolor{currentfill}{rgb}{0.243113,0.292092,0.538516}%
\pgfsetfillcolor{currentfill}%
\pgfsetfillopacity{0.700000}%
\pgfsetlinewidth{0.501875pt}%
\definecolor{currentstroke}{rgb}{1.000000,1.000000,1.000000}%
\pgfsetstrokecolor{currentstroke}%
\pgfsetstrokeopacity{0.700000}%
\pgfsetdash{}{0pt}%
\pgfpathmoveto{\pgfqpoint{3.225208in}{2.442397in}}%
\pgfpathcurveto{\pgfqpoint{3.238231in}{2.442397in}}{\pgfqpoint{3.250722in}{2.447571in}}{\pgfqpoint{3.259931in}{2.456779in}}%
\pgfpathcurveto{\pgfqpoint{3.269139in}{2.465988in}}{\pgfqpoint{3.274313in}{2.478479in}}{\pgfqpoint{3.274313in}{2.491502in}}%
\pgfpathcurveto{\pgfqpoint{3.274313in}{2.504524in}}{\pgfqpoint{3.269139in}{2.517015in}}{\pgfqpoint{3.259931in}{2.526224in}}%
\pgfpathcurveto{\pgfqpoint{3.250722in}{2.535432in}}{\pgfqpoint{3.238231in}{2.540606in}}{\pgfqpoint{3.225208in}{2.540606in}}%
\pgfpathcurveto{\pgfqpoint{3.212186in}{2.540606in}}{\pgfqpoint{3.199695in}{2.535432in}}{\pgfqpoint{3.190486in}{2.526224in}}%
\pgfpathcurveto{\pgfqpoint{3.181278in}{2.517015in}}{\pgfqpoint{3.176104in}{2.504524in}}{\pgfqpoint{3.176104in}{2.491502in}}%
\pgfpathcurveto{\pgfqpoint{3.176104in}{2.478479in}}{\pgfqpoint{3.181278in}{2.465988in}}{\pgfqpoint{3.190486in}{2.456779in}}%
\pgfpathcurveto{\pgfqpoint{3.199695in}{2.447571in}}{\pgfqpoint{3.212186in}{2.442397in}}{\pgfqpoint{3.225208in}{2.442397in}}%
\pgfpathlineto{\pgfqpoint{3.225208in}{2.442397in}}%
\pgfpathclose%
\pgfusepath{stroke,fill}%
\end{pgfscope}%
\begin{pgfscope}%
\pgfpathrectangle{\pgfqpoint{0.786164in}{0.768110in}}{\pgfqpoint{8.851069in}{7.081890in}}%
\pgfusepath{clip}%
\pgfsetbuttcap%
\pgfsetroundjoin%
\definecolor{currentfill}{rgb}{0.239346,0.300855,0.540844}%
\pgfsetfillcolor{currentfill}%
\pgfsetfillopacity{0.700000}%
\pgfsetlinewidth{0.501875pt}%
\definecolor{currentstroke}{rgb}{1.000000,1.000000,1.000000}%
\pgfsetstrokecolor{currentstroke}%
\pgfsetstrokeopacity{0.700000}%
\pgfsetdash{}{0pt}%
\pgfpathmoveto{\pgfqpoint{3.161275in}{2.464295in}}%
\pgfpathcurveto{\pgfqpoint{3.174298in}{2.464295in}}{\pgfqpoint{3.186789in}{2.469469in}}{\pgfqpoint{3.195998in}{2.478678in}}%
\pgfpathcurveto{\pgfqpoint{3.205206in}{2.487886in}}{\pgfqpoint{3.210380in}{2.500377in}}{\pgfqpoint{3.210380in}{2.513400in}}%
\pgfpathcurveto{\pgfqpoint{3.210380in}{2.526423in}}{\pgfqpoint{3.205206in}{2.538914in}}{\pgfqpoint{3.195998in}{2.548122in}}%
\pgfpathcurveto{\pgfqpoint{3.186789in}{2.557331in}}{\pgfqpoint{3.174298in}{2.562504in}}{\pgfqpoint{3.161275in}{2.562504in}}%
\pgfpathcurveto{\pgfqpoint{3.148253in}{2.562504in}}{\pgfqpoint{3.135762in}{2.557331in}}{\pgfqpoint{3.126553in}{2.548122in}}%
\pgfpathcurveto{\pgfqpoint{3.117345in}{2.538914in}}{\pgfqpoint{3.112171in}{2.526423in}}{\pgfqpoint{3.112171in}{2.513400in}}%
\pgfpathcurveto{\pgfqpoint{3.112171in}{2.500377in}}{\pgfqpoint{3.117345in}{2.487886in}}{\pgfqpoint{3.126553in}{2.478678in}}%
\pgfpathcurveto{\pgfqpoint{3.135762in}{2.469469in}}{\pgfqpoint{3.148253in}{2.464295in}}{\pgfqpoint{3.161275in}{2.464295in}}%
\pgfpathlineto{\pgfqpoint{3.161275in}{2.464295in}}%
\pgfpathclose%
\pgfusepath{stroke,fill}%
\end{pgfscope}%
\begin{pgfscope}%
\pgfpathrectangle{\pgfqpoint{0.786164in}{0.768110in}}{\pgfqpoint{8.851069in}{7.081890in}}%
\pgfusepath{clip}%
\pgfsetbuttcap%
\pgfsetroundjoin%
\definecolor{currentfill}{rgb}{0.235526,0.309527,0.542944}%
\pgfsetfillcolor{currentfill}%
\pgfsetfillopacity{0.700000}%
\pgfsetlinewidth{0.501875pt}%
\definecolor{currentstroke}{rgb}{1.000000,1.000000,1.000000}%
\pgfsetstrokecolor{currentstroke}%
\pgfsetstrokeopacity{0.700000}%
\pgfsetdash{}{0pt}%
\pgfpathmoveto{\pgfqpoint{3.143009in}{2.508092in}}%
\pgfpathcurveto{\pgfqpoint{3.156031in}{2.508092in}}{\pgfqpoint{3.168523in}{2.513266in}}{\pgfqpoint{3.177731in}{2.522474in}}%
\pgfpathcurveto{\pgfqpoint{3.186939in}{2.531683in}}{\pgfqpoint{3.192113in}{2.544174in}}{\pgfqpoint{3.192113in}{2.557196in}}%
\pgfpathcurveto{\pgfqpoint{3.192113in}{2.570219in}}{\pgfqpoint{3.186939in}{2.582710in}}{\pgfqpoint{3.177731in}{2.591919in}}%
\pgfpathcurveto{\pgfqpoint{3.168523in}{2.601127in}}{\pgfqpoint{3.156031in}{2.606301in}}{\pgfqpoint{3.143009in}{2.606301in}}%
\pgfpathcurveto{\pgfqpoint{3.129986in}{2.606301in}}{\pgfqpoint{3.117495in}{2.601127in}}{\pgfqpoint{3.108286in}{2.591919in}}%
\pgfpathcurveto{\pgfqpoint{3.099078in}{2.582710in}}{\pgfqpoint{3.093904in}{2.570219in}}{\pgfqpoint{3.093904in}{2.557196in}}%
\pgfpathcurveto{\pgfqpoint{3.093904in}{2.544174in}}{\pgfqpoint{3.099078in}{2.531683in}}{\pgfqpoint{3.108286in}{2.522474in}}%
\pgfpathcurveto{\pgfqpoint{3.117495in}{2.513266in}}{\pgfqpoint{3.129986in}{2.508092in}}{\pgfqpoint{3.143009in}{2.508092in}}%
\pgfpathlineto{\pgfqpoint{3.143009in}{2.508092in}}%
\pgfpathclose%
\pgfusepath{stroke,fill}%
\end{pgfscope}%
\begin{pgfscope}%
\pgfpathrectangle{\pgfqpoint{0.786164in}{0.768110in}}{\pgfqpoint{8.851069in}{7.081890in}}%
\pgfusepath{clip}%
\pgfsetbuttcap%
\pgfsetroundjoin%
\definecolor{currentfill}{rgb}{0.231674,0.318106,0.544834}%
\pgfsetfillcolor{currentfill}%
\pgfsetfillopacity{0.700000}%
\pgfsetlinewidth{0.501875pt}%
\definecolor{currentstroke}{rgb}{1.000000,1.000000,1.000000}%
\pgfsetstrokecolor{currentstroke}%
\pgfsetstrokeopacity{0.700000}%
\pgfsetdash{}{0pt}%
\pgfpathmoveto{\pgfqpoint{3.124742in}{2.420499in}}%
\pgfpathcurveto{\pgfqpoint{3.137765in}{2.420499in}}{\pgfqpoint{3.150256in}{2.425673in}}{\pgfqpoint{3.159464in}{2.434881in}}%
\pgfpathcurveto{\pgfqpoint{3.168673in}{2.444090in}}{\pgfqpoint{3.173847in}{2.456581in}}{\pgfqpoint{3.173847in}{2.469603in}}%
\pgfpathcurveto{\pgfqpoint{3.173847in}{2.482626in}}{\pgfqpoint{3.168673in}{2.495117in}}{\pgfqpoint{3.159464in}{2.504326in}}%
\pgfpathcurveto{\pgfqpoint{3.150256in}{2.513534in}}{\pgfqpoint{3.137765in}{2.518708in}}{\pgfqpoint{3.124742in}{2.518708in}}%
\pgfpathcurveto{\pgfqpoint{3.111719in}{2.518708in}}{\pgfqpoint{3.099228in}{2.513534in}}{\pgfqpoint{3.090020in}{2.504326in}}%
\pgfpathcurveto{\pgfqpoint{3.080811in}{2.495117in}}{\pgfqpoint{3.075638in}{2.482626in}}{\pgfqpoint{3.075638in}{2.469603in}}%
\pgfpathcurveto{\pgfqpoint{3.075638in}{2.456581in}}{\pgfqpoint{3.080811in}{2.444090in}}{\pgfqpoint{3.090020in}{2.434881in}}%
\pgfpathcurveto{\pgfqpoint{3.099228in}{2.425673in}}{\pgfqpoint{3.111719in}{2.420499in}}{\pgfqpoint{3.124742in}{2.420499in}}%
\pgfpathlineto{\pgfqpoint{3.124742in}{2.420499in}}%
\pgfpathclose%
\pgfusepath{stroke,fill}%
\end{pgfscope}%
\begin{pgfscope}%
\pgfpathrectangle{\pgfqpoint{0.786164in}{0.768110in}}{\pgfqpoint{8.851069in}{7.081890in}}%
\pgfusepath{clip}%
\pgfsetbuttcap%
\pgfsetroundjoin%
\definecolor{currentfill}{rgb}{0.225863,0.330805,0.547314}%
\pgfsetfillcolor{currentfill}%
\pgfsetfillopacity{0.700000}%
\pgfsetlinewidth{0.501875pt}%
\definecolor{currentstroke}{rgb}{1.000000,1.000000,1.000000}%
\pgfsetstrokecolor{currentstroke}%
\pgfsetstrokeopacity{0.700000}%
\pgfsetdash{}{0pt}%
\pgfpathmoveto{\pgfqpoint{3.060809in}{2.420499in}}%
\pgfpathcurveto{\pgfqpoint{3.073832in}{2.420499in}}{\pgfqpoint{3.086323in}{2.425673in}}{\pgfqpoint{3.095531in}{2.434881in}}%
\pgfpathcurveto{\pgfqpoint{3.104740in}{2.444090in}}{\pgfqpoint{3.109914in}{2.456581in}}{\pgfqpoint{3.109914in}{2.469603in}}%
\pgfpathcurveto{\pgfqpoint{3.109914in}{2.482626in}}{\pgfqpoint{3.104740in}{2.495117in}}{\pgfqpoint{3.095531in}{2.504326in}}%
\pgfpathcurveto{\pgfqpoint{3.086323in}{2.513534in}}{\pgfqpoint{3.073832in}{2.518708in}}{\pgfqpoint{3.060809in}{2.518708in}}%
\pgfpathcurveto{\pgfqpoint{3.047786in}{2.518708in}}{\pgfqpoint{3.035295in}{2.513534in}}{\pgfqpoint{3.026087in}{2.504326in}}%
\pgfpathcurveto{\pgfqpoint{3.016878in}{2.495117in}}{\pgfqpoint{3.011704in}{2.482626in}}{\pgfqpoint{3.011704in}{2.469603in}}%
\pgfpathcurveto{\pgfqpoint{3.011704in}{2.456581in}}{\pgfqpoint{3.016878in}{2.444090in}}{\pgfqpoint{3.026087in}{2.434881in}}%
\pgfpathcurveto{\pgfqpoint{3.035295in}{2.425673in}}{\pgfqpoint{3.047786in}{2.420499in}}{\pgfqpoint{3.060809in}{2.420499in}}%
\pgfpathlineto{\pgfqpoint{3.060809in}{2.420499in}}%
\pgfpathclose%
\pgfusepath{stroke,fill}%
\end{pgfscope}%
\begin{pgfscope}%
\pgfpathrectangle{\pgfqpoint{0.786164in}{0.768110in}}{\pgfqpoint{8.851069in}{7.081890in}}%
\pgfusepath{clip}%
\pgfsetbuttcap%
\pgfsetroundjoin%
\definecolor{currentfill}{rgb}{0.212395,0.359683,0.551710}%
\pgfsetfillcolor{currentfill}%
\pgfsetfillopacity{0.700000}%
\pgfsetlinewidth{0.501875pt}%
\definecolor{currentstroke}{rgb}{1.000000,1.000000,1.000000}%
\pgfsetstrokecolor{currentstroke}%
\pgfsetstrokeopacity{0.700000}%
\pgfsetdash{}{0pt}%
\pgfpathmoveto{\pgfqpoint{2.686344in}{2.245313in}}%
\pgfpathcurveto{\pgfqpoint{2.699367in}{2.245313in}}{\pgfqpoint{2.711858in}{2.250487in}}{\pgfqpoint{2.721067in}{2.259695in}}%
\pgfpathcurveto{\pgfqpoint{2.730275in}{2.268904in}}{\pgfqpoint{2.735449in}{2.281395in}}{\pgfqpoint{2.735449in}{2.294417in}}%
\pgfpathcurveto{\pgfqpoint{2.735449in}{2.307440in}}{\pgfqpoint{2.730275in}{2.319931in}}{\pgfqpoint{2.721067in}{2.329140in}}%
\pgfpathcurveto{\pgfqpoint{2.711858in}{2.338348in}}{\pgfqpoint{2.699367in}{2.343522in}}{\pgfqpoint{2.686344in}{2.343522in}}%
\pgfpathcurveto{\pgfqpoint{2.673322in}{2.343522in}}{\pgfqpoint{2.660831in}{2.338348in}}{\pgfqpoint{2.651622in}{2.329140in}}%
\pgfpathcurveto{\pgfqpoint{2.642414in}{2.319931in}}{\pgfqpoint{2.637240in}{2.307440in}}{\pgfqpoint{2.637240in}{2.294417in}}%
\pgfpathcurveto{\pgfqpoint{2.637240in}{2.281395in}}{\pgfqpoint{2.642414in}{2.268904in}}{\pgfqpoint{2.651622in}{2.259695in}}%
\pgfpathcurveto{\pgfqpoint{2.660831in}{2.250487in}}{\pgfqpoint{2.673322in}{2.245313in}}{\pgfqpoint{2.686344in}{2.245313in}}%
\pgfpathlineto{\pgfqpoint{2.686344in}{2.245313in}}%
\pgfpathclose%
\pgfusepath{stroke,fill}%
\end{pgfscope}%
\begin{pgfscope}%
\pgfpathrectangle{\pgfqpoint{0.786164in}{0.768110in}}{\pgfqpoint{8.851069in}{7.081890in}}%
\pgfusepath{clip}%
\pgfsetbuttcap%
\pgfsetroundjoin%
\definecolor{currentfill}{rgb}{0.210503,0.363727,0.552206}%
\pgfsetfillcolor{currentfill}%
\pgfsetfillopacity{0.700000}%
\pgfsetlinewidth{0.501875pt}%
\definecolor{currentstroke}{rgb}{1.000000,1.000000,1.000000}%
\pgfsetstrokecolor{currentstroke}%
\pgfsetstrokeopacity{0.700000}%
\pgfsetdash{}{0pt}%
\pgfpathmoveto{\pgfqpoint{2.869010in}{2.311008in}}%
\pgfpathcurveto{\pgfqpoint{2.882033in}{2.311008in}}{\pgfqpoint{2.894524in}{2.316182in}}{\pgfqpoint{2.903732in}{2.325390in}}%
\pgfpathcurveto{\pgfqpoint{2.912941in}{2.334598in}}{\pgfqpoint{2.918115in}{2.347089in}}{\pgfqpoint{2.918115in}{2.360112in}}%
\pgfpathcurveto{\pgfqpoint{2.918115in}{2.373135in}}{\pgfqpoint{2.912941in}{2.385626in}}{\pgfqpoint{2.903732in}{2.394834in}}%
\pgfpathcurveto{\pgfqpoint{2.894524in}{2.404043in}}{\pgfqpoint{2.882033in}{2.409217in}}{\pgfqpoint{2.869010in}{2.409217in}}%
\pgfpathcurveto{\pgfqpoint{2.855987in}{2.409217in}}{\pgfqpoint{2.843496in}{2.404043in}}{\pgfqpoint{2.834288in}{2.394834in}}%
\pgfpathcurveto{\pgfqpoint{2.825079in}{2.385626in}}{\pgfqpoint{2.819905in}{2.373135in}}{\pgfqpoint{2.819905in}{2.360112in}}%
\pgfpathcurveto{\pgfqpoint{2.819905in}{2.347089in}}{\pgfqpoint{2.825079in}{2.334598in}}{\pgfqpoint{2.834288in}{2.325390in}}%
\pgfpathcurveto{\pgfqpoint{2.843496in}{2.316182in}}{\pgfqpoint{2.855987in}{2.311008in}}{\pgfqpoint{2.869010in}{2.311008in}}%
\pgfpathlineto{\pgfqpoint{2.869010in}{2.311008in}}%
\pgfpathclose%
\pgfusepath{stroke,fill}%
\end{pgfscope}%
\begin{pgfscope}%
\pgfpathrectangle{\pgfqpoint{0.786164in}{0.768110in}}{\pgfqpoint{8.851069in}{7.081890in}}%
\pgfusepath{clip}%
\pgfsetbuttcap%
\pgfsetroundjoin%
\definecolor{currentfill}{rgb}{0.203063,0.379716,0.553925}%
\pgfsetfillcolor{currentfill}%
\pgfsetfillopacity{0.700000}%
\pgfsetlinewidth{0.501875pt}%
\definecolor{currentstroke}{rgb}{1.000000,1.000000,1.000000}%
\pgfsetstrokecolor{currentstroke}%
\pgfsetstrokeopacity{0.700000}%
\pgfsetdash{}{0pt}%
\pgfpathmoveto{\pgfqpoint{2.622411in}{2.267211in}}%
\pgfpathcurveto{\pgfqpoint{2.635434in}{2.267211in}}{\pgfqpoint{2.647925in}{2.272385in}}{\pgfqpoint{2.657134in}{2.281593in}}%
\pgfpathcurveto{\pgfqpoint{2.666342in}{2.290802in}}{\pgfqpoint{2.671516in}{2.303293in}}{\pgfqpoint{2.671516in}{2.316316in}}%
\pgfpathcurveto{\pgfqpoint{2.671516in}{2.329338in}}{\pgfqpoint{2.666342in}{2.341829in}}{\pgfqpoint{2.657134in}{2.351038in}}%
\pgfpathcurveto{\pgfqpoint{2.647925in}{2.360246in}}{\pgfqpoint{2.635434in}{2.365420in}}{\pgfqpoint{2.622411in}{2.365420in}}%
\pgfpathcurveto{\pgfqpoint{2.609389in}{2.365420in}}{\pgfqpoint{2.596898in}{2.360246in}}{\pgfqpoint{2.587689in}{2.351038in}}%
\pgfpathcurveto{\pgfqpoint{2.578481in}{2.341829in}}{\pgfqpoint{2.573307in}{2.329338in}}{\pgfqpoint{2.573307in}{2.316316in}}%
\pgfpathcurveto{\pgfqpoint{2.573307in}{2.303293in}}{\pgfqpoint{2.578481in}{2.290802in}}{\pgfqpoint{2.587689in}{2.281593in}}%
\pgfpathcurveto{\pgfqpoint{2.596898in}{2.272385in}}{\pgfqpoint{2.609389in}{2.267211in}}{\pgfqpoint{2.622411in}{2.267211in}}%
\pgfpathlineto{\pgfqpoint{2.622411in}{2.267211in}}%
\pgfpathclose%
\pgfusepath{stroke,fill}%
\end{pgfscope}%
\begin{pgfscope}%
\pgfpathrectangle{\pgfqpoint{0.786164in}{0.768110in}}{\pgfqpoint{8.851069in}{7.081890in}}%
\pgfusepath{clip}%
\pgfsetbuttcap%
\pgfsetroundjoin%
\definecolor{currentfill}{rgb}{0.182256,0.426184,0.557120}%
\pgfsetfillcolor{currentfill}%
\pgfsetfillopacity{0.700000}%
\pgfsetlinewidth{0.501875pt}%
\definecolor{currentstroke}{rgb}{1.000000,1.000000,1.000000}%
\pgfsetstrokecolor{currentstroke}%
\pgfsetstrokeopacity{0.700000}%
\pgfsetdash{}{0pt}%
\pgfpathmoveto{\pgfqpoint{1.773016in}{2.179618in}}%
\pgfpathcurveto{\pgfqpoint{1.786038in}{2.179618in}}{\pgfqpoint{1.798529in}{2.184792in}}{\pgfqpoint{1.807738in}{2.194001in}}%
\pgfpathcurveto{\pgfqpoint{1.816946in}{2.203209in}}{\pgfqpoint{1.822120in}{2.215700in}}{\pgfqpoint{1.822120in}{2.228723in}}%
\pgfpathcurveto{\pgfqpoint{1.822120in}{2.241745in}}{\pgfqpoint{1.816946in}{2.254237in}}{\pgfqpoint{1.807738in}{2.263445in}}%
\pgfpathcurveto{\pgfqpoint{1.798529in}{2.272653in}}{\pgfqpoint{1.786038in}{2.277827in}}{\pgfqpoint{1.773016in}{2.277827in}}%
\pgfpathcurveto{\pgfqpoint{1.759993in}{2.277827in}}{\pgfqpoint{1.747502in}{2.272653in}}{\pgfqpoint{1.738293in}{2.263445in}}%
\pgfpathcurveto{\pgfqpoint{1.729085in}{2.254237in}}{\pgfqpoint{1.723911in}{2.241745in}}{\pgfqpoint{1.723911in}{2.228723in}}%
\pgfpathcurveto{\pgfqpoint{1.723911in}{2.215700in}}{\pgfqpoint{1.729085in}{2.203209in}}{\pgfqpoint{1.738293in}{2.194001in}}%
\pgfpathcurveto{\pgfqpoint{1.747502in}{2.184792in}}{\pgfqpoint{1.759993in}{2.179618in}}{\pgfqpoint{1.773016in}{2.179618in}}%
\pgfpathlineto{\pgfqpoint{1.773016in}{2.179618in}}%
\pgfpathclose%
\pgfusepath{stroke,fill}%
\end{pgfscope}%
\begin{pgfscope}%
\pgfpathrectangle{\pgfqpoint{0.786164in}{0.768110in}}{\pgfqpoint{8.851069in}{7.081890in}}%
\pgfusepath{clip}%
\pgfsetbuttcap%
\pgfsetroundjoin%
\definecolor{currentfill}{rgb}{0.180629,0.429975,0.557282}%
\pgfsetfillcolor{currentfill}%
\pgfsetfillopacity{0.700000}%
\pgfsetlinewidth{0.501875pt}%
\definecolor{currentstroke}{rgb}{1.000000,1.000000,1.000000}%
\pgfsetstrokecolor{currentstroke}%
\pgfsetstrokeopacity{0.700000}%
\pgfsetdash{}{0pt}%
\pgfpathmoveto{\pgfqpoint{1.809549in}{2.201516in}}%
\pgfpathcurveto{\pgfqpoint{1.822571in}{2.201516in}}{\pgfqpoint{1.835063in}{2.206690in}}{\pgfqpoint{1.844271in}{2.215899in}}%
\pgfpathcurveto{\pgfqpoint{1.853479in}{2.225107in}}{\pgfqpoint{1.858653in}{2.237598in}}{\pgfqpoint{1.858653in}{2.250621in}}%
\pgfpathcurveto{\pgfqpoint{1.858653in}{2.263644in}}{\pgfqpoint{1.853479in}{2.276135in}}{\pgfqpoint{1.844271in}{2.285343in}}%
\pgfpathcurveto{\pgfqpoint{1.835063in}{2.294552in}}{\pgfqpoint{1.822571in}{2.299726in}}{\pgfqpoint{1.809549in}{2.299726in}}%
\pgfpathcurveto{\pgfqpoint{1.796526in}{2.299726in}}{\pgfqpoint{1.784035in}{2.294552in}}{\pgfqpoint{1.774827in}{2.285343in}}%
\pgfpathcurveto{\pgfqpoint{1.765618in}{2.276135in}}{\pgfqpoint{1.760444in}{2.263644in}}{\pgfqpoint{1.760444in}{2.250621in}}%
\pgfpathcurveto{\pgfqpoint{1.760444in}{2.237598in}}{\pgfqpoint{1.765618in}{2.225107in}}{\pgfqpoint{1.774827in}{2.215899in}}%
\pgfpathcurveto{\pgfqpoint{1.784035in}{2.206690in}}{\pgfqpoint{1.796526in}{2.201516in}}{\pgfqpoint{1.809549in}{2.201516in}}%
\pgfpathlineto{\pgfqpoint{1.809549in}{2.201516in}}%
\pgfpathclose%
\pgfusepath{stroke,fill}%
\end{pgfscope}%
\begin{pgfscope}%
\pgfpathrectangle{\pgfqpoint{0.786164in}{0.768110in}}{\pgfqpoint{8.851069in}{7.081890in}}%
\pgfusepath{clip}%
\pgfsetbuttcap%
\pgfsetroundjoin%
\definecolor{currentfill}{rgb}{0.187231,0.414746,0.556547}%
\pgfsetfillcolor{currentfill}%
\pgfsetfillopacity{0.700000}%
\pgfsetlinewidth{0.501875pt}%
\definecolor{currentstroke}{rgb}{1.000000,1.000000,1.000000}%
\pgfsetstrokecolor{currentstroke}%
\pgfsetstrokeopacity{0.700000}%
\pgfsetdash{}{0pt}%
\pgfpathmoveto{\pgfqpoint{1.800415in}{2.223415in}}%
\pgfpathcurveto{\pgfqpoint{1.813438in}{2.223415in}}{\pgfqpoint{1.825929in}{2.228589in}}{\pgfqpoint{1.835138in}{2.237797in}}%
\pgfpathcurveto{\pgfqpoint{1.844346in}{2.247005in}}{\pgfqpoint{1.849520in}{2.259497in}}{\pgfqpoint{1.849520in}{2.272519in}}%
\pgfpathcurveto{\pgfqpoint{1.849520in}{2.285542in}}{\pgfqpoint{1.844346in}{2.298033in}}{\pgfqpoint{1.835138in}{2.307241in}}%
\pgfpathcurveto{\pgfqpoint{1.825929in}{2.316450in}}{\pgfqpoint{1.813438in}{2.321624in}}{\pgfqpoint{1.800415in}{2.321624in}}%
\pgfpathcurveto{\pgfqpoint{1.787393in}{2.321624in}}{\pgfqpoint{1.774902in}{2.316450in}}{\pgfqpoint{1.765693in}{2.307241in}}%
\pgfpathcurveto{\pgfqpoint{1.756485in}{2.298033in}}{\pgfqpoint{1.751311in}{2.285542in}}{\pgfqpoint{1.751311in}{2.272519in}}%
\pgfpathcurveto{\pgfqpoint{1.751311in}{2.259497in}}{\pgfqpoint{1.756485in}{2.247005in}}{\pgfqpoint{1.765693in}{2.237797in}}%
\pgfpathcurveto{\pgfqpoint{1.774902in}{2.228589in}}{\pgfqpoint{1.787393in}{2.223415in}}{\pgfqpoint{1.800415in}{2.223415in}}%
\pgfpathlineto{\pgfqpoint{1.800415in}{2.223415in}}%
\pgfpathclose%
\pgfusepath{stroke,fill}%
\end{pgfscope}%
\begin{pgfscope}%
\pgfpathrectangle{\pgfqpoint{0.786164in}{0.768110in}}{\pgfqpoint{8.851069in}{7.081890in}}%
\pgfusepath{clip}%
\pgfsetbuttcap%
\pgfsetroundjoin%
\definecolor{currentfill}{rgb}{0.190631,0.407061,0.556089}%
\pgfsetfillcolor{currentfill}%
\pgfsetfillopacity{0.700000}%
\pgfsetlinewidth{0.501875pt}%
\definecolor{currentstroke}{rgb}{1.000000,1.000000,1.000000}%
\pgfsetstrokecolor{currentstroke}%
\pgfsetstrokeopacity{0.700000}%
\pgfsetdash{}{0pt}%
\pgfpathmoveto{\pgfqpoint{1.873482in}{2.354804in}}%
\pgfpathcurveto{\pgfqpoint{1.886504in}{2.354804in}}{\pgfqpoint{1.898996in}{2.359978in}}{\pgfqpoint{1.908204in}{2.369186in}}%
\pgfpathcurveto{\pgfqpoint{1.917412in}{2.378395in}}{\pgfqpoint{1.922586in}{2.390886in}}{\pgfqpoint{1.922586in}{2.403909in}}%
\pgfpathcurveto{\pgfqpoint{1.922586in}{2.416931in}}{\pgfqpoint{1.917412in}{2.429422in}}{\pgfqpoint{1.908204in}{2.438631in}}%
\pgfpathcurveto{\pgfqpoint{1.898996in}{2.447839in}}{\pgfqpoint{1.886504in}{2.453013in}}{\pgfqpoint{1.873482in}{2.453013in}}%
\pgfpathcurveto{\pgfqpoint{1.860459in}{2.453013in}}{\pgfqpoint{1.847968in}{2.447839in}}{\pgfqpoint{1.838760in}{2.438631in}}%
\pgfpathcurveto{\pgfqpoint{1.829551in}{2.429422in}}{\pgfqpoint{1.824377in}{2.416931in}}{\pgfqpoint{1.824377in}{2.403909in}}%
\pgfpathcurveto{\pgfqpoint{1.824377in}{2.390886in}}{\pgfqpoint{1.829551in}{2.378395in}}{\pgfqpoint{1.838760in}{2.369186in}}%
\pgfpathcurveto{\pgfqpoint{1.847968in}{2.359978in}}{\pgfqpoint{1.860459in}{2.354804in}}{\pgfqpoint{1.873482in}{2.354804in}}%
\pgfpathlineto{\pgfqpoint{1.873482in}{2.354804in}}%
\pgfpathclose%
\pgfusepath{stroke,fill}%
\end{pgfscope}%
\begin{pgfscope}%
\pgfpathrectangle{\pgfqpoint{0.786164in}{0.768110in}}{\pgfqpoint{8.851069in}{7.081890in}}%
\pgfusepath{clip}%
\pgfsetbuttcap%
\pgfsetroundjoin%
\definecolor{currentfill}{rgb}{0.190631,0.407061,0.556089}%
\pgfsetfillcolor{currentfill}%
\pgfsetfillopacity{0.700000}%
\pgfsetlinewidth{0.501875pt}%
\definecolor{currentstroke}{rgb}{1.000000,1.000000,1.000000}%
\pgfsetstrokecolor{currentstroke}%
\pgfsetstrokeopacity{0.700000}%
\pgfsetdash{}{0pt}%
\pgfpathmoveto{\pgfqpoint{1.818682in}{2.311008in}}%
\pgfpathcurveto{\pgfqpoint{1.831705in}{2.311008in}}{\pgfqpoint{1.844196in}{2.316182in}}{\pgfqpoint{1.853404in}{2.325390in}}%
\pgfpathcurveto{\pgfqpoint{1.862613in}{2.334598in}}{\pgfqpoint{1.867787in}{2.347089in}}{\pgfqpoint{1.867787in}{2.360112in}}%
\pgfpathcurveto{\pgfqpoint{1.867787in}{2.373135in}}{\pgfqpoint{1.862613in}{2.385626in}}{\pgfqpoint{1.853404in}{2.394834in}}%
\pgfpathcurveto{\pgfqpoint{1.844196in}{2.404043in}}{\pgfqpoint{1.831705in}{2.409217in}}{\pgfqpoint{1.818682in}{2.409217in}}%
\pgfpathcurveto{\pgfqpoint{1.805659in}{2.409217in}}{\pgfqpoint{1.793168in}{2.404043in}}{\pgfqpoint{1.783960in}{2.394834in}}%
\pgfpathcurveto{\pgfqpoint{1.774751in}{2.385626in}}{\pgfqpoint{1.769577in}{2.373135in}}{\pgfqpoint{1.769577in}{2.360112in}}%
\pgfpathcurveto{\pgfqpoint{1.769577in}{2.347089in}}{\pgfqpoint{1.774751in}{2.334598in}}{\pgfqpoint{1.783960in}{2.325390in}}%
\pgfpathcurveto{\pgfqpoint{1.793168in}{2.316182in}}{\pgfqpoint{1.805659in}{2.311008in}}{\pgfqpoint{1.818682in}{2.311008in}}%
\pgfpathlineto{\pgfqpoint{1.818682in}{2.311008in}}%
\pgfpathclose%
\pgfusepath{stroke,fill}%
\end{pgfscope}%
\begin{pgfscope}%
\pgfpathrectangle{\pgfqpoint{0.786164in}{0.768110in}}{\pgfqpoint{8.851069in}{7.081890in}}%
\pgfusepath{clip}%
\pgfsetbuttcap%
\pgfsetroundjoin%
\definecolor{currentfill}{rgb}{0.180629,0.429975,0.557282}%
\pgfsetfillcolor{currentfill}%
\pgfsetfillopacity{0.700000}%
\pgfsetlinewidth{0.501875pt}%
\definecolor{currentstroke}{rgb}{1.000000,1.000000,1.000000}%
\pgfsetstrokecolor{currentstroke}%
\pgfsetstrokeopacity{0.700000}%
\pgfsetdash{}{0pt}%
\pgfpathmoveto{\pgfqpoint{1.773016in}{2.201516in}}%
\pgfpathcurveto{\pgfqpoint{1.786038in}{2.201516in}}{\pgfqpoint{1.798529in}{2.206690in}}{\pgfqpoint{1.807738in}{2.215899in}}%
\pgfpathcurveto{\pgfqpoint{1.816946in}{2.225107in}}{\pgfqpoint{1.822120in}{2.237598in}}{\pgfqpoint{1.822120in}{2.250621in}}%
\pgfpathcurveto{\pgfqpoint{1.822120in}{2.263644in}}{\pgfqpoint{1.816946in}{2.276135in}}{\pgfqpoint{1.807738in}{2.285343in}}%
\pgfpathcurveto{\pgfqpoint{1.798529in}{2.294552in}}{\pgfqpoint{1.786038in}{2.299726in}}{\pgfqpoint{1.773016in}{2.299726in}}%
\pgfpathcurveto{\pgfqpoint{1.759993in}{2.299726in}}{\pgfqpoint{1.747502in}{2.294552in}}{\pgfqpoint{1.738293in}{2.285343in}}%
\pgfpathcurveto{\pgfqpoint{1.729085in}{2.276135in}}{\pgfqpoint{1.723911in}{2.263644in}}{\pgfqpoint{1.723911in}{2.250621in}}%
\pgfpathcurveto{\pgfqpoint{1.723911in}{2.237598in}}{\pgfqpoint{1.729085in}{2.225107in}}{\pgfqpoint{1.738293in}{2.215899in}}%
\pgfpathcurveto{\pgfqpoint{1.747502in}{2.206690in}}{\pgfqpoint{1.759993in}{2.201516in}}{\pgfqpoint{1.773016in}{2.201516in}}%
\pgfpathlineto{\pgfqpoint{1.773016in}{2.201516in}}%
\pgfpathclose%
\pgfusepath{stroke,fill}%
\end{pgfscope}%
\begin{pgfscope}%
\pgfpathrectangle{\pgfqpoint{0.786164in}{0.768110in}}{\pgfqpoint{8.851069in}{7.081890in}}%
\pgfusepath{clip}%
\pgfsetbuttcap%
\pgfsetroundjoin%
\definecolor{currentfill}{rgb}{0.171176,0.452530,0.557965}%
\pgfsetfillcolor{currentfill}%
\pgfsetfillopacity{0.700000}%
\pgfsetlinewidth{0.501875pt}%
\definecolor{currentstroke}{rgb}{1.000000,1.000000,1.000000}%
\pgfsetstrokecolor{currentstroke}%
\pgfsetstrokeopacity{0.700000}%
\pgfsetdash{}{0pt}%
\pgfpathmoveto{\pgfqpoint{1.754749in}{2.179618in}}%
\pgfpathcurveto{\pgfqpoint{1.767772in}{2.179618in}}{\pgfqpoint{1.780263in}{2.184792in}}{\pgfqpoint{1.789471in}{2.194001in}}%
\pgfpathcurveto{\pgfqpoint{1.798680in}{2.203209in}}{\pgfqpoint{1.803854in}{2.215700in}}{\pgfqpoint{1.803854in}{2.228723in}}%
\pgfpathcurveto{\pgfqpoint{1.803854in}{2.241745in}}{\pgfqpoint{1.798680in}{2.254237in}}{\pgfqpoint{1.789471in}{2.263445in}}%
\pgfpathcurveto{\pgfqpoint{1.780263in}{2.272653in}}{\pgfqpoint{1.767772in}{2.277827in}}{\pgfqpoint{1.754749in}{2.277827in}}%
\pgfpathcurveto{\pgfqpoint{1.741726in}{2.277827in}}{\pgfqpoint{1.729235in}{2.272653in}}{\pgfqpoint{1.720027in}{2.263445in}}%
\pgfpathcurveto{\pgfqpoint{1.710818in}{2.254237in}}{\pgfqpoint{1.705644in}{2.241745in}}{\pgfqpoint{1.705644in}{2.228723in}}%
\pgfpathcurveto{\pgfqpoint{1.705644in}{2.215700in}}{\pgfqpoint{1.710818in}{2.203209in}}{\pgfqpoint{1.720027in}{2.194001in}}%
\pgfpathcurveto{\pgfqpoint{1.729235in}{2.184792in}}{\pgfqpoint{1.741726in}{2.179618in}}{\pgfqpoint{1.754749in}{2.179618in}}%
\pgfpathlineto{\pgfqpoint{1.754749in}{2.179618in}}%
\pgfpathclose%
\pgfusepath{stroke,fill}%
\end{pgfscope}%
\begin{pgfscope}%
\pgfpathrectangle{\pgfqpoint{0.786164in}{0.768110in}}{\pgfqpoint{8.851069in}{7.081890in}}%
\pgfusepath{clip}%
\pgfsetbuttcap%
\pgfsetroundjoin%
\definecolor{currentfill}{rgb}{0.169646,0.456262,0.558030}%
\pgfsetfillcolor{currentfill}%
\pgfsetfillopacity{0.700000}%
\pgfsetlinewidth{0.501875pt}%
\definecolor{currentstroke}{rgb}{1.000000,1.000000,1.000000}%
\pgfsetstrokecolor{currentstroke}%
\pgfsetstrokeopacity{0.700000}%
\pgfsetdash{}{0pt}%
\pgfpathmoveto{\pgfqpoint{1.718216in}{2.245313in}}%
\pgfpathcurveto{\pgfqpoint{1.731239in}{2.245313in}}{\pgfqpoint{1.743730in}{2.250487in}}{\pgfqpoint{1.752938in}{2.259695in}}%
\pgfpathcurveto{\pgfqpoint{1.762147in}{2.268904in}}{\pgfqpoint{1.767321in}{2.281395in}}{\pgfqpoint{1.767321in}{2.294417in}}%
\pgfpathcurveto{\pgfqpoint{1.767321in}{2.307440in}}{\pgfqpoint{1.762147in}{2.319931in}}{\pgfqpoint{1.752938in}{2.329140in}}%
\pgfpathcurveto{\pgfqpoint{1.743730in}{2.338348in}}{\pgfqpoint{1.731239in}{2.343522in}}{\pgfqpoint{1.718216in}{2.343522in}}%
\pgfpathcurveto{\pgfqpoint{1.705193in}{2.343522in}}{\pgfqpoint{1.692702in}{2.338348in}}{\pgfqpoint{1.683494in}{2.329140in}}%
\pgfpathcurveto{\pgfqpoint{1.674285in}{2.319931in}}{\pgfqpoint{1.669111in}{2.307440in}}{\pgfqpoint{1.669111in}{2.294417in}}%
\pgfpathcurveto{\pgfqpoint{1.669111in}{2.281395in}}{\pgfqpoint{1.674285in}{2.268904in}}{\pgfqpoint{1.683494in}{2.259695in}}%
\pgfpathcurveto{\pgfqpoint{1.692702in}{2.250487in}}{\pgfqpoint{1.705193in}{2.245313in}}{\pgfqpoint{1.718216in}{2.245313in}}%
\pgfpathlineto{\pgfqpoint{1.718216in}{2.245313in}}%
\pgfpathclose%
\pgfusepath{stroke,fill}%
\end{pgfscope}%
\begin{pgfscope}%
\pgfpathrectangle{\pgfqpoint{0.786164in}{0.768110in}}{\pgfqpoint{8.851069in}{7.081890in}}%
\pgfusepath{clip}%
\pgfsetbuttcap%
\pgfsetroundjoin%
\definecolor{currentfill}{rgb}{0.162142,0.474838,0.558140}%
\pgfsetfillcolor{currentfill}%
\pgfsetfillopacity{0.700000}%
\pgfsetlinewidth{0.501875pt}%
\definecolor{currentstroke}{rgb}{1.000000,1.000000,1.000000}%
\pgfsetstrokecolor{currentstroke}%
\pgfsetstrokeopacity{0.700000}%
\pgfsetdash{}{0pt}%
\pgfpathmoveto{\pgfqpoint{1.626883in}{2.179618in}}%
\pgfpathcurveto{\pgfqpoint{1.639906in}{2.179618in}}{\pgfqpoint{1.652397in}{2.184792in}}{\pgfqpoint{1.661605in}{2.194001in}}%
\pgfpathcurveto{\pgfqpoint{1.670814in}{2.203209in}}{\pgfqpoint{1.675988in}{2.215700in}}{\pgfqpoint{1.675988in}{2.228723in}}%
\pgfpathcurveto{\pgfqpoint{1.675988in}{2.241745in}}{\pgfqpoint{1.670814in}{2.254237in}}{\pgfqpoint{1.661605in}{2.263445in}}%
\pgfpathcurveto{\pgfqpoint{1.652397in}{2.272653in}}{\pgfqpoint{1.639906in}{2.277827in}}{\pgfqpoint{1.626883in}{2.277827in}}%
\pgfpathcurveto{\pgfqpoint{1.613860in}{2.277827in}}{\pgfqpoint{1.601369in}{2.272653in}}{\pgfqpoint{1.592161in}{2.263445in}}%
\pgfpathcurveto{\pgfqpoint{1.582952in}{2.254237in}}{\pgfqpoint{1.577778in}{2.241745in}}{\pgfqpoint{1.577778in}{2.228723in}}%
\pgfpathcurveto{\pgfqpoint{1.577778in}{2.215700in}}{\pgfqpoint{1.582952in}{2.203209in}}{\pgfqpoint{1.592161in}{2.194001in}}%
\pgfpathcurveto{\pgfqpoint{1.601369in}{2.184792in}}{\pgfqpoint{1.613860in}{2.179618in}}{\pgfqpoint{1.626883in}{2.179618in}}%
\pgfpathlineto{\pgfqpoint{1.626883in}{2.179618in}}%
\pgfpathclose%
\pgfusepath{stroke,fill}%
\end{pgfscope}%
\begin{pgfscope}%
\pgfpathrectangle{\pgfqpoint{0.786164in}{0.768110in}}{\pgfqpoint{8.851069in}{7.081890in}}%
\pgfusepath{clip}%
\pgfsetbuttcap%
\pgfsetroundjoin%
\definecolor{currentfill}{rgb}{0.154815,0.493313,0.557840}%
\pgfsetfillcolor{currentfill}%
\pgfsetfillopacity{0.700000}%
\pgfsetlinewidth{0.501875pt}%
\definecolor{currentstroke}{rgb}{1.000000,1.000000,1.000000}%
\pgfsetstrokecolor{currentstroke}%
\pgfsetstrokeopacity{0.700000}%
\pgfsetdash{}{0pt}%
\pgfpathmoveto{\pgfqpoint{1.590350in}{2.092025in}}%
\pgfpathcurveto{\pgfqpoint{1.603373in}{2.092025in}}{\pgfqpoint{1.615864in}{2.097199in}}{\pgfqpoint{1.625072in}{2.106408in}}%
\pgfpathcurveto{\pgfqpoint{1.634281in}{2.115616in}}{\pgfqpoint{1.639454in}{2.128107in}}{\pgfqpoint{1.639454in}{2.141130in}}%
\pgfpathcurveto{\pgfqpoint{1.639454in}{2.154153in}}{\pgfqpoint{1.634281in}{2.166644in}}{\pgfqpoint{1.625072in}{2.175852in}}%
\pgfpathcurveto{\pgfqpoint{1.615864in}{2.185060in}}{\pgfqpoint{1.603373in}{2.190234in}}{\pgfqpoint{1.590350in}{2.190234in}}%
\pgfpathcurveto{\pgfqpoint{1.577327in}{2.190234in}}{\pgfqpoint{1.564836in}{2.185060in}}{\pgfqpoint{1.555628in}{2.175852in}}%
\pgfpathcurveto{\pgfqpoint{1.546419in}{2.166644in}}{\pgfqpoint{1.541245in}{2.154153in}}{\pgfqpoint{1.541245in}{2.141130in}}%
\pgfpathcurveto{\pgfqpoint{1.541245in}{2.128107in}}{\pgfqpoint{1.546419in}{2.115616in}}{\pgfqpoint{1.555628in}{2.106408in}}%
\pgfpathcurveto{\pgfqpoint{1.564836in}{2.097199in}}{\pgfqpoint{1.577327in}{2.092025in}}{\pgfqpoint{1.590350in}{2.092025in}}%
\pgfpathlineto{\pgfqpoint{1.590350in}{2.092025in}}%
\pgfpathclose%
\pgfusepath{stroke,fill}%
\end{pgfscope}%
\begin{pgfscope}%
\pgfpathrectangle{\pgfqpoint{0.786164in}{0.768110in}}{\pgfqpoint{8.851069in}{7.081890in}}%
\pgfusepath{clip}%
\pgfsetbuttcap%
\pgfsetroundjoin%
\definecolor{currentfill}{rgb}{0.156270,0.489624,0.557936}%
\pgfsetfillcolor{currentfill}%
\pgfsetfillopacity{0.700000}%
\pgfsetlinewidth{0.501875pt}%
\definecolor{currentstroke}{rgb}{1.000000,1.000000,1.000000}%
\pgfsetstrokecolor{currentstroke}%
\pgfsetstrokeopacity{0.700000}%
\pgfsetdash{}{0pt}%
\pgfpathmoveto{\pgfqpoint{1.508150in}{2.070127in}}%
\pgfpathcurveto{\pgfqpoint{1.521173in}{2.070127in}}{\pgfqpoint{1.533664in}{2.075301in}}{\pgfqpoint{1.542872in}{2.084509in}}%
\pgfpathcurveto{\pgfqpoint{1.552081in}{2.093718in}}{\pgfqpoint{1.557255in}{2.106209in}}{\pgfqpoint{1.557255in}{2.119232in}}%
\pgfpathcurveto{\pgfqpoint{1.557255in}{2.132254in}}{\pgfqpoint{1.552081in}{2.144745in}}{\pgfqpoint{1.542872in}{2.153954in}}%
\pgfpathcurveto{\pgfqpoint{1.533664in}{2.163162in}}{\pgfqpoint{1.521173in}{2.168336in}}{\pgfqpoint{1.508150in}{2.168336in}}%
\pgfpathcurveto{\pgfqpoint{1.495128in}{2.168336in}}{\pgfqpoint{1.482636in}{2.163162in}}{\pgfqpoint{1.473428in}{2.153954in}}%
\pgfpathcurveto{\pgfqpoint{1.464220in}{2.144745in}}{\pgfqpoint{1.459046in}{2.132254in}}{\pgfqpoint{1.459046in}{2.119232in}}%
\pgfpathcurveto{\pgfqpoint{1.459046in}{2.106209in}}{\pgfqpoint{1.464220in}{2.093718in}}{\pgfqpoint{1.473428in}{2.084509in}}%
\pgfpathcurveto{\pgfqpoint{1.482636in}{2.075301in}}{\pgfqpoint{1.495128in}{2.070127in}}{\pgfqpoint{1.508150in}{2.070127in}}%
\pgfpathlineto{\pgfqpoint{1.508150in}{2.070127in}}%
\pgfpathclose%
\pgfusepath{stroke,fill}%
\end{pgfscope}%
\begin{pgfscope}%
\pgfpathrectangle{\pgfqpoint{0.786164in}{0.768110in}}{\pgfqpoint{8.851069in}{7.081890in}}%
\pgfusepath{clip}%
\pgfsetbuttcap%
\pgfsetroundjoin%
\definecolor{currentfill}{rgb}{0.149039,0.508051,0.557250}%
\pgfsetfillcolor{currentfill}%
\pgfsetfillopacity{0.700000}%
\pgfsetlinewidth{0.501875pt}%
\definecolor{currentstroke}{rgb}{1.000000,1.000000,1.000000}%
\pgfsetstrokecolor{currentstroke}%
\pgfsetstrokeopacity{0.700000}%
\pgfsetdash{}{0pt}%
\pgfpathmoveto{\pgfqpoint{1.599483in}{2.113923in}}%
\pgfpathcurveto{\pgfqpoint{1.612506in}{2.113923in}}{\pgfqpoint{1.624997in}{2.119097in}}{\pgfqpoint{1.634205in}{2.128306in}}%
\pgfpathcurveto{\pgfqpoint{1.643414in}{2.137514in}}{\pgfqpoint{1.648588in}{2.150005in}}{\pgfqpoint{1.648588in}{2.163028in}}%
\pgfpathcurveto{\pgfqpoint{1.648588in}{2.176051in}}{\pgfqpoint{1.643414in}{2.188542in}}{\pgfqpoint{1.634205in}{2.197750in}}%
\pgfpathcurveto{\pgfqpoint{1.624997in}{2.206959in}}{\pgfqpoint{1.612506in}{2.212133in}}{\pgfqpoint{1.599483in}{2.212133in}}%
\pgfpathcurveto{\pgfqpoint{1.586460in}{2.212133in}}{\pgfqpoint{1.573969in}{2.206959in}}{\pgfqpoint{1.564761in}{2.197750in}}%
\pgfpathcurveto{\pgfqpoint{1.555552in}{2.188542in}}{\pgfqpoint{1.550379in}{2.176051in}}{\pgfqpoint{1.550379in}{2.163028in}}%
\pgfpathcurveto{\pgfqpoint{1.550379in}{2.150005in}}{\pgfqpoint{1.555552in}{2.137514in}}{\pgfqpoint{1.564761in}{2.128306in}}%
\pgfpathcurveto{\pgfqpoint{1.573969in}{2.119097in}}{\pgfqpoint{1.586460in}{2.113923in}}{\pgfqpoint{1.599483in}{2.113923in}}%
\pgfpathlineto{\pgfqpoint{1.599483in}{2.113923in}}%
\pgfpathclose%
\pgfusepath{stroke,fill}%
\end{pgfscope}%
\begin{pgfscope}%
\pgfpathrectangle{\pgfqpoint{0.786164in}{0.768110in}}{\pgfqpoint{8.851069in}{7.081890in}}%
\pgfusepath{clip}%
\pgfsetbuttcap%
\pgfsetroundjoin%
\definecolor{currentfill}{rgb}{0.153364,0.497000,0.557724}%
\pgfsetfillcolor{currentfill}%
\pgfsetfillopacity{0.700000}%
\pgfsetlinewidth{0.501875pt}%
\definecolor{currentstroke}{rgb}{1.000000,1.000000,1.000000}%
\pgfsetstrokecolor{currentstroke}%
\pgfsetstrokeopacity{0.700000}%
\pgfsetdash{}{0pt}%
\pgfpathmoveto{\pgfqpoint{1.636016in}{2.135822in}}%
\pgfpathcurveto{\pgfqpoint{1.649039in}{2.135822in}}{\pgfqpoint{1.661530in}{2.140996in}}{\pgfqpoint{1.670739in}{2.150204in}}%
\pgfpathcurveto{\pgfqpoint{1.679947in}{2.159413in}}{\pgfqpoint{1.685121in}{2.171904in}}{\pgfqpoint{1.685121in}{2.184926in}}%
\pgfpathcurveto{\pgfqpoint{1.685121in}{2.197949in}}{\pgfqpoint{1.679947in}{2.210440in}}{\pgfqpoint{1.670739in}{2.219649in}}%
\pgfpathcurveto{\pgfqpoint{1.661530in}{2.228857in}}{\pgfqpoint{1.649039in}{2.234031in}}{\pgfqpoint{1.636016in}{2.234031in}}%
\pgfpathcurveto{\pgfqpoint{1.622994in}{2.234031in}}{\pgfqpoint{1.610503in}{2.228857in}}{\pgfqpoint{1.601294in}{2.219649in}}%
\pgfpathcurveto{\pgfqpoint{1.592086in}{2.210440in}}{\pgfqpoint{1.586912in}{2.197949in}}{\pgfqpoint{1.586912in}{2.184926in}}%
\pgfpathcurveto{\pgfqpoint{1.586912in}{2.171904in}}{\pgfqpoint{1.592086in}{2.159413in}}{\pgfqpoint{1.601294in}{2.150204in}}%
\pgfpathcurveto{\pgfqpoint{1.610503in}{2.140996in}}{\pgfqpoint{1.622994in}{2.135822in}}{\pgfqpoint{1.636016in}{2.135822in}}%
\pgfpathlineto{\pgfqpoint{1.636016in}{2.135822in}}%
\pgfpathclose%
\pgfusepath{stroke,fill}%
\end{pgfscope}%
\begin{pgfscope}%
\pgfpathrectangle{\pgfqpoint{0.786164in}{0.768110in}}{\pgfqpoint{8.851069in}{7.081890in}}%
\pgfusepath{clip}%
\pgfsetbuttcap%
\pgfsetroundjoin%
\definecolor{currentfill}{rgb}{0.159194,0.482237,0.558073}%
\pgfsetfillcolor{currentfill}%
\pgfsetfillopacity{0.700000}%
\pgfsetlinewidth{0.501875pt}%
\definecolor{currentstroke}{rgb}{1.000000,1.000000,1.000000}%
\pgfsetstrokecolor{currentstroke}%
\pgfsetstrokeopacity{0.700000}%
\pgfsetdash{}{0pt}%
\pgfpathmoveto{\pgfqpoint{1.718216in}{2.245313in}}%
\pgfpathcurveto{\pgfqpoint{1.731239in}{2.245313in}}{\pgfqpoint{1.743730in}{2.250487in}}{\pgfqpoint{1.752938in}{2.259695in}}%
\pgfpathcurveto{\pgfqpoint{1.762147in}{2.268904in}}{\pgfqpoint{1.767321in}{2.281395in}}{\pgfqpoint{1.767321in}{2.294417in}}%
\pgfpathcurveto{\pgfqpoint{1.767321in}{2.307440in}}{\pgfqpoint{1.762147in}{2.319931in}}{\pgfqpoint{1.752938in}{2.329140in}}%
\pgfpathcurveto{\pgfqpoint{1.743730in}{2.338348in}}{\pgfqpoint{1.731239in}{2.343522in}}{\pgfqpoint{1.718216in}{2.343522in}}%
\pgfpathcurveto{\pgfqpoint{1.705193in}{2.343522in}}{\pgfqpoint{1.692702in}{2.338348in}}{\pgfqpoint{1.683494in}{2.329140in}}%
\pgfpathcurveto{\pgfqpoint{1.674285in}{2.319931in}}{\pgfqpoint{1.669111in}{2.307440in}}{\pgfqpoint{1.669111in}{2.294417in}}%
\pgfpathcurveto{\pgfqpoint{1.669111in}{2.281395in}}{\pgfqpoint{1.674285in}{2.268904in}}{\pgfqpoint{1.683494in}{2.259695in}}%
\pgfpathcurveto{\pgfqpoint{1.692702in}{2.250487in}}{\pgfqpoint{1.705193in}{2.245313in}}{\pgfqpoint{1.718216in}{2.245313in}}%
\pgfpathlineto{\pgfqpoint{1.718216in}{2.245313in}}%
\pgfpathclose%
\pgfusepath{stroke,fill}%
\end{pgfscope}%
\begin{pgfscope}%
\pgfpathrectangle{\pgfqpoint{0.786164in}{0.768110in}}{\pgfqpoint{8.851069in}{7.081890in}}%
\pgfusepath{clip}%
\pgfsetbuttcap%
\pgfsetroundjoin%
\definecolor{currentfill}{rgb}{0.154815,0.493313,0.557840}%
\pgfsetfillcolor{currentfill}%
\pgfsetfillopacity{0.700000}%
\pgfsetlinewidth{0.501875pt}%
\definecolor{currentstroke}{rgb}{1.000000,1.000000,1.000000}%
\pgfsetstrokecolor{currentstroke}%
\pgfsetstrokeopacity{0.700000}%
\pgfsetdash{}{0pt}%
\pgfpathmoveto{\pgfqpoint{1.699949in}{2.157720in}}%
\pgfpathcurveto{\pgfqpoint{1.712972in}{2.157720in}}{\pgfqpoint{1.725463in}{2.162894in}}{\pgfqpoint{1.734672in}{2.172102in}}%
\pgfpathcurveto{\pgfqpoint{1.743880in}{2.181311in}}{\pgfqpoint{1.749054in}{2.193802in}}{\pgfqpoint{1.749054in}{2.206825in}}%
\pgfpathcurveto{\pgfqpoint{1.749054in}{2.219847in}}{\pgfqpoint{1.743880in}{2.232338in}}{\pgfqpoint{1.734672in}{2.241547in}}%
\pgfpathcurveto{\pgfqpoint{1.725463in}{2.250755in}}{\pgfqpoint{1.712972in}{2.255929in}}{\pgfqpoint{1.699949in}{2.255929in}}%
\pgfpathcurveto{\pgfqpoint{1.686927in}{2.255929in}}{\pgfqpoint{1.674436in}{2.250755in}}{\pgfqpoint{1.665227in}{2.241547in}}%
\pgfpathcurveto{\pgfqpoint{1.656019in}{2.232338in}}{\pgfqpoint{1.650845in}{2.219847in}}{\pgfqpoint{1.650845in}{2.206825in}}%
\pgfpathcurveto{\pgfqpoint{1.650845in}{2.193802in}}{\pgfqpoint{1.656019in}{2.181311in}}{\pgfqpoint{1.665227in}{2.172102in}}%
\pgfpathcurveto{\pgfqpoint{1.674436in}{2.162894in}}{\pgfqpoint{1.686927in}{2.157720in}}{\pgfqpoint{1.699949in}{2.157720in}}%
\pgfpathlineto{\pgfqpoint{1.699949in}{2.157720in}}%
\pgfpathclose%
\pgfusepath{stroke,fill}%
\end{pgfscope}%
\begin{pgfscope}%
\pgfpathrectangle{\pgfqpoint{0.786164in}{0.768110in}}{\pgfqpoint{8.851069in}{7.081890in}}%
\pgfusepath{clip}%
\pgfsetbuttcap%
\pgfsetroundjoin%
\definecolor{currentfill}{rgb}{0.151918,0.500685,0.557587}%
\pgfsetfillcolor{currentfill}%
\pgfsetfillopacity{0.700000}%
\pgfsetlinewidth{0.501875pt}%
\definecolor{currentstroke}{rgb}{1.000000,1.000000,1.000000}%
\pgfsetstrokecolor{currentstroke}%
\pgfsetstrokeopacity{0.700000}%
\pgfsetdash{}{0pt}%
\pgfpathmoveto{\pgfqpoint{1.718216in}{2.157720in}}%
\pgfpathcurveto{\pgfqpoint{1.731239in}{2.157720in}}{\pgfqpoint{1.743730in}{2.162894in}}{\pgfqpoint{1.752938in}{2.172102in}}%
\pgfpathcurveto{\pgfqpoint{1.762147in}{2.181311in}}{\pgfqpoint{1.767321in}{2.193802in}}{\pgfqpoint{1.767321in}{2.206825in}}%
\pgfpathcurveto{\pgfqpoint{1.767321in}{2.219847in}}{\pgfqpoint{1.762147in}{2.232338in}}{\pgfqpoint{1.752938in}{2.241547in}}%
\pgfpathcurveto{\pgfqpoint{1.743730in}{2.250755in}}{\pgfqpoint{1.731239in}{2.255929in}}{\pgfqpoint{1.718216in}{2.255929in}}%
\pgfpathcurveto{\pgfqpoint{1.705193in}{2.255929in}}{\pgfqpoint{1.692702in}{2.250755in}}{\pgfqpoint{1.683494in}{2.241547in}}%
\pgfpathcurveto{\pgfqpoint{1.674285in}{2.232338in}}{\pgfqpoint{1.669111in}{2.219847in}}{\pgfqpoint{1.669111in}{2.206825in}}%
\pgfpathcurveto{\pgfqpoint{1.669111in}{2.193802in}}{\pgfqpoint{1.674285in}{2.181311in}}{\pgfqpoint{1.683494in}{2.172102in}}%
\pgfpathcurveto{\pgfqpoint{1.692702in}{2.162894in}}{\pgfqpoint{1.705193in}{2.157720in}}{\pgfqpoint{1.718216in}{2.157720in}}%
\pgfpathlineto{\pgfqpoint{1.718216in}{2.157720in}}%
\pgfpathclose%
\pgfusepath{stroke,fill}%
\end{pgfscope}%
\begin{pgfscope}%
\pgfpathrectangle{\pgfqpoint{0.786164in}{0.768110in}}{\pgfqpoint{8.851069in}{7.081890in}}%
\pgfusepath{clip}%
\pgfsetbuttcap%
\pgfsetroundjoin%
\definecolor{currentfill}{rgb}{0.137770,0.537492,0.554906}%
\pgfsetfillcolor{currentfill}%
\pgfsetfillopacity{0.700000}%
\pgfsetlinewidth{0.501875pt}%
\definecolor{currentstroke}{rgb}{1.000000,1.000000,1.000000}%
\pgfsetstrokecolor{currentstroke}%
\pgfsetstrokeopacity{0.700000}%
\pgfsetdash{}{0pt}%
\pgfpathmoveto{\pgfqpoint{1.626883in}{2.092025in}}%
\pgfpathcurveto{\pgfqpoint{1.639906in}{2.092025in}}{\pgfqpoint{1.652397in}{2.097199in}}{\pgfqpoint{1.661605in}{2.106408in}}%
\pgfpathcurveto{\pgfqpoint{1.670814in}{2.115616in}}{\pgfqpoint{1.675988in}{2.128107in}}{\pgfqpoint{1.675988in}{2.141130in}}%
\pgfpathcurveto{\pgfqpoint{1.675988in}{2.154153in}}{\pgfqpoint{1.670814in}{2.166644in}}{\pgfqpoint{1.661605in}{2.175852in}}%
\pgfpathcurveto{\pgfqpoint{1.652397in}{2.185060in}}{\pgfqpoint{1.639906in}{2.190234in}}{\pgfqpoint{1.626883in}{2.190234in}}%
\pgfpathcurveto{\pgfqpoint{1.613860in}{2.190234in}}{\pgfqpoint{1.601369in}{2.185060in}}{\pgfqpoint{1.592161in}{2.175852in}}%
\pgfpathcurveto{\pgfqpoint{1.582952in}{2.166644in}}{\pgfqpoint{1.577778in}{2.154153in}}{\pgfqpoint{1.577778in}{2.141130in}}%
\pgfpathcurveto{\pgfqpoint{1.577778in}{2.128107in}}{\pgfqpoint{1.582952in}{2.115616in}}{\pgfqpoint{1.592161in}{2.106408in}}%
\pgfpathcurveto{\pgfqpoint{1.601369in}{2.097199in}}{\pgfqpoint{1.613860in}{2.092025in}}{\pgfqpoint{1.626883in}{2.092025in}}%
\pgfpathlineto{\pgfqpoint{1.626883in}{2.092025in}}%
\pgfpathclose%
\pgfusepath{stroke,fill}%
\end{pgfscope}%
\begin{pgfscope}%
\pgfpathrectangle{\pgfqpoint{0.786164in}{0.768110in}}{\pgfqpoint{8.851069in}{7.081890in}}%
\pgfusepath{clip}%
\pgfsetbuttcap%
\pgfsetroundjoin%
\definecolor{currentfill}{rgb}{0.136408,0.541173,0.554483}%
\pgfsetfillcolor{currentfill}%
\pgfsetfillopacity{0.700000}%
\pgfsetlinewidth{0.501875pt}%
\definecolor{currentstroke}{rgb}{1.000000,1.000000,1.000000}%
\pgfsetstrokecolor{currentstroke}%
\pgfsetstrokeopacity{0.700000}%
\pgfsetdash{}{0pt}%
\pgfpathmoveto{\pgfqpoint{1.782149in}{2.135822in}}%
\pgfpathcurveto{\pgfqpoint{1.795172in}{2.135822in}}{\pgfqpoint{1.807663in}{2.140996in}}{\pgfqpoint{1.816871in}{2.150204in}}%
\pgfpathcurveto{\pgfqpoint{1.826080in}{2.159413in}}{\pgfqpoint{1.831254in}{2.171904in}}{\pgfqpoint{1.831254in}{2.184926in}}%
\pgfpathcurveto{\pgfqpoint{1.831254in}{2.197949in}}{\pgfqpoint{1.826080in}{2.210440in}}{\pgfqpoint{1.816871in}{2.219649in}}%
\pgfpathcurveto{\pgfqpoint{1.807663in}{2.228857in}}{\pgfqpoint{1.795172in}{2.234031in}}{\pgfqpoint{1.782149in}{2.234031in}}%
\pgfpathcurveto{\pgfqpoint{1.769126in}{2.234031in}}{\pgfqpoint{1.756635in}{2.228857in}}{\pgfqpoint{1.747427in}{2.219649in}}%
\pgfpathcurveto{\pgfqpoint{1.738218in}{2.210440in}}{\pgfqpoint{1.733044in}{2.197949in}}{\pgfqpoint{1.733044in}{2.184926in}}%
\pgfpathcurveto{\pgfqpoint{1.733044in}{2.171904in}}{\pgfqpoint{1.738218in}{2.159413in}}{\pgfqpoint{1.747427in}{2.150204in}}%
\pgfpathcurveto{\pgfqpoint{1.756635in}{2.140996in}}{\pgfqpoint{1.769126in}{2.135822in}}{\pgfqpoint{1.782149in}{2.135822in}}%
\pgfpathlineto{\pgfqpoint{1.782149in}{2.135822in}}%
\pgfpathclose%
\pgfusepath{stroke,fill}%
\end{pgfscope}%
\begin{pgfscope}%
\pgfpathrectangle{\pgfqpoint{0.786164in}{0.768110in}}{\pgfqpoint{8.851069in}{7.081890in}}%
\pgfusepath{clip}%
\pgfsetbuttcap%
\pgfsetroundjoin%
\definecolor{currentfill}{rgb}{0.147607,0.511733,0.557049}%
\pgfsetfillcolor{currentfill}%
\pgfsetfillopacity{0.700000}%
\pgfsetlinewidth{0.501875pt}%
\definecolor{currentstroke}{rgb}{1.000000,1.000000,1.000000}%
\pgfsetstrokecolor{currentstroke}%
\pgfsetstrokeopacity{0.700000}%
\pgfsetdash{}{0pt}%
\pgfpathmoveto{\pgfqpoint{1.836949in}{2.223415in}}%
\pgfpathcurveto{\pgfqpoint{1.849971in}{2.223415in}}{\pgfqpoint{1.862462in}{2.228589in}}{\pgfqpoint{1.871671in}{2.237797in}}%
\pgfpathcurveto{\pgfqpoint{1.880879in}{2.247005in}}{\pgfqpoint{1.886053in}{2.259497in}}{\pgfqpoint{1.886053in}{2.272519in}}%
\pgfpathcurveto{\pgfqpoint{1.886053in}{2.285542in}}{\pgfqpoint{1.880879in}{2.298033in}}{\pgfqpoint{1.871671in}{2.307241in}}%
\pgfpathcurveto{\pgfqpoint{1.862462in}{2.316450in}}{\pgfqpoint{1.849971in}{2.321624in}}{\pgfqpoint{1.836949in}{2.321624in}}%
\pgfpathcurveto{\pgfqpoint{1.823926in}{2.321624in}}{\pgfqpoint{1.811435in}{2.316450in}}{\pgfqpoint{1.802226in}{2.307241in}}%
\pgfpathcurveto{\pgfqpoint{1.793018in}{2.298033in}}{\pgfqpoint{1.787844in}{2.285542in}}{\pgfqpoint{1.787844in}{2.272519in}}%
\pgfpathcurveto{\pgfqpoint{1.787844in}{2.259497in}}{\pgfqpoint{1.793018in}{2.247005in}}{\pgfqpoint{1.802226in}{2.237797in}}%
\pgfpathcurveto{\pgfqpoint{1.811435in}{2.228589in}}{\pgfqpoint{1.823926in}{2.223415in}}{\pgfqpoint{1.836949in}{2.223415in}}%
\pgfpathlineto{\pgfqpoint{1.836949in}{2.223415in}}%
\pgfpathclose%
\pgfusepath{stroke,fill}%
\end{pgfscope}%
\begin{pgfscope}%
\pgfpathrectangle{\pgfqpoint{0.786164in}{0.768110in}}{\pgfqpoint{8.851069in}{7.081890in}}%
\pgfusepath{clip}%
\pgfsetbuttcap%
\pgfsetroundjoin%
\definecolor{currentfill}{rgb}{0.282327,0.094955,0.417331}%
\pgfsetfillcolor{currentfill}%
\pgfsetfillopacity{0.700000}%
\pgfsetlinewidth{0.501875pt}%
\definecolor{currentstroke}{rgb}{1.000000,1.000000,1.000000}%
\pgfsetstrokecolor{currentstroke}%
\pgfsetstrokeopacity{0.700000}%
\pgfsetdash{}{0pt}%
\pgfpathmoveto{\pgfqpoint{2.120081in}{2.639481in}}%
\pgfpathcurveto{\pgfqpoint{2.133103in}{2.639481in}}{\pgfqpoint{2.145594in}{2.644655in}}{\pgfqpoint{2.154803in}{2.653864in}}%
\pgfpathcurveto{\pgfqpoint{2.164011in}{2.663072in}}{\pgfqpoint{2.169185in}{2.675563in}}{\pgfqpoint{2.169185in}{2.688586in}}%
\pgfpathcurveto{\pgfqpoint{2.169185in}{2.701608in}}{\pgfqpoint{2.164011in}{2.714100in}}{\pgfqpoint{2.154803in}{2.723308in}}%
\pgfpathcurveto{\pgfqpoint{2.145594in}{2.732516in}}{\pgfqpoint{2.133103in}{2.737690in}}{\pgfqpoint{2.120081in}{2.737690in}}%
\pgfpathcurveto{\pgfqpoint{2.107058in}{2.737690in}}{\pgfqpoint{2.094567in}{2.732516in}}{\pgfqpoint{2.085358in}{2.723308in}}%
\pgfpathcurveto{\pgfqpoint{2.076150in}{2.714100in}}{\pgfqpoint{2.070976in}{2.701608in}}{\pgfqpoint{2.070976in}{2.688586in}}%
\pgfpathcurveto{\pgfqpoint{2.070976in}{2.675563in}}{\pgfqpoint{2.076150in}{2.663072in}}{\pgfqpoint{2.085358in}{2.653864in}}%
\pgfpathcurveto{\pgfqpoint{2.094567in}{2.644655in}}{\pgfqpoint{2.107058in}{2.639481in}}{\pgfqpoint{2.120081in}{2.639481in}}%
\pgfpathlineto{\pgfqpoint{2.120081in}{2.639481in}}%
\pgfpathclose%
\pgfusepath{stroke,fill}%
\end{pgfscope}%
\begin{pgfscope}%
\pgfpathrectangle{\pgfqpoint{0.786164in}{0.768110in}}{\pgfqpoint{8.851069in}{7.081890in}}%
\pgfusepath{clip}%
\pgfsetbuttcap%
\pgfsetroundjoin%
\definecolor{currentfill}{rgb}{0.282290,0.145912,0.461510}%
\pgfsetfillcolor{currentfill}%
\pgfsetfillopacity{0.700000}%
\pgfsetlinewidth{0.501875pt}%
\definecolor{currentstroke}{rgb}{1.000000,1.000000,1.000000}%
\pgfsetstrokecolor{currentstroke}%
\pgfsetstrokeopacity{0.700000}%
\pgfsetdash{}{0pt}%
\pgfpathmoveto{\pgfqpoint{2.257080in}{2.573786in}}%
\pgfpathcurveto{\pgfqpoint{2.270103in}{2.573786in}}{\pgfqpoint{2.282594in}{2.578960in}}{\pgfqpoint{2.291802in}{2.588169in}}%
\pgfpathcurveto{\pgfqpoint{2.301010in}{2.597377in}}{\pgfqpoint{2.306184in}{2.609868in}}{\pgfqpoint{2.306184in}{2.622891in}}%
\pgfpathcurveto{\pgfqpoint{2.306184in}{2.635914in}}{\pgfqpoint{2.301010in}{2.648405in}}{\pgfqpoint{2.291802in}{2.657613in}}%
\pgfpathcurveto{\pgfqpoint{2.282594in}{2.666822in}}{\pgfqpoint{2.270103in}{2.671996in}}{\pgfqpoint{2.257080in}{2.671996in}}%
\pgfpathcurveto{\pgfqpoint{2.244057in}{2.671996in}}{\pgfqpoint{2.231566in}{2.666822in}}{\pgfqpoint{2.222358in}{2.657613in}}%
\pgfpathcurveto{\pgfqpoint{2.213149in}{2.648405in}}{\pgfqpoint{2.207975in}{2.635914in}}{\pgfqpoint{2.207975in}{2.622891in}}%
\pgfpathcurveto{\pgfqpoint{2.207975in}{2.609868in}}{\pgfqpoint{2.213149in}{2.597377in}}{\pgfqpoint{2.222358in}{2.588169in}}%
\pgfpathcurveto{\pgfqpoint{2.231566in}{2.578960in}}{\pgfqpoint{2.244057in}{2.573786in}}{\pgfqpoint{2.257080in}{2.573786in}}%
\pgfpathlineto{\pgfqpoint{2.257080in}{2.573786in}}%
\pgfpathclose%
\pgfusepath{stroke,fill}%
\end{pgfscope}%
\begin{pgfscope}%
\pgfpathrectangle{\pgfqpoint{0.786164in}{0.768110in}}{\pgfqpoint{8.851069in}{7.081890in}}%
\pgfusepath{clip}%
\pgfsetbuttcap%
\pgfsetroundjoin%
\definecolor{currentfill}{rgb}{0.281412,0.155834,0.469201}%
\pgfsetfillcolor{currentfill}%
\pgfsetfillopacity{0.700000}%
\pgfsetlinewidth{0.501875pt}%
\definecolor{currentstroke}{rgb}{1.000000,1.000000,1.000000}%
\pgfsetstrokecolor{currentstroke}%
\pgfsetstrokeopacity{0.700000}%
\pgfsetdash{}{0pt}%
\pgfpathmoveto{\pgfqpoint{2.229680in}{2.595685in}}%
\pgfpathcurveto{\pgfqpoint{2.242703in}{2.595685in}}{\pgfqpoint{2.255194in}{2.600859in}}{\pgfqpoint{2.264402in}{2.610067in}}%
\pgfpathcurveto{\pgfqpoint{2.273611in}{2.619275in}}{\pgfqpoint{2.278785in}{2.631767in}}{\pgfqpoint{2.278785in}{2.644789in}}%
\pgfpathcurveto{\pgfqpoint{2.278785in}{2.657812in}}{\pgfqpoint{2.273611in}{2.670303in}}{\pgfqpoint{2.264402in}{2.679511in}}%
\pgfpathcurveto{\pgfqpoint{2.255194in}{2.688720in}}{\pgfqpoint{2.242703in}{2.693894in}}{\pgfqpoint{2.229680in}{2.693894in}}%
\pgfpathcurveto{\pgfqpoint{2.216657in}{2.693894in}}{\pgfqpoint{2.204166in}{2.688720in}}{\pgfqpoint{2.194958in}{2.679511in}}%
\pgfpathcurveto{\pgfqpoint{2.185749in}{2.670303in}}{\pgfqpoint{2.180575in}{2.657812in}}{\pgfqpoint{2.180575in}{2.644789in}}%
\pgfpathcurveto{\pgfqpoint{2.180575in}{2.631767in}}{\pgfqpoint{2.185749in}{2.619275in}}{\pgfqpoint{2.194958in}{2.610067in}}%
\pgfpathcurveto{\pgfqpoint{2.204166in}{2.600859in}}{\pgfqpoint{2.216657in}{2.595685in}}{\pgfqpoint{2.229680in}{2.595685in}}%
\pgfpathlineto{\pgfqpoint{2.229680in}{2.595685in}}%
\pgfpathclose%
\pgfusepath{stroke,fill}%
\end{pgfscope}%
\begin{pgfscope}%
\pgfpathrectangle{\pgfqpoint{0.786164in}{0.768110in}}{\pgfqpoint{8.851069in}{7.081890in}}%
\pgfusepath{clip}%
\pgfsetbuttcap%
\pgfsetroundjoin%
\definecolor{currentfill}{rgb}{0.278826,0.175490,0.483397}%
\pgfsetfillcolor{currentfill}%
\pgfsetfillopacity{0.700000}%
\pgfsetlinewidth{0.501875pt}%
\definecolor{currentstroke}{rgb}{1.000000,1.000000,1.000000}%
\pgfsetstrokecolor{currentstroke}%
\pgfsetstrokeopacity{0.700000}%
\pgfsetdash{}{0pt}%
\pgfpathmoveto{\pgfqpoint{2.174880in}{2.551888in}}%
\pgfpathcurveto{\pgfqpoint{2.187903in}{2.551888in}}{\pgfqpoint{2.200394in}{2.557062in}}{\pgfqpoint{2.209602in}{2.566271in}}%
\pgfpathcurveto{\pgfqpoint{2.218811in}{2.575479in}}{\pgfqpoint{2.223985in}{2.587970in}}{\pgfqpoint{2.223985in}{2.600993in}}%
\pgfpathcurveto{\pgfqpoint{2.223985in}{2.614015in}}{\pgfqpoint{2.218811in}{2.626507in}}{\pgfqpoint{2.209602in}{2.635715in}}%
\pgfpathcurveto{\pgfqpoint{2.200394in}{2.644923in}}{\pgfqpoint{2.187903in}{2.650097in}}{\pgfqpoint{2.174880in}{2.650097in}}%
\pgfpathcurveto{\pgfqpoint{2.161858in}{2.650097in}}{\pgfqpoint{2.149366in}{2.644923in}}{\pgfqpoint{2.140158in}{2.635715in}}%
\pgfpathcurveto{\pgfqpoint{2.130950in}{2.626507in}}{\pgfqpoint{2.125776in}{2.614015in}}{\pgfqpoint{2.125776in}{2.600993in}}%
\pgfpathcurveto{\pgfqpoint{2.125776in}{2.587970in}}{\pgfqpoint{2.130950in}{2.575479in}}{\pgfqpoint{2.140158in}{2.566271in}}%
\pgfpathcurveto{\pgfqpoint{2.149366in}{2.557062in}}{\pgfqpoint{2.161858in}{2.551888in}}{\pgfqpoint{2.174880in}{2.551888in}}%
\pgfpathlineto{\pgfqpoint{2.174880in}{2.551888in}}%
\pgfpathclose%
\pgfusepath{stroke,fill}%
\end{pgfscope}%
\begin{pgfscope}%
\pgfpathrectangle{\pgfqpoint{0.786164in}{0.768110in}}{\pgfqpoint{8.851069in}{7.081890in}}%
\pgfusepath{clip}%
\pgfsetbuttcap%
\pgfsetroundjoin%
\definecolor{currentfill}{rgb}{0.278826,0.175490,0.483397}%
\pgfsetfillcolor{currentfill}%
\pgfsetfillopacity{0.700000}%
\pgfsetlinewidth{0.501875pt}%
\definecolor{currentstroke}{rgb}{1.000000,1.000000,1.000000}%
\pgfsetstrokecolor{currentstroke}%
\pgfsetstrokeopacity{0.700000}%
\pgfsetdash{}{0pt}%
\pgfpathmoveto{\pgfqpoint{2.165747in}{2.464295in}}%
\pgfpathcurveto{\pgfqpoint{2.178770in}{2.464295in}}{\pgfqpoint{2.191261in}{2.469469in}}{\pgfqpoint{2.200469in}{2.478678in}}%
\pgfpathcurveto{\pgfqpoint{2.209678in}{2.487886in}}{\pgfqpoint{2.214852in}{2.500377in}}{\pgfqpoint{2.214852in}{2.513400in}}%
\pgfpathcurveto{\pgfqpoint{2.214852in}{2.526423in}}{\pgfqpoint{2.209678in}{2.538914in}}{\pgfqpoint{2.200469in}{2.548122in}}%
\pgfpathcurveto{\pgfqpoint{2.191261in}{2.557331in}}{\pgfqpoint{2.178770in}{2.562504in}}{\pgfqpoint{2.165747in}{2.562504in}}%
\pgfpathcurveto{\pgfqpoint{2.152724in}{2.562504in}}{\pgfqpoint{2.140233in}{2.557331in}}{\pgfqpoint{2.131025in}{2.548122in}}%
\pgfpathcurveto{\pgfqpoint{2.121816in}{2.538914in}}{\pgfqpoint{2.116642in}{2.526423in}}{\pgfqpoint{2.116642in}{2.513400in}}%
\pgfpathcurveto{\pgfqpoint{2.116642in}{2.500377in}}{\pgfqpoint{2.121816in}{2.487886in}}{\pgfqpoint{2.131025in}{2.478678in}}%
\pgfpathcurveto{\pgfqpoint{2.140233in}{2.469469in}}{\pgfqpoint{2.152724in}{2.464295in}}{\pgfqpoint{2.165747in}{2.464295in}}%
\pgfpathlineto{\pgfqpoint{2.165747in}{2.464295in}}%
\pgfpathclose%
\pgfusepath{stroke,fill}%
\end{pgfscope}%
\begin{pgfscope}%
\pgfpathrectangle{\pgfqpoint{0.786164in}{0.768110in}}{\pgfqpoint{8.851069in}{7.081890in}}%
\pgfusepath{clip}%
\pgfsetbuttcap%
\pgfsetroundjoin%
\definecolor{currentfill}{rgb}{0.265145,0.232956,0.516599}%
\pgfsetfillcolor{currentfill}%
\pgfsetfillopacity{0.700000}%
\pgfsetlinewidth{0.501875pt}%
\definecolor{currentstroke}{rgb}{1.000000,1.000000,1.000000}%
\pgfsetstrokecolor{currentstroke}%
\pgfsetstrokeopacity{0.700000}%
\pgfsetdash{}{0pt}%
\pgfpathmoveto{\pgfqpoint{2.056148in}{2.376702in}}%
\pgfpathcurveto{\pgfqpoint{2.069170in}{2.376702in}}{\pgfqpoint{2.081661in}{2.381876in}}{\pgfqpoint{2.090870in}{2.391085in}}%
\pgfpathcurveto{\pgfqpoint{2.100078in}{2.400293in}}{\pgfqpoint{2.105252in}{2.412784in}}{\pgfqpoint{2.105252in}{2.425807in}}%
\pgfpathcurveto{\pgfqpoint{2.105252in}{2.438830in}}{\pgfqpoint{2.100078in}{2.451321in}}{\pgfqpoint{2.090870in}{2.460529in}}%
\pgfpathcurveto{\pgfqpoint{2.081661in}{2.469738in}}{\pgfqpoint{2.069170in}{2.474912in}}{\pgfqpoint{2.056148in}{2.474912in}}%
\pgfpathcurveto{\pgfqpoint{2.043125in}{2.474912in}}{\pgfqpoint{2.030634in}{2.469738in}}{\pgfqpoint{2.021425in}{2.460529in}}%
\pgfpathcurveto{\pgfqpoint{2.012217in}{2.451321in}}{\pgfqpoint{2.007043in}{2.438830in}}{\pgfqpoint{2.007043in}{2.425807in}}%
\pgfpathcurveto{\pgfqpoint{2.007043in}{2.412784in}}{\pgfqpoint{2.012217in}{2.400293in}}{\pgfqpoint{2.021425in}{2.391085in}}%
\pgfpathcurveto{\pgfqpoint{2.030634in}{2.381876in}}{\pgfqpoint{2.043125in}{2.376702in}}{\pgfqpoint{2.056148in}{2.376702in}}%
\pgfpathlineto{\pgfqpoint{2.056148in}{2.376702in}}%
\pgfpathclose%
\pgfusepath{stroke,fill}%
\end{pgfscope}%
\begin{pgfscope}%
\pgfpathrectangle{\pgfqpoint{0.786164in}{0.768110in}}{\pgfqpoint{8.851069in}{7.081890in}}%
\pgfusepath{clip}%
\pgfsetbuttcap%
\pgfsetroundjoin%
\definecolor{currentfill}{rgb}{0.257322,0.256130,0.526563}%
\pgfsetfillcolor{currentfill}%
\pgfsetfillopacity{0.700000}%
\pgfsetlinewidth{0.501875pt}%
\definecolor{currentstroke}{rgb}{1.000000,1.000000,1.000000}%
\pgfsetstrokecolor{currentstroke}%
\pgfsetstrokeopacity{0.700000}%
\pgfsetdash{}{0pt}%
\pgfpathmoveto{\pgfqpoint{2.120081in}{2.376702in}}%
\pgfpathcurveto{\pgfqpoint{2.133103in}{2.376702in}}{\pgfqpoint{2.145594in}{2.381876in}}{\pgfqpoint{2.154803in}{2.391085in}}%
\pgfpathcurveto{\pgfqpoint{2.164011in}{2.400293in}}{\pgfqpoint{2.169185in}{2.412784in}}{\pgfqpoint{2.169185in}{2.425807in}}%
\pgfpathcurveto{\pgfqpoint{2.169185in}{2.438830in}}{\pgfqpoint{2.164011in}{2.451321in}}{\pgfqpoint{2.154803in}{2.460529in}}%
\pgfpathcurveto{\pgfqpoint{2.145594in}{2.469738in}}{\pgfqpoint{2.133103in}{2.474912in}}{\pgfqpoint{2.120081in}{2.474912in}}%
\pgfpathcurveto{\pgfqpoint{2.107058in}{2.474912in}}{\pgfqpoint{2.094567in}{2.469738in}}{\pgfqpoint{2.085358in}{2.460529in}}%
\pgfpathcurveto{\pgfqpoint{2.076150in}{2.451321in}}{\pgfqpoint{2.070976in}{2.438830in}}{\pgfqpoint{2.070976in}{2.425807in}}%
\pgfpathcurveto{\pgfqpoint{2.070976in}{2.412784in}}{\pgfqpoint{2.076150in}{2.400293in}}{\pgfqpoint{2.085358in}{2.391085in}}%
\pgfpathcurveto{\pgfqpoint{2.094567in}{2.381876in}}{\pgfqpoint{2.107058in}{2.376702in}}{\pgfqpoint{2.120081in}{2.376702in}}%
\pgfpathlineto{\pgfqpoint{2.120081in}{2.376702in}}%
\pgfpathclose%
\pgfusepath{stroke,fill}%
\end{pgfscope}%
\begin{pgfscope}%
\pgfpathrectangle{\pgfqpoint{0.786164in}{0.768110in}}{\pgfqpoint{8.851069in}{7.081890in}}%
\pgfusepath{clip}%
\pgfsetbuttcap%
\pgfsetroundjoin%
\definecolor{currentfill}{rgb}{0.250425,0.274290,0.533103}%
\pgfsetfillcolor{currentfill}%
\pgfsetfillopacity{0.700000}%
\pgfsetlinewidth{0.501875pt}%
\definecolor{currentstroke}{rgb}{1.000000,1.000000,1.000000}%
\pgfsetstrokecolor{currentstroke}%
\pgfsetstrokeopacity{0.700000}%
\pgfsetdash{}{0pt}%
\pgfpathmoveto{\pgfqpoint{2.110947in}{2.376702in}}%
\pgfpathcurveto{\pgfqpoint{2.123970in}{2.376702in}}{\pgfqpoint{2.136461in}{2.381876in}}{\pgfqpoint{2.145669in}{2.391085in}}%
\pgfpathcurveto{\pgfqpoint{2.154878in}{2.400293in}}{\pgfqpoint{2.160052in}{2.412784in}}{\pgfqpoint{2.160052in}{2.425807in}}%
\pgfpathcurveto{\pgfqpoint{2.160052in}{2.438830in}}{\pgfqpoint{2.154878in}{2.451321in}}{\pgfqpoint{2.145669in}{2.460529in}}%
\pgfpathcurveto{\pgfqpoint{2.136461in}{2.469738in}}{\pgfqpoint{2.123970in}{2.474912in}}{\pgfqpoint{2.110947in}{2.474912in}}%
\pgfpathcurveto{\pgfqpoint{2.097925in}{2.474912in}}{\pgfqpoint{2.085433in}{2.469738in}}{\pgfqpoint{2.076225in}{2.460529in}}%
\pgfpathcurveto{\pgfqpoint{2.067017in}{2.451321in}}{\pgfqpoint{2.061843in}{2.438830in}}{\pgfqpoint{2.061843in}{2.425807in}}%
\pgfpathcurveto{\pgfqpoint{2.061843in}{2.412784in}}{\pgfqpoint{2.067017in}{2.400293in}}{\pgfqpoint{2.076225in}{2.391085in}}%
\pgfpathcurveto{\pgfqpoint{2.085433in}{2.381876in}}{\pgfqpoint{2.097925in}{2.376702in}}{\pgfqpoint{2.110947in}{2.376702in}}%
\pgfpathlineto{\pgfqpoint{2.110947in}{2.376702in}}%
\pgfpathclose%
\pgfusepath{stroke,fill}%
\end{pgfscope}%
\begin{pgfscope}%
\pgfpathrectangle{\pgfqpoint{0.786164in}{0.768110in}}{\pgfqpoint{8.851069in}{7.081890in}}%
\pgfusepath{clip}%
\pgfsetbuttcap%
\pgfsetroundjoin%
\definecolor{currentfill}{rgb}{0.239346,0.300855,0.540844}%
\pgfsetfillcolor{currentfill}%
\pgfsetfillopacity{0.700000}%
\pgfsetlinewidth{0.501875pt}%
\definecolor{currentstroke}{rgb}{1.000000,1.000000,1.000000}%
\pgfsetstrokecolor{currentstroke}%
\pgfsetstrokeopacity{0.700000}%
\pgfsetdash{}{0pt}%
\pgfpathmoveto{\pgfqpoint{2.001348in}{2.289109in}}%
\pgfpathcurveto{\pgfqpoint{2.014370in}{2.289109in}}{\pgfqpoint{2.026862in}{2.294283in}}{\pgfqpoint{2.036070in}{2.303492in}}%
\pgfpathcurveto{\pgfqpoint{2.045278in}{2.312700in}}{\pgfqpoint{2.050452in}{2.325191in}}{\pgfqpoint{2.050452in}{2.338214in}}%
\pgfpathcurveto{\pgfqpoint{2.050452in}{2.351237in}}{\pgfqpoint{2.045278in}{2.363728in}}{\pgfqpoint{2.036070in}{2.372936in}}%
\pgfpathcurveto{\pgfqpoint{2.026862in}{2.382145in}}{\pgfqpoint{2.014370in}{2.387319in}}{\pgfqpoint{2.001348in}{2.387319in}}%
\pgfpathcurveto{\pgfqpoint{1.988325in}{2.387319in}}{\pgfqpoint{1.975834in}{2.382145in}}{\pgfqpoint{1.966626in}{2.372936in}}%
\pgfpathcurveto{\pgfqpoint{1.957417in}{2.363728in}}{\pgfqpoint{1.952243in}{2.351237in}}{\pgfqpoint{1.952243in}{2.338214in}}%
\pgfpathcurveto{\pgfqpoint{1.952243in}{2.325191in}}{\pgfqpoint{1.957417in}{2.312700in}}{\pgfqpoint{1.966626in}{2.303492in}}%
\pgfpathcurveto{\pgfqpoint{1.975834in}{2.294283in}}{\pgfqpoint{1.988325in}{2.289109in}}{\pgfqpoint{2.001348in}{2.289109in}}%
\pgfpathlineto{\pgfqpoint{2.001348in}{2.289109in}}%
\pgfpathclose%
\pgfusepath{stroke,fill}%
\end{pgfscope}%
\begin{pgfscope}%
\pgfpathrectangle{\pgfqpoint{0.786164in}{0.768110in}}{\pgfqpoint{8.851069in}{7.081890in}}%
\pgfusepath{clip}%
\pgfsetbuttcap%
\pgfsetroundjoin%
\definecolor{currentfill}{rgb}{0.233603,0.313828,0.543914}%
\pgfsetfillcolor{currentfill}%
\pgfsetfillopacity{0.700000}%
\pgfsetlinewidth{0.501875pt}%
\definecolor{currentstroke}{rgb}{1.000000,1.000000,1.000000}%
\pgfsetstrokecolor{currentstroke}%
\pgfsetstrokeopacity{0.700000}%
\pgfsetdash{}{0pt}%
\pgfpathmoveto{\pgfqpoint{1.946548in}{2.245313in}}%
\pgfpathcurveto{\pgfqpoint{1.959571in}{2.245313in}}{\pgfqpoint{1.972062in}{2.250487in}}{\pgfqpoint{1.981270in}{2.259695in}}%
\pgfpathcurveto{\pgfqpoint{1.990479in}{2.268904in}}{\pgfqpoint{1.995653in}{2.281395in}}{\pgfqpoint{1.995653in}{2.294417in}}%
\pgfpathcurveto{\pgfqpoint{1.995653in}{2.307440in}}{\pgfqpoint{1.990479in}{2.319931in}}{\pgfqpoint{1.981270in}{2.329140in}}%
\pgfpathcurveto{\pgfqpoint{1.972062in}{2.338348in}}{\pgfqpoint{1.959571in}{2.343522in}}{\pgfqpoint{1.946548in}{2.343522in}}%
\pgfpathcurveto{\pgfqpoint{1.933525in}{2.343522in}}{\pgfqpoint{1.921034in}{2.338348in}}{\pgfqpoint{1.911826in}{2.329140in}}%
\pgfpathcurveto{\pgfqpoint{1.902617in}{2.319931in}}{\pgfqpoint{1.897443in}{2.307440in}}{\pgfqpoint{1.897443in}{2.294417in}}%
\pgfpathcurveto{\pgfqpoint{1.897443in}{2.281395in}}{\pgfqpoint{1.902617in}{2.268904in}}{\pgfqpoint{1.911826in}{2.259695in}}%
\pgfpathcurveto{\pgfqpoint{1.921034in}{2.250487in}}{\pgfqpoint{1.933525in}{2.245313in}}{\pgfqpoint{1.946548in}{2.245313in}}%
\pgfpathlineto{\pgfqpoint{1.946548in}{2.245313in}}%
\pgfpathclose%
\pgfusepath{stroke,fill}%
\end{pgfscope}%
\begin{pgfscope}%
\pgfpathrectangle{\pgfqpoint{0.786164in}{0.768110in}}{\pgfqpoint{8.851069in}{7.081890in}}%
\pgfusepath{clip}%
\pgfsetbuttcap%
\pgfsetroundjoin%
\definecolor{currentfill}{rgb}{0.244972,0.287675,0.537260}%
\pgfsetfillcolor{currentfill}%
\pgfsetfillopacity{0.700000}%
\pgfsetlinewidth{0.501875pt}%
\definecolor{currentstroke}{rgb}{1.000000,1.000000,1.000000}%
\pgfsetstrokecolor{currentstroke}%
\pgfsetstrokeopacity{0.700000}%
\pgfsetdash{}{0pt}%
\pgfpathmoveto{\pgfqpoint{1.910015in}{2.223415in}}%
\pgfpathcurveto{\pgfqpoint{1.923038in}{2.223415in}}{\pgfqpoint{1.935529in}{2.228589in}}{\pgfqpoint{1.944737in}{2.237797in}}%
\pgfpathcurveto{\pgfqpoint{1.953946in}{2.247005in}}{\pgfqpoint{1.959120in}{2.259497in}}{\pgfqpoint{1.959120in}{2.272519in}}%
\pgfpathcurveto{\pgfqpoint{1.959120in}{2.285542in}}{\pgfqpoint{1.953946in}{2.298033in}}{\pgfqpoint{1.944737in}{2.307241in}}%
\pgfpathcurveto{\pgfqpoint{1.935529in}{2.316450in}}{\pgfqpoint{1.923038in}{2.321624in}}{\pgfqpoint{1.910015in}{2.321624in}}%
\pgfpathcurveto{\pgfqpoint{1.896992in}{2.321624in}}{\pgfqpoint{1.884501in}{2.316450in}}{\pgfqpoint{1.875293in}{2.307241in}}%
\pgfpathcurveto{\pgfqpoint{1.866084in}{2.298033in}}{\pgfqpoint{1.860910in}{2.285542in}}{\pgfqpoint{1.860910in}{2.272519in}}%
\pgfpathcurveto{\pgfqpoint{1.860910in}{2.259497in}}{\pgfqpoint{1.866084in}{2.247005in}}{\pgfqpoint{1.875293in}{2.237797in}}%
\pgfpathcurveto{\pgfqpoint{1.884501in}{2.228589in}}{\pgfqpoint{1.896992in}{2.223415in}}{\pgfqpoint{1.910015in}{2.223415in}}%
\pgfpathlineto{\pgfqpoint{1.910015in}{2.223415in}}%
\pgfpathclose%
\pgfusepath{stroke,fill}%
\end{pgfscope}%
\begin{pgfscope}%
\pgfpathrectangle{\pgfqpoint{0.786164in}{0.768110in}}{\pgfqpoint{8.851069in}{7.081890in}}%
\pgfusepath{clip}%
\pgfsetbuttcap%
\pgfsetroundjoin%
\definecolor{currentfill}{rgb}{0.246811,0.283237,0.535941}%
\pgfsetfillcolor{currentfill}%
\pgfsetfillopacity{0.700000}%
\pgfsetlinewidth{0.501875pt}%
\definecolor{currentstroke}{rgb}{1.000000,1.000000,1.000000}%
\pgfsetstrokecolor{currentstroke}%
\pgfsetstrokeopacity{0.700000}%
\pgfsetdash{}{0pt}%
\pgfpathmoveto{\pgfqpoint{2.037881in}{2.289109in}}%
\pgfpathcurveto{\pgfqpoint{2.050904in}{2.289109in}}{\pgfqpoint{2.063395in}{2.294283in}}{\pgfqpoint{2.072603in}{2.303492in}}%
\pgfpathcurveto{\pgfqpoint{2.081812in}{2.312700in}}{\pgfqpoint{2.086986in}{2.325191in}}{\pgfqpoint{2.086986in}{2.338214in}}%
\pgfpathcurveto{\pgfqpoint{2.086986in}{2.351237in}}{\pgfqpoint{2.081812in}{2.363728in}}{\pgfqpoint{2.072603in}{2.372936in}}%
\pgfpathcurveto{\pgfqpoint{2.063395in}{2.382145in}}{\pgfqpoint{2.050904in}{2.387319in}}{\pgfqpoint{2.037881in}{2.387319in}}%
\pgfpathcurveto{\pgfqpoint{2.024858in}{2.387319in}}{\pgfqpoint{2.012367in}{2.382145in}}{\pgfqpoint{2.003159in}{2.372936in}}%
\pgfpathcurveto{\pgfqpoint{1.993950in}{2.363728in}}{\pgfqpoint{1.988776in}{2.351237in}}{\pgfqpoint{1.988776in}{2.338214in}}%
\pgfpathcurveto{\pgfqpoint{1.988776in}{2.325191in}}{\pgfqpoint{1.993950in}{2.312700in}}{\pgfqpoint{2.003159in}{2.303492in}}%
\pgfpathcurveto{\pgfqpoint{2.012367in}{2.294283in}}{\pgfqpoint{2.024858in}{2.289109in}}{\pgfqpoint{2.037881in}{2.289109in}}%
\pgfpathlineto{\pgfqpoint{2.037881in}{2.289109in}}%
\pgfpathclose%
\pgfusepath{stroke,fill}%
\end{pgfscope}%
\begin{pgfscope}%
\pgfpathrectangle{\pgfqpoint{0.786164in}{0.768110in}}{\pgfqpoint{8.851069in}{7.081890in}}%
\pgfusepath{clip}%
\pgfsetbuttcap%
\pgfsetroundjoin%
\definecolor{currentfill}{rgb}{0.246811,0.283237,0.535941}%
\pgfsetfillcolor{currentfill}%
\pgfsetfillopacity{0.700000}%
\pgfsetlinewidth{0.501875pt}%
\definecolor{currentstroke}{rgb}{1.000000,1.000000,1.000000}%
\pgfsetstrokecolor{currentstroke}%
\pgfsetstrokeopacity{0.700000}%
\pgfsetdash{}{0pt}%
\pgfpathmoveto{\pgfqpoint{2.074414in}{2.332906in}}%
\pgfpathcurveto{\pgfqpoint{2.087437in}{2.332906in}}{\pgfqpoint{2.099928in}{2.338080in}}{\pgfqpoint{2.109136in}{2.347288in}}%
\pgfpathcurveto{\pgfqpoint{2.118345in}{2.356497in}}{\pgfqpoint{2.123519in}{2.368988in}}{\pgfqpoint{2.123519in}{2.382010in}}%
\pgfpathcurveto{\pgfqpoint{2.123519in}{2.395033in}}{\pgfqpoint{2.118345in}{2.407524in}}{\pgfqpoint{2.109136in}{2.416733in}}%
\pgfpathcurveto{\pgfqpoint{2.099928in}{2.425941in}}{\pgfqpoint{2.087437in}{2.431115in}}{\pgfqpoint{2.074414in}{2.431115in}}%
\pgfpathcurveto{\pgfqpoint{2.061391in}{2.431115in}}{\pgfqpoint{2.048900in}{2.425941in}}{\pgfqpoint{2.039692in}{2.416733in}}%
\pgfpathcurveto{\pgfqpoint{2.030483in}{2.407524in}}{\pgfqpoint{2.025309in}{2.395033in}}{\pgfqpoint{2.025309in}{2.382010in}}%
\pgfpathcurveto{\pgfqpoint{2.025309in}{2.368988in}}{\pgfqpoint{2.030483in}{2.356497in}}{\pgfqpoint{2.039692in}{2.347288in}}%
\pgfpathcurveto{\pgfqpoint{2.048900in}{2.338080in}}{\pgfqpoint{2.061391in}{2.332906in}}{\pgfqpoint{2.074414in}{2.332906in}}%
\pgfpathlineto{\pgfqpoint{2.074414in}{2.332906in}}%
\pgfpathclose%
\pgfusepath{stroke,fill}%
\end{pgfscope}%
\begin{pgfscope}%
\pgfpathrectangle{\pgfqpoint{0.786164in}{0.768110in}}{\pgfqpoint{8.851069in}{7.081890in}}%
\pgfusepath{clip}%
\pgfsetbuttcap%
\pgfsetroundjoin%
\definecolor{currentfill}{rgb}{0.252194,0.269783,0.531579}%
\pgfsetfillcolor{currentfill}%
\pgfsetfillopacity{0.700000}%
\pgfsetlinewidth{0.501875pt}%
\definecolor{currentstroke}{rgb}{1.000000,1.000000,1.000000}%
\pgfsetstrokecolor{currentstroke}%
\pgfsetstrokeopacity{0.700000}%
\pgfsetdash{}{0pt}%
\pgfpathmoveto{\pgfqpoint{2.156614in}{2.376702in}}%
\pgfpathcurveto{\pgfqpoint{2.169636in}{2.376702in}}{\pgfqpoint{2.182127in}{2.381876in}}{\pgfqpoint{2.191336in}{2.391085in}}%
\pgfpathcurveto{\pgfqpoint{2.200544in}{2.400293in}}{\pgfqpoint{2.205718in}{2.412784in}}{\pgfqpoint{2.205718in}{2.425807in}}%
\pgfpathcurveto{\pgfqpoint{2.205718in}{2.438830in}}{\pgfqpoint{2.200544in}{2.451321in}}{\pgfqpoint{2.191336in}{2.460529in}}%
\pgfpathcurveto{\pgfqpoint{2.182127in}{2.469738in}}{\pgfqpoint{2.169636in}{2.474912in}}{\pgfqpoint{2.156614in}{2.474912in}}%
\pgfpathcurveto{\pgfqpoint{2.143591in}{2.474912in}}{\pgfqpoint{2.131100in}{2.469738in}}{\pgfqpoint{2.121891in}{2.460529in}}%
\pgfpathcurveto{\pgfqpoint{2.112683in}{2.451321in}}{\pgfqpoint{2.107509in}{2.438830in}}{\pgfqpoint{2.107509in}{2.425807in}}%
\pgfpathcurveto{\pgfqpoint{2.107509in}{2.412784in}}{\pgfqpoint{2.112683in}{2.400293in}}{\pgfqpoint{2.121891in}{2.391085in}}%
\pgfpathcurveto{\pgfqpoint{2.131100in}{2.381876in}}{\pgfqpoint{2.143591in}{2.376702in}}{\pgfqpoint{2.156614in}{2.376702in}}%
\pgfpathlineto{\pgfqpoint{2.156614in}{2.376702in}}%
\pgfpathclose%
\pgfusepath{stroke,fill}%
\end{pgfscope}%
\begin{pgfscope}%
\pgfpathrectangle{\pgfqpoint{0.786164in}{0.768110in}}{\pgfqpoint{8.851069in}{7.081890in}}%
\pgfusepath{clip}%
\pgfsetbuttcap%
\pgfsetroundjoin%
\definecolor{currentfill}{rgb}{0.258965,0.251537,0.524736}%
\pgfsetfillcolor{currentfill}%
\pgfsetfillopacity{0.700000}%
\pgfsetlinewidth{0.501875pt}%
\definecolor{currentstroke}{rgb}{1.000000,1.000000,1.000000}%
\pgfsetstrokecolor{currentstroke}%
\pgfsetstrokeopacity{0.700000}%
\pgfsetdash{}{0pt}%
\pgfpathmoveto{\pgfqpoint{2.156614in}{2.376702in}}%
\pgfpathcurveto{\pgfqpoint{2.169636in}{2.376702in}}{\pgfqpoint{2.182127in}{2.381876in}}{\pgfqpoint{2.191336in}{2.391085in}}%
\pgfpathcurveto{\pgfqpoint{2.200544in}{2.400293in}}{\pgfqpoint{2.205718in}{2.412784in}}{\pgfqpoint{2.205718in}{2.425807in}}%
\pgfpathcurveto{\pgfqpoint{2.205718in}{2.438830in}}{\pgfqpoint{2.200544in}{2.451321in}}{\pgfqpoint{2.191336in}{2.460529in}}%
\pgfpathcurveto{\pgfqpoint{2.182127in}{2.469738in}}{\pgfqpoint{2.169636in}{2.474912in}}{\pgfqpoint{2.156614in}{2.474912in}}%
\pgfpathcurveto{\pgfqpoint{2.143591in}{2.474912in}}{\pgfqpoint{2.131100in}{2.469738in}}{\pgfqpoint{2.121891in}{2.460529in}}%
\pgfpathcurveto{\pgfqpoint{2.112683in}{2.451321in}}{\pgfqpoint{2.107509in}{2.438830in}}{\pgfqpoint{2.107509in}{2.425807in}}%
\pgfpathcurveto{\pgfqpoint{2.107509in}{2.412784in}}{\pgfqpoint{2.112683in}{2.400293in}}{\pgfqpoint{2.121891in}{2.391085in}}%
\pgfpathcurveto{\pgfqpoint{2.131100in}{2.381876in}}{\pgfqpoint{2.143591in}{2.376702in}}{\pgfqpoint{2.156614in}{2.376702in}}%
\pgfpathlineto{\pgfqpoint{2.156614in}{2.376702in}}%
\pgfpathclose%
\pgfusepath{stroke,fill}%
\end{pgfscope}%
\begin{pgfscope}%
\pgfpathrectangle{\pgfqpoint{0.786164in}{0.768110in}}{\pgfqpoint{8.851069in}{7.081890in}}%
\pgfusepath{clip}%
\pgfsetbuttcap%
\pgfsetroundjoin%
\definecolor{currentfill}{rgb}{0.258965,0.251537,0.524736}%
\pgfsetfillcolor{currentfill}%
\pgfsetfillopacity{0.700000}%
\pgfsetlinewidth{0.501875pt}%
\definecolor{currentstroke}{rgb}{1.000000,1.000000,1.000000}%
\pgfsetstrokecolor{currentstroke}%
\pgfsetstrokeopacity{0.700000}%
\pgfsetdash{}{0pt}%
\pgfpathmoveto{\pgfqpoint{2.165747in}{2.376702in}}%
\pgfpathcurveto{\pgfqpoint{2.178770in}{2.376702in}}{\pgfqpoint{2.191261in}{2.381876in}}{\pgfqpoint{2.200469in}{2.391085in}}%
\pgfpathcurveto{\pgfqpoint{2.209678in}{2.400293in}}{\pgfqpoint{2.214852in}{2.412784in}}{\pgfqpoint{2.214852in}{2.425807in}}%
\pgfpathcurveto{\pgfqpoint{2.214852in}{2.438830in}}{\pgfqpoint{2.209678in}{2.451321in}}{\pgfqpoint{2.200469in}{2.460529in}}%
\pgfpathcurveto{\pgfqpoint{2.191261in}{2.469738in}}{\pgfqpoint{2.178770in}{2.474912in}}{\pgfqpoint{2.165747in}{2.474912in}}%
\pgfpathcurveto{\pgfqpoint{2.152724in}{2.474912in}}{\pgfqpoint{2.140233in}{2.469738in}}{\pgfqpoint{2.131025in}{2.460529in}}%
\pgfpathcurveto{\pgfqpoint{2.121816in}{2.451321in}}{\pgfqpoint{2.116642in}{2.438830in}}{\pgfqpoint{2.116642in}{2.425807in}}%
\pgfpathcurveto{\pgfqpoint{2.116642in}{2.412784in}}{\pgfqpoint{2.121816in}{2.400293in}}{\pgfqpoint{2.131025in}{2.391085in}}%
\pgfpathcurveto{\pgfqpoint{2.140233in}{2.381876in}}{\pgfqpoint{2.152724in}{2.376702in}}{\pgfqpoint{2.165747in}{2.376702in}}%
\pgfpathlineto{\pgfqpoint{2.165747in}{2.376702in}}%
\pgfpathclose%
\pgfusepath{stroke,fill}%
\end{pgfscope}%
\begin{pgfscope}%
\pgfpathrectangle{\pgfqpoint{0.786164in}{0.768110in}}{\pgfqpoint{8.851069in}{7.081890in}}%
\pgfusepath{clip}%
\pgfsetbuttcap%
\pgfsetroundjoin%
\definecolor{currentfill}{rgb}{0.250425,0.274290,0.533103}%
\pgfsetfillcolor{currentfill}%
\pgfsetfillopacity{0.700000}%
\pgfsetlinewidth{0.501875pt}%
\definecolor{currentstroke}{rgb}{1.000000,1.000000,1.000000}%
\pgfsetstrokecolor{currentstroke}%
\pgfsetstrokeopacity{0.700000}%
\pgfsetdash{}{0pt}%
\pgfpathmoveto{\pgfqpoint{2.110947in}{2.267211in}}%
\pgfpathcurveto{\pgfqpoint{2.123970in}{2.267211in}}{\pgfqpoint{2.136461in}{2.272385in}}{\pgfqpoint{2.145669in}{2.281593in}}%
\pgfpathcurveto{\pgfqpoint{2.154878in}{2.290802in}}{\pgfqpoint{2.160052in}{2.303293in}}{\pgfqpoint{2.160052in}{2.316316in}}%
\pgfpathcurveto{\pgfqpoint{2.160052in}{2.329338in}}{\pgfqpoint{2.154878in}{2.341829in}}{\pgfqpoint{2.145669in}{2.351038in}}%
\pgfpathcurveto{\pgfqpoint{2.136461in}{2.360246in}}{\pgfqpoint{2.123970in}{2.365420in}}{\pgfqpoint{2.110947in}{2.365420in}}%
\pgfpathcurveto{\pgfqpoint{2.097925in}{2.365420in}}{\pgfqpoint{2.085433in}{2.360246in}}{\pgfqpoint{2.076225in}{2.351038in}}%
\pgfpathcurveto{\pgfqpoint{2.067017in}{2.341829in}}{\pgfqpoint{2.061843in}{2.329338in}}{\pgfqpoint{2.061843in}{2.316316in}}%
\pgfpathcurveto{\pgfqpoint{2.061843in}{2.303293in}}{\pgfqpoint{2.067017in}{2.290802in}}{\pgfqpoint{2.076225in}{2.281593in}}%
\pgfpathcurveto{\pgfqpoint{2.085433in}{2.272385in}}{\pgfqpoint{2.097925in}{2.267211in}}{\pgfqpoint{2.110947in}{2.267211in}}%
\pgfpathlineto{\pgfqpoint{2.110947in}{2.267211in}}%
\pgfpathclose%
\pgfusepath{stroke,fill}%
\end{pgfscope}%
\begin{pgfscope}%
\pgfpathrectangle{\pgfqpoint{0.786164in}{0.768110in}}{\pgfqpoint{8.851069in}{7.081890in}}%
\pgfusepath{clip}%
\pgfsetbuttcap%
\pgfsetroundjoin%
\definecolor{currentfill}{rgb}{0.248629,0.278775,0.534556}%
\pgfsetfillcolor{currentfill}%
\pgfsetfillopacity{0.700000}%
\pgfsetlinewidth{0.501875pt}%
\definecolor{currentstroke}{rgb}{1.000000,1.000000,1.000000}%
\pgfsetstrokecolor{currentstroke}%
\pgfsetstrokeopacity{0.700000}%
\pgfsetdash{}{0pt}%
\pgfpathmoveto{\pgfqpoint{2.220547in}{2.311008in}}%
\pgfpathcurveto{\pgfqpoint{2.233569in}{2.311008in}}{\pgfqpoint{2.246060in}{2.316182in}}{\pgfqpoint{2.255269in}{2.325390in}}%
\pgfpathcurveto{\pgfqpoint{2.264477in}{2.334598in}}{\pgfqpoint{2.269651in}{2.347089in}}{\pgfqpoint{2.269651in}{2.360112in}}%
\pgfpathcurveto{\pgfqpoint{2.269651in}{2.373135in}}{\pgfqpoint{2.264477in}{2.385626in}}{\pgfqpoint{2.255269in}{2.394834in}}%
\pgfpathcurveto{\pgfqpoint{2.246060in}{2.404043in}}{\pgfqpoint{2.233569in}{2.409217in}}{\pgfqpoint{2.220547in}{2.409217in}}%
\pgfpathcurveto{\pgfqpoint{2.207524in}{2.409217in}}{\pgfqpoint{2.195033in}{2.404043in}}{\pgfqpoint{2.185824in}{2.394834in}}%
\pgfpathcurveto{\pgfqpoint{2.176616in}{2.385626in}}{\pgfqpoint{2.171442in}{2.373135in}}{\pgfqpoint{2.171442in}{2.360112in}}%
\pgfpathcurveto{\pgfqpoint{2.171442in}{2.347089in}}{\pgfqpoint{2.176616in}{2.334598in}}{\pgfqpoint{2.185824in}{2.325390in}}%
\pgfpathcurveto{\pgfqpoint{2.195033in}{2.316182in}}{\pgfqpoint{2.207524in}{2.311008in}}{\pgfqpoint{2.220547in}{2.311008in}}%
\pgfpathlineto{\pgfqpoint{2.220547in}{2.311008in}}%
\pgfpathclose%
\pgfusepath{stroke,fill}%
\end{pgfscope}%
\begin{pgfscope}%
\pgfpathrectangle{\pgfqpoint{0.786164in}{0.768110in}}{\pgfqpoint{8.851069in}{7.081890in}}%
\pgfusepath{clip}%
\pgfsetbuttcap%
\pgfsetroundjoin%
\definecolor{currentfill}{rgb}{0.241237,0.296485,0.539709}%
\pgfsetfillcolor{currentfill}%
\pgfsetfillopacity{0.700000}%
\pgfsetlinewidth{0.501875pt}%
\definecolor{currentstroke}{rgb}{1.000000,1.000000,1.000000}%
\pgfsetstrokecolor{currentstroke}%
\pgfsetstrokeopacity{0.700000}%
\pgfsetdash{}{0pt}%
\pgfpathmoveto{\pgfqpoint{2.138347in}{2.245313in}}%
\pgfpathcurveto{\pgfqpoint{2.151370in}{2.245313in}}{\pgfqpoint{2.163861in}{2.250487in}}{\pgfqpoint{2.173069in}{2.259695in}}%
\pgfpathcurveto{\pgfqpoint{2.182278in}{2.268904in}}{\pgfqpoint{2.187452in}{2.281395in}}{\pgfqpoint{2.187452in}{2.294417in}}%
\pgfpathcurveto{\pgfqpoint{2.187452in}{2.307440in}}{\pgfqpoint{2.182278in}{2.319931in}}{\pgfqpoint{2.173069in}{2.329140in}}%
\pgfpathcurveto{\pgfqpoint{2.163861in}{2.338348in}}{\pgfqpoint{2.151370in}{2.343522in}}{\pgfqpoint{2.138347in}{2.343522in}}%
\pgfpathcurveto{\pgfqpoint{2.125324in}{2.343522in}}{\pgfqpoint{2.112833in}{2.338348in}}{\pgfqpoint{2.103625in}{2.329140in}}%
\pgfpathcurveto{\pgfqpoint{2.094416in}{2.319931in}}{\pgfqpoint{2.089242in}{2.307440in}}{\pgfqpoint{2.089242in}{2.294417in}}%
\pgfpathcurveto{\pgfqpoint{2.089242in}{2.281395in}}{\pgfqpoint{2.094416in}{2.268904in}}{\pgfqpoint{2.103625in}{2.259695in}}%
\pgfpathcurveto{\pgfqpoint{2.112833in}{2.250487in}}{\pgfqpoint{2.125324in}{2.245313in}}{\pgfqpoint{2.138347in}{2.245313in}}%
\pgfpathlineto{\pgfqpoint{2.138347in}{2.245313in}}%
\pgfpathclose%
\pgfusepath{stroke,fill}%
\end{pgfscope}%
\begin{pgfscope}%
\pgfpathrectangle{\pgfqpoint{0.786164in}{0.768110in}}{\pgfqpoint{8.851069in}{7.081890in}}%
\pgfusepath{clip}%
\pgfsetbuttcap%
\pgfsetroundjoin%
\definecolor{currentfill}{rgb}{0.277018,0.050344,0.375715}%
\pgfsetfillcolor{currentfill}%
\pgfsetfillopacity{0.700000}%
\pgfsetlinewidth{0.501875pt}%
\definecolor{currentstroke}{rgb}{1.000000,1.000000,1.000000}%
\pgfsetstrokecolor{currentstroke}%
\pgfsetstrokeopacity{0.700000}%
\pgfsetdash{}{0pt}%
\pgfpathmoveto{\pgfqpoint{3.188675in}{5.201575in}}%
\pgfpathcurveto{\pgfqpoint{3.201698in}{5.201575in}}{\pgfqpoint{3.214189in}{5.206749in}}{\pgfqpoint{3.223397in}{5.215957in}}%
\pgfpathcurveto{\pgfqpoint{3.232606in}{5.225166in}}{\pgfqpoint{3.237780in}{5.237657in}}{\pgfqpoint{3.237780in}{5.250679in}}%
\pgfpathcurveto{\pgfqpoint{3.237780in}{5.263702in}}{\pgfqpoint{3.232606in}{5.276193in}}{\pgfqpoint{3.223397in}{5.285402in}}%
\pgfpathcurveto{\pgfqpoint{3.214189in}{5.294610in}}{\pgfqpoint{3.201698in}{5.299784in}}{\pgfqpoint{3.188675in}{5.299784in}}%
\pgfpathcurveto{\pgfqpoint{3.175652in}{5.299784in}}{\pgfqpoint{3.163161in}{5.294610in}}{\pgfqpoint{3.153953in}{5.285402in}}%
\pgfpathcurveto{\pgfqpoint{3.144744in}{5.276193in}}{\pgfqpoint{3.139571in}{5.263702in}}{\pgfqpoint{3.139571in}{5.250679in}}%
\pgfpathcurveto{\pgfqpoint{3.139571in}{5.237657in}}{\pgfqpoint{3.144744in}{5.225166in}}{\pgfqpoint{3.153953in}{5.215957in}}%
\pgfpathcurveto{\pgfqpoint{3.163161in}{5.206749in}}{\pgfqpoint{3.175652in}{5.201575in}}{\pgfqpoint{3.188675in}{5.201575in}}%
\pgfpathlineto{\pgfqpoint{3.188675in}{5.201575in}}%
\pgfpathclose%
\pgfusepath{stroke,fill}%
\end{pgfscope}%
\begin{pgfscope}%
\pgfpathrectangle{\pgfqpoint{0.786164in}{0.768110in}}{\pgfqpoint{8.851069in}{7.081890in}}%
\pgfusepath{clip}%
\pgfsetbuttcap%
\pgfsetroundjoin%
\definecolor{currentfill}{rgb}{0.278791,0.062145,0.386592}%
\pgfsetfillcolor{currentfill}%
\pgfsetfillopacity{0.700000}%
\pgfsetlinewidth{0.501875pt}%
\definecolor{currentstroke}{rgb}{1.000000,1.000000,1.000000}%
\pgfsetstrokecolor{currentstroke}%
\pgfsetstrokeopacity{0.700000}%
\pgfsetdash{}{0pt}%
\pgfpathmoveto{\pgfqpoint{3.152142in}{5.223473in}}%
\pgfpathcurveto{\pgfqpoint{3.165165in}{5.223473in}}{\pgfqpoint{3.177656in}{5.228647in}}{\pgfqpoint{3.186864in}{5.237855in}}%
\pgfpathcurveto{\pgfqpoint{3.196073in}{5.247064in}}{\pgfqpoint{3.201247in}{5.259555in}}{\pgfqpoint{3.201247in}{5.272578in}}%
\pgfpathcurveto{\pgfqpoint{3.201247in}{5.285600in}}{\pgfqpoint{3.196073in}{5.298091in}}{\pgfqpoint{3.186864in}{5.307300in}}%
\pgfpathcurveto{\pgfqpoint{3.177656in}{5.316508in}}{\pgfqpoint{3.165165in}{5.321682in}}{\pgfqpoint{3.152142in}{5.321682in}}%
\pgfpathcurveto{\pgfqpoint{3.139119in}{5.321682in}}{\pgfqpoint{3.126628in}{5.316508in}}{\pgfqpoint{3.117420in}{5.307300in}}%
\pgfpathcurveto{\pgfqpoint{3.108211in}{5.298091in}}{\pgfqpoint{3.103037in}{5.285600in}}{\pgfqpoint{3.103037in}{5.272578in}}%
\pgfpathcurveto{\pgfqpoint{3.103037in}{5.259555in}}{\pgfqpoint{3.108211in}{5.247064in}}{\pgfqpoint{3.117420in}{5.237855in}}%
\pgfpathcurveto{\pgfqpoint{3.126628in}{5.228647in}}{\pgfqpoint{3.139119in}{5.223473in}}{\pgfqpoint{3.152142in}{5.223473in}}%
\pgfpathlineto{\pgfqpoint{3.152142in}{5.223473in}}%
\pgfpathclose%
\pgfusepath{stroke,fill}%
\end{pgfscope}%
\begin{pgfscope}%
\pgfpathrectangle{\pgfqpoint{0.786164in}{0.768110in}}{\pgfqpoint{8.851069in}{7.081890in}}%
\pgfusepath{clip}%
\pgfsetbuttcap%
\pgfsetroundjoin%
\definecolor{currentfill}{rgb}{0.279566,0.067836,0.391917}%
\pgfsetfillcolor{currentfill}%
\pgfsetfillopacity{0.700000}%
\pgfsetlinewidth{0.501875pt}%
\definecolor{currentstroke}{rgb}{1.000000,1.000000,1.000000}%
\pgfsetstrokecolor{currentstroke}%
\pgfsetstrokeopacity{0.700000}%
\pgfsetdash{}{0pt}%
\pgfpathmoveto{\pgfqpoint{3.106476in}{5.113982in}}%
\pgfpathcurveto{\pgfqpoint{3.119498in}{5.113982in}}{\pgfqpoint{3.131989in}{5.119156in}}{\pgfqpoint{3.141198in}{5.128364in}}%
\pgfpathcurveto{\pgfqpoint{3.150406in}{5.137573in}}{\pgfqpoint{3.155580in}{5.150064in}}{\pgfqpoint{3.155580in}{5.163086in}}%
\pgfpathcurveto{\pgfqpoint{3.155580in}{5.176109in}}{\pgfqpoint{3.150406in}{5.188600in}}{\pgfqpoint{3.141198in}{5.197809in}}%
\pgfpathcurveto{\pgfqpoint{3.131989in}{5.207017in}}{\pgfqpoint{3.119498in}{5.212191in}}{\pgfqpoint{3.106476in}{5.212191in}}%
\pgfpathcurveto{\pgfqpoint{3.093453in}{5.212191in}}{\pgfqpoint{3.080962in}{5.207017in}}{\pgfqpoint{3.071753in}{5.197809in}}%
\pgfpathcurveto{\pgfqpoint{3.062545in}{5.188600in}}{\pgfqpoint{3.057371in}{5.176109in}}{\pgfqpoint{3.057371in}{5.163086in}}%
\pgfpathcurveto{\pgfqpoint{3.057371in}{5.150064in}}{\pgfqpoint{3.062545in}{5.137573in}}{\pgfqpoint{3.071753in}{5.128364in}}%
\pgfpathcurveto{\pgfqpoint{3.080962in}{5.119156in}}{\pgfqpoint{3.093453in}{5.113982in}}{\pgfqpoint{3.106476in}{5.113982in}}%
\pgfpathlineto{\pgfqpoint{3.106476in}{5.113982in}}%
\pgfpathclose%
\pgfusepath{stroke,fill}%
\end{pgfscope}%
\begin{pgfscope}%
\pgfpathrectangle{\pgfqpoint{0.786164in}{0.768110in}}{\pgfqpoint{8.851069in}{7.081890in}}%
\pgfusepath{clip}%
\pgfsetbuttcap%
\pgfsetroundjoin%
\definecolor{currentfill}{rgb}{0.280894,0.078907,0.402329}%
\pgfsetfillcolor{currentfill}%
\pgfsetfillopacity{0.700000}%
\pgfsetlinewidth{0.501875pt}%
\definecolor{currentstroke}{rgb}{1.000000,1.000000,1.000000}%
\pgfsetstrokecolor{currentstroke}%
\pgfsetstrokeopacity{0.700000}%
\pgfsetdash{}{0pt}%
\pgfpathmoveto{\pgfqpoint{3.188675in}{4.894999in}}%
\pgfpathcurveto{\pgfqpoint{3.201698in}{4.894999in}}{\pgfqpoint{3.214189in}{4.900173in}}{\pgfqpoint{3.223397in}{4.909382in}}%
\pgfpathcurveto{\pgfqpoint{3.232606in}{4.918590in}}{\pgfqpoint{3.237780in}{4.931081in}}{\pgfqpoint{3.237780in}{4.944104in}}%
\pgfpathcurveto{\pgfqpoint{3.237780in}{4.957127in}}{\pgfqpoint{3.232606in}{4.969618in}}{\pgfqpoint{3.223397in}{4.978826in}}%
\pgfpathcurveto{\pgfqpoint{3.214189in}{4.988035in}}{\pgfqpoint{3.201698in}{4.993209in}}{\pgfqpoint{3.188675in}{4.993209in}}%
\pgfpathcurveto{\pgfqpoint{3.175652in}{4.993209in}}{\pgfqpoint{3.163161in}{4.988035in}}{\pgfqpoint{3.153953in}{4.978826in}}%
\pgfpathcurveto{\pgfqpoint{3.144744in}{4.969618in}}{\pgfqpoint{3.139571in}{4.957127in}}{\pgfqpoint{3.139571in}{4.944104in}}%
\pgfpathcurveto{\pgfqpoint{3.139571in}{4.931081in}}{\pgfqpoint{3.144744in}{4.918590in}}{\pgfqpoint{3.153953in}{4.909382in}}%
\pgfpathcurveto{\pgfqpoint{3.163161in}{4.900173in}}{\pgfqpoint{3.175652in}{4.894999in}}{\pgfqpoint{3.188675in}{4.894999in}}%
\pgfpathlineto{\pgfqpoint{3.188675in}{4.894999in}}%
\pgfpathclose%
\pgfusepath{stroke,fill}%
\end{pgfscope}%
\begin{pgfscope}%
\pgfpathrectangle{\pgfqpoint{0.786164in}{0.768110in}}{\pgfqpoint{8.851069in}{7.081890in}}%
\pgfusepath{clip}%
\pgfsetbuttcap%
\pgfsetroundjoin%
\definecolor{currentfill}{rgb}{0.281924,0.089666,0.412415}%
\pgfsetfillcolor{currentfill}%
\pgfsetfillopacity{0.700000}%
\pgfsetlinewidth{0.501875pt}%
\definecolor{currentstroke}{rgb}{1.000000,1.000000,1.000000}%
\pgfsetstrokecolor{currentstroke}%
\pgfsetstrokeopacity{0.700000}%
\pgfsetdash{}{0pt}%
\pgfpathmoveto{\pgfqpoint{3.060809in}{4.741712in}}%
\pgfpathcurveto{\pgfqpoint{3.073832in}{4.741712in}}{\pgfqpoint{3.086323in}{4.746886in}}{\pgfqpoint{3.095531in}{4.756094in}}%
\pgfpathcurveto{\pgfqpoint{3.104740in}{4.765303in}}{\pgfqpoint{3.109914in}{4.777794in}}{\pgfqpoint{3.109914in}{4.790816in}}%
\pgfpathcurveto{\pgfqpoint{3.109914in}{4.803839in}}{\pgfqpoint{3.104740in}{4.816330in}}{\pgfqpoint{3.095531in}{4.825539in}}%
\pgfpathcurveto{\pgfqpoint{3.086323in}{4.834747in}}{\pgfqpoint{3.073832in}{4.839921in}}{\pgfqpoint{3.060809in}{4.839921in}}%
\pgfpathcurveto{\pgfqpoint{3.047786in}{4.839921in}}{\pgfqpoint{3.035295in}{4.834747in}}{\pgfqpoint{3.026087in}{4.825539in}}%
\pgfpathcurveto{\pgfqpoint{3.016878in}{4.816330in}}{\pgfqpoint{3.011704in}{4.803839in}}{\pgfqpoint{3.011704in}{4.790816in}}%
\pgfpathcurveto{\pgfqpoint{3.011704in}{4.777794in}}{\pgfqpoint{3.016878in}{4.765303in}}{\pgfqpoint{3.026087in}{4.756094in}}%
\pgfpathcurveto{\pgfqpoint{3.035295in}{4.746886in}}{\pgfqpoint{3.047786in}{4.741712in}}{\pgfqpoint{3.060809in}{4.741712in}}%
\pgfpathlineto{\pgfqpoint{3.060809in}{4.741712in}}%
\pgfpathclose%
\pgfusepath{stroke,fill}%
\end{pgfscope}%
\begin{pgfscope}%
\pgfpathrectangle{\pgfqpoint{0.786164in}{0.768110in}}{\pgfqpoint{8.851069in}{7.081890in}}%
\pgfusepath{clip}%
\pgfsetbuttcap%
\pgfsetroundjoin%
\definecolor{currentfill}{rgb}{0.283091,0.110553,0.431554}%
\pgfsetfillcolor{currentfill}%
\pgfsetfillopacity{0.700000}%
\pgfsetlinewidth{0.501875pt}%
\definecolor{currentstroke}{rgb}{1.000000,1.000000,1.000000}%
\pgfsetstrokecolor{currentstroke}%
\pgfsetstrokeopacity{0.700000}%
\pgfsetdash{}{0pt}%
\pgfpathmoveto{\pgfqpoint{2.878143in}{4.369442in}}%
\pgfpathcurveto{\pgfqpoint{2.891166in}{4.369442in}}{\pgfqpoint{2.903657in}{4.374616in}}{\pgfqpoint{2.912866in}{4.383824in}}%
\pgfpathcurveto{\pgfqpoint{2.922074in}{4.393033in}}{\pgfqpoint{2.927248in}{4.405524in}}{\pgfqpoint{2.927248in}{4.418546in}}%
\pgfpathcurveto{\pgfqpoint{2.927248in}{4.431569in}}{\pgfqpoint{2.922074in}{4.444060in}}{\pgfqpoint{2.912866in}{4.453269in}}%
\pgfpathcurveto{\pgfqpoint{2.903657in}{4.462477in}}{\pgfqpoint{2.891166in}{4.467651in}}{\pgfqpoint{2.878143in}{4.467651in}}%
\pgfpathcurveto{\pgfqpoint{2.865121in}{4.467651in}}{\pgfqpoint{2.852630in}{4.462477in}}{\pgfqpoint{2.843421in}{4.453269in}}%
\pgfpathcurveto{\pgfqpoint{2.834213in}{4.444060in}}{\pgfqpoint{2.829039in}{4.431569in}}{\pgfqpoint{2.829039in}{4.418546in}}%
\pgfpathcurveto{\pgfqpoint{2.829039in}{4.405524in}}{\pgfqpoint{2.834213in}{4.393033in}}{\pgfqpoint{2.843421in}{4.383824in}}%
\pgfpathcurveto{\pgfqpoint{2.852630in}{4.374616in}}{\pgfqpoint{2.865121in}{4.369442in}}{\pgfqpoint{2.878143in}{4.369442in}}%
\pgfpathlineto{\pgfqpoint{2.878143in}{4.369442in}}%
\pgfpathclose%
\pgfusepath{stroke,fill}%
\end{pgfscope}%
\begin{pgfscope}%
\pgfpathrectangle{\pgfqpoint{0.786164in}{0.768110in}}{\pgfqpoint{8.851069in}{7.081890in}}%
\pgfusepath{clip}%
\pgfsetbuttcap%
\pgfsetroundjoin%
\definecolor{currentfill}{rgb}{0.283229,0.120777,0.440584}%
\pgfsetfillcolor{currentfill}%
\pgfsetfillopacity{0.700000}%
\pgfsetlinewidth{0.501875pt}%
\definecolor{currentstroke}{rgb}{1.000000,1.000000,1.000000}%
\pgfsetstrokecolor{currentstroke}%
\pgfsetstrokeopacity{0.700000}%
\pgfsetdash{}{0pt}%
\pgfpathmoveto{\pgfqpoint{2.823344in}{4.391340in}}%
\pgfpathcurveto{\pgfqpoint{2.836366in}{4.391340in}}{\pgfqpoint{2.848857in}{4.396514in}}{\pgfqpoint{2.858066in}{4.405722in}}%
\pgfpathcurveto{\pgfqpoint{2.867274in}{4.414931in}}{\pgfqpoint{2.872448in}{4.427422in}}{\pgfqpoint{2.872448in}{4.440445in}}%
\pgfpathcurveto{\pgfqpoint{2.872448in}{4.453467in}}{\pgfqpoint{2.867274in}{4.465958in}}{\pgfqpoint{2.858066in}{4.475167in}}%
\pgfpathcurveto{\pgfqpoint{2.848857in}{4.484375in}}{\pgfqpoint{2.836366in}{4.489549in}}{\pgfqpoint{2.823344in}{4.489549in}}%
\pgfpathcurveto{\pgfqpoint{2.810321in}{4.489549in}}{\pgfqpoint{2.797830in}{4.484375in}}{\pgfqpoint{2.788621in}{4.475167in}}%
\pgfpathcurveto{\pgfqpoint{2.779413in}{4.465958in}}{\pgfqpoint{2.774239in}{4.453467in}}{\pgfqpoint{2.774239in}{4.440445in}}%
\pgfpathcurveto{\pgfqpoint{2.774239in}{4.427422in}}{\pgfqpoint{2.779413in}{4.414931in}}{\pgfqpoint{2.788621in}{4.405722in}}%
\pgfpathcurveto{\pgfqpoint{2.797830in}{4.396514in}}{\pgfqpoint{2.810321in}{4.391340in}}{\pgfqpoint{2.823344in}{4.391340in}}%
\pgfpathlineto{\pgfqpoint{2.823344in}{4.391340in}}%
\pgfpathclose%
\pgfusepath{stroke,fill}%
\end{pgfscope}%
\begin{pgfscope}%
\pgfpathrectangle{\pgfqpoint{0.786164in}{0.768110in}}{\pgfqpoint{8.851069in}{7.081890in}}%
\pgfusepath{clip}%
\pgfsetbuttcap%
\pgfsetroundjoin%
\definecolor{currentfill}{rgb}{0.282623,0.140926,0.457517}%
\pgfsetfillcolor{currentfill}%
\pgfsetfillopacity{0.700000}%
\pgfsetlinewidth{0.501875pt}%
\definecolor{currentstroke}{rgb}{1.000000,1.000000,1.000000}%
\pgfsetstrokecolor{currentstroke}%
\pgfsetstrokeopacity{0.700000}%
\pgfsetdash{}{0pt}%
\pgfpathmoveto{\pgfqpoint{2.576745in}{4.128561in}}%
\pgfpathcurveto{\pgfqpoint{2.589768in}{4.128561in}}{\pgfqpoint{2.602259in}{4.133735in}}{\pgfqpoint{2.611467in}{4.142944in}}%
\pgfpathcurveto{\pgfqpoint{2.620676in}{4.152152in}}{\pgfqpoint{2.625850in}{4.164643in}}{\pgfqpoint{2.625850in}{4.177666in}}%
\pgfpathcurveto{\pgfqpoint{2.625850in}{4.190688in}}{\pgfqpoint{2.620676in}{4.203180in}}{\pgfqpoint{2.611467in}{4.212388in}}%
\pgfpathcurveto{\pgfqpoint{2.602259in}{4.221596in}}{\pgfqpoint{2.589768in}{4.226770in}}{\pgfqpoint{2.576745in}{4.226770in}}%
\pgfpathcurveto{\pgfqpoint{2.563722in}{4.226770in}}{\pgfqpoint{2.551231in}{4.221596in}}{\pgfqpoint{2.542023in}{4.212388in}}%
\pgfpathcurveto{\pgfqpoint{2.532814in}{4.203180in}}{\pgfqpoint{2.527640in}{4.190688in}}{\pgfqpoint{2.527640in}{4.177666in}}%
\pgfpathcurveto{\pgfqpoint{2.527640in}{4.164643in}}{\pgfqpoint{2.532814in}{4.152152in}}{\pgfqpoint{2.542023in}{4.142944in}}%
\pgfpathcurveto{\pgfqpoint{2.551231in}{4.133735in}}{\pgfqpoint{2.563722in}{4.128561in}}{\pgfqpoint{2.576745in}{4.128561in}}%
\pgfpathlineto{\pgfqpoint{2.576745in}{4.128561in}}%
\pgfpathclose%
\pgfusepath{stroke,fill}%
\end{pgfscope}%
\begin{pgfscope}%
\pgfpathrectangle{\pgfqpoint{0.786164in}{0.768110in}}{\pgfqpoint{8.851069in}{7.081890in}}%
\pgfusepath{clip}%
\pgfsetbuttcap%
\pgfsetroundjoin%
\definecolor{currentfill}{rgb}{0.282290,0.145912,0.461510}%
\pgfsetfillcolor{currentfill}%
\pgfsetfillopacity{0.700000}%
\pgfsetlinewidth{0.501875pt}%
\definecolor{currentstroke}{rgb}{1.000000,1.000000,1.000000}%
\pgfsetstrokecolor{currentstroke}%
\pgfsetstrokeopacity{0.700000}%
\pgfsetdash{}{0pt}%
\pgfpathmoveto{\pgfqpoint{2.595011in}{4.106663in}}%
\pgfpathcurveto{\pgfqpoint{2.608034in}{4.106663in}}{\pgfqpoint{2.620525in}{4.111837in}}{\pgfqpoint{2.629734in}{4.121045in}}%
\pgfpathcurveto{\pgfqpoint{2.638942in}{4.130254in}}{\pgfqpoint{2.644116in}{4.142745in}}{\pgfqpoint{2.644116in}{4.155768in}}%
\pgfpathcurveto{\pgfqpoint{2.644116in}{4.168790in}}{\pgfqpoint{2.638942in}{4.181281in}}{\pgfqpoint{2.629734in}{4.190490in}}%
\pgfpathcurveto{\pgfqpoint{2.620525in}{4.199698in}}{\pgfqpoint{2.608034in}{4.204872in}}{\pgfqpoint{2.595011in}{4.204872in}}%
\pgfpathcurveto{\pgfqpoint{2.581989in}{4.204872in}}{\pgfqpoint{2.569498in}{4.199698in}}{\pgfqpoint{2.560289in}{4.190490in}}%
\pgfpathcurveto{\pgfqpoint{2.551081in}{4.181281in}}{\pgfqpoint{2.545907in}{4.168790in}}{\pgfqpoint{2.545907in}{4.155768in}}%
\pgfpathcurveto{\pgfqpoint{2.545907in}{4.142745in}}{\pgfqpoint{2.551081in}{4.130254in}}{\pgfqpoint{2.560289in}{4.121045in}}%
\pgfpathcurveto{\pgfqpoint{2.569498in}{4.111837in}}{\pgfqpoint{2.581989in}{4.106663in}}{\pgfqpoint{2.595011in}{4.106663in}}%
\pgfpathlineto{\pgfqpoint{2.595011in}{4.106663in}}%
\pgfpathclose%
\pgfusepath{stroke,fill}%
\end{pgfscope}%
\begin{pgfscope}%
\pgfpathrectangle{\pgfqpoint{0.786164in}{0.768110in}}{\pgfqpoint{8.851069in}{7.081890in}}%
\pgfusepath{clip}%
\pgfsetbuttcap%
\pgfsetroundjoin%
\definecolor{currentfill}{rgb}{0.281412,0.155834,0.469201}%
\pgfsetfillcolor{currentfill}%
\pgfsetfillopacity{0.700000}%
\pgfsetlinewidth{0.501875pt}%
\definecolor{currentstroke}{rgb}{1.000000,1.000000,1.000000}%
\pgfsetstrokecolor{currentstroke}%
\pgfsetstrokeopacity{0.700000}%
\pgfsetdash{}{0pt}%
\pgfpathmoveto{\pgfqpoint{2.439746in}{4.128561in}}%
\pgfpathcurveto{\pgfqpoint{2.452768in}{4.128561in}}{\pgfqpoint{2.465259in}{4.133735in}}{\pgfqpoint{2.474468in}{4.142944in}}%
\pgfpathcurveto{\pgfqpoint{2.483676in}{4.152152in}}{\pgfqpoint{2.488850in}{4.164643in}}{\pgfqpoint{2.488850in}{4.177666in}}%
\pgfpathcurveto{\pgfqpoint{2.488850in}{4.190688in}}{\pgfqpoint{2.483676in}{4.203180in}}{\pgfqpoint{2.474468in}{4.212388in}}%
\pgfpathcurveto{\pgfqpoint{2.465259in}{4.221596in}}{\pgfqpoint{2.452768in}{4.226770in}}{\pgfqpoint{2.439746in}{4.226770in}}%
\pgfpathcurveto{\pgfqpoint{2.426723in}{4.226770in}}{\pgfqpoint{2.414232in}{4.221596in}}{\pgfqpoint{2.405023in}{4.212388in}}%
\pgfpathcurveto{\pgfqpoint{2.395815in}{4.203180in}}{\pgfqpoint{2.390641in}{4.190688in}}{\pgfqpoint{2.390641in}{4.177666in}}%
\pgfpathcurveto{\pgfqpoint{2.390641in}{4.164643in}}{\pgfqpoint{2.395815in}{4.152152in}}{\pgfqpoint{2.405023in}{4.142944in}}%
\pgfpathcurveto{\pgfqpoint{2.414232in}{4.133735in}}{\pgfqpoint{2.426723in}{4.128561in}}{\pgfqpoint{2.439746in}{4.128561in}}%
\pgfpathlineto{\pgfqpoint{2.439746in}{4.128561in}}%
\pgfpathclose%
\pgfusepath{stroke,fill}%
\end{pgfscope}%
\begin{pgfscope}%
\pgfpathrectangle{\pgfqpoint{0.786164in}{0.768110in}}{\pgfqpoint{8.851069in}{7.081890in}}%
\pgfusepath{clip}%
\pgfsetbuttcap%
\pgfsetroundjoin%
\definecolor{currentfill}{rgb}{0.278826,0.175490,0.483397}%
\pgfsetfillcolor{currentfill}%
\pgfsetfillopacity{0.700000}%
\pgfsetlinewidth{0.501875pt}%
\definecolor{currentstroke}{rgb}{1.000000,1.000000,1.000000}%
\pgfsetstrokecolor{currentstroke}%
\pgfsetstrokeopacity{0.700000}%
\pgfsetdash{}{0pt}%
\pgfpathmoveto{\pgfqpoint{2.458012in}{4.062866in}}%
\pgfpathcurveto{\pgfqpoint{2.471035in}{4.062866in}}{\pgfqpoint{2.483526in}{4.068040in}}{\pgfqpoint{2.492734in}{4.077249in}}%
\pgfpathcurveto{\pgfqpoint{2.501943in}{4.086457in}}{\pgfqpoint{2.507117in}{4.098948in}}{\pgfqpoint{2.507117in}{4.111971in}}%
\pgfpathcurveto{\pgfqpoint{2.507117in}{4.124994in}}{\pgfqpoint{2.501943in}{4.137485in}}{\pgfqpoint{2.492734in}{4.146693in}}%
\pgfpathcurveto{\pgfqpoint{2.483526in}{4.155902in}}{\pgfqpoint{2.471035in}{4.161076in}}{\pgfqpoint{2.458012in}{4.161076in}}%
\pgfpathcurveto{\pgfqpoint{2.444989in}{4.161076in}}{\pgfqpoint{2.432498in}{4.155902in}}{\pgfqpoint{2.423290in}{4.146693in}}%
\pgfpathcurveto{\pgfqpoint{2.414081in}{4.137485in}}{\pgfqpoint{2.408908in}{4.124994in}}{\pgfqpoint{2.408908in}{4.111971in}}%
\pgfpathcurveto{\pgfqpoint{2.408908in}{4.098948in}}{\pgfqpoint{2.414081in}{4.086457in}}{\pgfqpoint{2.423290in}{4.077249in}}%
\pgfpathcurveto{\pgfqpoint{2.432498in}{4.068040in}}{\pgfqpoint{2.444989in}{4.062866in}}{\pgfqpoint{2.458012in}{4.062866in}}%
\pgfpathlineto{\pgfqpoint{2.458012in}{4.062866in}}%
\pgfpathclose%
\pgfusepath{stroke,fill}%
\end{pgfscope}%
\begin{pgfscope}%
\pgfpathrectangle{\pgfqpoint{0.786164in}{0.768110in}}{\pgfqpoint{8.851069in}{7.081890in}}%
\pgfusepath{clip}%
\pgfsetbuttcap%
\pgfsetroundjoin%
\definecolor{currentfill}{rgb}{0.277134,0.185228,0.489898}%
\pgfsetfillcolor{currentfill}%
\pgfsetfillopacity{0.700000}%
\pgfsetlinewidth{0.501875pt}%
\definecolor{currentstroke}{rgb}{1.000000,1.000000,1.000000}%
\pgfsetstrokecolor{currentstroke}%
\pgfsetstrokeopacity{0.700000}%
\pgfsetdash{}{0pt}%
\pgfpathmoveto{\pgfqpoint{2.576745in}{4.194256in}}%
\pgfpathcurveto{\pgfqpoint{2.589768in}{4.194256in}}{\pgfqpoint{2.602259in}{4.199430in}}{\pgfqpoint{2.611467in}{4.208638in}}%
\pgfpathcurveto{\pgfqpoint{2.620676in}{4.217847in}}{\pgfqpoint{2.625850in}{4.230338in}}{\pgfqpoint{2.625850in}{4.243360in}}%
\pgfpathcurveto{\pgfqpoint{2.625850in}{4.256383in}}{\pgfqpoint{2.620676in}{4.268874in}}{\pgfqpoint{2.611467in}{4.278083in}}%
\pgfpathcurveto{\pgfqpoint{2.602259in}{4.287291in}}{\pgfqpoint{2.589768in}{4.292465in}}{\pgfqpoint{2.576745in}{4.292465in}}%
\pgfpathcurveto{\pgfqpoint{2.563722in}{4.292465in}}{\pgfqpoint{2.551231in}{4.287291in}}{\pgfqpoint{2.542023in}{4.278083in}}%
\pgfpathcurveto{\pgfqpoint{2.532814in}{4.268874in}}{\pgfqpoint{2.527640in}{4.256383in}}{\pgfqpoint{2.527640in}{4.243360in}}%
\pgfpathcurveto{\pgfqpoint{2.527640in}{4.230338in}}{\pgfqpoint{2.532814in}{4.217847in}}{\pgfqpoint{2.542023in}{4.208638in}}%
\pgfpathcurveto{\pgfqpoint{2.551231in}{4.199430in}}{\pgfqpoint{2.563722in}{4.194256in}}{\pgfqpoint{2.576745in}{4.194256in}}%
\pgfpathlineto{\pgfqpoint{2.576745in}{4.194256in}}%
\pgfpathclose%
\pgfusepath{stroke,fill}%
\end{pgfscope}%
\begin{pgfscope}%
\pgfpathrectangle{\pgfqpoint{0.786164in}{0.768110in}}{\pgfqpoint{8.851069in}{7.081890in}}%
\pgfusepath{clip}%
\pgfsetbuttcap%
\pgfsetroundjoin%
\definecolor{currentfill}{rgb}{0.276194,0.190074,0.493001}%
\pgfsetfillcolor{currentfill}%
\pgfsetfillopacity{0.700000}%
\pgfsetlinewidth{0.501875pt}%
\definecolor{currentstroke}{rgb}{1.000000,1.000000,1.000000}%
\pgfsetstrokecolor{currentstroke}%
\pgfsetstrokeopacity{0.700000}%
\pgfsetdash{}{0pt}%
\pgfpathmoveto{\pgfqpoint{2.686344in}{4.216154in}}%
\pgfpathcurveto{\pgfqpoint{2.699367in}{4.216154in}}{\pgfqpoint{2.711858in}{4.221328in}}{\pgfqpoint{2.721067in}{4.230537in}}%
\pgfpathcurveto{\pgfqpoint{2.730275in}{4.239745in}}{\pgfqpoint{2.735449in}{4.252236in}}{\pgfqpoint{2.735449in}{4.265259in}}%
\pgfpathcurveto{\pgfqpoint{2.735449in}{4.278281in}}{\pgfqpoint{2.730275in}{4.290773in}}{\pgfqpoint{2.721067in}{4.299981in}}%
\pgfpathcurveto{\pgfqpoint{2.711858in}{4.309189in}}{\pgfqpoint{2.699367in}{4.314363in}}{\pgfqpoint{2.686344in}{4.314363in}}%
\pgfpathcurveto{\pgfqpoint{2.673322in}{4.314363in}}{\pgfqpoint{2.660831in}{4.309189in}}{\pgfqpoint{2.651622in}{4.299981in}}%
\pgfpathcurveto{\pgfqpoint{2.642414in}{4.290773in}}{\pgfqpoint{2.637240in}{4.278281in}}{\pgfqpoint{2.637240in}{4.265259in}}%
\pgfpathcurveto{\pgfqpoint{2.637240in}{4.252236in}}{\pgfqpoint{2.642414in}{4.239745in}}{\pgfqpoint{2.651622in}{4.230537in}}%
\pgfpathcurveto{\pgfqpoint{2.660831in}{4.221328in}}{\pgfqpoint{2.673322in}{4.216154in}}{\pgfqpoint{2.686344in}{4.216154in}}%
\pgfpathlineto{\pgfqpoint{2.686344in}{4.216154in}}%
\pgfpathclose%
\pgfusepath{stroke,fill}%
\end{pgfscope}%
\begin{pgfscope}%
\pgfpathrectangle{\pgfqpoint{0.786164in}{0.768110in}}{\pgfqpoint{8.851069in}{7.081890in}}%
\pgfusepath{clip}%
\pgfsetbuttcap%
\pgfsetroundjoin%
\definecolor{currentfill}{rgb}{0.270595,0.214069,0.507052}%
\pgfsetfillcolor{currentfill}%
\pgfsetfillopacity{0.700000}%
\pgfsetlinewidth{0.501875pt}%
\definecolor{currentstroke}{rgb}{1.000000,1.000000,1.000000}%
\pgfsetstrokecolor{currentstroke}%
\pgfsetstrokeopacity{0.700000}%
\pgfsetdash{}{0pt}%
\pgfpathmoveto{\pgfqpoint{2.640678in}{4.259951in}}%
\pgfpathcurveto{\pgfqpoint{2.653701in}{4.259951in}}{\pgfqpoint{2.666192in}{4.265125in}}{\pgfqpoint{2.675400in}{4.274333in}}%
\pgfpathcurveto{\pgfqpoint{2.684609in}{4.283541in}}{\pgfqpoint{2.689783in}{4.296033in}}{\pgfqpoint{2.689783in}{4.309055in}}%
\pgfpathcurveto{\pgfqpoint{2.689783in}{4.322078in}}{\pgfqpoint{2.684609in}{4.334569in}}{\pgfqpoint{2.675400in}{4.343777in}}%
\pgfpathcurveto{\pgfqpoint{2.666192in}{4.352986in}}{\pgfqpoint{2.653701in}{4.358160in}}{\pgfqpoint{2.640678in}{4.358160in}}%
\pgfpathcurveto{\pgfqpoint{2.627655in}{4.358160in}}{\pgfqpoint{2.615164in}{4.352986in}}{\pgfqpoint{2.605956in}{4.343777in}}%
\pgfpathcurveto{\pgfqpoint{2.596747in}{4.334569in}}{\pgfqpoint{2.591573in}{4.322078in}}{\pgfqpoint{2.591573in}{4.309055in}}%
\pgfpathcurveto{\pgfqpoint{2.591573in}{4.296033in}}{\pgfqpoint{2.596747in}{4.283541in}}{\pgfqpoint{2.605956in}{4.274333in}}%
\pgfpathcurveto{\pgfqpoint{2.615164in}{4.265125in}}{\pgfqpoint{2.627655in}{4.259951in}}{\pgfqpoint{2.640678in}{4.259951in}}%
\pgfpathlineto{\pgfqpoint{2.640678in}{4.259951in}}%
\pgfpathclose%
\pgfusepath{stroke,fill}%
\end{pgfscope}%
\begin{pgfscope}%
\pgfpathrectangle{\pgfqpoint{0.786164in}{0.768110in}}{\pgfqpoint{8.851069in}{7.081890in}}%
\pgfusepath{clip}%
\pgfsetbuttcap%
\pgfsetroundjoin%
\definecolor{currentfill}{rgb}{0.267968,0.223549,0.512008}%
\pgfsetfillcolor{currentfill}%
\pgfsetfillopacity{0.700000}%
\pgfsetlinewidth{0.501875pt}%
\definecolor{currentstroke}{rgb}{1.000000,1.000000,1.000000}%
\pgfsetstrokecolor{currentstroke}%
\pgfsetstrokeopacity{0.700000}%
\pgfsetdash{}{0pt}%
\pgfpathmoveto{\pgfqpoint{2.613278in}{4.216154in}}%
\pgfpathcurveto{\pgfqpoint{2.626301in}{4.216154in}}{\pgfqpoint{2.638792in}{4.221328in}}{\pgfqpoint{2.648000in}{4.230537in}}%
\pgfpathcurveto{\pgfqpoint{2.657209in}{4.239745in}}{\pgfqpoint{2.662383in}{4.252236in}}{\pgfqpoint{2.662383in}{4.265259in}}%
\pgfpathcurveto{\pgfqpoint{2.662383in}{4.278281in}}{\pgfqpoint{2.657209in}{4.290773in}}{\pgfqpoint{2.648000in}{4.299981in}}%
\pgfpathcurveto{\pgfqpoint{2.638792in}{4.309189in}}{\pgfqpoint{2.626301in}{4.314363in}}{\pgfqpoint{2.613278in}{4.314363in}}%
\pgfpathcurveto{\pgfqpoint{2.600255in}{4.314363in}}{\pgfqpoint{2.587764in}{4.309189in}}{\pgfqpoint{2.578556in}{4.299981in}}%
\pgfpathcurveto{\pgfqpoint{2.569347in}{4.290773in}}{\pgfqpoint{2.564173in}{4.278281in}}{\pgfqpoint{2.564173in}{4.265259in}}%
\pgfpathcurveto{\pgfqpoint{2.564173in}{4.252236in}}{\pgfqpoint{2.569347in}{4.239745in}}{\pgfqpoint{2.578556in}{4.230537in}}%
\pgfpathcurveto{\pgfqpoint{2.587764in}{4.221328in}}{\pgfqpoint{2.600255in}{4.216154in}}{\pgfqpoint{2.613278in}{4.216154in}}%
\pgfpathlineto{\pgfqpoint{2.613278in}{4.216154in}}%
\pgfpathclose%
\pgfusepath{stroke,fill}%
\end{pgfscope}%
\begin{pgfscope}%
\pgfpathrectangle{\pgfqpoint{0.786164in}{0.768110in}}{\pgfqpoint{8.851069in}{7.081890in}}%
\pgfusepath{clip}%
\pgfsetbuttcap%
\pgfsetroundjoin%
\definecolor{currentfill}{rgb}{0.262138,0.242286,0.520837}%
\pgfsetfillcolor{currentfill}%
\pgfsetfillopacity{0.700000}%
\pgfsetlinewidth{0.501875pt}%
\definecolor{currentstroke}{rgb}{1.000000,1.000000,1.000000}%
\pgfsetstrokecolor{currentstroke}%
\pgfsetstrokeopacity{0.700000}%
\pgfsetdash{}{0pt}%
\pgfpathmoveto{\pgfqpoint{2.595011in}{4.062866in}}%
\pgfpathcurveto{\pgfqpoint{2.608034in}{4.062866in}}{\pgfqpoint{2.620525in}{4.068040in}}{\pgfqpoint{2.629734in}{4.077249in}}%
\pgfpathcurveto{\pgfqpoint{2.638942in}{4.086457in}}{\pgfqpoint{2.644116in}{4.098948in}}{\pgfqpoint{2.644116in}{4.111971in}}%
\pgfpathcurveto{\pgfqpoint{2.644116in}{4.124994in}}{\pgfqpoint{2.638942in}{4.137485in}}{\pgfqpoint{2.629734in}{4.146693in}}%
\pgfpathcurveto{\pgfqpoint{2.620525in}{4.155902in}}{\pgfqpoint{2.608034in}{4.161076in}}{\pgfqpoint{2.595011in}{4.161076in}}%
\pgfpathcurveto{\pgfqpoint{2.581989in}{4.161076in}}{\pgfqpoint{2.569498in}{4.155902in}}{\pgfqpoint{2.560289in}{4.146693in}}%
\pgfpathcurveto{\pgfqpoint{2.551081in}{4.137485in}}{\pgfqpoint{2.545907in}{4.124994in}}{\pgfqpoint{2.545907in}{4.111971in}}%
\pgfpathcurveto{\pgfqpoint{2.545907in}{4.098948in}}{\pgfqpoint{2.551081in}{4.086457in}}{\pgfqpoint{2.560289in}{4.077249in}}%
\pgfpathcurveto{\pgfqpoint{2.569498in}{4.068040in}}{\pgfqpoint{2.581989in}{4.062866in}}{\pgfqpoint{2.595011in}{4.062866in}}%
\pgfpathlineto{\pgfqpoint{2.595011in}{4.062866in}}%
\pgfpathclose%
\pgfusepath{stroke,fill}%
\end{pgfscope}%
\begin{pgfscope}%
\pgfpathrectangle{\pgfqpoint{0.786164in}{0.768110in}}{\pgfqpoint{8.851069in}{7.081890in}}%
\pgfusepath{clip}%
\pgfsetbuttcap%
\pgfsetroundjoin%
\definecolor{currentfill}{rgb}{0.233603,0.313828,0.543914}%
\pgfsetfillcolor{currentfill}%
\pgfsetfillopacity{0.700000}%
\pgfsetlinewidth{0.501875pt}%
\definecolor{currentstroke}{rgb}{1.000000,1.000000,1.000000}%
\pgfsetstrokecolor{currentstroke}%
\pgfsetstrokeopacity{0.700000}%
\pgfsetdash{}{0pt}%
\pgfpathmoveto{\pgfqpoint{2.339279in}{3.887681in}}%
\pgfpathcurveto{\pgfqpoint{2.352302in}{3.887681in}}{\pgfqpoint{2.364793in}{3.892855in}}{\pgfqpoint{2.374002in}{3.902063in}}%
\pgfpathcurveto{\pgfqpoint{2.383210in}{3.911271in}}{\pgfqpoint{2.388384in}{3.923762in}}{\pgfqpoint{2.388384in}{3.936785in}}%
\pgfpathcurveto{\pgfqpoint{2.388384in}{3.949808in}}{\pgfqpoint{2.383210in}{3.962299in}}{\pgfqpoint{2.374002in}{3.971507in}}%
\pgfpathcurveto{\pgfqpoint{2.364793in}{3.980716in}}{\pgfqpoint{2.352302in}{3.985890in}}{\pgfqpoint{2.339279in}{3.985890in}}%
\pgfpathcurveto{\pgfqpoint{2.326257in}{3.985890in}}{\pgfqpoint{2.313766in}{3.980716in}}{\pgfqpoint{2.304557in}{3.971507in}}%
\pgfpathcurveto{\pgfqpoint{2.295349in}{3.962299in}}{\pgfqpoint{2.290175in}{3.949808in}}{\pgfqpoint{2.290175in}{3.936785in}}%
\pgfpathcurveto{\pgfqpoint{2.290175in}{3.923762in}}{\pgfqpoint{2.295349in}{3.911271in}}{\pgfqpoint{2.304557in}{3.902063in}}%
\pgfpathcurveto{\pgfqpoint{2.313766in}{3.892855in}}{\pgfqpoint{2.326257in}{3.887681in}}{\pgfqpoint{2.339279in}{3.887681in}}%
\pgfpathlineto{\pgfqpoint{2.339279in}{3.887681in}}%
\pgfpathclose%
\pgfusepath{stroke,fill}%
\end{pgfscope}%
\begin{pgfscope}%
\pgfpathrectangle{\pgfqpoint{0.786164in}{0.768110in}}{\pgfqpoint{8.851069in}{7.081890in}}%
\pgfusepath{clip}%
\pgfsetbuttcap%
\pgfsetroundjoin%
\definecolor{currentfill}{rgb}{0.260571,0.246922,0.522828}%
\pgfsetfillcolor{currentfill}%
\pgfsetfillopacity{0.700000}%
\pgfsetlinewidth{0.501875pt}%
\definecolor{currentstroke}{rgb}{1.000000,1.000000,1.000000}%
\pgfsetstrokecolor{currentstroke}%
\pgfsetstrokeopacity{0.700000}%
\pgfsetdash{}{0pt}%
\pgfpathmoveto{\pgfqpoint{2.384946in}{3.975274in}}%
\pgfpathcurveto{\pgfqpoint{2.397969in}{3.975274in}}{\pgfqpoint{2.410460in}{3.980447in}}{\pgfqpoint{2.419668in}{3.989656in}}%
\pgfpathcurveto{\pgfqpoint{2.428877in}{3.998864in}}{\pgfqpoint{2.434050in}{4.011355in}}{\pgfqpoint{2.434050in}{4.024378in}}%
\pgfpathcurveto{\pgfqpoint{2.434050in}{4.037401in}}{\pgfqpoint{2.428877in}{4.049892in}}{\pgfqpoint{2.419668in}{4.059100in}}%
\pgfpathcurveto{\pgfqpoint{2.410460in}{4.068309in}}{\pgfqpoint{2.397969in}{4.073483in}}{\pgfqpoint{2.384946in}{4.073483in}}%
\pgfpathcurveto{\pgfqpoint{2.371923in}{4.073483in}}{\pgfqpoint{2.359432in}{4.068309in}}{\pgfqpoint{2.350224in}{4.059100in}}%
\pgfpathcurveto{\pgfqpoint{2.341015in}{4.049892in}}{\pgfqpoint{2.335841in}{4.037401in}}{\pgfqpoint{2.335841in}{4.024378in}}%
\pgfpathcurveto{\pgfqpoint{2.335841in}{4.011355in}}{\pgfqpoint{2.341015in}{3.998864in}}{\pgfqpoint{2.350224in}{3.989656in}}%
\pgfpathcurveto{\pgfqpoint{2.359432in}{3.980447in}}{\pgfqpoint{2.371923in}{3.975274in}}{\pgfqpoint{2.384946in}{3.975274in}}%
\pgfpathlineto{\pgfqpoint{2.384946in}{3.975274in}}%
\pgfpathclose%
\pgfusepath{stroke,fill}%
\end{pgfscope}%
\begin{pgfscope}%
\pgfpathrectangle{\pgfqpoint{0.786164in}{0.768110in}}{\pgfqpoint{8.851069in}{7.081890in}}%
\pgfusepath{clip}%
\pgfsetbuttcap%
\pgfsetroundjoin%
\definecolor{currentfill}{rgb}{0.255645,0.260703,0.528312}%
\pgfsetfillcolor{currentfill}%
\pgfsetfillopacity{0.700000}%
\pgfsetlinewidth{0.501875pt}%
\definecolor{currentstroke}{rgb}{1.000000,1.000000,1.000000}%
\pgfsetstrokecolor{currentstroke}%
\pgfsetstrokeopacity{0.700000}%
\pgfsetdash{}{0pt}%
\pgfpathmoveto{\pgfqpoint{2.412346in}{3.887681in}}%
\pgfpathcurveto{\pgfqpoint{2.425368in}{3.887681in}}{\pgfqpoint{2.437860in}{3.892855in}}{\pgfqpoint{2.447068in}{3.902063in}}%
\pgfpathcurveto{\pgfqpoint{2.456276in}{3.911271in}}{\pgfqpoint{2.461450in}{3.923762in}}{\pgfqpoint{2.461450in}{3.936785in}}%
\pgfpathcurveto{\pgfqpoint{2.461450in}{3.949808in}}{\pgfqpoint{2.456276in}{3.962299in}}{\pgfqpoint{2.447068in}{3.971507in}}%
\pgfpathcurveto{\pgfqpoint{2.437860in}{3.980716in}}{\pgfqpoint{2.425368in}{3.985890in}}{\pgfqpoint{2.412346in}{3.985890in}}%
\pgfpathcurveto{\pgfqpoint{2.399323in}{3.985890in}}{\pgfqpoint{2.386832in}{3.980716in}}{\pgfqpoint{2.377624in}{3.971507in}}%
\pgfpathcurveto{\pgfqpoint{2.368415in}{3.962299in}}{\pgfqpoint{2.363241in}{3.949808in}}{\pgfqpoint{2.363241in}{3.936785in}}%
\pgfpathcurveto{\pgfqpoint{2.363241in}{3.923762in}}{\pgfqpoint{2.368415in}{3.911271in}}{\pgfqpoint{2.377624in}{3.902063in}}%
\pgfpathcurveto{\pgfqpoint{2.386832in}{3.892855in}}{\pgfqpoint{2.399323in}{3.887681in}}{\pgfqpoint{2.412346in}{3.887681in}}%
\pgfpathlineto{\pgfqpoint{2.412346in}{3.887681in}}%
\pgfpathclose%
\pgfusepath{stroke,fill}%
\end{pgfscope}%
\begin{pgfscope}%
\pgfpathrectangle{\pgfqpoint{0.786164in}{0.768110in}}{\pgfqpoint{8.851069in}{7.081890in}}%
\pgfusepath{clip}%
\pgfsetbuttcap%
\pgfsetroundjoin%
\definecolor{currentfill}{rgb}{0.282884,0.135920,0.453427}%
\pgfsetfillcolor{currentfill}%
\pgfsetfillopacity{0.700000}%
\pgfsetlinewidth{0.501875pt}%
\definecolor{currentstroke}{rgb}{1.000000,1.000000,1.000000}%
\pgfsetstrokecolor{currentstroke}%
\pgfsetstrokeopacity{0.700000}%
\pgfsetdash{}{0pt}%
\pgfpathmoveto{\pgfqpoint{2.686344in}{3.186937in}}%
\pgfpathcurveto{\pgfqpoint{2.699367in}{3.186937in}}{\pgfqpoint{2.711858in}{3.192111in}}{\pgfqpoint{2.721067in}{3.201319in}}%
\pgfpathcurveto{\pgfqpoint{2.730275in}{3.210528in}}{\pgfqpoint{2.735449in}{3.223019in}}{\pgfqpoint{2.735449in}{3.236042in}}%
\pgfpathcurveto{\pgfqpoint{2.735449in}{3.249064in}}{\pgfqpoint{2.730275in}{3.261555in}}{\pgfqpoint{2.721067in}{3.270764in}}%
\pgfpathcurveto{\pgfqpoint{2.711858in}{3.279972in}}{\pgfqpoint{2.699367in}{3.285146in}}{\pgfqpoint{2.686344in}{3.285146in}}%
\pgfpathcurveto{\pgfqpoint{2.673322in}{3.285146in}}{\pgfqpoint{2.660831in}{3.279972in}}{\pgfqpoint{2.651622in}{3.270764in}}%
\pgfpathcurveto{\pgfqpoint{2.642414in}{3.261555in}}{\pgfqpoint{2.637240in}{3.249064in}}{\pgfqpoint{2.637240in}{3.236042in}}%
\pgfpathcurveto{\pgfqpoint{2.637240in}{3.223019in}}{\pgfqpoint{2.642414in}{3.210528in}}{\pgfqpoint{2.651622in}{3.201319in}}%
\pgfpathcurveto{\pgfqpoint{2.660831in}{3.192111in}}{\pgfqpoint{2.673322in}{3.186937in}}{\pgfqpoint{2.686344in}{3.186937in}}%
\pgfpathlineto{\pgfqpoint{2.686344in}{3.186937in}}%
\pgfpathclose%
\pgfusepath{stroke,fill}%
\end{pgfscope}%
\begin{pgfscope}%
\pgfpathrectangle{\pgfqpoint{0.786164in}{0.768110in}}{\pgfqpoint{8.851069in}{7.081890in}}%
\pgfusepath{clip}%
\pgfsetbuttcap%
\pgfsetroundjoin%
\definecolor{currentfill}{rgb}{0.281412,0.155834,0.469201}%
\pgfsetfillcolor{currentfill}%
\pgfsetfillopacity{0.700000}%
\pgfsetlinewidth{0.501875pt}%
\definecolor{currentstroke}{rgb}{1.000000,1.000000,1.000000}%
\pgfsetstrokecolor{currentstroke}%
\pgfsetstrokeopacity{0.700000}%
\pgfsetdash{}{0pt}%
\pgfpathmoveto{\pgfqpoint{2.722877in}{3.165039in}}%
\pgfpathcurveto{\pgfqpoint{2.735900in}{3.165039in}}{\pgfqpoint{2.748391in}{3.170213in}}{\pgfqpoint{2.757600in}{3.179421in}}%
\pgfpathcurveto{\pgfqpoint{2.766808in}{3.188630in}}{\pgfqpoint{2.771982in}{3.201121in}}{\pgfqpoint{2.771982in}{3.214143in}}%
\pgfpathcurveto{\pgfqpoint{2.771982in}{3.227166in}}{\pgfqpoint{2.766808in}{3.239657in}}{\pgfqpoint{2.757600in}{3.248866in}}%
\pgfpathcurveto{\pgfqpoint{2.748391in}{3.258074in}}{\pgfqpoint{2.735900in}{3.263248in}}{\pgfqpoint{2.722877in}{3.263248in}}%
\pgfpathcurveto{\pgfqpoint{2.709855in}{3.263248in}}{\pgfqpoint{2.697364in}{3.258074in}}{\pgfqpoint{2.688155in}{3.248866in}}%
\pgfpathcurveto{\pgfqpoint{2.678947in}{3.239657in}}{\pgfqpoint{2.673773in}{3.227166in}}{\pgfqpoint{2.673773in}{3.214143in}}%
\pgfpathcurveto{\pgfqpoint{2.673773in}{3.201121in}}{\pgfqpoint{2.678947in}{3.188630in}}{\pgfqpoint{2.688155in}{3.179421in}}%
\pgfpathcurveto{\pgfqpoint{2.697364in}{3.170213in}}{\pgfqpoint{2.709855in}{3.165039in}}{\pgfqpoint{2.722877in}{3.165039in}}%
\pgfpathlineto{\pgfqpoint{2.722877in}{3.165039in}}%
\pgfpathclose%
\pgfusepath{stroke,fill}%
\end{pgfscope}%
\begin{pgfscope}%
\pgfpathrectangle{\pgfqpoint{0.786164in}{0.768110in}}{\pgfqpoint{8.851069in}{7.081890in}}%
\pgfusepath{clip}%
\pgfsetbuttcap%
\pgfsetroundjoin%
\definecolor{currentfill}{rgb}{0.279574,0.170599,0.479997}%
\pgfsetfillcolor{currentfill}%
\pgfsetfillopacity{0.700000}%
\pgfsetlinewidth{0.501875pt}%
\definecolor{currentstroke}{rgb}{1.000000,1.000000,1.000000}%
\pgfsetstrokecolor{currentstroke}%
\pgfsetstrokeopacity{0.700000}%
\pgfsetdash{}{0pt}%
\pgfpathmoveto{\pgfqpoint{2.686344in}{3.121242in}}%
\pgfpathcurveto{\pgfqpoint{2.699367in}{3.121242in}}{\pgfqpoint{2.711858in}{3.126416in}}{\pgfqpoint{2.721067in}{3.135625in}}%
\pgfpathcurveto{\pgfqpoint{2.730275in}{3.144833in}}{\pgfqpoint{2.735449in}{3.157324in}}{\pgfqpoint{2.735449in}{3.170347in}}%
\pgfpathcurveto{\pgfqpoint{2.735449in}{3.183370in}}{\pgfqpoint{2.730275in}{3.195861in}}{\pgfqpoint{2.721067in}{3.205069in}}%
\pgfpathcurveto{\pgfqpoint{2.711858in}{3.214278in}}{\pgfqpoint{2.699367in}{3.219452in}}{\pgfqpoint{2.686344in}{3.219452in}}%
\pgfpathcurveto{\pgfqpoint{2.673322in}{3.219452in}}{\pgfqpoint{2.660831in}{3.214278in}}{\pgfqpoint{2.651622in}{3.205069in}}%
\pgfpathcurveto{\pgfqpoint{2.642414in}{3.195861in}}{\pgfqpoint{2.637240in}{3.183370in}}{\pgfqpoint{2.637240in}{3.170347in}}%
\pgfpathcurveto{\pgfqpoint{2.637240in}{3.157324in}}{\pgfqpoint{2.642414in}{3.144833in}}{\pgfqpoint{2.651622in}{3.135625in}}%
\pgfpathcurveto{\pgfqpoint{2.660831in}{3.126416in}}{\pgfqpoint{2.673322in}{3.121242in}}{\pgfqpoint{2.686344in}{3.121242in}}%
\pgfpathlineto{\pgfqpoint{2.686344in}{3.121242in}}%
\pgfpathclose%
\pgfusepath{stroke,fill}%
\end{pgfscope}%
\begin{pgfscope}%
\pgfpathrectangle{\pgfqpoint{0.786164in}{0.768110in}}{\pgfqpoint{8.851069in}{7.081890in}}%
\pgfusepath{clip}%
\pgfsetbuttcap%
\pgfsetroundjoin%
\definecolor{currentfill}{rgb}{0.274128,0.199721,0.498911}%
\pgfsetfillcolor{currentfill}%
\pgfsetfillopacity{0.700000}%
\pgfsetlinewidth{0.501875pt}%
\definecolor{currentstroke}{rgb}{1.000000,1.000000,1.000000}%
\pgfsetstrokecolor{currentstroke}%
\pgfsetstrokeopacity{0.700000}%
\pgfsetdash{}{0pt}%
\pgfpathmoveto{\pgfqpoint{2.668078in}{3.186937in}}%
\pgfpathcurveto{\pgfqpoint{2.681100in}{3.186937in}}{\pgfqpoint{2.693592in}{3.192111in}}{\pgfqpoint{2.702800in}{3.201319in}}%
\pgfpathcurveto{\pgfqpoint{2.712008in}{3.210528in}}{\pgfqpoint{2.717182in}{3.223019in}}{\pgfqpoint{2.717182in}{3.236042in}}%
\pgfpathcurveto{\pgfqpoint{2.717182in}{3.249064in}}{\pgfqpoint{2.712008in}{3.261555in}}{\pgfqpoint{2.702800in}{3.270764in}}%
\pgfpathcurveto{\pgfqpoint{2.693592in}{3.279972in}}{\pgfqpoint{2.681100in}{3.285146in}}{\pgfqpoint{2.668078in}{3.285146in}}%
\pgfpathcurveto{\pgfqpoint{2.655055in}{3.285146in}}{\pgfqpoint{2.642564in}{3.279972in}}{\pgfqpoint{2.633356in}{3.270764in}}%
\pgfpathcurveto{\pgfqpoint{2.624147in}{3.261555in}}{\pgfqpoint{2.618973in}{3.249064in}}{\pgfqpoint{2.618973in}{3.236042in}}%
\pgfpathcurveto{\pgfqpoint{2.618973in}{3.223019in}}{\pgfqpoint{2.624147in}{3.210528in}}{\pgfqpoint{2.633356in}{3.201319in}}%
\pgfpathcurveto{\pgfqpoint{2.642564in}{3.192111in}}{\pgfqpoint{2.655055in}{3.186937in}}{\pgfqpoint{2.668078in}{3.186937in}}%
\pgfpathlineto{\pgfqpoint{2.668078in}{3.186937in}}%
\pgfpathclose%
\pgfusepath{stroke,fill}%
\end{pgfscope}%
\begin{pgfscope}%
\pgfpathrectangle{\pgfqpoint{0.786164in}{0.768110in}}{\pgfqpoint{8.851069in}{7.081890in}}%
\pgfusepath{clip}%
\pgfsetbuttcap%
\pgfsetroundjoin%
\definecolor{currentfill}{rgb}{0.265145,0.232956,0.516599}%
\pgfsetfillcolor{currentfill}%
\pgfsetfillopacity{0.700000}%
\pgfsetlinewidth{0.501875pt}%
\definecolor{currentstroke}{rgb}{1.000000,1.000000,1.000000}%
\pgfsetstrokecolor{currentstroke}%
\pgfsetstrokeopacity{0.700000}%
\pgfsetdash{}{0pt}%
\pgfpathmoveto{\pgfqpoint{2.604145in}{3.055548in}}%
\pgfpathcurveto{\pgfqpoint{2.617167in}{3.055548in}}{\pgfqpoint{2.629659in}{3.060722in}}{\pgfqpoint{2.638867in}{3.069930in}}%
\pgfpathcurveto{\pgfqpoint{2.648075in}{3.079138in}}{\pgfqpoint{2.653249in}{3.091630in}}{\pgfqpoint{2.653249in}{3.104652in}}%
\pgfpathcurveto{\pgfqpoint{2.653249in}{3.117675in}}{\pgfqpoint{2.648075in}{3.130166in}}{\pgfqpoint{2.638867in}{3.139374in}}%
\pgfpathcurveto{\pgfqpoint{2.629659in}{3.148583in}}{\pgfqpoint{2.617167in}{3.153757in}}{\pgfqpoint{2.604145in}{3.153757in}}%
\pgfpathcurveto{\pgfqpoint{2.591122in}{3.153757in}}{\pgfqpoint{2.578631in}{3.148583in}}{\pgfqpoint{2.569423in}{3.139374in}}%
\pgfpathcurveto{\pgfqpoint{2.560214in}{3.130166in}}{\pgfqpoint{2.555040in}{3.117675in}}{\pgfqpoint{2.555040in}{3.104652in}}%
\pgfpathcurveto{\pgfqpoint{2.555040in}{3.091630in}}{\pgfqpoint{2.560214in}{3.079138in}}{\pgfqpoint{2.569423in}{3.069930in}}%
\pgfpathcurveto{\pgfqpoint{2.578631in}{3.060722in}}{\pgfqpoint{2.591122in}{3.055548in}}{\pgfqpoint{2.604145in}{3.055548in}}%
\pgfpathlineto{\pgfqpoint{2.604145in}{3.055548in}}%
\pgfpathclose%
\pgfusepath{stroke,fill}%
\end{pgfscope}%
\begin{pgfscope}%
\pgfpathrectangle{\pgfqpoint{0.786164in}{0.768110in}}{\pgfqpoint{8.851069in}{7.081890in}}%
\pgfusepath{clip}%
\pgfsetbuttcap%
\pgfsetroundjoin%
\definecolor{currentfill}{rgb}{0.257322,0.256130,0.526563}%
\pgfsetfillcolor{currentfill}%
\pgfsetfillopacity{0.700000}%
\pgfsetlinewidth{0.501875pt}%
\definecolor{currentstroke}{rgb}{1.000000,1.000000,1.000000}%
\pgfsetstrokecolor{currentstroke}%
\pgfsetstrokeopacity{0.700000}%
\pgfsetdash{}{0pt}%
\pgfpathmoveto{\pgfqpoint{2.412346in}{2.814667in}}%
\pgfpathcurveto{\pgfqpoint{2.425368in}{2.814667in}}{\pgfqpoint{2.437860in}{2.819841in}}{\pgfqpoint{2.447068in}{2.829049in}}%
\pgfpathcurveto{\pgfqpoint{2.456276in}{2.838258in}}{\pgfqpoint{2.461450in}{2.850749in}}{\pgfqpoint{2.461450in}{2.863772in}}%
\pgfpathcurveto{\pgfqpoint{2.461450in}{2.876794in}}{\pgfqpoint{2.456276in}{2.889285in}}{\pgfqpoint{2.447068in}{2.898494in}}%
\pgfpathcurveto{\pgfqpoint{2.437860in}{2.907702in}}{\pgfqpoint{2.425368in}{2.912876in}}{\pgfqpoint{2.412346in}{2.912876in}}%
\pgfpathcurveto{\pgfqpoint{2.399323in}{2.912876in}}{\pgfqpoint{2.386832in}{2.907702in}}{\pgfqpoint{2.377624in}{2.898494in}}%
\pgfpathcurveto{\pgfqpoint{2.368415in}{2.889285in}}{\pgfqpoint{2.363241in}{2.876794in}}{\pgfqpoint{2.363241in}{2.863772in}}%
\pgfpathcurveto{\pgfqpoint{2.363241in}{2.850749in}}{\pgfqpoint{2.368415in}{2.838258in}}{\pgfqpoint{2.377624in}{2.829049in}}%
\pgfpathcurveto{\pgfqpoint{2.386832in}{2.819841in}}{\pgfqpoint{2.399323in}{2.814667in}}{\pgfqpoint{2.412346in}{2.814667in}}%
\pgfpathlineto{\pgfqpoint{2.412346in}{2.814667in}}%
\pgfpathclose%
\pgfusepath{stroke,fill}%
\end{pgfscope}%
\begin{pgfscope}%
\pgfpathrectangle{\pgfqpoint{0.786164in}{0.768110in}}{\pgfqpoint{8.851069in}{7.081890in}}%
\pgfusepath{clip}%
\pgfsetbuttcap%
\pgfsetroundjoin%
\definecolor{currentfill}{rgb}{0.255645,0.260703,0.528312}%
\pgfsetfillcolor{currentfill}%
\pgfsetfillopacity{0.700000}%
\pgfsetlinewidth{0.501875pt}%
\definecolor{currentstroke}{rgb}{1.000000,1.000000,1.000000}%
\pgfsetstrokecolor{currentstroke}%
\pgfsetstrokeopacity{0.700000}%
\pgfsetdash{}{0pt}%
\pgfpathmoveto{\pgfqpoint{2.448879in}{2.836565in}}%
\pgfpathcurveto{\pgfqpoint{2.461902in}{2.836565in}}{\pgfqpoint{2.474393in}{2.841739in}}{\pgfqpoint{2.483601in}{2.850948in}}%
\pgfpathcurveto{\pgfqpoint{2.492810in}{2.860156in}}{\pgfqpoint{2.497984in}{2.872647in}}{\pgfqpoint{2.497984in}{2.885670in}}%
\pgfpathcurveto{\pgfqpoint{2.497984in}{2.898693in}}{\pgfqpoint{2.492810in}{2.911184in}}{\pgfqpoint{2.483601in}{2.920392in}}%
\pgfpathcurveto{\pgfqpoint{2.474393in}{2.929601in}}{\pgfqpoint{2.461902in}{2.934774in}}{\pgfqpoint{2.448879in}{2.934774in}}%
\pgfpathcurveto{\pgfqpoint{2.435856in}{2.934774in}}{\pgfqpoint{2.423365in}{2.929601in}}{\pgfqpoint{2.414157in}{2.920392in}}%
\pgfpathcurveto{\pgfqpoint{2.404948in}{2.911184in}}{\pgfqpoint{2.399774in}{2.898693in}}{\pgfqpoint{2.399774in}{2.885670in}}%
\pgfpathcurveto{\pgfqpoint{2.399774in}{2.872647in}}{\pgfqpoint{2.404948in}{2.860156in}}{\pgfqpoint{2.414157in}{2.850948in}}%
\pgfpathcurveto{\pgfqpoint{2.423365in}{2.841739in}}{\pgfqpoint{2.435856in}{2.836565in}}{\pgfqpoint{2.448879in}{2.836565in}}%
\pgfpathlineto{\pgfqpoint{2.448879in}{2.836565in}}%
\pgfpathclose%
\pgfusepath{stroke,fill}%
\end{pgfscope}%
\begin{pgfscope}%
\pgfpathrectangle{\pgfqpoint{0.786164in}{0.768110in}}{\pgfqpoint{8.851069in}{7.081890in}}%
\pgfusepath{clip}%
\pgfsetbuttcap%
\pgfsetroundjoin%
\definecolor{currentfill}{rgb}{0.257322,0.256130,0.526563}%
\pgfsetfillcolor{currentfill}%
\pgfsetfillopacity{0.700000}%
\pgfsetlinewidth{0.501875pt}%
\definecolor{currentstroke}{rgb}{1.000000,1.000000,1.000000}%
\pgfsetstrokecolor{currentstroke}%
\pgfsetstrokeopacity{0.700000}%
\pgfsetdash{}{0pt}%
\pgfpathmoveto{\pgfqpoint{2.366679in}{2.814667in}}%
\pgfpathcurveto{\pgfqpoint{2.379702in}{2.814667in}}{\pgfqpoint{2.392193in}{2.819841in}}{\pgfqpoint{2.401402in}{2.829049in}}%
\pgfpathcurveto{\pgfqpoint{2.410610in}{2.838258in}}{\pgfqpoint{2.415784in}{2.850749in}}{\pgfqpoint{2.415784in}{2.863772in}}%
\pgfpathcurveto{\pgfqpoint{2.415784in}{2.876794in}}{\pgfqpoint{2.410610in}{2.889285in}}{\pgfqpoint{2.401402in}{2.898494in}}%
\pgfpathcurveto{\pgfqpoint{2.392193in}{2.907702in}}{\pgfqpoint{2.379702in}{2.912876in}}{\pgfqpoint{2.366679in}{2.912876in}}%
\pgfpathcurveto{\pgfqpoint{2.353657in}{2.912876in}}{\pgfqpoint{2.341166in}{2.907702in}}{\pgfqpoint{2.331957in}{2.898494in}}%
\pgfpathcurveto{\pgfqpoint{2.322749in}{2.889285in}}{\pgfqpoint{2.317575in}{2.876794in}}{\pgfqpoint{2.317575in}{2.863772in}}%
\pgfpathcurveto{\pgfqpoint{2.317575in}{2.850749in}}{\pgfqpoint{2.322749in}{2.838258in}}{\pgfqpoint{2.331957in}{2.829049in}}%
\pgfpathcurveto{\pgfqpoint{2.341166in}{2.819841in}}{\pgfqpoint{2.353657in}{2.814667in}}{\pgfqpoint{2.366679in}{2.814667in}}%
\pgfpathlineto{\pgfqpoint{2.366679in}{2.814667in}}%
\pgfpathclose%
\pgfusepath{stroke,fill}%
\end{pgfscope}%
\begin{pgfscope}%
\pgfpathrectangle{\pgfqpoint{0.786164in}{0.768110in}}{\pgfqpoint{8.851069in}{7.081890in}}%
\pgfusepath{clip}%
\pgfsetbuttcap%
\pgfsetroundjoin%
\definecolor{currentfill}{rgb}{0.239346,0.300855,0.540844}%
\pgfsetfillcolor{currentfill}%
\pgfsetfillopacity{0.700000}%
\pgfsetlinewidth{0.501875pt}%
\definecolor{currentstroke}{rgb}{1.000000,1.000000,1.000000}%
\pgfsetstrokecolor{currentstroke}%
\pgfsetstrokeopacity{0.700000}%
\pgfsetdash{}{0pt}%
\pgfpathmoveto{\pgfqpoint{2.275346in}{2.770871in}}%
\pgfpathcurveto{\pgfqpoint{2.288369in}{2.770871in}}{\pgfqpoint{2.300860in}{2.776044in}}{\pgfqpoint{2.310069in}{2.785253in}}%
\pgfpathcurveto{\pgfqpoint{2.319277in}{2.794461in}}{\pgfqpoint{2.324451in}{2.806952in}}{\pgfqpoint{2.324451in}{2.819975in}}%
\pgfpathcurveto{\pgfqpoint{2.324451in}{2.832998in}}{\pgfqpoint{2.319277in}{2.845489in}}{\pgfqpoint{2.310069in}{2.854697in}}%
\pgfpathcurveto{\pgfqpoint{2.300860in}{2.863906in}}{\pgfqpoint{2.288369in}{2.869080in}}{\pgfqpoint{2.275346in}{2.869080in}}%
\pgfpathcurveto{\pgfqpoint{2.262324in}{2.869080in}}{\pgfqpoint{2.249833in}{2.863906in}}{\pgfqpoint{2.240624in}{2.854697in}}%
\pgfpathcurveto{\pgfqpoint{2.231416in}{2.845489in}}{\pgfqpoint{2.226242in}{2.832998in}}{\pgfqpoint{2.226242in}{2.819975in}}%
\pgfpathcurveto{\pgfqpoint{2.226242in}{2.806952in}}{\pgfqpoint{2.231416in}{2.794461in}}{\pgfqpoint{2.240624in}{2.785253in}}%
\pgfpathcurveto{\pgfqpoint{2.249833in}{2.776044in}}{\pgfqpoint{2.262324in}{2.770871in}}{\pgfqpoint{2.275346in}{2.770871in}}%
\pgfpathlineto{\pgfqpoint{2.275346in}{2.770871in}}%
\pgfpathclose%
\pgfusepath{stroke,fill}%
\end{pgfscope}%
\begin{pgfscope}%
\pgfpathrectangle{\pgfqpoint{0.786164in}{0.768110in}}{\pgfqpoint{8.851069in}{7.081890in}}%
\pgfusepath{clip}%
\pgfsetbuttcap%
\pgfsetroundjoin%
\definecolor{currentfill}{rgb}{0.229739,0.322361,0.545706}%
\pgfsetfillcolor{currentfill}%
\pgfsetfillopacity{0.700000}%
\pgfsetlinewidth{0.501875pt}%
\definecolor{currentstroke}{rgb}{1.000000,1.000000,1.000000}%
\pgfsetstrokecolor{currentstroke}%
\pgfsetstrokeopacity{0.700000}%
\pgfsetdash{}{0pt}%
\pgfpathmoveto{\pgfqpoint{2.174880in}{2.551888in}}%
\pgfpathcurveto{\pgfqpoint{2.187903in}{2.551888in}}{\pgfqpoint{2.200394in}{2.557062in}}{\pgfqpoint{2.209602in}{2.566271in}}%
\pgfpathcurveto{\pgfqpoint{2.218811in}{2.575479in}}{\pgfqpoint{2.223985in}{2.587970in}}{\pgfqpoint{2.223985in}{2.600993in}}%
\pgfpathcurveto{\pgfqpoint{2.223985in}{2.614015in}}{\pgfqpoint{2.218811in}{2.626507in}}{\pgfqpoint{2.209602in}{2.635715in}}%
\pgfpathcurveto{\pgfqpoint{2.200394in}{2.644923in}}{\pgfqpoint{2.187903in}{2.650097in}}{\pgfqpoint{2.174880in}{2.650097in}}%
\pgfpathcurveto{\pgfqpoint{2.161858in}{2.650097in}}{\pgfqpoint{2.149366in}{2.644923in}}{\pgfqpoint{2.140158in}{2.635715in}}%
\pgfpathcurveto{\pgfqpoint{2.130950in}{2.626507in}}{\pgfqpoint{2.125776in}{2.614015in}}{\pgfqpoint{2.125776in}{2.600993in}}%
\pgfpathcurveto{\pgfqpoint{2.125776in}{2.587970in}}{\pgfqpoint{2.130950in}{2.575479in}}{\pgfqpoint{2.140158in}{2.566271in}}%
\pgfpathcurveto{\pgfqpoint{2.149366in}{2.557062in}}{\pgfqpoint{2.161858in}{2.551888in}}{\pgfqpoint{2.174880in}{2.551888in}}%
\pgfpathlineto{\pgfqpoint{2.174880in}{2.551888in}}%
\pgfpathclose%
\pgfusepath{stroke,fill}%
\end{pgfscope}%
\begin{pgfscope}%
\pgfpathrectangle{\pgfqpoint{0.786164in}{0.768110in}}{\pgfqpoint{8.851069in}{7.081890in}}%
\pgfusepath{clip}%
\pgfsetbuttcap%
\pgfsetroundjoin%
\definecolor{currentfill}{rgb}{0.227802,0.326594,0.546532}%
\pgfsetfillcolor{currentfill}%
\pgfsetfillopacity{0.700000}%
\pgfsetlinewidth{0.501875pt}%
\definecolor{currentstroke}{rgb}{1.000000,1.000000,1.000000}%
\pgfsetstrokecolor{currentstroke}%
\pgfsetstrokeopacity{0.700000}%
\pgfsetdash{}{0pt}%
\pgfpathmoveto{\pgfqpoint{2.019614in}{2.464295in}}%
\pgfpathcurveto{\pgfqpoint{2.032637in}{2.464295in}}{\pgfqpoint{2.045128in}{2.469469in}}{\pgfqpoint{2.054337in}{2.478678in}}%
\pgfpathcurveto{\pgfqpoint{2.063545in}{2.487886in}}{\pgfqpoint{2.068719in}{2.500377in}}{\pgfqpoint{2.068719in}{2.513400in}}%
\pgfpathcurveto{\pgfqpoint{2.068719in}{2.526423in}}{\pgfqpoint{2.063545in}{2.538914in}}{\pgfqpoint{2.054337in}{2.548122in}}%
\pgfpathcurveto{\pgfqpoint{2.045128in}{2.557331in}}{\pgfqpoint{2.032637in}{2.562504in}}{\pgfqpoint{2.019614in}{2.562504in}}%
\pgfpathcurveto{\pgfqpoint{2.006592in}{2.562504in}}{\pgfqpoint{1.994101in}{2.557331in}}{\pgfqpoint{1.984892in}{2.548122in}}%
\pgfpathcurveto{\pgfqpoint{1.975684in}{2.538914in}}{\pgfqpoint{1.970510in}{2.526423in}}{\pgfqpoint{1.970510in}{2.513400in}}%
\pgfpathcurveto{\pgfqpoint{1.970510in}{2.500377in}}{\pgfqpoint{1.975684in}{2.487886in}}{\pgfqpoint{1.984892in}{2.478678in}}%
\pgfpathcurveto{\pgfqpoint{1.994101in}{2.469469in}}{\pgfqpoint{2.006592in}{2.464295in}}{\pgfqpoint{2.019614in}{2.464295in}}%
\pgfpathlineto{\pgfqpoint{2.019614in}{2.464295in}}%
\pgfpathclose%
\pgfusepath{stroke,fill}%
\end{pgfscope}%
\begin{pgfscope}%
\pgfpathrectangle{\pgfqpoint{0.786164in}{0.768110in}}{\pgfqpoint{8.851069in}{7.081890in}}%
\pgfusepath{clip}%
\pgfsetbuttcap%
\pgfsetroundjoin%
\definecolor{currentfill}{rgb}{0.223925,0.334994,0.548053}%
\pgfsetfillcolor{currentfill}%
\pgfsetfillopacity{0.700000}%
\pgfsetlinewidth{0.501875pt}%
\definecolor{currentstroke}{rgb}{1.000000,1.000000,1.000000}%
\pgfsetstrokecolor{currentstroke}%
\pgfsetstrokeopacity{0.700000}%
\pgfsetdash{}{0pt}%
\pgfpathmoveto{\pgfqpoint{2.120081in}{2.508092in}}%
\pgfpathcurveto{\pgfqpoint{2.133103in}{2.508092in}}{\pgfqpoint{2.145594in}{2.513266in}}{\pgfqpoint{2.154803in}{2.522474in}}%
\pgfpathcurveto{\pgfqpoint{2.164011in}{2.531683in}}{\pgfqpoint{2.169185in}{2.544174in}}{\pgfqpoint{2.169185in}{2.557196in}}%
\pgfpathcurveto{\pgfqpoint{2.169185in}{2.570219in}}{\pgfqpoint{2.164011in}{2.582710in}}{\pgfqpoint{2.154803in}{2.591919in}}%
\pgfpathcurveto{\pgfqpoint{2.145594in}{2.601127in}}{\pgfqpoint{2.133103in}{2.606301in}}{\pgfqpoint{2.120081in}{2.606301in}}%
\pgfpathcurveto{\pgfqpoint{2.107058in}{2.606301in}}{\pgfqpoint{2.094567in}{2.601127in}}{\pgfqpoint{2.085358in}{2.591919in}}%
\pgfpathcurveto{\pgfqpoint{2.076150in}{2.582710in}}{\pgfqpoint{2.070976in}{2.570219in}}{\pgfqpoint{2.070976in}{2.557196in}}%
\pgfpathcurveto{\pgfqpoint{2.070976in}{2.544174in}}{\pgfqpoint{2.076150in}{2.531683in}}{\pgfqpoint{2.085358in}{2.522474in}}%
\pgfpathcurveto{\pgfqpoint{2.094567in}{2.513266in}}{\pgfqpoint{2.107058in}{2.508092in}}{\pgfqpoint{2.120081in}{2.508092in}}%
\pgfpathlineto{\pgfqpoint{2.120081in}{2.508092in}}%
\pgfpathclose%
\pgfusepath{stroke,fill}%
\end{pgfscope}%
\begin{pgfscope}%
\pgfpathrectangle{\pgfqpoint{0.786164in}{0.768110in}}{\pgfqpoint{8.851069in}{7.081890in}}%
\pgfusepath{clip}%
\pgfsetbuttcap%
\pgfsetroundjoin%
\definecolor{currentfill}{rgb}{0.223925,0.334994,0.548053}%
\pgfsetfillcolor{currentfill}%
\pgfsetfillopacity{0.700000}%
\pgfsetlinewidth{0.501875pt}%
\definecolor{currentstroke}{rgb}{1.000000,1.000000,1.000000}%
\pgfsetstrokecolor{currentstroke}%
\pgfsetstrokeopacity{0.700000}%
\pgfsetdash{}{0pt}%
\pgfpathmoveto{\pgfqpoint{2.101814in}{2.486193in}}%
\pgfpathcurveto{\pgfqpoint{2.114837in}{2.486193in}}{\pgfqpoint{2.127328in}{2.491367in}}{\pgfqpoint{2.136536in}{2.500576in}}%
\pgfpathcurveto{\pgfqpoint{2.145745in}{2.509784in}}{\pgfqpoint{2.150919in}{2.522275in}}{\pgfqpoint{2.150919in}{2.535298in}}%
\pgfpathcurveto{\pgfqpoint{2.150919in}{2.548321in}}{\pgfqpoint{2.145745in}{2.560812in}}{\pgfqpoint{2.136536in}{2.570020in}}%
\pgfpathcurveto{\pgfqpoint{2.127328in}{2.579229in}}{\pgfqpoint{2.114837in}{2.584403in}}{\pgfqpoint{2.101814in}{2.584403in}}%
\pgfpathcurveto{\pgfqpoint{2.088791in}{2.584403in}}{\pgfqpoint{2.076300in}{2.579229in}}{\pgfqpoint{2.067092in}{2.570020in}}%
\pgfpathcurveto{\pgfqpoint{2.057883in}{2.560812in}}{\pgfqpoint{2.052709in}{2.548321in}}{\pgfqpoint{2.052709in}{2.535298in}}%
\pgfpathcurveto{\pgfqpoint{2.052709in}{2.522275in}}{\pgfqpoint{2.057883in}{2.509784in}}{\pgfqpoint{2.067092in}{2.500576in}}%
\pgfpathcurveto{\pgfqpoint{2.076300in}{2.491367in}}{\pgfqpoint{2.088791in}{2.486193in}}{\pgfqpoint{2.101814in}{2.486193in}}%
\pgfpathlineto{\pgfqpoint{2.101814in}{2.486193in}}%
\pgfpathclose%
\pgfusepath{stroke,fill}%
\end{pgfscope}%
\begin{pgfscope}%
\pgfpathrectangle{\pgfqpoint{0.786164in}{0.768110in}}{\pgfqpoint{8.851069in}{7.081890in}}%
\pgfusepath{clip}%
\pgfsetbuttcap%
\pgfsetroundjoin%
\definecolor{currentfill}{rgb}{0.218130,0.347432,0.550038}%
\pgfsetfillcolor{currentfill}%
\pgfsetfillopacity{0.700000}%
\pgfsetlinewidth{0.501875pt}%
\definecolor{currentstroke}{rgb}{1.000000,1.000000,1.000000}%
\pgfsetstrokecolor{currentstroke}%
\pgfsetstrokeopacity{0.700000}%
\pgfsetdash{}{0pt}%
\pgfpathmoveto{\pgfqpoint{2.120081in}{2.551888in}}%
\pgfpathcurveto{\pgfqpoint{2.133103in}{2.551888in}}{\pgfqpoint{2.145594in}{2.557062in}}{\pgfqpoint{2.154803in}{2.566271in}}%
\pgfpathcurveto{\pgfqpoint{2.164011in}{2.575479in}}{\pgfqpoint{2.169185in}{2.587970in}}{\pgfqpoint{2.169185in}{2.600993in}}%
\pgfpathcurveto{\pgfqpoint{2.169185in}{2.614015in}}{\pgfqpoint{2.164011in}{2.626507in}}{\pgfqpoint{2.154803in}{2.635715in}}%
\pgfpathcurveto{\pgfqpoint{2.145594in}{2.644923in}}{\pgfqpoint{2.133103in}{2.650097in}}{\pgfqpoint{2.120081in}{2.650097in}}%
\pgfpathcurveto{\pgfqpoint{2.107058in}{2.650097in}}{\pgfqpoint{2.094567in}{2.644923in}}{\pgfqpoint{2.085358in}{2.635715in}}%
\pgfpathcurveto{\pgfqpoint{2.076150in}{2.626507in}}{\pgfqpoint{2.070976in}{2.614015in}}{\pgfqpoint{2.070976in}{2.600993in}}%
\pgfpathcurveto{\pgfqpoint{2.070976in}{2.587970in}}{\pgfqpoint{2.076150in}{2.575479in}}{\pgfqpoint{2.085358in}{2.566271in}}%
\pgfpathcurveto{\pgfqpoint{2.094567in}{2.557062in}}{\pgfqpoint{2.107058in}{2.551888in}}{\pgfqpoint{2.120081in}{2.551888in}}%
\pgfpathlineto{\pgfqpoint{2.120081in}{2.551888in}}%
\pgfpathclose%
\pgfusepath{stroke,fill}%
\end{pgfscope}%
\begin{pgfscope}%
\pgfpathrectangle{\pgfqpoint{0.786164in}{0.768110in}}{\pgfqpoint{8.851069in}{7.081890in}}%
\pgfusepath{clip}%
\pgfsetbuttcap%
\pgfsetroundjoin%
\definecolor{currentfill}{rgb}{0.221989,0.339161,0.548752}%
\pgfsetfillcolor{currentfill}%
\pgfsetfillopacity{0.700000}%
\pgfsetlinewidth{0.501875pt}%
\definecolor{currentstroke}{rgb}{1.000000,1.000000,1.000000}%
\pgfsetstrokecolor{currentstroke}%
\pgfsetstrokeopacity{0.700000}%
\pgfsetdash{}{0pt}%
\pgfpathmoveto{\pgfqpoint{2.101814in}{2.464295in}}%
\pgfpathcurveto{\pgfqpoint{2.114837in}{2.464295in}}{\pgfqpoint{2.127328in}{2.469469in}}{\pgfqpoint{2.136536in}{2.478678in}}%
\pgfpathcurveto{\pgfqpoint{2.145745in}{2.487886in}}{\pgfqpoint{2.150919in}{2.500377in}}{\pgfqpoint{2.150919in}{2.513400in}}%
\pgfpathcurveto{\pgfqpoint{2.150919in}{2.526423in}}{\pgfqpoint{2.145745in}{2.538914in}}{\pgfqpoint{2.136536in}{2.548122in}}%
\pgfpathcurveto{\pgfqpoint{2.127328in}{2.557331in}}{\pgfqpoint{2.114837in}{2.562504in}}{\pgfqpoint{2.101814in}{2.562504in}}%
\pgfpathcurveto{\pgfqpoint{2.088791in}{2.562504in}}{\pgfqpoint{2.076300in}{2.557331in}}{\pgfqpoint{2.067092in}{2.548122in}}%
\pgfpathcurveto{\pgfqpoint{2.057883in}{2.538914in}}{\pgfqpoint{2.052709in}{2.526423in}}{\pgfqpoint{2.052709in}{2.513400in}}%
\pgfpathcurveto{\pgfqpoint{2.052709in}{2.500377in}}{\pgfqpoint{2.057883in}{2.487886in}}{\pgfqpoint{2.067092in}{2.478678in}}%
\pgfpathcurveto{\pgfqpoint{2.076300in}{2.469469in}}{\pgfqpoint{2.088791in}{2.464295in}}{\pgfqpoint{2.101814in}{2.464295in}}%
\pgfpathlineto{\pgfqpoint{2.101814in}{2.464295in}}%
\pgfpathclose%
\pgfusepath{stroke,fill}%
\end{pgfscope}%
\begin{pgfscope}%
\pgfpathrectangle{\pgfqpoint{0.786164in}{0.768110in}}{\pgfqpoint{8.851069in}{7.081890in}}%
\pgfusepath{clip}%
\pgfsetbuttcap%
\pgfsetroundjoin%
\definecolor{currentfill}{rgb}{0.218130,0.347432,0.550038}%
\pgfsetfillcolor{currentfill}%
\pgfsetfillopacity{0.700000}%
\pgfsetlinewidth{0.501875pt}%
\definecolor{currentstroke}{rgb}{1.000000,1.000000,1.000000}%
\pgfsetstrokecolor{currentstroke}%
\pgfsetstrokeopacity{0.700000}%
\pgfsetdash{}{0pt}%
\pgfpathmoveto{\pgfqpoint{2.101814in}{2.464295in}}%
\pgfpathcurveto{\pgfqpoint{2.114837in}{2.464295in}}{\pgfqpoint{2.127328in}{2.469469in}}{\pgfqpoint{2.136536in}{2.478678in}}%
\pgfpathcurveto{\pgfqpoint{2.145745in}{2.487886in}}{\pgfqpoint{2.150919in}{2.500377in}}{\pgfqpoint{2.150919in}{2.513400in}}%
\pgfpathcurveto{\pgfqpoint{2.150919in}{2.526423in}}{\pgfqpoint{2.145745in}{2.538914in}}{\pgfqpoint{2.136536in}{2.548122in}}%
\pgfpathcurveto{\pgfqpoint{2.127328in}{2.557331in}}{\pgfqpoint{2.114837in}{2.562504in}}{\pgfqpoint{2.101814in}{2.562504in}}%
\pgfpathcurveto{\pgfqpoint{2.088791in}{2.562504in}}{\pgfqpoint{2.076300in}{2.557331in}}{\pgfqpoint{2.067092in}{2.548122in}}%
\pgfpathcurveto{\pgfqpoint{2.057883in}{2.538914in}}{\pgfqpoint{2.052709in}{2.526423in}}{\pgfqpoint{2.052709in}{2.513400in}}%
\pgfpathcurveto{\pgfqpoint{2.052709in}{2.500377in}}{\pgfqpoint{2.057883in}{2.487886in}}{\pgfqpoint{2.067092in}{2.478678in}}%
\pgfpathcurveto{\pgfqpoint{2.076300in}{2.469469in}}{\pgfqpoint{2.088791in}{2.464295in}}{\pgfqpoint{2.101814in}{2.464295in}}%
\pgfpathlineto{\pgfqpoint{2.101814in}{2.464295in}}%
\pgfpathclose%
\pgfusepath{stroke,fill}%
\end{pgfscope}%
\begin{pgfscope}%
\pgfpathrectangle{\pgfqpoint{0.786164in}{0.768110in}}{\pgfqpoint{8.851069in}{7.081890in}}%
\pgfusepath{clip}%
\pgfsetbuttcap%
\pgfsetroundjoin%
\definecolor{currentfill}{rgb}{0.203063,0.379716,0.553925}%
\pgfsetfillcolor{currentfill}%
\pgfsetfillopacity{0.700000}%
\pgfsetlinewidth{0.501875pt}%
\definecolor{currentstroke}{rgb}{1.000000,1.000000,1.000000}%
\pgfsetstrokecolor{currentstroke}%
\pgfsetstrokeopacity{0.700000}%
\pgfsetdash{}{0pt}%
\pgfpathmoveto{\pgfqpoint{1.910015in}{2.332906in}}%
\pgfpathcurveto{\pgfqpoint{1.923038in}{2.332906in}}{\pgfqpoint{1.935529in}{2.338080in}}{\pgfqpoint{1.944737in}{2.347288in}}%
\pgfpathcurveto{\pgfqpoint{1.953946in}{2.356497in}}{\pgfqpoint{1.959120in}{2.368988in}}{\pgfqpoint{1.959120in}{2.382010in}}%
\pgfpathcurveto{\pgfqpoint{1.959120in}{2.395033in}}{\pgfqpoint{1.953946in}{2.407524in}}{\pgfqpoint{1.944737in}{2.416733in}}%
\pgfpathcurveto{\pgfqpoint{1.935529in}{2.425941in}}{\pgfqpoint{1.923038in}{2.431115in}}{\pgfqpoint{1.910015in}{2.431115in}}%
\pgfpathcurveto{\pgfqpoint{1.896992in}{2.431115in}}{\pgfqpoint{1.884501in}{2.425941in}}{\pgfqpoint{1.875293in}{2.416733in}}%
\pgfpathcurveto{\pgfqpoint{1.866084in}{2.407524in}}{\pgfqpoint{1.860910in}{2.395033in}}{\pgfqpoint{1.860910in}{2.382010in}}%
\pgfpathcurveto{\pgfqpoint{1.860910in}{2.368988in}}{\pgfqpoint{1.866084in}{2.356497in}}{\pgfqpoint{1.875293in}{2.347288in}}%
\pgfpathcurveto{\pgfqpoint{1.884501in}{2.338080in}}{\pgfqpoint{1.896992in}{2.332906in}}{\pgfqpoint{1.910015in}{2.332906in}}%
\pgfpathlineto{\pgfqpoint{1.910015in}{2.332906in}}%
\pgfpathclose%
\pgfusepath{stroke,fill}%
\end{pgfscope}%
\begin{pgfscope}%
\pgfpathrectangle{\pgfqpoint{0.786164in}{0.768110in}}{\pgfqpoint{8.851069in}{7.081890in}}%
\pgfusepath{clip}%
\pgfsetbuttcap%
\pgfsetroundjoin%
\definecolor{currentfill}{rgb}{0.214298,0.355619,0.551184}%
\pgfsetfillcolor{currentfill}%
\pgfsetfillopacity{0.700000}%
\pgfsetlinewidth{0.501875pt}%
\definecolor{currentstroke}{rgb}{1.000000,1.000000,1.000000}%
\pgfsetstrokecolor{currentstroke}%
\pgfsetstrokeopacity{0.700000}%
\pgfsetdash{}{0pt}%
\pgfpathmoveto{\pgfqpoint{2.120081in}{2.486193in}}%
\pgfpathcurveto{\pgfqpoint{2.133103in}{2.486193in}}{\pgfqpoint{2.145594in}{2.491367in}}{\pgfqpoint{2.154803in}{2.500576in}}%
\pgfpathcurveto{\pgfqpoint{2.164011in}{2.509784in}}{\pgfqpoint{2.169185in}{2.522275in}}{\pgfqpoint{2.169185in}{2.535298in}}%
\pgfpathcurveto{\pgfqpoint{2.169185in}{2.548321in}}{\pgfqpoint{2.164011in}{2.560812in}}{\pgfqpoint{2.154803in}{2.570020in}}%
\pgfpathcurveto{\pgfqpoint{2.145594in}{2.579229in}}{\pgfqpoint{2.133103in}{2.584403in}}{\pgfqpoint{2.120081in}{2.584403in}}%
\pgfpathcurveto{\pgfqpoint{2.107058in}{2.584403in}}{\pgfqpoint{2.094567in}{2.579229in}}{\pgfqpoint{2.085358in}{2.570020in}}%
\pgfpathcurveto{\pgfqpoint{2.076150in}{2.560812in}}{\pgfqpoint{2.070976in}{2.548321in}}{\pgfqpoint{2.070976in}{2.535298in}}%
\pgfpathcurveto{\pgfqpoint{2.070976in}{2.522275in}}{\pgfqpoint{2.076150in}{2.509784in}}{\pgfqpoint{2.085358in}{2.500576in}}%
\pgfpathcurveto{\pgfqpoint{2.094567in}{2.491367in}}{\pgfqpoint{2.107058in}{2.486193in}}{\pgfqpoint{2.120081in}{2.486193in}}%
\pgfpathlineto{\pgfqpoint{2.120081in}{2.486193in}}%
\pgfpathclose%
\pgfusepath{stroke,fill}%
\end{pgfscope}%
\begin{pgfscope}%
\pgfpathrectangle{\pgfqpoint{0.786164in}{0.768110in}}{\pgfqpoint{8.851069in}{7.081890in}}%
\pgfusepath{clip}%
\pgfsetbuttcap%
\pgfsetroundjoin%
\definecolor{currentfill}{rgb}{0.212395,0.359683,0.551710}%
\pgfsetfillcolor{currentfill}%
\pgfsetfillopacity{0.700000}%
\pgfsetlinewidth{0.501875pt}%
\definecolor{currentstroke}{rgb}{1.000000,1.000000,1.000000}%
\pgfsetstrokecolor{currentstroke}%
\pgfsetstrokeopacity{0.700000}%
\pgfsetdash{}{0pt}%
\pgfpathmoveto{\pgfqpoint{2.028748in}{2.486193in}}%
\pgfpathcurveto{\pgfqpoint{2.041770in}{2.486193in}}{\pgfqpoint{2.054261in}{2.491367in}}{\pgfqpoint{2.063470in}{2.500576in}}%
\pgfpathcurveto{\pgfqpoint{2.072678in}{2.509784in}}{\pgfqpoint{2.077852in}{2.522275in}}{\pgfqpoint{2.077852in}{2.535298in}}%
\pgfpathcurveto{\pgfqpoint{2.077852in}{2.548321in}}{\pgfqpoint{2.072678in}{2.560812in}}{\pgfqpoint{2.063470in}{2.570020in}}%
\pgfpathcurveto{\pgfqpoint{2.054261in}{2.579229in}}{\pgfqpoint{2.041770in}{2.584403in}}{\pgfqpoint{2.028748in}{2.584403in}}%
\pgfpathcurveto{\pgfqpoint{2.015725in}{2.584403in}}{\pgfqpoint{2.003234in}{2.579229in}}{\pgfqpoint{1.994025in}{2.570020in}}%
\pgfpathcurveto{\pgfqpoint{1.984817in}{2.560812in}}{\pgfqpoint{1.979643in}{2.548321in}}{\pgfqpoint{1.979643in}{2.535298in}}%
\pgfpathcurveto{\pgfqpoint{1.979643in}{2.522275in}}{\pgfqpoint{1.984817in}{2.509784in}}{\pgfqpoint{1.994025in}{2.500576in}}%
\pgfpathcurveto{\pgfqpoint{2.003234in}{2.491367in}}{\pgfqpoint{2.015725in}{2.486193in}}{\pgfqpoint{2.028748in}{2.486193in}}%
\pgfpathlineto{\pgfqpoint{2.028748in}{2.486193in}}%
\pgfpathclose%
\pgfusepath{stroke,fill}%
\end{pgfscope}%
\begin{pgfscope}%
\pgfpathrectangle{\pgfqpoint{0.786164in}{0.768110in}}{\pgfqpoint{8.851069in}{7.081890in}}%
\pgfusepath{clip}%
\pgfsetbuttcap%
\pgfsetroundjoin%
\definecolor{currentfill}{rgb}{0.225863,0.330805,0.547314}%
\pgfsetfillcolor{currentfill}%
\pgfsetfillopacity{0.700000}%
\pgfsetlinewidth{0.501875pt}%
\definecolor{currentstroke}{rgb}{1.000000,1.000000,1.000000}%
\pgfsetstrokecolor{currentstroke}%
\pgfsetstrokeopacity{0.700000}%
\pgfsetdash{}{0pt}%
\pgfpathmoveto{\pgfqpoint{2.193147in}{2.092025in}}%
\pgfpathcurveto{\pgfqpoint{2.206170in}{2.092025in}}{\pgfqpoint{2.218661in}{2.097199in}}{\pgfqpoint{2.227869in}{2.106408in}}%
\pgfpathcurveto{\pgfqpoint{2.237077in}{2.115616in}}{\pgfqpoint{2.242251in}{2.128107in}}{\pgfqpoint{2.242251in}{2.141130in}}%
\pgfpathcurveto{\pgfqpoint{2.242251in}{2.154153in}}{\pgfqpoint{2.237077in}{2.166644in}}{\pgfqpoint{2.227869in}{2.175852in}}%
\pgfpathcurveto{\pgfqpoint{2.218661in}{2.185060in}}{\pgfqpoint{2.206170in}{2.190234in}}{\pgfqpoint{2.193147in}{2.190234in}}%
\pgfpathcurveto{\pgfqpoint{2.180124in}{2.190234in}}{\pgfqpoint{2.167633in}{2.185060in}}{\pgfqpoint{2.158425in}{2.175852in}}%
\pgfpathcurveto{\pgfqpoint{2.149216in}{2.166644in}}{\pgfqpoint{2.144042in}{2.154153in}}{\pgfqpoint{2.144042in}{2.141130in}}%
\pgfpathcurveto{\pgfqpoint{2.144042in}{2.128107in}}{\pgfqpoint{2.149216in}{2.115616in}}{\pgfqpoint{2.158425in}{2.106408in}}%
\pgfpathcurveto{\pgfqpoint{2.167633in}{2.097199in}}{\pgfqpoint{2.180124in}{2.092025in}}{\pgfqpoint{2.193147in}{2.092025in}}%
\pgfpathlineto{\pgfqpoint{2.193147in}{2.092025in}}%
\pgfpathclose%
\pgfusepath{stroke,fill}%
\end{pgfscope}%
\begin{pgfscope}%
\pgfpathrectangle{\pgfqpoint{0.786164in}{0.768110in}}{\pgfqpoint{8.851069in}{7.081890in}}%
\pgfusepath{clip}%
\pgfsetbuttcap%
\pgfsetroundjoin%
\definecolor{currentfill}{rgb}{0.229739,0.322361,0.545706}%
\pgfsetfillcolor{currentfill}%
\pgfsetfillopacity{0.700000}%
\pgfsetlinewidth{0.501875pt}%
\definecolor{currentstroke}{rgb}{1.000000,1.000000,1.000000}%
\pgfsetstrokecolor{currentstroke}%
\pgfsetstrokeopacity{0.700000}%
\pgfsetdash{}{0pt}%
\pgfpathmoveto{\pgfqpoint{2.083547in}{2.223415in}}%
\pgfpathcurveto{\pgfqpoint{2.096570in}{2.223415in}}{\pgfqpoint{2.109061in}{2.228589in}}{\pgfqpoint{2.118270in}{2.237797in}}%
\pgfpathcurveto{\pgfqpoint{2.127478in}{2.247005in}}{\pgfqpoint{2.132652in}{2.259497in}}{\pgfqpoint{2.132652in}{2.272519in}}%
\pgfpathcurveto{\pgfqpoint{2.132652in}{2.285542in}}{\pgfqpoint{2.127478in}{2.298033in}}{\pgfqpoint{2.118270in}{2.307241in}}%
\pgfpathcurveto{\pgfqpoint{2.109061in}{2.316450in}}{\pgfqpoint{2.096570in}{2.321624in}}{\pgfqpoint{2.083547in}{2.321624in}}%
\pgfpathcurveto{\pgfqpoint{2.070525in}{2.321624in}}{\pgfqpoint{2.058034in}{2.316450in}}{\pgfqpoint{2.048825in}{2.307241in}}%
\pgfpathcurveto{\pgfqpoint{2.039617in}{2.298033in}}{\pgfqpoint{2.034443in}{2.285542in}}{\pgfqpoint{2.034443in}{2.272519in}}%
\pgfpathcurveto{\pgfqpoint{2.034443in}{2.259497in}}{\pgfqpoint{2.039617in}{2.247005in}}{\pgfqpoint{2.048825in}{2.237797in}}%
\pgfpathcurveto{\pgfqpoint{2.058034in}{2.228589in}}{\pgfqpoint{2.070525in}{2.223415in}}{\pgfqpoint{2.083547in}{2.223415in}}%
\pgfpathlineto{\pgfqpoint{2.083547in}{2.223415in}}%
\pgfpathclose%
\pgfusepath{stroke,fill}%
\end{pgfscope}%
\begin{pgfscope}%
\pgfpathrectangle{\pgfqpoint{0.786164in}{0.768110in}}{\pgfqpoint{8.851069in}{7.081890in}}%
\pgfusepath{clip}%
\pgfsetbuttcap%
\pgfsetroundjoin%
\definecolor{currentfill}{rgb}{0.227802,0.326594,0.546532}%
\pgfsetfillcolor{currentfill}%
\pgfsetfillopacity{0.700000}%
\pgfsetlinewidth{0.501875pt}%
\definecolor{currentstroke}{rgb}{1.000000,1.000000,1.000000}%
\pgfsetstrokecolor{currentstroke}%
\pgfsetstrokeopacity{0.700000}%
\pgfsetdash{}{0pt}%
\pgfpathmoveto{\pgfqpoint{2.074414in}{2.289109in}}%
\pgfpathcurveto{\pgfqpoint{2.087437in}{2.289109in}}{\pgfqpoint{2.099928in}{2.294283in}}{\pgfqpoint{2.109136in}{2.303492in}}%
\pgfpathcurveto{\pgfqpoint{2.118345in}{2.312700in}}{\pgfqpoint{2.123519in}{2.325191in}}{\pgfqpoint{2.123519in}{2.338214in}}%
\pgfpathcurveto{\pgfqpoint{2.123519in}{2.351237in}}{\pgfqpoint{2.118345in}{2.363728in}}{\pgfqpoint{2.109136in}{2.372936in}}%
\pgfpathcurveto{\pgfqpoint{2.099928in}{2.382145in}}{\pgfqpoint{2.087437in}{2.387319in}}{\pgfqpoint{2.074414in}{2.387319in}}%
\pgfpathcurveto{\pgfqpoint{2.061391in}{2.387319in}}{\pgfqpoint{2.048900in}{2.382145in}}{\pgfqpoint{2.039692in}{2.372936in}}%
\pgfpathcurveto{\pgfqpoint{2.030483in}{2.363728in}}{\pgfqpoint{2.025309in}{2.351237in}}{\pgfqpoint{2.025309in}{2.338214in}}%
\pgfpathcurveto{\pgfqpoint{2.025309in}{2.325191in}}{\pgfqpoint{2.030483in}{2.312700in}}{\pgfqpoint{2.039692in}{2.303492in}}%
\pgfpathcurveto{\pgfqpoint{2.048900in}{2.294283in}}{\pgfqpoint{2.061391in}{2.289109in}}{\pgfqpoint{2.074414in}{2.289109in}}%
\pgfpathlineto{\pgfqpoint{2.074414in}{2.289109in}}%
\pgfpathclose%
\pgfusepath{stroke,fill}%
\end{pgfscope}%
\begin{pgfscope}%
\pgfpathrectangle{\pgfqpoint{0.786164in}{0.768110in}}{\pgfqpoint{8.851069in}{7.081890in}}%
\pgfusepath{clip}%
\pgfsetbuttcap%
\pgfsetroundjoin%
\definecolor{currentfill}{rgb}{0.231674,0.318106,0.544834}%
\pgfsetfillcolor{currentfill}%
\pgfsetfillopacity{0.700000}%
\pgfsetlinewidth{0.501875pt}%
\definecolor{currentstroke}{rgb}{1.000000,1.000000,1.000000}%
\pgfsetstrokecolor{currentstroke}%
\pgfsetstrokeopacity{0.700000}%
\pgfsetdash{}{0pt}%
\pgfpathmoveto{\pgfqpoint{2.165747in}{2.332906in}}%
\pgfpathcurveto{\pgfqpoint{2.178770in}{2.332906in}}{\pgfqpoint{2.191261in}{2.338080in}}{\pgfqpoint{2.200469in}{2.347288in}}%
\pgfpathcurveto{\pgfqpoint{2.209678in}{2.356497in}}{\pgfqpoint{2.214852in}{2.368988in}}{\pgfqpoint{2.214852in}{2.382010in}}%
\pgfpathcurveto{\pgfqpoint{2.214852in}{2.395033in}}{\pgfqpoint{2.209678in}{2.407524in}}{\pgfqpoint{2.200469in}{2.416733in}}%
\pgfpathcurveto{\pgfqpoint{2.191261in}{2.425941in}}{\pgfqpoint{2.178770in}{2.431115in}}{\pgfqpoint{2.165747in}{2.431115in}}%
\pgfpathcurveto{\pgfqpoint{2.152724in}{2.431115in}}{\pgfqpoint{2.140233in}{2.425941in}}{\pgfqpoint{2.131025in}{2.416733in}}%
\pgfpathcurveto{\pgfqpoint{2.121816in}{2.407524in}}{\pgfqpoint{2.116642in}{2.395033in}}{\pgfqpoint{2.116642in}{2.382010in}}%
\pgfpathcurveto{\pgfqpoint{2.116642in}{2.368988in}}{\pgfqpoint{2.121816in}{2.356497in}}{\pgfqpoint{2.131025in}{2.347288in}}%
\pgfpathcurveto{\pgfqpoint{2.140233in}{2.338080in}}{\pgfqpoint{2.152724in}{2.332906in}}{\pgfqpoint{2.165747in}{2.332906in}}%
\pgfpathlineto{\pgfqpoint{2.165747in}{2.332906in}}%
\pgfpathclose%
\pgfusepath{stroke,fill}%
\end{pgfscope}%
\begin{pgfscope}%
\pgfpathrectangle{\pgfqpoint{0.786164in}{0.768110in}}{\pgfqpoint{8.851069in}{7.081890in}}%
\pgfusepath{clip}%
\pgfsetbuttcap%
\pgfsetroundjoin%
\definecolor{currentfill}{rgb}{0.221989,0.339161,0.548752}%
\pgfsetfillcolor{currentfill}%
\pgfsetfillopacity{0.700000}%
\pgfsetlinewidth{0.501875pt}%
\definecolor{currentstroke}{rgb}{1.000000,1.000000,1.000000}%
\pgfsetstrokecolor{currentstroke}%
\pgfsetstrokeopacity{0.700000}%
\pgfsetdash{}{0pt}%
\pgfpathmoveto{\pgfqpoint{2.229680in}{2.223415in}}%
\pgfpathcurveto{\pgfqpoint{2.242703in}{2.223415in}}{\pgfqpoint{2.255194in}{2.228589in}}{\pgfqpoint{2.264402in}{2.237797in}}%
\pgfpathcurveto{\pgfqpoint{2.273611in}{2.247005in}}{\pgfqpoint{2.278785in}{2.259497in}}{\pgfqpoint{2.278785in}{2.272519in}}%
\pgfpathcurveto{\pgfqpoint{2.278785in}{2.285542in}}{\pgfqpoint{2.273611in}{2.298033in}}{\pgfqpoint{2.264402in}{2.307241in}}%
\pgfpathcurveto{\pgfqpoint{2.255194in}{2.316450in}}{\pgfqpoint{2.242703in}{2.321624in}}{\pgfqpoint{2.229680in}{2.321624in}}%
\pgfpathcurveto{\pgfqpoint{2.216657in}{2.321624in}}{\pgfqpoint{2.204166in}{2.316450in}}{\pgfqpoint{2.194958in}{2.307241in}}%
\pgfpathcurveto{\pgfqpoint{2.185749in}{2.298033in}}{\pgfqpoint{2.180575in}{2.285542in}}{\pgfqpoint{2.180575in}{2.272519in}}%
\pgfpathcurveto{\pgfqpoint{2.180575in}{2.259497in}}{\pgfqpoint{2.185749in}{2.247005in}}{\pgfqpoint{2.194958in}{2.237797in}}%
\pgfpathcurveto{\pgfqpoint{2.204166in}{2.228589in}}{\pgfqpoint{2.216657in}{2.223415in}}{\pgfqpoint{2.229680in}{2.223415in}}%
\pgfpathlineto{\pgfqpoint{2.229680in}{2.223415in}}%
\pgfpathclose%
\pgfusepath{stroke,fill}%
\end{pgfscope}%
\begin{pgfscope}%
\pgfpathrectangle{\pgfqpoint{0.786164in}{0.768110in}}{\pgfqpoint{8.851069in}{7.081890in}}%
\pgfusepath{clip}%
\pgfsetbuttcap%
\pgfsetroundjoin%
\definecolor{currentfill}{rgb}{0.206756,0.371758,0.553117}%
\pgfsetfillcolor{currentfill}%
\pgfsetfillopacity{0.700000}%
\pgfsetlinewidth{0.501875pt}%
\definecolor{currentstroke}{rgb}{1.000000,1.000000,1.000000}%
\pgfsetstrokecolor{currentstroke}%
\pgfsetstrokeopacity{0.700000}%
\pgfsetdash{}{0pt}%
\pgfpathmoveto{\pgfqpoint{2.156614in}{1.894941in}}%
\pgfpathcurveto{\pgfqpoint{2.169636in}{1.894941in}}{\pgfqpoint{2.182127in}{1.900115in}}{\pgfqpoint{2.191336in}{1.909323in}}%
\pgfpathcurveto{\pgfqpoint{2.200544in}{1.918532in}}{\pgfqpoint{2.205718in}{1.931023in}}{\pgfqpoint{2.205718in}{1.944046in}}%
\pgfpathcurveto{\pgfqpoint{2.205718in}{1.957068in}}{\pgfqpoint{2.200544in}{1.969559in}}{\pgfqpoint{2.191336in}{1.978768in}}%
\pgfpathcurveto{\pgfqpoint{2.182127in}{1.987976in}}{\pgfqpoint{2.169636in}{1.993150in}}{\pgfqpoint{2.156614in}{1.993150in}}%
\pgfpathcurveto{\pgfqpoint{2.143591in}{1.993150in}}{\pgfqpoint{2.131100in}{1.987976in}}{\pgfqpoint{2.121891in}{1.978768in}}%
\pgfpathcurveto{\pgfqpoint{2.112683in}{1.969559in}}{\pgfqpoint{2.107509in}{1.957068in}}{\pgfqpoint{2.107509in}{1.944046in}}%
\pgfpathcurveto{\pgfqpoint{2.107509in}{1.931023in}}{\pgfqpoint{2.112683in}{1.918532in}}{\pgfqpoint{2.121891in}{1.909323in}}%
\pgfpathcurveto{\pgfqpoint{2.131100in}{1.900115in}}{\pgfqpoint{2.143591in}{1.894941in}}{\pgfqpoint{2.156614in}{1.894941in}}%
\pgfpathlineto{\pgfqpoint{2.156614in}{1.894941in}}%
\pgfpathclose%
\pgfusepath{stroke,fill}%
\end{pgfscope}%
\begin{pgfscope}%
\pgfpathrectangle{\pgfqpoint{0.786164in}{0.768110in}}{\pgfqpoint{8.851069in}{7.081890in}}%
\pgfusepath{clip}%
\pgfsetbuttcap%
\pgfsetroundjoin%
\definecolor{currentfill}{rgb}{0.208623,0.367752,0.552675}%
\pgfsetfillcolor{currentfill}%
\pgfsetfillopacity{0.700000}%
\pgfsetlinewidth{0.501875pt}%
\definecolor{currentstroke}{rgb}{1.000000,1.000000,1.000000}%
\pgfsetstrokecolor{currentstroke}%
\pgfsetstrokeopacity{0.700000}%
\pgfsetdash{}{0pt}%
\pgfpathmoveto{\pgfqpoint{1.690816in}{1.763552in}}%
\pgfpathcurveto{\pgfqpoint{1.703839in}{1.763552in}}{\pgfqpoint{1.716330in}{1.768726in}}{\pgfqpoint{1.725538in}{1.777934in}}%
\pgfpathcurveto{\pgfqpoint{1.734747in}{1.787143in}}{\pgfqpoint{1.739921in}{1.799634in}}{\pgfqpoint{1.739921in}{1.812656in}}%
\pgfpathcurveto{\pgfqpoint{1.739921in}{1.825679in}}{\pgfqpoint{1.734747in}{1.838170in}}{\pgfqpoint{1.725538in}{1.847379in}}%
\pgfpathcurveto{\pgfqpoint{1.716330in}{1.856587in}}{\pgfqpoint{1.703839in}{1.861761in}}{\pgfqpoint{1.690816in}{1.861761in}}%
\pgfpathcurveto{\pgfqpoint{1.677793in}{1.861761in}}{\pgfqpoint{1.665302in}{1.856587in}}{\pgfqpoint{1.656094in}{1.847379in}}%
\pgfpathcurveto{\pgfqpoint{1.646885in}{1.838170in}}{\pgfqpoint{1.641711in}{1.825679in}}{\pgfqpoint{1.641711in}{1.812656in}}%
\pgfpathcurveto{\pgfqpoint{1.641711in}{1.799634in}}{\pgfqpoint{1.646885in}{1.787143in}}{\pgfqpoint{1.656094in}{1.777934in}}%
\pgfpathcurveto{\pgfqpoint{1.665302in}{1.768726in}}{\pgfqpoint{1.677793in}{1.763552in}}{\pgfqpoint{1.690816in}{1.763552in}}%
\pgfpathlineto{\pgfqpoint{1.690816in}{1.763552in}}%
\pgfpathclose%
\pgfusepath{stroke,fill}%
\end{pgfscope}%
\begin{pgfscope}%
\pgfpathrectangle{\pgfqpoint{0.786164in}{0.768110in}}{\pgfqpoint{8.851069in}{7.081890in}}%
\pgfusepath{clip}%
\pgfsetbuttcap%
\pgfsetroundjoin%
\definecolor{currentfill}{rgb}{0.204903,0.375746,0.553533}%
\pgfsetfillcolor{currentfill}%
\pgfsetfillopacity{0.700000}%
\pgfsetlinewidth{0.501875pt}%
\definecolor{currentstroke}{rgb}{1.000000,1.000000,1.000000}%
\pgfsetstrokecolor{currentstroke}%
\pgfsetstrokeopacity{0.700000}%
\pgfsetdash{}{0pt}%
\pgfpathmoveto{\pgfqpoint{1.654283in}{1.785450in}}%
\pgfpathcurveto{\pgfqpoint{1.667306in}{1.785450in}}{\pgfqpoint{1.679797in}{1.790624in}}{\pgfqpoint{1.689005in}{1.799832in}}%
\pgfpathcurveto{\pgfqpoint{1.698214in}{1.809041in}}{\pgfqpoint{1.703388in}{1.821532in}}{\pgfqpoint{1.703388in}{1.834555in}}%
\pgfpathcurveto{\pgfqpoint{1.703388in}{1.847577in}}{\pgfqpoint{1.698214in}{1.860068in}}{\pgfqpoint{1.689005in}{1.869277in}}%
\pgfpathcurveto{\pgfqpoint{1.679797in}{1.878485in}}{\pgfqpoint{1.667306in}{1.883659in}}{\pgfqpoint{1.654283in}{1.883659in}}%
\pgfpathcurveto{\pgfqpoint{1.641260in}{1.883659in}}{\pgfqpoint{1.628769in}{1.878485in}}{\pgfqpoint{1.619561in}{1.869277in}}%
\pgfpathcurveto{\pgfqpoint{1.610352in}{1.860068in}}{\pgfqpoint{1.605178in}{1.847577in}}{\pgfqpoint{1.605178in}{1.834555in}}%
\pgfpathcurveto{\pgfqpoint{1.605178in}{1.821532in}}{\pgfqpoint{1.610352in}{1.809041in}}{\pgfqpoint{1.619561in}{1.799832in}}%
\pgfpathcurveto{\pgfqpoint{1.628769in}{1.790624in}}{\pgfqpoint{1.641260in}{1.785450in}}{\pgfqpoint{1.654283in}{1.785450in}}%
\pgfpathlineto{\pgfqpoint{1.654283in}{1.785450in}}%
\pgfpathclose%
\pgfusepath{stroke,fill}%
\end{pgfscope}%
\begin{pgfscope}%
\pgfpathrectangle{\pgfqpoint{0.786164in}{0.768110in}}{\pgfqpoint{8.851069in}{7.081890in}}%
\pgfusepath{clip}%
\pgfsetbuttcap%
\pgfsetroundjoin%
\definecolor{currentfill}{rgb}{0.195860,0.395433,0.555276}%
\pgfsetfillcolor{currentfill}%
\pgfsetfillopacity{0.700000}%
\pgfsetlinewidth{0.501875pt}%
\definecolor{currentstroke}{rgb}{1.000000,1.000000,1.000000}%
\pgfsetstrokecolor{currentstroke}%
\pgfsetstrokeopacity{0.700000}%
\pgfsetdash{}{0pt}%
\pgfpathmoveto{\pgfqpoint{1.699949in}{1.829246in}}%
\pgfpathcurveto{\pgfqpoint{1.712972in}{1.829246in}}{\pgfqpoint{1.725463in}{1.834420in}}{\pgfqpoint{1.734672in}{1.843629in}}%
\pgfpathcurveto{\pgfqpoint{1.743880in}{1.852837in}}{\pgfqpoint{1.749054in}{1.865328in}}{\pgfqpoint{1.749054in}{1.878351in}}%
\pgfpathcurveto{\pgfqpoint{1.749054in}{1.891374in}}{\pgfqpoint{1.743880in}{1.903865in}}{\pgfqpoint{1.734672in}{1.913073in}}%
\pgfpathcurveto{\pgfqpoint{1.725463in}{1.922282in}}{\pgfqpoint{1.712972in}{1.927456in}}{\pgfqpoint{1.699949in}{1.927456in}}%
\pgfpathcurveto{\pgfqpoint{1.686927in}{1.927456in}}{\pgfqpoint{1.674436in}{1.922282in}}{\pgfqpoint{1.665227in}{1.913073in}}%
\pgfpathcurveto{\pgfqpoint{1.656019in}{1.903865in}}{\pgfqpoint{1.650845in}{1.891374in}}{\pgfqpoint{1.650845in}{1.878351in}}%
\pgfpathcurveto{\pgfqpoint{1.650845in}{1.865328in}}{\pgfqpoint{1.656019in}{1.852837in}}{\pgfqpoint{1.665227in}{1.843629in}}%
\pgfpathcurveto{\pgfqpoint{1.674436in}{1.834420in}}{\pgfqpoint{1.686927in}{1.829246in}}{\pgfqpoint{1.699949in}{1.829246in}}%
\pgfpathlineto{\pgfqpoint{1.699949in}{1.829246in}}%
\pgfpathclose%
\pgfusepath{stroke,fill}%
\end{pgfscope}%
\begin{pgfscope}%
\pgfpathrectangle{\pgfqpoint{0.786164in}{0.768110in}}{\pgfqpoint{8.851069in}{7.081890in}}%
\pgfusepath{clip}%
\pgfsetbuttcap%
\pgfsetroundjoin%
\definecolor{currentfill}{rgb}{0.187231,0.414746,0.556547}%
\pgfsetfillcolor{currentfill}%
\pgfsetfillopacity{0.700000}%
\pgfsetlinewidth{0.501875pt}%
\definecolor{currentstroke}{rgb}{1.000000,1.000000,1.000000}%
\pgfsetstrokecolor{currentstroke}%
\pgfsetstrokeopacity{0.700000}%
\pgfsetdash{}{0pt}%
\pgfpathmoveto{\pgfqpoint{1.663416in}{1.807348in}}%
\pgfpathcurveto{\pgfqpoint{1.676439in}{1.807348in}}{\pgfqpoint{1.688930in}{1.812522in}}{\pgfqpoint{1.698138in}{1.821731in}}%
\pgfpathcurveto{\pgfqpoint{1.707347in}{1.830939in}}{\pgfqpoint{1.712521in}{1.843430in}}{\pgfqpoint{1.712521in}{1.856453in}}%
\pgfpathcurveto{\pgfqpoint{1.712521in}{1.869475in}}{\pgfqpoint{1.707347in}{1.881967in}}{\pgfqpoint{1.698138in}{1.891175in}}%
\pgfpathcurveto{\pgfqpoint{1.688930in}{1.900383in}}{\pgfqpoint{1.676439in}{1.905557in}}{\pgfqpoint{1.663416in}{1.905557in}}%
\pgfpathcurveto{\pgfqpoint{1.650393in}{1.905557in}}{\pgfqpoint{1.637902in}{1.900383in}}{\pgfqpoint{1.628694in}{1.891175in}}%
\pgfpathcurveto{\pgfqpoint{1.619485in}{1.881967in}}{\pgfqpoint{1.614312in}{1.869475in}}{\pgfqpoint{1.614312in}{1.856453in}}%
\pgfpathcurveto{\pgfqpoint{1.614312in}{1.843430in}}{\pgfqpoint{1.619485in}{1.830939in}}{\pgfqpoint{1.628694in}{1.821731in}}%
\pgfpathcurveto{\pgfqpoint{1.637902in}{1.812522in}}{\pgfqpoint{1.650393in}{1.807348in}}{\pgfqpoint{1.663416in}{1.807348in}}%
\pgfpathlineto{\pgfqpoint{1.663416in}{1.807348in}}%
\pgfpathclose%
\pgfusepath{stroke,fill}%
\end{pgfscope}%
\begin{pgfscope}%
\pgfpathrectangle{\pgfqpoint{0.786164in}{0.768110in}}{\pgfqpoint{8.851069in}{7.081890in}}%
\pgfusepath{clip}%
\pgfsetbuttcap%
\pgfsetroundjoin%
\definecolor{currentfill}{rgb}{0.180629,0.429975,0.557282}%
\pgfsetfillcolor{currentfill}%
\pgfsetfillopacity{0.700000}%
\pgfsetlinewidth{0.501875pt}%
\definecolor{currentstroke}{rgb}{1.000000,1.000000,1.000000}%
\pgfsetstrokecolor{currentstroke}%
\pgfsetstrokeopacity{0.700000}%
\pgfsetdash{}{0pt}%
\pgfpathmoveto{\pgfqpoint{1.663416in}{1.851145in}}%
\pgfpathcurveto{\pgfqpoint{1.676439in}{1.851145in}}{\pgfqpoint{1.688930in}{1.856319in}}{\pgfqpoint{1.698138in}{1.865527in}}%
\pgfpathcurveto{\pgfqpoint{1.707347in}{1.874735in}}{\pgfqpoint{1.712521in}{1.887227in}}{\pgfqpoint{1.712521in}{1.900249in}}%
\pgfpathcurveto{\pgfqpoint{1.712521in}{1.913272in}}{\pgfqpoint{1.707347in}{1.925763in}}{\pgfqpoint{1.698138in}{1.934971in}}%
\pgfpathcurveto{\pgfqpoint{1.688930in}{1.944180in}}{\pgfqpoint{1.676439in}{1.949354in}}{\pgfqpoint{1.663416in}{1.949354in}}%
\pgfpathcurveto{\pgfqpoint{1.650393in}{1.949354in}}{\pgfqpoint{1.637902in}{1.944180in}}{\pgfqpoint{1.628694in}{1.934971in}}%
\pgfpathcurveto{\pgfqpoint{1.619485in}{1.925763in}}{\pgfqpoint{1.614312in}{1.913272in}}{\pgfqpoint{1.614312in}{1.900249in}}%
\pgfpathcurveto{\pgfqpoint{1.614312in}{1.887227in}}{\pgfqpoint{1.619485in}{1.874735in}}{\pgfqpoint{1.628694in}{1.865527in}}%
\pgfpathcurveto{\pgfqpoint{1.637902in}{1.856319in}}{\pgfqpoint{1.650393in}{1.851145in}}{\pgfqpoint{1.663416in}{1.851145in}}%
\pgfpathlineto{\pgfqpoint{1.663416in}{1.851145in}}%
\pgfpathclose%
\pgfusepath{stroke,fill}%
\end{pgfscope}%
\begin{pgfscope}%
\pgfpathrectangle{\pgfqpoint{0.786164in}{0.768110in}}{\pgfqpoint{8.851069in}{7.081890in}}%
\pgfusepath{clip}%
\pgfsetbuttcap%
\pgfsetroundjoin%
\definecolor{currentfill}{rgb}{0.168126,0.459988,0.558082}%
\pgfsetfillcolor{currentfill}%
\pgfsetfillopacity{0.700000}%
\pgfsetlinewidth{0.501875pt}%
\definecolor{currentstroke}{rgb}{1.000000,1.000000,1.000000}%
\pgfsetstrokecolor{currentstroke}%
\pgfsetstrokeopacity{0.700000}%
\pgfsetdash{}{0pt}%
\pgfpathmoveto{\pgfqpoint{1.690816in}{1.960636in}}%
\pgfpathcurveto{\pgfqpoint{1.703839in}{1.960636in}}{\pgfqpoint{1.716330in}{1.965810in}}{\pgfqpoint{1.725538in}{1.975018in}}%
\pgfpathcurveto{\pgfqpoint{1.734747in}{1.984227in}}{\pgfqpoint{1.739921in}{1.996718in}}{\pgfqpoint{1.739921in}{2.009740in}}%
\pgfpathcurveto{\pgfqpoint{1.739921in}{2.022763in}}{\pgfqpoint{1.734747in}{2.035254in}}{\pgfqpoint{1.725538in}{2.044463in}}%
\pgfpathcurveto{\pgfqpoint{1.716330in}{2.053671in}}{\pgfqpoint{1.703839in}{2.058845in}}{\pgfqpoint{1.690816in}{2.058845in}}%
\pgfpathcurveto{\pgfqpoint{1.677793in}{2.058845in}}{\pgfqpoint{1.665302in}{2.053671in}}{\pgfqpoint{1.656094in}{2.044463in}}%
\pgfpathcurveto{\pgfqpoint{1.646885in}{2.035254in}}{\pgfqpoint{1.641711in}{2.022763in}}{\pgfqpoint{1.641711in}{2.009740in}}%
\pgfpathcurveto{\pgfqpoint{1.641711in}{1.996718in}}{\pgfqpoint{1.646885in}{1.984227in}}{\pgfqpoint{1.656094in}{1.975018in}}%
\pgfpathcurveto{\pgfqpoint{1.665302in}{1.965810in}}{\pgfqpoint{1.677793in}{1.960636in}}{\pgfqpoint{1.690816in}{1.960636in}}%
\pgfpathlineto{\pgfqpoint{1.690816in}{1.960636in}}%
\pgfpathclose%
\pgfusepath{stroke,fill}%
\end{pgfscope}%
\begin{pgfscope}%
\pgfpathrectangle{\pgfqpoint{0.786164in}{0.768110in}}{\pgfqpoint{8.851069in}{7.081890in}}%
\pgfusepath{clip}%
\pgfsetbuttcap%
\pgfsetroundjoin%
\definecolor{currentfill}{rgb}{0.168126,0.459988,0.558082}%
\pgfsetfillcolor{currentfill}%
\pgfsetfillopacity{0.700000}%
\pgfsetlinewidth{0.501875pt}%
\definecolor{currentstroke}{rgb}{1.000000,1.000000,1.000000}%
\pgfsetstrokecolor{currentstroke}%
\pgfsetstrokeopacity{0.700000}%
\pgfsetdash{}{0pt}%
\pgfpathmoveto{\pgfqpoint{1.800415in}{2.026330in}}%
\pgfpathcurveto{\pgfqpoint{1.813438in}{2.026330in}}{\pgfqpoint{1.825929in}{2.031504in}}{\pgfqpoint{1.835138in}{2.040713in}}%
\pgfpathcurveto{\pgfqpoint{1.844346in}{2.049921in}}{\pgfqpoint{1.849520in}{2.062412in}}{\pgfqpoint{1.849520in}{2.075435in}}%
\pgfpathcurveto{\pgfqpoint{1.849520in}{2.088458in}}{\pgfqpoint{1.844346in}{2.100949in}}{\pgfqpoint{1.835138in}{2.110157in}}%
\pgfpathcurveto{\pgfqpoint{1.825929in}{2.119366in}}{\pgfqpoint{1.813438in}{2.124540in}}{\pgfqpoint{1.800415in}{2.124540in}}%
\pgfpathcurveto{\pgfqpoint{1.787393in}{2.124540in}}{\pgfqpoint{1.774902in}{2.119366in}}{\pgfqpoint{1.765693in}{2.110157in}}%
\pgfpathcurveto{\pgfqpoint{1.756485in}{2.100949in}}{\pgfqpoint{1.751311in}{2.088458in}}{\pgfqpoint{1.751311in}{2.075435in}}%
\pgfpathcurveto{\pgfqpoint{1.751311in}{2.062412in}}{\pgfqpoint{1.756485in}{2.049921in}}{\pgfqpoint{1.765693in}{2.040713in}}%
\pgfpathcurveto{\pgfqpoint{1.774902in}{2.031504in}}{\pgfqpoint{1.787393in}{2.026330in}}{\pgfqpoint{1.800415in}{2.026330in}}%
\pgfpathlineto{\pgfqpoint{1.800415in}{2.026330in}}%
\pgfpathclose%
\pgfusepath{stroke,fill}%
\end{pgfscope}%
\begin{pgfscope}%
\pgfpathrectangle{\pgfqpoint{0.786164in}{0.768110in}}{\pgfqpoint{8.851069in}{7.081890in}}%
\pgfusepath{clip}%
\pgfsetbuttcap%
\pgfsetroundjoin%
\definecolor{currentfill}{rgb}{0.166617,0.463708,0.558119}%
\pgfsetfillcolor{currentfill}%
\pgfsetfillopacity{0.700000}%
\pgfsetlinewidth{0.501875pt}%
\definecolor{currentstroke}{rgb}{1.000000,1.000000,1.000000}%
\pgfsetstrokecolor{currentstroke}%
\pgfsetstrokeopacity{0.700000}%
\pgfsetdash{}{0pt}%
\pgfpathmoveto{\pgfqpoint{1.864348in}{2.113923in}}%
\pgfpathcurveto{\pgfqpoint{1.877371in}{2.113923in}}{\pgfqpoint{1.889862in}{2.119097in}}{\pgfqpoint{1.899071in}{2.128306in}}%
\pgfpathcurveto{\pgfqpoint{1.908279in}{2.137514in}}{\pgfqpoint{1.913453in}{2.150005in}}{\pgfqpoint{1.913453in}{2.163028in}}%
\pgfpathcurveto{\pgfqpoint{1.913453in}{2.176051in}}{\pgfqpoint{1.908279in}{2.188542in}}{\pgfqpoint{1.899071in}{2.197750in}}%
\pgfpathcurveto{\pgfqpoint{1.889862in}{2.206959in}}{\pgfqpoint{1.877371in}{2.212133in}}{\pgfqpoint{1.864348in}{2.212133in}}%
\pgfpathcurveto{\pgfqpoint{1.851326in}{2.212133in}}{\pgfqpoint{1.838835in}{2.206959in}}{\pgfqpoint{1.829626in}{2.197750in}}%
\pgfpathcurveto{\pgfqpoint{1.820418in}{2.188542in}}{\pgfqpoint{1.815244in}{2.176051in}}{\pgfqpoint{1.815244in}{2.163028in}}%
\pgfpathcurveto{\pgfqpoint{1.815244in}{2.150005in}}{\pgfqpoint{1.820418in}{2.137514in}}{\pgfqpoint{1.829626in}{2.128306in}}%
\pgfpathcurveto{\pgfqpoint{1.838835in}{2.119097in}}{\pgfqpoint{1.851326in}{2.113923in}}{\pgfqpoint{1.864348in}{2.113923in}}%
\pgfpathlineto{\pgfqpoint{1.864348in}{2.113923in}}%
\pgfpathclose%
\pgfusepath{stroke,fill}%
\end{pgfscope}%
\begin{pgfscope}%
\pgfpathrectangle{\pgfqpoint{0.786164in}{0.768110in}}{\pgfqpoint{8.851069in}{7.081890in}}%
\pgfusepath{clip}%
\pgfsetbuttcap%
\pgfsetroundjoin%
\definecolor{currentfill}{rgb}{0.174274,0.445044,0.557792}%
\pgfsetfillcolor{currentfill}%
\pgfsetfillopacity{0.700000}%
\pgfsetlinewidth{0.501875pt}%
\definecolor{currentstroke}{rgb}{1.000000,1.000000,1.000000}%
\pgfsetstrokecolor{currentstroke}%
\pgfsetstrokeopacity{0.700000}%
\pgfsetdash{}{0pt}%
\pgfpathmoveto{\pgfqpoint{1.946548in}{2.157720in}}%
\pgfpathcurveto{\pgfqpoint{1.959571in}{2.157720in}}{\pgfqpoint{1.972062in}{2.162894in}}{\pgfqpoint{1.981270in}{2.172102in}}%
\pgfpathcurveto{\pgfqpoint{1.990479in}{2.181311in}}{\pgfqpoint{1.995653in}{2.193802in}}{\pgfqpoint{1.995653in}{2.206825in}}%
\pgfpathcurveto{\pgfqpoint{1.995653in}{2.219847in}}{\pgfqpoint{1.990479in}{2.232338in}}{\pgfqpoint{1.981270in}{2.241547in}}%
\pgfpathcurveto{\pgfqpoint{1.972062in}{2.250755in}}{\pgfqpoint{1.959571in}{2.255929in}}{\pgfqpoint{1.946548in}{2.255929in}}%
\pgfpathcurveto{\pgfqpoint{1.933525in}{2.255929in}}{\pgfqpoint{1.921034in}{2.250755in}}{\pgfqpoint{1.911826in}{2.241547in}}%
\pgfpathcurveto{\pgfqpoint{1.902617in}{2.232338in}}{\pgfqpoint{1.897443in}{2.219847in}}{\pgfqpoint{1.897443in}{2.206825in}}%
\pgfpathcurveto{\pgfqpoint{1.897443in}{2.193802in}}{\pgfqpoint{1.902617in}{2.181311in}}{\pgfqpoint{1.911826in}{2.172102in}}%
\pgfpathcurveto{\pgfqpoint{1.921034in}{2.162894in}}{\pgfqpoint{1.933525in}{2.157720in}}{\pgfqpoint{1.946548in}{2.157720in}}%
\pgfpathlineto{\pgfqpoint{1.946548in}{2.157720in}}%
\pgfpathclose%
\pgfusepath{stroke,fill}%
\end{pgfscope}%
\begin{pgfscope}%
\pgfpathrectangle{\pgfqpoint{0.786164in}{0.768110in}}{\pgfqpoint{8.851069in}{7.081890in}}%
\pgfusepath{clip}%
\pgfsetbuttcap%
\pgfsetroundjoin%
\definecolor{currentfill}{rgb}{0.169646,0.456262,0.558030}%
\pgfsetfillcolor{currentfill}%
\pgfsetfillopacity{0.700000}%
\pgfsetlinewidth{0.501875pt}%
\definecolor{currentstroke}{rgb}{1.000000,1.000000,1.000000}%
\pgfsetstrokecolor{currentstroke}%
\pgfsetstrokeopacity{0.700000}%
\pgfsetdash{}{0pt}%
\pgfpathmoveto{\pgfqpoint{1.928281in}{2.157720in}}%
\pgfpathcurveto{\pgfqpoint{1.941304in}{2.157720in}}{\pgfqpoint{1.953795in}{2.162894in}}{\pgfqpoint{1.963004in}{2.172102in}}%
\pgfpathcurveto{\pgfqpoint{1.972212in}{2.181311in}}{\pgfqpoint{1.977386in}{2.193802in}}{\pgfqpoint{1.977386in}{2.206825in}}%
\pgfpathcurveto{\pgfqpoint{1.977386in}{2.219847in}}{\pgfqpoint{1.972212in}{2.232338in}}{\pgfqpoint{1.963004in}{2.241547in}}%
\pgfpathcurveto{\pgfqpoint{1.953795in}{2.250755in}}{\pgfqpoint{1.941304in}{2.255929in}}{\pgfqpoint{1.928281in}{2.255929in}}%
\pgfpathcurveto{\pgfqpoint{1.915259in}{2.255929in}}{\pgfqpoint{1.902768in}{2.250755in}}{\pgfqpoint{1.893559in}{2.241547in}}%
\pgfpathcurveto{\pgfqpoint{1.884351in}{2.232338in}}{\pgfqpoint{1.879177in}{2.219847in}}{\pgfqpoint{1.879177in}{2.206825in}}%
\pgfpathcurveto{\pgfqpoint{1.879177in}{2.193802in}}{\pgfqpoint{1.884351in}{2.181311in}}{\pgfqpoint{1.893559in}{2.172102in}}%
\pgfpathcurveto{\pgfqpoint{1.902768in}{2.162894in}}{\pgfqpoint{1.915259in}{2.157720in}}{\pgfqpoint{1.928281in}{2.157720in}}%
\pgfpathlineto{\pgfqpoint{1.928281in}{2.157720in}}%
\pgfpathclose%
\pgfusepath{stroke,fill}%
\end{pgfscope}%
\begin{pgfscope}%
\pgfpathrectangle{\pgfqpoint{0.786164in}{0.768110in}}{\pgfqpoint{8.851069in}{7.081890in}}%
\pgfusepath{clip}%
\pgfsetbuttcap%
\pgfsetroundjoin%
\definecolor{currentfill}{rgb}{0.162142,0.474838,0.558140}%
\pgfsetfillcolor{currentfill}%
\pgfsetfillopacity{0.700000}%
\pgfsetlinewidth{0.501875pt}%
\definecolor{currentstroke}{rgb}{1.000000,1.000000,1.000000}%
\pgfsetstrokecolor{currentstroke}%
\pgfsetstrokeopacity{0.700000}%
\pgfsetdash{}{0pt}%
\pgfpathmoveto{\pgfqpoint{1.910015in}{2.135822in}}%
\pgfpathcurveto{\pgfqpoint{1.923038in}{2.135822in}}{\pgfqpoint{1.935529in}{2.140996in}}{\pgfqpoint{1.944737in}{2.150204in}}%
\pgfpathcurveto{\pgfqpoint{1.953946in}{2.159413in}}{\pgfqpoint{1.959120in}{2.171904in}}{\pgfqpoint{1.959120in}{2.184926in}}%
\pgfpathcurveto{\pgfqpoint{1.959120in}{2.197949in}}{\pgfqpoint{1.953946in}{2.210440in}}{\pgfqpoint{1.944737in}{2.219649in}}%
\pgfpathcurveto{\pgfqpoint{1.935529in}{2.228857in}}{\pgfqpoint{1.923038in}{2.234031in}}{\pgfqpoint{1.910015in}{2.234031in}}%
\pgfpathcurveto{\pgfqpoint{1.896992in}{2.234031in}}{\pgfqpoint{1.884501in}{2.228857in}}{\pgfqpoint{1.875293in}{2.219649in}}%
\pgfpathcurveto{\pgfqpoint{1.866084in}{2.210440in}}{\pgfqpoint{1.860910in}{2.197949in}}{\pgfqpoint{1.860910in}{2.184926in}}%
\pgfpathcurveto{\pgfqpoint{1.860910in}{2.171904in}}{\pgfqpoint{1.866084in}{2.159413in}}{\pgfqpoint{1.875293in}{2.150204in}}%
\pgfpathcurveto{\pgfqpoint{1.884501in}{2.140996in}}{\pgfqpoint{1.896992in}{2.135822in}}{\pgfqpoint{1.910015in}{2.135822in}}%
\pgfpathlineto{\pgfqpoint{1.910015in}{2.135822in}}%
\pgfpathclose%
\pgfusepath{stroke,fill}%
\end{pgfscope}%
\begin{pgfscope}%
\pgfpathrectangle{\pgfqpoint{0.786164in}{0.768110in}}{\pgfqpoint{8.851069in}{7.081890in}}%
\pgfusepath{clip}%
\pgfsetbuttcap%
\pgfsetroundjoin%
\definecolor{currentfill}{rgb}{0.153364,0.497000,0.557724}%
\pgfsetfillcolor{currentfill}%
\pgfsetfillopacity{0.700000}%
\pgfsetlinewidth{0.501875pt}%
\definecolor{currentstroke}{rgb}{1.000000,1.000000,1.000000}%
\pgfsetstrokecolor{currentstroke}%
\pgfsetstrokeopacity{0.700000}%
\pgfsetdash{}{0pt}%
\pgfpathmoveto{\pgfqpoint{2.037881in}{2.179618in}}%
\pgfpathcurveto{\pgfqpoint{2.050904in}{2.179618in}}{\pgfqpoint{2.063395in}{2.184792in}}{\pgfqpoint{2.072603in}{2.194001in}}%
\pgfpathcurveto{\pgfqpoint{2.081812in}{2.203209in}}{\pgfqpoint{2.086986in}{2.215700in}}{\pgfqpoint{2.086986in}{2.228723in}}%
\pgfpathcurveto{\pgfqpoint{2.086986in}{2.241745in}}{\pgfqpoint{2.081812in}{2.254237in}}{\pgfqpoint{2.072603in}{2.263445in}}%
\pgfpathcurveto{\pgfqpoint{2.063395in}{2.272653in}}{\pgfqpoint{2.050904in}{2.277827in}}{\pgfqpoint{2.037881in}{2.277827in}}%
\pgfpathcurveto{\pgfqpoint{2.024858in}{2.277827in}}{\pgfqpoint{2.012367in}{2.272653in}}{\pgfqpoint{2.003159in}{2.263445in}}%
\pgfpathcurveto{\pgfqpoint{1.993950in}{2.254237in}}{\pgfqpoint{1.988776in}{2.241745in}}{\pgfqpoint{1.988776in}{2.228723in}}%
\pgfpathcurveto{\pgfqpoint{1.988776in}{2.215700in}}{\pgfqpoint{1.993950in}{2.203209in}}{\pgfqpoint{2.003159in}{2.194001in}}%
\pgfpathcurveto{\pgfqpoint{2.012367in}{2.184792in}}{\pgfqpoint{2.024858in}{2.179618in}}{\pgfqpoint{2.037881in}{2.179618in}}%
\pgfpathlineto{\pgfqpoint{2.037881in}{2.179618in}}%
\pgfpathclose%
\pgfusepath{stroke,fill}%
\end{pgfscope}%
\begin{pgfscope}%
\pgfpathrectangle{\pgfqpoint{0.786164in}{0.768110in}}{\pgfqpoint{8.851069in}{7.081890in}}%
\pgfusepath{clip}%
\pgfsetbuttcap%
\pgfsetroundjoin%
\definecolor{currentfill}{rgb}{0.146180,0.515413,0.556823}%
\pgfsetfillcolor{currentfill}%
\pgfsetfillopacity{0.700000}%
\pgfsetlinewidth{0.501875pt}%
\definecolor{currentstroke}{rgb}{1.000000,1.000000,1.000000}%
\pgfsetstrokecolor{currentstroke}%
\pgfsetstrokeopacity{0.700000}%
\pgfsetdash{}{0pt}%
\pgfpathmoveto{\pgfqpoint{1.910015in}{2.026330in}}%
\pgfpathcurveto{\pgfqpoint{1.923038in}{2.026330in}}{\pgfqpoint{1.935529in}{2.031504in}}{\pgfqpoint{1.944737in}{2.040713in}}%
\pgfpathcurveto{\pgfqpoint{1.953946in}{2.049921in}}{\pgfqpoint{1.959120in}{2.062412in}}{\pgfqpoint{1.959120in}{2.075435in}}%
\pgfpathcurveto{\pgfqpoint{1.959120in}{2.088458in}}{\pgfqpoint{1.953946in}{2.100949in}}{\pgfqpoint{1.944737in}{2.110157in}}%
\pgfpathcurveto{\pgfqpoint{1.935529in}{2.119366in}}{\pgfqpoint{1.923038in}{2.124540in}}{\pgfqpoint{1.910015in}{2.124540in}}%
\pgfpathcurveto{\pgfqpoint{1.896992in}{2.124540in}}{\pgfqpoint{1.884501in}{2.119366in}}{\pgfqpoint{1.875293in}{2.110157in}}%
\pgfpathcurveto{\pgfqpoint{1.866084in}{2.100949in}}{\pgfqpoint{1.860910in}{2.088458in}}{\pgfqpoint{1.860910in}{2.075435in}}%
\pgfpathcurveto{\pgfqpoint{1.860910in}{2.062412in}}{\pgfqpoint{1.866084in}{2.049921in}}{\pgfqpoint{1.875293in}{2.040713in}}%
\pgfpathcurveto{\pgfqpoint{1.884501in}{2.031504in}}{\pgfqpoint{1.896992in}{2.026330in}}{\pgfqpoint{1.910015in}{2.026330in}}%
\pgfpathlineto{\pgfqpoint{1.910015in}{2.026330in}}%
\pgfpathclose%
\pgfusepath{stroke,fill}%
\end{pgfscope}%
\begin{pgfscope}%
\pgfpathrectangle{\pgfqpoint{0.786164in}{0.768110in}}{\pgfqpoint{8.851069in}{7.081890in}}%
\pgfusepath{clip}%
\pgfsetbuttcap%
\pgfsetroundjoin%
\definecolor{currentfill}{rgb}{0.271305,0.019942,0.347269}%
\pgfsetfillcolor{currentfill}%
\pgfsetfillopacity{0.700000}%
\pgfsetlinewidth{0.501875pt}%
\definecolor{currentstroke}{rgb}{1.000000,1.000000,1.000000}%
\pgfsetstrokecolor{currentstroke}%
\pgfsetstrokeopacity{0.700000}%
\pgfsetdash{}{0pt}%
\pgfpathmoveto{\pgfqpoint{9.161845in}{7.435195in}}%
\pgfpathcurveto{\pgfqpoint{9.174868in}{7.435195in}}{\pgfqpoint{9.187359in}{7.440369in}}{\pgfqpoint{9.196567in}{7.449577in}}%
\pgfpathcurveto{\pgfqpoint{9.205776in}{7.458786in}}{\pgfqpoint{9.210950in}{7.471277in}}{\pgfqpoint{9.210950in}{7.484299in}}%
\pgfpathcurveto{\pgfqpoint{9.210950in}{7.497322in}}{\pgfqpoint{9.205776in}{7.509813in}}{\pgfqpoint{9.196567in}{7.519022in}}%
\pgfpathcurveto{\pgfqpoint{9.187359in}{7.528230in}}{\pgfqpoint{9.174868in}{7.533404in}}{\pgfqpoint{9.161845in}{7.533404in}}%
\pgfpathcurveto{\pgfqpoint{9.148822in}{7.533404in}}{\pgfqpoint{9.136331in}{7.528230in}}{\pgfqpoint{9.127123in}{7.519022in}}%
\pgfpathcurveto{\pgfqpoint{9.117915in}{7.509813in}}{\pgfqpoint{9.112741in}{7.497322in}}{\pgfqpoint{9.112741in}{7.484299in}}%
\pgfpathcurveto{\pgfqpoint{9.112741in}{7.471277in}}{\pgfqpoint{9.117915in}{7.458786in}}{\pgfqpoint{9.127123in}{7.449577in}}%
\pgfpathcurveto{\pgfqpoint{9.136331in}{7.440369in}}{\pgfqpoint{9.148822in}{7.435195in}}{\pgfqpoint{9.161845in}{7.435195in}}%
\pgfpathlineto{\pgfqpoint{9.161845in}{7.435195in}}%
\pgfpathclose%
\pgfusepath{stroke,fill}%
\end{pgfscope}%
\begin{pgfscope}%
\pgfpathrectangle{\pgfqpoint{0.786164in}{0.768110in}}{\pgfqpoint{8.851069in}{7.081890in}}%
\pgfusepath{clip}%
\pgfsetbuttcap%
\pgfsetroundjoin%
\definecolor{currentfill}{rgb}{0.273809,0.031497,0.358853}%
\pgfsetfillcolor{currentfill}%
\pgfsetfillopacity{0.700000}%
\pgfsetlinewidth{0.501875pt}%
\definecolor{currentstroke}{rgb}{1.000000,1.000000,1.000000}%
\pgfsetstrokecolor{currentstroke}%
\pgfsetstrokeopacity{0.700000}%
\pgfsetdash{}{0pt}%
\pgfpathmoveto{\pgfqpoint{9.234911in}{7.478991in}}%
\pgfpathcurveto{\pgfqpoint{9.247934in}{7.478991in}}{\pgfqpoint{9.260425in}{7.484165in}}{\pgfqpoint{9.269634in}{7.493374in}}%
\pgfpathcurveto{\pgfqpoint{9.278842in}{7.502582in}}{\pgfqpoint{9.284016in}{7.515073in}}{\pgfqpoint{9.284016in}{7.528096in}}%
\pgfpathcurveto{\pgfqpoint{9.284016in}{7.541119in}}{\pgfqpoint{9.278842in}{7.553610in}}{\pgfqpoint{9.269634in}{7.562818in}}%
\pgfpathcurveto{\pgfqpoint{9.260425in}{7.572027in}}{\pgfqpoint{9.247934in}{7.577201in}}{\pgfqpoint{9.234911in}{7.577201in}}%
\pgfpathcurveto{\pgfqpoint{9.221889in}{7.577201in}}{\pgfqpoint{9.209398in}{7.572027in}}{\pgfqpoint{9.200189in}{7.562818in}}%
\pgfpathcurveto{\pgfqpoint{9.190981in}{7.553610in}}{\pgfqpoint{9.185807in}{7.541119in}}{\pgfqpoint{9.185807in}{7.528096in}}%
\pgfpathcurveto{\pgfqpoint{9.185807in}{7.515073in}}{\pgfqpoint{9.190981in}{7.502582in}}{\pgfqpoint{9.200189in}{7.493374in}}%
\pgfpathcurveto{\pgfqpoint{9.209398in}{7.484165in}}{\pgfqpoint{9.221889in}{7.478991in}}{\pgfqpoint{9.234911in}{7.478991in}}%
\pgfpathlineto{\pgfqpoint{9.234911in}{7.478991in}}%
\pgfpathclose%
\pgfusepath{stroke,fill}%
\end{pgfscope}%
\begin{pgfscope}%
\pgfpathrectangle{\pgfqpoint{0.786164in}{0.768110in}}{\pgfqpoint{8.851069in}{7.081890in}}%
\pgfusepath{clip}%
\pgfsetbuttcap%
\pgfsetroundjoin%
\definecolor{currentfill}{rgb}{0.273809,0.031497,0.358853}%
\pgfsetfillcolor{currentfill}%
\pgfsetfillopacity{0.700000}%
\pgfsetlinewidth{0.501875pt}%
\definecolor{currentstroke}{rgb}{1.000000,1.000000,1.000000}%
\pgfsetstrokecolor{currentstroke}%
\pgfsetstrokeopacity{0.700000}%
\pgfsetdash{}{0pt}%
\pgfpathmoveto{\pgfqpoint{8.933513in}{7.303805in}}%
\pgfpathcurveto{\pgfqpoint{8.946536in}{7.303805in}}{\pgfqpoint{8.959027in}{7.308979in}}{\pgfqpoint{8.968235in}{7.318188in}}%
\pgfpathcurveto{\pgfqpoint{8.977444in}{7.327396in}}{\pgfqpoint{8.982618in}{7.339887in}}{\pgfqpoint{8.982618in}{7.352910in}}%
\pgfpathcurveto{\pgfqpoint{8.982618in}{7.365933in}}{\pgfqpoint{8.977444in}{7.378424in}}{\pgfqpoint{8.968235in}{7.387632in}}%
\pgfpathcurveto{\pgfqpoint{8.959027in}{7.396841in}}{\pgfqpoint{8.946536in}{7.402015in}}{\pgfqpoint{8.933513in}{7.402015in}}%
\pgfpathcurveto{\pgfqpoint{8.920490in}{7.402015in}}{\pgfqpoint{8.907999in}{7.396841in}}{\pgfqpoint{8.898791in}{7.387632in}}%
\pgfpathcurveto{\pgfqpoint{8.889582in}{7.378424in}}{\pgfqpoint{8.884408in}{7.365933in}}{\pgfqpoint{8.884408in}{7.352910in}}%
\pgfpathcurveto{\pgfqpoint{8.884408in}{7.339887in}}{\pgfqpoint{8.889582in}{7.327396in}}{\pgfqpoint{8.898791in}{7.318188in}}%
\pgfpathcurveto{\pgfqpoint{8.907999in}{7.308979in}}{\pgfqpoint{8.920490in}{7.303805in}}{\pgfqpoint{8.933513in}{7.303805in}}%
\pgfpathlineto{\pgfqpoint{8.933513in}{7.303805in}}%
\pgfpathclose%
\pgfusepath{stroke,fill}%
\end{pgfscope}%
\begin{pgfscope}%
\pgfpathrectangle{\pgfqpoint{0.786164in}{0.768110in}}{\pgfqpoint{8.851069in}{7.081890in}}%
\pgfusepath{clip}%
\pgfsetbuttcap%
\pgfsetroundjoin%
\definecolor{currentfill}{rgb}{0.278791,0.062145,0.386592}%
\pgfsetfillcolor{currentfill}%
\pgfsetfillopacity{0.700000}%
\pgfsetlinewidth{0.501875pt}%
\definecolor{currentstroke}{rgb}{1.000000,1.000000,1.000000}%
\pgfsetstrokecolor{currentstroke}%
\pgfsetstrokeopacity{0.700000}%
\pgfsetdash{}{0pt}%
\pgfpathmoveto{\pgfqpoint{8.641248in}{7.019128in}}%
\pgfpathcurveto{\pgfqpoint{8.654270in}{7.019128in}}{\pgfqpoint{8.666762in}{7.024302in}}{\pgfqpoint{8.675970in}{7.033511in}}%
\pgfpathcurveto{\pgfqpoint{8.685178in}{7.042719in}}{\pgfqpoint{8.690352in}{7.055210in}}{\pgfqpoint{8.690352in}{7.068233in}}%
\pgfpathcurveto{\pgfqpoint{8.690352in}{7.081256in}}{\pgfqpoint{8.685178in}{7.093747in}}{\pgfqpoint{8.675970in}{7.102955in}}%
\pgfpathcurveto{\pgfqpoint{8.666762in}{7.112164in}}{\pgfqpoint{8.654270in}{7.117338in}}{\pgfqpoint{8.641248in}{7.117338in}}%
\pgfpathcurveto{\pgfqpoint{8.628225in}{7.117338in}}{\pgfqpoint{8.615734in}{7.112164in}}{\pgfqpoint{8.606526in}{7.102955in}}%
\pgfpathcurveto{\pgfqpoint{8.597317in}{7.093747in}}{\pgfqpoint{8.592143in}{7.081256in}}{\pgfqpoint{8.592143in}{7.068233in}}%
\pgfpathcurveto{\pgfqpoint{8.592143in}{7.055210in}}{\pgfqpoint{8.597317in}{7.042719in}}{\pgfqpoint{8.606526in}{7.033511in}}%
\pgfpathcurveto{\pgfqpoint{8.615734in}{7.024302in}}{\pgfqpoint{8.628225in}{7.019128in}}{\pgfqpoint{8.641248in}{7.019128in}}%
\pgfpathlineto{\pgfqpoint{8.641248in}{7.019128in}}%
\pgfpathclose%
\pgfusepath{stroke,fill}%
\end{pgfscope}%
\begin{pgfscope}%
\pgfpathrectangle{\pgfqpoint{0.786164in}{0.768110in}}{\pgfqpoint{8.851069in}{7.081890in}}%
\pgfusepath{clip}%
\pgfsetbuttcap%
\pgfsetroundjoin%
\definecolor{currentfill}{rgb}{0.278791,0.062145,0.386592}%
\pgfsetfillcolor{currentfill}%
\pgfsetfillopacity{0.700000}%
\pgfsetlinewidth{0.501875pt}%
\definecolor{currentstroke}{rgb}{1.000000,1.000000,1.000000}%
\pgfsetstrokecolor{currentstroke}%
\pgfsetstrokeopacity{0.700000}%
\pgfsetdash{}{0pt}%
\pgfpathmoveto{\pgfqpoint{8.485982in}{6.865841in}}%
\pgfpathcurveto{\pgfqpoint{8.499005in}{6.865841in}}{\pgfqpoint{8.511496in}{6.871015in}}{\pgfqpoint{8.520704in}{6.880223in}}%
\pgfpathcurveto{\pgfqpoint{8.529913in}{6.889432in}}{\pgfqpoint{8.535087in}{6.901923in}}{\pgfqpoint{8.535087in}{6.914945in}}%
\pgfpathcurveto{\pgfqpoint{8.535087in}{6.927968in}}{\pgfqpoint{8.529913in}{6.940459in}}{\pgfqpoint{8.520704in}{6.949668in}}%
\pgfpathcurveto{\pgfqpoint{8.511496in}{6.958876in}}{\pgfqpoint{8.499005in}{6.964050in}}{\pgfqpoint{8.485982in}{6.964050in}}%
\pgfpathcurveto{\pgfqpoint{8.472959in}{6.964050in}}{\pgfqpoint{8.460468in}{6.958876in}}{\pgfqpoint{8.451260in}{6.949668in}}%
\pgfpathcurveto{\pgfqpoint{8.442051in}{6.940459in}}{\pgfqpoint{8.436877in}{6.927968in}}{\pgfqpoint{8.436877in}{6.914945in}}%
\pgfpathcurveto{\pgfqpoint{8.436877in}{6.901923in}}{\pgfqpoint{8.442051in}{6.889432in}}{\pgfqpoint{8.451260in}{6.880223in}}%
\pgfpathcurveto{\pgfqpoint{8.460468in}{6.871015in}}{\pgfqpoint{8.472959in}{6.865841in}}{\pgfqpoint{8.485982in}{6.865841in}}%
\pgfpathlineto{\pgfqpoint{8.485982in}{6.865841in}}%
\pgfpathclose%
\pgfusepath{stroke,fill}%
\end{pgfscope}%
\begin{pgfscope}%
\pgfpathrectangle{\pgfqpoint{0.786164in}{0.768110in}}{\pgfqpoint{8.851069in}{7.081890in}}%
\pgfusepath{clip}%
\pgfsetbuttcap%
\pgfsetroundjoin%
\definecolor{currentfill}{rgb}{0.278791,0.062145,0.386592}%
\pgfsetfillcolor{currentfill}%
\pgfsetfillopacity{0.700000}%
\pgfsetlinewidth{0.501875pt}%
\definecolor{currentstroke}{rgb}{1.000000,1.000000,1.000000}%
\pgfsetstrokecolor{currentstroke}%
\pgfsetstrokeopacity{0.700000}%
\pgfsetdash{}{0pt}%
\pgfpathmoveto{\pgfqpoint{7.837518in}{6.493571in}}%
\pgfpathcurveto{\pgfqpoint{7.850541in}{6.493571in}}{\pgfqpoint{7.863032in}{6.498745in}}{\pgfqpoint{7.872241in}{6.507953in}}%
\pgfpathcurveto{\pgfqpoint{7.881449in}{6.517162in}}{\pgfqpoint{7.886623in}{6.529653in}}{\pgfqpoint{7.886623in}{6.542675in}}%
\pgfpathcurveto{\pgfqpoint{7.886623in}{6.555698in}}{\pgfqpoint{7.881449in}{6.568189in}}{\pgfqpoint{7.872241in}{6.577398in}}%
\pgfpathcurveto{\pgfqpoint{7.863032in}{6.586606in}}{\pgfqpoint{7.850541in}{6.591780in}}{\pgfqpoint{7.837518in}{6.591780in}}%
\pgfpathcurveto{\pgfqpoint{7.824496in}{6.591780in}}{\pgfqpoint{7.812005in}{6.586606in}}{\pgfqpoint{7.802796in}{6.577398in}}%
\pgfpathcurveto{\pgfqpoint{7.793588in}{6.568189in}}{\pgfqpoint{7.788414in}{6.555698in}}{\pgfqpoint{7.788414in}{6.542675in}}%
\pgfpathcurveto{\pgfqpoint{7.788414in}{6.529653in}}{\pgfqpoint{7.793588in}{6.517162in}}{\pgfqpoint{7.802796in}{6.507953in}}%
\pgfpathcurveto{\pgfqpoint{7.812005in}{6.498745in}}{\pgfqpoint{7.824496in}{6.493571in}}{\pgfqpoint{7.837518in}{6.493571in}}%
\pgfpathlineto{\pgfqpoint{7.837518in}{6.493571in}}%
\pgfpathclose%
\pgfusepath{stroke,fill}%
\end{pgfscope}%
\begin{pgfscope}%
\pgfpathrectangle{\pgfqpoint{0.786164in}{0.768110in}}{\pgfqpoint{8.851069in}{7.081890in}}%
\pgfusepath{clip}%
\pgfsetbuttcap%
\pgfsetroundjoin%
\definecolor{currentfill}{rgb}{0.278791,0.062145,0.386592}%
\pgfsetfillcolor{currentfill}%
\pgfsetfillopacity{0.700000}%
\pgfsetlinewidth{0.501875pt}%
\definecolor{currentstroke}{rgb}{1.000000,1.000000,1.000000}%
\pgfsetstrokecolor{currentstroke}%
\pgfsetstrokeopacity{0.700000}%
\pgfsetdash{}{0pt}%
\pgfpathmoveto{\pgfqpoint{8.175450in}{6.734451in}}%
\pgfpathcurveto{\pgfqpoint{8.188473in}{6.734451in}}{\pgfqpoint{8.200964in}{6.739625in}}{\pgfqpoint{8.210172in}{6.748834in}}%
\pgfpathcurveto{\pgfqpoint{8.219381in}{6.758042in}}{\pgfqpoint{8.224555in}{6.770533in}}{\pgfqpoint{8.224555in}{6.783556in}}%
\pgfpathcurveto{\pgfqpoint{8.224555in}{6.796579in}}{\pgfqpoint{8.219381in}{6.809070in}}{\pgfqpoint{8.210172in}{6.818278in}}%
\pgfpathcurveto{\pgfqpoint{8.200964in}{6.827487in}}{\pgfqpoint{8.188473in}{6.832661in}}{\pgfqpoint{8.175450in}{6.832661in}}%
\pgfpathcurveto{\pgfqpoint{8.162427in}{6.832661in}}{\pgfqpoint{8.149936in}{6.827487in}}{\pgfqpoint{8.140728in}{6.818278in}}%
\pgfpathcurveto{\pgfqpoint{8.131519in}{6.809070in}}{\pgfqpoint{8.126345in}{6.796579in}}{\pgfqpoint{8.126345in}{6.783556in}}%
\pgfpathcurveto{\pgfqpoint{8.126345in}{6.770533in}}{\pgfqpoint{8.131519in}{6.758042in}}{\pgfqpoint{8.140728in}{6.748834in}}%
\pgfpathcurveto{\pgfqpoint{8.149936in}{6.739625in}}{\pgfqpoint{8.162427in}{6.734451in}}{\pgfqpoint{8.175450in}{6.734451in}}%
\pgfpathlineto{\pgfqpoint{8.175450in}{6.734451in}}%
\pgfpathclose%
\pgfusepath{stroke,fill}%
\end{pgfscope}%
\begin{pgfscope}%
\pgfpathrectangle{\pgfqpoint{0.786164in}{0.768110in}}{\pgfqpoint{8.851069in}{7.081890in}}%
\pgfusepath{clip}%
\pgfsetbuttcap%
\pgfsetroundjoin%
\definecolor{currentfill}{rgb}{0.278791,0.062145,0.386592}%
\pgfsetfillcolor{currentfill}%
\pgfsetfillopacity{0.700000}%
\pgfsetlinewidth{0.501875pt}%
\definecolor{currentstroke}{rgb}{1.000000,1.000000,1.000000}%
\pgfsetstrokecolor{currentstroke}%
\pgfsetstrokeopacity{0.700000}%
\pgfsetdash{}{0pt}%
\pgfpathmoveto{\pgfqpoint{7.855785in}{6.493571in}}%
\pgfpathcurveto{\pgfqpoint{7.868808in}{6.493571in}}{\pgfqpoint{7.881299in}{6.498745in}}{\pgfqpoint{7.890507in}{6.507953in}}%
\pgfpathcurveto{\pgfqpoint{7.899716in}{6.517162in}}{\pgfqpoint{7.904890in}{6.529653in}}{\pgfqpoint{7.904890in}{6.542675in}}%
\pgfpathcurveto{\pgfqpoint{7.904890in}{6.555698in}}{\pgfqpoint{7.899716in}{6.568189in}}{\pgfqpoint{7.890507in}{6.577398in}}%
\pgfpathcurveto{\pgfqpoint{7.881299in}{6.586606in}}{\pgfqpoint{7.868808in}{6.591780in}}{\pgfqpoint{7.855785in}{6.591780in}}%
\pgfpathcurveto{\pgfqpoint{7.842762in}{6.591780in}}{\pgfqpoint{7.830271in}{6.586606in}}{\pgfqpoint{7.821063in}{6.577398in}}%
\pgfpathcurveto{\pgfqpoint{7.811854in}{6.568189in}}{\pgfqpoint{7.806680in}{6.555698in}}{\pgfqpoint{7.806680in}{6.542675in}}%
\pgfpathcurveto{\pgfqpoint{7.806680in}{6.529653in}}{\pgfqpoint{7.811854in}{6.517162in}}{\pgfqpoint{7.821063in}{6.507953in}}%
\pgfpathcurveto{\pgfqpoint{7.830271in}{6.498745in}}{\pgfqpoint{7.842762in}{6.493571in}}{\pgfqpoint{7.855785in}{6.493571in}}%
\pgfpathlineto{\pgfqpoint{7.855785in}{6.493571in}}%
\pgfpathclose%
\pgfusepath{stroke,fill}%
\end{pgfscope}%
\begin{pgfscope}%
\pgfpathrectangle{\pgfqpoint{0.786164in}{0.768110in}}{\pgfqpoint{8.851069in}{7.081890in}}%
\pgfusepath{clip}%
\pgfsetbuttcap%
\pgfsetroundjoin%
\definecolor{currentfill}{rgb}{0.279566,0.067836,0.391917}%
\pgfsetfillcolor{currentfill}%
\pgfsetfillopacity{0.700000}%
\pgfsetlinewidth{0.501875pt}%
\definecolor{currentstroke}{rgb}{1.000000,1.000000,1.000000}%
\pgfsetstrokecolor{currentstroke}%
\pgfsetstrokeopacity{0.700000}%
\pgfsetdash{}{0pt}%
\pgfpathmoveto{\pgfqpoint{7.472187in}{6.208894in}}%
\pgfpathcurveto{\pgfqpoint{7.485210in}{6.208894in}}{\pgfqpoint{7.497701in}{6.214068in}}{\pgfqpoint{7.506909in}{6.223276in}}%
\pgfpathcurveto{\pgfqpoint{7.516118in}{6.232484in}}{\pgfqpoint{7.521292in}{6.244976in}}{\pgfqpoint{7.521292in}{6.257998in}}%
\pgfpathcurveto{\pgfqpoint{7.521292in}{6.271021in}}{\pgfqpoint{7.516118in}{6.283512in}}{\pgfqpoint{7.506909in}{6.292720in}}%
\pgfpathcurveto{\pgfqpoint{7.497701in}{6.301929in}}{\pgfqpoint{7.485210in}{6.307103in}}{\pgfqpoint{7.472187in}{6.307103in}}%
\pgfpathcurveto{\pgfqpoint{7.459164in}{6.307103in}}{\pgfqpoint{7.446673in}{6.301929in}}{\pgfqpoint{7.437465in}{6.292720in}}%
\pgfpathcurveto{\pgfqpoint{7.428256in}{6.283512in}}{\pgfqpoint{7.423082in}{6.271021in}}{\pgfqpoint{7.423082in}{6.257998in}}%
\pgfpathcurveto{\pgfqpoint{7.423082in}{6.244976in}}{\pgfqpoint{7.428256in}{6.232484in}}{\pgfqpoint{7.437465in}{6.223276in}}%
\pgfpathcurveto{\pgfqpoint{7.446673in}{6.214068in}}{\pgfqpoint{7.459164in}{6.208894in}}{\pgfqpoint{7.472187in}{6.208894in}}%
\pgfpathlineto{\pgfqpoint{7.472187in}{6.208894in}}%
\pgfpathclose%
\pgfusepath{stroke,fill}%
\end{pgfscope}%
\begin{pgfscope}%
\pgfpathrectangle{\pgfqpoint{0.786164in}{0.768110in}}{\pgfqpoint{8.851069in}{7.081890in}}%
\pgfusepath{clip}%
\pgfsetbuttcap%
\pgfsetroundjoin%
\definecolor{currentfill}{rgb}{0.280894,0.078907,0.402329}%
\pgfsetfillcolor{currentfill}%
\pgfsetfillopacity{0.700000}%
\pgfsetlinewidth{0.501875pt}%
\definecolor{currentstroke}{rgb}{1.000000,1.000000,1.000000}%
\pgfsetstrokecolor{currentstroke}%
\pgfsetstrokeopacity{0.700000}%
\pgfsetdash{}{0pt}%
\pgfpathmoveto{\pgfqpoint{7.097722in}{5.814725in}}%
\pgfpathcurveto{\pgfqpoint{7.110745in}{5.814725in}}{\pgfqpoint{7.123236in}{5.819899in}}{\pgfqpoint{7.132444in}{5.829108in}}%
\pgfpathcurveto{\pgfqpoint{7.141653in}{5.838316in}}{\pgfqpoint{7.146827in}{5.850807in}}{\pgfqpoint{7.146827in}{5.863830in}}%
\pgfpathcurveto{\pgfqpoint{7.146827in}{5.876853in}}{\pgfqpoint{7.141653in}{5.889344in}}{\pgfqpoint{7.132444in}{5.898552in}}%
\pgfpathcurveto{\pgfqpoint{7.123236in}{5.907761in}}{\pgfqpoint{7.110745in}{5.912935in}}{\pgfqpoint{7.097722in}{5.912935in}}%
\pgfpathcurveto{\pgfqpoint{7.084699in}{5.912935in}}{\pgfqpoint{7.072208in}{5.907761in}}{\pgfqpoint{7.063000in}{5.898552in}}%
\pgfpathcurveto{\pgfqpoint{7.053792in}{5.889344in}}{\pgfqpoint{7.048618in}{5.876853in}}{\pgfqpoint{7.048618in}{5.863830in}}%
\pgfpathcurveto{\pgfqpoint{7.048618in}{5.850807in}}{\pgfqpoint{7.053792in}{5.838316in}}{\pgfqpoint{7.063000in}{5.829108in}}%
\pgfpathcurveto{\pgfqpoint{7.072208in}{5.819899in}}{\pgfqpoint{7.084699in}{5.814725in}}{\pgfqpoint{7.097722in}{5.814725in}}%
\pgfpathlineto{\pgfqpoint{7.097722in}{5.814725in}}%
\pgfpathclose%
\pgfusepath{stroke,fill}%
\end{pgfscope}%
\begin{pgfscope}%
\pgfpathrectangle{\pgfqpoint{0.786164in}{0.768110in}}{\pgfqpoint{8.851069in}{7.081890in}}%
\pgfusepath{clip}%
\pgfsetbuttcap%
\pgfsetroundjoin%
\definecolor{currentfill}{rgb}{0.282327,0.094955,0.417331}%
\pgfsetfillcolor{currentfill}%
\pgfsetfillopacity{0.700000}%
\pgfsetlinewidth{0.501875pt}%
\definecolor{currentstroke}{rgb}{1.000000,1.000000,1.000000}%
\pgfsetstrokecolor{currentstroke}%
\pgfsetstrokeopacity{0.700000}%
\pgfsetdash{}{0pt}%
\pgfpathmoveto{\pgfqpoint{6.750657in}{5.573845in}}%
\pgfpathcurveto{\pgfqpoint{6.763680in}{5.573845in}}{\pgfqpoint{6.776171in}{5.579019in}}{\pgfqpoint{6.785379in}{5.588227in}}%
\pgfpathcurveto{\pgfqpoint{6.794588in}{5.597436in}}{\pgfqpoint{6.799762in}{5.609927in}}{\pgfqpoint{6.799762in}{5.622949in}}%
\pgfpathcurveto{\pgfqpoint{6.799762in}{5.635972in}}{\pgfqpoint{6.794588in}{5.648463in}}{\pgfqpoint{6.785379in}{5.657672in}}%
\pgfpathcurveto{\pgfqpoint{6.776171in}{5.666880in}}{\pgfqpoint{6.763680in}{5.672054in}}{\pgfqpoint{6.750657in}{5.672054in}}%
\pgfpathcurveto{\pgfqpoint{6.737635in}{5.672054in}}{\pgfqpoint{6.725143in}{5.666880in}}{\pgfqpoint{6.715935in}{5.657672in}}%
\pgfpathcurveto{\pgfqpoint{6.706727in}{5.648463in}}{\pgfqpoint{6.701553in}{5.635972in}}{\pgfqpoint{6.701553in}{5.622949in}}%
\pgfpathcurveto{\pgfqpoint{6.701553in}{5.609927in}}{\pgfqpoint{6.706727in}{5.597436in}}{\pgfqpoint{6.715935in}{5.588227in}}%
\pgfpathcurveto{\pgfqpoint{6.725143in}{5.579019in}}{\pgfqpoint{6.737635in}{5.573845in}}{\pgfqpoint{6.750657in}{5.573845in}}%
\pgfpathlineto{\pgfqpoint{6.750657in}{5.573845in}}%
\pgfpathclose%
\pgfusepath{stroke,fill}%
\end{pgfscope}%
\begin{pgfscope}%
\pgfpathrectangle{\pgfqpoint{0.786164in}{0.768110in}}{\pgfqpoint{8.851069in}{7.081890in}}%
\pgfusepath{clip}%
\pgfsetbuttcap%
\pgfsetroundjoin%
\definecolor{currentfill}{rgb}{0.282910,0.105393,0.426902}%
\pgfsetfillcolor{currentfill}%
\pgfsetfillopacity{0.700000}%
\pgfsetlinewidth{0.501875pt}%
\definecolor{currentstroke}{rgb}{1.000000,1.000000,1.000000}%
\pgfsetstrokecolor{currentstroke}%
\pgfsetstrokeopacity{0.700000}%
\pgfsetdash{}{0pt}%
\pgfpathmoveto{\pgfqpoint{6.513192in}{5.376761in}}%
\pgfpathcurveto{\pgfqpoint{6.526215in}{5.376761in}}{\pgfqpoint{6.538706in}{5.381935in}}{\pgfqpoint{6.547914in}{5.391143in}}%
\pgfpathcurveto{\pgfqpoint{6.557122in}{5.400351in}}{\pgfqpoint{6.562296in}{5.412843in}}{\pgfqpoint{6.562296in}{5.425865in}}%
\pgfpathcurveto{\pgfqpoint{6.562296in}{5.438888in}}{\pgfqpoint{6.557122in}{5.451379in}}{\pgfqpoint{6.547914in}{5.460587in}}%
\pgfpathcurveto{\pgfqpoint{6.538706in}{5.469796in}}{\pgfqpoint{6.526215in}{5.474970in}}{\pgfqpoint{6.513192in}{5.474970in}}%
\pgfpathcurveto{\pgfqpoint{6.500169in}{5.474970in}}{\pgfqpoint{6.487678in}{5.469796in}}{\pgfqpoint{6.478470in}{5.460587in}}%
\pgfpathcurveto{\pgfqpoint{6.469261in}{5.451379in}}{\pgfqpoint{6.464087in}{5.438888in}}{\pgfqpoint{6.464087in}{5.425865in}}%
\pgfpathcurveto{\pgfqpoint{6.464087in}{5.412843in}}{\pgfqpoint{6.469261in}{5.400351in}}{\pgfqpoint{6.478470in}{5.391143in}}%
\pgfpathcurveto{\pgfqpoint{6.487678in}{5.381935in}}{\pgfqpoint{6.500169in}{5.376761in}}{\pgfqpoint{6.513192in}{5.376761in}}%
\pgfpathlineto{\pgfqpoint{6.513192in}{5.376761in}}%
\pgfpathclose%
\pgfusepath{stroke,fill}%
\end{pgfscope}%
\begin{pgfscope}%
\pgfpathrectangle{\pgfqpoint{0.786164in}{0.768110in}}{\pgfqpoint{8.851069in}{7.081890in}}%
\pgfusepath{clip}%
\pgfsetbuttcap%
\pgfsetroundjoin%
\definecolor{currentfill}{rgb}{0.283197,0.115680,0.436115}%
\pgfsetfillcolor{currentfill}%
\pgfsetfillopacity{0.700000}%
\pgfsetlinewidth{0.501875pt}%
\definecolor{currentstroke}{rgb}{1.000000,1.000000,1.000000}%
\pgfsetstrokecolor{currentstroke}%
\pgfsetstrokeopacity{0.700000}%
\pgfsetdash{}{0pt}%
\pgfpathmoveto{\pgfqpoint{6.385326in}{5.201575in}}%
\pgfpathcurveto{\pgfqpoint{6.398348in}{5.201575in}}{\pgfqpoint{6.410840in}{5.206749in}}{\pgfqpoint{6.420048in}{5.215957in}}%
\pgfpathcurveto{\pgfqpoint{6.429256in}{5.225166in}}{\pgfqpoint{6.434430in}{5.237657in}}{\pgfqpoint{6.434430in}{5.250679in}}%
\pgfpathcurveto{\pgfqpoint{6.434430in}{5.263702in}}{\pgfqpoint{6.429256in}{5.276193in}}{\pgfqpoint{6.420048in}{5.285402in}}%
\pgfpathcurveto{\pgfqpoint{6.410840in}{5.294610in}}{\pgfqpoint{6.398348in}{5.299784in}}{\pgfqpoint{6.385326in}{5.299784in}}%
\pgfpathcurveto{\pgfqpoint{6.372303in}{5.299784in}}{\pgfqpoint{6.359812in}{5.294610in}}{\pgfqpoint{6.350604in}{5.285402in}}%
\pgfpathcurveto{\pgfqpoint{6.341395in}{5.276193in}}{\pgfqpoint{6.336221in}{5.263702in}}{\pgfqpoint{6.336221in}{5.250679in}}%
\pgfpathcurveto{\pgfqpoint{6.336221in}{5.237657in}}{\pgfqpoint{6.341395in}{5.225166in}}{\pgfqpoint{6.350604in}{5.215957in}}%
\pgfpathcurveto{\pgfqpoint{6.359812in}{5.206749in}}{\pgfqpoint{6.372303in}{5.201575in}}{\pgfqpoint{6.385326in}{5.201575in}}%
\pgfpathlineto{\pgfqpoint{6.385326in}{5.201575in}}%
\pgfpathclose%
\pgfusepath{stroke,fill}%
\end{pgfscope}%
\begin{pgfscope}%
\pgfpathrectangle{\pgfqpoint{0.786164in}{0.768110in}}{\pgfqpoint{8.851069in}{7.081890in}}%
\pgfusepath{clip}%
\pgfsetbuttcap%
\pgfsetroundjoin%
\definecolor{currentfill}{rgb}{0.283072,0.130895,0.449241}%
\pgfsetfillcolor{currentfill}%
\pgfsetfillopacity{0.700000}%
\pgfsetlinewidth{0.501875pt}%
\definecolor{currentstroke}{rgb}{1.000000,1.000000,1.000000}%
\pgfsetstrokecolor{currentstroke}%
\pgfsetstrokeopacity{0.700000}%
\pgfsetdash{}{0pt}%
\pgfpathmoveto{\pgfqpoint{6.440125in}{5.289168in}}%
\pgfpathcurveto{\pgfqpoint{6.453148in}{5.289168in}}{\pgfqpoint{6.465639in}{5.294342in}}{\pgfqpoint{6.474848in}{5.303550in}}%
\pgfpathcurveto{\pgfqpoint{6.484056in}{5.312759in}}{\pgfqpoint{6.489230in}{5.325250in}}{\pgfqpoint{6.489230in}{5.338272in}}%
\pgfpathcurveto{\pgfqpoint{6.489230in}{5.351295in}}{\pgfqpoint{6.484056in}{5.363786in}}{\pgfqpoint{6.474848in}{5.372995in}}%
\pgfpathcurveto{\pgfqpoint{6.465639in}{5.382203in}}{\pgfqpoint{6.453148in}{5.387377in}}{\pgfqpoint{6.440125in}{5.387377in}}%
\pgfpathcurveto{\pgfqpoint{6.427103in}{5.387377in}}{\pgfqpoint{6.414612in}{5.382203in}}{\pgfqpoint{6.405403in}{5.372995in}}%
\pgfpathcurveto{\pgfqpoint{6.396195in}{5.363786in}}{\pgfqpoint{6.391021in}{5.351295in}}{\pgfqpoint{6.391021in}{5.338272in}}%
\pgfpathcurveto{\pgfqpoint{6.391021in}{5.325250in}}{\pgfqpoint{6.396195in}{5.312759in}}{\pgfqpoint{6.405403in}{5.303550in}}%
\pgfpathcurveto{\pgfqpoint{6.414612in}{5.294342in}}{\pgfqpoint{6.427103in}{5.289168in}}{\pgfqpoint{6.440125in}{5.289168in}}%
\pgfpathlineto{\pgfqpoint{6.440125in}{5.289168in}}%
\pgfpathclose%
\pgfusepath{stroke,fill}%
\end{pgfscope}%
\begin{pgfscope}%
\pgfpathrectangle{\pgfqpoint{0.786164in}{0.768110in}}{\pgfqpoint{8.851069in}{7.081890in}}%
\pgfusepath{clip}%
\pgfsetbuttcap%
\pgfsetroundjoin%
\definecolor{currentfill}{rgb}{0.277134,0.185228,0.489898}%
\pgfsetfillcolor{currentfill}%
\pgfsetfillopacity{0.700000}%
\pgfsetlinewidth{0.501875pt}%
\definecolor{currentstroke}{rgb}{1.000000,1.000000,1.000000}%
\pgfsetstrokecolor{currentstroke}%
\pgfsetstrokeopacity{0.700000}%
\pgfsetdash{}{0pt}%
\pgfpathmoveto{\pgfqpoint{6.577125in}{5.398659in}}%
\pgfpathcurveto{\pgfqpoint{6.590148in}{5.398659in}}{\pgfqpoint{6.602639in}{5.403833in}}{\pgfqpoint{6.611847in}{5.413041in}}%
\pgfpathcurveto{\pgfqpoint{6.621055in}{5.422250in}}{\pgfqpoint{6.626229in}{5.434741in}}{\pgfqpoint{6.626229in}{5.447763in}}%
\pgfpathcurveto{\pgfqpoint{6.626229in}{5.460786in}}{\pgfqpoint{6.621055in}{5.473277in}}{\pgfqpoint{6.611847in}{5.482486in}}%
\pgfpathcurveto{\pgfqpoint{6.602639in}{5.491694in}}{\pgfqpoint{6.590148in}{5.496868in}}{\pgfqpoint{6.577125in}{5.496868in}}%
\pgfpathcurveto{\pgfqpoint{6.564102in}{5.496868in}}{\pgfqpoint{6.551611in}{5.491694in}}{\pgfqpoint{6.542403in}{5.482486in}}%
\pgfpathcurveto{\pgfqpoint{6.533194in}{5.473277in}}{\pgfqpoint{6.528020in}{5.460786in}}{\pgfqpoint{6.528020in}{5.447763in}}%
\pgfpathcurveto{\pgfqpoint{6.528020in}{5.434741in}}{\pgfqpoint{6.533194in}{5.422250in}}{\pgfqpoint{6.542403in}{5.413041in}}%
\pgfpathcurveto{\pgfqpoint{6.551611in}{5.403833in}}{\pgfqpoint{6.564102in}{5.398659in}}{\pgfqpoint{6.577125in}{5.398659in}}%
\pgfpathlineto{\pgfqpoint{6.577125in}{5.398659in}}%
\pgfpathclose%
\pgfusepath{stroke,fill}%
\end{pgfscope}%
\begin{pgfscope}%
\pgfpathrectangle{\pgfqpoint{0.786164in}{0.768110in}}{\pgfqpoint{8.851069in}{7.081890in}}%
\pgfusepath{clip}%
\pgfsetbuttcap%
\pgfsetroundjoin%
\definecolor{currentfill}{rgb}{0.282290,0.145912,0.461510}%
\pgfsetfillcolor{currentfill}%
\pgfsetfillopacity{0.700000}%
\pgfsetlinewidth{0.501875pt}%
\definecolor{currentstroke}{rgb}{1.000000,1.000000,1.000000}%
\pgfsetstrokecolor{currentstroke}%
\pgfsetstrokeopacity{0.700000}%
\pgfsetdash{}{0pt}%
\pgfpathmoveto{\pgfqpoint{6.504059in}{5.311066in}}%
\pgfpathcurveto{\pgfqpoint{6.517081in}{5.311066in}}{\pgfqpoint{6.529572in}{5.316240in}}{\pgfqpoint{6.538781in}{5.325448in}}%
\pgfpathcurveto{\pgfqpoint{6.547989in}{5.334657in}}{\pgfqpoint{6.553163in}{5.347148in}}{\pgfqpoint{6.553163in}{5.360171in}}%
\pgfpathcurveto{\pgfqpoint{6.553163in}{5.373193in}}{\pgfqpoint{6.547989in}{5.385684in}}{\pgfqpoint{6.538781in}{5.394893in}}%
\pgfpathcurveto{\pgfqpoint{6.529572in}{5.404101in}}{\pgfqpoint{6.517081in}{5.409275in}}{\pgfqpoint{6.504059in}{5.409275in}}%
\pgfpathcurveto{\pgfqpoint{6.491036in}{5.409275in}}{\pgfqpoint{6.478545in}{5.404101in}}{\pgfqpoint{6.469336in}{5.394893in}}%
\pgfpathcurveto{\pgfqpoint{6.460128in}{5.385684in}}{\pgfqpoint{6.454954in}{5.373193in}}{\pgfqpoint{6.454954in}{5.360171in}}%
\pgfpathcurveto{\pgfqpoint{6.454954in}{5.347148in}}{\pgfqpoint{6.460128in}{5.334657in}}{\pgfqpoint{6.469336in}{5.325448in}}%
\pgfpathcurveto{\pgfqpoint{6.478545in}{5.316240in}}{\pgfqpoint{6.491036in}{5.311066in}}{\pgfqpoint{6.504059in}{5.311066in}}%
\pgfpathlineto{\pgfqpoint{6.504059in}{5.311066in}}%
\pgfpathclose%
\pgfusepath{stroke,fill}%
\end{pgfscope}%
\begin{pgfscope}%
\pgfpathrectangle{\pgfqpoint{0.786164in}{0.768110in}}{\pgfqpoint{8.851069in}{7.081890in}}%
\pgfusepath{clip}%
\pgfsetbuttcap%
\pgfsetroundjoin%
\definecolor{currentfill}{rgb}{0.263663,0.237631,0.518762}%
\pgfsetfillcolor{currentfill}%
\pgfsetfillopacity{0.700000}%
\pgfsetlinewidth{0.501875pt}%
\definecolor{currentstroke}{rgb}{1.000000,1.000000,1.000000}%
\pgfsetstrokecolor{currentstroke}%
\pgfsetstrokeopacity{0.700000}%
\pgfsetdash{}{0pt}%
\pgfpathmoveto{\pgfqpoint{5.581596in}{4.566526in}}%
\pgfpathcurveto{\pgfqpoint{5.594619in}{4.566526in}}{\pgfqpoint{5.607110in}{4.571700in}}{\pgfqpoint{5.616319in}{4.580908in}}%
\pgfpathcurveto{\pgfqpoint{5.625527in}{4.590117in}}{\pgfqpoint{5.630701in}{4.602608in}}{\pgfqpoint{5.630701in}{4.615631in}}%
\pgfpathcurveto{\pgfqpoint{5.630701in}{4.628653in}}{\pgfqpoint{5.625527in}{4.641144in}}{\pgfqpoint{5.616319in}{4.650353in}}%
\pgfpathcurveto{\pgfqpoint{5.607110in}{4.659561in}}{\pgfqpoint{5.594619in}{4.664735in}}{\pgfqpoint{5.581596in}{4.664735in}}%
\pgfpathcurveto{\pgfqpoint{5.568574in}{4.664735in}}{\pgfqpoint{5.556083in}{4.659561in}}{\pgfqpoint{5.546874in}{4.650353in}}%
\pgfpathcurveto{\pgfqpoint{5.537666in}{4.641144in}}{\pgfqpoint{5.532492in}{4.628653in}}{\pgfqpoint{5.532492in}{4.615631in}}%
\pgfpathcurveto{\pgfqpoint{5.532492in}{4.602608in}}{\pgfqpoint{5.537666in}{4.590117in}}{\pgfqpoint{5.546874in}{4.580908in}}%
\pgfpathcurveto{\pgfqpoint{5.556083in}{4.571700in}}{\pgfqpoint{5.568574in}{4.566526in}}{\pgfqpoint{5.581596in}{4.566526in}}%
\pgfpathlineto{\pgfqpoint{5.581596in}{4.566526in}}%
\pgfpathclose%
\pgfusepath{stroke,fill}%
\end{pgfscope}%
\begin{pgfscope}%
\pgfpathrectangle{\pgfqpoint{0.786164in}{0.768110in}}{\pgfqpoint{8.851069in}{7.081890in}}%
\pgfusepath{clip}%
\pgfsetbuttcap%
\pgfsetroundjoin%
\definecolor{currentfill}{rgb}{0.263663,0.237631,0.518762}%
\pgfsetfillcolor{currentfill}%
\pgfsetfillopacity{0.700000}%
\pgfsetlinewidth{0.501875pt}%
\definecolor{currentstroke}{rgb}{1.000000,1.000000,1.000000}%
\pgfsetstrokecolor{currentstroke}%
\pgfsetstrokeopacity{0.700000}%
\pgfsetdash{}{0pt}%
\pgfpathmoveto{\pgfqpoint{5.855595in}{4.676017in}}%
\pgfpathcurveto{\pgfqpoint{5.868618in}{4.676017in}}{\pgfqpoint{5.881109in}{4.681191in}}{\pgfqpoint{5.890317in}{4.690399in}}%
\pgfpathcurveto{\pgfqpoint{5.899526in}{4.699608in}}{\pgfqpoint{5.904700in}{4.712099in}}{\pgfqpoint{5.904700in}{4.725122in}}%
\pgfpathcurveto{\pgfqpoint{5.904700in}{4.738144in}}{\pgfqpoint{5.899526in}{4.750635in}}{\pgfqpoint{5.890317in}{4.759844in}}%
\pgfpathcurveto{\pgfqpoint{5.881109in}{4.769052in}}{\pgfqpoint{5.868618in}{4.774226in}}{\pgfqpoint{5.855595in}{4.774226in}}%
\pgfpathcurveto{\pgfqpoint{5.842572in}{4.774226in}}{\pgfqpoint{5.830081in}{4.769052in}}{\pgfqpoint{5.820873in}{4.759844in}}%
\pgfpathcurveto{\pgfqpoint{5.811664in}{4.750635in}}{\pgfqpoint{5.806490in}{4.738144in}}{\pgfqpoint{5.806490in}{4.725122in}}%
\pgfpathcurveto{\pgfqpoint{5.806490in}{4.712099in}}{\pgfqpoint{5.811664in}{4.699608in}}{\pgfqpoint{5.820873in}{4.690399in}}%
\pgfpathcurveto{\pgfqpoint{5.830081in}{4.681191in}}{\pgfqpoint{5.842572in}{4.676017in}}{\pgfqpoint{5.855595in}{4.676017in}}%
\pgfpathlineto{\pgfqpoint{5.855595in}{4.676017in}}%
\pgfpathclose%
\pgfusepath{stroke,fill}%
\end{pgfscope}%
\begin{pgfscope}%
\pgfpathrectangle{\pgfqpoint{0.786164in}{0.768110in}}{\pgfqpoint{8.851069in}{7.081890in}}%
\pgfusepath{clip}%
\pgfsetbuttcap%
\pgfsetroundjoin%
\definecolor{currentfill}{rgb}{0.246811,0.283237,0.535941}%
\pgfsetfillcolor{currentfill}%
\pgfsetfillopacity{0.700000}%
\pgfsetlinewidth{0.501875pt}%
\definecolor{currentstroke}{rgb}{1.000000,1.000000,1.000000}%
\pgfsetstrokecolor{currentstroke}%
\pgfsetstrokeopacity{0.700000}%
\pgfsetdash{}{0pt}%
\pgfpathmoveto{\pgfqpoint{5.188865in}{4.194256in}}%
\pgfpathcurveto{\pgfqpoint{5.201888in}{4.194256in}}{\pgfqpoint{5.214379in}{4.199430in}}{\pgfqpoint{5.223587in}{4.208638in}}%
\pgfpathcurveto{\pgfqpoint{5.232796in}{4.217847in}}{\pgfqpoint{5.237970in}{4.230338in}}{\pgfqpoint{5.237970in}{4.243360in}}%
\pgfpathcurveto{\pgfqpoint{5.237970in}{4.256383in}}{\pgfqpoint{5.232796in}{4.268874in}}{\pgfqpoint{5.223587in}{4.278083in}}%
\pgfpathcurveto{\pgfqpoint{5.214379in}{4.287291in}}{\pgfqpoint{5.201888in}{4.292465in}}{\pgfqpoint{5.188865in}{4.292465in}}%
\pgfpathcurveto{\pgfqpoint{5.175842in}{4.292465in}}{\pgfqpoint{5.163351in}{4.287291in}}{\pgfqpoint{5.154143in}{4.278083in}}%
\pgfpathcurveto{\pgfqpoint{5.144934in}{4.268874in}}{\pgfqpoint{5.139760in}{4.256383in}}{\pgfqpoint{5.139760in}{4.243360in}}%
\pgfpathcurveto{\pgfqpoint{5.139760in}{4.230338in}}{\pgfqpoint{5.144934in}{4.217847in}}{\pgfqpoint{5.154143in}{4.208638in}}%
\pgfpathcurveto{\pgfqpoint{5.163351in}{4.199430in}}{\pgfqpoint{5.175842in}{4.194256in}}{\pgfqpoint{5.188865in}{4.194256in}}%
\pgfpathlineto{\pgfqpoint{5.188865in}{4.194256in}}%
\pgfpathclose%
\pgfusepath{stroke,fill}%
\end{pgfscope}%
\begin{pgfscope}%
\pgfpathrectangle{\pgfqpoint{0.786164in}{0.768110in}}{\pgfqpoint{8.851069in}{7.081890in}}%
\pgfusepath{clip}%
\pgfsetbuttcap%
\pgfsetroundjoin%
\definecolor{currentfill}{rgb}{0.129933,0.559582,0.551864}%
\pgfsetfillcolor{currentfill}%
\pgfsetfillopacity{0.700000}%
\pgfsetlinewidth{0.501875pt}%
\definecolor{currentstroke}{rgb}{1.000000,1.000000,1.000000}%
\pgfsetstrokecolor{currentstroke}%
\pgfsetstrokeopacity{0.700000}%
\pgfsetdash{}{0pt}%
\pgfpathmoveto{\pgfqpoint{1.654283in}{1.456976in}}%
\pgfpathcurveto{\pgfqpoint{1.667306in}{1.456976in}}{\pgfqpoint{1.679797in}{1.462150in}}{\pgfqpoint{1.689005in}{1.471359in}}%
\pgfpathcurveto{\pgfqpoint{1.698214in}{1.480567in}}{\pgfqpoint{1.703388in}{1.493058in}}{\pgfqpoint{1.703388in}{1.506081in}}%
\pgfpathcurveto{\pgfqpoint{1.703388in}{1.519104in}}{\pgfqpoint{1.698214in}{1.531595in}}{\pgfqpoint{1.689005in}{1.540803in}}%
\pgfpathcurveto{\pgfqpoint{1.679797in}{1.550012in}}{\pgfqpoint{1.667306in}{1.555186in}}{\pgfqpoint{1.654283in}{1.555186in}}%
\pgfpathcurveto{\pgfqpoint{1.641260in}{1.555186in}}{\pgfqpoint{1.628769in}{1.550012in}}{\pgfqpoint{1.619561in}{1.540803in}}%
\pgfpathcurveto{\pgfqpoint{1.610352in}{1.531595in}}{\pgfqpoint{1.605178in}{1.519104in}}{\pgfqpoint{1.605178in}{1.506081in}}%
\pgfpathcurveto{\pgfqpoint{1.605178in}{1.493058in}}{\pgfqpoint{1.610352in}{1.480567in}}{\pgfqpoint{1.619561in}{1.471359in}}%
\pgfpathcurveto{\pgfqpoint{1.628769in}{1.462150in}}{\pgfqpoint{1.641260in}{1.456976in}}{\pgfqpoint{1.654283in}{1.456976in}}%
\pgfpathlineto{\pgfqpoint{1.654283in}{1.456976in}}%
\pgfpathclose%
\pgfusepath{stroke,fill}%
\end{pgfscope}%
\begin{pgfscope}%
\pgfpathrectangle{\pgfqpoint{0.786164in}{0.768110in}}{\pgfqpoint{8.851069in}{7.081890in}}%
\pgfusepath{clip}%
\pgfsetbuttcap%
\pgfsetroundjoin%
\definecolor{currentfill}{rgb}{0.132444,0.552216,0.553018}%
\pgfsetfillcolor{currentfill}%
\pgfsetfillopacity{0.700000}%
\pgfsetlinewidth{0.501875pt}%
\definecolor{currentstroke}{rgb}{1.000000,1.000000,1.000000}%
\pgfsetstrokecolor{currentstroke}%
\pgfsetstrokeopacity{0.700000}%
\pgfsetdash{}{0pt}%
\pgfpathmoveto{\pgfqpoint{1.709083in}{1.544569in}}%
\pgfpathcurveto{\pgfqpoint{1.722105in}{1.544569in}}{\pgfqpoint{1.734596in}{1.549743in}}{\pgfqpoint{1.743805in}{1.558952in}}%
\pgfpathcurveto{\pgfqpoint{1.753013in}{1.568160in}}{\pgfqpoint{1.758187in}{1.580651in}}{\pgfqpoint{1.758187in}{1.593674in}}%
\pgfpathcurveto{\pgfqpoint{1.758187in}{1.606697in}}{\pgfqpoint{1.753013in}{1.619188in}}{\pgfqpoint{1.743805in}{1.628396in}}%
\pgfpathcurveto{\pgfqpoint{1.734596in}{1.637605in}}{\pgfqpoint{1.722105in}{1.642779in}}{\pgfqpoint{1.709083in}{1.642779in}}%
\pgfpathcurveto{\pgfqpoint{1.696060in}{1.642779in}}{\pgfqpoint{1.683569in}{1.637605in}}{\pgfqpoint{1.674360in}{1.628396in}}%
\pgfpathcurveto{\pgfqpoint{1.665152in}{1.619188in}}{\pgfqpoint{1.659978in}{1.606697in}}{\pgfqpoint{1.659978in}{1.593674in}}%
\pgfpathcurveto{\pgfqpoint{1.659978in}{1.580651in}}{\pgfqpoint{1.665152in}{1.568160in}}{\pgfqpoint{1.674360in}{1.558952in}}%
\pgfpathcurveto{\pgfqpoint{1.683569in}{1.549743in}}{\pgfqpoint{1.696060in}{1.544569in}}{\pgfqpoint{1.709083in}{1.544569in}}%
\pgfpathlineto{\pgfqpoint{1.709083in}{1.544569in}}%
\pgfpathclose%
\pgfusepath{stroke,fill}%
\end{pgfscope}%
\begin{pgfscope}%
\pgfpathrectangle{\pgfqpoint{0.786164in}{0.768110in}}{\pgfqpoint{8.851069in}{7.081890in}}%
\pgfusepath{clip}%
\pgfsetbuttcap%
\pgfsetroundjoin%
\definecolor{currentfill}{rgb}{0.137770,0.537492,0.554906}%
\pgfsetfillcolor{currentfill}%
\pgfsetfillopacity{0.700000}%
\pgfsetlinewidth{0.501875pt}%
\definecolor{currentstroke}{rgb}{1.000000,1.000000,1.000000}%
\pgfsetstrokecolor{currentstroke}%
\pgfsetstrokeopacity{0.700000}%
\pgfsetdash{}{0pt}%
\pgfpathmoveto{\pgfqpoint{1.791282in}{1.522671in}}%
\pgfpathcurveto{\pgfqpoint{1.804305in}{1.522671in}}{\pgfqpoint{1.816796in}{1.527845in}}{\pgfqpoint{1.826004in}{1.537053in}}%
\pgfpathcurveto{\pgfqpoint{1.835213in}{1.546262in}}{\pgfqpoint{1.840387in}{1.558753in}}{\pgfqpoint{1.840387in}{1.571776in}}%
\pgfpathcurveto{\pgfqpoint{1.840387in}{1.584798in}}{\pgfqpoint{1.835213in}{1.597289in}}{\pgfqpoint{1.826004in}{1.606498in}}%
\pgfpathcurveto{\pgfqpoint{1.816796in}{1.615706in}}{\pgfqpoint{1.804305in}{1.620880in}}{\pgfqpoint{1.791282in}{1.620880in}}%
\pgfpathcurveto{\pgfqpoint{1.778259in}{1.620880in}}{\pgfqpoint{1.765768in}{1.615706in}}{\pgfqpoint{1.756560in}{1.606498in}}%
\pgfpathcurveto{\pgfqpoint{1.747352in}{1.597289in}}{\pgfqpoint{1.742178in}{1.584798in}}{\pgfqpoint{1.742178in}{1.571776in}}%
\pgfpathcurveto{\pgfqpoint{1.742178in}{1.558753in}}{\pgfqpoint{1.747352in}{1.546262in}}{\pgfqpoint{1.756560in}{1.537053in}}%
\pgfpathcurveto{\pgfqpoint{1.765768in}{1.527845in}}{\pgfqpoint{1.778259in}{1.522671in}}{\pgfqpoint{1.791282in}{1.522671in}}%
\pgfpathlineto{\pgfqpoint{1.791282in}{1.522671in}}%
\pgfpathclose%
\pgfusepath{stroke,fill}%
\end{pgfscope}%
\begin{pgfscope}%
\pgfpathrectangle{\pgfqpoint{0.786164in}{0.768110in}}{\pgfqpoint{8.851069in}{7.081890in}}%
\pgfusepath{clip}%
\pgfsetbuttcap%
\pgfsetroundjoin%
\definecolor{currentfill}{rgb}{0.146180,0.515413,0.556823}%
\pgfsetfillcolor{currentfill}%
\pgfsetfillopacity{0.700000}%
\pgfsetlinewidth{0.501875pt}%
\definecolor{currentstroke}{rgb}{1.000000,1.000000,1.000000}%
\pgfsetstrokecolor{currentstroke}%
\pgfsetstrokeopacity{0.700000}%
\pgfsetdash{}{0pt}%
\pgfpathmoveto{\pgfqpoint{1.855215in}{1.610264in}}%
\pgfpathcurveto{\pgfqpoint{1.868238in}{1.610264in}}{\pgfqpoint{1.880729in}{1.615438in}}{\pgfqpoint{1.889937in}{1.624646in}}%
\pgfpathcurveto{\pgfqpoint{1.899146in}{1.633855in}}{\pgfqpoint{1.904320in}{1.646346in}}{\pgfqpoint{1.904320in}{1.659369in}}%
\pgfpathcurveto{\pgfqpoint{1.904320in}{1.672391in}}{\pgfqpoint{1.899146in}{1.684882in}}{\pgfqpoint{1.889937in}{1.694091in}}%
\pgfpathcurveto{\pgfqpoint{1.880729in}{1.703299in}}{\pgfqpoint{1.868238in}{1.708473in}}{\pgfqpoint{1.855215in}{1.708473in}}%
\pgfpathcurveto{\pgfqpoint{1.842192in}{1.708473in}}{\pgfqpoint{1.829701in}{1.703299in}}{\pgfqpoint{1.820493in}{1.694091in}}%
\pgfpathcurveto{\pgfqpoint{1.811285in}{1.684882in}}{\pgfqpoint{1.806111in}{1.672391in}}{\pgfqpoint{1.806111in}{1.659369in}}%
\pgfpathcurveto{\pgfqpoint{1.806111in}{1.646346in}}{\pgfqpoint{1.811285in}{1.633855in}}{\pgfqpoint{1.820493in}{1.624646in}}%
\pgfpathcurveto{\pgfqpoint{1.829701in}{1.615438in}}{\pgfqpoint{1.842192in}{1.610264in}}{\pgfqpoint{1.855215in}{1.610264in}}%
\pgfpathlineto{\pgfqpoint{1.855215in}{1.610264in}}%
\pgfpathclose%
\pgfusepath{stroke,fill}%
\end{pgfscope}%
\begin{pgfscope}%
\pgfpathrectangle{\pgfqpoint{0.786164in}{0.768110in}}{\pgfqpoint{8.851069in}{7.081890in}}%
\pgfusepath{clip}%
\pgfsetbuttcap%
\pgfsetroundjoin%
\definecolor{currentfill}{rgb}{0.144759,0.519093,0.556572}%
\pgfsetfillcolor{currentfill}%
\pgfsetfillopacity{0.700000}%
\pgfsetlinewidth{0.501875pt}%
\definecolor{currentstroke}{rgb}{1.000000,1.000000,1.000000}%
\pgfsetstrokecolor{currentstroke}%
\pgfsetstrokeopacity{0.700000}%
\pgfsetdash{}{0pt}%
\pgfpathmoveto{\pgfqpoint{1.791282in}{1.544569in}}%
\pgfpathcurveto{\pgfqpoint{1.804305in}{1.544569in}}{\pgfqpoint{1.816796in}{1.549743in}}{\pgfqpoint{1.826004in}{1.558952in}}%
\pgfpathcurveto{\pgfqpoint{1.835213in}{1.568160in}}{\pgfqpoint{1.840387in}{1.580651in}}{\pgfqpoint{1.840387in}{1.593674in}}%
\pgfpathcurveto{\pgfqpoint{1.840387in}{1.606697in}}{\pgfqpoint{1.835213in}{1.619188in}}{\pgfqpoint{1.826004in}{1.628396in}}%
\pgfpathcurveto{\pgfqpoint{1.816796in}{1.637605in}}{\pgfqpoint{1.804305in}{1.642779in}}{\pgfqpoint{1.791282in}{1.642779in}}%
\pgfpathcurveto{\pgfqpoint{1.778259in}{1.642779in}}{\pgfqpoint{1.765768in}{1.637605in}}{\pgfqpoint{1.756560in}{1.628396in}}%
\pgfpathcurveto{\pgfqpoint{1.747352in}{1.619188in}}{\pgfqpoint{1.742178in}{1.606697in}}{\pgfqpoint{1.742178in}{1.593674in}}%
\pgfpathcurveto{\pgfqpoint{1.742178in}{1.580651in}}{\pgfqpoint{1.747352in}{1.568160in}}{\pgfqpoint{1.756560in}{1.558952in}}%
\pgfpathcurveto{\pgfqpoint{1.765768in}{1.549743in}}{\pgfqpoint{1.778259in}{1.544569in}}{\pgfqpoint{1.791282in}{1.544569in}}%
\pgfpathlineto{\pgfqpoint{1.791282in}{1.544569in}}%
\pgfpathclose%
\pgfusepath{stroke,fill}%
\end{pgfscope}%
\begin{pgfscope}%
\pgfpathrectangle{\pgfqpoint{0.786164in}{0.768110in}}{\pgfqpoint{8.851069in}{7.081890in}}%
\pgfusepath{clip}%
\pgfsetbuttcap%
\pgfsetroundjoin%
\definecolor{currentfill}{rgb}{0.123463,0.581687,0.547445}%
\pgfsetfillcolor{currentfill}%
\pgfsetfillopacity{0.700000}%
\pgfsetlinewidth{0.501875pt}%
\definecolor{currentstroke}{rgb}{1.000000,1.000000,1.000000}%
\pgfsetstrokecolor{currentstroke}%
\pgfsetstrokeopacity{0.700000}%
\pgfsetdash{}{0pt}%
\pgfpathmoveto{\pgfqpoint{1.763882in}{1.785450in}}%
\pgfpathcurveto{\pgfqpoint{1.776905in}{1.785450in}}{\pgfqpoint{1.789396in}{1.790624in}}{\pgfqpoint{1.798605in}{1.799832in}}%
\pgfpathcurveto{\pgfqpoint{1.807813in}{1.809041in}}{\pgfqpoint{1.812987in}{1.821532in}}{\pgfqpoint{1.812987in}{1.834555in}}%
\pgfpathcurveto{\pgfqpoint{1.812987in}{1.847577in}}{\pgfqpoint{1.807813in}{1.860068in}}{\pgfqpoint{1.798605in}{1.869277in}}%
\pgfpathcurveto{\pgfqpoint{1.789396in}{1.878485in}}{\pgfqpoint{1.776905in}{1.883659in}}{\pgfqpoint{1.763882in}{1.883659in}}%
\pgfpathcurveto{\pgfqpoint{1.750860in}{1.883659in}}{\pgfqpoint{1.738369in}{1.878485in}}{\pgfqpoint{1.729160in}{1.869277in}}%
\pgfpathcurveto{\pgfqpoint{1.719952in}{1.860068in}}{\pgfqpoint{1.714778in}{1.847577in}}{\pgfqpoint{1.714778in}{1.834555in}}%
\pgfpathcurveto{\pgfqpoint{1.714778in}{1.821532in}}{\pgfqpoint{1.719952in}{1.809041in}}{\pgfqpoint{1.729160in}{1.799832in}}%
\pgfpathcurveto{\pgfqpoint{1.738369in}{1.790624in}}{\pgfqpoint{1.750860in}{1.785450in}}{\pgfqpoint{1.763882in}{1.785450in}}%
\pgfpathlineto{\pgfqpoint{1.763882in}{1.785450in}}%
\pgfpathclose%
\pgfusepath{stroke,fill}%
\end{pgfscope}%
\begin{pgfscope}%
\pgfpathrectangle{\pgfqpoint{0.786164in}{0.768110in}}{\pgfqpoint{8.851069in}{7.081890in}}%
\pgfusepath{clip}%
\pgfsetbuttcap%
\pgfsetroundjoin%
\definecolor{currentfill}{rgb}{0.141935,0.526453,0.555991}%
\pgfsetfillcolor{currentfill}%
\pgfsetfillopacity{0.700000}%
\pgfsetlinewidth{0.501875pt}%
\definecolor{currentstroke}{rgb}{1.000000,1.000000,1.000000}%
\pgfsetstrokecolor{currentstroke}%
\pgfsetstrokeopacity{0.700000}%
\pgfsetdash{}{0pt}%
\pgfpathmoveto{\pgfqpoint{1.855215in}{2.092025in}}%
\pgfpathcurveto{\pgfqpoint{1.868238in}{2.092025in}}{\pgfqpoint{1.880729in}{2.097199in}}{\pgfqpoint{1.889937in}{2.106408in}}%
\pgfpathcurveto{\pgfqpoint{1.899146in}{2.115616in}}{\pgfqpoint{1.904320in}{2.128107in}}{\pgfqpoint{1.904320in}{2.141130in}}%
\pgfpathcurveto{\pgfqpoint{1.904320in}{2.154153in}}{\pgfqpoint{1.899146in}{2.166644in}}{\pgfqpoint{1.889937in}{2.175852in}}%
\pgfpathcurveto{\pgfqpoint{1.880729in}{2.185060in}}{\pgfqpoint{1.868238in}{2.190234in}}{\pgfqpoint{1.855215in}{2.190234in}}%
\pgfpathcurveto{\pgfqpoint{1.842192in}{2.190234in}}{\pgfqpoint{1.829701in}{2.185060in}}{\pgfqpoint{1.820493in}{2.175852in}}%
\pgfpathcurveto{\pgfqpoint{1.811285in}{2.166644in}}{\pgfqpoint{1.806111in}{2.154153in}}{\pgfqpoint{1.806111in}{2.141130in}}%
\pgfpathcurveto{\pgfqpoint{1.806111in}{2.128107in}}{\pgfqpoint{1.811285in}{2.115616in}}{\pgfqpoint{1.820493in}{2.106408in}}%
\pgfpathcurveto{\pgfqpoint{1.829701in}{2.097199in}}{\pgfqpoint{1.842192in}{2.092025in}}{\pgfqpoint{1.855215in}{2.092025in}}%
\pgfpathlineto{\pgfqpoint{1.855215in}{2.092025in}}%
\pgfpathclose%
\pgfusepath{stroke,fill}%
\end{pgfscope}%
\begin{pgfscope}%
\pgfpathrectangle{\pgfqpoint{0.786164in}{0.768110in}}{\pgfqpoint{8.851069in}{7.081890in}}%
\pgfusepath{clip}%
\pgfsetbuttcap%
\pgfsetroundjoin%
\definecolor{currentfill}{rgb}{0.126453,0.570633,0.549841}%
\pgfsetfillcolor{currentfill}%
\pgfsetfillopacity{0.700000}%
\pgfsetlinewidth{0.501875pt}%
\definecolor{currentstroke}{rgb}{1.000000,1.000000,1.000000}%
\pgfsetstrokecolor{currentstroke}%
\pgfsetstrokeopacity{0.700000}%
\pgfsetdash{}{0pt}%
\pgfpathmoveto{\pgfqpoint{1.773016in}{2.004432in}}%
\pgfpathcurveto{\pgfqpoint{1.786038in}{2.004432in}}{\pgfqpoint{1.798529in}{2.009606in}}{\pgfqpoint{1.807738in}{2.018815in}}%
\pgfpathcurveto{\pgfqpoint{1.816946in}{2.028023in}}{\pgfqpoint{1.822120in}{2.040514in}}{\pgfqpoint{1.822120in}{2.053537in}}%
\pgfpathcurveto{\pgfqpoint{1.822120in}{2.066560in}}{\pgfqpoint{1.816946in}{2.079051in}}{\pgfqpoint{1.807738in}{2.088259in}}%
\pgfpathcurveto{\pgfqpoint{1.798529in}{2.097468in}}{\pgfqpoint{1.786038in}{2.102642in}}{\pgfqpoint{1.773016in}{2.102642in}}%
\pgfpathcurveto{\pgfqpoint{1.759993in}{2.102642in}}{\pgfqpoint{1.747502in}{2.097468in}}{\pgfqpoint{1.738293in}{2.088259in}}%
\pgfpathcurveto{\pgfqpoint{1.729085in}{2.079051in}}{\pgfqpoint{1.723911in}{2.066560in}}{\pgfqpoint{1.723911in}{2.053537in}}%
\pgfpathcurveto{\pgfqpoint{1.723911in}{2.040514in}}{\pgfqpoint{1.729085in}{2.028023in}}{\pgfqpoint{1.738293in}{2.018815in}}%
\pgfpathcurveto{\pgfqpoint{1.747502in}{2.009606in}}{\pgfqpoint{1.759993in}{2.004432in}}{\pgfqpoint{1.773016in}{2.004432in}}%
\pgfpathlineto{\pgfqpoint{1.773016in}{2.004432in}}%
\pgfpathclose%
\pgfusepath{stroke,fill}%
\end{pgfscope}%
\begin{pgfscope}%
\pgfpathrectangle{\pgfqpoint{0.786164in}{0.768110in}}{\pgfqpoint{8.851069in}{7.081890in}}%
\pgfusepath{clip}%
\pgfsetbuttcap%
\pgfsetroundjoin%
\definecolor{currentfill}{rgb}{0.119738,0.603785,0.541400}%
\pgfsetfillcolor{currentfill}%
\pgfsetfillopacity{0.700000}%
\pgfsetlinewidth{0.501875pt}%
\definecolor{currentstroke}{rgb}{1.000000,1.000000,1.000000}%
\pgfsetstrokecolor{currentstroke}%
\pgfsetstrokeopacity{0.700000}%
\pgfsetdash{}{0pt}%
\pgfpathmoveto{\pgfqpoint{1.864348in}{1.851145in}}%
\pgfpathcurveto{\pgfqpoint{1.877371in}{1.851145in}}{\pgfqpoint{1.889862in}{1.856319in}}{\pgfqpoint{1.899071in}{1.865527in}}%
\pgfpathcurveto{\pgfqpoint{1.908279in}{1.874735in}}{\pgfqpoint{1.913453in}{1.887227in}}{\pgfqpoint{1.913453in}{1.900249in}}%
\pgfpathcurveto{\pgfqpoint{1.913453in}{1.913272in}}{\pgfqpoint{1.908279in}{1.925763in}}{\pgfqpoint{1.899071in}{1.934971in}}%
\pgfpathcurveto{\pgfqpoint{1.889862in}{1.944180in}}{\pgfqpoint{1.877371in}{1.949354in}}{\pgfqpoint{1.864348in}{1.949354in}}%
\pgfpathcurveto{\pgfqpoint{1.851326in}{1.949354in}}{\pgfqpoint{1.838835in}{1.944180in}}{\pgfqpoint{1.829626in}{1.934971in}}%
\pgfpathcurveto{\pgfqpoint{1.820418in}{1.925763in}}{\pgfqpoint{1.815244in}{1.913272in}}{\pgfqpoint{1.815244in}{1.900249in}}%
\pgfpathcurveto{\pgfqpoint{1.815244in}{1.887227in}}{\pgfqpoint{1.820418in}{1.874735in}}{\pgfqpoint{1.829626in}{1.865527in}}%
\pgfpathcurveto{\pgfqpoint{1.838835in}{1.856319in}}{\pgfqpoint{1.851326in}{1.851145in}}{\pgfqpoint{1.864348in}{1.851145in}}%
\pgfpathlineto{\pgfqpoint{1.864348in}{1.851145in}}%
\pgfpathclose%
\pgfusepath{stroke,fill}%
\end{pgfscope}%
\begin{pgfscope}%
\pgfpathrectangle{\pgfqpoint{0.786164in}{0.768110in}}{\pgfqpoint{8.851069in}{7.081890in}}%
\pgfusepath{clip}%
\pgfsetbuttcap%
\pgfsetroundjoin%
\definecolor{currentfill}{rgb}{0.120081,0.622161,0.534946}%
\pgfsetfillcolor{currentfill}%
\pgfsetfillopacity{0.700000}%
\pgfsetlinewidth{0.501875pt}%
\definecolor{currentstroke}{rgb}{1.000000,1.000000,1.000000}%
\pgfsetstrokecolor{currentstroke}%
\pgfsetstrokeopacity{0.700000}%
\pgfsetdash{}{0pt}%
\pgfpathmoveto{\pgfqpoint{1.855215in}{1.982534in}}%
\pgfpathcurveto{\pgfqpoint{1.868238in}{1.982534in}}{\pgfqpoint{1.880729in}{1.987708in}}{\pgfqpoint{1.889937in}{1.996916in}}%
\pgfpathcurveto{\pgfqpoint{1.899146in}{2.006125in}}{\pgfqpoint{1.904320in}{2.018616in}}{\pgfqpoint{1.904320in}{2.031639in}}%
\pgfpathcurveto{\pgfqpoint{1.904320in}{2.044661in}}{\pgfqpoint{1.899146in}{2.057152in}}{\pgfqpoint{1.889937in}{2.066361in}}%
\pgfpathcurveto{\pgfqpoint{1.880729in}{2.075569in}}{\pgfqpoint{1.868238in}{2.080743in}}{\pgfqpoint{1.855215in}{2.080743in}}%
\pgfpathcurveto{\pgfqpoint{1.842192in}{2.080743in}}{\pgfqpoint{1.829701in}{2.075569in}}{\pgfqpoint{1.820493in}{2.066361in}}%
\pgfpathcurveto{\pgfqpoint{1.811285in}{2.057152in}}{\pgfqpoint{1.806111in}{2.044661in}}{\pgfqpoint{1.806111in}{2.031639in}}%
\pgfpathcurveto{\pgfqpoint{1.806111in}{2.018616in}}{\pgfqpoint{1.811285in}{2.006125in}}{\pgfqpoint{1.820493in}{1.996916in}}%
\pgfpathcurveto{\pgfqpoint{1.829701in}{1.987708in}}{\pgfqpoint{1.842192in}{1.982534in}}{\pgfqpoint{1.855215in}{1.982534in}}%
\pgfpathlineto{\pgfqpoint{1.855215in}{1.982534in}}%
\pgfpathclose%
\pgfusepath{stroke,fill}%
\end{pgfscope}%
\begin{pgfscope}%
\pgfpathrectangle{\pgfqpoint{0.786164in}{0.768110in}}{\pgfqpoint{8.851069in}{7.081890in}}%
\pgfusepath{clip}%
\pgfsetbuttcap%
\pgfsetroundjoin%
\definecolor{currentfill}{rgb}{0.126326,0.644107,0.525311}%
\pgfsetfillcolor{currentfill}%
\pgfsetfillopacity{0.700000}%
\pgfsetlinewidth{0.501875pt}%
\definecolor{currentstroke}{rgb}{1.000000,1.000000,1.000000}%
\pgfsetstrokecolor{currentstroke}%
\pgfsetstrokeopacity{0.700000}%
\pgfsetdash{}{0pt}%
\pgfpathmoveto{\pgfqpoint{1.873482in}{2.420499in}}%
\pgfpathcurveto{\pgfqpoint{1.886504in}{2.420499in}}{\pgfqpoint{1.898996in}{2.425673in}}{\pgfqpoint{1.908204in}{2.434881in}}%
\pgfpathcurveto{\pgfqpoint{1.917412in}{2.444090in}}{\pgfqpoint{1.922586in}{2.456581in}}{\pgfqpoint{1.922586in}{2.469603in}}%
\pgfpathcurveto{\pgfqpoint{1.922586in}{2.482626in}}{\pgfqpoint{1.917412in}{2.495117in}}{\pgfqpoint{1.908204in}{2.504326in}}%
\pgfpathcurveto{\pgfqpoint{1.898996in}{2.513534in}}{\pgfqpoint{1.886504in}{2.518708in}}{\pgfqpoint{1.873482in}{2.518708in}}%
\pgfpathcurveto{\pgfqpoint{1.860459in}{2.518708in}}{\pgfqpoint{1.847968in}{2.513534in}}{\pgfqpoint{1.838760in}{2.504326in}}%
\pgfpathcurveto{\pgfqpoint{1.829551in}{2.495117in}}{\pgfqpoint{1.824377in}{2.482626in}}{\pgfqpoint{1.824377in}{2.469603in}}%
\pgfpathcurveto{\pgfqpoint{1.824377in}{2.456581in}}{\pgfqpoint{1.829551in}{2.444090in}}{\pgfqpoint{1.838760in}{2.434881in}}%
\pgfpathcurveto{\pgfqpoint{1.847968in}{2.425673in}}{\pgfqpoint{1.860459in}{2.420499in}}{\pgfqpoint{1.873482in}{2.420499in}}%
\pgfpathlineto{\pgfqpoint{1.873482in}{2.420499in}}%
\pgfpathclose%
\pgfusepath{stroke,fill}%
\end{pgfscope}%
\begin{pgfscope}%
\pgfpathrectangle{\pgfqpoint{0.786164in}{0.768110in}}{\pgfqpoint{8.851069in}{7.081890in}}%
\pgfusepath{clip}%
\pgfsetbuttcap%
\pgfsetroundjoin%
\definecolor{currentfill}{rgb}{0.121380,0.629492,0.531973}%
\pgfsetfillcolor{currentfill}%
\pgfsetfillopacity{0.700000}%
\pgfsetlinewidth{0.501875pt}%
\definecolor{currentstroke}{rgb}{1.000000,1.000000,1.000000}%
\pgfsetstrokecolor{currentstroke}%
\pgfsetstrokeopacity{0.700000}%
\pgfsetdash{}{0pt}%
\pgfpathmoveto{\pgfqpoint{1.846082in}{2.289109in}}%
\pgfpathcurveto{\pgfqpoint{1.859105in}{2.289109in}}{\pgfqpoint{1.871596in}{2.294283in}}{\pgfqpoint{1.880804in}{2.303492in}}%
\pgfpathcurveto{\pgfqpoint{1.890013in}{2.312700in}}{\pgfqpoint{1.895187in}{2.325191in}}{\pgfqpoint{1.895187in}{2.338214in}}%
\pgfpathcurveto{\pgfqpoint{1.895187in}{2.351237in}}{\pgfqpoint{1.890013in}{2.363728in}}{\pgfqpoint{1.880804in}{2.372936in}}%
\pgfpathcurveto{\pgfqpoint{1.871596in}{2.382145in}}{\pgfqpoint{1.859105in}{2.387319in}}{\pgfqpoint{1.846082in}{2.387319in}}%
\pgfpathcurveto{\pgfqpoint{1.833059in}{2.387319in}}{\pgfqpoint{1.820568in}{2.382145in}}{\pgfqpoint{1.811360in}{2.372936in}}%
\pgfpathcurveto{\pgfqpoint{1.802151in}{2.363728in}}{\pgfqpoint{1.796977in}{2.351237in}}{\pgfqpoint{1.796977in}{2.338214in}}%
\pgfpathcurveto{\pgfqpoint{1.796977in}{2.325191in}}{\pgfqpoint{1.802151in}{2.312700in}}{\pgfqpoint{1.811360in}{2.303492in}}%
\pgfpathcurveto{\pgfqpoint{1.820568in}{2.294283in}}{\pgfqpoint{1.833059in}{2.289109in}}{\pgfqpoint{1.846082in}{2.289109in}}%
\pgfpathlineto{\pgfqpoint{1.846082in}{2.289109in}}%
\pgfpathclose%
\pgfusepath{stroke,fill}%
\end{pgfscope}%
\begin{pgfscope}%
\pgfpathrectangle{\pgfqpoint{0.786164in}{0.768110in}}{\pgfqpoint{8.851069in}{7.081890in}}%
\pgfusepath{clip}%
\pgfsetbuttcap%
\pgfsetroundjoin%
\definecolor{currentfill}{rgb}{0.120081,0.622161,0.534946}%
\pgfsetfillcolor{currentfill}%
\pgfsetfillopacity{0.700000}%
\pgfsetlinewidth{0.501875pt}%
\definecolor{currentstroke}{rgb}{1.000000,1.000000,1.000000}%
\pgfsetstrokecolor{currentstroke}%
\pgfsetstrokeopacity{0.700000}%
\pgfsetdash{}{0pt}%
\pgfpathmoveto{\pgfqpoint{1.873482in}{2.092025in}}%
\pgfpathcurveto{\pgfqpoint{1.886504in}{2.092025in}}{\pgfqpoint{1.898996in}{2.097199in}}{\pgfqpoint{1.908204in}{2.106408in}}%
\pgfpathcurveto{\pgfqpoint{1.917412in}{2.115616in}}{\pgfqpoint{1.922586in}{2.128107in}}{\pgfqpoint{1.922586in}{2.141130in}}%
\pgfpathcurveto{\pgfqpoint{1.922586in}{2.154153in}}{\pgfqpoint{1.917412in}{2.166644in}}{\pgfqpoint{1.908204in}{2.175852in}}%
\pgfpathcurveto{\pgfqpoint{1.898996in}{2.185060in}}{\pgfqpoint{1.886504in}{2.190234in}}{\pgfqpoint{1.873482in}{2.190234in}}%
\pgfpathcurveto{\pgfqpoint{1.860459in}{2.190234in}}{\pgfqpoint{1.847968in}{2.185060in}}{\pgfqpoint{1.838760in}{2.175852in}}%
\pgfpathcurveto{\pgfqpoint{1.829551in}{2.166644in}}{\pgfqpoint{1.824377in}{2.154153in}}{\pgfqpoint{1.824377in}{2.141130in}}%
\pgfpathcurveto{\pgfqpoint{1.824377in}{2.128107in}}{\pgfqpoint{1.829551in}{2.115616in}}{\pgfqpoint{1.838760in}{2.106408in}}%
\pgfpathcurveto{\pgfqpoint{1.847968in}{2.097199in}}{\pgfqpoint{1.860459in}{2.092025in}}{\pgfqpoint{1.873482in}{2.092025in}}%
\pgfpathlineto{\pgfqpoint{1.873482in}{2.092025in}}%
\pgfpathclose%
\pgfusepath{stroke,fill}%
\end{pgfscope}%
\begin{pgfscope}%
\pgfpathrectangle{\pgfqpoint{0.786164in}{0.768110in}}{\pgfqpoint{8.851069in}{7.081890in}}%
\pgfusepath{clip}%
\pgfsetbuttcap%
\pgfsetroundjoin%
\definecolor{currentfill}{rgb}{0.130067,0.651384,0.521608}%
\pgfsetfillcolor{currentfill}%
\pgfsetfillopacity{0.700000}%
\pgfsetlinewidth{0.501875pt}%
\definecolor{currentstroke}{rgb}{1.000000,1.000000,1.000000}%
\pgfsetstrokecolor{currentstroke}%
\pgfsetstrokeopacity{0.700000}%
\pgfsetdash{}{0pt}%
\pgfpathmoveto{\pgfqpoint{1.973948in}{1.960636in}}%
\pgfpathcurveto{\pgfqpoint{1.986971in}{1.960636in}}{\pgfqpoint{1.999462in}{1.965810in}}{\pgfqpoint{2.008670in}{1.975018in}}%
\pgfpathcurveto{\pgfqpoint{2.017879in}{1.984227in}}{\pgfqpoint{2.023053in}{1.996718in}}{\pgfqpoint{2.023053in}{2.009740in}}%
\pgfpathcurveto{\pgfqpoint{2.023053in}{2.022763in}}{\pgfqpoint{2.017879in}{2.035254in}}{\pgfqpoint{2.008670in}{2.044463in}}%
\pgfpathcurveto{\pgfqpoint{1.999462in}{2.053671in}}{\pgfqpoint{1.986971in}{2.058845in}}{\pgfqpoint{1.973948in}{2.058845in}}%
\pgfpathcurveto{\pgfqpoint{1.960925in}{2.058845in}}{\pgfqpoint{1.948434in}{2.053671in}}{\pgfqpoint{1.939226in}{2.044463in}}%
\pgfpathcurveto{\pgfqpoint{1.930017in}{2.035254in}}{\pgfqpoint{1.924843in}{2.022763in}}{\pgfqpoint{1.924843in}{2.009740in}}%
\pgfpathcurveto{\pgfqpoint{1.924843in}{1.996718in}}{\pgfqpoint{1.930017in}{1.984227in}}{\pgfqpoint{1.939226in}{1.975018in}}%
\pgfpathcurveto{\pgfqpoint{1.948434in}{1.965810in}}{\pgfqpoint{1.960925in}{1.960636in}}{\pgfqpoint{1.973948in}{1.960636in}}%
\pgfpathlineto{\pgfqpoint{1.973948in}{1.960636in}}%
\pgfpathclose%
\pgfusepath{stroke,fill}%
\end{pgfscope}%
\begin{pgfscope}%
\pgfpathrectangle{\pgfqpoint{0.786164in}{0.768110in}}{\pgfqpoint{8.851069in}{7.081890in}}%
\pgfusepath{clip}%
\pgfsetbuttcap%
\pgfsetroundjoin%
\definecolor{currentfill}{rgb}{0.140210,0.665859,0.513427}%
\pgfsetfillcolor{currentfill}%
\pgfsetfillopacity{0.700000}%
\pgfsetlinewidth{0.501875pt}%
\definecolor{currentstroke}{rgb}{1.000000,1.000000,1.000000}%
\pgfsetstrokecolor{currentstroke}%
\pgfsetstrokeopacity{0.700000}%
\pgfsetdash{}{0pt}%
\pgfpathmoveto{\pgfqpoint{2.092681in}{2.311008in}}%
\pgfpathcurveto{\pgfqpoint{2.105703in}{2.311008in}}{\pgfqpoint{2.118194in}{2.316182in}}{\pgfqpoint{2.127403in}{2.325390in}}%
\pgfpathcurveto{\pgfqpoint{2.136611in}{2.334598in}}{\pgfqpoint{2.141785in}{2.347089in}}{\pgfqpoint{2.141785in}{2.360112in}}%
\pgfpathcurveto{\pgfqpoint{2.141785in}{2.373135in}}{\pgfqpoint{2.136611in}{2.385626in}}{\pgfqpoint{2.127403in}{2.394834in}}%
\pgfpathcurveto{\pgfqpoint{2.118194in}{2.404043in}}{\pgfqpoint{2.105703in}{2.409217in}}{\pgfqpoint{2.092681in}{2.409217in}}%
\pgfpathcurveto{\pgfqpoint{2.079658in}{2.409217in}}{\pgfqpoint{2.067167in}{2.404043in}}{\pgfqpoint{2.057958in}{2.394834in}}%
\pgfpathcurveto{\pgfqpoint{2.048750in}{2.385626in}}{\pgfqpoint{2.043576in}{2.373135in}}{\pgfqpoint{2.043576in}{2.360112in}}%
\pgfpathcurveto{\pgfqpoint{2.043576in}{2.347089in}}{\pgfqpoint{2.048750in}{2.334598in}}{\pgfqpoint{2.057958in}{2.325390in}}%
\pgfpathcurveto{\pgfqpoint{2.067167in}{2.316182in}}{\pgfqpoint{2.079658in}{2.311008in}}{\pgfqpoint{2.092681in}{2.311008in}}%
\pgfpathlineto{\pgfqpoint{2.092681in}{2.311008in}}%
\pgfpathclose%
\pgfusepath{stroke,fill}%
\end{pgfscope}%
\begin{pgfscope}%
\pgfpathrectangle{\pgfqpoint{0.786164in}{0.768110in}}{\pgfqpoint{8.851069in}{7.081890in}}%
\pgfusepath{clip}%
\pgfsetbuttcap%
\pgfsetroundjoin%
\definecolor{currentfill}{rgb}{0.150148,0.676631,0.506589}%
\pgfsetfillcolor{currentfill}%
\pgfsetfillopacity{0.700000}%
\pgfsetlinewidth{0.501875pt}%
\definecolor{currentstroke}{rgb}{1.000000,1.000000,1.000000}%
\pgfsetstrokecolor{currentstroke}%
\pgfsetstrokeopacity{0.700000}%
\pgfsetdash{}{0pt}%
\pgfpathmoveto{\pgfqpoint{2.047014in}{2.135822in}}%
\pgfpathcurveto{\pgfqpoint{2.060037in}{2.135822in}}{\pgfqpoint{2.072528in}{2.140996in}}{\pgfqpoint{2.081736in}{2.150204in}}%
\pgfpathcurveto{\pgfqpoint{2.090945in}{2.159413in}}{\pgfqpoint{2.096119in}{2.171904in}}{\pgfqpoint{2.096119in}{2.184926in}}%
\pgfpathcurveto{\pgfqpoint{2.096119in}{2.197949in}}{\pgfqpoint{2.090945in}{2.210440in}}{\pgfqpoint{2.081736in}{2.219649in}}%
\pgfpathcurveto{\pgfqpoint{2.072528in}{2.228857in}}{\pgfqpoint{2.060037in}{2.234031in}}{\pgfqpoint{2.047014in}{2.234031in}}%
\pgfpathcurveto{\pgfqpoint{2.033992in}{2.234031in}}{\pgfqpoint{2.021500in}{2.228857in}}{\pgfqpoint{2.012292in}{2.219649in}}%
\pgfpathcurveto{\pgfqpoint{2.003084in}{2.210440in}}{\pgfqpoint{1.997910in}{2.197949in}}{\pgfqpoint{1.997910in}{2.184926in}}%
\pgfpathcurveto{\pgfqpoint{1.997910in}{2.171904in}}{\pgfqpoint{2.003084in}{2.159413in}}{\pgfqpoint{2.012292in}{2.150204in}}%
\pgfpathcurveto{\pgfqpoint{2.021500in}{2.140996in}}{\pgfqpoint{2.033992in}{2.135822in}}{\pgfqpoint{2.047014in}{2.135822in}}%
\pgfpathlineto{\pgfqpoint{2.047014in}{2.135822in}}%
\pgfpathclose%
\pgfusepath{stroke,fill}%
\end{pgfscope}%
\begin{pgfscope}%
\pgfpathrectangle{\pgfqpoint{0.786164in}{0.768110in}}{\pgfqpoint{8.851069in}{7.081890in}}%
\pgfusepath{clip}%
\pgfsetbuttcap%
\pgfsetroundjoin%
\definecolor{currentfill}{rgb}{0.170948,0.694384,0.493803}%
\pgfsetfillcolor{currentfill}%
\pgfsetfillopacity{0.700000}%
\pgfsetlinewidth{0.501875pt}%
\definecolor{currentstroke}{rgb}{1.000000,1.000000,1.000000}%
\pgfsetstrokecolor{currentstroke}%
\pgfsetstrokeopacity{0.700000}%
\pgfsetdash{}{0pt}%
\pgfpathmoveto{\pgfqpoint{1.919148in}{2.332906in}}%
\pgfpathcurveto{\pgfqpoint{1.932171in}{2.332906in}}{\pgfqpoint{1.944662in}{2.338080in}}{\pgfqpoint{1.953870in}{2.347288in}}%
\pgfpathcurveto{\pgfqpoint{1.963079in}{2.356497in}}{\pgfqpoint{1.968253in}{2.368988in}}{\pgfqpoint{1.968253in}{2.382010in}}%
\pgfpathcurveto{\pgfqpoint{1.968253in}{2.395033in}}{\pgfqpoint{1.963079in}{2.407524in}}{\pgfqpoint{1.953870in}{2.416733in}}%
\pgfpathcurveto{\pgfqpoint{1.944662in}{2.425941in}}{\pgfqpoint{1.932171in}{2.431115in}}{\pgfqpoint{1.919148in}{2.431115in}}%
\pgfpathcurveto{\pgfqpoint{1.906125in}{2.431115in}}{\pgfqpoint{1.893634in}{2.425941in}}{\pgfqpoint{1.884426in}{2.416733in}}%
\pgfpathcurveto{\pgfqpoint{1.875218in}{2.407524in}}{\pgfqpoint{1.870044in}{2.395033in}}{\pgfqpoint{1.870044in}{2.382010in}}%
\pgfpathcurveto{\pgfqpoint{1.870044in}{2.368988in}}{\pgfqpoint{1.875218in}{2.356497in}}{\pgfqpoint{1.884426in}{2.347288in}}%
\pgfpathcurveto{\pgfqpoint{1.893634in}{2.338080in}}{\pgfqpoint{1.906125in}{2.332906in}}{\pgfqpoint{1.919148in}{2.332906in}}%
\pgfpathlineto{\pgfqpoint{1.919148in}{2.332906in}}%
\pgfpathclose%
\pgfusepath{stroke,fill}%
\end{pgfscope}%
\begin{pgfscope}%
\pgfpathrectangle{\pgfqpoint{0.786164in}{0.768110in}}{\pgfqpoint{8.851069in}{7.081890in}}%
\pgfusepath{clip}%
\pgfsetbuttcap%
\pgfsetroundjoin%
\definecolor{currentfill}{rgb}{0.170948,0.694384,0.493803}%
\pgfsetfillcolor{currentfill}%
\pgfsetfillopacity{0.700000}%
\pgfsetlinewidth{0.501875pt}%
\definecolor{currentstroke}{rgb}{1.000000,1.000000,1.000000}%
\pgfsetstrokecolor{currentstroke}%
\pgfsetstrokeopacity{0.700000}%
\pgfsetdash{}{0pt}%
\pgfpathmoveto{\pgfqpoint{2.037881in}{2.551888in}}%
\pgfpathcurveto{\pgfqpoint{2.050904in}{2.551888in}}{\pgfqpoint{2.063395in}{2.557062in}}{\pgfqpoint{2.072603in}{2.566271in}}%
\pgfpathcurveto{\pgfqpoint{2.081812in}{2.575479in}}{\pgfqpoint{2.086986in}{2.587970in}}{\pgfqpoint{2.086986in}{2.600993in}}%
\pgfpathcurveto{\pgfqpoint{2.086986in}{2.614015in}}{\pgfqpoint{2.081812in}{2.626507in}}{\pgfqpoint{2.072603in}{2.635715in}}%
\pgfpathcurveto{\pgfqpoint{2.063395in}{2.644923in}}{\pgfqpoint{2.050904in}{2.650097in}}{\pgfqpoint{2.037881in}{2.650097in}}%
\pgfpathcurveto{\pgfqpoint{2.024858in}{2.650097in}}{\pgfqpoint{2.012367in}{2.644923in}}{\pgfqpoint{2.003159in}{2.635715in}}%
\pgfpathcurveto{\pgfqpoint{1.993950in}{2.626507in}}{\pgfqpoint{1.988776in}{2.614015in}}{\pgfqpoint{1.988776in}{2.600993in}}%
\pgfpathcurveto{\pgfqpoint{1.988776in}{2.587970in}}{\pgfqpoint{1.993950in}{2.575479in}}{\pgfqpoint{2.003159in}{2.566271in}}%
\pgfpathcurveto{\pgfqpoint{2.012367in}{2.557062in}}{\pgfqpoint{2.024858in}{2.551888in}}{\pgfqpoint{2.037881in}{2.551888in}}%
\pgfpathlineto{\pgfqpoint{2.037881in}{2.551888in}}%
\pgfpathclose%
\pgfusepath{stroke,fill}%
\end{pgfscope}%
\begin{pgfscope}%
\pgfpathrectangle{\pgfqpoint{0.786164in}{0.768110in}}{\pgfqpoint{8.851069in}{7.081890in}}%
\pgfusepath{clip}%
\pgfsetbuttcap%
\pgfsetroundjoin%
\definecolor{currentfill}{rgb}{0.191090,0.708366,0.482284}%
\pgfsetfillcolor{currentfill}%
\pgfsetfillopacity{0.700000}%
\pgfsetlinewidth{0.501875pt}%
\definecolor{currentstroke}{rgb}{1.000000,1.000000,1.000000}%
\pgfsetstrokecolor{currentstroke}%
\pgfsetstrokeopacity{0.700000}%
\pgfsetdash{}{0pt}%
\pgfpathmoveto{\pgfqpoint{1.964815in}{2.836565in}}%
\pgfpathcurveto{\pgfqpoint{1.977837in}{2.836565in}}{\pgfqpoint{1.990328in}{2.841739in}}{\pgfqpoint{1.999537in}{2.850948in}}%
\pgfpathcurveto{\pgfqpoint{2.008745in}{2.860156in}}{\pgfqpoint{2.013919in}{2.872647in}}{\pgfqpoint{2.013919in}{2.885670in}}%
\pgfpathcurveto{\pgfqpoint{2.013919in}{2.898693in}}{\pgfqpoint{2.008745in}{2.911184in}}{\pgfqpoint{1.999537in}{2.920392in}}%
\pgfpathcurveto{\pgfqpoint{1.990328in}{2.929601in}}{\pgfqpoint{1.977837in}{2.934774in}}{\pgfqpoint{1.964815in}{2.934774in}}%
\pgfpathcurveto{\pgfqpoint{1.951792in}{2.934774in}}{\pgfqpoint{1.939301in}{2.929601in}}{\pgfqpoint{1.930092in}{2.920392in}}%
\pgfpathcurveto{\pgfqpoint{1.920884in}{2.911184in}}{\pgfqpoint{1.915710in}{2.898693in}}{\pgfqpoint{1.915710in}{2.885670in}}%
\pgfpathcurveto{\pgfqpoint{1.915710in}{2.872647in}}{\pgfqpoint{1.920884in}{2.860156in}}{\pgfqpoint{1.930092in}{2.850948in}}%
\pgfpathcurveto{\pgfqpoint{1.939301in}{2.841739in}}{\pgfqpoint{1.951792in}{2.836565in}}{\pgfqpoint{1.964815in}{2.836565in}}%
\pgfpathlineto{\pgfqpoint{1.964815in}{2.836565in}}%
\pgfpathclose%
\pgfusepath{stroke,fill}%
\end{pgfscope}%
\begin{pgfscope}%
\pgfpathrectangle{\pgfqpoint{0.786164in}{0.768110in}}{\pgfqpoint{8.851069in}{7.081890in}}%
\pgfusepath{clip}%
\pgfsetbuttcap%
\pgfsetroundjoin%
\definecolor{currentfill}{rgb}{0.267004,0.004874,0.329415}%
\pgfsetfillcolor{currentfill}%
\pgfsetfillopacity{0.700000}%
\pgfsetlinewidth{0.501875pt}%
\definecolor{currentstroke}{rgb}{1.000000,1.000000,1.000000}%
\pgfsetstrokecolor{currentstroke}%
\pgfsetstrokeopacity{0.700000}%
\pgfsetdash{}{0pt}%
\pgfpathmoveto{\pgfqpoint{1.992214in}{2.902260in}}%
\pgfpathcurveto{\pgfqpoint{2.005237in}{2.902260in}}{\pgfqpoint{2.017728in}{2.907434in}}{\pgfqpoint{2.026937in}{2.916642in}}%
\pgfpathcurveto{\pgfqpoint{2.036145in}{2.925851in}}{\pgfqpoint{2.041319in}{2.938342in}}{\pgfqpoint{2.041319in}{2.951365in}}%
\pgfpathcurveto{\pgfqpoint{2.041319in}{2.964387in}}{\pgfqpoint{2.036145in}{2.976878in}}{\pgfqpoint{2.026937in}{2.986087in}}%
\pgfpathcurveto{\pgfqpoint{2.017728in}{2.995295in}}{\pgfqpoint{2.005237in}{3.000469in}}{\pgfqpoint{1.992214in}{3.000469in}}%
\pgfpathcurveto{\pgfqpoint{1.979192in}{3.000469in}}{\pgfqpoint{1.966701in}{2.995295in}}{\pgfqpoint{1.957492in}{2.986087in}}%
\pgfpathcurveto{\pgfqpoint{1.948284in}{2.976878in}}{\pgfqpoint{1.943110in}{2.964387in}}{\pgfqpoint{1.943110in}{2.951365in}}%
\pgfpathcurveto{\pgfqpoint{1.943110in}{2.938342in}}{\pgfqpoint{1.948284in}{2.925851in}}{\pgfqpoint{1.957492in}{2.916642in}}%
\pgfpathcurveto{\pgfqpoint{1.966701in}{2.907434in}}{\pgfqpoint{1.979192in}{2.902260in}}{\pgfqpoint{1.992214in}{2.902260in}}%
\pgfpathlineto{\pgfqpoint{1.992214in}{2.902260in}}%
\pgfpathclose%
\pgfusepath{stroke,fill}%
\end{pgfscope}%
\begin{pgfscope}%
\pgfpathrectangle{\pgfqpoint{0.786164in}{0.768110in}}{\pgfqpoint{8.851069in}{7.081890in}}%
\pgfusepath{clip}%
\pgfsetbuttcap%
\pgfsetroundjoin%
\definecolor{currentfill}{rgb}{0.267004,0.004874,0.329415}%
\pgfsetfillcolor{currentfill}%
\pgfsetfillopacity{0.700000}%
\pgfsetlinewidth{0.501875pt}%
\definecolor{currentstroke}{rgb}{1.000000,1.000000,1.000000}%
\pgfsetstrokecolor{currentstroke}%
\pgfsetstrokeopacity{0.700000}%
\pgfsetdash{}{0pt}%
\pgfpathmoveto{\pgfqpoint{1.955681in}{2.792769in}}%
\pgfpathcurveto{\pgfqpoint{1.968704in}{2.792769in}}{\pgfqpoint{1.981195in}{2.797943in}}{\pgfqpoint{1.990404in}{2.807151in}}%
\pgfpathcurveto{\pgfqpoint{1.999612in}{2.816360in}}{\pgfqpoint{2.004786in}{2.828851in}}{\pgfqpoint{2.004786in}{2.841873in}}%
\pgfpathcurveto{\pgfqpoint{2.004786in}{2.854896in}}{\pgfqpoint{1.999612in}{2.867387in}}{\pgfqpoint{1.990404in}{2.876596in}}%
\pgfpathcurveto{\pgfqpoint{1.981195in}{2.885804in}}{\pgfqpoint{1.968704in}{2.890978in}}{\pgfqpoint{1.955681in}{2.890978in}}%
\pgfpathcurveto{\pgfqpoint{1.942659in}{2.890978in}}{\pgfqpoint{1.930168in}{2.885804in}}{\pgfqpoint{1.920959in}{2.876596in}}%
\pgfpathcurveto{\pgfqpoint{1.911751in}{2.867387in}}{\pgfqpoint{1.906577in}{2.854896in}}{\pgfqpoint{1.906577in}{2.841873in}}%
\pgfpathcurveto{\pgfqpoint{1.906577in}{2.828851in}}{\pgfqpoint{1.911751in}{2.816360in}}{\pgfqpoint{1.920959in}{2.807151in}}%
\pgfpathcurveto{\pgfqpoint{1.930168in}{2.797943in}}{\pgfqpoint{1.942659in}{2.792769in}}{\pgfqpoint{1.955681in}{2.792769in}}%
\pgfpathlineto{\pgfqpoint{1.955681in}{2.792769in}}%
\pgfpathclose%
\pgfusepath{stroke,fill}%
\end{pgfscope}%
\begin{pgfscope}%
\pgfpathrectangle{\pgfqpoint{0.786164in}{0.768110in}}{\pgfqpoint{8.851069in}{7.081890in}}%
\pgfusepath{clip}%
\pgfsetbuttcap%
\pgfsetroundjoin%
\definecolor{currentfill}{rgb}{0.267004,0.004874,0.329415}%
\pgfsetfillcolor{currentfill}%
\pgfsetfillopacity{0.700000}%
\pgfsetlinewidth{0.501875pt}%
\definecolor{currentstroke}{rgb}{1.000000,1.000000,1.000000}%
\pgfsetstrokecolor{currentstroke}%
\pgfsetstrokeopacity{0.700000}%
\pgfsetdash{}{0pt}%
\pgfpathmoveto{\pgfqpoint{1.946548in}{2.792769in}}%
\pgfpathcurveto{\pgfqpoint{1.959571in}{2.792769in}}{\pgfqpoint{1.972062in}{2.797943in}}{\pgfqpoint{1.981270in}{2.807151in}}%
\pgfpathcurveto{\pgfqpoint{1.990479in}{2.816360in}}{\pgfqpoint{1.995653in}{2.828851in}}{\pgfqpoint{1.995653in}{2.841873in}}%
\pgfpathcurveto{\pgfqpoint{1.995653in}{2.854896in}}{\pgfqpoint{1.990479in}{2.867387in}}{\pgfqpoint{1.981270in}{2.876596in}}%
\pgfpathcurveto{\pgfqpoint{1.972062in}{2.885804in}}{\pgfqpoint{1.959571in}{2.890978in}}{\pgfqpoint{1.946548in}{2.890978in}}%
\pgfpathcurveto{\pgfqpoint{1.933525in}{2.890978in}}{\pgfqpoint{1.921034in}{2.885804in}}{\pgfqpoint{1.911826in}{2.876596in}}%
\pgfpathcurveto{\pgfqpoint{1.902617in}{2.867387in}}{\pgfqpoint{1.897443in}{2.854896in}}{\pgfqpoint{1.897443in}{2.841873in}}%
\pgfpathcurveto{\pgfqpoint{1.897443in}{2.828851in}}{\pgfqpoint{1.902617in}{2.816360in}}{\pgfqpoint{1.911826in}{2.807151in}}%
\pgfpathcurveto{\pgfqpoint{1.921034in}{2.797943in}}{\pgfqpoint{1.933525in}{2.792769in}}{\pgfqpoint{1.946548in}{2.792769in}}%
\pgfpathlineto{\pgfqpoint{1.946548in}{2.792769in}}%
\pgfpathclose%
\pgfusepath{stroke,fill}%
\end{pgfscope}%
\begin{pgfscope}%
\pgfpathrectangle{\pgfqpoint{0.786164in}{0.768110in}}{\pgfqpoint{8.851069in}{7.081890in}}%
\pgfusepath{clip}%
\pgfsetbuttcap%
\pgfsetroundjoin%
\definecolor{currentfill}{rgb}{0.267004,0.004874,0.329415}%
\pgfsetfillcolor{currentfill}%
\pgfsetfillopacity{0.700000}%
\pgfsetlinewidth{0.501875pt}%
\definecolor{currentstroke}{rgb}{1.000000,1.000000,1.000000}%
\pgfsetstrokecolor{currentstroke}%
\pgfsetstrokeopacity{0.700000}%
\pgfsetdash{}{0pt}%
\pgfpathmoveto{\pgfqpoint{2.010481in}{2.858463in}}%
\pgfpathcurveto{\pgfqpoint{2.023504in}{2.858463in}}{\pgfqpoint{2.035995in}{2.863637in}}{\pgfqpoint{2.045203in}{2.872846in}}%
\pgfpathcurveto{\pgfqpoint{2.054412in}{2.882054in}}{\pgfqpoint{2.059586in}{2.894545in}}{\pgfqpoint{2.059586in}{2.907568in}}%
\pgfpathcurveto{\pgfqpoint{2.059586in}{2.920591in}}{\pgfqpoint{2.054412in}{2.933082in}}{\pgfqpoint{2.045203in}{2.942290in}}%
\pgfpathcurveto{\pgfqpoint{2.035995in}{2.951499in}}{\pgfqpoint{2.023504in}{2.956673in}}{\pgfqpoint{2.010481in}{2.956673in}}%
\pgfpathcurveto{\pgfqpoint{1.997458in}{2.956673in}}{\pgfqpoint{1.984967in}{2.951499in}}{\pgfqpoint{1.975759in}{2.942290in}}%
\pgfpathcurveto{\pgfqpoint{1.966550in}{2.933082in}}{\pgfqpoint{1.961376in}{2.920591in}}{\pgfqpoint{1.961376in}{2.907568in}}%
\pgfpathcurveto{\pgfqpoint{1.961376in}{2.894545in}}{\pgfqpoint{1.966550in}{2.882054in}}{\pgfqpoint{1.975759in}{2.872846in}}%
\pgfpathcurveto{\pgfqpoint{1.984967in}{2.863637in}}{\pgfqpoint{1.997458in}{2.858463in}}{\pgfqpoint{2.010481in}{2.858463in}}%
\pgfpathlineto{\pgfqpoint{2.010481in}{2.858463in}}%
\pgfpathclose%
\pgfusepath{stroke,fill}%
\end{pgfscope}%
\begin{pgfscope}%
\pgfpathrectangle{\pgfqpoint{0.786164in}{0.768110in}}{\pgfqpoint{8.851069in}{7.081890in}}%
\pgfusepath{clip}%
\pgfsetbuttcap%
\pgfsetroundjoin%
\definecolor{currentfill}{rgb}{0.267004,0.004874,0.329415}%
\pgfsetfillcolor{currentfill}%
\pgfsetfillopacity{0.700000}%
\pgfsetlinewidth{0.501875pt}%
\definecolor{currentstroke}{rgb}{1.000000,1.000000,1.000000}%
\pgfsetstrokecolor{currentstroke}%
\pgfsetstrokeopacity{0.700000}%
\pgfsetdash{}{0pt}%
\pgfpathmoveto{\pgfqpoint{2.019614in}{2.814667in}}%
\pgfpathcurveto{\pgfqpoint{2.032637in}{2.814667in}}{\pgfqpoint{2.045128in}{2.819841in}}{\pgfqpoint{2.054337in}{2.829049in}}%
\pgfpathcurveto{\pgfqpoint{2.063545in}{2.838258in}}{\pgfqpoint{2.068719in}{2.850749in}}{\pgfqpoint{2.068719in}{2.863772in}}%
\pgfpathcurveto{\pgfqpoint{2.068719in}{2.876794in}}{\pgfqpoint{2.063545in}{2.889285in}}{\pgfqpoint{2.054337in}{2.898494in}}%
\pgfpathcurveto{\pgfqpoint{2.045128in}{2.907702in}}{\pgfqpoint{2.032637in}{2.912876in}}{\pgfqpoint{2.019614in}{2.912876in}}%
\pgfpathcurveto{\pgfqpoint{2.006592in}{2.912876in}}{\pgfqpoint{1.994101in}{2.907702in}}{\pgfqpoint{1.984892in}{2.898494in}}%
\pgfpathcurveto{\pgfqpoint{1.975684in}{2.889285in}}{\pgfqpoint{1.970510in}{2.876794in}}{\pgfqpoint{1.970510in}{2.863772in}}%
\pgfpathcurveto{\pgfqpoint{1.970510in}{2.850749in}}{\pgfqpoint{1.975684in}{2.838258in}}{\pgfqpoint{1.984892in}{2.829049in}}%
\pgfpathcurveto{\pgfqpoint{1.994101in}{2.819841in}}{\pgfqpoint{2.006592in}{2.814667in}}{\pgfqpoint{2.019614in}{2.814667in}}%
\pgfpathlineto{\pgfqpoint{2.019614in}{2.814667in}}%
\pgfpathclose%
\pgfusepath{stroke,fill}%
\end{pgfscope}%
\begin{pgfscope}%
\pgfpathrectangle{\pgfqpoint{0.786164in}{0.768110in}}{\pgfqpoint{8.851069in}{7.081890in}}%
\pgfusepath{clip}%
\pgfsetbuttcap%
\pgfsetroundjoin%
\definecolor{currentfill}{rgb}{0.267004,0.004874,0.329415}%
\pgfsetfillcolor{currentfill}%
\pgfsetfillopacity{0.700000}%
\pgfsetlinewidth{0.501875pt}%
\definecolor{currentstroke}{rgb}{1.000000,1.000000,1.000000}%
\pgfsetstrokecolor{currentstroke}%
\pgfsetstrokeopacity{0.700000}%
\pgfsetdash{}{0pt}%
\pgfpathmoveto{\pgfqpoint{1.818682in}{2.705176in}}%
\pgfpathcurveto{\pgfqpoint{1.831705in}{2.705176in}}{\pgfqpoint{1.844196in}{2.710350in}}{\pgfqpoint{1.853404in}{2.719558in}}%
\pgfpathcurveto{\pgfqpoint{1.862613in}{2.728767in}}{\pgfqpoint{1.867787in}{2.741258in}}{\pgfqpoint{1.867787in}{2.754280in}}%
\pgfpathcurveto{\pgfqpoint{1.867787in}{2.767303in}}{\pgfqpoint{1.862613in}{2.779794in}}{\pgfqpoint{1.853404in}{2.789003in}}%
\pgfpathcurveto{\pgfqpoint{1.844196in}{2.798211in}}{\pgfqpoint{1.831705in}{2.803385in}}{\pgfqpoint{1.818682in}{2.803385in}}%
\pgfpathcurveto{\pgfqpoint{1.805659in}{2.803385in}}{\pgfqpoint{1.793168in}{2.798211in}}{\pgfqpoint{1.783960in}{2.789003in}}%
\pgfpathcurveto{\pgfqpoint{1.774751in}{2.779794in}}{\pgfqpoint{1.769577in}{2.767303in}}{\pgfqpoint{1.769577in}{2.754280in}}%
\pgfpathcurveto{\pgfqpoint{1.769577in}{2.741258in}}{\pgfqpoint{1.774751in}{2.728767in}}{\pgfqpoint{1.783960in}{2.719558in}}%
\pgfpathcurveto{\pgfqpoint{1.793168in}{2.710350in}}{\pgfqpoint{1.805659in}{2.705176in}}{\pgfqpoint{1.818682in}{2.705176in}}%
\pgfpathlineto{\pgfqpoint{1.818682in}{2.705176in}}%
\pgfpathclose%
\pgfusepath{stroke,fill}%
\end{pgfscope}%
\begin{pgfscope}%
\pgfpathrectangle{\pgfqpoint{0.786164in}{0.768110in}}{\pgfqpoint{8.851069in}{7.081890in}}%
\pgfusepath{clip}%
\pgfsetbuttcap%
\pgfsetroundjoin%
\definecolor{currentfill}{rgb}{0.271305,0.019942,0.347269}%
\pgfsetfillcolor{currentfill}%
\pgfsetfillopacity{0.700000}%
\pgfsetlinewidth{0.501875pt}%
\definecolor{currentstroke}{rgb}{1.000000,1.000000,1.000000}%
\pgfsetstrokecolor{currentstroke}%
\pgfsetstrokeopacity{0.700000}%
\pgfsetdash{}{0pt}%
\pgfpathmoveto{\pgfqpoint{1.919148in}{2.748972in}}%
\pgfpathcurveto{\pgfqpoint{1.932171in}{2.748972in}}{\pgfqpoint{1.944662in}{2.754146in}}{\pgfqpoint{1.953870in}{2.763355in}}%
\pgfpathcurveto{\pgfqpoint{1.963079in}{2.772563in}}{\pgfqpoint{1.968253in}{2.785054in}}{\pgfqpoint{1.968253in}{2.798077in}}%
\pgfpathcurveto{\pgfqpoint{1.968253in}{2.811100in}}{\pgfqpoint{1.963079in}{2.823591in}}{\pgfqpoint{1.953870in}{2.832799in}}%
\pgfpathcurveto{\pgfqpoint{1.944662in}{2.842008in}}{\pgfqpoint{1.932171in}{2.847182in}}{\pgfqpoint{1.919148in}{2.847182in}}%
\pgfpathcurveto{\pgfqpoint{1.906125in}{2.847182in}}{\pgfqpoint{1.893634in}{2.842008in}}{\pgfqpoint{1.884426in}{2.832799in}}%
\pgfpathcurveto{\pgfqpoint{1.875218in}{2.823591in}}{\pgfqpoint{1.870044in}{2.811100in}}{\pgfqpoint{1.870044in}{2.798077in}}%
\pgfpathcurveto{\pgfqpoint{1.870044in}{2.785054in}}{\pgfqpoint{1.875218in}{2.772563in}}{\pgfqpoint{1.884426in}{2.763355in}}%
\pgfpathcurveto{\pgfqpoint{1.893634in}{2.754146in}}{\pgfqpoint{1.906125in}{2.748972in}}{\pgfqpoint{1.919148in}{2.748972in}}%
\pgfpathlineto{\pgfqpoint{1.919148in}{2.748972in}}%
\pgfpathclose%
\pgfusepath{stroke,fill}%
\end{pgfscope}%
\begin{pgfscope}%
\pgfpathrectangle{\pgfqpoint{0.786164in}{0.768110in}}{\pgfqpoint{8.851069in}{7.081890in}}%
\pgfusepath{clip}%
\pgfsetbuttcap%
\pgfsetroundjoin%
\definecolor{currentfill}{rgb}{0.274952,0.037752,0.364543}%
\pgfsetfillcolor{currentfill}%
\pgfsetfillopacity{0.700000}%
\pgfsetlinewidth{0.501875pt}%
\definecolor{currentstroke}{rgb}{1.000000,1.000000,1.000000}%
\pgfsetstrokecolor{currentstroke}%
\pgfsetstrokeopacity{0.700000}%
\pgfsetdash{}{0pt}%
\pgfpathmoveto{\pgfqpoint{1.910015in}{2.748972in}}%
\pgfpathcurveto{\pgfqpoint{1.923038in}{2.748972in}}{\pgfqpoint{1.935529in}{2.754146in}}{\pgfqpoint{1.944737in}{2.763355in}}%
\pgfpathcurveto{\pgfqpoint{1.953946in}{2.772563in}}{\pgfqpoint{1.959120in}{2.785054in}}{\pgfqpoint{1.959120in}{2.798077in}}%
\pgfpathcurveto{\pgfqpoint{1.959120in}{2.811100in}}{\pgfqpoint{1.953946in}{2.823591in}}{\pgfqpoint{1.944737in}{2.832799in}}%
\pgfpathcurveto{\pgfqpoint{1.935529in}{2.842008in}}{\pgfqpoint{1.923038in}{2.847182in}}{\pgfqpoint{1.910015in}{2.847182in}}%
\pgfpathcurveto{\pgfqpoint{1.896992in}{2.847182in}}{\pgfqpoint{1.884501in}{2.842008in}}{\pgfqpoint{1.875293in}{2.832799in}}%
\pgfpathcurveto{\pgfqpoint{1.866084in}{2.823591in}}{\pgfqpoint{1.860910in}{2.811100in}}{\pgfqpoint{1.860910in}{2.798077in}}%
\pgfpathcurveto{\pgfqpoint{1.860910in}{2.785054in}}{\pgfqpoint{1.866084in}{2.772563in}}{\pgfqpoint{1.875293in}{2.763355in}}%
\pgfpathcurveto{\pgfqpoint{1.884501in}{2.754146in}}{\pgfqpoint{1.896992in}{2.748972in}}{\pgfqpoint{1.910015in}{2.748972in}}%
\pgfpathlineto{\pgfqpoint{1.910015in}{2.748972in}}%
\pgfpathclose%
\pgfusepath{stroke,fill}%
\end{pgfscope}%
\begin{pgfscope}%
\pgfpathrectangle{\pgfqpoint{0.786164in}{0.768110in}}{\pgfqpoint{8.851069in}{7.081890in}}%
\pgfusepath{clip}%
\pgfsetbuttcap%
\pgfsetroundjoin%
\definecolor{currentfill}{rgb}{0.278791,0.062145,0.386592}%
\pgfsetfillcolor{currentfill}%
\pgfsetfillopacity{0.700000}%
\pgfsetlinewidth{0.501875pt}%
\definecolor{currentstroke}{rgb}{1.000000,1.000000,1.000000}%
\pgfsetstrokecolor{currentstroke}%
\pgfsetstrokeopacity{0.700000}%
\pgfsetdash{}{0pt}%
\pgfpathmoveto{\pgfqpoint{1.983081in}{2.814667in}}%
\pgfpathcurveto{\pgfqpoint{1.996104in}{2.814667in}}{\pgfqpoint{2.008595in}{2.819841in}}{\pgfqpoint{2.017803in}{2.829049in}}%
\pgfpathcurveto{\pgfqpoint{2.027012in}{2.838258in}}{\pgfqpoint{2.032186in}{2.850749in}}{\pgfqpoint{2.032186in}{2.863772in}}%
\pgfpathcurveto{\pgfqpoint{2.032186in}{2.876794in}}{\pgfqpoint{2.027012in}{2.889285in}}{\pgfqpoint{2.017803in}{2.898494in}}%
\pgfpathcurveto{\pgfqpoint{2.008595in}{2.907702in}}{\pgfqpoint{1.996104in}{2.912876in}}{\pgfqpoint{1.983081in}{2.912876in}}%
\pgfpathcurveto{\pgfqpoint{1.970059in}{2.912876in}}{\pgfqpoint{1.957567in}{2.907702in}}{\pgfqpoint{1.948359in}{2.898494in}}%
\pgfpathcurveto{\pgfqpoint{1.939151in}{2.889285in}}{\pgfqpoint{1.933977in}{2.876794in}}{\pgfqpoint{1.933977in}{2.863772in}}%
\pgfpathcurveto{\pgfqpoint{1.933977in}{2.850749in}}{\pgfqpoint{1.939151in}{2.838258in}}{\pgfqpoint{1.948359in}{2.829049in}}%
\pgfpathcurveto{\pgfqpoint{1.957567in}{2.819841in}}{\pgfqpoint{1.970059in}{2.814667in}}{\pgfqpoint{1.983081in}{2.814667in}}%
\pgfpathlineto{\pgfqpoint{1.983081in}{2.814667in}}%
\pgfpathclose%
\pgfusepath{stroke,fill}%
\end{pgfscope}%
\begin{pgfscope}%
\pgfpathrectangle{\pgfqpoint{0.786164in}{0.768110in}}{\pgfqpoint{8.851069in}{7.081890in}}%
\pgfusepath{clip}%
\pgfsetbuttcap%
\pgfsetroundjoin%
\definecolor{currentfill}{rgb}{0.281446,0.084320,0.407414}%
\pgfsetfillcolor{currentfill}%
\pgfsetfillopacity{0.700000}%
\pgfsetlinewidth{0.501875pt}%
\definecolor{currentstroke}{rgb}{1.000000,1.000000,1.000000}%
\pgfsetstrokecolor{currentstroke}%
\pgfsetstrokeopacity{0.700000}%
\pgfsetdash{}{0pt}%
\pgfpathmoveto{\pgfqpoint{1.736482in}{2.639481in}}%
\pgfpathcurveto{\pgfqpoint{1.749505in}{2.639481in}}{\pgfqpoint{1.761996in}{2.644655in}}{\pgfqpoint{1.771205in}{2.653864in}}%
\pgfpathcurveto{\pgfqpoint{1.780413in}{2.663072in}}{\pgfqpoint{1.785587in}{2.675563in}}{\pgfqpoint{1.785587in}{2.688586in}}%
\pgfpathcurveto{\pgfqpoint{1.785587in}{2.701608in}}{\pgfqpoint{1.780413in}{2.714100in}}{\pgfqpoint{1.771205in}{2.723308in}}%
\pgfpathcurveto{\pgfqpoint{1.761996in}{2.732516in}}{\pgfqpoint{1.749505in}{2.737690in}}{\pgfqpoint{1.736482in}{2.737690in}}%
\pgfpathcurveto{\pgfqpoint{1.723460in}{2.737690in}}{\pgfqpoint{1.710969in}{2.732516in}}{\pgfqpoint{1.701760in}{2.723308in}}%
\pgfpathcurveto{\pgfqpoint{1.692552in}{2.714100in}}{\pgfqpoint{1.687378in}{2.701608in}}{\pgfqpoint{1.687378in}{2.688586in}}%
\pgfpathcurveto{\pgfqpoint{1.687378in}{2.675563in}}{\pgfqpoint{1.692552in}{2.663072in}}{\pgfqpoint{1.701760in}{2.653864in}}%
\pgfpathcurveto{\pgfqpoint{1.710969in}{2.644655in}}{\pgfqpoint{1.723460in}{2.639481in}}{\pgfqpoint{1.736482in}{2.639481in}}%
\pgfpathlineto{\pgfqpoint{1.736482in}{2.639481in}}%
\pgfpathclose%
\pgfusepath{stroke,fill}%
\end{pgfscope}%
\begin{pgfscope}%
\pgfpathrectangle{\pgfqpoint{0.786164in}{0.768110in}}{\pgfqpoint{8.851069in}{7.081890in}}%
\pgfusepath{clip}%
\pgfsetbuttcap%
\pgfsetroundjoin%
\definecolor{currentfill}{rgb}{0.282910,0.105393,0.426902}%
\pgfsetfillcolor{currentfill}%
\pgfsetfillopacity{0.700000}%
\pgfsetlinewidth{0.501875pt}%
\definecolor{currentstroke}{rgb}{1.000000,1.000000,1.000000}%
\pgfsetstrokecolor{currentstroke}%
\pgfsetstrokeopacity{0.700000}%
\pgfsetdash{}{0pt}%
\pgfpathmoveto{\pgfqpoint{1.745616in}{2.617583in}}%
\pgfpathcurveto{\pgfqpoint{1.758638in}{2.617583in}}{\pgfqpoint{1.771130in}{2.622757in}}{\pgfqpoint{1.780338in}{2.631965in}}%
\pgfpathcurveto{\pgfqpoint{1.789546in}{2.641174in}}{\pgfqpoint{1.794720in}{2.653665in}}{\pgfqpoint{1.794720in}{2.666687in}}%
\pgfpathcurveto{\pgfqpoint{1.794720in}{2.679710in}}{\pgfqpoint{1.789546in}{2.692201in}}{\pgfqpoint{1.780338in}{2.701410in}}%
\pgfpathcurveto{\pgfqpoint{1.771130in}{2.710618in}}{\pgfqpoint{1.758638in}{2.715792in}}{\pgfqpoint{1.745616in}{2.715792in}}%
\pgfpathcurveto{\pgfqpoint{1.732593in}{2.715792in}}{\pgfqpoint{1.720102in}{2.710618in}}{\pgfqpoint{1.710894in}{2.701410in}}%
\pgfpathcurveto{\pgfqpoint{1.701685in}{2.692201in}}{\pgfqpoint{1.696511in}{2.679710in}}{\pgfqpoint{1.696511in}{2.666687in}}%
\pgfpathcurveto{\pgfqpoint{1.696511in}{2.653665in}}{\pgfqpoint{1.701685in}{2.641174in}}{\pgfqpoint{1.710894in}{2.631965in}}%
\pgfpathcurveto{\pgfqpoint{1.720102in}{2.622757in}}{\pgfqpoint{1.732593in}{2.617583in}}{\pgfqpoint{1.745616in}{2.617583in}}%
\pgfpathlineto{\pgfqpoint{1.745616in}{2.617583in}}%
\pgfpathclose%
\pgfusepath{stroke,fill}%
\end{pgfscope}%
\begin{pgfscope}%
\pgfpathrectangle{\pgfqpoint{0.786164in}{0.768110in}}{\pgfqpoint{8.851069in}{7.081890in}}%
\pgfusepath{clip}%
\pgfsetbuttcap%
\pgfsetroundjoin%
\definecolor{currentfill}{rgb}{0.283091,0.110553,0.431554}%
\pgfsetfillcolor{currentfill}%
\pgfsetfillopacity{0.700000}%
\pgfsetlinewidth{0.501875pt}%
\definecolor{currentstroke}{rgb}{1.000000,1.000000,1.000000}%
\pgfsetstrokecolor{currentstroke}%
\pgfsetstrokeopacity{0.700000}%
\pgfsetdash{}{0pt}%
\pgfpathmoveto{\pgfqpoint{1.416817in}{2.245313in}}%
\pgfpathcurveto{\pgfqpoint{1.429840in}{2.245313in}}{\pgfqpoint{1.442331in}{2.250487in}}{\pgfqpoint{1.451540in}{2.259695in}}%
\pgfpathcurveto{\pgfqpoint{1.460748in}{2.268904in}}{\pgfqpoint{1.465922in}{2.281395in}}{\pgfqpoint{1.465922in}{2.294417in}}%
\pgfpathcurveto{\pgfqpoint{1.465922in}{2.307440in}}{\pgfqpoint{1.460748in}{2.319931in}}{\pgfqpoint{1.451540in}{2.329140in}}%
\pgfpathcurveto{\pgfqpoint{1.442331in}{2.338348in}}{\pgfqpoint{1.429840in}{2.343522in}}{\pgfqpoint{1.416817in}{2.343522in}}%
\pgfpathcurveto{\pgfqpoint{1.403795in}{2.343522in}}{\pgfqpoint{1.391304in}{2.338348in}}{\pgfqpoint{1.382095in}{2.329140in}}%
\pgfpathcurveto{\pgfqpoint{1.372887in}{2.319931in}}{\pgfqpoint{1.367713in}{2.307440in}}{\pgfqpoint{1.367713in}{2.294417in}}%
\pgfpathcurveto{\pgfqpoint{1.367713in}{2.281395in}}{\pgfqpoint{1.372887in}{2.268904in}}{\pgfqpoint{1.382095in}{2.259695in}}%
\pgfpathcurveto{\pgfqpoint{1.391304in}{2.250487in}}{\pgfqpoint{1.403795in}{2.245313in}}{\pgfqpoint{1.416817in}{2.245313in}}%
\pgfpathlineto{\pgfqpoint{1.416817in}{2.245313in}}%
\pgfpathclose%
\pgfusepath{stroke,fill}%
\end{pgfscope}%
\begin{pgfscope}%
\pgfpathrectangle{\pgfqpoint{0.786164in}{0.768110in}}{\pgfqpoint{8.851069in}{7.081890in}}%
\pgfusepath{clip}%
\pgfsetbuttcap%
\pgfsetroundjoin%
\definecolor{currentfill}{rgb}{0.283072,0.130895,0.449241}%
\pgfsetfillcolor{currentfill}%
\pgfsetfillopacity{0.700000}%
\pgfsetlinewidth{0.501875pt}%
\definecolor{currentstroke}{rgb}{1.000000,1.000000,1.000000}%
\pgfsetstrokecolor{currentstroke}%
\pgfsetstrokeopacity{0.700000}%
\pgfsetdash{}{0pt}%
\pgfpathmoveto{\pgfqpoint{1.298085in}{2.092025in}}%
\pgfpathcurveto{\pgfqpoint{1.311107in}{2.092025in}}{\pgfqpoint{1.323598in}{2.097199in}}{\pgfqpoint{1.332807in}{2.106408in}}%
\pgfpathcurveto{\pgfqpoint{1.342015in}{2.115616in}}{\pgfqpoint{1.347189in}{2.128107in}}{\pgfqpoint{1.347189in}{2.141130in}}%
\pgfpathcurveto{\pgfqpoint{1.347189in}{2.154153in}}{\pgfqpoint{1.342015in}{2.166644in}}{\pgfqpoint{1.332807in}{2.175852in}}%
\pgfpathcurveto{\pgfqpoint{1.323598in}{2.185060in}}{\pgfqpoint{1.311107in}{2.190234in}}{\pgfqpoint{1.298085in}{2.190234in}}%
\pgfpathcurveto{\pgfqpoint{1.285062in}{2.190234in}}{\pgfqpoint{1.272571in}{2.185060in}}{\pgfqpoint{1.263362in}{2.175852in}}%
\pgfpathcurveto{\pgfqpoint{1.254154in}{2.166644in}}{\pgfqpoint{1.248980in}{2.154153in}}{\pgfqpoint{1.248980in}{2.141130in}}%
\pgfpathcurveto{\pgfqpoint{1.248980in}{2.128107in}}{\pgfqpoint{1.254154in}{2.115616in}}{\pgfqpoint{1.263362in}{2.106408in}}%
\pgfpathcurveto{\pgfqpoint{1.272571in}{2.097199in}}{\pgfqpoint{1.285062in}{2.092025in}}{\pgfqpoint{1.298085in}{2.092025in}}%
\pgfpathlineto{\pgfqpoint{1.298085in}{2.092025in}}%
\pgfpathclose%
\pgfusepath{stroke,fill}%
\end{pgfscope}%
\begin{pgfscope}%
\pgfpathrectangle{\pgfqpoint{0.786164in}{0.768110in}}{\pgfqpoint{8.851069in}{7.081890in}}%
\pgfusepath{clip}%
\pgfsetbuttcap%
\pgfsetroundjoin%
\definecolor{currentfill}{rgb}{0.281887,0.150881,0.465405}%
\pgfsetfillcolor{currentfill}%
\pgfsetfillopacity{0.700000}%
\pgfsetlinewidth{0.501875pt}%
\definecolor{currentstroke}{rgb}{1.000000,1.000000,1.000000}%
\pgfsetstrokecolor{currentstroke}%
\pgfsetstrokeopacity{0.700000}%
\pgfsetdash{}{0pt}%
\pgfpathmoveto{\pgfqpoint{1.453351in}{2.157720in}}%
\pgfpathcurveto{\pgfqpoint{1.466373in}{2.157720in}}{\pgfqpoint{1.478864in}{2.162894in}}{\pgfqpoint{1.488073in}{2.172102in}}%
\pgfpathcurveto{\pgfqpoint{1.497281in}{2.181311in}}{\pgfqpoint{1.502455in}{2.193802in}}{\pgfqpoint{1.502455in}{2.206825in}}%
\pgfpathcurveto{\pgfqpoint{1.502455in}{2.219847in}}{\pgfqpoint{1.497281in}{2.232338in}}{\pgfqpoint{1.488073in}{2.241547in}}%
\pgfpathcurveto{\pgfqpoint{1.478864in}{2.250755in}}{\pgfqpoint{1.466373in}{2.255929in}}{\pgfqpoint{1.453351in}{2.255929in}}%
\pgfpathcurveto{\pgfqpoint{1.440328in}{2.255929in}}{\pgfqpoint{1.427837in}{2.250755in}}{\pgfqpoint{1.418628in}{2.241547in}}%
\pgfpathcurveto{\pgfqpoint{1.409420in}{2.232338in}}{\pgfqpoint{1.404246in}{2.219847in}}{\pgfqpoint{1.404246in}{2.206825in}}%
\pgfpathcurveto{\pgfqpoint{1.404246in}{2.193802in}}{\pgfqpoint{1.409420in}{2.181311in}}{\pgfqpoint{1.418628in}{2.172102in}}%
\pgfpathcurveto{\pgfqpoint{1.427837in}{2.162894in}}{\pgfqpoint{1.440328in}{2.157720in}}{\pgfqpoint{1.453351in}{2.157720in}}%
\pgfpathlineto{\pgfqpoint{1.453351in}{2.157720in}}%
\pgfpathclose%
\pgfusepath{stroke,fill}%
\end{pgfscope}%
\begin{pgfscope}%
\pgfpathrectangle{\pgfqpoint{0.786164in}{0.768110in}}{\pgfqpoint{8.851069in}{7.081890in}}%
\pgfusepath{clip}%
\pgfsetbuttcap%
\pgfsetroundjoin%
\definecolor{currentfill}{rgb}{0.280255,0.165693,0.476498}%
\pgfsetfillcolor{currentfill}%
\pgfsetfillopacity{0.700000}%
\pgfsetlinewidth{0.501875pt}%
\definecolor{currentstroke}{rgb}{1.000000,1.000000,1.000000}%
\pgfsetstrokecolor{currentstroke}%
\pgfsetstrokeopacity{0.700000}%
\pgfsetdash{}{0pt}%
\pgfpathmoveto{\pgfqpoint{1.416817in}{2.135822in}}%
\pgfpathcurveto{\pgfqpoint{1.429840in}{2.135822in}}{\pgfqpoint{1.442331in}{2.140996in}}{\pgfqpoint{1.451540in}{2.150204in}}%
\pgfpathcurveto{\pgfqpoint{1.460748in}{2.159413in}}{\pgfqpoint{1.465922in}{2.171904in}}{\pgfqpoint{1.465922in}{2.184926in}}%
\pgfpathcurveto{\pgfqpoint{1.465922in}{2.197949in}}{\pgfqpoint{1.460748in}{2.210440in}}{\pgfqpoint{1.451540in}{2.219649in}}%
\pgfpathcurveto{\pgfqpoint{1.442331in}{2.228857in}}{\pgfqpoint{1.429840in}{2.234031in}}{\pgfqpoint{1.416817in}{2.234031in}}%
\pgfpathcurveto{\pgfqpoint{1.403795in}{2.234031in}}{\pgfqpoint{1.391304in}{2.228857in}}{\pgfqpoint{1.382095in}{2.219649in}}%
\pgfpathcurveto{\pgfqpoint{1.372887in}{2.210440in}}{\pgfqpoint{1.367713in}{2.197949in}}{\pgfqpoint{1.367713in}{2.184926in}}%
\pgfpathcurveto{\pgfqpoint{1.367713in}{2.171904in}}{\pgfqpoint{1.372887in}{2.159413in}}{\pgfqpoint{1.382095in}{2.150204in}}%
\pgfpathcurveto{\pgfqpoint{1.391304in}{2.140996in}}{\pgfqpoint{1.403795in}{2.135822in}}{\pgfqpoint{1.416817in}{2.135822in}}%
\pgfpathlineto{\pgfqpoint{1.416817in}{2.135822in}}%
\pgfpathclose%
\pgfusepath{stroke,fill}%
\end{pgfscope}%
\begin{pgfscope}%
\pgfpathrectangle{\pgfqpoint{0.786164in}{0.768110in}}{\pgfqpoint{8.851069in}{7.081890in}}%
\pgfusepath{clip}%
\pgfsetbuttcap%
\pgfsetroundjoin%
\definecolor{currentfill}{rgb}{0.279574,0.170599,0.479997}%
\pgfsetfillcolor{currentfill}%
\pgfsetfillopacity{0.700000}%
\pgfsetlinewidth{0.501875pt}%
\definecolor{currentstroke}{rgb}{1.000000,1.000000,1.000000}%
\pgfsetstrokecolor{currentstroke}%
\pgfsetstrokeopacity{0.700000}%
\pgfsetdash{}{0pt}%
\pgfpathmoveto{\pgfqpoint{1.453351in}{2.157720in}}%
\pgfpathcurveto{\pgfqpoint{1.466373in}{2.157720in}}{\pgfqpoint{1.478864in}{2.162894in}}{\pgfqpoint{1.488073in}{2.172102in}}%
\pgfpathcurveto{\pgfqpoint{1.497281in}{2.181311in}}{\pgfqpoint{1.502455in}{2.193802in}}{\pgfqpoint{1.502455in}{2.206825in}}%
\pgfpathcurveto{\pgfqpoint{1.502455in}{2.219847in}}{\pgfqpoint{1.497281in}{2.232338in}}{\pgfqpoint{1.488073in}{2.241547in}}%
\pgfpathcurveto{\pgfqpoint{1.478864in}{2.250755in}}{\pgfqpoint{1.466373in}{2.255929in}}{\pgfqpoint{1.453351in}{2.255929in}}%
\pgfpathcurveto{\pgfqpoint{1.440328in}{2.255929in}}{\pgfqpoint{1.427837in}{2.250755in}}{\pgfqpoint{1.418628in}{2.241547in}}%
\pgfpathcurveto{\pgfqpoint{1.409420in}{2.232338in}}{\pgfqpoint{1.404246in}{2.219847in}}{\pgfqpoint{1.404246in}{2.206825in}}%
\pgfpathcurveto{\pgfqpoint{1.404246in}{2.193802in}}{\pgfqpoint{1.409420in}{2.181311in}}{\pgfqpoint{1.418628in}{2.172102in}}%
\pgfpathcurveto{\pgfqpoint{1.427837in}{2.162894in}}{\pgfqpoint{1.440328in}{2.157720in}}{\pgfqpoint{1.453351in}{2.157720in}}%
\pgfpathlineto{\pgfqpoint{1.453351in}{2.157720in}}%
\pgfpathclose%
\pgfusepath{stroke,fill}%
\end{pgfscope}%
\begin{pgfscope}%
\pgfpathrectangle{\pgfqpoint{0.786164in}{0.768110in}}{\pgfqpoint{8.851069in}{7.081890in}}%
\pgfusepath{clip}%
\pgfsetbuttcap%
\pgfsetroundjoin%
\definecolor{currentfill}{rgb}{0.269308,0.218818,0.509577}%
\pgfsetfillcolor{currentfill}%
\pgfsetfillopacity{0.700000}%
\pgfsetlinewidth{0.501875pt}%
\definecolor{currentstroke}{rgb}{1.000000,1.000000,1.000000}%
\pgfsetstrokecolor{currentstroke}%
\pgfsetstrokeopacity{0.700000}%
\pgfsetdash{}{0pt}%
\pgfpathmoveto{\pgfqpoint{1.188485in}{1.982534in}}%
\pgfpathcurveto{\pgfqpoint{1.201508in}{1.982534in}}{\pgfqpoint{1.213999in}{1.987708in}}{\pgfqpoint{1.223207in}{1.996916in}}%
\pgfpathcurveto{\pgfqpoint{1.232416in}{2.006125in}}{\pgfqpoint{1.237590in}{2.018616in}}{\pgfqpoint{1.237590in}{2.031639in}}%
\pgfpathcurveto{\pgfqpoint{1.237590in}{2.044661in}}{\pgfqpoint{1.232416in}{2.057152in}}{\pgfqpoint{1.223207in}{2.066361in}}%
\pgfpathcurveto{\pgfqpoint{1.213999in}{2.075569in}}{\pgfqpoint{1.201508in}{2.080743in}}{\pgfqpoint{1.188485in}{2.080743in}}%
\pgfpathcurveto{\pgfqpoint{1.175462in}{2.080743in}}{\pgfqpoint{1.162971in}{2.075569in}}{\pgfqpoint{1.153763in}{2.066361in}}%
\pgfpathcurveto{\pgfqpoint{1.144555in}{2.057152in}}{\pgfqpoint{1.139381in}{2.044661in}}{\pgfqpoint{1.139381in}{2.031639in}}%
\pgfpathcurveto{\pgfqpoint{1.139381in}{2.018616in}}{\pgfqpoint{1.144555in}{2.006125in}}{\pgfqpoint{1.153763in}{1.996916in}}%
\pgfpathcurveto{\pgfqpoint{1.162971in}{1.987708in}}{\pgfqpoint{1.175462in}{1.982534in}}{\pgfqpoint{1.188485in}{1.982534in}}%
\pgfpathlineto{\pgfqpoint{1.188485in}{1.982534in}}%
\pgfpathclose%
\pgfusepath{stroke,fill}%
\end{pgfscope}%
\begin{pgfscope}%
\pgfpathrectangle{\pgfqpoint{0.786164in}{0.768110in}}{\pgfqpoint{8.851069in}{7.081890in}}%
\pgfusepath{clip}%
\pgfsetbuttcap%
\pgfsetroundjoin%
\definecolor{currentfill}{rgb}{0.258965,0.251537,0.524736}%
\pgfsetfillcolor{currentfill}%
\pgfsetfillopacity{0.700000}%
\pgfsetlinewidth{0.501875pt}%
\definecolor{currentstroke}{rgb}{1.000000,1.000000,1.000000}%
\pgfsetstrokecolor{currentstroke}%
\pgfsetstrokeopacity{0.700000}%
\pgfsetdash{}{0pt}%
\pgfpathmoveto{\pgfqpoint{1.225018in}{2.004432in}}%
\pgfpathcurveto{\pgfqpoint{1.238041in}{2.004432in}}{\pgfqpoint{1.250532in}{2.009606in}}{\pgfqpoint{1.259741in}{2.018815in}}%
\pgfpathcurveto{\pgfqpoint{1.268949in}{2.028023in}}{\pgfqpoint{1.274123in}{2.040514in}}{\pgfqpoint{1.274123in}{2.053537in}}%
\pgfpathcurveto{\pgfqpoint{1.274123in}{2.066560in}}{\pgfqpoint{1.268949in}{2.079051in}}{\pgfqpoint{1.259741in}{2.088259in}}%
\pgfpathcurveto{\pgfqpoint{1.250532in}{2.097468in}}{\pgfqpoint{1.238041in}{2.102642in}}{\pgfqpoint{1.225018in}{2.102642in}}%
\pgfpathcurveto{\pgfqpoint{1.211996in}{2.102642in}}{\pgfqpoint{1.199505in}{2.097468in}}{\pgfqpoint{1.190296in}{2.088259in}}%
\pgfpathcurveto{\pgfqpoint{1.181088in}{2.079051in}}{\pgfqpoint{1.175914in}{2.066560in}}{\pgfqpoint{1.175914in}{2.053537in}}%
\pgfpathcurveto{\pgfqpoint{1.175914in}{2.040514in}}{\pgfqpoint{1.181088in}{2.028023in}}{\pgfqpoint{1.190296in}{2.018815in}}%
\pgfpathcurveto{\pgfqpoint{1.199505in}{2.009606in}}{\pgfqpoint{1.211996in}{2.004432in}}{\pgfqpoint{1.225018in}{2.004432in}}%
\pgfpathlineto{\pgfqpoint{1.225018in}{2.004432in}}%
\pgfpathclose%
\pgfusepath{stroke,fill}%
\end{pgfscope}%
\begin{pgfscope}%
\pgfpathrectangle{\pgfqpoint{0.786164in}{0.768110in}}{\pgfqpoint{8.851069in}{7.081890in}}%
\pgfusepath{clip}%
\pgfsetbuttcap%
\pgfsetroundjoin%
\definecolor{currentfill}{rgb}{0.253935,0.265254,0.529983}%
\pgfsetfillcolor{currentfill}%
\pgfsetfillopacity{0.700000}%
\pgfsetlinewidth{0.501875pt}%
\definecolor{currentstroke}{rgb}{1.000000,1.000000,1.000000}%
\pgfsetstrokecolor{currentstroke}%
\pgfsetstrokeopacity{0.700000}%
\pgfsetdash{}{0pt}%
\pgfpathmoveto{\pgfqpoint{1.398551in}{2.113923in}}%
\pgfpathcurveto{\pgfqpoint{1.411574in}{2.113923in}}{\pgfqpoint{1.424065in}{2.119097in}}{\pgfqpoint{1.433273in}{2.128306in}}%
\pgfpathcurveto{\pgfqpoint{1.442481in}{2.137514in}}{\pgfqpoint{1.447655in}{2.150005in}}{\pgfqpoint{1.447655in}{2.163028in}}%
\pgfpathcurveto{\pgfqpoint{1.447655in}{2.176051in}}{\pgfqpoint{1.442481in}{2.188542in}}{\pgfqpoint{1.433273in}{2.197750in}}%
\pgfpathcurveto{\pgfqpoint{1.424065in}{2.206959in}}{\pgfqpoint{1.411574in}{2.212133in}}{\pgfqpoint{1.398551in}{2.212133in}}%
\pgfpathcurveto{\pgfqpoint{1.385528in}{2.212133in}}{\pgfqpoint{1.373037in}{2.206959in}}{\pgfqpoint{1.363829in}{2.197750in}}%
\pgfpathcurveto{\pgfqpoint{1.354620in}{2.188542in}}{\pgfqpoint{1.349446in}{2.176051in}}{\pgfqpoint{1.349446in}{2.163028in}}%
\pgfpathcurveto{\pgfqpoint{1.349446in}{2.150005in}}{\pgfqpoint{1.354620in}{2.137514in}}{\pgfqpoint{1.363829in}{2.128306in}}%
\pgfpathcurveto{\pgfqpoint{1.373037in}{2.119097in}}{\pgfqpoint{1.385528in}{2.113923in}}{\pgfqpoint{1.398551in}{2.113923in}}%
\pgfpathlineto{\pgfqpoint{1.398551in}{2.113923in}}%
\pgfpathclose%
\pgfusepath{stroke,fill}%
\end{pgfscope}%
\begin{pgfscope}%
\pgfpathrectangle{\pgfqpoint{0.786164in}{0.768110in}}{\pgfqpoint{8.851069in}{7.081890in}}%
\pgfusepath{clip}%
\pgfsetbuttcap%
\pgfsetroundjoin%
\definecolor{currentfill}{rgb}{0.276022,0.044167,0.370164}%
\pgfsetfillcolor{currentfill}%
\pgfsetfillopacity{0.700000}%
\pgfsetlinewidth{0.501875pt}%
\definecolor{currentstroke}{rgb}{1.000000,1.000000,1.000000}%
\pgfsetstrokecolor{currentstroke}%
\pgfsetstrokeopacity{0.700000}%
\pgfsetdash{}{0pt}%
\pgfpathmoveto{\pgfqpoint{3.882805in}{4.216154in}}%
\pgfpathcurveto{\pgfqpoint{3.895828in}{4.216154in}}{\pgfqpoint{3.908319in}{4.221328in}}{\pgfqpoint{3.917527in}{4.230537in}}%
\pgfpathcurveto{\pgfqpoint{3.926736in}{4.239745in}}{\pgfqpoint{3.931910in}{4.252236in}}{\pgfqpoint{3.931910in}{4.265259in}}%
\pgfpathcurveto{\pgfqpoint{3.931910in}{4.278281in}}{\pgfqpoint{3.926736in}{4.290773in}}{\pgfqpoint{3.917527in}{4.299981in}}%
\pgfpathcurveto{\pgfqpoint{3.908319in}{4.309189in}}{\pgfqpoint{3.895828in}{4.314363in}}{\pgfqpoint{3.882805in}{4.314363in}}%
\pgfpathcurveto{\pgfqpoint{3.869782in}{4.314363in}}{\pgfqpoint{3.857291in}{4.309189in}}{\pgfqpoint{3.848083in}{4.299981in}}%
\pgfpathcurveto{\pgfqpoint{3.838874in}{4.290773in}}{\pgfqpoint{3.833700in}{4.278281in}}{\pgfqpoint{3.833700in}{4.265259in}}%
\pgfpathcurveto{\pgfqpoint{3.833700in}{4.252236in}}{\pgfqpoint{3.838874in}{4.239745in}}{\pgfqpoint{3.848083in}{4.230537in}}%
\pgfpathcurveto{\pgfqpoint{3.857291in}{4.221328in}}{\pgfqpoint{3.869782in}{4.216154in}}{\pgfqpoint{3.882805in}{4.216154in}}%
\pgfpathlineto{\pgfqpoint{3.882805in}{4.216154in}}%
\pgfpathclose%
\pgfusepath{stroke,fill}%
\end{pgfscope}%
\begin{pgfscope}%
\pgfpathrectangle{\pgfqpoint{0.786164in}{0.768110in}}{\pgfqpoint{8.851069in}{7.081890in}}%
\pgfusepath{clip}%
\pgfsetbuttcap%
\pgfsetroundjoin%
\definecolor{currentfill}{rgb}{0.277941,0.056324,0.381191}%
\pgfsetfillcolor{currentfill}%
\pgfsetfillopacity{0.700000}%
\pgfsetlinewidth{0.501875pt}%
\definecolor{currentstroke}{rgb}{1.000000,1.000000,1.000000}%
\pgfsetstrokecolor{currentstroke}%
\pgfsetstrokeopacity{0.700000}%
\pgfsetdash{}{0pt}%
\pgfpathmoveto{\pgfqpoint{3.800605in}{4.128561in}}%
\pgfpathcurveto{\pgfqpoint{3.813628in}{4.128561in}}{\pgfqpoint{3.826119in}{4.133735in}}{\pgfqpoint{3.835328in}{4.142944in}}%
\pgfpathcurveto{\pgfqpoint{3.844536in}{4.152152in}}{\pgfqpoint{3.849710in}{4.164643in}}{\pgfqpoint{3.849710in}{4.177666in}}%
\pgfpathcurveto{\pgfqpoint{3.849710in}{4.190688in}}{\pgfqpoint{3.844536in}{4.203180in}}{\pgfqpoint{3.835328in}{4.212388in}}%
\pgfpathcurveto{\pgfqpoint{3.826119in}{4.221596in}}{\pgfqpoint{3.813628in}{4.226770in}}{\pgfqpoint{3.800605in}{4.226770in}}%
\pgfpathcurveto{\pgfqpoint{3.787583in}{4.226770in}}{\pgfqpoint{3.775092in}{4.221596in}}{\pgfqpoint{3.765883in}{4.212388in}}%
\pgfpathcurveto{\pgfqpoint{3.756675in}{4.203180in}}{\pgfqpoint{3.751501in}{4.190688in}}{\pgfqpoint{3.751501in}{4.177666in}}%
\pgfpathcurveto{\pgfqpoint{3.751501in}{4.164643in}}{\pgfqpoint{3.756675in}{4.152152in}}{\pgfqpoint{3.765883in}{4.142944in}}%
\pgfpathcurveto{\pgfqpoint{3.775092in}{4.133735in}}{\pgfqpoint{3.787583in}{4.128561in}}{\pgfqpoint{3.800605in}{4.128561in}}%
\pgfpathlineto{\pgfqpoint{3.800605in}{4.128561in}}%
\pgfpathclose%
\pgfusepath{stroke,fill}%
\end{pgfscope}%
\begin{pgfscope}%
\pgfpathrectangle{\pgfqpoint{0.786164in}{0.768110in}}{\pgfqpoint{8.851069in}{7.081890in}}%
\pgfusepath{clip}%
\pgfsetbuttcap%
\pgfsetroundjoin%
\definecolor{currentfill}{rgb}{0.278791,0.062145,0.386592}%
\pgfsetfillcolor{currentfill}%
\pgfsetfillopacity{0.700000}%
\pgfsetlinewidth{0.501875pt}%
\definecolor{currentstroke}{rgb}{1.000000,1.000000,1.000000}%
\pgfsetstrokecolor{currentstroke}%
\pgfsetstrokeopacity{0.700000}%
\pgfsetdash{}{0pt}%
\pgfpathmoveto{\pgfqpoint{3.754939in}{4.062866in}}%
\pgfpathcurveto{\pgfqpoint{3.767962in}{4.062866in}}{\pgfqpoint{3.780453in}{4.068040in}}{\pgfqpoint{3.789661in}{4.077249in}}%
\pgfpathcurveto{\pgfqpoint{3.798870in}{4.086457in}}{\pgfqpoint{3.804044in}{4.098948in}}{\pgfqpoint{3.804044in}{4.111971in}}%
\pgfpathcurveto{\pgfqpoint{3.804044in}{4.124994in}}{\pgfqpoint{3.798870in}{4.137485in}}{\pgfqpoint{3.789661in}{4.146693in}}%
\pgfpathcurveto{\pgfqpoint{3.780453in}{4.155902in}}{\pgfqpoint{3.767962in}{4.161076in}}{\pgfqpoint{3.754939in}{4.161076in}}%
\pgfpathcurveto{\pgfqpoint{3.741916in}{4.161076in}}{\pgfqpoint{3.729425in}{4.155902in}}{\pgfqpoint{3.720217in}{4.146693in}}%
\pgfpathcurveto{\pgfqpoint{3.711008in}{4.137485in}}{\pgfqpoint{3.705834in}{4.124994in}}{\pgfqpoint{3.705834in}{4.111971in}}%
\pgfpathcurveto{\pgfqpoint{3.705834in}{4.098948in}}{\pgfqpoint{3.711008in}{4.086457in}}{\pgfqpoint{3.720217in}{4.077249in}}%
\pgfpathcurveto{\pgfqpoint{3.729425in}{4.068040in}}{\pgfqpoint{3.741916in}{4.062866in}}{\pgfqpoint{3.754939in}{4.062866in}}%
\pgfpathlineto{\pgfqpoint{3.754939in}{4.062866in}}%
\pgfpathclose%
\pgfusepath{stroke,fill}%
\end{pgfscope}%
\begin{pgfscope}%
\pgfpathrectangle{\pgfqpoint{0.786164in}{0.768110in}}{\pgfqpoint{8.851069in}{7.081890in}}%
\pgfusepath{clip}%
\pgfsetbuttcap%
\pgfsetroundjoin%
\definecolor{currentfill}{rgb}{0.280267,0.073417,0.397163}%
\pgfsetfillcolor{currentfill}%
\pgfsetfillopacity{0.700000}%
\pgfsetlinewidth{0.501875pt}%
\definecolor{currentstroke}{rgb}{1.000000,1.000000,1.000000}%
\pgfsetstrokecolor{currentstroke}%
\pgfsetstrokeopacity{0.700000}%
\pgfsetdash{}{0pt}%
\pgfpathmoveto{\pgfqpoint{3.736672in}{4.040968in}}%
\pgfpathcurveto{\pgfqpoint{3.749695in}{4.040968in}}{\pgfqpoint{3.762186in}{4.046142in}}{\pgfqpoint{3.771395in}{4.055351in}}%
\pgfpathcurveto{\pgfqpoint{3.780603in}{4.064559in}}{\pgfqpoint{3.785777in}{4.077050in}}{\pgfqpoint{3.785777in}{4.090073in}}%
\pgfpathcurveto{\pgfqpoint{3.785777in}{4.103096in}}{\pgfqpoint{3.780603in}{4.115587in}}{\pgfqpoint{3.771395in}{4.124795in}}%
\pgfpathcurveto{\pgfqpoint{3.762186in}{4.134004in}}{\pgfqpoint{3.749695in}{4.139177in}}{\pgfqpoint{3.736672in}{4.139177in}}%
\pgfpathcurveto{\pgfqpoint{3.723650in}{4.139177in}}{\pgfqpoint{3.711159in}{4.134004in}}{\pgfqpoint{3.701950in}{4.124795in}}%
\pgfpathcurveto{\pgfqpoint{3.692742in}{4.115587in}}{\pgfqpoint{3.687568in}{4.103096in}}{\pgfqpoint{3.687568in}{4.090073in}}%
\pgfpathcurveto{\pgfqpoint{3.687568in}{4.077050in}}{\pgfqpoint{3.692742in}{4.064559in}}{\pgfqpoint{3.701950in}{4.055351in}}%
\pgfpathcurveto{\pgfqpoint{3.711159in}{4.046142in}}{\pgfqpoint{3.723650in}{4.040968in}}{\pgfqpoint{3.736672in}{4.040968in}}%
\pgfpathlineto{\pgfqpoint{3.736672in}{4.040968in}}%
\pgfpathclose%
\pgfusepath{stroke,fill}%
\end{pgfscope}%
\begin{pgfscope}%
\pgfpathrectangle{\pgfqpoint{0.786164in}{0.768110in}}{\pgfqpoint{8.851069in}{7.081890in}}%
\pgfusepath{clip}%
\pgfsetbuttcap%
\pgfsetroundjoin%
\definecolor{currentfill}{rgb}{0.280894,0.078907,0.402329}%
\pgfsetfillcolor{currentfill}%
\pgfsetfillopacity{0.700000}%
\pgfsetlinewidth{0.501875pt}%
\definecolor{currentstroke}{rgb}{1.000000,1.000000,1.000000}%
\pgfsetstrokecolor{currentstroke}%
\pgfsetstrokeopacity{0.700000}%
\pgfsetdash{}{0pt}%
\pgfpathmoveto{\pgfqpoint{3.754939in}{4.019070in}}%
\pgfpathcurveto{\pgfqpoint{3.767962in}{4.019070in}}{\pgfqpoint{3.780453in}{4.024244in}}{\pgfqpoint{3.789661in}{4.033452in}}%
\pgfpathcurveto{\pgfqpoint{3.798870in}{4.042661in}}{\pgfqpoint{3.804044in}{4.055152in}}{\pgfqpoint{3.804044in}{4.068175in}}%
\pgfpathcurveto{\pgfqpoint{3.804044in}{4.081197in}}{\pgfqpoint{3.798870in}{4.093688in}}{\pgfqpoint{3.789661in}{4.102897in}}%
\pgfpathcurveto{\pgfqpoint{3.780453in}{4.112105in}}{\pgfqpoint{3.767962in}{4.117279in}}{\pgfqpoint{3.754939in}{4.117279in}}%
\pgfpathcurveto{\pgfqpoint{3.741916in}{4.117279in}}{\pgfqpoint{3.729425in}{4.112105in}}{\pgfqpoint{3.720217in}{4.102897in}}%
\pgfpathcurveto{\pgfqpoint{3.711008in}{4.093688in}}{\pgfqpoint{3.705834in}{4.081197in}}{\pgfqpoint{3.705834in}{4.068175in}}%
\pgfpathcurveto{\pgfqpoint{3.705834in}{4.055152in}}{\pgfqpoint{3.711008in}{4.042661in}}{\pgfqpoint{3.720217in}{4.033452in}}%
\pgfpathcurveto{\pgfqpoint{3.729425in}{4.024244in}}{\pgfqpoint{3.741916in}{4.019070in}}{\pgfqpoint{3.754939in}{4.019070in}}%
\pgfpathlineto{\pgfqpoint{3.754939in}{4.019070in}}%
\pgfpathclose%
\pgfusepath{stroke,fill}%
\end{pgfscope}%
\begin{pgfscope}%
\pgfpathrectangle{\pgfqpoint{0.786164in}{0.768110in}}{\pgfqpoint{8.851069in}{7.081890in}}%
\pgfusepath{clip}%
\pgfsetbuttcap%
\pgfsetroundjoin%
\definecolor{currentfill}{rgb}{0.282327,0.094955,0.417331}%
\pgfsetfillcolor{currentfill}%
\pgfsetfillopacity{0.700000}%
\pgfsetlinewidth{0.501875pt}%
\definecolor{currentstroke}{rgb}{1.000000,1.000000,1.000000}%
\pgfsetstrokecolor{currentstroke}%
\pgfsetstrokeopacity{0.700000}%
\pgfsetdash{}{0pt}%
\pgfpathmoveto{\pgfqpoint{3.608806in}{3.909579in}}%
\pgfpathcurveto{\pgfqpoint{3.621829in}{3.909579in}}{\pgfqpoint{3.634320in}{3.914753in}}{\pgfqpoint{3.643529in}{3.923961in}}%
\pgfpathcurveto{\pgfqpoint{3.652737in}{3.933170in}}{\pgfqpoint{3.657911in}{3.945661in}}{\pgfqpoint{3.657911in}{3.958683in}}%
\pgfpathcurveto{\pgfqpoint{3.657911in}{3.971706in}}{\pgfqpoint{3.652737in}{3.984197in}}{\pgfqpoint{3.643529in}{3.993406in}}%
\pgfpathcurveto{\pgfqpoint{3.634320in}{4.002614in}}{\pgfqpoint{3.621829in}{4.007788in}}{\pgfqpoint{3.608806in}{4.007788in}}%
\pgfpathcurveto{\pgfqpoint{3.595784in}{4.007788in}}{\pgfqpoint{3.583293in}{4.002614in}}{\pgfqpoint{3.574084in}{3.993406in}}%
\pgfpathcurveto{\pgfqpoint{3.564876in}{3.984197in}}{\pgfqpoint{3.559702in}{3.971706in}}{\pgfqpoint{3.559702in}{3.958683in}}%
\pgfpathcurveto{\pgfqpoint{3.559702in}{3.945661in}}{\pgfqpoint{3.564876in}{3.933170in}}{\pgfqpoint{3.574084in}{3.923961in}}%
\pgfpathcurveto{\pgfqpoint{3.583293in}{3.914753in}}{\pgfqpoint{3.595784in}{3.909579in}}{\pgfqpoint{3.608806in}{3.909579in}}%
\pgfpathlineto{\pgfqpoint{3.608806in}{3.909579in}}%
\pgfpathclose%
\pgfusepath{stroke,fill}%
\end{pgfscope}%
\begin{pgfscope}%
\pgfpathrectangle{\pgfqpoint{0.786164in}{0.768110in}}{\pgfqpoint{8.851069in}{7.081890in}}%
\pgfusepath{clip}%
\pgfsetbuttcap%
\pgfsetroundjoin%
\definecolor{currentfill}{rgb}{0.281446,0.084320,0.407414}%
\pgfsetfillcolor{currentfill}%
\pgfsetfillopacity{0.700000}%
\pgfsetlinewidth{0.501875pt}%
\definecolor{currentstroke}{rgb}{1.000000,1.000000,1.000000}%
\pgfsetstrokecolor{currentstroke}%
\pgfsetstrokeopacity{0.700000}%
\pgfsetdash{}{0pt}%
\pgfpathmoveto{\pgfqpoint{3.818872in}{4.062866in}}%
\pgfpathcurveto{\pgfqpoint{3.831895in}{4.062866in}}{\pgfqpoint{3.844386in}{4.068040in}}{\pgfqpoint{3.853594in}{4.077249in}}%
\pgfpathcurveto{\pgfqpoint{3.862803in}{4.086457in}}{\pgfqpoint{3.867977in}{4.098948in}}{\pgfqpoint{3.867977in}{4.111971in}}%
\pgfpathcurveto{\pgfqpoint{3.867977in}{4.124994in}}{\pgfqpoint{3.862803in}{4.137485in}}{\pgfqpoint{3.853594in}{4.146693in}}%
\pgfpathcurveto{\pgfqpoint{3.844386in}{4.155902in}}{\pgfqpoint{3.831895in}{4.161076in}}{\pgfqpoint{3.818872in}{4.161076in}}%
\pgfpathcurveto{\pgfqpoint{3.805849in}{4.161076in}}{\pgfqpoint{3.793358in}{4.155902in}}{\pgfqpoint{3.784150in}{4.146693in}}%
\pgfpathcurveto{\pgfqpoint{3.774941in}{4.137485in}}{\pgfqpoint{3.769767in}{4.124994in}}{\pgfqpoint{3.769767in}{4.111971in}}%
\pgfpathcurveto{\pgfqpoint{3.769767in}{4.098948in}}{\pgfqpoint{3.774941in}{4.086457in}}{\pgfqpoint{3.784150in}{4.077249in}}%
\pgfpathcurveto{\pgfqpoint{3.793358in}{4.068040in}}{\pgfqpoint{3.805849in}{4.062866in}}{\pgfqpoint{3.818872in}{4.062866in}}%
\pgfpathlineto{\pgfqpoint{3.818872in}{4.062866in}}%
\pgfpathclose%
\pgfusepath{stroke,fill}%
\end{pgfscope}%
\begin{pgfscope}%
\pgfpathrectangle{\pgfqpoint{0.786164in}{0.768110in}}{\pgfqpoint{8.851069in}{7.081890in}}%
\pgfusepath{clip}%
\pgfsetbuttcap%
\pgfsetroundjoin%
\definecolor{currentfill}{rgb}{0.282656,0.100196,0.422160}%
\pgfsetfillcolor{currentfill}%
\pgfsetfillopacity{0.700000}%
\pgfsetlinewidth{0.501875pt}%
\definecolor{currentstroke}{rgb}{1.000000,1.000000,1.000000}%
\pgfsetstrokecolor{currentstroke}%
\pgfsetstrokeopacity{0.700000}%
\pgfsetdash{}{0pt}%
\pgfpathmoveto{\pgfqpoint{3.544873in}{3.865782in}}%
\pgfpathcurveto{\pgfqpoint{3.557896in}{3.865782in}}{\pgfqpoint{3.570387in}{3.870956in}}{\pgfqpoint{3.579596in}{3.880165in}}%
\pgfpathcurveto{\pgfqpoint{3.588804in}{3.889373in}}{\pgfqpoint{3.593978in}{3.901864in}}{\pgfqpoint{3.593978in}{3.914887in}}%
\pgfpathcurveto{\pgfqpoint{3.593978in}{3.927910in}}{\pgfqpoint{3.588804in}{3.940401in}}{\pgfqpoint{3.579596in}{3.949609in}}%
\pgfpathcurveto{\pgfqpoint{3.570387in}{3.958818in}}{\pgfqpoint{3.557896in}{3.963992in}}{\pgfqpoint{3.544873in}{3.963992in}}%
\pgfpathcurveto{\pgfqpoint{3.531851in}{3.963992in}}{\pgfqpoint{3.519360in}{3.958818in}}{\pgfqpoint{3.510151in}{3.949609in}}%
\pgfpathcurveto{\pgfqpoint{3.500943in}{3.940401in}}{\pgfqpoint{3.495769in}{3.927910in}}{\pgfqpoint{3.495769in}{3.914887in}}%
\pgfpathcurveto{\pgfqpoint{3.495769in}{3.901864in}}{\pgfqpoint{3.500943in}{3.889373in}}{\pgfqpoint{3.510151in}{3.880165in}}%
\pgfpathcurveto{\pgfqpoint{3.519360in}{3.870956in}}{\pgfqpoint{3.531851in}{3.865782in}}{\pgfqpoint{3.544873in}{3.865782in}}%
\pgfpathlineto{\pgfqpoint{3.544873in}{3.865782in}}%
\pgfpathclose%
\pgfusepath{stroke,fill}%
\end{pgfscope}%
\begin{pgfscope}%
\pgfpathrectangle{\pgfqpoint{0.786164in}{0.768110in}}{\pgfqpoint{8.851069in}{7.081890in}}%
\pgfusepath{clip}%
\pgfsetbuttcap%
\pgfsetroundjoin%
\definecolor{currentfill}{rgb}{0.282656,0.100196,0.422160}%
\pgfsetfillcolor{currentfill}%
\pgfsetfillopacity{0.700000}%
\pgfsetlinewidth{0.501875pt}%
\definecolor{currentstroke}{rgb}{1.000000,1.000000,1.000000}%
\pgfsetstrokecolor{currentstroke}%
\pgfsetstrokeopacity{0.700000}%
\pgfsetdash{}{0pt}%
\pgfpathmoveto{\pgfqpoint{3.508340in}{3.778189in}}%
\pgfpathcurveto{\pgfqpoint{3.521363in}{3.778189in}}{\pgfqpoint{3.533854in}{3.783363in}}{\pgfqpoint{3.543062in}{3.792572in}}%
\pgfpathcurveto{\pgfqpoint{3.552271in}{3.801780in}}{\pgfqpoint{3.557445in}{3.814271in}}{\pgfqpoint{3.557445in}{3.827294in}}%
\pgfpathcurveto{\pgfqpoint{3.557445in}{3.840317in}}{\pgfqpoint{3.552271in}{3.852808in}}{\pgfqpoint{3.543062in}{3.862016in}}%
\pgfpathcurveto{\pgfqpoint{3.533854in}{3.871225in}}{\pgfqpoint{3.521363in}{3.876399in}}{\pgfqpoint{3.508340in}{3.876399in}}%
\pgfpathcurveto{\pgfqpoint{3.495318in}{3.876399in}}{\pgfqpoint{3.482826in}{3.871225in}}{\pgfqpoint{3.473618in}{3.862016in}}%
\pgfpathcurveto{\pgfqpoint{3.464410in}{3.852808in}}{\pgfqpoint{3.459236in}{3.840317in}}{\pgfqpoint{3.459236in}{3.827294in}}%
\pgfpathcurveto{\pgfqpoint{3.459236in}{3.814271in}}{\pgfqpoint{3.464410in}{3.801780in}}{\pgfqpoint{3.473618in}{3.792572in}}%
\pgfpathcurveto{\pgfqpoint{3.482826in}{3.783363in}}{\pgfqpoint{3.495318in}{3.778189in}}{\pgfqpoint{3.508340in}{3.778189in}}%
\pgfpathlineto{\pgfqpoint{3.508340in}{3.778189in}}%
\pgfpathclose%
\pgfusepath{stroke,fill}%
\end{pgfscope}%
\begin{pgfscope}%
\pgfpathrectangle{\pgfqpoint{0.786164in}{0.768110in}}{\pgfqpoint{8.851069in}{7.081890in}}%
\pgfusepath{clip}%
\pgfsetbuttcap%
\pgfsetroundjoin%
\definecolor{currentfill}{rgb}{0.282656,0.100196,0.422160}%
\pgfsetfillcolor{currentfill}%
\pgfsetfillopacity{0.700000}%
\pgfsetlinewidth{0.501875pt}%
\definecolor{currentstroke}{rgb}{1.000000,1.000000,1.000000}%
\pgfsetstrokecolor{currentstroke}%
\pgfsetstrokeopacity{0.700000}%
\pgfsetdash{}{0pt}%
\pgfpathmoveto{\pgfqpoint{3.471807in}{3.756291in}}%
\pgfpathcurveto{\pgfqpoint{3.484830in}{3.756291in}}{\pgfqpoint{3.497321in}{3.761465in}}{\pgfqpoint{3.506529in}{3.770674in}}%
\pgfpathcurveto{\pgfqpoint{3.515738in}{3.779882in}}{\pgfqpoint{3.520912in}{3.792373in}}{\pgfqpoint{3.520912in}{3.805396in}}%
\pgfpathcurveto{\pgfqpoint{3.520912in}{3.818418in}}{\pgfqpoint{3.515738in}{3.830910in}}{\pgfqpoint{3.506529in}{3.840118in}}%
\pgfpathcurveto{\pgfqpoint{3.497321in}{3.849326in}}{\pgfqpoint{3.484830in}{3.854500in}}{\pgfqpoint{3.471807in}{3.854500in}}%
\pgfpathcurveto{\pgfqpoint{3.458784in}{3.854500in}}{\pgfqpoint{3.446293in}{3.849326in}}{\pgfqpoint{3.437085in}{3.840118in}}%
\pgfpathcurveto{\pgfqpoint{3.427876in}{3.830910in}}{\pgfqpoint{3.422702in}{3.818418in}}{\pgfqpoint{3.422702in}{3.805396in}}%
\pgfpathcurveto{\pgfqpoint{3.422702in}{3.792373in}}{\pgfqpoint{3.427876in}{3.779882in}}{\pgfqpoint{3.437085in}{3.770674in}}%
\pgfpathcurveto{\pgfqpoint{3.446293in}{3.761465in}}{\pgfqpoint{3.458784in}{3.756291in}}{\pgfqpoint{3.471807in}{3.756291in}}%
\pgfpathlineto{\pgfqpoint{3.471807in}{3.756291in}}%
\pgfpathclose%
\pgfusepath{stroke,fill}%
\end{pgfscope}%
\begin{pgfscope}%
\pgfpathrectangle{\pgfqpoint{0.786164in}{0.768110in}}{\pgfqpoint{8.851069in}{7.081890in}}%
\pgfusepath{clip}%
\pgfsetbuttcap%
\pgfsetroundjoin%
\definecolor{currentfill}{rgb}{0.283197,0.115680,0.436115}%
\pgfsetfillcolor{currentfill}%
\pgfsetfillopacity{0.700000}%
\pgfsetlinewidth{0.501875pt}%
\definecolor{currentstroke}{rgb}{1.000000,1.000000,1.000000}%
\pgfsetstrokecolor{currentstroke}%
\pgfsetstrokeopacity{0.700000}%
\pgfsetdash{}{0pt}%
\pgfpathmoveto{\pgfqpoint{3.252608in}{3.668698in}}%
\pgfpathcurveto{\pgfqpoint{3.265631in}{3.668698in}}{\pgfqpoint{3.278122in}{3.673872in}}{\pgfqpoint{3.287330in}{3.683081in}}%
\pgfpathcurveto{\pgfqpoint{3.296539in}{3.692289in}}{\pgfqpoint{3.301713in}{3.704780in}}{\pgfqpoint{3.301713in}{3.717803in}}%
\pgfpathcurveto{\pgfqpoint{3.301713in}{3.730826in}}{\pgfqpoint{3.296539in}{3.743317in}}{\pgfqpoint{3.287330in}{3.752525in}}%
\pgfpathcurveto{\pgfqpoint{3.278122in}{3.761733in}}{\pgfqpoint{3.265631in}{3.766907in}}{\pgfqpoint{3.252608in}{3.766907in}}%
\pgfpathcurveto{\pgfqpoint{3.239585in}{3.766907in}}{\pgfqpoint{3.227094in}{3.761733in}}{\pgfqpoint{3.217886in}{3.752525in}}%
\pgfpathcurveto{\pgfqpoint{3.208678in}{3.743317in}}{\pgfqpoint{3.203504in}{3.730826in}}{\pgfqpoint{3.203504in}{3.717803in}}%
\pgfpathcurveto{\pgfqpoint{3.203504in}{3.704780in}}{\pgfqpoint{3.208678in}{3.692289in}}{\pgfqpoint{3.217886in}{3.683081in}}%
\pgfpathcurveto{\pgfqpoint{3.227094in}{3.673872in}}{\pgfqpoint{3.239585in}{3.668698in}}{\pgfqpoint{3.252608in}{3.668698in}}%
\pgfpathlineto{\pgfqpoint{3.252608in}{3.668698in}}%
\pgfpathclose%
\pgfusepath{stroke,fill}%
\end{pgfscope}%
\begin{pgfscope}%
\pgfpathrectangle{\pgfqpoint{0.786164in}{0.768110in}}{\pgfqpoint{8.851069in}{7.081890in}}%
\pgfusepath{clip}%
\pgfsetbuttcap%
\pgfsetroundjoin%
\definecolor{currentfill}{rgb}{0.283229,0.120777,0.440584}%
\pgfsetfillcolor{currentfill}%
\pgfsetfillopacity{0.700000}%
\pgfsetlinewidth{0.501875pt}%
\definecolor{currentstroke}{rgb}{1.000000,1.000000,1.000000}%
\pgfsetstrokecolor{currentstroke}%
\pgfsetstrokeopacity{0.700000}%
\pgfsetdash{}{0pt}%
\pgfpathmoveto{\pgfqpoint{3.051676in}{3.756291in}}%
\pgfpathcurveto{\pgfqpoint{3.064699in}{3.756291in}}{\pgfqpoint{3.077190in}{3.761465in}}{\pgfqpoint{3.086398in}{3.770674in}}%
\pgfpathcurveto{\pgfqpoint{3.095607in}{3.779882in}}{\pgfqpoint{3.100780in}{3.792373in}}{\pgfqpoint{3.100780in}{3.805396in}}%
\pgfpathcurveto{\pgfqpoint{3.100780in}{3.818418in}}{\pgfqpoint{3.095607in}{3.830910in}}{\pgfqpoint{3.086398in}{3.840118in}}%
\pgfpathcurveto{\pgfqpoint{3.077190in}{3.849326in}}{\pgfqpoint{3.064699in}{3.854500in}}{\pgfqpoint{3.051676in}{3.854500in}}%
\pgfpathcurveto{\pgfqpoint{3.038653in}{3.854500in}}{\pgfqpoint{3.026162in}{3.849326in}}{\pgfqpoint{3.016954in}{3.840118in}}%
\pgfpathcurveto{\pgfqpoint{3.007745in}{3.830910in}}{\pgfqpoint{3.002571in}{3.818418in}}{\pgfqpoint{3.002571in}{3.805396in}}%
\pgfpathcurveto{\pgfqpoint{3.002571in}{3.792373in}}{\pgfqpoint{3.007745in}{3.779882in}}{\pgfqpoint{3.016954in}{3.770674in}}%
\pgfpathcurveto{\pgfqpoint{3.026162in}{3.761465in}}{\pgfqpoint{3.038653in}{3.756291in}}{\pgfqpoint{3.051676in}{3.756291in}}%
\pgfpathlineto{\pgfqpoint{3.051676in}{3.756291in}}%
\pgfpathclose%
\pgfusepath{stroke,fill}%
\end{pgfscope}%
\begin{pgfscope}%
\pgfpathrectangle{\pgfqpoint{0.786164in}{0.768110in}}{\pgfqpoint{8.851069in}{7.081890in}}%
\pgfusepath{clip}%
\pgfsetbuttcap%
\pgfsetroundjoin%
\definecolor{currentfill}{rgb}{0.283187,0.125848,0.444960}%
\pgfsetfillcolor{currentfill}%
\pgfsetfillopacity{0.700000}%
\pgfsetlinewidth{0.501875pt}%
\definecolor{currentstroke}{rgb}{1.000000,1.000000,1.000000}%
\pgfsetstrokecolor{currentstroke}%
\pgfsetstrokeopacity{0.700000}%
\pgfsetdash{}{0pt}%
\pgfpathmoveto{\pgfqpoint{3.234342in}{3.756291in}}%
\pgfpathcurveto{\pgfqpoint{3.247364in}{3.756291in}}{\pgfqpoint{3.259855in}{3.761465in}}{\pgfqpoint{3.269064in}{3.770674in}}%
\pgfpathcurveto{\pgfqpoint{3.278272in}{3.779882in}}{\pgfqpoint{3.283446in}{3.792373in}}{\pgfqpoint{3.283446in}{3.805396in}}%
\pgfpathcurveto{\pgfqpoint{3.283446in}{3.818418in}}{\pgfqpoint{3.278272in}{3.830910in}}{\pgfqpoint{3.269064in}{3.840118in}}%
\pgfpathcurveto{\pgfqpoint{3.259855in}{3.849326in}}{\pgfqpoint{3.247364in}{3.854500in}}{\pgfqpoint{3.234342in}{3.854500in}}%
\pgfpathcurveto{\pgfqpoint{3.221319in}{3.854500in}}{\pgfqpoint{3.208828in}{3.849326in}}{\pgfqpoint{3.199619in}{3.840118in}}%
\pgfpathcurveto{\pgfqpoint{3.190411in}{3.830910in}}{\pgfqpoint{3.185237in}{3.818418in}}{\pgfqpoint{3.185237in}{3.805396in}}%
\pgfpathcurveto{\pgfqpoint{3.185237in}{3.792373in}}{\pgfqpoint{3.190411in}{3.779882in}}{\pgfqpoint{3.199619in}{3.770674in}}%
\pgfpathcurveto{\pgfqpoint{3.208828in}{3.761465in}}{\pgfqpoint{3.221319in}{3.756291in}}{\pgfqpoint{3.234342in}{3.756291in}}%
\pgfpathlineto{\pgfqpoint{3.234342in}{3.756291in}}%
\pgfpathclose%
\pgfusepath{stroke,fill}%
\end{pgfscope}%
\begin{pgfscope}%
\pgfpathrectangle{\pgfqpoint{0.786164in}{0.768110in}}{\pgfqpoint{8.851069in}{7.081890in}}%
\pgfusepath{clip}%
\pgfsetbuttcap%
\pgfsetroundjoin%
\definecolor{currentfill}{rgb}{0.282884,0.135920,0.453427}%
\pgfsetfillcolor{currentfill}%
\pgfsetfillopacity{0.700000}%
\pgfsetlinewidth{0.501875pt}%
\definecolor{currentstroke}{rgb}{1.000000,1.000000,1.000000}%
\pgfsetstrokecolor{currentstroke}%
\pgfsetstrokeopacity{0.700000}%
\pgfsetdash{}{0pt}%
\pgfpathmoveto{\pgfqpoint{3.143009in}{3.690596in}}%
\pgfpathcurveto{\pgfqpoint{3.156031in}{3.690596in}}{\pgfqpoint{3.168523in}{3.695770in}}{\pgfqpoint{3.177731in}{3.704979in}}%
\pgfpathcurveto{\pgfqpoint{3.186939in}{3.714187in}}{\pgfqpoint{3.192113in}{3.726678in}}{\pgfqpoint{3.192113in}{3.739701in}}%
\pgfpathcurveto{\pgfqpoint{3.192113in}{3.752724in}}{\pgfqpoint{3.186939in}{3.765215in}}{\pgfqpoint{3.177731in}{3.774423in}}%
\pgfpathcurveto{\pgfqpoint{3.168523in}{3.783632in}}{\pgfqpoint{3.156031in}{3.788806in}}{\pgfqpoint{3.143009in}{3.788806in}}%
\pgfpathcurveto{\pgfqpoint{3.129986in}{3.788806in}}{\pgfqpoint{3.117495in}{3.783632in}}{\pgfqpoint{3.108286in}{3.774423in}}%
\pgfpathcurveto{\pgfqpoint{3.099078in}{3.765215in}}{\pgfqpoint{3.093904in}{3.752724in}}{\pgfqpoint{3.093904in}{3.739701in}}%
\pgfpathcurveto{\pgfqpoint{3.093904in}{3.726678in}}{\pgfqpoint{3.099078in}{3.714187in}}{\pgfqpoint{3.108286in}{3.704979in}}%
\pgfpathcurveto{\pgfqpoint{3.117495in}{3.695770in}}{\pgfqpoint{3.129986in}{3.690596in}}{\pgfqpoint{3.143009in}{3.690596in}}%
\pgfpathlineto{\pgfqpoint{3.143009in}{3.690596in}}%
\pgfpathclose%
\pgfusepath{stroke,fill}%
\end{pgfscope}%
\begin{pgfscope}%
\pgfpathrectangle{\pgfqpoint{0.786164in}{0.768110in}}{\pgfqpoint{8.851069in}{7.081890in}}%
\pgfusepath{clip}%
\pgfsetbuttcap%
\pgfsetroundjoin%
\definecolor{currentfill}{rgb}{0.281412,0.155834,0.469201}%
\pgfsetfillcolor{currentfill}%
\pgfsetfillopacity{0.700000}%
\pgfsetlinewidth{0.501875pt}%
\definecolor{currentstroke}{rgb}{1.000000,1.000000,1.000000}%
\pgfsetstrokecolor{currentstroke}%
\pgfsetstrokeopacity{0.700000}%
\pgfsetdash{}{0pt}%
\pgfpathmoveto{\pgfqpoint{2.969476in}{3.603003in}}%
\pgfpathcurveto{\pgfqpoint{2.982499in}{3.603003in}}{\pgfqpoint{2.994990in}{3.608177in}}{\pgfqpoint{3.004198in}{3.617386in}}%
\pgfpathcurveto{\pgfqpoint{3.013407in}{3.626594in}}{\pgfqpoint{3.018581in}{3.639085in}}{\pgfqpoint{3.018581in}{3.652108in}}%
\pgfpathcurveto{\pgfqpoint{3.018581in}{3.665131in}}{\pgfqpoint{3.013407in}{3.677622in}}{\pgfqpoint{3.004198in}{3.686830in}}%
\pgfpathcurveto{\pgfqpoint{2.994990in}{3.696039in}}{\pgfqpoint{2.982499in}{3.701213in}}{\pgfqpoint{2.969476in}{3.701213in}}%
\pgfpathcurveto{\pgfqpoint{2.956454in}{3.701213in}}{\pgfqpoint{2.943962in}{3.696039in}}{\pgfqpoint{2.934754in}{3.686830in}}%
\pgfpathcurveto{\pgfqpoint{2.925546in}{3.677622in}}{\pgfqpoint{2.920372in}{3.665131in}}{\pgfqpoint{2.920372in}{3.652108in}}%
\pgfpathcurveto{\pgfqpoint{2.920372in}{3.639085in}}{\pgfqpoint{2.925546in}{3.626594in}}{\pgfqpoint{2.934754in}{3.617386in}}%
\pgfpathcurveto{\pgfqpoint{2.943962in}{3.608177in}}{\pgfqpoint{2.956454in}{3.603003in}}{\pgfqpoint{2.969476in}{3.603003in}}%
\pgfpathlineto{\pgfqpoint{2.969476in}{3.603003in}}%
\pgfpathclose%
\pgfusepath{stroke,fill}%
\end{pgfscope}%
\begin{pgfscope}%
\pgfpathrectangle{\pgfqpoint{0.786164in}{0.768110in}}{\pgfqpoint{8.851069in}{7.081890in}}%
\pgfusepath{clip}%
\pgfsetbuttcap%
\pgfsetroundjoin%
\definecolor{currentfill}{rgb}{0.277134,0.185228,0.489898}%
\pgfsetfillcolor{currentfill}%
\pgfsetfillopacity{0.700000}%
\pgfsetlinewidth{0.501875pt}%
\definecolor{currentstroke}{rgb}{1.000000,1.000000,1.000000}%
\pgfsetstrokecolor{currentstroke}%
\pgfsetstrokeopacity{0.700000}%
\pgfsetdash{}{0pt}%
\pgfpathmoveto{\pgfqpoint{3.234342in}{3.515411in}}%
\pgfpathcurveto{\pgfqpoint{3.247364in}{3.515411in}}{\pgfqpoint{3.259855in}{3.520585in}}{\pgfqpoint{3.269064in}{3.529793in}}%
\pgfpathcurveto{\pgfqpoint{3.278272in}{3.539001in}}{\pgfqpoint{3.283446in}{3.551492in}}{\pgfqpoint{3.283446in}{3.564515in}}%
\pgfpathcurveto{\pgfqpoint{3.283446in}{3.577538in}}{\pgfqpoint{3.278272in}{3.590029in}}{\pgfqpoint{3.269064in}{3.599237in}}%
\pgfpathcurveto{\pgfqpoint{3.259855in}{3.608446in}}{\pgfqpoint{3.247364in}{3.613620in}}{\pgfqpoint{3.234342in}{3.613620in}}%
\pgfpathcurveto{\pgfqpoint{3.221319in}{3.613620in}}{\pgfqpoint{3.208828in}{3.608446in}}{\pgfqpoint{3.199619in}{3.599237in}}%
\pgfpathcurveto{\pgfqpoint{3.190411in}{3.590029in}}{\pgfqpoint{3.185237in}{3.577538in}}{\pgfqpoint{3.185237in}{3.564515in}}%
\pgfpathcurveto{\pgfqpoint{3.185237in}{3.551492in}}{\pgfqpoint{3.190411in}{3.539001in}}{\pgfqpoint{3.199619in}{3.529793in}}%
\pgfpathcurveto{\pgfqpoint{3.208828in}{3.520585in}}{\pgfqpoint{3.221319in}{3.515411in}}{\pgfqpoint{3.234342in}{3.515411in}}%
\pgfpathlineto{\pgfqpoint{3.234342in}{3.515411in}}%
\pgfpathclose%
\pgfusepath{stroke,fill}%
\end{pgfscope}%
\begin{pgfscope}%
\pgfpathrectangle{\pgfqpoint{0.786164in}{0.768110in}}{\pgfqpoint{8.851069in}{7.081890in}}%
\pgfusepath{clip}%
\pgfsetbuttcap%
\pgfsetroundjoin%
\definecolor{currentfill}{rgb}{0.250425,0.274290,0.533103}%
\pgfsetfillcolor{currentfill}%
\pgfsetfillopacity{0.700000}%
\pgfsetlinewidth{0.501875pt}%
\definecolor{currentstroke}{rgb}{1.000000,1.000000,1.000000}%
\pgfsetstrokecolor{currentstroke}%
\pgfsetstrokeopacity{0.700000}%
\pgfsetdash{}{0pt}%
\pgfpathmoveto{\pgfqpoint{2.932943in}{3.230733in}}%
\pgfpathcurveto{\pgfqpoint{2.945966in}{3.230733in}}{\pgfqpoint{2.958457in}{3.235907in}}{\pgfqpoint{2.967665in}{3.245116in}}%
\pgfpathcurveto{\pgfqpoint{2.976874in}{3.254324in}}{\pgfqpoint{2.982048in}{3.266815in}}{\pgfqpoint{2.982048in}{3.279838in}}%
\pgfpathcurveto{\pgfqpoint{2.982048in}{3.292861in}}{\pgfqpoint{2.976874in}{3.305352in}}{\pgfqpoint{2.967665in}{3.314560in}}%
\pgfpathcurveto{\pgfqpoint{2.958457in}{3.323769in}}{\pgfqpoint{2.945966in}{3.328943in}}{\pgfqpoint{2.932943in}{3.328943in}}%
\pgfpathcurveto{\pgfqpoint{2.919920in}{3.328943in}}{\pgfqpoint{2.907429in}{3.323769in}}{\pgfqpoint{2.898221in}{3.314560in}}%
\pgfpathcurveto{\pgfqpoint{2.889012in}{3.305352in}}{\pgfqpoint{2.883838in}{3.292861in}}{\pgfqpoint{2.883838in}{3.279838in}}%
\pgfpathcurveto{\pgfqpoint{2.883838in}{3.266815in}}{\pgfqpoint{2.889012in}{3.254324in}}{\pgfqpoint{2.898221in}{3.245116in}}%
\pgfpathcurveto{\pgfqpoint{2.907429in}{3.235907in}}{\pgfqpoint{2.919920in}{3.230733in}}{\pgfqpoint{2.932943in}{3.230733in}}%
\pgfpathlineto{\pgfqpoint{2.932943in}{3.230733in}}%
\pgfpathclose%
\pgfusepath{stroke,fill}%
\end{pgfscope}%
\begin{pgfscope}%
\pgfpathrectangle{\pgfqpoint{0.786164in}{0.768110in}}{\pgfqpoint{8.851069in}{7.081890in}}%
\pgfusepath{clip}%
\pgfsetbuttcap%
\pgfsetroundjoin%
\definecolor{currentfill}{rgb}{0.255645,0.260703,0.528312}%
\pgfsetfillcolor{currentfill}%
\pgfsetfillopacity{0.700000}%
\pgfsetlinewidth{0.501875pt}%
\definecolor{currentstroke}{rgb}{1.000000,1.000000,1.000000}%
\pgfsetstrokecolor{currentstroke}%
\pgfsetstrokeopacity{0.700000}%
\pgfsetdash{}{0pt}%
\pgfpathmoveto{\pgfqpoint{3.033409in}{3.252632in}}%
\pgfpathcurveto{\pgfqpoint{3.046432in}{3.252632in}}{\pgfqpoint{3.058923in}{3.257806in}}{\pgfqpoint{3.068131in}{3.267014in}}%
\pgfpathcurveto{\pgfqpoint{3.077340in}{3.276223in}}{\pgfqpoint{3.082514in}{3.288714in}}{\pgfqpoint{3.082514in}{3.301736in}}%
\pgfpathcurveto{\pgfqpoint{3.082514in}{3.314759in}}{\pgfqpoint{3.077340in}{3.327250in}}{\pgfqpoint{3.068131in}{3.336459in}}%
\pgfpathcurveto{\pgfqpoint{3.058923in}{3.345667in}}{\pgfqpoint{3.046432in}{3.350841in}}{\pgfqpoint{3.033409in}{3.350841in}}%
\pgfpathcurveto{\pgfqpoint{3.020387in}{3.350841in}}{\pgfqpoint{3.007895in}{3.345667in}}{\pgfqpoint{2.998687in}{3.336459in}}%
\pgfpathcurveto{\pgfqpoint{2.989479in}{3.327250in}}{\pgfqpoint{2.984305in}{3.314759in}}{\pgfqpoint{2.984305in}{3.301736in}}%
\pgfpathcurveto{\pgfqpoint{2.984305in}{3.288714in}}{\pgfqpoint{2.989479in}{3.276223in}}{\pgfqpoint{2.998687in}{3.267014in}}%
\pgfpathcurveto{\pgfqpoint{3.007895in}{3.257806in}}{\pgfqpoint{3.020387in}{3.252632in}}{\pgfqpoint{3.033409in}{3.252632in}}%
\pgfpathlineto{\pgfqpoint{3.033409in}{3.252632in}}%
\pgfpathclose%
\pgfusepath{stroke,fill}%
\end{pgfscope}%
\begin{pgfscope}%
\pgfpathrectangle{\pgfqpoint{0.786164in}{0.768110in}}{\pgfqpoint{8.851069in}{7.081890in}}%
\pgfusepath{clip}%
\pgfsetbuttcap%
\pgfsetroundjoin%
\definecolor{currentfill}{rgb}{0.243113,0.292092,0.538516}%
\pgfsetfillcolor{currentfill}%
\pgfsetfillopacity{0.700000}%
\pgfsetlinewidth{0.501875pt}%
\definecolor{currentstroke}{rgb}{1.000000,1.000000,1.000000}%
\pgfsetstrokecolor{currentstroke}%
\pgfsetstrokeopacity{0.700000}%
\pgfsetdash{}{0pt}%
\pgfpathmoveto{\pgfqpoint{2.768544in}{3.099344in}}%
\pgfpathcurveto{\pgfqpoint{2.781567in}{3.099344in}}{\pgfqpoint{2.794058in}{3.104518in}}{\pgfqpoint{2.803266in}{3.113726in}}%
\pgfpathcurveto{\pgfqpoint{2.812475in}{3.122935in}}{\pgfqpoint{2.817649in}{3.135426in}}{\pgfqpoint{2.817649in}{3.148449in}}%
\pgfpathcurveto{\pgfqpoint{2.817649in}{3.161471in}}{\pgfqpoint{2.812475in}{3.173962in}}{\pgfqpoint{2.803266in}{3.183171in}}%
\pgfpathcurveto{\pgfqpoint{2.794058in}{3.192379in}}{\pgfqpoint{2.781567in}{3.197553in}}{\pgfqpoint{2.768544in}{3.197553in}}%
\pgfpathcurveto{\pgfqpoint{2.755521in}{3.197553in}}{\pgfqpoint{2.743030in}{3.192379in}}{\pgfqpoint{2.733822in}{3.183171in}}%
\pgfpathcurveto{\pgfqpoint{2.724613in}{3.173962in}}{\pgfqpoint{2.719439in}{3.161471in}}{\pgfqpoint{2.719439in}{3.148449in}}%
\pgfpathcurveto{\pgfqpoint{2.719439in}{3.135426in}}{\pgfqpoint{2.724613in}{3.122935in}}{\pgfqpoint{2.733822in}{3.113726in}}%
\pgfpathcurveto{\pgfqpoint{2.743030in}{3.104518in}}{\pgfqpoint{2.755521in}{3.099344in}}{\pgfqpoint{2.768544in}{3.099344in}}%
\pgfpathlineto{\pgfqpoint{2.768544in}{3.099344in}}%
\pgfpathclose%
\pgfusepath{stroke,fill}%
\end{pgfscope}%
\begin{pgfscope}%
\pgfpathrectangle{\pgfqpoint{0.786164in}{0.768110in}}{\pgfqpoint{8.851069in}{7.081890in}}%
\pgfusepath{clip}%
\pgfsetbuttcap%
\pgfsetroundjoin%
\definecolor{currentfill}{rgb}{0.282290,0.145912,0.461510}%
\pgfsetfillcolor{currentfill}%
\pgfsetfillopacity{0.700000}%
\pgfsetlinewidth{0.501875pt}%
\definecolor{currentstroke}{rgb}{1.000000,1.000000,1.000000}%
\pgfsetstrokecolor{currentstroke}%
\pgfsetstrokeopacity{0.700000}%
\pgfsetdash{}{0pt}%
\pgfpathmoveto{\pgfqpoint{1.955681in}{3.033649in}}%
\pgfpathcurveto{\pgfqpoint{1.968704in}{3.033649in}}{\pgfqpoint{1.981195in}{3.038823in}}{\pgfqpoint{1.990404in}{3.048032in}}%
\pgfpathcurveto{\pgfqpoint{1.999612in}{3.057240in}}{\pgfqpoint{2.004786in}{3.069731in}}{\pgfqpoint{2.004786in}{3.082754in}}%
\pgfpathcurveto{\pgfqpoint{2.004786in}{3.095777in}}{\pgfqpoint{1.999612in}{3.108268in}}{\pgfqpoint{1.990404in}{3.117476in}}%
\pgfpathcurveto{\pgfqpoint{1.981195in}{3.126685in}}{\pgfqpoint{1.968704in}{3.131859in}}{\pgfqpoint{1.955681in}{3.131859in}}%
\pgfpathcurveto{\pgfqpoint{1.942659in}{3.131859in}}{\pgfqpoint{1.930168in}{3.126685in}}{\pgfqpoint{1.920959in}{3.117476in}}%
\pgfpathcurveto{\pgfqpoint{1.911751in}{3.108268in}}{\pgfqpoint{1.906577in}{3.095777in}}{\pgfqpoint{1.906577in}{3.082754in}}%
\pgfpathcurveto{\pgfqpoint{1.906577in}{3.069731in}}{\pgfqpoint{1.911751in}{3.057240in}}{\pgfqpoint{1.920959in}{3.048032in}}%
\pgfpathcurveto{\pgfqpoint{1.930168in}{3.038823in}}{\pgfqpoint{1.942659in}{3.033649in}}{\pgfqpoint{1.955681in}{3.033649in}}%
\pgfpathlineto{\pgfqpoint{1.955681in}{3.033649in}}%
\pgfpathclose%
\pgfusepath{stroke,fill}%
\end{pgfscope}%
\begin{pgfscope}%
\pgfpathrectangle{\pgfqpoint{0.786164in}{0.768110in}}{\pgfqpoint{8.851069in}{7.081890in}}%
\pgfusepath{clip}%
\pgfsetbuttcap%
\pgfsetroundjoin%
\definecolor{currentfill}{rgb}{0.282290,0.145912,0.461510}%
\pgfsetfillcolor{currentfill}%
\pgfsetfillopacity{0.700000}%
\pgfsetlinewidth{0.501875pt}%
\definecolor{currentstroke}{rgb}{1.000000,1.000000,1.000000}%
\pgfsetstrokecolor{currentstroke}%
\pgfsetstrokeopacity{0.700000}%
\pgfsetdash{}{0pt}%
\pgfpathmoveto{\pgfqpoint{1.973948in}{3.033649in}}%
\pgfpathcurveto{\pgfqpoint{1.986971in}{3.033649in}}{\pgfqpoint{1.999462in}{3.038823in}}{\pgfqpoint{2.008670in}{3.048032in}}%
\pgfpathcurveto{\pgfqpoint{2.017879in}{3.057240in}}{\pgfqpoint{2.023053in}{3.069731in}}{\pgfqpoint{2.023053in}{3.082754in}}%
\pgfpathcurveto{\pgfqpoint{2.023053in}{3.095777in}}{\pgfqpoint{2.017879in}{3.108268in}}{\pgfqpoint{2.008670in}{3.117476in}}%
\pgfpathcurveto{\pgfqpoint{1.999462in}{3.126685in}}{\pgfqpoint{1.986971in}{3.131859in}}{\pgfqpoint{1.973948in}{3.131859in}}%
\pgfpathcurveto{\pgfqpoint{1.960925in}{3.131859in}}{\pgfqpoint{1.948434in}{3.126685in}}{\pgfqpoint{1.939226in}{3.117476in}}%
\pgfpathcurveto{\pgfqpoint{1.930017in}{3.108268in}}{\pgfqpoint{1.924843in}{3.095777in}}{\pgfqpoint{1.924843in}{3.082754in}}%
\pgfpathcurveto{\pgfqpoint{1.924843in}{3.069731in}}{\pgfqpoint{1.930017in}{3.057240in}}{\pgfqpoint{1.939226in}{3.048032in}}%
\pgfpathcurveto{\pgfqpoint{1.948434in}{3.038823in}}{\pgfqpoint{1.960925in}{3.033649in}}{\pgfqpoint{1.973948in}{3.033649in}}%
\pgfpathlineto{\pgfqpoint{1.973948in}{3.033649in}}%
\pgfpathclose%
\pgfusepath{stroke,fill}%
\end{pgfscope}%
\begin{pgfscope}%
\pgfpathrectangle{\pgfqpoint{0.786164in}{0.768110in}}{\pgfqpoint{8.851069in}{7.081890in}}%
\pgfusepath{clip}%
\pgfsetbuttcap%
\pgfsetroundjoin%
\definecolor{currentfill}{rgb}{0.282290,0.145912,0.461510}%
\pgfsetfillcolor{currentfill}%
\pgfsetfillopacity{0.700000}%
\pgfsetlinewidth{0.501875pt}%
\definecolor{currentstroke}{rgb}{1.000000,1.000000,1.000000}%
\pgfsetstrokecolor{currentstroke}%
\pgfsetstrokeopacity{0.700000}%
\pgfsetdash{}{0pt}%
\pgfpathmoveto{\pgfqpoint{2.083547in}{3.165039in}}%
\pgfpathcurveto{\pgfqpoint{2.096570in}{3.165039in}}{\pgfqpoint{2.109061in}{3.170213in}}{\pgfqpoint{2.118270in}{3.179421in}}%
\pgfpathcurveto{\pgfqpoint{2.127478in}{3.188630in}}{\pgfqpoint{2.132652in}{3.201121in}}{\pgfqpoint{2.132652in}{3.214143in}}%
\pgfpathcurveto{\pgfqpoint{2.132652in}{3.227166in}}{\pgfqpoint{2.127478in}{3.239657in}}{\pgfqpoint{2.118270in}{3.248866in}}%
\pgfpathcurveto{\pgfqpoint{2.109061in}{3.258074in}}{\pgfqpoint{2.096570in}{3.263248in}}{\pgfqpoint{2.083547in}{3.263248in}}%
\pgfpathcurveto{\pgfqpoint{2.070525in}{3.263248in}}{\pgfqpoint{2.058034in}{3.258074in}}{\pgfqpoint{2.048825in}{3.248866in}}%
\pgfpathcurveto{\pgfqpoint{2.039617in}{3.239657in}}{\pgfqpoint{2.034443in}{3.227166in}}{\pgfqpoint{2.034443in}{3.214143in}}%
\pgfpathcurveto{\pgfqpoint{2.034443in}{3.201121in}}{\pgfqpoint{2.039617in}{3.188630in}}{\pgfqpoint{2.048825in}{3.179421in}}%
\pgfpathcurveto{\pgfqpoint{2.058034in}{3.170213in}}{\pgfqpoint{2.070525in}{3.165039in}}{\pgfqpoint{2.083547in}{3.165039in}}%
\pgfpathlineto{\pgfqpoint{2.083547in}{3.165039in}}%
\pgfpathclose%
\pgfusepath{stroke,fill}%
\end{pgfscope}%
\begin{pgfscope}%
\pgfpathrectangle{\pgfqpoint{0.786164in}{0.768110in}}{\pgfqpoint{8.851069in}{7.081890in}}%
\pgfusepath{clip}%
\pgfsetbuttcap%
\pgfsetroundjoin%
\definecolor{currentfill}{rgb}{0.282290,0.145912,0.461510}%
\pgfsetfillcolor{currentfill}%
\pgfsetfillopacity{0.700000}%
\pgfsetlinewidth{0.501875pt}%
\definecolor{currentstroke}{rgb}{1.000000,1.000000,1.000000}%
\pgfsetstrokecolor{currentstroke}%
\pgfsetstrokeopacity{0.700000}%
\pgfsetdash{}{0pt}%
\pgfpathmoveto{\pgfqpoint{2.074414in}{3.186937in}}%
\pgfpathcurveto{\pgfqpoint{2.087437in}{3.186937in}}{\pgfqpoint{2.099928in}{3.192111in}}{\pgfqpoint{2.109136in}{3.201319in}}%
\pgfpathcurveto{\pgfqpoint{2.118345in}{3.210528in}}{\pgfqpoint{2.123519in}{3.223019in}}{\pgfqpoint{2.123519in}{3.236042in}}%
\pgfpathcurveto{\pgfqpoint{2.123519in}{3.249064in}}{\pgfqpoint{2.118345in}{3.261555in}}{\pgfqpoint{2.109136in}{3.270764in}}%
\pgfpathcurveto{\pgfqpoint{2.099928in}{3.279972in}}{\pgfqpoint{2.087437in}{3.285146in}}{\pgfqpoint{2.074414in}{3.285146in}}%
\pgfpathcurveto{\pgfqpoint{2.061391in}{3.285146in}}{\pgfqpoint{2.048900in}{3.279972in}}{\pgfqpoint{2.039692in}{3.270764in}}%
\pgfpathcurveto{\pgfqpoint{2.030483in}{3.261555in}}{\pgfqpoint{2.025309in}{3.249064in}}{\pgfqpoint{2.025309in}{3.236042in}}%
\pgfpathcurveto{\pgfqpoint{2.025309in}{3.223019in}}{\pgfqpoint{2.030483in}{3.210528in}}{\pgfqpoint{2.039692in}{3.201319in}}%
\pgfpathcurveto{\pgfqpoint{2.048900in}{3.192111in}}{\pgfqpoint{2.061391in}{3.186937in}}{\pgfqpoint{2.074414in}{3.186937in}}%
\pgfpathlineto{\pgfqpoint{2.074414in}{3.186937in}}%
\pgfpathclose%
\pgfusepath{stroke,fill}%
\end{pgfscope}%
\begin{pgfscope}%
\pgfpathrectangle{\pgfqpoint{0.786164in}{0.768110in}}{\pgfqpoint{8.851069in}{7.081890in}}%
\pgfusepath{clip}%
\pgfsetbuttcap%
\pgfsetroundjoin%
\definecolor{currentfill}{rgb}{0.280868,0.160771,0.472899}%
\pgfsetfillcolor{currentfill}%
\pgfsetfillopacity{0.700000}%
\pgfsetlinewidth{0.501875pt}%
\definecolor{currentstroke}{rgb}{1.000000,1.000000,1.000000}%
\pgfsetstrokecolor{currentstroke}%
\pgfsetstrokeopacity{0.700000}%
\pgfsetdash{}{0pt}%
\pgfpathmoveto{\pgfqpoint{2.101814in}{3.165039in}}%
\pgfpathcurveto{\pgfqpoint{2.114837in}{3.165039in}}{\pgfqpoint{2.127328in}{3.170213in}}{\pgfqpoint{2.136536in}{3.179421in}}%
\pgfpathcurveto{\pgfqpoint{2.145745in}{3.188630in}}{\pgfqpoint{2.150919in}{3.201121in}}{\pgfqpoint{2.150919in}{3.214143in}}%
\pgfpathcurveto{\pgfqpoint{2.150919in}{3.227166in}}{\pgfqpoint{2.145745in}{3.239657in}}{\pgfqpoint{2.136536in}{3.248866in}}%
\pgfpathcurveto{\pgfqpoint{2.127328in}{3.258074in}}{\pgfqpoint{2.114837in}{3.263248in}}{\pgfqpoint{2.101814in}{3.263248in}}%
\pgfpathcurveto{\pgfqpoint{2.088791in}{3.263248in}}{\pgfqpoint{2.076300in}{3.258074in}}{\pgfqpoint{2.067092in}{3.248866in}}%
\pgfpathcurveto{\pgfqpoint{2.057883in}{3.239657in}}{\pgfqpoint{2.052709in}{3.227166in}}{\pgfqpoint{2.052709in}{3.214143in}}%
\pgfpathcurveto{\pgfqpoint{2.052709in}{3.201121in}}{\pgfqpoint{2.057883in}{3.188630in}}{\pgfqpoint{2.067092in}{3.179421in}}%
\pgfpathcurveto{\pgfqpoint{2.076300in}{3.170213in}}{\pgfqpoint{2.088791in}{3.165039in}}{\pgfqpoint{2.101814in}{3.165039in}}%
\pgfpathlineto{\pgfqpoint{2.101814in}{3.165039in}}%
\pgfpathclose%
\pgfusepath{stroke,fill}%
\end{pgfscope}%
\begin{pgfscope}%
\pgfpathrectangle{\pgfqpoint{0.786164in}{0.768110in}}{\pgfqpoint{8.851069in}{7.081890in}}%
\pgfusepath{clip}%
\pgfsetbuttcap%
\pgfsetroundjoin%
\definecolor{currentfill}{rgb}{0.278012,0.180367,0.486697}%
\pgfsetfillcolor{currentfill}%
\pgfsetfillopacity{0.700000}%
\pgfsetlinewidth{0.501875pt}%
\definecolor{currentstroke}{rgb}{1.000000,1.000000,1.000000}%
\pgfsetstrokecolor{currentstroke}%
\pgfsetstrokeopacity{0.700000}%
\pgfsetdash{}{0pt}%
\pgfpathmoveto{\pgfqpoint{2.019614in}{3.055548in}}%
\pgfpathcurveto{\pgfqpoint{2.032637in}{3.055548in}}{\pgfqpoint{2.045128in}{3.060722in}}{\pgfqpoint{2.054337in}{3.069930in}}%
\pgfpathcurveto{\pgfqpoint{2.063545in}{3.079138in}}{\pgfqpoint{2.068719in}{3.091630in}}{\pgfqpoint{2.068719in}{3.104652in}}%
\pgfpathcurveto{\pgfqpoint{2.068719in}{3.117675in}}{\pgfqpoint{2.063545in}{3.130166in}}{\pgfqpoint{2.054337in}{3.139374in}}%
\pgfpathcurveto{\pgfqpoint{2.045128in}{3.148583in}}{\pgfqpoint{2.032637in}{3.153757in}}{\pgfqpoint{2.019614in}{3.153757in}}%
\pgfpathcurveto{\pgfqpoint{2.006592in}{3.153757in}}{\pgfqpoint{1.994101in}{3.148583in}}{\pgfqpoint{1.984892in}{3.139374in}}%
\pgfpathcurveto{\pgfqpoint{1.975684in}{3.130166in}}{\pgfqpoint{1.970510in}{3.117675in}}{\pgfqpoint{1.970510in}{3.104652in}}%
\pgfpathcurveto{\pgfqpoint{1.970510in}{3.091630in}}{\pgfqpoint{1.975684in}{3.079138in}}{\pgfqpoint{1.984892in}{3.069930in}}%
\pgfpathcurveto{\pgfqpoint{1.994101in}{3.060722in}}{\pgfqpoint{2.006592in}{3.055548in}}{\pgfqpoint{2.019614in}{3.055548in}}%
\pgfpathlineto{\pgfqpoint{2.019614in}{3.055548in}}%
\pgfpathclose%
\pgfusepath{stroke,fill}%
\end{pgfscope}%
\begin{pgfscope}%
\pgfpathrectangle{\pgfqpoint{0.786164in}{0.768110in}}{\pgfqpoint{8.851069in}{7.081890in}}%
\pgfusepath{clip}%
\pgfsetbuttcap%
\pgfsetroundjoin%
\definecolor{currentfill}{rgb}{0.276194,0.190074,0.493001}%
\pgfsetfillcolor{currentfill}%
\pgfsetfillopacity{0.700000}%
\pgfsetlinewidth{0.501875pt}%
\definecolor{currentstroke}{rgb}{1.000000,1.000000,1.000000}%
\pgfsetstrokecolor{currentstroke}%
\pgfsetstrokeopacity{0.700000}%
\pgfsetdash{}{0pt}%
\pgfpathmoveto{\pgfqpoint{2.193147in}{3.165039in}}%
\pgfpathcurveto{\pgfqpoint{2.206170in}{3.165039in}}{\pgfqpoint{2.218661in}{3.170213in}}{\pgfqpoint{2.227869in}{3.179421in}}%
\pgfpathcurveto{\pgfqpoint{2.237077in}{3.188630in}}{\pgfqpoint{2.242251in}{3.201121in}}{\pgfqpoint{2.242251in}{3.214143in}}%
\pgfpathcurveto{\pgfqpoint{2.242251in}{3.227166in}}{\pgfqpoint{2.237077in}{3.239657in}}{\pgfqpoint{2.227869in}{3.248866in}}%
\pgfpathcurveto{\pgfqpoint{2.218661in}{3.258074in}}{\pgfqpoint{2.206170in}{3.263248in}}{\pgfqpoint{2.193147in}{3.263248in}}%
\pgfpathcurveto{\pgfqpoint{2.180124in}{3.263248in}}{\pgfqpoint{2.167633in}{3.258074in}}{\pgfqpoint{2.158425in}{3.248866in}}%
\pgfpathcurveto{\pgfqpoint{2.149216in}{3.239657in}}{\pgfqpoint{2.144042in}{3.227166in}}{\pgfqpoint{2.144042in}{3.214143in}}%
\pgfpathcurveto{\pgfqpoint{2.144042in}{3.201121in}}{\pgfqpoint{2.149216in}{3.188630in}}{\pgfqpoint{2.158425in}{3.179421in}}%
\pgfpathcurveto{\pgfqpoint{2.167633in}{3.170213in}}{\pgfqpoint{2.180124in}{3.165039in}}{\pgfqpoint{2.193147in}{3.165039in}}%
\pgfpathlineto{\pgfqpoint{2.193147in}{3.165039in}}%
\pgfpathclose%
\pgfusepath{stroke,fill}%
\end{pgfscope}%
\begin{pgfscope}%
\pgfpathrectangle{\pgfqpoint{0.786164in}{0.768110in}}{\pgfqpoint{8.851069in}{7.081890in}}%
\pgfusepath{clip}%
\pgfsetbuttcap%
\pgfsetroundjoin%
\definecolor{currentfill}{rgb}{0.271828,0.209303,0.504434}%
\pgfsetfillcolor{currentfill}%
\pgfsetfillopacity{0.700000}%
\pgfsetlinewidth{0.501875pt}%
\definecolor{currentstroke}{rgb}{1.000000,1.000000,1.000000}%
\pgfsetstrokecolor{currentstroke}%
\pgfsetstrokeopacity{0.700000}%
\pgfsetdash{}{0pt}%
\pgfpathmoveto{\pgfqpoint{2.193147in}{3.121242in}}%
\pgfpathcurveto{\pgfqpoint{2.206170in}{3.121242in}}{\pgfqpoint{2.218661in}{3.126416in}}{\pgfqpoint{2.227869in}{3.135625in}}%
\pgfpathcurveto{\pgfqpoint{2.237077in}{3.144833in}}{\pgfqpoint{2.242251in}{3.157324in}}{\pgfqpoint{2.242251in}{3.170347in}}%
\pgfpathcurveto{\pgfqpoint{2.242251in}{3.183370in}}{\pgfqpoint{2.237077in}{3.195861in}}{\pgfqpoint{2.227869in}{3.205069in}}%
\pgfpathcurveto{\pgfqpoint{2.218661in}{3.214278in}}{\pgfqpoint{2.206170in}{3.219452in}}{\pgfqpoint{2.193147in}{3.219452in}}%
\pgfpathcurveto{\pgfqpoint{2.180124in}{3.219452in}}{\pgfqpoint{2.167633in}{3.214278in}}{\pgfqpoint{2.158425in}{3.205069in}}%
\pgfpathcurveto{\pgfqpoint{2.149216in}{3.195861in}}{\pgfqpoint{2.144042in}{3.183370in}}{\pgfqpoint{2.144042in}{3.170347in}}%
\pgfpathcurveto{\pgfqpoint{2.144042in}{3.157324in}}{\pgfqpoint{2.149216in}{3.144833in}}{\pgfqpoint{2.158425in}{3.135625in}}%
\pgfpathcurveto{\pgfqpoint{2.167633in}{3.126416in}}{\pgfqpoint{2.180124in}{3.121242in}}{\pgfqpoint{2.193147in}{3.121242in}}%
\pgfpathlineto{\pgfqpoint{2.193147in}{3.121242in}}%
\pgfpathclose%
\pgfusepath{stroke,fill}%
\end{pgfscope}%
\begin{pgfscope}%
\pgfpathrectangle{\pgfqpoint{0.786164in}{0.768110in}}{\pgfqpoint{8.851069in}{7.081890in}}%
\pgfusepath{clip}%
\pgfsetbuttcap%
\pgfsetroundjoin%
\definecolor{currentfill}{rgb}{0.267968,0.223549,0.512008}%
\pgfsetfillcolor{currentfill}%
\pgfsetfillopacity{0.700000}%
\pgfsetlinewidth{0.501875pt}%
\definecolor{currentstroke}{rgb}{1.000000,1.000000,1.000000}%
\pgfsetstrokecolor{currentstroke}%
\pgfsetstrokeopacity{0.700000}%
\pgfsetdash{}{0pt}%
\pgfpathmoveto{\pgfqpoint{2.101814in}{3.077446in}}%
\pgfpathcurveto{\pgfqpoint{2.114837in}{3.077446in}}{\pgfqpoint{2.127328in}{3.082620in}}{\pgfqpoint{2.136536in}{3.091828in}}%
\pgfpathcurveto{\pgfqpoint{2.145745in}{3.101037in}}{\pgfqpoint{2.150919in}{3.113528in}}{\pgfqpoint{2.150919in}{3.126550in}}%
\pgfpathcurveto{\pgfqpoint{2.150919in}{3.139573in}}{\pgfqpoint{2.145745in}{3.152064in}}{\pgfqpoint{2.136536in}{3.161273in}}%
\pgfpathcurveto{\pgfqpoint{2.127328in}{3.170481in}}{\pgfqpoint{2.114837in}{3.175655in}}{\pgfqpoint{2.101814in}{3.175655in}}%
\pgfpathcurveto{\pgfqpoint{2.088791in}{3.175655in}}{\pgfqpoint{2.076300in}{3.170481in}}{\pgfqpoint{2.067092in}{3.161273in}}%
\pgfpathcurveto{\pgfqpoint{2.057883in}{3.152064in}}{\pgfqpoint{2.052709in}{3.139573in}}{\pgfqpoint{2.052709in}{3.126550in}}%
\pgfpathcurveto{\pgfqpoint{2.052709in}{3.113528in}}{\pgfqpoint{2.057883in}{3.101037in}}{\pgfqpoint{2.067092in}{3.091828in}}%
\pgfpathcurveto{\pgfqpoint{2.076300in}{3.082620in}}{\pgfqpoint{2.088791in}{3.077446in}}{\pgfqpoint{2.101814in}{3.077446in}}%
\pgfpathlineto{\pgfqpoint{2.101814in}{3.077446in}}%
\pgfpathclose%
\pgfusepath{stroke,fill}%
\end{pgfscope}%
\begin{pgfscope}%
\pgfpathrectangle{\pgfqpoint{0.786164in}{0.768110in}}{\pgfqpoint{8.851069in}{7.081890in}}%
\pgfusepath{clip}%
\pgfsetbuttcap%
\pgfsetroundjoin%
\definecolor{currentfill}{rgb}{0.265145,0.232956,0.516599}%
\pgfsetfillcolor{currentfill}%
\pgfsetfillopacity{0.700000}%
\pgfsetlinewidth{0.501875pt}%
\definecolor{currentstroke}{rgb}{1.000000,1.000000,1.000000}%
\pgfsetstrokecolor{currentstroke}%
\pgfsetstrokeopacity{0.700000}%
\pgfsetdash{}{0pt}%
\pgfpathmoveto{\pgfqpoint{2.120081in}{3.033649in}}%
\pgfpathcurveto{\pgfqpoint{2.133103in}{3.033649in}}{\pgfqpoint{2.145594in}{3.038823in}}{\pgfqpoint{2.154803in}{3.048032in}}%
\pgfpathcurveto{\pgfqpoint{2.164011in}{3.057240in}}{\pgfqpoint{2.169185in}{3.069731in}}{\pgfqpoint{2.169185in}{3.082754in}}%
\pgfpathcurveto{\pgfqpoint{2.169185in}{3.095777in}}{\pgfqpoint{2.164011in}{3.108268in}}{\pgfqpoint{2.154803in}{3.117476in}}%
\pgfpathcurveto{\pgfqpoint{2.145594in}{3.126685in}}{\pgfqpoint{2.133103in}{3.131859in}}{\pgfqpoint{2.120081in}{3.131859in}}%
\pgfpathcurveto{\pgfqpoint{2.107058in}{3.131859in}}{\pgfqpoint{2.094567in}{3.126685in}}{\pgfqpoint{2.085358in}{3.117476in}}%
\pgfpathcurveto{\pgfqpoint{2.076150in}{3.108268in}}{\pgfqpoint{2.070976in}{3.095777in}}{\pgfqpoint{2.070976in}{3.082754in}}%
\pgfpathcurveto{\pgfqpoint{2.070976in}{3.069731in}}{\pgfqpoint{2.076150in}{3.057240in}}{\pgfqpoint{2.085358in}{3.048032in}}%
\pgfpathcurveto{\pgfqpoint{2.094567in}{3.038823in}}{\pgfqpoint{2.107058in}{3.033649in}}{\pgfqpoint{2.120081in}{3.033649in}}%
\pgfpathlineto{\pgfqpoint{2.120081in}{3.033649in}}%
\pgfpathclose%
\pgfusepath{stroke,fill}%
\end{pgfscope}%
\begin{pgfscope}%
\pgfpathrectangle{\pgfqpoint{0.786164in}{0.768110in}}{\pgfqpoint{8.851069in}{7.081890in}}%
\pgfusepath{clip}%
\pgfsetbuttcap%
\pgfsetroundjoin%
\definecolor{currentfill}{rgb}{0.265145,0.232956,0.516599}%
\pgfsetfillcolor{currentfill}%
\pgfsetfillopacity{0.700000}%
\pgfsetlinewidth{0.501875pt}%
\definecolor{currentstroke}{rgb}{1.000000,1.000000,1.000000}%
\pgfsetstrokecolor{currentstroke}%
\pgfsetstrokeopacity{0.700000}%
\pgfsetdash{}{0pt}%
\pgfpathmoveto{\pgfqpoint{2.019614in}{3.011751in}}%
\pgfpathcurveto{\pgfqpoint{2.032637in}{3.011751in}}{\pgfqpoint{2.045128in}{3.016925in}}{\pgfqpoint{2.054337in}{3.026134in}}%
\pgfpathcurveto{\pgfqpoint{2.063545in}{3.035342in}}{\pgfqpoint{2.068719in}{3.047833in}}{\pgfqpoint{2.068719in}{3.060856in}}%
\pgfpathcurveto{\pgfqpoint{2.068719in}{3.073878in}}{\pgfqpoint{2.063545in}{3.086370in}}{\pgfqpoint{2.054337in}{3.095578in}}%
\pgfpathcurveto{\pgfqpoint{2.045128in}{3.104786in}}{\pgfqpoint{2.032637in}{3.109960in}}{\pgfqpoint{2.019614in}{3.109960in}}%
\pgfpathcurveto{\pgfqpoint{2.006592in}{3.109960in}}{\pgfqpoint{1.994101in}{3.104786in}}{\pgfqpoint{1.984892in}{3.095578in}}%
\pgfpathcurveto{\pgfqpoint{1.975684in}{3.086370in}}{\pgfqpoint{1.970510in}{3.073878in}}{\pgfqpoint{1.970510in}{3.060856in}}%
\pgfpathcurveto{\pgfqpoint{1.970510in}{3.047833in}}{\pgfqpoint{1.975684in}{3.035342in}}{\pgfqpoint{1.984892in}{3.026134in}}%
\pgfpathcurveto{\pgfqpoint{1.994101in}{3.016925in}}{\pgfqpoint{2.006592in}{3.011751in}}{\pgfqpoint{2.019614in}{3.011751in}}%
\pgfpathlineto{\pgfqpoint{2.019614in}{3.011751in}}%
\pgfpathclose%
\pgfusepath{stroke,fill}%
\end{pgfscope}%
\begin{pgfscope}%
\pgfpathrectangle{\pgfqpoint{0.786164in}{0.768110in}}{\pgfqpoint{8.851069in}{7.081890in}}%
\pgfusepath{clip}%
\pgfsetbuttcap%
\pgfsetroundjoin%
\definecolor{currentfill}{rgb}{0.263663,0.237631,0.518762}%
\pgfsetfillcolor{currentfill}%
\pgfsetfillopacity{0.700000}%
\pgfsetlinewidth{0.501875pt}%
\definecolor{currentstroke}{rgb}{1.000000,1.000000,1.000000}%
\pgfsetstrokecolor{currentstroke}%
\pgfsetstrokeopacity{0.700000}%
\pgfsetdash{}{0pt}%
\pgfpathmoveto{\pgfqpoint{2.037881in}{3.033649in}}%
\pgfpathcurveto{\pgfqpoint{2.050904in}{3.033649in}}{\pgfqpoint{2.063395in}{3.038823in}}{\pgfqpoint{2.072603in}{3.048032in}}%
\pgfpathcurveto{\pgfqpoint{2.081812in}{3.057240in}}{\pgfqpoint{2.086986in}{3.069731in}}{\pgfqpoint{2.086986in}{3.082754in}}%
\pgfpathcurveto{\pgfqpoint{2.086986in}{3.095777in}}{\pgfqpoint{2.081812in}{3.108268in}}{\pgfqpoint{2.072603in}{3.117476in}}%
\pgfpathcurveto{\pgfqpoint{2.063395in}{3.126685in}}{\pgfqpoint{2.050904in}{3.131859in}}{\pgfqpoint{2.037881in}{3.131859in}}%
\pgfpathcurveto{\pgfqpoint{2.024858in}{3.131859in}}{\pgfqpoint{2.012367in}{3.126685in}}{\pgfqpoint{2.003159in}{3.117476in}}%
\pgfpathcurveto{\pgfqpoint{1.993950in}{3.108268in}}{\pgfqpoint{1.988776in}{3.095777in}}{\pgfqpoint{1.988776in}{3.082754in}}%
\pgfpathcurveto{\pgfqpoint{1.988776in}{3.069731in}}{\pgfqpoint{1.993950in}{3.057240in}}{\pgfqpoint{2.003159in}{3.048032in}}%
\pgfpathcurveto{\pgfqpoint{2.012367in}{3.038823in}}{\pgfqpoint{2.024858in}{3.033649in}}{\pgfqpoint{2.037881in}{3.033649in}}%
\pgfpathlineto{\pgfqpoint{2.037881in}{3.033649in}}%
\pgfpathclose%
\pgfusepath{stroke,fill}%
\end{pgfscope}%
\begin{pgfscope}%
\pgfpathrectangle{\pgfqpoint{0.786164in}{0.768110in}}{\pgfqpoint{8.851069in}{7.081890in}}%
\pgfusepath{clip}%
\pgfsetbuttcap%
\pgfsetroundjoin%
\definecolor{currentfill}{rgb}{0.266580,0.228262,0.514349}%
\pgfsetfillcolor{currentfill}%
\pgfsetfillopacity{0.700000}%
\pgfsetlinewidth{0.501875pt}%
\definecolor{currentstroke}{rgb}{1.000000,1.000000,1.000000}%
\pgfsetstrokecolor{currentstroke}%
\pgfsetstrokeopacity{0.700000}%
\pgfsetdash{}{0pt}%
\pgfpathmoveto{\pgfqpoint{2.156614in}{3.077446in}}%
\pgfpathcurveto{\pgfqpoint{2.169636in}{3.077446in}}{\pgfqpoint{2.182127in}{3.082620in}}{\pgfqpoint{2.191336in}{3.091828in}}%
\pgfpathcurveto{\pgfqpoint{2.200544in}{3.101037in}}{\pgfqpoint{2.205718in}{3.113528in}}{\pgfqpoint{2.205718in}{3.126550in}}%
\pgfpathcurveto{\pgfqpoint{2.205718in}{3.139573in}}{\pgfqpoint{2.200544in}{3.152064in}}{\pgfqpoint{2.191336in}{3.161273in}}%
\pgfpathcurveto{\pgfqpoint{2.182127in}{3.170481in}}{\pgfqpoint{2.169636in}{3.175655in}}{\pgfqpoint{2.156614in}{3.175655in}}%
\pgfpathcurveto{\pgfqpoint{2.143591in}{3.175655in}}{\pgfqpoint{2.131100in}{3.170481in}}{\pgfqpoint{2.121891in}{3.161273in}}%
\pgfpathcurveto{\pgfqpoint{2.112683in}{3.152064in}}{\pgfqpoint{2.107509in}{3.139573in}}{\pgfqpoint{2.107509in}{3.126550in}}%
\pgfpathcurveto{\pgfqpoint{2.107509in}{3.113528in}}{\pgfqpoint{2.112683in}{3.101037in}}{\pgfqpoint{2.121891in}{3.091828in}}%
\pgfpathcurveto{\pgfqpoint{2.131100in}{3.082620in}}{\pgfqpoint{2.143591in}{3.077446in}}{\pgfqpoint{2.156614in}{3.077446in}}%
\pgfpathlineto{\pgfqpoint{2.156614in}{3.077446in}}%
\pgfpathclose%
\pgfusepath{stroke,fill}%
\end{pgfscope}%
\begin{pgfscope}%
\pgfpathrectangle{\pgfqpoint{0.786164in}{0.768110in}}{\pgfqpoint{8.851069in}{7.081890in}}%
\pgfusepath{clip}%
\pgfsetbuttcap%
\pgfsetroundjoin%
\definecolor{currentfill}{rgb}{0.267968,0.223549,0.512008}%
\pgfsetfillcolor{currentfill}%
\pgfsetfillopacity{0.700000}%
\pgfsetlinewidth{0.501875pt}%
\definecolor{currentstroke}{rgb}{1.000000,1.000000,1.000000}%
\pgfsetstrokecolor{currentstroke}%
\pgfsetstrokeopacity{0.700000}%
\pgfsetdash{}{0pt}%
\pgfpathmoveto{\pgfqpoint{2.257080in}{3.143141in}}%
\pgfpathcurveto{\pgfqpoint{2.270103in}{3.143141in}}{\pgfqpoint{2.282594in}{3.148314in}}{\pgfqpoint{2.291802in}{3.157523in}}%
\pgfpathcurveto{\pgfqpoint{2.301010in}{3.166731in}}{\pgfqpoint{2.306184in}{3.179222in}}{\pgfqpoint{2.306184in}{3.192245in}}%
\pgfpathcurveto{\pgfqpoint{2.306184in}{3.205268in}}{\pgfqpoint{2.301010in}{3.217759in}}{\pgfqpoint{2.291802in}{3.226967in}}%
\pgfpathcurveto{\pgfqpoint{2.282594in}{3.236176in}}{\pgfqpoint{2.270103in}{3.241350in}}{\pgfqpoint{2.257080in}{3.241350in}}%
\pgfpathcurveto{\pgfqpoint{2.244057in}{3.241350in}}{\pgfqpoint{2.231566in}{3.236176in}}{\pgfqpoint{2.222358in}{3.226967in}}%
\pgfpathcurveto{\pgfqpoint{2.213149in}{3.217759in}}{\pgfqpoint{2.207975in}{3.205268in}}{\pgfqpoint{2.207975in}{3.192245in}}%
\pgfpathcurveto{\pgfqpoint{2.207975in}{3.179222in}}{\pgfqpoint{2.213149in}{3.166731in}}{\pgfqpoint{2.222358in}{3.157523in}}%
\pgfpathcurveto{\pgfqpoint{2.231566in}{3.148314in}}{\pgfqpoint{2.244057in}{3.143141in}}{\pgfqpoint{2.257080in}{3.143141in}}%
\pgfpathlineto{\pgfqpoint{2.257080in}{3.143141in}}%
\pgfpathclose%
\pgfusepath{stroke,fill}%
\end{pgfscope}%
\begin{pgfscope}%
\pgfpathrectangle{\pgfqpoint{0.786164in}{0.768110in}}{\pgfqpoint{8.851069in}{7.081890in}}%
\pgfusepath{clip}%
\pgfsetbuttcap%
\pgfsetroundjoin%
\definecolor{currentfill}{rgb}{0.243113,0.292092,0.538516}%
\pgfsetfillcolor{currentfill}%
\pgfsetfillopacity{0.700000}%
\pgfsetlinewidth{0.501875pt}%
\definecolor{currentstroke}{rgb}{1.000000,1.000000,1.000000}%
\pgfsetstrokecolor{currentstroke}%
\pgfsetstrokeopacity{0.700000}%
\pgfsetdash{}{0pt}%
\pgfpathmoveto{\pgfqpoint{2.375813in}{3.165039in}}%
\pgfpathcurveto{\pgfqpoint{2.388835in}{3.165039in}}{\pgfqpoint{2.401326in}{3.170213in}}{\pgfqpoint{2.410535in}{3.179421in}}%
\pgfpathcurveto{\pgfqpoint{2.419743in}{3.188630in}}{\pgfqpoint{2.424917in}{3.201121in}}{\pgfqpoint{2.424917in}{3.214143in}}%
\pgfpathcurveto{\pgfqpoint{2.424917in}{3.227166in}}{\pgfqpoint{2.419743in}{3.239657in}}{\pgfqpoint{2.410535in}{3.248866in}}%
\pgfpathcurveto{\pgfqpoint{2.401326in}{3.258074in}}{\pgfqpoint{2.388835in}{3.263248in}}{\pgfqpoint{2.375813in}{3.263248in}}%
\pgfpathcurveto{\pgfqpoint{2.362790in}{3.263248in}}{\pgfqpoint{2.350299in}{3.258074in}}{\pgfqpoint{2.341090in}{3.248866in}}%
\pgfpathcurveto{\pgfqpoint{2.331882in}{3.239657in}}{\pgfqpoint{2.326708in}{3.227166in}}{\pgfqpoint{2.326708in}{3.214143in}}%
\pgfpathcurveto{\pgfqpoint{2.326708in}{3.201121in}}{\pgfqpoint{2.331882in}{3.188630in}}{\pgfqpoint{2.341090in}{3.179421in}}%
\pgfpathcurveto{\pgfqpoint{2.350299in}{3.170213in}}{\pgfqpoint{2.362790in}{3.165039in}}{\pgfqpoint{2.375813in}{3.165039in}}%
\pgfpathlineto{\pgfqpoint{2.375813in}{3.165039in}}%
\pgfpathclose%
\pgfusepath{stroke,fill}%
\end{pgfscope}%
\begin{pgfscope}%
\pgfpathrectangle{\pgfqpoint{0.786164in}{0.768110in}}{\pgfqpoint{8.851069in}{7.081890in}}%
\pgfusepath{clip}%
\pgfsetbuttcap%
\pgfsetroundjoin%
\definecolor{currentfill}{rgb}{0.239346,0.300855,0.540844}%
\pgfsetfillcolor{currentfill}%
\pgfsetfillopacity{0.700000}%
\pgfsetlinewidth{0.501875pt}%
\definecolor{currentstroke}{rgb}{1.000000,1.000000,1.000000}%
\pgfsetstrokecolor{currentstroke}%
\pgfsetstrokeopacity{0.700000}%
\pgfsetdash{}{0pt}%
\pgfpathmoveto{\pgfqpoint{2.284480in}{3.143141in}}%
\pgfpathcurveto{\pgfqpoint{2.297502in}{3.143141in}}{\pgfqpoint{2.309993in}{3.148314in}}{\pgfqpoint{2.319202in}{3.157523in}}%
\pgfpathcurveto{\pgfqpoint{2.328410in}{3.166731in}}{\pgfqpoint{2.333584in}{3.179222in}}{\pgfqpoint{2.333584in}{3.192245in}}%
\pgfpathcurveto{\pgfqpoint{2.333584in}{3.205268in}}{\pgfqpoint{2.328410in}{3.217759in}}{\pgfqpoint{2.319202in}{3.226967in}}%
\pgfpathcurveto{\pgfqpoint{2.309993in}{3.236176in}}{\pgfqpoint{2.297502in}{3.241350in}}{\pgfqpoint{2.284480in}{3.241350in}}%
\pgfpathcurveto{\pgfqpoint{2.271457in}{3.241350in}}{\pgfqpoint{2.258966in}{3.236176in}}{\pgfqpoint{2.249757in}{3.226967in}}%
\pgfpathcurveto{\pgfqpoint{2.240549in}{3.217759in}}{\pgfqpoint{2.235375in}{3.205268in}}{\pgfqpoint{2.235375in}{3.192245in}}%
\pgfpathcurveto{\pgfqpoint{2.235375in}{3.179222in}}{\pgfqpoint{2.240549in}{3.166731in}}{\pgfqpoint{2.249757in}{3.157523in}}%
\pgfpathcurveto{\pgfqpoint{2.258966in}{3.148314in}}{\pgfqpoint{2.271457in}{3.143141in}}{\pgfqpoint{2.284480in}{3.143141in}}%
\pgfpathlineto{\pgfqpoint{2.284480in}{3.143141in}}%
\pgfpathclose%
\pgfusepath{stroke,fill}%
\end{pgfscope}%
\begin{pgfscope}%
\pgfpathrectangle{\pgfqpoint{0.786164in}{0.768110in}}{\pgfqpoint{8.851069in}{7.081890in}}%
\pgfusepath{clip}%
\pgfsetbuttcap%
\pgfsetroundjoin%
\definecolor{currentfill}{rgb}{0.233603,0.313828,0.543914}%
\pgfsetfillcolor{currentfill}%
\pgfsetfillopacity{0.700000}%
\pgfsetlinewidth{0.501875pt}%
\definecolor{currentstroke}{rgb}{1.000000,1.000000,1.000000}%
\pgfsetstrokecolor{currentstroke}%
\pgfsetstrokeopacity{0.700000}%
\pgfsetdash{}{0pt}%
\pgfpathmoveto{\pgfqpoint{2.211413in}{3.033649in}}%
\pgfpathcurveto{\pgfqpoint{2.224436in}{3.033649in}}{\pgfqpoint{2.236927in}{3.038823in}}{\pgfqpoint{2.246136in}{3.048032in}}%
\pgfpathcurveto{\pgfqpoint{2.255344in}{3.057240in}}{\pgfqpoint{2.260518in}{3.069731in}}{\pgfqpoint{2.260518in}{3.082754in}}%
\pgfpathcurveto{\pgfqpoint{2.260518in}{3.095777in}}{\pgfqpoint{2.255344in}{3.108268in}}{\pgfqpoint{2.246136in}{3.117476in}}%
\pgfpathcurveto{\pgfqpoint{2.236927in}{3.126685in}}{\pgfqpoint{2.224436in}{3.131859in}}{\pgfqpoint{2.211413in}{3.131859in}}%
\pgfpathcurveto{\pgfqpoint{2.198391in}{3.131859in}}{\pgfqpoint{2.185900in}{3.126685in}}{\pgfqpoint{2.176691in}{3.117476in}}%
\pgfpathcurveto{\pgfqpoint{2.167483in}{3.108268in}}{\pgfqpoint{2.162309in}{3.095777in}}{\pgfqpoint{2.162309in}{3.082754in}}%
\pgfpathcurveto{\pgfqpoint{2.162309in}{3.069731in}}{\pgfqpoint{2.167483in}{3.057240in}}{\pgfqpoint{2.176691in}{3.048032in}}%
\pgfpathcurveto{\pgfqpoint{2.185900in}{3.038823in}}{\pgfqpoint{2.198391in}{3.033649in}}{\pgfqpoint{2.211413in}{3.033649in}}%
\pgfpathlineto{\pgfqpoint{2.211413in}{3.033649in}}%
\pgfpathclose%
\pgfusepath{stroke,fill}%
\end{pgfscope}%
\begin{pgfscope}%
\pgfpathrectangle{\pgfqpoint{0.786164in}{0.768110in}}{\pgfqpoint{8.851069in}{7.081890in}}%
\pgfusepath{clip}%
\pgfsetbuttcap%
\pgfsetroundjoin%
\definecolor{currentfill}{rgb}{0.237441,0.305202,0.541921}%
\pgfsetfillcolor{currentfill}%
\pgfsetfillopacity{0.700000}%
\pgfsetlinewidth{0.501875pt}%
\definecolor{currentstroke}{rgb}{1.000000,1.000000,1.000000}%
\pgfsetstrokecolor{currentstroke}%
\pgfsetstrokeopacity{0.700000}%
\pgfsetdash{}{0pt}%
\pgfpathmoveto{\pgfqpoint{2.439746in}{3.252632in}}%
\pgfpathcurveto{\pgfqpoint{2.452768in}{3.252632in}}{\pgfqpoint{2.465259in}{3.257806in}}{\pgfqpoint{2.474468in}{3.267014in}}%
\pgfpathcurveto{\pgfqpoint{2.483676in}{3.276223in}}{\pgfqpoint{2.488850in}{3.288714in}}{\pgfqpoint{2.488850in}{3.301736in}}%
\pgfpathcurveto{\pgfqpoint{2.488850in}{3.314759in}}{\pgfqpoint{2.483676in}{3.327250in}}{\pgfqpoint{2.474468in}{3.336459in}}%
\pgfpathcurveto{\pgfqpoint{2.465259in}{3.345667in}}{\pgfqpoint{2.452768in}{3.350841in}}{\pgfqpoint{2.439746in}{3.350841in}}%
\pgfpathcurveto{\pgfqpoint{2.426723in}{3.350841in}}{\pgfqpoint{2.414232in}{3.345667in}}{\pgfqpoint{2.405023in}{3.336459in}}%
\pgfpathcurveto{\pgfqpoint{2.395815in}{3.327250in}}{\pgfqpoint{2.390641in}{3.314759in}}{\pgfqpoint{2.390641in}{3.301736in}}%
\pgfpathcurveto{\pgfqpoint{2.390641in}{3.288714in}}{\pgfqpoint{2.395815in}{3.276223in}}{\pgfqpoint{2.405023in}{3.267014in}}%
\pgfpathcurveto{\pgfqpoint{2.414232in}{3.257806in}}{\pgfqpoint{2.426723in}{3.252632in}}{\pgfqpoint{2.439746in}{3.252632in}}%
\pgfpathlineto{\pgfqpoint{2.439746in}{3.252632in}}%
\pgfpathclose%
\pgfusepath{stroke,fill}%
\end{pgfscope}%
\begin{pgfscope}%
\pgfpathrectangle{\pgfqpoint{0.786164in}{0.768110in}}{\pgfqpoint{8.851069in}{7.081890in}}%
\pgfusepath{clip}%
\pgfsetbuttcap%
\pgfsetroundjoin%
\definecolor{currentfill}{rgb}{0.227802,0.326594,0.546532}%
\pgfsetfillcolor{currentfill}%
\pgfsetfillopacity{0.700000}%
\pgfsetlinewidth{0.501875pt}%
\definecolor{currentstroke}{rgb}{1.000000,1.000000,1.000000}%
\pgfsetstrokecolor{currentstroke}%
\pgfsetstrokeopacity{0.700000}%
\pgfsetdash{}{0pt}%
\pgfpathmoveto{\pgfqpoint{2.311880in}{3.077446in}}%
\pgfpathcurveto{\pgfqpoint{2.324902in}{3.077446in}}{\pgfqpoint{2.337393in}{3.082620in}}{\pgfqpoint{2.346602in}{3.091828in}}%
\pgfpathcurveto{\pgfqpoint{2.355810in}{3.101037in}}{\pgfqpoint{2.360984in}{3.113528in}}{\pgfqpoint{2.360984in}{3.126550in}}%
\pgfpathcurveto{\pgfqpoint{2.360984in}{3.139573in}}{\pgfqpoint{2.355810in}{3.152064in}}{\pgfqpoint{2.346602in}{3.161273in}}%
\pgfpathcurveto{\pgfqpoint{2.337393in}{3.170481in}}{\pgfqpoint{2.324902in}{3.175655in}}{\pgfqpoint{2.311880in}{3.175655in}}%
\pgfpathcurveto{\pgfqpoint{2.298857in}{3.175655in}}{\pgfqpoint{2.286366in}{3.170481in}}{\pgfqpoint{2.277157in}{3.161273in}}%
\pgfpathcurveto{\pgfqpoint{2.267949in}{3.152064in}}{\pgfqpoint{2.262775in}{3.139573in}}{\pgfqpoint{2.262775in}{3.126550in}}%
\pgfpathcurveto{\pgfqpoint{2.262775in}{3.113528in}}{\pgfqpoint{2.267949in}{3.101037in}}{\pgfqpoint{2.277157in}{3.091828in}}%
\pgfpathcurveto{\pgfqpoint{2.286366in}{3.082620in}}{\pgfqpoint{2.298857in}{3.077446in}}{\pgfqpoint{2.311880in}{3.077446in}}%
\pgfpathlineto{\pgfqpoint{2.311880in}{3.077446in}}%
\pgfpathclose%
\pgfusepath{stroke,fill}%
\end{pgfscope}%
\begin{pgfscope}%
\pgfpathrectangle{\pgfqpoint{0.786164in}{0.768110in}}{\pgfqpoint{8.851069in}{7.081890in}}%
\pgfusepath{clip}%
\pgfsetbuttcap%
\pgfsetroundjoin%
\definecolor{currentfill}{rgb}{0.210503,0.363727,0.552206}%
\pgfsetfillcolor{currentfill}%
\pgfsetfillopacity{0.700000}%
\pgfsetlinewidth{0.501875pt}%
\definecolor{currentstroke}{rgb}{1.000000,1.000000,1.000000}%
\pgfsetstrokecolor{currentstroke}%
\pgfsetstrokeopacity{0.700000}%
\pgfsetdash{}{0pt}%
\pgfpathmoveto{\pgfqpoint{1.983081in}{2.727074in}}%
\pgfpathcurveto{\pgfqpoint{1.996104in}{2.727074in}}{\pgfqpoint{2.008595in}{2.732248in}}{\pgfqpoint{2.017803in}{2.741456in}}%
\pgfpathcurveto{\pgfqpoint{2.027012in}{2.750665in}}{\pgfqpoint{2.032186in}{2.763156in}}{\pgfqpoint{2.032186in}{2.776179in}}%
\pgfpathcurveto{\pgfqpoint{2.032186in}{2.789201in}}{\pgfqpoint{2.027012in}{2.801692in}}{\pgfqpoint{2.017803in}{2.810901in}}%
\pgfpathcurveto{\pgfqpoint{2.008595in}{2.820109in}}{\pgfqpoint{1.996104in}{2.825283in}}{\pgfqpoint{1.983081in}{2.825283in}}%
\pgfpathcurveto{\pgfqpoint{1.970059in}{2.825283in}}{\pgfqpoint{1.957567in}{2.820109in}}{\pgfqpoint{1.948359in}{2.810901in}}%
\pgfpathcurveto{\pgfqpoint{1.939151in}{2.801692in}}{\pgfqpoint{1.933977in}{2.789201in}}{\pgfqpoint{1.933977in}{2.776179in}}%
\pgfpathcurveto{\pgfqpoint{1.933977in}{2.763156in}}{\pgfqpoint{1.939151in}{2.750665in}}{\pgfqpoint{1.948359in}{2.741456in}}%
\pgfpathcurveto{\pgfqpoint{1.957567in}{2.732248in}}{\pgfqpoint{1.970059in}{2.727074in}}{\pgfqpoint{1.983081in}{2.727074in}}%
\pgfpathlineto{\pgfqpoint{1.983081in}{2.727074in}}%
\pgfpathclose%
\pgfusepath{stroke,fill}%
\end{pgfscope}%
\begin{pgfscope}%
\pgfpathrectangle{\pgfqpoint{0.786164in}{0.768110in}}{\pgfqpoint{8.851069in}{7.081890in}}%
\pgfusepath{clip}%
\pgfsetbuttcap%
\pgfsetroundjoin%
\definecolor{currentfill}{rgb}{0.208623,0.367752,0.552675}%
\pgfsetfillcolor{currentfill}%
\pgfsetfillopacity{0.700000}%
\pgfsetlinewidth{0.501875pt}%
\definecolor{currentstroke}{rgb}{1.000000,1.000000,1.000000}%
\pgfsetstrokecolor{currentstroke}%
\pgfsetstrokeopacity{0.700000}%
\pgfsetdash{}{0pt}%
\pgfpathmoveto{\pgfqpoint{2.037881in}{2.924158in}}%
\pgfpathcurveto{\pgfqpoint{2.050904in}{2.924158in}}{\pgfqpoint{2.063395in}{2.929332in}}{\pgfqpoint{2.072603in}{2.938541in}}%
\pgfpathcurveto{\pgfqpoint{2.081812in}{2.947749in}}{\pgfqpoint{2.086986in}{2.960240in}}{\pgfqpoint{2.086986in}{2.973263in}}%
\pgfpathcurveto{\pgfqpoint{2.086986in}{2.986286in}}{\pgfqpoint{2.081812in}{2.998777in}}{\pgfqpoint{2.072603in}{3.007985in}}%
\pgfpathcurveto{\pgfqpoint{2.063395in}{3.017193in}}{\pgfqpoint{2.050904in}{3.022367in}}{\pgfqpoint{2.037881in}{3.022367in}}%
\pgfpathcurveto{\pgfqpoint{2.024858in}{3.022367in}}{\pgfqpoint{2.012367in}{3.017193in}}{\pgfqpoint{2.003159in}{3.007985in}}%
\pgfpathcurveto{\pgfqpoint{1.993950in}{2.998777in}}{\pgfqpoint{1.988776in}{2.986286in}}{\pgfqpoint{1.988776in}{2.973263in}}%
\pgfpathcurveto{\pgfqpoint{1.988776in}{2.960240in}}{\pgfqpoint{1.993950in}{2.947749in}}{\pgfqpoint{2.003159in}{2.938541in}}%
\pgfpathcurveto{\pgfqpoint{2.012367in}{2.929332in}}{\pgfqpoint{2.024858in}{2.924158in}}{\pgfqpoint{2.037881in}{2.924158in}}%
\pgfpathlineto{\pgfqpoint{2.037881in}{2.924158in}}%
\pgfpathclose%
\pgfusepath{stroke,fill}%
\end{pgfscope}%
\begin{pgfscope}%
\pgfpathrectangle{\pgfqpoint{0.786164in}{0.768110in}}{\pgfqpoint{8.851069in}{7.081890in}}%
\pgfusepath{clip}%
\pgfsetbuttcap%
\pgfsetroundjoin%
\definecolor{currentfill}{rgb}{0.199430,0.387607,0.554642}%
\pgfsetfillcolor{currentfill}%
\pgfsetfillopacity{0.700000}%
\pgfsetlinewidth{0.501875pt}%
\definecolor{currentstroke}{rgb}{1.000000,1.000000,1.000000}%
\pgfsetstrokecolor{currentstroke}%
\pgfsetstrokeopacity{0.700000}%
\pgfsetdash{}{0pt}%
\pgfpathmoveto{\pgfqpoint{1.955681in}{2.727074in}}%
\pgfpathcurveto{\pgfqpoint{1.968704in}{2.727074in}}{\pgfqpoint{1.981195in}{2.732248in}}{\pgfqpoint{1.990404in}{2.741456in}}%
\pgfpathcurveto{\pgfqpoint{1.999612in}{2.750665in}}{\pgfqpoint{2.004786in}{2.763156in}}{\pgfqpoint{2.004786in}{2.776179in}}%
\pgfpathcurveto{\pgfqpoint{2.004786in}{2.789201in}}{\pgfqpoint{1.999612in}{2.801692in}}{\pgfqpoint{1.990404in}{2.810901in}}%
\pgfpathcurveto{\pgfqpoint{1.981195in}{2.820109in}}{\pgfqpoint{1.968704in}{2.825283in}}{\pgfqpoint{1.955681in}{2.825283in}}%
\pgfpathcurveto{\pgfqpoint{1.942659in}{2.825283in}}{\pgfqpoint{1.930168in}{2.820109in}}{\pgfqpoint{1.920959in}{2.810901in}}%
\pgfpathcurveto{\pgfqpoint{1.911751in}{2.801692in}}{\pgfqpoint{1.906577in}{2.789201in}}{\pgfqpoint{1.906577in}{2.776179in}}%
\pgfpathcurveto{\pgfqpoint{1.906577in}{2.763156in}}{\pgfqpoint{1.911751in}{2.750665in}}{\pgfqpoint{1.920959in}{2.741456in}}%
\pgfpathcurveto{\pgfqpoint{1.930168in}{2.732248in}}{\pgfqpoint{1.942659in}{2.727074in}}{\pgfqpoint{1.955681in}{2.727074in}}%
\pgfpathlineto{\pgfqpoint{1.955681in}{2.727074in}}%
\pgfpathclose%
\pgfusepath{stroke,fill}%
\end{pgfscope}%
\begin{pgfscope}%
\pgfpathrectangle{\pgfqpoint{0.786164in}{0.768110in}}{\pgfqpoint{8.851069in}{7.081890in}}%
\pgfusepath{clip}%
\pgfsetbuttcap%
\pgfsetroundjoin%
\definecolor{currentfill}{rgb}{0.192357,0.403199,0.555836}%
\pgfsetfillcolor{currentfill}%
\pgfsetfillopacity{0.700000}%
\pgfsetlinewidth{0.501875pt}%
\definecolor{currentstroke}{rgb}{1.000000,1.000000,1.000000}%
\pgfsetstrokecolor{currentstroke}%
\pgfsetstrokeopacity{0.700000}%
\pgfsetdash{}{0pt}%
\pgfpathmoveto{\pgfqpoint{1.937415in}{2.639481in}}%
\pgfpathcurveto{\pgfqpoint{1.950437in}{2.639481in}}{\pgfqpoint{1.962929in}{2.644655in}}{\pgfqpoint{1.972137in}{2.653864in}}%
\pgfpathcurveto{\pgfqpoint{1.981345in}{2.663072in}}{\pgfqpoint{1.986519in}{2.675563in}}{\pgfqpoint{1.986519in}{2.688586in}}%
\pgfpathcurveto{\pgfqpoint{1.986519in}{2.701608in}}{\pgfqpoint{1.981345in}{2.714100in}}{\pgfqpoint{1.972137in}{2.723308in}}%
\pgfpathcurveto{\pgfqpoint{1.962929in}{2.732516in}}{\pgfqpoint{1.950437in}{2.737690in}}{\pgfqpoint{1.937415in}{2.737690in}}%
\pgfpathcurveto{\pgfqpoint{1.924392in}{2.737690in}}{\pgfqpoint{1.911901in}{2.732516in}}{\pgfqpoint{1.902693in}{2.723308in}}%
\pgfpathcurveto{\pgfqpoint{1.893484in}{2.714100in}}{\pgfqpoint{1.888310in}{2.701608in}}{\pgfqpoint{1.888310in}{2.688586in}}%
\pgfpathcurveto{\pgfqpoint{1.888310in}{2.675563in}}{\pgfqpoint{1.893484in}{2.663072in}}{\pgfqpoint{1.902693in}{2.653864in}}%
\pgfpathcurveto{\pgfqpoint{1.911901in}{2.644655in}}{\pgfqpoint{1.924392in}{2.639481in}}{\pgfqpoint{1.937415in}{2.639481in}}%
\pgfpathlineto{\pgfqpoint{1.937415in}{2.639481in}}%
\pgfpathclose%
\pgfusepath{stroke,fill}%
\end{pgfscope}%
\begin{pgfscope}%
\pgfpathrectangle{\pgfqpoint{0.786164in}{0.768110in}}{\pgfqpoint{8.851069in}{7.081890in}}%
\pgfusepath{clip}%
\pgfsetbuttcap%
\pgfsetroundjoin%
\definecolor{currentfill}{rgb}{0.185556,0.418570,0.556753}%
\pgfsetfillcolor{currentfill}%
\pgfsetfillopacity{0.700000}%
\pgfsetlinewidth{0.501875pt}%
\definecolor{currentstroke}{rgb}{1.000000,1.000000,1.000000}%
\pgfsetstrokecolor{currentstroke}%
\pgfsetstrokeopacity{0.700000}%
\pgfsetdash{}{0pt}%
\pgfpathmoveto{\pgfqpoint{1.919148in}{2.464295in}}%
\pgfpathcurveto{\pgfqpoint{1.932171in}{2.464295in}}{\pgfqpoint{1.944662in}{2.469469in}}{\pgfqpoint{1.953870in}{2.478678in}}%
\pgfpathcurveto{\pgfqpoint{1.963079in}{2.487886in}}{\pgfqpoint{1.968253in}{2.500377in}}{\pgfqpoint{1.968253in}{2.513400in}}%
\pgfpathcurveto{\pgfqpoint{1.968253in}{2.526423in}}{\pgfqpoint{1.963079in}{2.538914in}}{\pgfqpoint{1.953870in}{2.548122in}}%
\pgfpathcurveto{\pgfqpoint{1.944662in}{2.557331in}}{\pgfqpoint{1.932171in}{2.562504in}}{\pgfqpoint{1.919148in}{2.562504in}}%
\pgfpathcurveto{\pgfqpoint{1.906125in}{2.562504in}}{\pgfqpoint{1.893634in}{2.557331in}}{\pgfqpoint{1.884426in}{2.548122in}}%
\pgfpathcurveto{\pgfqpoint{1.875218in}{2.538914in}}{\pgfqpoint{1.870044in}{2.526423in}}{\pgfqpoint{1.870044in}{2.513400in}}%
\pgfpathcurveto{\pgfqpoint{1.870044in}{2.500377in}}{\pgfqpoint{1.875218in}{2.487886in}}{\pgfqpoint{1.884426in}{2.478678in}}%
\pgfpathcurveto{\pgfqpoint{1.893634in}{2.469469in}}{\pgfqpoint{1.906125in}{2.464295in}}{\pgfqpoint{1.919148in}{2.464295in}}%
\pgfpathlineto{\pgfqpoint{1.919148in}{2.464295in}}%
\pgfpathclose%
\pgfusepath{stroke,fill}%
\end{pgfscope}%
\begin{pgfscope}%
\pgfpathrectangle{\pgfqpoint{0.786164in}{0.768110in}}{\pgfqpoint{8.851069in}{7.081890in}}%
\pgfusepath{clip}%
\pgfsetbuttcap%
\pgfsetroundjoin%
\definecolor{currentfill}{rgb}{0.175841,0.441290,0.557685}%
\pgfsetfillcolor{currentfill}%
\pgfsetfillopacity{0.700000}%
\pgfsetlinewidth{0.501875pt}%
\definecolor{currentstroke}{rgb}{1.000000,1.000000,1.000000}%
\pgfsetstrokecolor{currentstroke}%
\pgfsetstrokeopacity{0.700000}%
\pgfsetdash{}{0pt}%
\pgfpathmoveto{\pgfqpoint{1.919148in}{2.376702in}}%
\pgfpathcurveto{\pgfqpoint{1.932171in}{2.376702in}}{\pgfqpoint{1.944662in}{2.381876in}}{\pgfqpoint{1.953870in}{2.391085in}}%
\pgfpathcurveto{\pgfqpoint{1.963079in}{2.400293in}}{\pgfqpoint{1.968253in}{2.412784in}}{\pgfqpoint{1.968253in}{2.425807in}}%
\pgfpathcurveto{\pgfqpoint{1.968253in}{2.438830in}}{\pgfqpoint{1.963079in}{2.451321in}}{\pgfqpoint{1.953870in}{2.460529in}}%
\pgfpathcurveto{\pgfqpoint{1.944662in}{2.469738in}}{\pgfqpoint{1.932171in}{2.474912in}}{\pgfqpoint{1.919148in}{2.474912in}}%
\pgfpathcurveto{\pgfqpoint{1.906125in}{2.474912in}}{\pgfqpoint{1.893634in}{2.469738in}}{\pgfqpoint{1.884426in}{2.460529in}}%
\pgfpathcurveto{\pgfqpoint{1.875218in}{2.451321in}}{\pgfqpoint{1.870044in}{2.438830in}}{\pgfqpoint{1.870044in}{2.425807in}}%
\pgfpathcurveto{\pgfqpoint{1.870044in}{2.412784in}}{\pgfqpoint{1.875218in}{2.400293in}}{\pgfqpoint{1.884426in}{2.391085in}}%
\pgfpathcurveto{\pgfqpoint{1.893634in}{2.381876in}}{\pgfqpoint{1.906125in}{2.376702in}}{\pgfqpoint{1.919148in}{2.376702in}}%
\pgfpathlineto{\pgfqpoint{1.919148in}{2.376702in}}%
\pgfpathclose%
\pgfusepath{stroke,fill}%
\end{pgfscope}%
\begin{pgfscope}%
\pgfpathrectangle{\pgfqpoint{0.786164in}{0.768110in}}{\pgfqpoint{8.851069in}{7.081890in}}%
\pgfusepath{clip}%
\pgfsetbuttcap%
\pgfsetroundjoin%
\definecolor{currentfill}{rgb}{0.177423,0.437527,0.557565}%
\pgfsetfillcolor{currentfill}%
\pgfsetfillopacity{0.700000}%
\pgfsetlinewidth{0.501875pt}%
\definecolor{currentstroke}{rgb}{1.000000,1.000000,1.000000}%
\pgfsetstrokecolor{currentstroke}%
\pgfsetstrokeopacity{0.700000}%
\pgfsetdash{}{0pt}%
\pgfpathmoveto{\pgfqpoint{1.827815in}{2.376702in}}%
\pgfpathcurveto{\pgfqpoint{1.840838in}{2.376702in}}{\pgfqpoint{1.853329in}{2.381876in}}{\pgfqpoint{1.862538in}{2.391085in}}%
\pgfpathcurveto{\pgfqpoint{1.871746in}{2.400293in}}{\pgfqpoint{1.876920in}{2.412784in}}{\pgfqpoint{1.876920in}{2.425807in}}%
\pgfpathcurveto{\pgfqpoint{1.876920in}{2.438830in}}{\pgfqpoint{1.871746in}{2.451321in}}{\pgfqpoint{1.862538in}{2.460529in}}%
\pgfpathcurveto{\pgfqpoint{1.853329in}{2.469738in}}{\pgfqpoint{1.840838in}{2.474912in}}{\pgfqpoint{1.827815in}{2.474912in}}%
\pgfpathcurveto{\pgfqpoint{1.814793in}{2.474912in}}{\pgfqpoint{1.802302in}{2.469738in}}{\pgfqpoint{1.793093in}{2.460529in}}%
\pgfpathcurveto{\pgfqpoint{1.783885in}{2.451321in}}{\pgfqpoint{1.778711in}{2.438830in}}{\pgfqpoint{1.778711in}{2.425807in}}%
\pgfpathcurveto{\pgfqpoint{1.778711in}{2.412784in}}{\pgfqpoint{1.783885in}{2.400293in}}{\pgfqpoint{1.793093in}{2.391085in}}%
\pgfpathcurveto{\pgfqpoint{1.802302in}{2.381876in}}{\pgfqpoint{1.814793in}{2.376702in}}{\pgfqpoint{1.827815in}{2.376702in}}%
\pgfpathlineto{\pgfqpoint{1.827815in}{2.376702in}}%
\pgfpathclose%
\pgfusepath{stroke,fill}%
\end{pgfscope}%
\begin{pgfscope}%
\pgfpathrectangle{\pgfqpoint{0.786164in}{0.768110in}}{\pgfqpoint{8.851069in}{7.081890in}}%
\pgfusepath{clip}%
\pgfsetbuttcap%
\pgfsetroundjoin%
\definecolor{currentfill}{rgb}{0.174274,0.445044,0.557792}%
\pgfsetfillcolor{currentfill}%
\pgfsetfillopacity{0.700000}%
\pgfsetlinewidth{0.501875pt}%
\definecolor{currentstroke}{rgb}{1.000000,1.000000,1.000000}%
\pgfsetstrokecolor{currentstroke}%
\pgfsetstrokeopacity{0.700000}%
\pgfsetdash{}{0pt}%
\pgfpathmoveto{\pgfqpoint{1.773016in}{2.420499in}}%
\pgfpathcurveto{\pgfqpoint{1.786038in}{2.420499in}}{\pgfqpoint{1.798529in}{2.425673in}}{\pgfqpoint{1.807738in}{2.434881in}}%
\pgfpathcurveto{\pgfqpoint{1.816946in}{2.444090in}}{\pgfqpoint{1.822120in}{2.456581in}}{\pgfqpoint{1.822120in}{2.469603in}}%
\pgfpathcurveto{\pgfqpoint{1.822120in}{2.482626in}}{\pgfqpoint{1.816946in}{2.495117in}}{\pgfqpoint{1.807738in}{2.504326in}}%
\pgfpathcurveto{\pgfqpoint{1.798529in}{2.513534in}}{\pgfqpoint{1.786038in}{2.518708in}}{\pgfqpoint{1.773016in}{2.518708in}}%
\pgfpathcurveto{\pgfqpoint{1.759993in}{2.518708in}}{\pgfqpoint{1.747502in}{2.513534in}}{\pgfqpoint{1.738293in}{2.504326in}}%
\pgfpathcurveto{\pgfqpoint{1.729085in}{2.495117in}}{\pgfqpoint{1.723911in}{2.482626in}}{\pgfqpoint{1.723911in}{2.469603in}}%
\pgfpathcurveto{\pgfqpoint{1.723911in}{2.456581in}}{\pgfqpoint{1.729085in}{2.444090in}}{\pgfqpoint{1.738293in}{2.434881in}}%
\pgfpathcurveto{\pgfqpoint{1.747502in}{2.425673in}}{\pgfqpoint{1.759993in}{2.420499in}}{\pgfqpoint{1.773016in}{2.420499in}}%
\pgfpathlineto{\pgfqpoint{1.773016in}{2.420499in}}%
\pgfpathclose%
\pgfusepath{stroke,fill}%
\end{pgfscope}%
\begin{pgfscope}%
\pgfpathrectangle{\pgfqpoint{0.786164in}{0.768110in}}{\pgfqpoint{8.851069in}{7.081890in}}%
\pgfusepath{clip}%
\pgfsetbuttcap%
\pgfsetroundjoin%
\definecolor{currentfill}{rgb}{0.174274,0.445044,0.557792}%
\pgfsetfillcolor{currentfill}%
\pgfsetfillopacity{0.700000}%
\pgfsetlinewidth{0.501875pt}%
\definecolor{currentstroke}{rgb}{1.000000,1.000000,1.000000}%
\pgfsetstrokecolor{currentstroke}%
\pgfsetstrokeopacity{0.700000}%
\pgfsetdash{}{0pt}%
\pgfpathmoveto{\pgfqpoint{1.699949in}{2.420499in}}%
\pgfpathcurveto{\pgfqpoint{1.712972in}{2.420499in}}{\pgfqpoint{1.725463in}{2.425673in}}{\pgfqpoint{1.734672in}{2.434881in}}%
\pgfpathcurveto{\pgfqpoint{1.743880in}{2.444090in}}{\pgfqpoint{1.749054in}{2.456581in}}{\pgfqpoint{1.749054in}{2.469603in}}%
\pgfpathcurveto{\pgfqpoint{1.749054in}{2.482626in}}{\pgfqpoint{1.743880in}{2.495117in}}{\pgfqpoint{1.734672in}{2.504326in}}%
\pgfpathcurveto{\pgfqpoint{1.725463in}{2.513534in}}{\pgfqpoint{1.712972in}{2.518708in}}{\pgfqpoint{1.699949in}{2.518708in}}%
\pgfpathcurveto{\pgfqpoint{1.686927in}{2.518708in}}{\pgfqpoint{1.674436in}{2.513534in}}{\pgfqpoint{1.665227in}{2.504326in}}%
\pgfpathcurveto{\pgfqpoint{1.656019in}{2.495117in}}{\pgfqpoint{1.650845in}{2.482626in}}{\pgfqpoint{1.650845in}{2.469603in}}%
\pgfpathcurveto{\pgfqpoint{1.650845in}{2.456581in}}{\pgfqpoint{1.656019in}{2.444090in}}{\pgfqpoint{1.665227in}{2.434881in}}%
\pgfpathcurveto{\pgfqpoint{1.674436in}{2.425673in}}{\pgfqpoint{1.686927in}{2.420499in}}{\pgfqpoint{1.699949in}{2.420499in}}%
\pgfpathlineto{\pgfqpoint{1.699949in}{2.420499in}}%
\pgfpathclose%
\pgfusepath{stroke,fill}%
\end{pgfscope}%
\begin{pgfscope}%
\pgfpathrectangle{\pgfqpoint{0.786164in}{0.768110in}}{\pgfqpoint{8.851069in}{7.081890in}}%
\pgfusepath{clip}%
\pgfsetbuttcap%
\pgfsetroundjoin%
\definecolor{currentfill}{rgb}{0.168126,0.459988,0.558082}%
\pgfsetfillcolor{currentfill}%
\pgfsetfillopacity{0.700000}%
\pgfsetlinewidth{0.501875pt}%
\definecolor{currentstroke}{rgb}{1.000000,1.000000,1.000000}%
\pgfsetstrokecolor{currentstroke}%
\pgfsetstrokeopacity{0.700000}%
\pgfsetdash{}{0pt}%
\pgfpathmoveto{\pgfqpoint{1.709083in}{2.420499in}}%
\pgfpathcurveto{\pgfqpoint{1.722105in}{2.420499in}}{\pgfqpoint{1.734596in}{2.425673in}}{\pgfqpoint{1.743805in}{2.434881in}}%
\pgfpathcurveto{\pgfqpoint{1.753013in}{2.444090in}}{\pgfqpoint{1.758187in}{2.456581in}}{\pgfqpoint{1.758187in}{2.469603in}}%
\pgfpathcurveto{\pgfqpoint{1.758187in}{2.482626in}}{\pgfqpoint{1.753013in}{2.495117in}}{\pgfqpoint{1.743805in}{2.504326in}}%
\pgfpathcurveto{\pgfqpoint{1.734596in}{2.513534in}}{\pgfqpoint{1.722105in}{2.518708in}}{\pgfqpoint{1.709083in}{2.518708in}}%
\pgfpathcurveto{\pgfqpoint{1.696060in}{2.518708in}}{\pgfqpoint{1.683569in}{2.513534in}}{\pgfqpoint{1.674360in}{2.504326in}}%
\pgfpathcurveto{\pgfqpoint{1.665152in}{2.495117in}}{\pgfqpoint{1.659978in}{2.482626in}}{\pgfqpoint{1.659978in}{2.469603in}}%
\pgfpathcurveto{\pgfqpoint{1.659978in}{2.456581in}}{\pgfqpoint{1.665152in}{2.444090in}}{\pgfqpoint{1.674360in}{2.434881in}}%
\pgfpathcurveto{\pgfqpoint{1.683569in}{2.425673in}}{\pgfqpoint{1.696060in}{2.420499in}}{\pgfqpoint{1.709083in}{2.420499in}}%
\pgfpathlineto{\pgfqpoint{1.709083in}{2.420499in}}%
\pgfpathclose%
\pgfusepath{stroke,fill}%
\end{pgfscope}%
\begin{pgfscope}%
\pgfpathrectangle{\pgfqpoint{0.786164in}{0.768110in}}{\pgfqpoint{8.851069in}{7.081890in}}%
\pgfusepath{clip}%
\pgfsetbuttcap%
\pgfsetroundjoin%
\definecolor{currentfill}{rgb}{0.146180,0.515413,0.556823}%
\pgfsetfillcolor{currentfill}%
\pgfsetfillopacity{0.700000}%
\pgfsetlinewidth{0.501875pt}%
\definecolor{currentstroke}{rgb}{1.000000,1.000000,1.000000}%
\pgfsetstrokecolor{currentstroke}%
\pgfsetstrokeopacity{0.700000}%
\pgfsetdash{}{0pt}%
\pgfpathmoveto{\pgfqpoint{1.690816in}{2.311008in}}%
\pgfpathcurveto{\pgfqpoint{1.703839in}{2.311008in}}{\pgfqpoint{1.716330in}{2.316182in}}{\pgfqpoint{1.725538in}{2.325390in}}%
\pgfpathcurveto{\pgfqpoint{1.734747in}{2.334598in}}{\pgfqpoint{1.739921in}{2.347089in}}{\pgfqpoint{1.739921in}{2.360112in}}%
\pgfpathcurveto{\pgfqpoint{1.739921in}{2.373135in}}{\pgfqpoint{1.734747in}{2.385626in}}{\pgfqpoint{1.725538in}{2.394834in}}%
\pgfpathcurveto{\pgfqpoint{1.716330in}{2.404043in}}{\pgfqpoint{1.703839in}{2.409217in}}{\pgfqpoint{1.690816in}{2.409217in}}%
\pgfpathcurveto{\pgfqpoint{1.677793in}{2.409217in}}{\pgfqpoint{1.665302in}{2.404043in}}{\pgfqpoint{1.656094in}{2.394834in}}%
\pgfpathcurveto{\pgfqpoint{1.646885in}{2.385626in}}{\pgfqpoint{1.641711in}{2.373135in}}{\pgfqpoint{1.641711in}{2.360112in}}%
\pgfpathcurveto{\pgfqpoint{1.641711in}{2.347089in}}{\pgfqpoint{1.646885in}{2.334598in}}{\pgfqpoint{1.656094in}{2.325390in}}%
\pgfpathcurveto{\pgfqpoint{1.665302in}{2.316182in}}{\pgfqpoint{1.677793in}{2.311008in}}{\pgfqpoint{1.690816in}{2.311008in}}%
\pgfpathlineto{\pgfqpoint{1.690816in}{2.311008in}}%
\pgfpathclose%
\pgfusepath{stroke,fill}%
\end{pgfscope}%
\begin{pgfscope}%
\pgfpathrectangle{\pgfqpoint{0.786164in}{0.768110in}}{\pgfqpoint{8.851069in}{7.081890in}}%
\pgfusepath{clip}%
\pgfsetbuttcap%
\pgfsetroundjoin%
\definecolor{currentfill}{rgb}{0.140536,0.530132,0.555659}%
\pgfsetfillcolor{currentfill}%
\pgfsetfillopacity{0.700000}%
\pgfsetlinewidth{0.501875pt}%
\definecolor{currentstroke}{rgb}{1.000000,1.000000,1.000000}%
\pgfsetstrokecolor{currentstroke}%
\pgfsetstrokeopacity{0.700000}%
\pgfsetdash{}{0pt}%
\pgfpathmoveto{\pgfqpoint{1.782149in}{2.442397in}}%
\pgfpathcurveto{\pgfqpoint{1.795172in}{2.442397in}}{\pgfqpoint{1.807663in}{2.447571in}}{\pgfqpoint{1.816871in}{2.456779in}}%
\pgfpathcurveto{\pgfqpoint{1.826080in}{2.465988in}}{\pgfqpoint{1.831254in}{2.478479in}}{\pgfqpoint{1.831254in}{2.491502in}}%
\pgfpathcurveto{\pgfqpoint{1.831254in}{2.504524in}}{\pgfqpoint{1.826080in}{2.517015in}}{\pgfqpoint{1.816871in}{2.526224in}}%
\pgfpathcurveto{\pgfqpoint{1.807663in}{2.535432in}}{\pgfqpoint{1.795172in}{2.540606in}}{\pgfqpoint{1.782149in}{2.540606in}}%
\pgfpathcurveto{\pgfqpoint{1.769126in}{2.540606in}}{\pgfqpoint{1.756635in}{2.535432in}}{\pgfqpoint{1.747427in}{2.526224in}}%
\pgfpathcurveto{\pgfqpoint{1.738218in}{2.517015in}}{\pgfqpoint{1.733044in}{2.504524in}}{\pgfqpoint{1.733044in}{2.491502in}}%
\pgfpathcurveto{\pgfqpoint{1.733044in}{2.478479in}}{\pgfqpoint{1.738218in}{2.465988in}}{\pgfqpoint{1.747427in}{2.456779in}}%
\pgfpathcurveto{\pgfqpoint{1.756635in}{2.447571in}}{\pgfqpoint{1.769126in}{2.442397in}}{\pgfqpoint{1.782149in}{2.442397in}}%
\pgfpathlineto{\pgfqpoint{1.782149in}{2.442397in}}%
\pgfpathclose%
\pgfusepath{stroke,fill}%
\end{pgfscope}%
\begin{pgfscope}%
\pgfpathrectangle{\pgfqpoint{0.786164in}{0.768110in}}{\pgfqpoint{8.851069in}{7.081890in}}%
\pgfusepath{clip}%
\pgfsetbuttcap%
\pgfsetroundjoin%
\definecolor{currentfill}{rgb}{0.139147,0.533812,0.555298}%
\pgfsetfillcolor{currentfill}%
\pgfsetfillopacity{0.700000}%
\pgfsetlinewidth{0.501875pt}%
\definecolor{currentstroke}{rgb}{1.000000,1.000000,1.000000}%
\pgfsetstrokecolor{currentstroke}%
\pgfsetstrokeopacity{0.700000}%
\pgfsetdash{}{0pt}%
\pgfpathmoveto{\pgfqpoint{1.800415in}{2.508092in}}%
\pgfpathcurveto{\pgfqpoint{1.813438in}{2.508092in}}{\pgfqpoint{1.825929in}{2.513266in}}{\pgfqpoint{1.835138in}{2.522474in}}%
\pgfpathcurveto{\pgfqpoint{1.844346in}{2.531683in}}{\pgfqpoint{1.849520in}{2.544174in}}{\pgfqpoint{1.849520in}{2.557196in}}%
\pgfpathcurveto{\pgfqpoint{1.849520in}{2.570219in}}{\pgfqpoint{1.844346in}{2.582710in}}{\pgfqpoint{1.835138in}{2.591919in}}%
\pgfpathcurveto{\pgfqpoint{1.825929in}{2.601127in}}{\pgfqpoint{1.813438in}{2.606301in}}{\pgfqpoint{1.800415in}{2.606301in}}%
\pgfpathcurveto{\pgfqpoint{1.787393in}{2.606301in}}{\pgfqpoint{1.774902in}{2.601127in}}{\pgfqpoint{1.765693in}{2.591919in}}%
\pgfpathcurveto{\pgfqpoint{1.756485in}{2.582710in}}{\pgfqpoint{1.751311in}{2.570219in}}{\pgfqpoint{1.751311in}{2.557196in}}%
\pgfpathcurveto{\pgfqpoint{1.751311in}{2.544174in}}{\pgfqpoint{1.756485in}{2.531683in}}{\pgfqpoint{1.765693in}{2.522474in}}%
\pgfpathcurveto{\pgfqpoint{1.774902in}{2.513266in}}{\pgfqpoint{1.787393in}{2.508092in}}{\pgfqpoint{1.800415in}{2.508092in}}%
\pgfpathlineto{\pgfqpoint{1.800415in}{2.508092in}}%
\pgfpathclose%
\pgfusepath{stroke,fill}%
\end{pgfscope}%
\begin{pgfscope}%
\pgfpathrectangle{\pgfqpoint{0.786164in}{0.768110in}}{\pgfqpoint{8.851069in}{7.081890in}}%
\pgfusepath{clip}%
\pgfsetbuttcap%
\pgfsetroundjoin%
\definecolor{currentfill}{rgb}{0.140536,0.530132,0.555659}%
\pgfsetfillcolor{currentfill}%
\pgfsetfillopacity{0.700000}%
\pgfsetlinewidth{0.501875pt}%
\definecolor{currentstroke}{rgb}{1.000000,1.000000,1.000000}%
\pgfsetstrokecolor{currentstroke}%
\pgfsetstrokeopacity{0.700000}%
\pgfsetdash{}{0pt}%
\pgfpathmoveto{\pgfqpoint{1.891748in}{3.077446in}}%
\pgfpathcurveto{\pgfqpoint{1.904771in}{3.077446in}}{\pgfqpoint{1.917262in}{3.082620in}}{\pgfqpoint{1.926471in}{3.091828in}}%
\pgfpathcurveto{\pgfqpoint{1.935679in}{3.101037in}}{\pgfqpoint{1.940853in}{3.113528in}}{\pgfqpoint{1.940853in}{3.126550in}}%
\pgfpathcurveto{\pgfqpoint{1.940853in}{3.139573in}}{\pgfqpoint{1.935679in}{3.152064in}}{\pgfqpoint{1.926471in}{3.161273in}}%
\pgfpathcurveto{\pgfqpoint{1.917262in}{3.170481in}}{\pgfqpoint{1.904771in}{3.175655in}}{\pgfqpoint{1.891748in}{3.175655in}}%
\pgfpathcurveto{\pgfqpoint{1.878726in}{3.175655in}}{\pgfqpoint{1.866235in}{3.170481in}}{\pgfqpoint{1.857026in}{3.161273in}}%
\pgfpathcurveto{\pgfqpoint{1.847818in}{3.152064in}}{\pgfqpoint{1.842644in}{3.139573in}}{\pgfqpoint{1.842644in}{3.126550in}}%
\pgfpathcurveto{\pgfqpoint{1.842644in}{3.113528in}}{\pgfqpoint{1.847818in}{3.101037in}}{\pgfqpoint{1.857026in}{3.091828in}}%
\pgfpathcurveto{\pgfqpoint{1.866235in}{3.082620in}}{\pgfqpoint{1.878726in}{3.077446in}}{\pgfqpoint{1.891748in}{3.077446in}}%
\pgfpathlineto{\pgfqpoint{1.891748in}{3.077446in}}%
\pgfpathclose%
\pgfusepath{stroke,fill}%
\end{pgfscope}%
\begin{pgfscope}%
\pgfpathrectangle{\pgfqpoint{0.786164in}{0.768110in}}{\pgfqpoint{8.851069in}{7.081890in}}%
\pgfusepath{clip}%
\pgfsetbuttcap%
\pgfsetroundjoin%
\definecolor{currentfill}{rgb}{0.143343,0.522773,0.556295}%
\pgfsetfillcolor{currentfill}%
\pgfsetfillopacity{0.700000}%
\pgfsetlinewidth{0.501875pt}%
\definecolor{currentstroke}{rgb}{1.000000,1.000000,1.000000}%
\pgfsetstrokecolor{currentstroke}%
\pgfsetstrokeopacity{0.700000}%
\pgfsetdash{}{0pt}%
\pgfpathmoveto{\pgfqpoint{1.882615in}{2.464295in}}%
\pgfpathcurveto{\pgfqpoint{1.895638in}{2.464295in}}{\pgfqpoint{1.908129in}{2.469469in}}{\pgfqpoint{1.917337in}{2.478678in}}%
\pgfpathcurveto{\pgfqpoint{1.926546in}{2.487886in}}{\pgfqpoint{1.931720in}{2.500377in}}{\pgfqpoint{1.931720in}{2.513400in}}%
\pgfpathcurveto{\pgfqpoint{1.931720in}{2.526423in}}{\pgfqpoint{1.926546in}{2.538914in}}{\pgfqpoint{1.917337in}{2.548122in}}%
\pgfpathcurveto{\pgfqpoint{1.908129in}{2.557331in}}{\pgfqpoint{1.895638in}{2.562504in}}{\pgfqpoint{1.882615in}{2.562504in}}%
\pgfpathcurveto{\pgfqpoint{1.869592in}{2.562504in}}{\pgfqpoint{1.857101in}{2.557331in}}{\pgfqpoint{1.847893in}{2.548122in}}%
\pgfpathcurveto{\pgfqpoint{1.838684in}{2.538914in}}{\pgfqpoint{1.833510in}{2.526423in}}{\pgfqpoint{1.833510in}{2.513400in}}%
\pgfpathcurveto{\pgfqpoint{1.833510in}{2.500377in}}{\pgfqpoint{1.838684in}{2.487886in}}{\pgfqpoint{1.847893in}{2.478678in}}%
\pgfpathcurveto{\pgfqpoint{1.857101in}{2.469469in}}{\pgfqpoint{1.869592in}{2.464295in}}{\pgfqpoint{1.882615in}{2.464295in}}%
\pgfpathlineto{\pgfqpoint{1.882615in}{2.464295in}}%
\pgfpathclose%
\pgfusepath{stroke,fill}%
\end{pgfscope}%
\begin{pgfscope}%
\pgfpathrectangle{\pgfqpoint{0.786164in}{0.768110in}}{\pgfqpoint{8.851069in}{7.081890in}}%
\pgfusepath{clip}%
\pgfsetbuttcap%
\pgfsetroundjoin%
\definecolor{currentfill}{rgb}{0.140536,0.530132,0.555659}%
\pgfsetfillcolor{currentfill}%
\pgfsetfillopacity{0.700000}%
\pgfsetlinewidth{0.501875pt}%
\definecolor{currentstroke}{rgb}{1.000000,1.000000,1.000000}%
\pgfsetstrokecolor{currentstroke}%
\pgfsetstrokeopacity{0.700000}%
\pgfsetdash{}{0pt}%
\pgfpathmoveto{\pgfqpoint{1.836949in}{2.376702in}}%
\pgfpathcurveto{\pgfqpoint{1.849971in}{2.376702in}}{\pgfqpoint{1.862462in}{2.381876in}}{\pgfqpoint{1.871671in}{2.391085in}}%
\pgfpathcurveto{\pgfqpoint{1.880879in}{2.400293in}}{\pgfqpoint{1.886053in}{2.412784in}}{\pgfqpoint{1.886053in}{2.425807in}}%
\pgfpathcurveto{\pgfqpoint{1.886053in}{2.438830in}}{\pgfqpoint{1.880879in}{2.451321in}}{\pgfqpoint{1.871671in}{2.460529in}}%
\pgfpathcurveto{\pgfqpoint{1.862462in}{2.469738in}}{\pgfqpoint{1.849971in}{2.474912in}}{\pgfqpoint{1.836949in}{2.474912in}}%
\pgfpathcurveto{\pgfqpoint{1.823926in}{2.474912in}}{\pgfqpoint{1.811435in}{2.469738in}}{\pgfqpoint{1.802226in}{2.460529in}}%
\pgfpathcurveto{\pgfqpoint{1.793018in}{2.451321in}}{\pgfqpoint{1.787844in}{2.438830in}}{\pgfqpoint{1.787844in}{2.425807in}}%
\pgfpathcurveto{\pgfqpoint{1.787844in}{2.412784in}}{\pgfqpoint{1.793018in}{2.400293in}}{\pgfqpoint{1.802226in}{2.391085in}}%
\pgfpathcurveto{\pgfqpoint{1.811435in}{2.381876in}}{\pgfqpoint{1.823926in}{2.376702in}}{\pgfqpoint{1.836949in}{2.376702in}}%
\pgfpathlineto{\pgfqpoint{1.836949in}{2.376702in}}%
\pgfpathclose%
\pgfusepath{stroke,fill}%
\end{pgfscope}%
\begin{pgfscope}%
\pgfpathrectangle{\pgfqpoint{0.786164in}{0.768110in}}{\pgfqpoint{8.851069in}{7.081890in}}%
\pgfusepath{clip}%
\pgfsetbuttcap%
\pgfsetroundjoin%
\definecolor{currentfill}{rgb}{0.124395,0.578002,0.548287}%
\pgfsetfillcolor{currentfill}%
\pgfsetfillopacity{0.700000}%
\pgfsetlinewidth{0.501875pt}%
\definecolor{currentstroke}{rgb}{1.000000,1.000000,1.000000}%
\pgfsetstrokecolor{currentstroke}%
\pgfsetstrokeopacity{0.700000}%
\pgfsetdash{}{0pt}%
\pgfpathmoveto{\pgfqpoint{1.599483in}{2.179618in}}%
\pgfpathcurveto{\pgfqpoint{1.612506in}{2.179618in}}{\pgfqpoint{1.624997in}{2.184792in}}{\pgfqpoint{1.634205in}{2.194001in}}%
\pgfpathcurveto{\pgfqpoint{1.643414in}{2.203209in}}{\pgfqpoint{1.648588in}{2.215700in}}{\pgfqpoint{1.648588in}{2.228723in}}%
\pgfpathcurveto{\pgfqpoint{1.648588in}{2.241745in}}{\pgfqpoint{1.643414in}{2.254237in}}{\pgfqpoint{1.634205in}{2.263445in}}%
\pgfpathcurveto{\pgfqpoint{1.624997in}{2.272653in}}{\pgfqpoint{1.612506in}{2.277827in}}{\pgfqpoint{1.599483in}{2.277827in}}%
\pgfpathcurveto{\pgfqpoint{1.586460in}{2.277827in}}{\pgfqpoint{1.573969in}{2.272653in}}{\pgfqpoint{1.564761in}{2.263445in}}%
\pgfpathcurveto{\pgfqpoint{1.555552in}{2.254237in}}{\pgfqpoint{1.550379in}{2.241745in}}{\pgfqpoint{1.550379in}{2.228723in}}%
\pgfpathcurveto{\pgfqpoint{1.550379in}{2.215700in}}{\pgfqpoint{1.555552in}{2.203209in}}{\pgfqpoint{1.564761in}{2.194001in}}%
\pgfpathcurveto{\pgfqpoint{1.573969in}{2.184792in}}{\pgfqpoint{1.586460in}{2.179618in}}{\pgfqpoint{1.599483in}{2.179618in}}%
\pgfpathlineto{\pgfqpoint{1.599483in}{2.179618in}}%
\pgfpathclose%
\pgfusepath{stroke,fill}%
\end{pgfscope}%
\begin{pgfscope}%
\pgfpathrectangle{\pgfqpoint{0.786164in}{0.768110in}}{\pgfqpoint{8.851069in}{7.081890in}}%
\pgfusepath{clip}%
\pgfsetbuttcap%
\pgfsetroundjoin%
\definecolor{currentfill}{rgb}{0.124395,0.578002,0.548287}%
\pgfsetfillcolor{currentfill}%
\pgfsetfillopacity{0.700000}%
\pgfsetlinewidth{0.501875pt}%
\definecolor{currentstroke}{rgb}{1.000000,1.000000,1.000000}%
\pgfsetstrokecolor{currentstroke}%
\pgfsetstrokeopacity{0.700000}%
\pgfsetdash{}{0pt}%
\pgfpathmoveto{\pgfqpoint{1.590350in}{2.113923in}}%
\pgfpathcurveto{\pgfqpoint{1.603373in}{2.113923in}}{\pgfqpoint{1.615864in}{2.119097in}}{\pgfqpoint{1.625072in}{2.128306in}}%
\pgfpathcurveto{\pgfqpoint{1.634281in}{2.137514in}}{\pgfqpoint{1.639454in}{2.150005in}}{\pgfqpoint{1.639454in}{2.163028in}}%
\pgfpathcurveto{\pgfqpoint{1.639454in}{2.176051in}}{\pgfqpoint{1.634281in}{2.188542in}}{\pgfqpoint{1.625072in}{2.197750in}}%
\pgfpathcurveto{\pgfqpoint{1.615864in}{2.206959in}}{\pgfqpoint{1.603373in}{2.212133in}}{\pgfqpoint{1.590350in}{2.212133in}}%
\pgfpathcurveto{\pgfqpoint{1.577327in}{2.212133in}}{\pgfqpoint{1.564836in}{2.206959in}}{\pgfqpoint{1.555628in}{2.197750in}}%
\pgfpathcurveto{\pgfqpoint{1.546419in}{2.188542in}}{\pgfqpoint{1.541245in}{2.176051in}}{\pgfqpoint{1.541245in}{2.163028in}}%
\pgfpathcurveto{\pgfqpoint{1.541245in}{2.150005in}}{\pgfqpoint{1.546419in}{2.137514in}}{\pgfqpoint{1.555628in}{2.128306in}}%
\pgfpathcurveto{\pgfqpoint{1.564836in}{2.119097in}}{\pgfqpoint{1.577327in}{2.113923in}}{\pgfqpoint{1.590350in}{2.113923in}}%
\pgfpathlineto{\pgfqpoint{1.590350in}{2.113923in}}%
\pgfpathclose%
\pgfusepath{stroke,fill}%
\end{pgfscope}%
\begin{pgfscope}%
\pgfpathrectangle{\pgfqpoint{0.786164in}{0.768110in}}{\pgfqpoint{8.851069in}{7.081890in}}%
\pgfusepath{clip}%
\pgfsetbuttcap%
\pgfsetroundjoin%
\definecolor{currentfill}{rgb}{0.121831,0.589055,0.545623}%
\pgfsetfillcolor{currentfill}%
\pgfsetfillopacity{0.700000}%
\pgfsetlinewidth{0.501875pt}%
\definecolor{currentstroke}{rgb}{1.000000,1.000000,1.000000}%
\pgfsetstrokecolor{currentstroke}%
\pgfsetstrokeopacity{0.700000}%
\pgfsetdash{}{0pt}%
\pgfpathmoveto{\pgfqpoint{1.699949in}{2.179618in}}%
\pgfpathcurveto{\pgfqpoint{1.712972in}{2.179618in}}{\pgfqpoint{1.725463in}{2.184792in}}{\pgfqpoint{1.734672in}{2.194001in}}%
\pgfpathcurveto{\pgfqpoint{1.743880in}{2.203209in}}{\pgfqpoint{1.749054in}{2.215700in}}{\pgfqpoint{1.749054in}{2.228723in}}%
\pgfpathcurveto{\pgfqpoint{1.749054in}{2.241745in}}{\pgfqpoint{1.743880in}{2.254237in}}{\pgfqpoint{1.734672in}{2.263445in}}%
\pgfpathcurveto{\pgfqpoint{1.725463in}{2.272653in}}{\pgfqpoint{1.712972in}{2.277827in}}{\pgfqpoint{1.699949in}{2.277827in}}%
\pgfpathcurveto{\pgfqpoint{1.686927in}{2.277827in}}{\pgfqpoint{1.674436in}{2.272653in}}{\pgfqpoint{1.665227in}{2.263445in}}%
\pgfpathcurveto{\pgfqpoint{1.656019in}{2.254237in}}{\pgfqpoint{1.650845in}{2.241745in}}{\pgfqpoint{1.650845in}{2.228723in}}%
\pgfpathcurveto{\pgfqpoint{1.650845in}{2.215700in}}{\pgfqpoint{1.656019in}{2.203209in}}{\pgfqpoint{1.665227in}{2.194001in}}%
\pgfpathcurveto{\pgfqpoint{1.674436in}{2.184792in}}{\pgfqpoint{1.686927in}{2.179618in}}{\pgfqpoint{1.699949in}{2.179618in}}%
\pgfpathlineto{\pgfqpoint{1.699949in}{2.179618in}}%
\pgfpathclose%
\pgfusepath{stroke,fill}%
\end{pgfscope}%
\begin{pgfscope}%
\pgfpathrectangle{\pgfqpoint{0.786164in}{0.768110in}}{\pgfqpoint{8.851069in}{7.081890in}}%
\pgfusepath{clip}%
\pgfsetbuttcap%
\pgfsetroundjoin%
\definecolor{currentfill}{rgb}{0.229739,0.322361,0.545706}%
\pgfsetfillcolor{currentfill}%
\pgfsetfillopacity{0.700000}%
\pgfsetlinewidth{0.501875pt}%
\definecolor{currentstroke}{rgb}{1.000000,1.000000,1.000000}%
\pgfsetstrokecolor{currentstroke}%
\pgfsetstrokeopacity{0.700000}%
\pgfsetdash{}{0pt}%
\pgfpathmoveto{\pgfqpoint{1.462484in}{2.267211in}}%
\pgfpathcurveto{\pgfqpoint{1.475507in}{2.267211in}}{\pgfqpoint{1.487998in}{2.272385in}}{\pgfqpoint{1.497206in}{2.281593in}}%
\pgfpathcurveto{\pgfqpoint{1.506414in}{2.290802in}}{\pgfqpoint{1.511588in}{2.303293in}}{\pgfqpoint{1.511588in}{2.316316in}}%
\pgfpathcurveto{\pgfqpoint{1.511588in}{2.329338in}}{\pgfqpoint{1.506414in}{2.341829in}}{\pgfqpoint{1.497206in}{2.351038in}}%
\pgfpathcurveto{\pgfqpoint{1.487998in}{2.360246in}}{\pgfqpoint{1.475507in}{2.365420in}}{\pgfqpoint{1.462484in}{2.365420in}}%
\pgfpathcurveto{\pgfqpoint{1.449461in}{2.365420in}}{\pgfqpoint{1.436970in}{2.360246in}}{\pgfqpoint{1.427762in}{2.351038in}}%
\pgfpathcurveto{\pgfqpoint{1.418553in}{2.341829in}}{\pgfqpoint{1.413379in}{2.329338in}}{\pgfqpoint{1.413379in}{2.316316in}}%
\pgfpathcurveto{\pgfqpoint{1.413379in}{2.303293in}}{\pgfqpoint{1.418553in}{2.290802in}}{\pgfqpoint{1.427762in}{2.281593in}}%
\pgfpathcurveto{\pgfqpoint{1.436970in}{2.272385in}}{\pgfqpoint{1.449461in}{2.267211in}}{\pgfqpoint{1.462484in}{2.267211in}}%
\pgfpathlineto{\pgfqpoint{1.462484in}{2.267211in}}%
\pgfpathclose%
\pgfusepath{stroke,fill}%
\end{pgfscope}%
\begin{pgfscope}%
\pgfpathrectangle{\pgfqpoint{0.786164in}{0.768110in}}{\pgfqpoint{8.851069in}{7.081890in}}%
\pgfusepath{clip}%
\pgfsetbuttcap%
\pgfsetroundjoin%
\definecolor{currentfill}{rgb}{0.223925,0.334994,0.548053}%
\pgfsetfillcolor{currentfill}%
\pgfsetfillopacity{0.700000}%
\pgfsetlinewidth{0.501875pt}%
\definecolor{currentstroke}{rgb}{1.000000,1.000000,1.000000}%
\pgfsetstrokecolor{currentstroke}%
\pgfsetstrokeopacity{0.700000}%
\pgfsetdash{}{0pt}%
\pgfpathmoveto{\pgfqpoint{1.416817in}{2.245313in}}%
\pgfpathcurveto{\pgfqpoint{1.429840in}{2.245313in}}{\pgfqpoint{1.442331in}{2.250487in}}{\pgfqpoint{1.451540in}{2.259695in}}%
\pgfpathcurveto{\pgfqpoint{1.460748in}{2.268904in}}{\pgfqpoint{1.465922in}{2.281395in}}{\pgfqpoint{1.465922in}{2.294417in}}%
\pgfpathcurveto{\pgfqpoint{1.465922in}{2.307440in}}{\pgfqpoint{1.460748in}{2.319931in}}{\pgfqpoint{1.451540in}{2.329140in}}%
\pgfpathcurveto{\pgfqpoint{1.442331in}{2.338348in}}{\pgfqpoint{1.429840in}{2.343522in}}{\pgfqpoint{1.416817in}{2.343522in}}%
\pgfpathcurveto{\pgfqpoint{1.403795in}{2.343522in}}{\pgfqpoint{1.391304in}{2.338348in}}{\pgfqpoint{1.382095in}{2.329140in}}%
\pgfpathcurveto{\pgfqpoint{1.372887in}{2.319931in}}{\pgfqpoint{1.367713in}{2.307440in}}{\pgfqpoint{1.367713in}{2.294417in}}%
\pgfpathcurveto{\pgfqpoint{1.367713in}{2.281395in}}{\pgfqpoint{1.372887in}{2.268904in}}{\pgfqpoint{1.382095in}{2.259695in}}%
\pgfpathcurveto{\pgfqpoint{1.391304in}{2.250487in}}{\pgfqpoint{1.403795in}{2.245313in}}{\pgfqpoint{1.416817in}{2.245313in}}%
\pgfpathlineto{\pgfqpoint{1.416817in}{2.245313in}}%
\pgfpathclose%
\pgfusepath{stroke,fill}%
\end{pgfscope}%
\begin{pgfscope}%
\pgfpathrectangle{\pgfqpoint{0.786164in}{0.768110in}}{\pgfqpoint{8.851069in}{7.081890in}}%
\pgfusepath{clip}%
\pgfsetbuttcap%
\pgfsetroundjoin%
\definecolor{currentfill}{rgb}{0.225863,0.330805,0.547314}%
\pgfsetfillcolor{currentfill}%
\pgfsetfillopacity{0.700000}%
\pgfsetlinewidth{0.501875pt}%
\definecolor{currentstroke}{rgb}{1.000000,1.000000,1.000000}%
\pgfsetstrokecolor{currentstroke}%
\pgfsetstrokeopacity{0.700000}%
\pgfsetdash{}{0pt}%
\pgfpathmoveto{\pgfqpoint{1.489884in}{2.289109in}}%
\pgfpathcurveto{\pgfqpoint{1.502906in}{2.289109in}}{\pgfqpoint{1.515397in}{2.294283in}}{\pgfqpoint{1.524606in}{2.303492in}}%
\pgfpathcurveto{\pgfqpoint{1.533814in}{2.312700in}}{\pgfqpoint{1.538988in}{2.325191in}}{\pgfqpoint{1.538988in}{2.338214in}}%
\pgfpathcurveto{\pgfqpoint{1.538988in}{2.351237in}}{\pgfqpoint{1.533814in}{2.363728in}}{\pgfqpoint{1.524606in}{2.372936in}}%
\pgfpathcurveto{\pgfqpoint{1.515397in}{2.382145in}}{\pgfqpoint{1.502906in}{2.387319in}}{\pgfqpoint{1.489884in}{2.387319in}}%
\pgfpathcurveto{\pgfqpoint{1.476861in}{2.387319in}}{\pgfqpoint{1.464370in}{2.382145in}}{\pgfqpoint{1.455161in}{2.372936in}}%
\pgfpathcurveto{\pgfqpoint{1.445953in}{2.363728in}}{\pgfqpoint{1.440779in}{2.351237in}}{\pgfqpoint{1.440779in}{2.338214in}}%
\pgfpathcurveto{\pgfqpoint{1.440779in}{2.325191in}}{\pgfqpoint{1.445953in}{2.312700in}}{\pgfqpoint{1.455161in}{2.303492in}}%
\pgfpathcurveto{\pgfqpoint{1.464370in}{2.294283in}}{\pgfqpoint{1.476861in}{2.289109in}}{\pgfqpoint{1.489884in}{2.289109in}}%
\pgfpathlineto{\pgfqpoint{1.489884in}{2.289109in}}%
\pgfpathclose%
\pgfusepath{stroke,fill}%
\end{pgfscope}%
\begin{pgfscope}%
\pgfpathrectangle{\pgfqpoint{0.786164in}{0.768110in}}{\pgfqpoint{8.851069in}{7.081890in}}%
\pgfusepath{clip}%
\pgfsetbuttcap%
\pgfsetroundjoin%
\definecolor{currentfill}{rgb}{0.218130,0.347432,0.550038}%
\pgfsetfillcolor{currentfill}%
\pgfsetfillopacity{0.700000}%
\pgfsetlinewidth{0.501875pt}%
\definecolor{currentstroke}{rgb}{1.000000,1.000000,1.000000}%
\pgfsetstrokecolor{currentstroke}%
\pgfsetstrokeopacity{0.700000}%
\pgfsetdash{}{0pt}%
\pgfpathmoveto{\pgfqpoint{1.508150in}{2.311008in}}%
\pgfpathcurveto{\pgfqpoint{1.521173in}{2.311008in}}{\pgfqpoint{1.533664in}{2.316182in}}{\pgfqpoint{1.542872in}{2.325390in}}%
\pgfpathcurveto{\pgfqpoint{1.552081in}{2.334598in}}{\pgfqpoint{1.557255in}{2.347089in}}{\pgfqpoint{1.557255in}{2.360112in}}%
\pgfpathcurveto{\pgfqpoint{1.557255in}{2.373135in}}{\pgfqpoint{1.552081in}{2.385626in}}{\pgfqpoint{1.542872in}{2.394834in}}%
\pgfpathcurveto{\pgfqpoint{1.533664in}{2.404043in}}{\pgfqpoint{1.521173in}{2.409217in}}{\pgfqpoint{1.508150in}{2.409217in}}%
\pgfpathcurveto{\pgfqpoint{1.495128in}{2.409217in}}{\pgfqpoint{1.482636in}{2.404043in}}{\pgfqpoint{1.473428in}{2.394834in}}%
\pgfpathcurveto{\pgfqpoint{1.464220in}{2.385626in}}{\pgfqpoint{1.459046in}{2.373135in}}{\pgfqpoint{1.459046in}{2.360112in}}%
\pgfpathcurveto{\pgfqpoint{1.459046in}{2.347089in}}{\pgfqpoint{1.464220in}{2.334598in}}{\pgfqpoint{1.473428in}{2.325390in}}%
\pgfpathcurveto{\pgfqpoint{1.482636in}{2.316182in}}{\pgfqpoint{1.495128in}{2.311008in}}{\pgfqpoint{1.508150in}{2.311008in}}%
\pgfpathlineto{\pgfqpoint{1.508150in}{2.311008in}}%
\pgfpathclose%
\pgfusepath{stroke,fill}%
\end{pgfscope}%
\begin{pgfscope}%
\pgfpathrectangle{\pgfqpoint{0.786164in}{0.768110in}}{\pgfqpoint{8.851069in}{7.081890in}}%
\pgfusepath{clip}%
\pgfsetbuttcap%
\pgfsetroundjoin%
\definecolor{currentfill}{rgb}{0.203063,0.379716,0.553925}%
\pgfsetfillcolor{currentfill}%
\pgfsetfillopacity{0.700000}%
\pgfsetlinewidth{0.501875pt}%
\definecolor{currentstroke}{rgb}{1.000000,1.000000,1.000000}%
\pgfsetstrokecolor{currentstroke}%
\pgfsetstrokeopacity{0.700000}%
\pgfsetdash{}{0pt}%
\pgfpathmoveto{\pgfqpoint{1.535550in}{2.289109in}}%
\pgfpathcurveto{\pgfqpoint{1.548573in}{2.289109in}}{\pgfqpoint{1.561064in}{2.294283in}}{\pgfqpoint{1.570272in}{2.303492in}}%
\pgfpathcurveto{\pgfqpoint{1.579481in}{2.312700in}}{\pgfqpoint{1.584655in}{2.325191in}}{\pgfqpoint{1.584655in}{2.338214in}}%
\pgfpathcurveto{\pgfqpoint{1.584655in}{2.351237in}}{\pgfqpoint{1.579481in}{2.363728in}}{\pgfqpoint{1.570272in}{2.372936in}}%
\pgfpathcurveto{\pgfqpoint{1.561064in}{2.382145in}}{\pgfqpoint{1.548573in}{2.387319in}}{\pgfqpoint{1.535550in}{2.387319in}}%
\pgfpathcurveto{\pgfqpoint{1.522527in}{2.387319in}}{\pgfqpoint{1.510036in}{2.382145in}}{\pgfqpoint{1.500828in}{2.372936in}}%
\pgfpathcurveto{\pgfqpoint{1.491619in}{2.363728in}}{\pgfqpoint{1.486445in}{2.351237in}}{\pgfqpoint{1.486445in}{2.338214in}}%
\pgfpathcurveto{\pgfqpoint{1.486445in}{2.325191in}}{\pgfqpoint{1.491619in}{2.312700in}}{\pgfqpoint{1.500828in}{2.303492in}}%
\pgfpathcurveto{\pgfqpoint{1.510036in}{2.294283in}}{\pgfqpoint{1.522527in}{2.289109in}}{\pgfqpoint{1.535550in}{2.289109in}}%
\pgfpathlineto{\pgfqpoint{1.535550in}{2.289109in}}%
\pgfpathclose%
\pgfusepath{stroke,fill}%
\end{pgfscope}%
\begin{pgfscope}%
\pgfpathrectangle{\pgfqpoint{0.786164in}{0.768110in}}{\pgfqpoint{8.851069in}{7.081890in}}%
\pgfusepath{clip}%
\pgfsetbuttcap%
\pgfsetroundjoin%
\definecolor{currentfill}{rgb}{0.190631,0.407061,0.556089}%
\pgfsetfillcolor{currentfill}%
\pgfsetfillopacity{0.700000}%
\pgfsetlinewidth{0.501875pt}%
\definecolor{currentstroke}{rgb}{1.000000,1.000000,1.000000}%
\pgfsetstrokecolor{currentstroke}%
\pgfsetstrokeopacity{0.700000}%
\pgfsetdash{}{0pt}%
\pgfpathmoveto{\pgfqpoint{1.334618in}{2.113923in}}%
\pgfpathcurveto{\pgfqpoint{1.347641in}{2.113923in}}{\pgfqpoint{1.360132in}{2.119097in}}{\pgfqpoint{1.369340in}{2.128306in}}%
\pgfpathcurveto{\pgfqpoint{1.378548in}{2.137514in}}{\pgfqpoint{1.383722in}{2.150005in}}{\pgfqpoint{1.383722in}{2.163028in}}%
\pgfpathcurveto{\pgfqpoint{1.383722in}{2.176051in}}{\pgfqpoint{1.378548in}{2.188542in}}{\pgfqpoint{1.369340in}{2.197750in}}%
\pgfpathcurveto{\pgfqpoint{1.360132in}{2.206959in}}{\pgfqpoint{1.347641in}{2.212133in}}{\pgfqpoint{1.334618in}{2.212133in}}%
\pgfpathcurveto{\pgfqpoint{1.321595in}{2.212133in}}{\pgfqpoint{1.309104in}{2.206959in}}{\pgfqpoint{1.299896in}{2.197750in}}%
\pgfpathcurveto{\pgfqpoint{1.290687in}{2.188542in}}{\pgfqpoint{1.285513in}{2.176051in}}{\pgfqpoint{1.285513in}{2.163028in}}%
\pgfpathcurveto{\pgfqpoint{1.285513in}{2.150005in}}{\pgfqpoint{1.290687in}{2.137514in}}{\pgfqpoint{1.299896in}{2.128306in}}%
\pgfpathcurveto{\pgfqpoint{1.309104in}{2.119097in}}{\pgfqpoint{1.321595in}{2.113923in}}{\pgfqpoint{1.334618in}{2.113923in}}%
\pgfpathlineto{\pgfqpoint{1.334618in}{2.113923in}}%
\pgfpathclose%
\pgfusepath{stroke,fill}%
\end{pgfscope}%
\begin{pgfscope}%
\pgfpathrectangle{\pgfqpoint{0.786164in}{0.768110in}}{\pgfqpoint{8.851069in}{7.081890in}}%
\pgfusepath{clip}%
\pgfsetbuttcap%
\pgfsetroundjoin%
\definecolor{currentfill}{rgb}{0.185556,0.418570,0.556753}%
\pgfsetfillcolor{currentfill}%
\pgfsetfillopacity{0.700000}%
\pgfsetlinewidth{0.501875pt}%
\definecolor{currentstroke}{rgb}{1.000000,1.000000,1.000000}%
\pgfsetstrokecolor{currentstroke}%
\pgfsetstrokeopacity{0.700000}%
\pgfsetdash{}{0pt}%
\pgfpathmoveto{\pgfqpoint{1.362018in}{2.004432in}}%
\pgfpathcurveto{\pgfqpoint{1.375040in}{2.004432in}}{\pgfqpoint{1.387531in}{2.009606in}}{\pgfqpoint{1.396740in}{2.018815in}}%
\pgfpathcurveto{\pgfqpoint{1.405948in}{2.028023in}}{\pgfqpoint{1.411122in}{2.040514in}}{\pgfqpoint{1.411122in}{2.053537in}}%
\pgfpathcurveto{\pgfqpoint{1.411122in}{2.066560in}}{\pgfqpoint{1.405948in}{2.079051in}}{\pgfqpoint{1.396740in}{2.088259in}}%
\pgfpathcurveto{\pgfqpoint{1.387531in}{2.097468in}}{\pgfqpoint{1.375040in}{2.102642in}}{\pgfqpoint{1.362018in}{2.102642in}}%
\pgfpathcurveto{\pgfqpoint{1.348995in}{2.102642in}}{\pgfqpoint{1.336504in}{2.097468in}}{\pgfqpoint{1.327295in}{2.088259in}}%
\pgfpathcurveto{\pgfqpoint{1.318087in}{2.079051in}}{\pgfqpoint{1.312913in}{2.066560in}}{\pgfqpoint{1.312913in}{2.053537in}}%
\pgfpathcurveto{\pgfqpoint{1.312913in}{2.040514in}}{\pgfqpoint{1.318087in}{2.028023in}}{\pgfqpoint{1.327295in}{2.018815in}}%
\pgfpathcurveto{\pgfqpoint{1.336504in}{2.009606in}}{\pgfqpoint{1.348995in}{2.004432in}}{\pgfqpoint{1.362018in}{2.004432in}}%
\pgfpathlineto{\pgfqpoint{1.362018in}{2.004432in}}%
\pgfpathclose%
\pgfusepath{stroke,fill}%
\end{pgfscope}%
\begin{pgfscope}%
\pgfpathrectangle{\pgfqpoint{0.786164in}{0.768110in}}{\pgfqpoint{8.851069in}{7.081890in}}%
\pgfusepath{clip}%
\pgfsetbuttcap%
\pgfsetroundjoin%
\definecolor{currentfill}{rgb}{0.192357,0.403199,0.555836}%
\pgfsetfillcolor{currentfill}%
\pgfsetfillopacity{0.700000}%
\pgfsetlinewidth{0.501875pt}%
\definecolor{currentstroke}{rgb}{1.000000,1.000000,1.000000}%
\pgfsetstrokecolor{currentstroke}%
\pgfsetstrokeopacity{0.700000}%
\pgfsetdash{}{0pt}%
\pgfpathmoveto{\pgfqpoint{1.389418in}{2.070127in}}%
\pgfpathcurveto{\pgfqpoint{1.402440in}{2.070127in}}{\pgfqpoint{1.414931in}{2.075301in}}{\pgfqpoint{1.424140in}{2.084509in}}%
\pgfpathcurveto{\pgfqpoint{1.433348in}{2.093718in}}{\pgfqpoint{1.438522in}{2.106209in}}{\pgfqpoint{1.438522in}{2.119232in}}%
\pgfpathcurveto{\pgfqpoint{1.438522in}{2.132254in}}{\pgfqpoint{1.433348in}{2.144745in}}{\pgfqpoint{1.424140in}{2.153954in}}%
\pgfpathcurveto{\pgfqpoint{1.414931in}{2.163162in}}{\pgfqpoint{1.402440in}{2.168336in}}{\pgfqpoint{1.389418in}{2.168336in}}%
\pgfpathcurveto{\pgfqpoint{1.376395in}{2.168336in}}{\pgfqpoint{1.363904in}{2.163162in}}{\pgfqpoint{1.354695in}{2.153954in}}%
\pgfpathcurveto{\pgfqpoint{1.345487in}{2.144745in}}{\pgfqpoint{1.340313in}{2.132254in}}{\pgfqpoint{1.340313in}{2.119232in}}%
\pgfpathcurveto{\pgfqpoint{1.340313in}{2.106209in}}{\pgfqpoint{1.345487in}{2.093718in}}{\pgfqpoint{1.354695in}{2.084509in}}%
\pgfpathcurveto{\pgfqpoint{1.363904in}{2.075301in}}{\pgfqpoint{1.376395in}{2.070127in}}{\pgfqpoint{1.389418in}{2.070127in}}%
\pgfpathlineto{\pgfqpoint{1.389418in}{2.070127in}}%
\pgfpathclose%
\pgfusepath{stroke,fill}%
\end{pgfscope}%
\begin{pgfscope}%
\pgfpathrectangle{\pgfqpoint{0.786164in}{0.768110in}}{\pgfqpoint{8.851069in}{7.081890in}}%
\pgfusepath{clip}%
\pgfsetbuttcap%
\pgfsetroundjoin%
\definecolor{currentfill}{rgb}{0.185556,0.418570,0.556753}%
\pgfsetfillcolor{currentfill}%
\pgfsetfillopacity{0.700000}%
\pgfsetlinewidth{0.501875pt}%
\definecolor{currentstroke}{rgb}{1.000000,1.000000,1.000000}%
\pgfsetstrokecolor{currentstroke}%
\pgfsetstrokeopacity{0.700000}%
\pgfsetdash{}{0pt}%
\pgfpathmoveto{\pgfqpoint{1.389418in}{2.026330in}}%
\pgfpathcurveto{\pgfqpoint{1.402440in}{2.026330in}}{\pgfqpoint{1.414931in}{2.031504in}}{\pgfqpoint{1.424140in}{2.040713in}}%
\pgfpathcurveto{\pgfqpoint{1.433348in}{2.049921in}}{\pgfqpoint{1.438522in}{2.062412in}}{\pgfqpoint{1.438522in}{2.075435in}}%
\pgfpathcurveto{\pgfqpoint{1.438522in}{2.088458in}}{\pgfqpoint{1.433348in}{2.100949in}}{\pgfqpoint{1.424140in}{2.110157in}}%
\pgfpathcurveto{\pgfqpoint{1.414931in}{2.119366in}}{\pgfqpoint{1.402440in}{2.124540in}}{\pgfqpoint{1.389418in}{2.124540in}}%
\pgfpathcurveto{\pgfqpoint{1.376395in}{2.124540in}}{\pgfqpoint{1.363904in}{2.119366in}}{\pgfqpoint{1.354695in}{2.110157in}}%
\pgfpathcurveto{\pgfqpoint{1.345487in}{2.100949in}}{\pgfqpoint{1.340313in}{2.088458in}}{\pgfqpoint{1.340313in}{2.075435in}}%
\pgfpathcurveto{\pgfqpoint{1.340313in}{2.062412in}}{\pgfqpoint{1.345487in}{2.049921in}}{\pgfqpoint{1.354695in}{2.040713in}}%
\pgfpathcurveto{\pgfqpoint{1.363904in}{2.031504in}}{\pgfqpoint{1.376395in}{2.026330in}}{\pgfqpoint{1.389418in}{2.026330in}}%
\pgfpathlineto{\pgfqpoint{1.389418in}{2.026330in}}%
\pgfpathclose%
\pgfusepath{stroke,fill}%
\end{pgfscope}%
\begin{pgfscope}%
\pgfpathrectangle{\pgfqpoint{0.786164in}{0.768110in}}{\pgfqpoint{8.851069in}{7.081890in}}%
\pgfusepath{clip}%
\pgfsetbuttcap%
\pgfsetroundjoin%
\definecolor{currentfill}{rgb}{0.179019,0.433756,0.557430}%
\pgfsetfillcolor{currentfill}%
\pgfsetfillopacity{0.700000}%
\pgfsetlinewidth{0.501875pt}%
\definecolor{currentstroke}{rgb}{1.000000,1.000000,1.000000}%
\pgfsetstrokecolor{currentstroke}%
\pgfsetstrokeopacity{0.700000}%
\pgfsetdash{}{0pt}%
\pgfpathmoveto{\pgfqpoint{1.261552in}{1.894941in}}%
\pgfpathcurveto{\pgfqpoint{1.274574in}{1.894941in}}{\pgfqpoint{1.287065in}{1.900115in}}{\pgfqpoint{1.296274in}{1.909323in}}%
\pgfpathcurveto{\pgfqpoint{1.305482in}{1.918532in}}{\pgfqpoint{1.310656in}{1.931023in}}{\pgfqpoint{1.310656in}{1.944046in}}%
\pgfpathcurveto{\pgfqpoint{1.310656in}{1.957068in}}{\pgfqpoint{1.305482in}{1.969559in}}{\pgfqpoint{1.296274in}{1.978768in}}%
\pgfpathcurveto{\pgfqpoint{1.287065in}{1.987976in}}{\pgfqpoint{1.274574in}{1.993150in}}{\pgfqpoint{1.261552in}{1.993150in}}%
\pgfpathcurveto{\pgfqpoint{1.248529in}{1.993150in}}{\pgfqpoint{1.236038in}{1.987976in}}{\pgfqpoint{1.226829in}{1.978768in}}%
\pgfpathcurveto{\pgfqpoint{1.217621in}{1.969559in}}{\pgfqpoint{1.212447in}{1.957068in}}{\pgfqpoint{1.212447in}{1.944046in}}%
\pgfpathcurveto{\pgfqpoint{1.212447in}{1.931023in}}{\pgfqpoint{1.217621in}{1.918532in}}{\pgfqpoint{1.226829in}{1.909323in}}%
\pgfpathcurveto{\pgfqpoint{1.236038in}{1.900115in}}{\pgfqpoint{1.248529in}{1.894941in}}{\pgfqpoint{1.261552in}{1.894941in}}%
\pgfpathlineto{\pgfqpoint{1.261552in}{1.894941in}}%
\pgfpathclose%
\pgfusepath{stroke,fill}%
\end{pgfscope}%
\begin{pgfscope}%
\pgfpathrectangle{\pgfqpoint{0.786164in}{0.768110in}}{\pgfqpoint{8.851069in}{7.081890in}}%
\pgfusepath{clip}%
\pgfsetbuttcap%
\pgfsetroundjoin%
\definecolor{currentfill}{rgb}{0.172719,0.448791,0.557885}%
\pgfsetfillcolor{currentfill}%
\pgfsetfillopacity{0.700000}%
\pgfsetlinewidth{0.501875pt}%
\definecolor{currentstroke}{rgb}{1.000000,1.000000,1.000000}%
\pgfsetstrokecolor{currentstroke}%
\pgfsetstrokeopacity{0.700000}%
\pgfsetdash{}{0pt}%
\pgfpathmoveto{\pgfqpoint{1.252418in}{1.785450in}}%
\pgfpathcurveto{\pgfqpoint{1.265441in}{1.785450in}}{\pgfqpoint{1.277932in}{1.790624in}}{\pgfqpoint{1.287140in}{1.799832in}}%
\pgfpathcurveto{\pgfqpoint{1.296349in}{1.809041in}}{\pgfqpoint{1.301523in}{1.821532in}}{\pgfqpoint{1.301523in}{1.834555in}}%
\pgfpathcurveto{\pgfqpoint{1.301523in}{1.847577in}}{\pgfqpoint{1.296349in}{1.860068in}}{\pgfqpoint{1.287140in}{1.869277in}}%
\pgfpathcurveto{\pgfqpoint{1.277932in}{1.878485in}}{\pgfqpoint{1.265441in}{1.883659in}}{\pgfqpoint{1.252418in}{1.883659in}}%
\pgfpathcurveto{\pgfqpoint{1.239396in}{1.883659in}}{\pgfqpoint{1.226904in}{1.878485in}}{\pgfqpoint{1.217696in}{1.869277in}}%
\pgfpathcurveto{\pgfqpoint{1.208488in}{1.860068in}}{\pgfqpoint{1.203314in}{1.847577in}}{\pgfqpoint{1.203314in}{1.834555in}}%
\pgfpathcurveto{\pgfqpoint{1.203314in}{1.821532in}}{\pgfqpoint{1.208488in}{1.809041in}}{\pgfqpoint{1.217696in}{1.799832in}}%
\pgfpathcurveto{\pgfqpoint{1.226904in}{1.790624in}}{\pgfqpoint{1.239396in}{1.785450in}}{\pgfqpoint{1.252418in}{1.785450in}}%
\pgfpathlineto{\pgfqpoint{1.252418in}{1.785450in}}%
\pgfpathclose%
\pgfusepath{stroke,fill}%
\end{pgfscope}%
\begin{pgfscope}%
\pgfpathrectangle{\pgfqpoint{0.786164in}{0.768110in}}{\pgfqpoint{8.851069in}{7.081890in}}%
\pgfusepath{clip}%
\pgfsetbuttcap%
\pgfsetroundjoin%
\definecolor{currentfill}{rgb}{0.174274,0.445044,0.557792}%
\pgfsetfillcolor{currentfill}%
\pgfsetfillopacity{0.700000}%
\pgfsetlinewidth{0.501875pt}%
\definecolor{currentstroke}{rgb}{1.000000,1.000000,1.000000}%
\pgfsetstrokecolor{currentstroke}%
\pgfsetstrokeopacity{0.700000}%
\pgfsetdash{}{0pt}%
\pgfpathmoveto{\pgfqpoint{1.288951in}{1.785450in}}%
\pgfpathcurveto{\pgfqpoint{1.301974in}{1.785450in}}{\pgfqpoint{1.314465in}{1.790624in}}{\pgfqpoint{1.323674in}{1.799832in}}%
\pgfpathcurveto{\pgfqpoint{1.332882in}{1.809041in}}{\pgfqpoint{1.338056in}{1.821532in}}{\pgfqpoint{1.338056in}{1.834555in}}%
\pgfpathcurveto{\pgfqpoint{1.338056in}{1.847577in}}{\pgfqpoint{1.332882in}{1.860068in}}{\pgfqpoint{1.323674in}{1.869277in}}%
\pgfpathcurveto{\pgfqpoint{1.314465in}{1.878485in}}{\pgfqpoint{1.301974in}{1.883659in}}{\pgfqpoint{1.288951in}{1.883659in}}%
\pgfpathcurveto{\pgfqpoint{1.275929in}{1.883659in}}{\pgfqpoint{1.263438in}{1.878485in}}{\pgfqpoint{1.254229in}{1.869277in}}%
\pgfpathcurveto{\pgfqpoint{1.245021in}{1.860068in}}{\pgfqpoint{1.239847in}{1.847577in}}{\pgfqpoint{1.239847in}{1.834555in}}%
\pgfpathcurveto{\pgfqpoint{1.239847in}{1.821532in}}{\pgfqpoint{1.245021in}{1.809041in}}{\pgfqpoint{1.254229in}{1.799832in}}%
\pgfpathcurveto{\pgfqpoint{1.263438in}{1.790624in}}{\pgfqpoint{1.275929in}{1.785450in}}{\pgfqpoint{1.288951in}{1.785450in}}%
\pgfpathlineto{\pgfqpoint{1.288951in}{1.785450in}}%
\pgfpathclose%
\pgfusepath{stroke,fill}%
\end{pgfscope}%
\begin{pgfscope}%
\pgfpathrectangle{\pgfqpoint{0.786164in}{0.768110in}}{\pgfqpoint{8.851069in}{7.081890in}}%
\pgfusepath{clip}%
\pgfsetbuttcap%
\pgfsetroundjoin%
\definecolor{currentfill}{rgb}{0.172719,0.448791,0.557885}%
\pgfsetfillcolor{currentfill}%
\pgfsetfillopacity{0.700000}%
\pgfsetlinewidth{0.501875pt}%
\definecolor{currentstroke}{rgb}{1.000000,1.000000,1.000000}%
\pgfsetstrokecolor{currentstroke}%
\pgfsetstrokeopacity{0.700000}%
\pgfsetdash{}{0pt}%
\pgfpathmoveto{\pgfqpoint{1.298085in}{1.741653in}}%
\pgfpathcurveto{\pgfqpoint{1.311107in}{1.741653in}}{\pgfqpoint{1.323598in}{1.746827in}}{\pgfqpoint{1.332807in}{1.756036in}}%
\pgfpathcurveto{\pgfqpoint{1.342015in}{1.765244in}}{\pgfqpoint{1.347189in}{1.777735in}}{\pgfqpoint{1.347189in}{1.790758in}}%
\pgfpathcurveto{\pgfqpoint{1.347189in}{1.803781in}}{\pgfqpoint{1.342015in}{1.816272in}}{\pgfqpoint{1.332807in}{1.825480in}}%
\pgfpathcurveto{\pgfqpoint{1.323598in}{1.834689in}}{\pgfqpoint{1.311107in}{1.839863in}}{\pgfqpoint{1.298085in}{1.839863in}}%
\pgfpathcurveto{\pgfqpoint{1.285062in}{1.839863in}}{\pgfqpoint{1.272571in}{1.834689in}}{\pgfqpoint{1.263362in}{1.825480in}}%
\pgfpathcurveto{\pgfqpoint{1.254154in}{1.816272in}}{\pgfqpoint{1.248980in}{1.803781in}}{\pgfqpoint{1.248980in}{1.790758in}}%
\pgfpathcurveto{\pgfqpoint{1.248980in}{1.777735in}}{\pgfqpoint{1.254154in}{1.765244in}}{\pgfqpoint{1.263362in}{1.756036in}}%
\pgfpathcurveto{\pgfqpoint{1.272571in}{1.746827in}}{\pgfqpoint{1.285062in}{1.741653in}}{\pgfqpoint{1.298085in}{1.741653in}}%
\pgfpathlineto{\pgfqpoint{1.298085in}{1.741653in}}%
\pgfpathclose%
\pgfusepath{stroke,fill}%
\end{pgfscope}%
\begin{pgfscope}%
\pgfpathrectangle{\pgfqpoint{0.786164in}{0.768110in}}{\pgfqpoint{8.851069in}{7.081890in}}%
\pgfusepath{clip}%
\pgfsetbuttcap%
\pgfsetroundjoin%
\definecolor{currentfill}{rgb}{0.175841,0.441290,0.557685}%
\pgfsetfillcolor{currentfill}%
\pgfsetfillopacity{0.700000}%
\pgfsetlinewidth{0.501875pt}%
\definecolor{currentstroke}{rgb}{1.000000,1.000000,1.000000}%
\pgfsetstrokecolor{currentstroke}%
\pgfsetstrokeopacity{0.700000}%
\pgfsetdash{}{0pt}%
\pgfpathmoveto{\pgfqpoint{1.389418in}{1.807348in}}%
\pgfpathcurveto{\pgfqpoint{1.402440in}{1.807348in}}{\pgfqpoint{1.414931in}{1.812522in}}{\pgfqpoint{1.424140in}{1.821731in}}%
\pgfpathcurveto{\pgfqpoint{1.433348in}{1.830939in}}{\pgfqpoint{1.438522in}{1.843430in}}{\pgfqpoint{1.438522in}{1.856453in}}%
\pgfpathcurveto{\pgfqpoint{1.438522in}{1.869475in}}{\pgfqpoint{1.433348in}{1.881967in}}{\pgfqpoint{1.424140in}{1.891175in}}%
\pgfpathcurveto{\pgfqpoint{1.414931in}{1.900383in}}{\pgfqpoint{1.402440in}{1.905557in}}{\pgfqpoint{1.389418in}{1.905557in}}%
\pgfpathcurveto{\pgfqpoint{1.376395in}{1.905557in}}{\pgfqpoint{1.363904in}{1.900383in}}{\pgfqpoint{1.354695in}{1.891175in}}%
\pgfpathcurveto{\pgfqpoint{1.345487in}{1.881967in}}{\pgfqpoint{1.340313in}{1.869475in}}{\pgfqpoint{1.340313in}{1.856453in}}%
\pgfpathcurveto{\pgfqpoint{1.340313in}{1.843430in}}{\pgfqpoint{1.345487in}{1.830939in}}{\pgfqpoint{1.354695in}{1.821731in}}%
\pgfpathcurveto{\pgfqpoint{1.363904in}{1.812522in}}{\pgfqpoint{1.376395in}{1.807348in}}{\pgfqpoint{1.389418in}{1.807348in}}%
\pgfpathlineto{\pgfqpoint{1.389418in}{1.807348in}}%
\pgfpathclose%
\pgfusepath{stroke,fill}%
\end{pgfscope}%
\begin{pgfscope}%
\pgfpathrectangle{\pgfqpoint{0.786164in}{0.768110in}}{\pgfqpoint{8.851069in}{7.081890in}}%
\pgfusepath{clip}%
\pgfsetbuttcap%
\pgfsetroundjoin%
\definecolor{currentfill}{rgb}{0.179019,0.433756,0.557430}%
\pgfsetfillcolor{currentfill}%
\pgfsetfillopacity{0.700000}%
\pgfsetlinewidth{0.501875pt}%
\definecolor{currentstroke}{rgb}{1.000000,1.000000,1.000000}%
\pgfsetstrokecolor{currentstroke}%
\pgfsetstrokeopacity{0.700000}%
\pgfsetdash{}{0pt}%
\pgfpathmoveto{\pgfqpoint{1.398551in}{1.829246in}}%
\pgfpathcurveto{\pgfqpoint{1.411574in}{1.829246in}}{\pgfqpoint{1.424065in}{1.834420in}}{\pgfqpoint{1.433273in}{1.843629in}}%
\pgfpathcurveto{\pgfqpoint{1.442481in}{1.852837in}}{\pgfqpoint{1.447655in}{1.865328in}}{\pgfqpoint{1.447655in}{1.878351in}}%
\pgfpathcurveto{\pgfqpoint{1.447655in}{1.891374in}}{\pgfqpoint{1.442481in}{1.903865in}}{\pgfqpoint{1.433273in}{1.913073in}}%
\pgfpathcurveto{\pgfqpoint{1.424065in}{1.922282in}}{\pgfqpoint{1.411574in}{1.927456in}}{\pgfqpoint{1.398551in}{1.927456in}}%
\pgfpathcurveto{\pgfqpoint{1.385528in}{1.927456in}}{\pgfqpoint{1.373037in}{1.922282in}}{\pgfqpoint{1.363829in}{1.913073in}}%
\pgfpathcurveto{\pgfqpoint{1.354620in}{1.903865in}}{\pgfqpoint{1.349446in}{1.891374in}}{\pgfqpoint{1.349446in}{1.878351in}}%
\pgfpathcurveto{\pgfqpoint{1.349446in}{1.865328in}}{\pgfqpoint{1.354620in}{1.852837in}}{\pgfqpoint{1.363829in}{1.843629in}}%
\pgfpathcurveto{\pgfqpoint{1.373037in}{1.834420in}}{\pgfqpoint{1.385528in}{1.829246in}}{\pgfqpoint{1.398551in}{1.829246in}}%
\pgfpathlineto{\pgfqpoint{1.398551in}{1.829246in}}%
\pgfpathclose%
\pgfusepath{stroke,fill}%
\end{pgfscope}%
\begin{pgfscope}%
\pgfpathrectangle{\pgfqpoint{0.786164in}{0.768110in}}{\pgfqpoint{8.851069in}{7.081890in}}%
\pgfusepath{clip}%
\pgfsetbuttcap%
\pgfsetroundjoin%
\definecolor{currentfill}{rgb}{0.177423,0.437527,0.557565}%
\pgfsetfillcolor{currentfill}%
\pgfsetfillopacity{0.700000}%
\pgfsetlinewidth{0.501875pt}%
\definecolor{currentstroke}{rgb}{1.000000,1.000000,1.000000}%
\pgfsetstrokecolor{currentstroke}%
\pgfsetstrokeopacity{0.700000}%
\pgfsetdash{}{0pt}%
\pgfpathmoveto{\pgfqpoint{1.389418in}{1.807348in}}%
\pgfpathcurveto{\pgfqpoint{1.402440in}{1.807348in}}{\pgfqpoint{1.414931in}{1.812522in}}{\pgfqpoint{1.424140in}{1.821731in}}%
\pgfpathcurveto{\pgfqpoint{1.433348in}{1.830939in}}{\pgfqpoint{1.438522in}{1.843430in}}{\pgfqpoint{1.438522in}{1.856453in}}%
\pgfpathcurveto{\pgfqpoint{1.438522in}{1.869475in}}{\pgfqpoint{1.433348in}{1.881967in}}{\pgfqpoint{1.424140in}{1.891175in}}%
\pgfpathcurveto{\pgfqpoint{1.414931in}{1.900383in}}{\pgfqpoint{1.402440in}{1.905557in}}{\pgfqpoint{1.389418in}{1.905557in}}%
\pgfpathcurveto{\pgfqpoint{1.376395in}{1.905557in}}{\pgfqpoint{1.363904in}{1.900383in}}{\pgfqpoint{1.354695in}{1.891175in}}%
\pgfpathcurveto{\pgfqpoint{1.345487in}{1.881967in}}{\pgfqpoint{1.340313in}{1.869475in}}{\pgfqpoint{1.340313in}{1.856453in}}%
\pgfpathcurveto{\pgfqpoint{1.340313in}{1.843430in}}{\pgfqpoint{1.345487in}{1.830939in}}{\pgfqpoint{1.354695in}{1.821731in}}%
\pgfpathcurveto{\pgfqpoint{1.363904in}{1.812522in}}{\pgfqpoint{1.376395in}{1.807348in}}{\pgfqpoint{1.389418in}{1.807348in}}%
\pgfpathlineto{\pgfqpoint{1.389418in}{1.807348in}}%
\pgfpathclose%
\pgfusepath{stroke,fill}%
\end{pgfscope}%
\begin{pgfscope}%
\pgfpathrectangle{\pgfqpoint{0.786164in}{0.768110in}}{\pgfqpoint{8.851069in}{7.081890in}}%
\pgfusepath{clip}%
\pgfsetbuttcap%
\pgfsetroundjoin%
\definecolor{currentfill}{rgb}{0.175841,0.441290,0.557685}%
\pgfsetfillcolor{currentfill}%
\pgfsetfillopacity{0.700000}%
\pgfsetlinewidth{0.501875pt}%
\definecolor{currentstroke}{rgb}{1.000000,1.000000,1.000000}%
\pgfsetstrokecolor{currentstroke}%
\pgfsetstrokeopacity{0.700000}%
\pgfsetdash{}{0pt}%
\pgfpathmoveto{\pgfqpoint{1.343751in}{1.719755in}}%
\pgfpathcurveto{\pgfqpoint{1.356774in}{1.719755in}}{\pgfqpoint{1.369265in}{1.724929in}}{\pgfqpoint{1.378473in}{1.734138in}}%
\pgfpathcurveto{\pgfqpoint{1.387682in}{1.743346in}}{\pgfqpoint{1.392856in}{1.755837in}}{\pgfqpoint{1.392856in}{1.768860in}}%
\pgfpathcurveto{\pgfqpoint{1.392856in}{1.781883in}}{\pgfqpoint{1.387682in}{1.794374in}}{\pgfqpoint{1.378473in}{1.803582in}}%
\pgfpathcurveto{\pgfqpoint{1.369265in}{1.812790in}}{\pgfqpoint{1.356774in}{1.817964in}}{\pgfqpoint{1.343751in}{1.817964in}}%
\pgfpathcurveto{\pgfqpoint{1.330728in}{1.817964in}}{\pgfqpoint{1.318237in}{1.812790in}}{\pgfqpoint{1.309029in}{1.803582in}}%
\pgfpathcurveto{\pgfqpoint{1.299820in}{1.794374in}}{\pgfqpoint{1.294646in}{1.781883in}}{\pgfqpoint{1.294646in}{1.768860in}}%
\pgfpathcurveto{\pgfqpoint{1.294646in}{1.755837in}}{\pgfqpoint{1.299820in}{1.743346in}}{\pgfqpoint{1.309029in}{1.734138in}}%
\pgfpathcurveto{\pgfqpoint{1.318237in}{1.724929in}}{\pgfqpoint{1.330728in}{1.719755in}}{\pgfqpoint{1.343751in}{1.719755in}}%
\pgfpathlineto{\pgfqpoint{1.343751in}{1.719755in}}%
\pgfpathclose%
\pgfusepath{stroke,fill}%
\end{pgfscope}%
\begin{pgfscope}%
\pgfpathrectangle{\pgfqpoint{0.786164in}{0.768110in}}{\pgfqpoint{8.851069in}{7.081890in}}%
\pgfusepath{clip}%
\pgfsetbuttcap%
\pgfsetroundjoin%
\definecolor{currentfill}{rgb}{0.179019,0.433756,0.557430}%
\pgfsetfillcolor{currentfill}%
\pgfsetfillopacity{0.700000}%
\pgfsetlinewidth{0.501875pt}%
\definecolor{currentstroke}{rgb}{1.000000,1.000000,1.000000}%
\pgfsetstrokecolor{currentstroke}%
\pgfsetstrokeopacity{0.700000}%
\pgfsetdash{}{0pt}%
\pgfpathmoveto{\pgfqpoint{1.453351in}{1.785450in}}%
\pgfpathcurveto{\pgfqpoint{1.466373in}{1.785450in}}{\pgfqpoint{1.478864in}{1.790624in}}{\pgfqpoint{1.488073in}{1.799832in}}%
\pgfpathcurveto{\pgfqpoint{1.497281in}{1.809041in}}{\pgfqpoint{1.502455in}{1.821532in}}{\pgfqpoint{1.502455in}{1.834555in}}%
\pgfpathcurveto{\pgfqpoint{1.502455in}{1.847577in}}{\pgfqpoint{1.497281in}{1.860068in}}{\pgfqpoint{1.488073in}{1.869277in}}%
\pgfpathcurveto{\pgfqpoint{1.478864in}{1.878485in}}{\pgfqpoint{1.466373in}{1.883659in}}{\pgfqpoint{1.453351in}{1.883659in}}%
\pgfpathcurveto{\pgfqpoint{1.440328in}{1.883659in}}{\pgfqpoint{1.427837in}{1.878485in}}{\pgfqpoint{1.418628in}{1.869277in}}%
\pgfpathcurveto{\pgfqpoint{1.409420in}{1.860068in}}{\pgfqpoint{1.404246in}{1.847577in}}{\pgfqpoint{1.404246in}{1.834555in}}%
\pgfpathcurveto{\pgfqpoint{1.404246in}{1.821532in}}{\pgfqpoint{1.409420in}{1.809041in}}{\pgfqpoint{1.418628in}{1.799832in}}%
\pgfpathcurveto{\pgfqpoint{1.427837in}{1.790624in}}{\pgfqpoint{1.440328in}{1.785450in}}{\pgfqpoint{1.453351in}{1.785450in}}%
\pgfpathlineto{\pgfqpoint{1.453351in}{1.785450in}}%
\pgfpathclose%
\pgfusepath{stroke,fill}%
\end{pgfscope}%
\begin{pgfscope}%
\pgfpathrectangle{\pgfqpoint{0.786164in}{0.768110in}}{\pgfqpoint{8.851069in}{7.081890in}}%
\pgfusepath{clip}%
\pgfsetbuttcap%
\pgfsetroundjoin%
\definecolor{currentfill}{rgb}{0.177423,0.437527,0.557565}%
\pgfsetfillcolor{currentfill}%
\pgfsetfillopacity{0.700000}%
\pgfsetlinewidth{0.501875pt}%
\definecolor{currentstroke}{rgb}{1.000000,1.000000,1.000000}%
\pgfsetstrokecolor{currentstroke}%
\pgfsetstrokeopacity{0.700000}%
\pgfsetdash{}{0pt}%
\pgfpathmoveto{\pgfqpoint{1.362018in}{1.741653in}}%
\pgfpathcurveto{\pgfqpoint{1.375040in}{1.741653in}}{\pgfqpoint{1.387531in}{1.746827in}}{\pgfqpoint{1.396740in}{1.756036in}}%
\pgfpathcurveto{\pgfqpoint{1.405948in}{1.765244in}}{\pgfqpoint{1.411122in}{1.777735in}}{\pgfqpoint{1.411122in}{1.790758in}}%
\pgfpathcurveto{\pgfqpoint{1.411122in}{1.803781in}}{\pgfqpoint{1.405948in}{1.816272in}}{\pgfqpoint{1.396740in}{1.825480in}}%
\pgfpathcurveto{\pgfqpoint{1.387531in}{1.834689in}}{\pgfqpoint{1.375040in}{1.839863in}}{\pgfqpoint{1.362018in}{1.839863in}}%
\pgfpathcurveto{\pgfqpoint{1.348995in}{1.839863in}}{\pgfqpoint{1.336504in}{1.834689in}}{\pgfqpoint{1.327295in}{1.825480in}}%
\pgfpathcurveto{\pgfqpoint{1.318087in}{1.816272in}}{\pgfqpoint{1.312913in}{1.803781in}}{\pgfqpoint{1.312913in}{1.790758in}}%
\pgfpathcurveto{\pgfqpoint{1.312913in}{1.777735in}}{\pgfqpoint{1.318087in}{1.765244in}}{\pgfqpoint{1.327295in}{1.756036in}}%
\pgfpathcurveto{\pgfqpoint{1.336504in}{1.746827in}}{\pgfqpoint{1.348995in}{1.741653in}}{\pgfqpoint{1.362018in}{1.741653in}}%
\pgfpathlineto{\pgfqpoint{1.362018in}{1.741653in}}%
\pgfpathclose%
\pgfusepath{stroke,fill}%
\end{pgfscope}%
\begin{pgfscope}%
\pgfpathrectangle{\pgfqpoint{0.786164in}{0.768110in}}{\pgfqpoint{8.851069in}{7.081890in}}%
\pgfusepath{clip}%
\pgfsetbuttcap%
\pgfsetroundjoin%
\definecolor{currentfill}{rgb}{0.124780,0.640461,0.527068}%
\pgfsetfillcolor{currentfill}%
\pgfsetfillopacity{0.700000}%
\pgfsetlinewidth{0.501875pt}%
\definecolor{currentstroke}{rgb}{1.000000,1.000000,1.000000}%
\pgfsetstrokecolor{currentstroke}%
\pgfsetstrokeopacity{0.700000}%
\pgfsetdash{}{0pt}%
\pgfpathmoveto{\pgfqpoint{4.997066in}{1.544569in}}%
\pgfpathcurveto{\pgfqpoint{5.010089in}{1.544569in}}{\pgfqpoint{5.022580in}{1.549743in}}{\pgfqpoint{5.031788in}{1.558952in}}%
\pgfpathcurveto{\pgfqpoint{5.040997in}{1.568160in}}{\pgfqpoint{5.046171in}{1.580651in}}{\pgfqpoint{5.046171in}{1.593674in}}%
\pgfpathcurveto{\pgfqpoint{5.046171in}{1.606697in}}{\pgfqpoint{5.040997in}{1.619188in}}{\pgfqpoint{5.031788in}{1.628396in}}%
\pgfpathcurveto{\pgfqpoint{5.022580in}{1.637605in}}{\pgfqpoint{5.010089in}{1.642779in}}{\pgfqpoint{4.997066in}{1.642779in}}%
\pgfpathcurveto{\pgfqpoint{4.984043in}{1.642779in}}{\pgfqpoint{4.971552in}{1.637605in}}{\pgfqpoint{4.962344in}{1.628396in}}%
\pgfpathcurveto{\pgfqpoint{4.953135in}{1.619188in}}{\pgfqpoint{4.947961in}{1.606697in}}{\pgfqpoint{4.947961in}{1.593674in}}%
\pgfpathcurveto{\pgfqpoint{4.947961in}{1.580651in}}{\pgfqpoint{4.953135in}{1.568160in}}{\pgfqpoint{4.962344in}{1.558952in}}%
\pgfpathcurveto{\pgfqpoint{4.971552in}{1.549743in}}{\pgfqpoint{4.984043in}{1.544569in}}{\pgfqpoint{4.997066in}{1.544569in}}%
\pgfpathlineto{\pgfqpoint{4.997066in}{1.544569in}}%
\pgfpathclose%
\pgfusepath{stroke,fill}%
\end{pgfscope}%
\begin{pgfscope}%
\pgfpathrectangle{\pgfqpoint{0.786164in}{0.768110in}}{\pgfqpoint{8.851069in}{7.081890in}}%
\pgfusepath{clip}%
\pgfsetbuttcap%
\pgfsetroundjoin%
\definecolor{currentfill}{rgb}{0.137339,0.662252,0.515571}%
\pgfsetfillcolor{currentfill}%
\pgfsetfillopacity{0.700000}%
\pgfsetlinewidth{0.501875pt}%
\definecolor{currentstroke}{rgb}{1.000000,1.000000,1.000000}%
\pgfsetstrokecolor{currentstroke}%
\pgfsetstrokeopacity{0.700000}%
\pgfsetdash{}{0pt}%
\pgfpathmoveto{\pgfqpoint{4.832667in}{1.435078in}}%
\pgfpathcurveto{\pgfqpoint{4.845690in}{1.435078in}}{\pgfqpoint{4.858181in}{1.440252in}}{\pgfqpoint{4.867389in}{1.449461in}}%
\pgfpathcurveto{\pgfqpoint{4.876598in}{1.458669in}}{\pgfqpoint{4.881772in}{1.471160in}}{\pgfqpoint{4.881772in}{1.484183in}}%
\pgfpathcurveto{\pgfqpoint{4.881772in}{1.497205in}}{\pgfqpoint{4.876598in}{1.509697in}}{\pgfqpoint{4.867389in}{1.518905in}}%
\pgfpathcurveto{\pgfqpoint{4.858181in}{1.528113in}}{\pgfqpoint{4.845690in}{1.533287in}}{\pgfqpoint{4.832667in}{1.533287in}}%
\pgfpathcurveto{\pgfqpoint{4.819644in}{1.533287in}}{\pgfqpoint{4.807153in}{1.528113in}}{\pgfqpoint{4.797945in}{1.518905in}}%
\pgfpathcurveto{\pgfqpoint{4.788736in}{1.509697in}}{\pgfqpoint{4.783562in}{1.497205in}}{\pgfqpoint{4.783562in}{1.484183in}}%
\pgfpathcurveto{\pgfqpoint{4.783562in}{1.471160in}}{\pgfqpoint{4.788736in}{1.458669in}}{\pgfqpoint{4.797945in}{1.449461in}}%
\pgfpathcurveto{\pgfqpoint{4.807153in}{1.440252in}}{\pgfqpoint{4.819644in}{1.435078in}}{\pgfqpoint{4.832667in}{1.435078in}}%
\pgfpathlineto{\pgfqpoint{4.832667in}{1.435078in}}%
\pgfpathclose%
\pgfusepath{stroke,fill}%
\end{pgfscope}%
\begin{pgfscope}%
\pgfpathrectangle{\pgfqpoint{0.786164in}{0.768110in}}{\pgfqpoint{8.851069in}{7.081890in}}%
\pgfusepath{clip}%
\pgfsetbuttcap%
\pgfsetroundjoin%
\definecolor{currentfill}{rgb}{0.162016,0.687316,0.499129}%
\pgfsetfillcolor{currentfill}%
\pgfsetfillopacity{0.700000}%
\pgfsetlinewidth{0.501875pt}%
\definecolor{currentstroke}{rgb}{1.000000,1.000000,1.000000}%
\pgfsetstrokecolor{currentstroke}%
\pgfsetstrokeopacity{0.700000}%
\pgfsetdash{}{0pt}%
\pgfpathmoveto{\pgfqpoint{4.668268in}{1.325587in}}%
\pgfpathcurveto{\pgfqpoint{4.681290in}{1.325587in}}{\pgfqpoint{4.693782in}{1.330761in}}{\pgfqpoint{4.702990in}{1.339969in}}%
\pgfpathcurveto{\pgfqpoint{4.712198in}{1.349178in}}{\pgfqpoint{4.717372in}{1.361669in}}{\pgfqpoint{4.717372in}{1.374692in}}%
\pgfpathcurveto{\pgfqpoint{4.717372in}{1.387714in}}{\pgfqpoint{4.712198in}{1.400205in}}{\pgfqpoint{4.702990in}{1.409414in}}%
\pgfpathcurveto{\pgfqpoint{4.693782in}{1.418622in}}{\pgfqpoint{4.681290in}{1.423796in}}{\pgfqpoint{4.668268in}{1.423796in}}%
\pgfpathcurveto{\pgfqpoint{4.655245in}{1.423796in}}{\pgfqpoint{4.642754in}{1.418622in}}{\pgfqpoint{4.633546in}{1.409414in}}%
\pgfpathcurveto{\pgfqpoint{4.624337in}{1.400205in}}{\pgfqpoint{4.619163in}{1.387714in}}{\pgfqpoint{4.619163in}{1.374692in}}%
\pgfpathcurveto{\pgfqpoint{4.619163in}{1.361669in}}{\pgfqpoint{4.624337in}{1.349178in}}{\pgfqpoint{4.633546in}{1.339969in}}%
\pgfpathcurveto{\pgfqpoint{4.642754in}{1.330761in}}{\pgfqpoint{4.655245in}{1.325587in}}{\pgfqpoint{4.668268in}{1.325587in}}%
\pgfpathlineto{\pgfqpoint{4.668268in}{1.325587in}}%
\pgfpathclose%
\pgfusepath{stroke,fill}%
\end{pgfscope}%
\begin{pgfscope}%
\pgfpathrectangle{\pgfqpoint{0.786164in}{0.768110in}}{\pgfqpoint{8.851069in}{7.081890in}}%
\pgfusepath{clip}%
\pgfsetbuttcap%
\pgfsetroundjoin%
\definecolor{currentfill}{rgb}{0.191090,0.708366,0.482284}%
\pgfsetfillcolor{currentfill}%
\pgfsetfillopacity{0.700000}%
\pgfsetlinewidth{0.501875pt}%
\definecolor{currentstroke}{rgb}{1.000000,1.000000,1.000000}%
\pgfsetstrokecolor{currentstroke}%
\pgfsetstrokeopacity{0.700000}%
\pgfsetdash{}{0pt}%
\pgfpathmoveto{\pgfqpoint{4.604335in}{1.303689in}}%
\pgfpathcurveto{\pgfqpoint{4.617357in}{1.303689in}}{\pgfqpoint{4.629848in}{1.308863in}}{\pgfqpoint{4.639057in}{1.318071in}}%
\pgfpathcurveto{\pgfqpoint{4.648265in}{1.327280in}}{\pgfqpoint{4.653439in}{1.339771in}}{\pgfqpoint{4.653439in}{1.352793in}}%
\pgfpathcurveto{\pgfqpoint{4.653439in}{1.365816in}}{\pgfqpoint{4.648265in}{1.378307in}}{\pgfqpoint{4.639057in}{1.387516in}}%
\pgfpathcurveto{\pgfqpoint{4.629848in}{1.396724in}}{\pgfqpoint{4.617357in}{1.401898in}}{\pgfqpoint{4.604335in}{1.401898in}}%
\pgfpathcurveto{\pgfqpoint{4.591312in}{1.401898in}}{\pgfqpoint{4.578821in}{1.396724in}}{\pgfqpoint{4.569612in}{1.387516in}}%
\pgfpathcurveto{\pgfqpoint{4.560404in}{1.378307in}}{\pgfqpoint{4.555230in}{1.365816in}}{\pgfqpoint{4.555230in}{1.352793in}}%
\pgfpathcurveto{\pgfqpoint{4.555230in}{1.339771in}}{\pgfqpoint{4.560404in}{1.327280in}}{\pgfqpoint{4.569612in}{1.318071in}}%
\pgfpathcurveto{\pgfqpoint{4.578821in}{1.308863in}}{\pgfqpoint{4.591312in}{1.303689in}}{\pgfqpoint{4.604335in}{1.303689in}}%
\pgfpathlineto{\pgfqpoint{4.604335in}{1.303689in}}%
\pgfpathclose%
\pgfusepath{stroke,fill}%
\end{pgfscope}%
\begin{pgfscope}%
\pgfpathrectangle{\pgfqpoint{0.786164in}{0.768110in}}{\pgfqpoint{8.851069in}{7.081890in}}%
\pgfusepath{clip}%
\pgfsetbuttcap%
\pgfsetroundjoin%
\definecolor{currentfill}{rgb}{0.208030,0.718701,0.472873}%
\pgfsetfillcolor{currentfill}%
\pgfsetfillopacity{0.700000}%
\pgfsetlinewidth{0.501875pt}%
\definecolor{currentstroke}{rgb}{1.000000,1.000000,1.000000}%
\pgfsetstrokecolor{currentstroke}%
\pgfsetstrokeopacity{0.700000}%
\pgfsetdash{}{0pt}%
\pgfpathmoveto{\pgfqpoint{4.549535in}{1.259892in}}%
\pgfpathcurveto{\pgfqpoint{4.562558in}{1.259892in}}{\pgfqpoint{4.575049in}{1.265066in}}{\pgfqpoint{4.584257in}{1.274275in}}%
\pgfpathcurveto{\pgfqpoint{4.593466in}{1.283483in}}{\pgfqpoint{4.598640in}{1.295974in}}{\pgfqpoint{4.598640in}{1.308997in}}%
\pgfpathcurveto{\pgfqpoint{4.598640in}{1.322020in}}{\pgfqpoint{4.593466in}{1.334511in}}{\pgfqpoint{4.584257in}{1.343719in}}%
\pgfpathcurveto{\pgfqpoint{4.575049in}{1.352928in}}{\pgfqpoint{4.562558in}{1.358101in}}{\pgfqpoint{4.549535in}{1.358101in}}%
\pgfpathcurveto{\pgfqpoint{4.536512in}{1.358101in}}{\pgfqpoint{4.524021in}{1.352928in}}{\pgfqpoint{4.514813in}{1.343719in}}%
\pgfpathcurveto{\pgfqpoint{4.505604in}{1.334511in}}{\pgfqpoint{4.500430in}{1.322020in}}{\pgfqpoint{4.500430in}{1.308997in}}%
\pgfpathcurveto{\pgfqpoint{4.500430in}{1.295974in}}{\pgfqpoint{4.505604in}{1.283483in}}{\pgfqpoint{4.514813in}{1.274275in}}%
\pgfpathcurveto{\pgfqpoint{4.524021in}{1.265066in}}{\pgfqpoint{4.536512in}{1.259892in}}{\pgfqpoint{4.549535in}{1.259892in}}%
\pgfpathlineto{\pgfqpoint{4.549535in}{1.259892in}}%
\pgfpathclose%
\pgfusepath{stroke,fill}%
\end{pgfscope}%
\begin{pgfscope}%
\pgfpathrectangle{\pgfqpoint{0.786164in}{0.768110in}}{\pgfqpoint{8.851069in}{7.081890in}}%
\pgfusepath{clip}%
\pgfsetbuttcap%
\pgfsetroundjoin%
\definecolor{currentfill}{rgb}{0.288921,0.758394,0.428426}%
\pgfsetfillcolor{currentfill}%
\pgfsetfillopacity{0.700000}%
\pgfsetlinewidth{0.501875pt}%
\definecolor{currentstroke}{rgb}{1.000000,1.000000,1.000000}%
\pgfsetstrokecolor{currentstroke}%
\pgfsetstrokeopacity{0.700000}%
\pgfsetdash{}{0pt}%
\pgfpathmoveto{\pgfqpoint{4.102004in}{1.172299in}}%
\pgfpathcurveto{\pgfqpoint{4.115027in}{1.172299in}}{\pgfqpoint{4.127518in}{1.177473in}}{\pgfqpoint{4.136726in}{1.186682in}}%
\pgfpathcurveto{\pgfqpoint{4.145935in}{1.195890in}}{\pgfqpoint{4.151109in}{1.208381in}}{\pgfqpoint{4.151109in}{1.221404in}}%
\pgfpathcurveto{\pgfqpoint{4.151109in}{1.234427in}}{\pgfqpoint{4.145935in}{1.246918in}}{\pgfqpoint{4.136726in}{1.256126in}}%
\pgfpathcurveto{\pgfqpoint{4.127518in}{1.265335in}}{\pgfqpoint{4.115027in}{1.270509in}}{\pgfqpoint{4.102004in}{1.270509in}}%
\pgfpathcurveto{\pgfqpoint{4.088981in}{1.270509in}}{\pgfqpoint{4.076490in}{1.265335in}}{\pgfqpoint{4.067282in}{1.256126in}}%
\pgfpathcurveto{\pgfqpoint{4.058073in}{1.246918in}}{\pgfqpoint{4.052899in}{1.234427in}}{\pgfqpoint{4.052899in}{1.221404in}}%
\pgfpathcurveto{\pgfqpoint{4.052899in}{1.208381in}}{\pgfqpoint{4.058073in}{1.195890in}}{\pgfqpoint{4.067282in}{1.186682in}}%
\pgfpathcurveto{\pgfqpoint{4.076490in}{1.177473in}}{\pgfqpoint{4.088981in}{1.172299in}}{\pgfqpoint{4.102004in}{1.172299in}}%
\pgfpathlineto{\pgfqpoint{4.102004in}{1.172299in}}%
\pgfpathclose%
\pgfusepath{stroke,fill}%
\end{pgfscope}%
\begin{pgfscope}%
\pgfpathrectangle{\pgfqpoint{0.786164in}{0.768110in}}{\pgfqpoint{8.851069in}{7.081890in}}%
\pgfusepath{clip}%
\pgfsetbuttcap%
\pgfsetroundjoin%
\definecolor{currentfill}{rgb}{0.259857,0.745492,0.444467}%
\pgfsetfillcolor{currentfill}%
\pgfsetfillopacity{0.700000}%
\pgfsetlinewidth{0.501875pt}%
\definecolor{currentstroke}{rgb}{1.000000,1.000000,1.000000}%
\pgfsetstrokecolor{currentstroke}%
\pgfsetstrokeopacity{0.700000}%
\pgfsetdash{}{0pt}%
\pgfpathmoveto{\pgfqpoint{4.576935in}{1.237994in}}%
\pgfpathcurveto{\pgfqpoint{4.589958in}{1.237994in}}{\pgfqpoint{4.602449in}{1.243168in}}{\pgfqpoint{4.611657in}{1.252376in}}%
\pgfpathcurveto{\pgfqpoint{4.620866in}{1.261585in}}{\pgfqpoint{4.626039in}{1.274076in}}{\pgfqpoint{4.626039in}{1.287099in}}%
\pgfpathcurveto{\pgfqpoint{4.626039in}{1.300121in}}{\pgfqpoint{4.620866in}{1.312612in}}{\pgfqpoint{4.611657in}{1.321821in}}%
\pgfpathcurveto{\pgfqpoint{4.602449in}{1.331029in}}{\pgfqpoint{4.589958in}{1.336203in}}{\pgfqpoint{4.576935in}{1.336203in}}%
\pgfpathcurveto{\pgfqpoint{4.563912in}{1.336203in}}{\pgfqpoint{4.551421in}{1.331029in}}{\pgfqpoint{4.542213in}{1.321821in}}%
\pgfpathcurveto{\pgfqpoint{4.533004in}{1.312612in}}{\pgfqpoint{4.527830in}{1.300121in}}{\pgfqpoint{4.527830in}{1.287099in}}%
\pgfpathcurveto{\pgfqpoint{4.527830in}{1.274076in}}{\pgfqpoint{4.533004in}{1.261585in}}{\pgfqpoint{4.542213in}{1.252376in}}%
\pgfpathcurveto{\pgfqpoint{4.551421in}{1.243168in}}{\pgfqpoint{4.563912in}{1.237994in}}{\pgfqpoint{4.576935in}{1.237994in}}%
\pgfpathlineto{\pgfqpoint{4.576935in}{1.237994in}}%
\pgfpathclose%
\pgfusepath{stroke,fill}%
\end{pgfscope}%
\begin{pgfscope}%
\pgfpathrectangle{\pgfqpoint{0.786164in}{0.768110in}}{\pgfqpoint{8.851069in}{7.081890in}}%
\pgfusepath{clip}%
\pgfsetbuttcap%
\pgfsetroundjoin%
\definecolor{currentfill}{rgb}{0.304148,0.764704,0.419943}%
\pgfsetfillcolor{currentfill}%
\pgfsetfillopacity{0.700000}%
\pgfsetlinewidth{0.501875pt}%
\definecolor{currentstroke}{rgb}{1.000000,1.000000,1.000000}%
\pgfsetstrokecolor{currentstroke}%
\pgfsetstrokeopacity{0.700000}%
\pgfsetdash{}{0pt}%
\pgfpathmoveto{\pgfqpoint{4.467335in}{1.172299in}}%
\pgfpathcurveto{\pgfqpoint{4.480358in}{1.172299in}}{\pgfqpoint{4.492849in}{1.177473in}}{\pgfqpoint{4.502058in}{1.186682in}}%
\pgfpathcurveto{\pgfqpoint{4.511266in}{1.195890in}}{\pgfqpoint{4.516440in}{1.208381in}}{\pgfqpoint{4.516440in}{1.221404in}}%
\pgfpathcurveto{\pgfqpoint{4.516440in}{1.234427in}}{\pgfqpoint{4.511266in}{1.246918in}}{\pgfqpoint{4.502058in}{1.256126in}}%
\pgfpathcurveto{\pgfqpoint{4.492849in}{1.265335in}}{\pgfqpoint{4.480358in}{1.270509in}}{\pgfqpoint{4.467335in}{1.270509in}}%
\pgfpathcurveto{\pgfqpoint{4.454313in}{1.270509in}}{\pgfqpoint{4.441822in}{1.265335in}}{\pgfqpoint{4.432613in}{1.256126in}}%
\pgfpathcurveto{\pgfqpoint{4.423405in}{1.246918in}}{\pgfqpoint{4.418231in}{1.234427in}}{\pgfqpoint{4.418231in}{1.221404in}}%
\pgfpathcurveto{\pgfqpoint{4.418231in}{1.208381in}}{\pgfqpoint{4.423405in}{1.195890in}}{\pgfqpoint{4.432613in}{1.186682in}}%
\pgfpathcurveto{\pgfqpoint{4.441822in}{1.177473in}}{\pgfqpoint{4.454313in}{1.172299in}}{\pgfqpoint{4.467335in}{1.172299in}}%
\pgfpathlineto{\pgfqpoint{4.467335in}{1.172299in}}%
\pgfpathclose%
\pgfusepath{stroke,fill}%
\end{pgfscope}%
\begin{pgfscope}%
\pgfpathrectangle{\pgfqpoint{0.786164in}{0.768110in}}{\pgfqpoint{8.851069in}{7.081890in}}%
\pgfusepath{clip}%
\pgfsetbuttcap%
\pgfsetroundjoin%
\definecolor{currentfill}{rgb}{0.360741,0.785964,0.387814}%
\pgfsetfillcolor{currentfill}%
\pgfsetfillopacity{0.700000}%
\pgfsetlinewidth{0.501875pt}%
\definecolor{currentstroke}{rgb}{1.000000,1.000000,1.000000}%
\pgfsetstrokecolor{currentstroke}%
\pgfsetstrokeopacity{0.700000}%
\pgfsetdash{}{0pt}%
\pgfpathmoveto{\pgfqpoint{4.421669in}{1.106605in}}%
\pgfpathcurveto{\pgfqpoint{4.434692in}{1.106605in}}{\pgfqpoint{4.447183in}{1.111779in}}{\pgfqpoint{4.456391in}{1.120987in}}%
\pgfpathcurveto{\pgfqpoint{4.465600in}{1.130195in}}{\pgfqpoint{4.470774in}{1.142686in}}{\pgfqpoint{4.470774in}{1.155709in}}%
\pgfpathcurveto{\pgfqpoint{4.470774in}{1.168732in}}{\pgfqpoint{4.465600in}{1.181223in}}{\pgfqpoint{4.456391in}{1.190431in}}%
\pgfpathcurveto{\pgfqpoint{4.447183in}{1.199640in}}{\pgfqpoint{4.434692in}{1.204814in}}{\pgfqpoint{4.421669in}{1.204814in}}%
\pgfpathcurveto{\pgfqpoint{4.408646in}{1.204814in}}{\pgfqpoint{4.396155in}{1.199640in}}{\pgfqpoint{4.386947in}{1.190431in}}%
\pgfpathcurveto{\pgfqpoint{4.377738in}{1.181223in}}{\pgfqpoint{4.372564in}{1.168732in}}{\pgfqpoint{4.372564in}{1.155709in}}%
\pgfpathcurveto{\pgfqpoint{4.372564in}{1.142686in}}{\pgfqpoint{4.377738in}{1.130195in}}{\pgfqpoint{4.386947in}{1.120987in}}%
\pgfpathcurveto{\pgfqpoint{4.396155in}{1.111779in}}{\pgfqpoint{4.408646in}{1.106605in}}{\pgfqpoint{4.421669in}{1.106605in}}%
\pgfpathlineto{\pgfqpoint{4.421669in}{1.106605in}}%
\pgfpathclose%
\pgfusepath{stroke,fill}%
\end{pgfscope}%
\begin{pgfscope}%
\pgfpathrectangle{\pgfqpoint{0.786164in}{0.768110in}}{\pgfqpoint{8.851069in}{7.081890in}}%
\pgfusepath{clip}%
\pgfsetbuttcap%
\pgfsetroundjoin%
\definecolor{currentfill}{rgb}{0.386433,0.794644,0.372886}%
\pgfsetfillcolor{currentfill}%
\pgfsetfillopacity{0.700000}%
\pgfsetlinewidth{0.501875pt}%
\definecolor{currentstroke}{rgb}{1.000000,1.000000,1.000000}%
\pgfsetstrokecolor{currentstroke}%
\pgfsetstrokeopacity{0.700000}%
\pgfsetdash{}{0pt}%
\pgfpathmoveto{\pgfqpoint{4.293803in}{1.040910in}}%
\pgfpathcurveto{\pgfqpoint{4.306826in}{1.040910in}}{\pgfqpoint{4.319317in}{1.046084in}}{\pgfqpoint{4.328525in}{1.055292in}}%
\pgfpathcurveto{\pgfqpoint{4.337734in}{1.064501in}}{\pgfqpoint{4.342908in}{1.076992in}}{\pgfqpoint{4.342908in}{1.090014in}}%
\pgfpathcurveto{\pgfqpoint{4.342908in}{1.103037in}}{\pgfqpoint{4.337734in}{1.115528in}}{\pgfqpoint{4.328525in}{1.124737in}}%
\pgfpathcurveto{\pgfqpoint{4.319317in}{1.133945in}}{\pgfqpoint{4.306826in}{1.139119in}}{\pgfqpoint{4.293803in}{1.139119in}}%
\pgfpathcurveto{\pgfqpoint{4.280780in}{1.139119in}}{\pgfqpoint{4.268289in}{1.133945in}}{\pgfqpoint{4.259081in}{1.124737in}}%
\pgfpathcurveto{\pgfqpoint{4.249872in}{1.115528in}}{\pgfqpoint{4.244698in}{1.103037in}}{\pgfqpoint{4.244698in}{1.090014in}}%
\pgfpathcurveto{\pgfqpoint{4.244698in}{1.076992in}}{\pgfqpoint{4.249872in}{1.064501in}}{\pgfqpoint{4.259081in}{1.055292in}}%
\pgfpathcurveto{\pgfqpoint{4.268289in}{1.046084in}}{\pgfqpoint{4.280780in}{1.040910in}}{\pgfqpoint{4.293803in}{1.040910in}}%
\pgfpathlineto{\pgfqpoint{4.293803in}{1.040910in}}%
\pgfpathclose%
\pgfusepath{stroke,fill}%
\end{pgfscope}%
\begin{pgfscope}%
\pgfpathrectangle{\pgfqpoint{0.786164in}{0.768110in}}{\pgfqpoint{8.851069in}{7.081890in}}%
\pgfusepath{clip}%
\pgfsetbuttcap%
\pgfsetroundjoin%
\definecolor{currentfill}{rgb}{0.421908,0.805774,0.351910}%
\pgfsetfillcolor{currentfill}%
\pgfsetfillopacity{0.700000}%
\pgfsetlinewidth{0.501875pt}%
\definecolor{currentstroke}{rgb}{1.000000,1.000000,1.000000}%
\pgfsetstrokecolor{currentstroke}%
\pgfsetstrokeopacity{0.700000}%
\pgfsetdash{}{0pt}%
\pgfpathmoveto{\pgfqpoint{4.202470in}{1.040910in}}%
\pgfpathcurveto{\pgfqpoint{4.215493in}{1.040910in}}{\pgfqpoint{4.227984in}{1.046084in}}{\pgfqpoint{4.237192in}{1.055292in}}%
\pgfpathcurveto{\pgfqpoint{4.246401in}{1.064501in}}{\pgfqpoint{4.251575in}{1.076992in}}{\pgfqpoint{4.251575in}{1.090014in}}%
\pgfpathcurveto{\pgfqpoint{4.251575in}{1.103037in}}{\pgfqpoint{4.246401in}{1.115528in}}{\pgfqpoint{4.237192in}{1.124737in}}%
\pgfpathcurveto{\pgfqpoint{4.227984in}{1.133945in}}{\pgfqpoint{4.215493in}{1.139119in}}{\pgfqpoint{4.202470in}{1.139119in}}%
\pgfpathcurveto{\pgfqpoint{4.189447in}{1.139119in}}{\pgfqpoint{4.176956in}{1.133945in}}{\pgfqpoint{4.167748in}{1.124737in}}%
\pgfpathcurveto{\pgfqpoint{4.158539in}{1.115528in}}{\pgfqpoint{4.153365in}{1.103037in}}{\pgfqpoint{4.153365in}{1.090014in}}%
\pgfpathcurveto{\pgfqpoint{4.153365in}{1.076992in}}{\pgfqpoint{4.158539in}{1.064501in}}{\pgfqpoint{4.167748in}{1.055292in}}%
\pgfpathcurveto{\pgfqpoint{4.176956in}{1.046084in}}{\pgfqpoint{4.189447in}{1.040910in}}{\pgfqpoint{4.202470in}{1.040910in}}%
\pgfpathlineto{\pgfqpoint{4.202470in}{1.040910in}}%
\pgfpathclose%
\pgfusepath{stroke,fill}%
\end{pgfscope}%
\begin{pgfscope}%
\pgfpathrectangle{\pgfqpoint{0.786164in}{0.768110in}}{\pgfqpoint{8.851069in}{7.081890in}}%
\pgfusepath{clip}%
\pgfsetbuttcap%
\pgfsetroundjoin%
\definecolor{currentfill}{rgb}{0.458674,0.816363,0.329727}%
\pgfsetfillcolor{currentfill}%
\pgfsetfillopacity{0.700000}%
\pgfsetlinewidth{0.501875pt}%
\definecolor{currentstroke}{rgb}{1.000000,1.000000,1.000000}%
\pgfsetstrokecolor{currentstroke}%
\pgfsetstrokeopacity{0.700000}%
\pgfsetdash{}{0pt}%
\pgfpathmoveto{\pgfqpoint{3.955871in}{1.084706in}}%
\pgfpathcurveto{\pgfqpoint{3.968894in}{1.084706in}}{\pgfqpoint{3.981385in}{1.089880in}}{\pgfqpoint{3.990594in}{1.099089in}}%
\pgfpathcurveto{\pgfqpoint{3.999802in}{1.108297in}}{\pgfqpoint{4.004976in}{1.120788in}}{\pgfqpoint{4.004976in}{1.133811in}}%
\pgfpathcurveto{\pgfqpoint{4.004976in}{1.146834in}}{\pgfqpoint{3.999802in}{1.159325in}}{\pgfqpoint{3.990594in}{1.168533in}}%
\pgfpathcurveto{\pgfqpoint{3.981385in}{1.177742in}}{\pgfqpoint{3.968894in}{1.182916in}}{\pgfqpoint{3.955871in}{1.182916in}}%
\pgfpathcurveto{\pgfqpoint{3.942849in}{1.182916in}}{\pgfqpoint{3.930358in}{1.177742in}}{\pgfqpoint{3.921149in}{1.168533in}}%
\pgfpathcurveto{\pgfqpoint{3.911941in}{1.159325in}}{\pgfqpoint{3.906767in}{1.146834in}}{\pgfqpoint{3.906767in}{1.133811in}}%
\pgfpathcurveto{\pgfqpoint{3.906767in}{1.120788in}}{\pgfqpoint{3.911941in}{1.108297in}}{\pgfqpoint{3.921149in}{1.099089in}}%
\pgfpathcurveto{\pgfqpoint{3.930358in}{1.089880in}}{\pgfqpoint{3.942849in}{1.084706in}}{\pgfqpoint{3.955871in}{1.084706in}}%
\pgfpathlineto{\pgfqpoint{3.955871in}{1.084706in}}%
\pgfpathclose%
\pgfusepath{stroke,fill}%
\end{pgfscope}%
\begin{pgfscope}%
\pgfpathrectangle{\pgfqpoint{0.786164in}{0.768110in}}{\pgfqpoint{8.851069in}{7.081890in}}%
\pgfusepath{clip}%
\pgfsetbuttcap%
\pgfsetroundjoin%
\definecolor{currentfill}{rgb}{0.468053,0.818921,0.323998}%
\pgfsetfillcolor{currentfill}%
\pgfsetfillopacity{0.700000}%
\pgfsetlinewidth{0.501875pt}%
\definecolor{currentstroke}{rgb}{1.000000,1.000000,1.000000}%
\pgfsetstrokecolor{currentstroke}%
\pgfsetstrokeopacity{0.700000}%
\pgfsetdash{}{0pt}%
\pgfpathmoveto{\pgfqpoint{4.047204in}{1.128503in}}%
\pgfpathcurveto{\pgfqpoint{4.060227in}{1.128503in}}{\pgfqpoint{4.072718in}{1.133677in}}{\pgfqpoint{4.081926in}{1.142885in}}%
\pgfpathcurveto{\pgfqpoint{4.091135in}{1.152094in}}{\pgfqpoint{4.096309in}{1.164585in}}{\pgfqpoint{4.096309in}{1.177607in}}%
\pgfpathcurveto{\pgfqpoint{4.096309in}{1.190630in}}{\pgfqpoint{4.091135in}{1.203121in}}{\pgfqpoint{4.081926in}{1.212330in}}%
\pgfpathcurveto{\pgfqpoint{4.072718in}{1.221538in}}{\pgfqpoint{4.060227in}{1.226712in}}{\pgfqpoint{4.047204in}{1.226712in}}%
\pgfpathcurveto{\pgfqpoint{4.034181in}{1.226712in}}{\pgfqpoint{4.021690in}{1.221538in}}{\pgfqpoint{4.012482in}{1.212330in}}%
\pgfpathcurveto{\pgfqpoint{4.003274in}{1.203121in}}{\pgfqpoint{3.998100in}{1.190630in}}{\pgfqpoint{3.998100in}{1.177607in}}%
\pgfpathcurveto{\pgfqpoint{3.998100in}{1.164585in}}{\pgfqpoint{4.003274in}{1.152094in}}{\pgfqpoint{4.012482in}{1.142885in}}%
\pgfpathcurveto{\pgfqpoint{4.021690in}{1.133677in}}{\pgfqpoint{4.034181in}{1.128503in}}{\pgfqpoint{4.047204in}{1.128503in}}%
\pgfpathlineto{\pgfqpoint{4.047204in}{1.128503in}}%
\pgfpathclose%
\pgfusepath{stroke,fill}%
\end{pgfscope}%
\begin{pgfscope}%
\pgfpathrectangle{\pgfqpoint{0.786164in}{0.768110in}}{\pgfqpoint{8.851069in}{7.081890in}}%
\pgfusepath{clip}%
\pgfsetbuttcap%
\pgfsetroundjoin%
\definecolor{currentfill}{rgb}{0.506271,0.828786,0.300362}%
\pgfsetfillcolor{currentfill}%
\pgfsetfillopacity{0.700000}%
\pgfsetlinewidth{0.501875pt}%
\definecolor{currentstroke}{rgb}{1.000000,1.000000,1.000000}%
\pgfsetstrokecolor{currentstroke}%
\pgfsetstrokeopacity{0.700000}%
\pgfsetdash{}{0pt}%
\pgfpathmoveto{\pgfqpoint{4.083737in}{1.237994in}}%
\pgfpathcurveto{\pgfqpoint{4.096760in}{1.237994in}}{\pgfqpoint{4.109251in}{1.243168in}}{\pgfqpoint{4.118460in}{1.252376in}}%
\pgfpathcurveto{\pgfqpoint{4.127668in}{1.261585in}}{\pgfqpoint{4.132842in}{1.274076in}}{\pgfqpoint{4.132842in}{1.287099in}}%
\pgfpathcurveto{\pgfqpoint{4.132842in}{1.300121in}}{\pgfqpoint{4.127668in}{1.312612in}}{\pgfqpoint{4.118460in}{1.321821in}}%
\pgfpathcurveto{\pgfqpoint{4.109251in}{1.331029in}}{\pgfqpoint{4.096760in}{1.336203in}}{\pgfqpoint{4.083737in}{1.336203in}}%
\pgfpathcurveto{\pgfqpoint{4.070715in}{1.336203in}}{\pgfqpoint{4.058224in}{1.331029in}}{\pgfqpoint{4.049015in}{1.321821in}}%
\pgfpathcurveto{\pgfqpoint{4.039807in}{1.312612in}}{\pgfqpoint{4.034633in}{1.300121in}}{\pgfqpoint{4.034633in}{1.287099in}}%
\pgfpathcurveto{\pgfqpoint{4.034633in}{1.274076in}}{\pgfqpoint{4.039807in}{1.261585in}}{\pgfqpoint{4.049015in}{1.252376in}}%
\pgfpathcurveto{\pgfqpoint{4.058224in}{1.243168in}}{\pgfqpoint{4.070715in}{1.237994in}}{\pgfqpoint{4.083737in}{1.237994in}}%
\pgfpathlineto{\pgfqpoint{4.083737in}{1.237994in}}%
\pgfpathclose%
\pgfusepath{stroke,fill}%
\end{pgfscope}%
\begin{pgfscope}%
\pgfpathrectangle{\pgfqpoint{0.786164in}{0.768110in}}{\pgfqpoint{8.851069in}{7.081890in}}%
\pgfusepath{clip}%
\pgfsetbuttcap%
\pgfsetroundjoin%
\definecolor{currentfill}{rgb}{0.525776,0.833491,0.288127}%
\pgfsetfillcolor{currentfill}%
\pgfsetfillopacity{0.700000}%
\pgfsetlinewidth{0.501875pt}%
\definecolor{currentstroke}{rgb}{1.000000,1.000000,1.000000}%
\pgfsetstrokecolor{currentstroke}%
\pgfsetstrokeopacity{0.700000}%
\pgfsetdash{}{0pt}%
\pgfpathmoveto{\pgfqpoint{4.111137in}{1.369383in}}%
\pgfpathcurveto{\pgfqpoint{4.124160in}{1.369383in}}{\pgfqpoint{4.136651in}{1.374557in}}{\pgfqpoint{4.145859in}{1.383766in}}%
\pgfpathcurveto{\pgfqpoint{4.155068in}{1.392974in}}{\pgfqpoint{4.160242in}{1.405465in}}{\pgfqpoint{4.160242in}{1.418488in}}%
\pgfpathcurveto{\pgfqpoint{4.160242in}{1.431511in}}{\pgfqpoint{4.155068in}{1.444002in}}{\pgfqpoint{4.145859in}{1.453210in}}%
\pgfpathcurveto{\pgfqpoint{4.136651in}{1.462419in}}{\pgfqpoint{4.124160in}{1.467593in}}{\pgfqpoint{4.111137in}{1.467593in}}%
\pgfpathcurveto{\pgfqpoint{4.098114in}{1.467593in}}{\pgfqpoint{4.085623in}{1.462419in}}{\pgfqpoint{4.076415in}{1.453210in}}%
\pgfpathcurveto{\pgfqpoint{4.067207in}{1.444002in}}{\pgfqpoint{4.062033in}{1.431511in}}{\pgfqpoint{4.062033in}{1.418488in}}%
\pgfpathcurveto{\pgfqpoint{4.062033in}{1.405465in}}{\pgfqpoint{4.067207in}{1.392974in}}{\pgfqpoint{4.076415in}{1.383766in}}%
\pgfpathcurveto{\pgfqpoint{4.085623in}{1.374557in}}{\pgfqpoint{4.098114in}{1.369383in}}{\pgfqpoint{4.111137in}{1.369383in}}%
\pgfpathlineto{\pgfqpoint{4.111137in}{1.369383in}}%
\pgfpathclose%
\pgfusepath{stroke,fill}%
\end{pgfscope}%
\begin{pgfscope}%
\pgfpathrectangle{\pgfqpoint{0.786164in}{0.768110in}}{\pgfqpoint{8.851069in}{7.081890in}}%
\pgfusepath{clip}%
\pgfsetbuttcap%
\pgfsetroundjoin%
\definecolor{currentfill}{rgb}{0.595839,0.848717,0.243329}%
\pgfsetfillcolor{currentfill}%
\pgfsetfillopacity{0.700000}%
\pgfsetlinewidth{0.501875pt}%
\definecolor{currentstroke}{rgb}{1.000000,1.000000,1.000000}%
\pgfsetstrokecolor{currentstroke}%
\pgfsetstrokeopacity{0.700000}%
\pgfsetdash{}{0pt}%
\pgfpathmoveto{\pgfqpoint{3.937605in}{1.281790in}}%
\pgfpathcurveto{\pgfqpoint{3.950627in}{1.281790in}}{\pgfqpoint{3.963119in}{1.286964in}}{\pgfqpoint{3.972327in}{1.296173in}}%
\pgfpathcurveto{\pgfqpoint{3.981535in}{1.305381in}}{\pgfqpoint{3.986709in}{1.317872in}}{\pgfqpoint{3.986709in}{1.330895in}}%
\pgfpathcurveto{\pgfqpoint{3.986709in}{1.343918in}}{\pgfqpoint{3.981535in}{1.356409in}}{\pgfqpoint{3.972327in}{1.365617in}}%
\pgfpathcurveto{\pgfqpoint{3.963119in}{1.374826in}}{\pgfqpoint{3.950627in}{1.380000in}}{\pgfqpoint{3.937605in}{1.380000in}}%
\pgfpathcurveto{\pgfqpoint{3.924582in}{1.380000in}}{\pgfqpoint{3.912091in}{1.374826in}}{\pgfqpoint{3.902883in}{1.365617in}}%
\pgfpathcurveto{\pgfqpoint{3.893674in}{1.356409in}}{\pgfqpoint{3.888500in}{1.343918in}}{\pgfqpoint{3.888500in}{1.330895in}}%
\pgfpathcurveto{\pgfqpoint{3.888500in}{1.317872in}}{\pgfqpoint{3.893674in}{1.305381in}}{\pgfqpoint{3.902883in}{1.296173in}}%
\pgfpathcurveto{\pgfqpoint{3.912091in}{1.286964in}}{\pgfqpoint{3.924582in}{1.281790in}}{\pgfqpoint{3.937605in}{1.281790in}}%
\pgfpathlineto{\pgfqpoint{3.937605in}{1.281790in}}%
\pgfpathclose%
\pgfusepath{stroke,fill}%
\end{pgfscope}%
\begin{pgfscope}%
\pgfpathrectangle{\pgfqpoint{0.786164in}{0.768110in}}{\pgfqpoint{8.851069in}{7.081890in}}%
\pgfusepath{clip}%
\pgfsetbuttcap%
\pgfsetroundjoin%
\definecolor{currentfill}{rgb}{0.772852,0.877868,0.131109}%
\pgfsetfillcolor{currentfill}%
\pgfsetfillopacity{0.700000}%
\pgfsetlinewidth{0.501875pt}%
\definecolor{currentstroke}{rgb}{1.000000,1.000000,1.000000}%
\pgfsetstrokecolor{currentstroke}%
\pgfsetstrokeopacity{0.700000}%
\pgfsetdash{}{0pt}%
\pgfpathmoveto{\pgfqpoint{3.517474in}{1.106605in}}%
\pgfpathcurveto{\pgfqpoint{3.530496in}{1.106605in}}{\pgfqpoint{3.542987in}{1.111779in}}{\pgfqpoint{3.552196in}{1.120987in}}%
\pgfpathcurveto{\pgfqpoint{3.561404in}{1.130195in}}{\pgfqpoint{3.566578in}{1.142686in}}{\pgfqpoint{3.566578in}{1.155709in}}%
\pgfpathcurveto{\pgfqpoint{3.566578in}{1.168732in}}{\pgfqpoint{3.561404in}{1.181223in}}{\pgfqpoint{3.552196in}{1.190431in}}%
\pgfpathcurveto{\pgfqpoint{3.542987in}{1.199640in}}{\pgfqpoint{3.530496in}{1.204814in}}{\pgfqpoint{3.517474in}{1.204814in}}%
\pgfpathcurveto{\pgfqpoint{3.504451in}{1.204814in}}{\pgfqpoint{3.491960in}{1.199640in}}{\pgfqpoint{3.482751in}{1.190431in}}%
\pgfpathcurveto{\pgfqpoint{3.473543in}{1.181223in}}{\pgfqpoint{3.468369in}{1.168732in}}{\pgfqpoint{3.468369in}{1.155709in}}%
\pgfpathcurveto{\pgfqpoint{3.468369in}{1.142686in}}{\pgfqpoint{3.473543in}{1.130195in}}{\pgfqpoint{3.482751in}{1.120987in}}%
\pgfpathcurveto{\pgfqpoint{3.491960in}{1.111779in}}{\pgfqpoint{3.504451in}{1.106605in}}{\pgfqpoint{3.517474in}{1.106605in}}%
\pgfpathlineto{\pgfqpoint{3.517474in}{1.106605in}}%
\pgfpathclose%
\pgfusepath{stroke,fill}%
\end{pgfscope}%
\begin{pgfscope}%
\pgfpathrectangle{\pgfqpoint{0.786164in}{0.768110in}}{\pgfqpoint{8.851069in}{7.081890in}}%
\pgfusepath{clip}%
\pgfsetbuttcap%
\pgfsetroundjoin%
\definecolor{currentfill}{rgb}{0.876168,0.891125,0.095250}%
\pgfsetfillcolor{currentfill}%
\pgfsetfillopacity{0.700000}%
\pgfsetlinewidth{0.501875pt}%
\definecolor{currentstroke}{rgb}{1.000000,1.000000,1.000000}%
\pgfsetstrokecolor{currentstroke}%
\pgfsetstrokeopacity{0.700000}%
\pgfsetdash{}{0pt}%
\pgfpathmoveto{\pgfqpoint{3.672739in}{1.128503in}}%
\pgfpathcurveto{\pgfqpoint{3.685762in}{1.128503in}}{\pgfqpoint{3.698253in}{1.133677in}}{\pgfqpoint{3.707462in}{1.142885in}}%
\pgfpathcurveto{\pgfqpoint{3.716670in}{1.152094in}}{\pgfqpoint{3.721844in}{1.164585in}}{\pgfqpoint{3.721844in}{1.177607in}}%
\pgfpathcurveto{\pgfqpoint{3.721844in}{1.190630in}}{\pgfqpoint{3.716670in}{1.203121in}}{\pgfqpoint{3.707462in}{1.212330in}}%
\pgfpathcurveto{\pgfqpoint{3.698253in}{1.221538in}}{\pgfqpoint{3.685762in}{1.226712in}}{\pgfqpoint{3.672739in}{1.226712in}}%
\pgfpathcurveto{\pgfqpoint{3.659717in}{1.226712in}}{\pgfqpoint{3.647226in}{1.221538in}}{\pgfqpoint{3.638017in}{1.212330in}}%
\pgfpathcurveto{\pgfqpoint{3.628809in}{1.203121in}}{\pgfqpoint{3.623635in}{1.190630in}}{\pgfqpoint{3.623635in}{1.177607in}}%
\pgfpathcurveto{\pgfqpoint{3.623635in}{1.164585in}}{\pgfqpoint{3.628809in}{1.152094in}}{\pgfqpoint{3.638017in}{1.142885in}}%
\pgfpathcurveto{\pgfqpoint{3.647226in}{1.133677in}}{\pgfqpoint{3.659717in}{1.128503in}}{\pgfqpoint{3.672739in}{1.128503in}}%
\pgfpathlineto{\pgfqpoint{3.672739in}{1.128503in}}%
\pgfpathclose%
\pgfusepath{stroke,fill}%
\end{pgfscope}%
\begin{pgfscope}%
\pgfpathrectangle{\pgfqpoint{0.786164in}{0.768110in}}{\pgfqpoint{8.851069in}{7.081890in}}%
\pgfusepath{clip}%
\pgfsetbuttcap%
\pgfsetroundjoin%
\definecolor{currentfill}{rgb}{0.993248,0.906157,0.143936}%
\pgfsetfillcolor{currentfill}%
\pgfsetfillopacity{0.700000}%
\pgfsetlinewidth{0.501875pt}%
\definecolor{currentstroke}{rgb}{1.000000,1.000000,1.000000}%
\pgfsetstrokecolor{currentstroke}%
\pgfsetstrokeopacity{0.700000}%
\pgfsetdash{}{0pt}%
\pgfpathmoveto{\pgfqpoint{3.572273in}{1.150401in}}%
\pgfpathcurveto{\pgfqpoint{3.585296in}{1.150401in}}{\pgfqpoint{3.597787in}{1.155575in}}{\pgfqpoint{3.606995in}{1.164783in}}%
\pgfpathcurveto{\pgfqpoint{3.616204in}{1.173992in}}{\pgfqpoint{3.621378in}{1.186483in}}{\pgfqpoint{3.621378in}{1.199506in}}%
\pgfpathcurveto{\pgfqpoint{3.621378in}{1.212528in}}{\pgfqpoint{3.616204in}{1.225019in}}{\pgfqpoint{3.606995in}{1.234228in}}%
\pgfpathcurveto{\pgfqpoint{3.597787in}{1.243436in}}{\pgfqpoint{3.585296in}{1.248610in}}{\pgfqpoint{3.572273in}{1.248610in}}%
\pgfpathcurveto{\pgfqpoint{3.559251in}{1.248610in}}{\pgfqpoint{3.546759in}{1.243436in}}{\pgfqpoint{3.537551in}{1.234228in}}%
\pgfpathcurveto{\pgfqpoint{3.528343in}{1.225019in}}{\pgfqpoint{3.523169in}{1.212528in}}{\pgfqpoint{3.523169in}{1.199506in}}%
\pgfpathcurveto{\pgfqpoint{3.523169in}{1.186483in}}{\pgfqpoint{3.528343in}{1.173992in}}{\pgfqpoint{3.537551in}{1.164783in}}%
\pgfpathcurveto{\pgfqpoint{3.546759in}{1.155575in}}{\pgfqpoint{3.559251in}{1.150401in}}{\pgfqpoint{3.572273in}{1.150401in}}%
\pgfpathlineto{\pgfqpoint{3.572273in}{1.150401in}}%
\pgfpathclose%
\pgfusepath{stroke,fill}%
\end{pgfscope}%
\begin{pgfscope}%
\pgfpathrectangle{\pgfqpoint{0.786164in}{0.768110in}}{\pgfqpoint{8.851069in}{7.081890in}}%
\pgfusepath{clip}%
\pgfsetbuttcap%
\pgfsetroundjoin%
\definecolor{currentfill}{rgb}{0.216210,0.351535,0.550627}%
\pgfsetfillcolor{currentfill}%
\pgfsetfillopacity{0.700000}%
\pgfsetlinewidth{0.501875pt}%
\definecolor{currentstroke}{rgb}{1.000000,1.000000,1.000000}%
\pgfsetstrokecolor{currentstroke}%
\pgfsetstrokeopacity{0.700000}%
\pgfsetdash{}{0pt}%
\pgfpathmoveto{\pgfqpoint{3.069942in}{2.464295in}}%
\pgfpathcurveto{\pgfqpoint{3.082965in}{2.464295in}}{\pgfqpoint{3.095456in}{2.469469in}}{\pgfqpoint{3.104665in}{2.478678in}}%
\pgfpathcurveto{\pgfqpoint{3.113873in}{2.487886in}}{\pgfqpoint{3.119047in}{2.500377in}}{\pgfqpoint{3.119047in}{2.513400in}}%
\pgfpathcurveto{\pgfqpoint{3.119047in}{2.526423in}}{\pgfqpoint{3.113873in}{2.538914in}}{\pgfqpoint{3.104665in}{2.548122in}}%
\pgfpathcurveto{\pgfqpoint{3.095456in}{2.557331in}}{\pgfqpoint{3.082965in}{2.562504in}}{\pgfqpoint{3.069942in}{2.562504in}}%
\pgfpathcurveto{\pgfqpoint{3.056920in}{2.562504in}}{\pgfqpoint{3.044429in}{2.557331in}}{\pgfqpoint{3.035220in}{2.548122in}}%
\pgfpathcurveto{\pgfqpoint{3.026012in}{2.538914in}}{\pgfqpoint{3.020838in}{2.526423in}}{\pgfqpoint{3.020838in}{2.513400in}}%
\pgfpathcurveto{\pgfqpoint{3.020838in}{2.500377in}}{\pgfqpoint{3.026012in}{2.487886in}}{\pgfqpoint{3.035220in}{2.478678in}}%
\pgfpathcurveto{\pgfqpoint{3.044429in}{2.469469in}}{\pgfqpoint{3.056920in}{2.464295in}}{\pgfqpoint{3.069942in}{2.464295in}}%
\pgfpathlineto{\pgfqpoint{3.069942in}{2.464295in}}%
\pgfpathclose%
\pgfusepath{stroke,fill}%
\end{pgfscope}%
\begin{pgfscope}%
\pgfpathrectangle{\pgfqpoint{0.786164in}{0.768110in}}{\pgfqpoint{8.851069in}{7.081890in}}%
\pgfusepath{clip}%
\pgfsetbuttcap%
\pgfsetroundjoin%
\definecolor{currentfill}{rgb}{0.206756,0.371758,0.553117}%
\pgfsetfillcolor{currentfill}%
\pgfsetfillopacity{0.700000}%
\pgfsetlinewidth{0.501875pt}%
\definecolor{currentstroke}{rgb}{1.000000,1.000000,1.000000}%
\pgfsetstrokecolor{currentstroke}%
\pgfsetstrokeopacity{0.700000}%
\pgfsetdash{}{0pt}%
\pgfpathmoveto{\pgfqpoint{3.179542in}{2.486193in}}%
\pgfpathcurveto{\pgfqpoint{3.192565in}{2.486193in}}{\pgfqpoint{3.205056in}{2.491367in}}{\pgfqpoint{3.214264in}{2.500576in}}%
\pgfpathcurveto{\pgfqpoint{3.223473in}{2.509784in}}{\pgfqpoint{3.228647in}{2.522275in}}{\pgfqpoint{3.228647in}{2.535298in}}%
\pgfpathcurveto{\pgfqpoint{3.228647in}{2.548321in}}{\pgfqpoint{3.223473in}{2.560812in}}{\pgfqpoint{3.214264in}{2.570020in}}%
\pgfpathcurveto{\pgfqpoint{3.205056in}{2.579229in}}{\pgfqpoint{3.192565in}{2.584403in}}{\pgfqpoint{3.179542in}{2.584403in}}%
\pgfpathcurveto{\pgfqpoint{3.166519in}{2.584403in}}{\pgfqpoint{3.154028in}{2.579229in}}{\pgfqpoint{3.144820in}{2.570020in}}%
\pgfpathcurveto{\pgfqpoint{3.135611in}{2.560812in}}{\pgfqpoint{3.130437in}{2.548321in}}{\pgfqpoint{3.130437in}{2.535298in}}%
\pgfpathcurveto{\pgfqpoint{3.130437in}{2.522275in}}{\pgfqpoint{3.135611in}{2.509784in}}{\pgfqpoint{3.144820in}{2.500576in}}%
\pgfpathcurveto{\pgfqpoint{3.154028in}{2.491367in}}{\pgfqpoint{3.166519in}{2.486193in}}{\pgfqpoint{3.179542in}{2.486193in}}%
\pgfpathlineto{\pgfqpoint{3.179542in}{2.486193in}}%
\pgfpathclose%
\pgfusepath{stroke,fill}%
\end{pgfscope}%
\begin{pgfscope}%
\pgfpathrectangle{\pgfqpoint{0.786164in}{0.768110in}}{\pgfqpoint{8.851069in}{7.081890in}}%
\pgfusepath{clip}%
\pgfsetbuttcap%
\pgfsetroundjoin%
\definecolor{currentfill}{rgb}{0.216210,0.351535,0.550627}%
\pgfsetfillcolor{currentfill}%
\pgfsetfillopacity{0.700000}%
\pgfsetlinewidth{0.501875pt}%
\definecolor{currentstroke}{rgb}{1.000000,1.000000,1.000000}%
\pgfsetstrokecolor{currentstroke}%
\pgfsetstrokeopacity{0.700000}%
\pgfsetdash{}{0pt}%
\pgfpathmoveto{\pgfqpoint{3.143009in}{2.508092in}}%
\pgfpathcurveto{\pgfqpoint{3.156031in}{2.508092in}}{\pgfqpoint{3.168523in}{2.513266in}}{\pgfqpoint{3.177731in}{2.522474in}}%
\pgfpathcurveto{\pgfqpoint{3.186939in}{2.531683in}}{\pgfqpoint{3.192113in}{2.544174in}}{\pgfqpoint{3.192113in}{2.557196in}}%
\pgfpathcurveto{\pgfqpoint{3.192113in}{2.570219in}}{\pgfqpoint{3.186939in}{2.582710in}}{\pgfqpoint{3.177731in}{2.591919in}}%
\pgfpathcurveto{\pgfqpoint{3.168523in}{2.601127in}}{\pgfqpoint{3.156031in}{2.606301in}}{\pgfqpoint{3.143009in}{2.606301in}}%
\pgfpathcurveto{\pgfqpoint{3.129986in}{2.606301in}}{\pgfqpoint{3.117495in}{2.601127in}}{\pgfqpoint{3.108286in}{2.591919in}}%
\pgfpathcurveto{\pgfqpoint{3.099078in}{2.582710in}}{\pgfqpoint{3.093904in}{2.570219in}}{\pgfqpoint{3.093904in}{2.557196in}}%
\pgfpathcurveto{\pgfqpoint{3.093904in}{2.544174in}}{\pgfqpoint{3.099078in}{2.531683in}}{\pgfqpoint{3.108286in}{2.522474in}}%
\pgfpathcurveto{\pgfqpoint{3.117495in}{2.513266in}}{\pgfqpoint{3.129986in}{2.508092in}}{\pgfqpoint{3.143009in}{2.508092in}}%
\pgfpathlineto{\pgfqpoint{3.143009in}{2.508092in}}%
\pgfpathclose%
\pgfusepath{stroke,fill}%
\end{pgfscope}%
\begin{pgfscope}%
\pgfpathrectangle{\pgfqpoint{0.786164in}{0.768110in}}{\pgfqpoint{8.851069in}{7.081890in}}%
\pgfusepath{clip}%
\pgfsetbuttcap%
\pgfsetroundjoin%
\definecolor{currentfill}{rgb}{0.206756,0.371758,0.553117}%
\pgfsetfillcolor{currentfill}%
\pgfsetfillopacity{0.700000}%
\pgfsetlinewidth{0.501875pt}%
\definecolor{currentstroke}{rgb}{1.000000,1.000000,1.000000}%
\pgfsetstrokecolor{currentstroke}%
\pgfsetstrokeopacity{0.700000}%
\pgfsetdash{}{0pt}%
\pgfpathmoveto{\pgfqpoint{3.170409in}{2.486193in}}%
\pgfpathcurveto{\pgfqpoint{3.183431in}{2.486193in}}{\pgfqpoint{3.195922in}{2.491367in}}{\pgfqpoint{3.205131in}{2.500576in}}%
\pgfpathcurveto{\pgfqpoint{3.214339in}{2.509784in}}{\pgfqpoint{3.219513in}{2.522275in}}{\pgfqpoint{3.219513in}{2.535298in}}%
\pgfpathcurveto{\pgfqpoint{3.219513in}{2.548321in}}{\pgfqpoint{3.214339in}{2.560812in}}{\pgfqpoint{3.205131in}{2.570020in}}%
\pgfpathcurveto{\pgfqpoint{3.195922in}{2.579229in}}{\pgfqpoint{3.183431in}{2.584403in}}{\pgfqpoint{3.170409in}{2.584403in}}%
\pgfpathcurveto{\pgfqpoint{3.157386in}{2.584403in}}{\pgfqpoint{3.144895in}{2.579229in}}{\pgfqpoint{3.135686in}{2.570020in}}%
\pgfpathcurveto{\pgfqpoint{3.126478in}{2.560812in}}{\pgfqpoint{3.121304in}{2.548321in}}{\pgfqpoint{3.121304in}{2.535298in}}%
\pgfpathcurveto{\pgfqpoint{3.121304in}{2.522275in}}{\pgfqpoint{3.126478in}{2.509784in}}{\pgfqpoint{3.135686in}{2.500576in}}%
\pgfpathcurveto{\pgfqpoint{3.144895in}{2.491367in}}{\pgfqpoint{3.157386in}{2.486193in}}{\pgfqpoint{3.170409in}{2.486193in}}%
\pgfpathlineto{\pgfqpoint{3.170409in}{2.486193in}}%
\pgfpathclose%
\pgfusepath{stroke,fill}%
\end{pgfscope}%
\begin{pgfscope}%
\pgfpathrectangle{\pgfqpoint{0.786164in}{0.768110in}}{\pgfqpoint{8.851069in}{7.081890in}}%
\pgfusepath{clip}%
\pgfsetbuttcap%
\pgfsetroundjoin%
\definecolor{currentfill}{rgb}{0.214298,0.355619,0.551184}%
\pgfsetfillcolor{currentfill}%
\pgfsetfillopacity{0.700000}%
\pgfsetlinewidth{0.501875pt}%
\definecolor{currentstroke}{rgb}{1.000000,1.000000,1.000000}%
\pgfsetstrokecolor{currentstroke}%
\pgfsetstrokeopacity{0.700000}%
\pgfsetdash{}{0pt}%
\pgfpathmoveto{\pgfqpoint{3.362208in}{2.595685in}}%
\pgfpathcurveto{\pgfqpoint{3.375230in}{2.595685in}}{\pgfqpoint{3.387721in}{2.600859in}}{\pgfqpoint{3.396930in}{2.610067in}}%
\pgfpathcurveto{\pgfqpoint{3.406138in}{2.619275in}}{\pgfqpoint{3.411312in}{2.631767in}}{\pgfqpoint{3.411312in}{2.644789in}}%
\pgfpathcurveto{\pgfqpoint{3.411312in}{2.657812in}}{\pgfqpoint{3.406138in}{2.670303in}}{\pgfqpoint{3.396930in}{2.679511in}}%
\pgfpathcurveto{\pgfqpoint{3.387721in}{2.688720in}}{\pgfqpoint{3.375230in}{2.693894in}}{\pgfqpoint{3.362208in}{2.693894in}}%
\pgfpathcurveto{\pgfqpoint{3.349185in}{2.693894in}}{\pgfqpoint{3.336694in}{2.688720in}}{\pgfqpoint{3.327485in}{2.679511in}}%
\pgfpathcurveto{\pgfqpoint{3.318277in}{2.670303in}}{\pgfqpoint{3.313103in}{2.657812in}}{\pgfqpoint{3.313103in}{2.644789in}}%
\pgfpathcurveto{\pgfqpoint{3.313103in}{2.631767in}}{\pgfqpoint{3.318277in}{2.619275in}}{\pgfqpoint{3.327485in}{2.610067in}}%
\pgfpathcurveto{\pgfqpoint{3.336694in}{2.600859in}}{\pgfqpoint{3.349185in}{2.595685in}}{\pgfqpoint{3.362208in}{2.595685in}}%
\pgfpathlineto{\pgfqpoint{3.362208in}{2.595685in}}%
\pgfpathclose%
\pgfusepath{stroke,fill}%
\end{pgfscope}%
\begin{pgfscope}%
\pgfpathrectangle{\pgfqpoint{0.786164in}{0.768110in}}{\pgfqpoint{8.851069in}{7.081890in}}%
\pgfusepath{clip}%
\pgfsetbuttcap%
\pgfsetroundjoin%
\definecolor{currentfill}{rgb}{0.199430,0.387607,0.554642}%
\pgfsetfillcolor{currentfill}%
\pgfsetfillopacity{0.700000}%
\pgfsetlinewidth{0.501875pt}%
\definecolor{currentstroke}{rgb}{1.000000,1.000000,1.000000}%
\pgfsetstrokecolor{currentstroke}%
\pgfsetstrokeopacity{0.700000}%
\pgfsetdash{}{0pt}%
\pgfpathmoveto{\pgfqpoint{2.923810in}{2.354804in}}%
\pgfpathcurveto{\pgfqpoint{2.936833in}{2.354804in}}{\pgfqpoint{2.949324in}{2.359978in}}{\pgfqpoint{2.958532in}{2.369186in}}%
\pgfpathcurveto{\pgfqpoint{2.967740in}{2.378395in}}{\pgfqpoint{2.972914in}{2.390886in}}{\pgfqpoint{2.972914in}{2.403909in}}%
\pgfpathcurveto{\pgfqpoint{2.972914in}{2.416931in}}{\pgfqpoint{2.967740in}{2.429422in}}{\pgfqpoint{2.958532in}{2.438631in}}%
\pgfpathcurveto{\pgfqpoint{2.949324in}{2.447839in}}{\pgfqpoint{2.936833in}{2.453013in}}{\pgfqpoint{2.923810in}{2.453013in}}%
\pgfpathcurveto{\pgfqpoint{2.910787in}{2.453013in}}{\pgfqpoint{2.898296in}{2.447839in}}{\pgfqpoint{2.889088in}{2.438631in}}%
\pgfpathcurveto{\pgfqpoint{2.879879in}{2.429422in}}{\pgfqpoint{2.874705in}{2.416931in}}{\pgfqpoint{2.874705in}{2.403909in}}%
\pgfpathcurveto{\pgfqpoint{2.874705in}{2.390886in}}{\pgfqpoint{2.879879in}{2.378395in}}{\pgfqpoint{2.889088in}{2.369186in}}%
\pgfpathcurveto{\pgfqpoint{2.898296in}{2.359978in}}{\pgfqpoint{2.910787in}{2.354804in}}{\pgfqpoint{2.923810in}{2.354804in}}%
\pgfpathlineto{\pgfqpoint{2.923810in}{2.354804in}}%
\pgfpathclose%
\pgfusepath{stroke,fill}%
\end{pgfscope}%
\begin{pgfscope}%
\pgfpathrectangle{\pgfqpoint{0.786164in}{0.768110in}}{\pgfqpoint{8.851069in}{7.081890in}}%
\pgfusepath{clip}%
\pgfsetbuttcap%
\pgfsetroundjoin%
\definecolor{currentfill}{rgb}{0.197636,0.391528,0.554969}%
\pgfsetfillcolor{currentfill}%
\pgfsetfillopacity{0.700000}%
\pgfsetlinewidth{0.501875pt}%
\definecolor{currentstroke}{rgb}{1.000000,1.000000,1.000000}%
\pgfsetstrokecolor{currentstroke}%
\pgfsetstrokeopacity{0.700000}%
\pgfsetdash{}{0pt}%
\pgfpathmoveto{\pgfqpoint{3.006009in}{2.376702in}}%
\pgfpathcurveto{\pgfqpoint{3.019032in}{2.376702in}}{\pgfqpoint{3.031523in}{2.381876in}}{\pgfqpoint{3.040732in}{2.391085in}}%
\pgfpathcurveto{\pgfqpoint{3.049940in}{2.400293in}}{\pgfqpoint{3.055114in}{2.412784in}}{\pgfqpoint{3.055114in}{2.425807in}}%
\pgfpathcurveto{\pgfqpoint{3.055114in}{2.438830in}}{\pgfqpoint{3.049940in}{2.451321in}}{\pgfqpoint{3.040732in}{2.460529in}}%
\pgfpathcurveto{\pgfqpoint{3.031523in}{2.469738in}}{\pgfqpoint{3.019032in}{2.474912in}}{\pgfqpoint{3.006009in}{2.474912in}}%
\pgfpathcurveto{\pgfqpoint{2.992987in}{2.474912in}}{\pgfqpoint{2.980496in}{2.469738in}}{\pgfqpoint{2.971287in}{2.460529in}}%
\pgfpathcurveto{\pgfqpoint{2.962079in}{2.451321in}}{\pgfqpoint{2.956905in}{2.438830in}}{\pgfqpoint{2.956905in}{2.425807in}}%
\pgfpathcurveto{\pgfqpoint{2.956905in}{2.412784in}}{\pgfqpoint{2.962079in}{2.400293in}}{\pgfqpoint{2.971287in}{2.391085in}}%
\pgfpathcurveto{\pgfqpoint{2.980496in}{2.381876in}}{\pgfqpoint{2.992987in}{2.376702in}}{\pgfqpoint{3.006009in}{2.376702in}}%
\pgfpathlineto{\pgfqpoint{3.006009in}{2.376702in}}%
\pgfpathclose%
\pgfusepath{stroke,fill}%
\end{pgfscope}%
\begin{pgfscope}%
\pgfpathrectangle{\pgfqpoint{0.786164in}{0.768110in}}{\pgfqpoint{8.851069in}{7.081890in}}%
\pgfusepath{clip}%
\pgfsetbuttcap%
\pgfsetroundjoin%
\definecolor{currentfill}{rgb}{0.199430,0.387607,0.554642}%
\pgfsetfillcolor{currentfill}%
\pgfsetfillopacity{0.700000}%
\pgfsetlinewidth{0.501875pt}%
\definecolor{currentstroke}{rgb}{1.000000,1.000000,1.000000}%
\pgfsetstrokecolor{currentstroke}%
\pgfsetstrokeopacity{0.700000}%
\pgfsetdash{}{0pt}%
\pgfpathmoveto{\pgfqpoint{3.033409in}{2.376702in}}%
\pgfpathcurveto{\pgfqpoint{3.046432in}{2.376702in}}{\pgfqpoint{3.058923in}{2.381876in}}{\pgfqpoint{3.068131in}{2.391085in}}%
\pgfpathcurveto{\pgfqpoint{3.077340in}{2.400293in}}{\pgfqpoint{3.082514in}{2.412784in}}{\pgfqpoint{3.082514in}{2.425807in}}%
\pgfpathcurveto{\pgfqpoint{3.082514in}{2.438830in}}{\pgfqpoint{3.077340in}{2.451321in}}{\pgfqpoint{3.068131in}{2.460529in}}%
\pgfpathcurveto{\pgfqpoint{3.058923in}{2.469738in}}{\pgfqpoint{3.046432in}{2.474912in}}{\pgfqpoint{3.033409in}{2.474912in}}%
\pgfpathcurveto{\pgfqpoint{3.020387in}{2.474912in}}{\pgfqpoint{3.007895in}{2.469738in}}{\pgfqpoint{2.998687in}{2.460529in}}%
\pgfpathcurveto{\pgfqpoint{2.989479in}{2.451321in}}{\pgfqpoint{2.984305in}{2.438830in}}{\pgfqpoint{2.984305in}{2.425807in}}%
\pgfpathcurveto{\pgfqpoint{2.984305in}{2.412784in}}{\pgfqpoint{2.989479in}{2.400293in}}{\pgfqpoint{2.998687in}{2.391085in}}%
\pgfpathcurveto{\pgfqpoint{3.007895in}{2.381876in}}{\pgfqpoint{3.020387in}{2.376702in}}{\pgfqpoint{3.033409in}{2.376702in}}%
\pgfpathlineto{\pgfqpoint{3.033409in}{2.376702in}}%
\pgfpathclose%
\pgfusepath{stroke,fill}%
\end{pgfscope}%
\begin{pgfscope}%
\pgfpathrectangle{\pgfqpoint{0.786164in}{0.768110in}}{\pgfqpoint{8.851069in}{7.081890in}}%
\pgfusepath{clip}%
\pgfsetbuttcap%
\pgfsetroundjoin%
\definecolor{currentfill}{rgb}{0.194100,0.399323,0.555565}%
\pgfsetfillcolor{currentfill}%
\pgfsetfillopacity{0.700000}%
\pgfsetlinewidth{0.501875pt}%
\definecolor{currentstroke}{rgb}{1.000000,1.000000,1.000000}%
\pgfsetstrokecolor{currentstroke}%
\pgfsetstrokeopacity{0.700000}%
\pgfsetdash{}{0pt}%
\pgfpathmoveto{\pgfqpoint{2.905543in}{2.311008in}}%
\pgfpathcurveto{\pgfqpoint{2.918566in}{2.311008in}}{\pgfqpoint{2.931057in}{2.316182in}}{\pgfqpoint{2.940265in}{2.325390in}}%
\pgfpathcurveto{\pgfqpoint{2.949474in}{2.334598in}}{\pgfqpoint{2.954648in}{2.347089in}}{\pgfqpoint{2.954648in}{2.360112in}}%
\pgfpathcurveto{\pgfqpoint{2.954648in}{2.373135in}}{\pgfqpoint{2.949474in}{2.385626in}}{\pgfqpoint{2.940265in}{2.394834in}}%
\pgfpathcurveto{\pgfqpoint{2.931057in}{2.404043in}}{\pgfqpoint{2.918566in}{2.409217in}}{\pgfqpoint{2.905543in}{2.409217in}}%
\pgfpathcurveto{\pgfqpoint{2.892521in}{2.409217in}}{\pgfqpoint{2.880029in}{2.404043in}}{\pgfqpoint{2.870821in}{2.394834in}}%
\pgfpathcurveto{\pgfqpoint{2.861613in}{2.385626in}}{\pgfqpoint{2.856439in}{2.373135in}}{\pgfqpoint{2.856439in}{2.360112in}}%
\pgfpathcurveto{\pgfqpoint{2.856439in}{2.347089in}}{\pgfqpoint{2.861613in}{2.334598in}}{\pgfqpoint{2.870821in}{2.325390in}}%
\pgfpathcurveto{\pgfqpoint{2.880029in}{2.316182in}}{\pgfqpoint{2.892521in}{2.311008in}}{\pgfqpoint{2.905543in}{2.311008in}}%
\pgfpathlineto{\pgfqpoint{2.905543in}{2.311008in}}%
\pgfpathclose%
\pgfusepath{stroke,fill}%
\end{pgfscope}%
\begin{pgfscope}%
\pgfpathrectangle{\pgfqpoint{0.786164in}{0.768110in}}{\pgfqpoint{8.851069in}{7.081890in}}%
\pgfusepath{clip}%
\pgfsetbuttcap%
\pgfsetroundjoin%
\definecolor{currentfill}{rgb}{0.183898,0.422383,0.556944}%
\pgfsetfillcolor{currentfill}%
\pgfsetfillopacity{0.700000}%
\pgfsetlinewidth{0.501875pt}%
\definecolor{currentstroke}{rgb}{1.000000,1.000000,1.000000}%
\pgfsetstrokecolor{currentstroke}%
\pgfsetstrokeopacity{0.700000}%
\pgfsetdash{}{0pt}%
\pgfpathmoveto{\pgfqpoint{2.823344in}{2.420499in}}%
\pgfpathcurveto{\pgfqpoint{2.836366in}{2.420499in}}{\pgfqpoint{2.848857in}{2.425673in}}{\pgfqpoint{2.858066in}{2.434881in}}%
\pgfpathcurveto{\pgfqpoint{2.867274in}{2.444090in}}{\pgfqpoint{2.872448in}{2.456581in}}{\pgfqpoint{2.872448in}{2.469603in}}%
\pgfpathcurveto{\pgfqpoint{2.872448in}{2.482626in}}{\pgfqpoint{2.867274in}{2.495117in}}{\pgfqpoint{2.858066in}{2.504326in}}%
\pgfpathcurveto{\pgfqpoint{2.848857in}{2.513534in}}{\pgfqpoint{2.836366in}{2.518708in}}{\pgfqpoint{2.823344in}{2.518708in}}%
\pgfpathcurveto{\pgfqpoint{2.810321in}{2.518708in}}{\pgfqpoint{2.797830in}{2.513534in}}{\pgfqpoint{2.788621in}{2.504326in}}%
\pgfpathcurveto{\pgfqpoint{2.779413in}{2.495117in}}{\pgfqpoint{2.774239in}{2.482626in}}{\pgfqpoint{2.774239in}{2.469603in}}%
\pgfpathcurveto{\pgfqpoint{2.774239in}{2.456581in}}{\pgfqpoint{2.779413in}{2.444090in}}{\pgfqpoint{2.788621in}{2.434881in}}%
\pgfpathcurveto{\pgfqpoint{2.797830in}{2.425673in}}{\pgfqpoint{2.810321in}{2.420499in}}{\pgfqpoint{2.823344in}{2.420499in}}%
\pgfpathlineto{\pgfqpoint{2.823344in}{2.420499in}}%
\pgfpathclose%
\pgfusepath{stroke,fill}%
\end{pgfscope}%
\begin{pgfscope}%
\pgfpathrectangle{\pgfqpoint{0.786164in}{0.768110in}}{\pgfqpoint{8.851069in}{7.081890in}}%
\pgfusepath{clip}%
\pgfsetbuttcap%
\pgfsetroundjoin%
\definecolor{currentfill}{rgb}{0.188923,0.410910,0.556326}%
\pgfsetfillcolor{currentfill}%
\pgfsetfillopacity{0.700000}%
\pgfsetlinewidth{0.501875pt}%
\definecolor{currentstroke}{rgb}{1.000000,1.000000,1.000000}%
\pgfsetstrokecolor{currentstroke}%
\pgfsetstrokeopacity{0.700000}%
\pgfsetdash{}{0pt}%
\pgfpathmoveto{\pgfqpoint{2.704611in}{2.858463in}}%
\pgfpathcurveto{\pgfqpoint{2.717634in}{2.858463in}}{\pgfqpoint{2.730125in}{2.863637in}}{\pgfqpoint{2.739333in}{2.872846in}}%
\pgfpathcurveto{\pgfqpoint{2.748542in}{2.882054in}}{\pgfqpoint{2.753716in}{2.894545in}}{\pgfqpoint{2.753716in}{2.907568in}}%
\pgfpathcurveto{\pgfqpoint{2.753716in}{2.920591in}}{\pgfqpoint{2.748542in}{2.933082in}}{\pgfqpoint{2.739333in}{2.942290in}}%
\pgfpathcurveto{\pgfqpoint{2.730125in}{2.951499in}}{\pgfqpoint{2.717634in}{2.956673in}}{\pgfqpoint{2.704611in}{2.956673in}}%
\pgfpathcurveto{\pgfqpoint{2.691588in}{2.956673in}}{\pgfqpoint{2.679097in}{2.951499in}}{\pgfqpoint{2.669889in}{2.942290in}}%
\pgfpathcurveto{\pgfqpoint{2.660680in}{2.933082in}}{\pgfqpoint{2.655506in}{2.920591in}}{\pgfqpoint{2.655506in}{2.907568in}}%
\pgfpathcurveto{\pgfqpoint{2.655506in}{2.894545in}}{\pgfqpoint{2.660680in}{2.882054in}}{\pgfqpoint{2.669889in}{2.872846in}}%
\pgfpathcurveto{\pgfqpoint{2.679097in}{2.863637in}}{\pgfqpoint{2.691588in}{2.858463in}}{\pgfqpoint{2.704611in}{2.858463in}}%
\pgfpathlineto{\pgfqpoint{2.704611in}{2.858463in}}%
\pgfpathclose%
\pgfusepath{stroke,fill}%
\end{pgfscope}%
\begin{pgfscope}%
\pgfpathrectangle{\pgfqpoint{0.786164in}{0.768110in}}{\pgfqpoint{8.851069in}{7.081890in}}%
\pgfusepath{clip}%
\pgfsetbuttcap%
\pgfsetroundjoin%
\definecolor{currentfill}{rgb}{0.185556,0.418570,0.556753}%
\pgfsetfillcolor{currentfill}%
\pgfsetfillopacity{0.700000}%
\pgfsetlinewidth{0.501875pt}%
\definecolor{currentstroke}{rgb}{1.000000,1.000000,1.000000}%
\pgfsetstrokecolor{currentstroke}%
\pgfsetstrokeopacity{0.700000}%
\pgfsetdash{}{0pt}%
\pgfpathmoveto{\pgfqpoint{2.677211in}{2.880362in}}%
\pgfpathcurveto{\pgfqpoint{2.690234in}{2.880362in}}{\pgfqpoint{2.702725in}{2.885536in}}{\pgfqpoint{2.711933in}{2.894744in}}%
\pgfpathcurveto{\pgfqpoint{2.721142in}{2.903953in}}{\pgfqpoint{2.726316in}{2.916444in}}{\pgfqpoint{2.726316in}{2.929466in}}%
\pgfpathcurveto{\pgfqpoint{2.726316in}{2.942489in}}{\pgfqpoint{2.721142in}{2.954980in}}{\pgfqpoint{2.711933in}{2.964189in}}%
\pgfpathcurveto{\pgfqpoint{2.702725in}{2.973397in}}{\pgfqpoint{2.690234in}{2.978571in}}{\pgfqpoint{2.677211in}{2.978571in}}%
\pgfpathcurveto{\pgfqpoint{2.664188in}{2.978571in}}{\pgfqpoint{2.651697in}{2.973397in}}{\pgfqpoint{2.642489in}{2.964189in}}%
\pgfpathcurveto{\pgfqpoint{2.633280in}{2.954980in}}{\pgfqpoint{2.628106in}{2.942489in}}{\pgfqpoint{2.628106in}{2.929466in}}%
\pgfpathcurveto{\pgfqpoint{2.628106in}{2.916444in}}{\pgfqpoint{2.633280in}{2.903953in}}{\pgfqpoint{2.642489in}{2.894744in}}%
\pgfpathcurveto{\pgfqpoint{2.651697in}{2.885536in}}{\pgfqpoint{2.664188in}{2.880362in}}{\pgfqpoint{2.677211in}{2.880362in}}%
\pgfpathlineto{\pgfqpoint{2.677211in}{2.880362in}}%
\pgfpathclose%
\pgfusepath{stroke,fill}%
\end{pgfscope}%
\begin{pgfscope}%
\pgfpathrectangle{\pgfqpoint{0.786164in}{0.768110in}}{\pgfqpoint{8.851069in}{7.081890in}}%
\pgfusepath{clip}%
\pgfsetbuttcap%
\pgfsetroundjoin%
\definecolor{currentfill}{rgb}{0.192357,0.403199,0.555836}%
\pgfsetfillcolor{currentfill}%
\pgfsetfillopacity{0.700000}%
\pgfsetlinewidth{0.501875pt}%
\definecolor{currentstroke}{rgb}{1.000000,1.000000,1.000000}%
\pgfsetstrokecolor{currentstroke}%
\pgfsetstrokeopacity{0.700000}%
\pgfsetdash{}{0pt}%
\pgfpathmoveto{\pgfqpoint{2.768544in}{2.989853in}}%
\pgfpathcurveto{\pgfqpoint{2.781567in}{2.989853in}}{\pgfqpoint{2.794058in}{2.995027in}}{\pgfqpoint{2.803266in}{3.004235in}}%
\pgfpathcurveto{\pgfqpoint{2.812475in}{3.013444in}}{\pgfqpoint{2.817649in}{3.025935in}}{\pgfqpoint{2.817649in}{3.038958in}}%
\pgfpathcurveto{\pgfqpoint{2.817649in}{3.051980in}}{\pgfqpoint{2.812475in}{3.064471in}}{\pgfqpoint{2.803266in}{3.073680in}}%
\pgfpathcurveto{\pgfqpoint{2.794058in}{3.082888in}}{\pgfqpoint{2.781567in}{3.088062in}}{\pgfqpoint{2.768544in}{3.088062in}}%
\pgfpathcurveto{\pgfqpoint{2.755521in}{3.088062in}}{\pgfqpoint{2.743030in}{3.082888in}}{\pgfqpoint{2.733822in}{3.073680in}}%
\pgfpathcurveto{\pgfqpoint{2.724613in}{3.064471in}}{\pgfqpoint{2.719439in}{3.051980in}}{\pgfqpoint{2.719439in}{3.038958in}}%
\pgfpathcurveto{\pgfqpoint{2.719439in}{3.025935in}}{\pgfqpoint{2.724613in}{3.013444in}}{\pgfqpoint{2.733822in}{3.004235in}}%
\pgfpathcurveto{\pgfqpoint{2.743030in}{2.995027in}}{\pgfqpoint{2.755521in}{2.989853in}}{\pgfqpoint{2.768544in}{2.989853in}}%
\pgfpathlineto{\pgfqpoint{2.768544in}{2.989853in}}%
\pgfpathclose%
\pgfusepath{stroke,fill}%
\end{pgfscope}%
\begin{pgfscope}%
\pgfpathrectangle{\pgfqpoint{0.786164in}{0.768110in}}{\pgfqpoint{8.851069in}{7.081890in}}%
\pgfusepath{clip}%
\pgfsetbuttcap%
\pgfsetroundjoin%
\definecolor{currentfill}{rgb}{0.194100,0.399323,0.555565}%
\pgfsetfillcolor{currentfill}%
\pgfsetfillopacity{0.700000}%
\pgfsetlinewidth{0.501875pt}%
\definecolor{currentstroke}{rgb}{1.000000,1.000000,1.000000}%
\pgfsetstrokecolor{currentstroke}%
\pgfsetstrokeopacity{0.700000}%
\pgfsetdash{}{0pt}%
\pgfpathmoveto{\pgfqpoint{2.850744in}{2.989853in}}%
\pgfpathcurveto{\pgfqpoint{2.863766in}{2.989853in}}{\pgfqpoint{2.876257in}{2.995027in}}{\pgfqpoint{2.885466in}{3.004235in}}%
\pgfpathcurveto{\pgfqpoint{2.894674in}{3.013444in}}{\pgfqpoint{2.899848in}{3.025935in}}{\pgfqpoint{2.899848in}{3.038958in}}%
\pgfpathcurveto{\pgfqpoint{2.899848in}{3.051980in}}{\pgfqpoint{2.894674in}{3.064471in}}{\pgfqpoint{2.885466in}{3.073680in}}%
\pgfpathcurveto{\pgfqpoint{2.876257in}{3.082888in}}{\pgfqpoint{2.863766in}{3.088062in}}{\pgfqpoint{2.850744in}{3.088062in}}%
\pgfpathcurveto{\pgfqpoint{2.837721in}{3.088062in}}{\pgfqpoint{2.825230in}{3.082888in}}{\pgfqpoint{2.816021in}{3.073680in}}%
\pgfpathcurveto{\pgfqpoint{2.806813in}{3.064471in}}{\pgfqpoint{2.801639in}{3.051980in}}{\pgfqpoint{2.801639in}{3.038958in}}%
\pgfpathcurveto{\pgfqpoint{2.801639in}{3.025935in}}{\pgfqpoint{2.806813in}{3.013444in}}{\pgfqpoint{2.816021in}{3.004235in}}%
\pgfpathcurveto{\pgfqpoint{2.825230in}{2.995027in}}{\pgfqpoint{2.837721in}{2.989853in}}{\pgfqpoint{2.850744in}{2.989853in}}%
\pgfpathlineto{\pgfqpoint{2.850744in}{2.989853in}}%
\pgfpathclose%
\pgfusepath{stroke,fill}%
\end{pgfscope}%
\begin{pgfscope}%
\pgfpathrectangle{\pgfqpoint{0.786164in}{0.768110in}}{\pgfqpoint{8.851069in}{7.081890in}}%
\pgfusepath{clip}%
\pgfsetbuttcap%
\pgfsetroundjoin%
\definecolor{currentfill}{rgb}{0.195860,0.395433,0.555276}%
\pgfsetfillcolor{currentfill}%
\pgfsetfillopacity{0.700000}%
\pgfsetlinewidth{0.501875pt}%
\definecolor{currentstroke}{rgb}{1.000000,1.000000,1.000000}%
\pgfsetstrokecolor{currentstroke}%
\pgfsetstrokeopacity{0.700000}%
\pgfsetdash{}{0pt}%
\pgfpathmoveto{\pgfqpoint{2.805077in}{2.967955in}}%
\pgfpathcurveto{\pgfqpoint{2.818100in}{2.967955in}}{\pgfqpoint{2.830591in}{2.973129in}}{\pgfqpoint{2.839799in}{2.982337in}}%
\pgfpathcurveto{\pgfqpoint{2.849008in}{2.991545in}}{\pgfqpoint{2.854182in}{3.004037in}}{\pgfqpoint{2.854182in}{3.017059in}}%
\pgfpathcurveto{\pgfqpoint{2.854182in}{3.030082in}}{\pgfqpoint{2.849008in}{3.042573in}}{\pgfqpoint{2.839799in}{3.051781in}}%
\pgfpathcurveto{\pgfqpoint{2.830591in}{3.060990in}}{\pgfqpoint{2.818100in}{3.066164in}}{\pgfqpoint{2.805077in}{3.066164in}}%
\pgfpathcurveto{\pgfqpoint{2.792054in}{3.066164in}}{\pgfqpoint{2.779563in}{3.060990in}}{\pgfqpoint{2.770355in}{3.051781in}}%
\pgfpathcurveto{\pgfqpoint{2.761146in}{3.042573in}}{\pgfqpoint{2.755972in}{3.030082in}}{\pgfqpoint{2.755972in}{3.017059in}}%
\pgfpathcurveto{\pgfqpoint{2.755972in}{3.004037in}}{\pgfqpoint{2.761146in}{2.991545in}}{\pgfqpoint{2.770355in}{2.982337in}}%
\pgfpathcurveto{\pgfqpoint{2.779563in}{2.973129in}}{\pgfqpoint{2.792054in}{2.967955in}}{\pgfqpoint{2.805077in}{2.967955in}}%
\pgfpathlineto{\pgfqpoint{2.805077in}{2.967955in}}%
\pgfpathclose%
\pgfusepath{stroke,fill}%
\end{pgfscope}%
\begin{pgfscope}%
\pgfpathrectangle{\pgfqpoint{0.786164in}{0.768110in}}{\pgfqpoint{8.851069in}{7.081890in}}%
\pgfusepath{clip}%
\pgfsetbuttcap%
\pgfsetroundjoin%
\definecolor{currentfill}{rgb}{0.192357,0.403199,0.555836}%
\pgfsetfillcolor{currentfill}%
\pgfsetfillopacity{0.700000}%
\pgfsetlinewidth{0.501875pt}%
\definecolor{currentstroke}{rgb}{1.000000,1.000000,1.000000}%
\pgfsetstrokecolor{currentstroke}%
\pgfsetstrokeopacity{0.700000}%
\pgfsetdash{}{0pt}%
\pgfpathmoveto{\pgfqpoint{2.722877in}{2.748972in}}%
\pgfpathcurveto{\pgfqpoint{2.735900in}{2.748972in}}{\pgfqpoint{2.748391in}{2.754146in}}{\pgfqpoint{2.757600in}{2.763355in}}%
\pgfpathcurveto{\pgfqpoint{2.766808in}{2.772563in}}{\pgfqpoint{2.771982in}{2.785054in}}{\pgfqpoint{2.771982in}{2.798077in}}%
\pgfpathcurveto{\pgfqpoint{2.771982in}{2.811100in}}{\pgfqpoint{2.766808in}{2.823591in}}{\pgfqpoint{2.757600in}{2.832799in}}%
\pgfpathcurveto{\pgfqpoint{2.748391in}{2.842008in}}{\pgfqpoint{2.735900in}{2.847182in}}{\pgfqpoint{2.722877in}{2.847182in}}%
\pgfpathcurveto{\pgfqpoint{2.709855in}{2.847182in}}{\pgfqpoint{2.697364in}{2.842008in}}{\pgfqpoint{2.688155in}{2.832799in}}%
\pgfpathcurveto{\pgfqpoint{2.678947in}{2.823591in}}{\pgfqpoint{2.673773in}{2.811100in}}{\pgfqpoint{2.673773in}{2.798077in}}%
\pgfpathcurveto{\pgfqpoint{2.673773in}{2.785054in}}{\pgfqpoint{2.678947in}{2.772563in}}{\pgfqpoint{2.688155in}{2.763355in}}%
\pgfpathcurveto{\pgfqpoint{2.697364in}{2.754146in}}{\pgfqpoint{2.709855in}{2.748972in}}{\pgfqpoint{2.722877in}{2.748972in}}%
\pgfpathlineto{\pgfqpoint{2.722877in}{2.748972in}}%
\pgfpathclose%
\pgfusepath{stroke,fill}%
\end{pgfscope}%
\begin{pgfscope}%
\pgfpathrectangle{\pgfqpoint{0.786164in}{0.768110in}}{\pgfqpoint{8.851069in}{7.081890in}}%
\pgfusepath{clip}%
\pgfsetbuttcap%
\pgfsetroundjoin%
\definecolor{currentfill}{rgb}{0.172719,0.448791,0.557885}%
\pgfsetfillcolor{currentfill}%
\pgfsetfillopacity{0.700000}%
\pgfsetlinewidth{0.501875pt}%
\definecolor{currentstroke}{rgb}{1.000000,1.000000,1.000000}%
\pgfsetstrokecolor{currentstroke}%
\pgfsetstrokeopacity{0.700000}%
\pgfsetdash{}{0pt}%
\pgfpathmoveto{\pgfqpoint{2.549345in}{2.639481in}}%
\pgfpathcurveto{\pgfqpoint{2.562368in}{2.639481in}}{\pgfqpoint{2.574859in}{2.644655in}}{\pgfqpoint{2.584067in}{2.653864in}}%
\pgfpathcurveto{\pgfqpoint{2.593276in}{2.663072in}}{\pgfqpoint{2.598450in}{2.675563in}}{\pgfqpoint{2.598450in}{2.688586in}}%
\pgfpathcurveto{\pgfqpoint{2.598450in}{2.701608in}}{\pgfqpoint{2.593276in}{2.714100in}}{\pgfqpoint{2.584067in}{2.723308in}}%
\pgfpathcurveto{\pgfqpoint{2.574859in}{2.732516in}}{\pgfqpoint{2.562368in}{2.737690in}}{\pgfqpoint{2.549345in}{2.737690in}}%
\pgfpathcurveto{\pgfqpoint{2.536322in}{2.737690in}}{\pgfqpoint{2.523831in}{2.732516in}}{\pgfqpoint{2.514623in}{2.723308in}}%
\pgfpathcurveto{\pgfqpoint{2.505414in}{2.714100in}}{\pgfqpoint{2.500240in}{2.701608in}}{\pgfqpoint{2.500240in}{2.688586in}}%
\pgfpathcurveto{\pgfqpoint{2.500240in}{2.675563in}}{\pgfqpoint{2.505414in}{2.663072in}}{\pgfqpoint{2.514623in}{2.653864in}}%
\pgfpathcurveto{\pgfqpoint{2.523831in}{2.644655in}}{\pgfqpoint{2.536322in}{2.639481in}}{\pgfqpoint{2.549345in}{2.639481in}}%
\pgfpathlineto{\pgfqpoint{2.549345in}{2.639481in}}%
\pgfpathclose%
\pgfusepath{stroke,fill}%
\end{pgfscope}%
\begin{pgfscope}%
\pgfpathrectangle{\pgfqpoint{0.786164in}{0.768110in}}{\pgfqpoint{8.851069in}{7.081890in}}%
\pgfusepath{clip}%
\pgfsetbuttcap%
\pgfsetroundjoin%
\definecolor{currentfill}{rgb}{0.172719,0.448791,0.557885}%
\pgfsetfillcolor{currentfill}%
\pgfsetfillopacity{0.700000}%
\pgfsetlinewidth{0.501875pt}%
\definecolor{currentstroke}{rgb}{1.000000,1.000000,1.000000}%
\pgfsetstrokecolor{currentstroke}%
\pgfsetstrokeopacity{0.700000}%
\pgfsetdash{}{0pt}%
\pgfpathmoveto{\pgfqpoint{2.613278in}{2.661379in}}%
\pgfpathcurveto{\pgfqpoint{2.626301in}{2.661379in}}{\pgfqpoint{2.638792in}{2.666553in}}{\pgfqpoint{2.648000in}{2.675762in}}%
\pgfpathcurveto{\pgfqpoint{2.657209in}{2.684970in}}{\pgfqpoint{2.662383in}{2.697461in}}{\pgfqpoint{2.662383in}{2.710484in}}%
\pgfpathcurveto{\pgfqpoint{2.662383in}{2.723507in}}{\pgfqpoint{2.657209in}{2.735998in}}{\pgfqpoint{2.648000in}{2.745206in}}%
\pgfpathcurveto{\pgfqpoint{2.638792in}{2.754415in}}{\pgfqpoint{2.626301in}{2.759589in}}{\pgfqpoint{2.613278in}{2.759589in}}%
\pgfpathcurveto{\pgfqpoint{2.600255in}{2.759589in}}{\pgfqpoint{2.587764in}{2.754415in}}{\pgfqpoint{2.578556in}{2.745206in}}%
\pgfpathcurveto{\pgfqpoint{2.569347in}{2.735998in}}{\pgfqpoint{2.564173in}{2.723507in}}{\pgfqpoint{2.564173in}{2.710484in}}%
\pgfpathcurveto{\pgfqpoint{2.564173in}{2.697461in}}{\pgfqpoint{2.569347in}{2.684970in}}{\pgfqpoint{2.578556in}{2.675762in}}%
\pgfpathcurveto{\pgfqpoint{2.587764in}{2.666553in}}{\pgfqpoint{2.600255in}{2.661379in}}{\pgfqpoint{2.613278in}{2.661379in}}%
\pgfpathlineto{\pgfqpoint{2.613278in}{2.661379in}}%
\pgfpathclose%
\pgfusepath{stroke,fill}%
\end{pgfscope}%
\begin{pgfscope}%
\pgfpathrectangle{\pgfqpoint{0.786164in}{0.768110in}}{\pgfqpoint{8.851069in}{7.081890in}}%
\pgfusepath{clip}%
\pgfsetbuttcap%
\pgfsetroundjoin%
\definecolor{currentfill}{rgb}{0.172719,0.448791,0.557885}%
\pgfsetfillcolor{currentfill}%
\pgfsetfillopacity{0.700000}%
\pgfsetlinewidth{0.501875pt}%
\definecolor{currentstroke}{rgb}{1.000000,1.000000,1.000000}%
\pgfsetstrokecolor{currentstroke}%
\pgfsetstrokeopacity{0.700000}%
\pgfsetdash{}{0pt}%
\pgfpathmoveto{\pgfqpoint{2.549345in}{2.617583in}}%
\pgfpathcurveto{\pgfqpoint{2.562368in}{2.617583in}}{\pgfqpoint{2.574859in}{2.622757in}}{\pgfqpoint{2.584067in}{2.631965in}}%
\pgfpathcurveto{\pgfqpoint{2.593276in}{2.641174in}}{\pgfqpoint{2.598450in}{2.653665in}}{\pgfqpoint{2.598450in}{2.666687in}}%
\pgfpathcurveto{\pgfqpoint{2.598450in}{2.679710in}}{\pgfqpoint{2.593276in}{2.692201in}}{\pgfqpoint{2.584067in}{2.701410in}}%
\pgfpathcurveto{\pgfqpoint{2.574859in}{2.710618in}}{\pgfqpoint{2.562368in}{2.715792in}}{\pgfqpoint{2.549345in}{2.715792in}}%
\pgfpathcurveto{\pgfqpoint{2.536322in}{2.715792in}}{\pgfqpoint{2.523831in}{2.710618in}}{\pgfqpoint{2.514623in}{2.701410in}}%
\pgfpathcurveto{\pgfqpoint{2.505414in}{2.692201in}}{\pgfqpoint{2.500240in}{2.679710in}}{\pgfqpoint{2.500240in}{2.666687in}}%
\pgfpathcurveto{\pgfqpoint{2.500240in}{2.653665in}}{\pgfqpoint{2.505414in}{2.641174in}}{\pgfqpoint{2.514623in}{2.631965in}}%
\pgfpathcurveto{\pgfqpoint{2.523831in}{2.622757in}}{\pgfqpoint{2.536322in}{2.617583in}}{\pgfqpoint{2.549345in}{2.617583in}}%
\pgfpathlineto{\pgfqpoint{2.549345in}{2.617583in}}%
\pgfpathclose%
\pgfusepath{stroke,fill}%
\end{pgfscope}%
\begin{pgfscope}%
\pgfpathrectangle{\pgfqpoint{0.786164in}{0.768110in}}{\pgfqpoint{8.851069in}{7.081890in}}%
\pgfusepath{clip}%
\pgfsetbuttcap%
\pgfsetroundjoin%
\definecolor{currentfill}{rgb}{0.282884,0.135920,0.453427}%
\pgfsetfillcolor{currentfill}%
\pgfsetfillopacity{0.700000}%
\pgfsetlinewidth{0.501875pt}%
\definecolor{currentstroke}{rgb}{1.000000,1.000000,1.000000}%
\pgfsetstrokecolor{currentstroke}%
\pgfsetstrokeopacity{0.700000}%
\pgfsetdash{}{0pt}%
\pgfpathmoveto{\pgfqpoint{2.750277in}{2.748972in}}%
\pgfpathcurveto{\pgfqpoint{2.763300in}{2.748972in}}{\pgfqpoint{2.775791in}{2.754146in}}{\pgfqpoint{2.785000in}{2.763355in}}%
\pgfpathcurveto{\pgfqpoint{2.794208in}{2.772563in}}{\pgfqpoint{2.799382in}{2.785054in}}{\pgfqpoint{2.799382in}{2.798077in}}%
\pgfpathcurveto{\pgfqpoint{2.799382in}{2.811100in}}{\pgfqpoint{2.794208in}{2.823591in}}{\pgfqpoint{2.785000in}{2.832799in}}%
\pgfpathcurveto{\pgfqpoint{2.775791in}{2.842008in}}{\pgfqpoint{2.763300in}{2.847182in}}{\pgfqpoint{2.750277in}{2.847182in}}%
\pgfpathcurveto{\pgfqpoint{2.737255in}{2.847182in}}{\pgfqpoint{2.724764in}{2.842008in}}{\pgfqpoint{2.715555in}{2.832799in}}%
\pgfpathcurveto{\pgfqpoint{2.706347in}{2.823591in}}{\pgfqpoint{2.701173in}{2.811100in}}{\pgfqpoint{2.701173in}{2.798077in}}%
\pgfpathcurveto{\pgfqpoint{2.701173in}{2.785054in}}{\pgfqpoint{2.706347in}{2.772563in}}{\pgfqpoint{2.715555in}{2.763355in}}%
\pgfpathcurveto{\pgfqpoint{2.724764in}{2.754146in}}{\pgfqpoint{2.737255in}{2.748972in}}{\pgfqpoint{2.750277in}{2.748972in}}%
\pgfpathlineto{\pgfqpoint{2.750277in}{2.748972in}}%
\pgfpathclose%
\pgfusepath{stroke,fill}%
\end{pgfscope}%
\begin{pgfscope}%
\pgfpathrectangle{\pgfqpoint{0.786164in}{0.768110in}}{\pgfqpoint{8.851069in}{7.081890in}}%
\pgfusepath{clip}%
\pgfsetbuttcap%
\pgfsetroundjoin%
\definecolor{currentfill}{rgb}{0.282884,0.135920,0.453427}%
\pgfsetfillcolor{currentfill}%
\pgfsetfillopacity{0.700000}%
\pgfsetlinewidth{0.501875pt}%
\definecolor{currentstroke}{rgb}{1.000000,1.000000,1.000000}%
\pgfsetstrokecolor{currentstroke}%
\pgfsetstrokeopacity{0.700000}%
\pgfsetdash{}{0pt}%
\pgfpathmoveto{\pgfqpoint{2.832477in}{2.924158in}}%
\pgfpathcurveto{\pgfqpoint{2.845500in}{2.924158in}}{\pgfqpoint{2.857991in}{2.929332in}}{\pgfqpoint{2.867199in}{2.938541in}}%
\pgfpathcurveto{\pgfqpoint{2.876408in}{2.947749in}}{\pgfqpoint{2.881582in}{2.960240in}}{\pgfqpoint{2.881582in}{2.973263in}}%
\pgfpathcurveto{\pgfqpoint{2.881582in}{2.986286in}}{\pgfqpoint{2.876408in}{2.998777in}}{\pgfqpoint{2.867199in}{3.007985in}}%
\pgfpathcurveto{\pgfqpoint{2.857991in}{3.017193in}}{\pgfqpoint{2.845500in}{3.022367in}}{\pgfqpoint{2.832477in}{3.022367in}}%
\pgfpathcurveto{\pgfqpoint{2.819454in}{3.022367in}}{\pgfqpoint{2.806963in}{3.017193in}}{\pgfqpoint{2.797755in}{3.007985in}}%
\pgfpathcurveto{\pgfqpoint{2.788546in}{2.998777in}}{\pgfqpoint{2.783372in}{2.986286in}}{\pgfqpoint{2.783372in}{2.973263in}}%
\pgfpathcurveto{\pgfqpoint{2.783372in}{2.960240in}}{\pgfqpoint{2.788546in}{2.947749in}}{\pgfqpoint{2.797755in}{2.938541in}}%
\pgfpathcurveto{\pgfqpoint{2.806963in}{2.929332in}}{\pgfqpoint{2.819454in}{2.924158in}}{\pgfqpoint{2.832477in}{2.924158in}}%
\pgfpathlineto{\pgfqpoint{2.832477in}{2.924158in}}%
\pgfpathclose%
\pgfusepath{stroke,fill}%
\end{pgfscope}%
\begin{pgfscope}%
\pgfpathrectangle{\pgfqpoint{0.786164in}{0.768110in}}{\pgfqpoint{8.851069in}{7.081890in}}%
\pgfusepath{clip}%
\pgfsetbuttcap%
\pgfsetroundjoin%
\definecolor{currentfill}{rgb}{0.282623,0.140926,0.457517}%
\pgfsetfillcolor{currentfill}%
\pgfsetfillopacity{0.700000}%
\pgfsetlinewidth{0.501875pt}%
\definecolor{currentstroke}{rgb}{1.000000,1.000000,1.000000}%
\pgfsetstrokecolor{currentstroke}%
\pgfsetstrokeopacity{0.700000}%
\pgfsetdash{}{0pt}%
\pgfpathmoveto{\pgfqpoint{2.805077in}{2.792769in}}%
\pgfpathcurveto{\pgfqpoint{2.818100in}{2.792769in}}{\pgfqpoint{2.830591in}{2.797943in}}{\pgfqpoint{2.839799in}{2.807151in}}%
\pgfpathcurveto{\pgfqpoint{2.849008in}{2.816360in}}{\pgfqpoint{2.854182in}{2.828851in}}{\pgfqpoint{2.854182in}{2.841873in}}%
\pgfpathcurveto{\pgfqpoint{2.854182in}{2.854896in}}{\pgfqpoint{2.849008in}{2.867387in}}{\pgfqpoint{2.839799in}{2.876596in}}%
\pgfpathcurveto{\pgfqpoint{2.830591in}{2.885804in}}{\pgfqpoint{2.818100in}{2.890978in}}{\pgfqpoint{2.805077in}{2.890978in}}%
\pgfpathcurveto{\pgfqpoint{2.792054in}{2.890978in}}{\pgfqpoint{2.779563in}{2.885804in}}{\pgfqpoint{2.770355in}{2.876596in}}%
\pgfpathcurveto{\pgfqpoint{2.761146in}{2.867387in}}{\pgfqpoint{2.755972in}{2.854896in}}{\pgfqpoint{2.755972in}{2.841873in}}%
\pgfpathcurveto{\pgfqpoint{2.755972in}{2.828851in}}{\pgfqpoint{2.761146in}{2.816360in}}{\pgfqpoint{2.770355in}{2.807151in}}%
\pgfpathcurveto{\pgfqpoint{2.779563in}{2.797943in}}{\pgfqpoint{2.792054in}{2.792769in}}{\pgfqpoint{2.805077in}{2.792769in}}%
\pgfpathlineto{\pgfqpoint{2.805077in}{2.792769in}}%
\pgfpathclose%
\pgfusepath{stroke,fill}%
\end{pgfscope}%
\begin{pgfscope}%
\pgfpathrectangle{\pgfqpoint{0.786164in}{0.768110in}}{\pgfqpoint{8.851069in}{7.081890in}}%
\pgfusepath{clip}%
\pgfsetbuttcap%
\pgfsetroundjoin%
\definecolor{currentfill}{rgb}{0.280868,0.160771,0.472899}%
\pgfsetfillcolor{currentfill}%
\pgfsetfillopacity{0.700000}%
\pgfsetlinewidth{0.501875pt}%
\definecolor{currentstroke}{rgb}{1.000000,1.000000,1.000000}%
\pgfsetstrokecolor{currentstroke}%
\pgfsetstrokeopacity{0.700000}%
\pgfsetdash{}{0pt}%
\pgfpathmoveto{\pgfqpoint{2.668078in}{2.727074in}}%
\pgfpathcurveto{\pgfqpoint{2.681100in}{2.727074in}}{\pgfqpoint{2.693592in}{2.732248in}}{\pgfqpoint{2.702800in}{2.741456in}}%
\pgfpathcurveto{\pgfqpoint{2.712008in}{2.750665in}}{\pgfqpoint{2.717182in}{2.763156in}}{\pgfqpoint{2.717182in}{2.776179in}}%
\pgfpathcurveto{\pgfqpoint{2.717182in}{2.789201in}}{\pgfqpoint{2.712008in}{2.801692in}}{\pgfqpoint{2.702800in}{2.810901in}}%
\pgfpathcurveto{\pgfqpoint{2.693592in}{2.820109in}}{\pgfqpoint{2.681100in}{2.825283in}}{\pgfqpoint{2.668078in}{2.825283in}}%
\pgfpathcurveto{\pgfqpoint{2.655055in}{2.825283in}}{\pgfqpoint{2.642564in}{2.820109in}}{\pgfqpoint{2.633356in}{2.810901in}}%
\pgfpathcurveto{\pgfqpoint{2.624147in}{2.801692in}}{\pgfqpoint{2.618973in}{2.789201in}}{\pgfqpoint{2.618973in}{2.776179in}}%
\pgfpathcurveto{\pgfqpoint{2.618973in}{2.763156in}}{\pgfqpoint{2.624147in}{2.750665in}}{\pgfqpoint{2.633356in}{2.741456in}}%
\pgfpathcurveto{\pgfqpoint{2.642564in}{2.732248in}}{\pgfqpoint{2.655055in}{2.727074in}}{\pgfqpoint{2.668078in}{2.727074in}}%
\pgfpathlineto{\pgfqpoint{2.668078in}{2.727074in}}%
\pgfpathclose%
\pgfusepath{stroke,fill}%
\end{pgfscope}%
\begin{pgfscope}%
\pgfpathrectangle{\pgfqpoint{0.786164in}{0.768110in}}{\pgfqpoint{8.851069in}{7.081890in}}%
\pgfusepath{clip}%
\pgfsetbuttcap%
\pgfsetroundjoin%
\definecolor{currentfill}{rgb}{0.280868,0.160771,0.472899}%
\pgfsetfillcolor{currentfill}%
\pgfsetfillopacity{0.700000}%
\pgfsetlinewidth{0.501875pt}%
\definecolor{currentstroke}{rgb}{1.000000,1.000000,1.000000}%
\pgfsetstrokecolor{currentstroke}%
\pgfsetstrokeopacity{0.700000}%
\pgfsetdash{}{0pt}%
\pgfpathmoveto{\pgfqpoint{2.759411in}{2.792769in}}%
\pgfpathcurveto{\pgfqpoint{2.772433in}{2.792769in}}{\pgfqpoint{2.784924in}{2.797943in}}{\pgfqpoint{2.794133in}{2.807151in}}%
\pgfpathcurveto{\pgfqpoint{2.803341in}{2.816360in}}{\pgfqpoint{2.808515in}{2.828851in}}{\pgfqpoint{2.808515in}{2.841873in}}%
\pgfpathcurveto{\pgfqpoint{2.808515in}{2.854896in}}{\pgfqpoint{2.803341in}{2.867387in}}{\pgfqpoint{2.794133in}{2.876596in}}%
\pgfpathcurveto{\pgfqpoint{2.784924in}{2.885804in}}{\pgfqpoint{2.772433in}{2.890978in}}{\pgfqpoint{2.759411in}{2.890978in}}%
\pgfpathcurveto{\pgfqpoint{2.746388in}{2.890978in}}{\pgfqpoint{2.733897in}{2.885804in}}{\pgfqpoint{2.724688in}{2.876596in}}%
\pgfpathcurveto{\pgfqpoint{2.715480in}{2.867387in}}{\pgfqpoint{2.710306in}{2.854896in}}{\pgfqpoint{2.710306in}{2.841873in}}%
\pgfpathcurveto{\pgfqpoint{2.710306in}{2.828851in}}{\pgfqpoint{2.715480in}{2.816360in}}{\pgfqpoint{2.724688in}{2.807151in}}%
\pgfpathcurveto{\pgfqpoint{2.733897in}{2.797943in}}{\pgfqpoint{2.746388in}{2.792769in}}{\pgfqpoint{2.759411in}{2.792769in}}%
\pgfpathlineto{\pgfqpoint{2.759411in}{2.792769in}}%
\pgfpathclose%
\pgfusepath{stroke,fill}%
\end{pgfscope}%
\begin{pgfscope}%
\pgfpathrectangle{\pgfqpoint{0.786164in}{0.768110in}}{\pgfqpoint{8.851069in}{7.081890in}}%
\pgfusepath{clip}%
\pgfsetbuttcap%
\pgfsetroundjoin%
\definecolor{currentfill}{rgb}{0.276194,0.190074,0.493001}%
\pgfsetfillcolor{currentfill}%
\pgfsetfillopacity{0.700000}%
\pgfsetlinewidth{0.501875pt}%
\definecolor{currentstroke}{rgb}{1.000000,1.000000,1.000000}%
\pgfsetstrokecolor{currentstroke}%
\pgfsetstrokeopacity{0.700000}%
\pgfsetdash{}{0pt}%
\pgfpathmoveto{\pgfqpoint{2.503679in}{2.639481in}}%
\pgfpathcurveto{\pgfqpoint{2.516701in}{2.639481in}}{\pgfqpoint{2.529192in}{2.644655in}}{\pgfqpoint{2.538401in}{2.653864in}}%
\pgfpathcurveto{\pgfqpoint{2.547609in}{2.663072in}}{\pgfqpoint{2.552783in}{2.675563in}}{\pgfqpoint{2.552783in}{2.688586in}}%
\pgfpathcurveto{\pgfqpoint{2.552783in}{2.701608in}}{\pgfqpoint{2.547609in}{2.714100in}}{\pgfqpoint{2.538401in}{2.723308in}}%
\pgfpathcurveto{\pgfqpoint{2.529192in}{2.732516in}}{\pgfqpoint{2.516701in}{2.737690in}}{\pgfqpoint{2.503679in}{2.737690in}}%
\pgfpathcurveto{\pgfqpoint{2.490656in}{2.737690in}}{\pgfqpoint{2.478165in}{2.732516in}}{\pgfqpoint{2.468956in}{2.723308in}}%
\pgfpathcurveto{\pgfqpoint{2.459748in}{2.714100in}}{\pgfqpoint{2.454574in}{2.701608in}}{\pgfqpoint{2.454574in}{2.688586in}}%
\pgfpathcurveto{\pgfqpoint{2.454574in}{2.675563in}}{\pgfqpoint{2.459748in}{2.663072in}}{\pgfqpoint{2.468956in}{2.653864in}}%
\pgfpathcurveto{\pgfqpoint{2.478165in}{2.644655in}}{\pgfqpoint{2.490656in}{2.639481in}}{\pgfqpoint{2.503679in}{2.639481in}}%
\pgfpathlineto{\pgfqpoint{2.503679in}{2.639481in}}%
\pgfpathclose%
\pgfusepath{stroke,fill}%
\end{pgfscope}%
\begin{pgfscope}%
\pgfpathrectangle{\pgfqpoint{0.786164in}{0.768110in}}{\pgfqpoint{8.851069in}{7.081890in}}%
\pgfusepath{clip}%
\pgfsetbuttcap%
\pgfsetroundjoin%
\definecolor{currentfill}{rgb}{0.277134,0.185228,0.489898}%
\pgfsetfillcolor{currentfill}%
\pgfsetfillopacity{0.700000}%
\pgfsetlinewidth{0.501875pt}%
\definecolor{currentstroke}{rgb}{1.000000,1.000000,1.000000}%
\pgfsetstrokecolor{currentstroke}%
\pgfsetstrokeopacity{0.700000}%
\pgfsetdash{}{0pt}%
\pgfpathmoveto{\pgfqpoint{2.814210in}{2.705176in}}%
\pgfpathcurveto{\pgfqpoint{2.827233in}{2.705176in}}{\pgfqpoint{2.839724in}{2.710350in}}{\pgfqpoint{2.848933in}{2.719558in}}%
\pgfpathcurveto{\pgfqpoint{2.858141in}{2.728767in}}{\pgfqpoint{2.863315in}{2.741258in}}{\pgfqpoint{2.863315in}{2.754280in}}%
\pgfpathcurveto{\pgfqpoint{2.863315in}{2.767303in}}{\pgfqpoint{2.858141in}{2.779794in}}{\pgfqpoint{2.848933in}{2.789003in}}%
\pgfpathcurveto{\pgfqpoint{2.839724in}{2.798211in}}{\pgfqpoint{2.827233in}{2.803385in}}{\pgfqpoint{2.814210in}{2.803385in}}%
\pgfpathcurveto{\pgfqpoint{2.801188in}{2.803385in}}{\pgfqpoint{2.788697in}{2.798211in}}{\pgfqpoint{2.779488in}{2.789003in}}%
\pgfpathcurveto{\pgfqpoint{2.770280in}{2.779794in}}{\pgfqpoint{2.765106in}{2.767303in}}{\pgfqpoint{2.765106in}{2.754280in}}%
\pgfpathcurveto{\pgfqpoint{2.765106in}{2.741258in}}{\pgfqpoint{2.770280in}{2.728767in}}{\pgfqpoint{2.779488in}{2.719558in}}%
\pgfpathcurveto{\pgfqpoint{2.788697in}{2.710350in}}{\pgfqpoint{2.801188in}{2.705176in}}{\pgfqpoint{2.814210in}{2.705176in}}%
\pgfpathlineto{\pgfqpoint{2.814210in}{2.705176in}}%
\pgfpathclose%
\pgfusepath{stroke,fill}%
\end{pgfscope}%
\begin{pgfscope}%
\pgfpathrectangle{\pgfqpoint{0.786164in}{0.768110in}}{\pgfqpoint{8.851069in}{7.081890in}}%
\pgfusepath{clip}%
\pgfsetbuttcap%
\pgfsetroundjoin%
\definecolor{currentfill}{rgb}{0.271828,0.209303,0.504434}%
\pgfsetfillcolor{currentfill}%
\pgfsetfillopacity{0.700000}%
\pgfsetlinewidth{0.501875pt}%
\definecolor{currentstroke}{rgb}{1.000000,1.000000,1.000000}%
\pgfsetstrokecolor{currentstroke}%
\pgfsetstrokeopacity{0.700000}%
\pgfsetdash{}{0pt}%
\pgfpathmoveto{\pgfqpoint{2.686344in}{2.639481in}}%
\pgfpathcurveto{\pgfqpoint{2.699367in}{2.639481in}}{\pgfqpoint{2.711858in}{2.644655in}}{\pgfqpoint{2.721067in}{2.653864in}}%
\pgfpathcurveto{\pgfqpoint{2.730275in}{2.663072in}}{\pgfqpoint{2.735449in}{2.675563in}}{\pgfqpoint{2.735449in}{2.688586in}}%
\pgfpathcurveto{\pgfqpoint{2.735449in}{2.701608in}}{\pgfqpoint{2.730275in}{2.714100in}}{\pgfqpoint{2.721067in}{2.723308in}}%
\pgfpathcurveto{\pgfqpoint{2.711858in}{2.732516in}}{\pgfqpoint{2.699367in}{2.737690in}}{\pgfqpoint{2.686344in}{2.737690in}}%
\pgfpathcurveto{\pgfqpoint{2.673322in}{2.737690in}}{\pgfqpoint{2.660831in}{2.732516in}}{\pgfqpoint{2.651622in}{2.723308in}}%
\pgfpathcurveto{\pgfqpoint{2.642414in}{2.714100in}}{\pgfqpoint{2.637240in}{2.701608in}}{\pgfqpoint{2.637240in}{2.688586in}}%
\pgfpathcurveto{\pgfqpoint{2.637240in}{2.675563in}}{\pgfqpoint{2.642414in}{2.663072in}}{\pgfqpoint{2.651622in}{2.653864in}}%
\pgfpathcurveto{\pgfqpoint{2.660831in}{2.644655in}}{\pgfqpoint{2.673322in}{2.639481in}}{\pgfqpoint{2.686344in}{2.639481in}}%
\pgfpathlineto{\pgfqpoint{2.686344in}{2.639481in}}%
\pgfpathclose%
\pgfusepath{stroke,fill}%
\end{pgfscope}%
\begin{pgfscope}%
\pgfpathrectangle{\pgfqpoint{0.786164in}{0.768110in}}{\pgfqpoint{8.851069in}{7.081890in}}%
\pgfusepath{clip}%
\pgfsetbuttcap%
\pgfsetroundjoin%
\definecolor{currentfill}{rgb}{0.270595,0.214069,0.507052}%
\pgfsetfillcolor{currentfill}%
\pgfsetfillopacity{0.700000}%
\pgfsetlinewidth{0.501875pt}%
\definecolor{currentstroke}{rgb}{1.000000,1.000000,1.000000}%
\pgfsetstrokecolor{currentstroke}%
\pgfsetstrokeopacity{0.700000}%
\pgfsetdash{}{0pt}%
\pgfpathmoveto{\pgfqpoint{2.631545in}{2.486193in}}%
\pgfpathcurveto{\pgfqpoint{2.644567in}{2.486193in}}{\pgfqpoint{2.657058in}{2.491367in}}{\pgfqpoint{2.666267in}{2.500576in}}%
\pgfpathcurveto{\pgfqpoint{2.675475in}{2.509784in}}{\pgfqpoint{2.680649in}{2.522275in}}{\pgfqpoint{2.680649in}{2.535298in}}%
\pgfpathcurveto{\pgfqpoint{2.680649in}{2.548321in}}{\pgfqpoint{2.675475in}{2.560812in}}{\pgfqpoint{2.666267in}{2.570020in}}%
\pgfpathcurveto{\pgfqpoint{2.657058in}{2.579229in}}{\pgfqpoint{2.644567in}{2.584403in}}{\pgfqpoint{2.631545in}{2.584403in}}%
\pgfpathcurveto{\pgfqpoint{2.618522in}{2.584403in}}{\pgfqpoint{2.606031in}{2.579229in}}{\pgfqpoint{2.596822in}{2.570020in}}%
\pgfpathcurveto{\pgfqpoint{2.587614in}{2.560812in}}{\pgfqpoint{2.582440in}{2.548321in}}{\pgfqpoint{2.582440in}{2.535298in}}%
\pgfpathcurveto{\pgfqpoint{2.582440in}{2.522275in}}{\pgfqpoint{2.587614in}{2.509784in}}{\pgfqpoint{2.596822in}{2.500576in}}%
\pgfpathcurveto{\pgfqpoint{2.606031in}{2.491367in}}{\pgfqpoint{2.618522in}{2.486193in}}{\pgfqpoint{2.631545in}{2.486193in}}%
\pgfpathlineto{\pgfqpoint{2.631545in}{2.486193in}}%
\pgfpathclose%
\pgfusepath{stroke,fill}%
\end{pgfscope}%
\begin{pgfscope}%
\pgfpathrectangle{\pgfqpoint{0.786164in}{0.768110in}}{\pgfqpoint{8.851069in}{7.081890in}}%
\pgfusepath{clip}%
\pgfsetbuttcap%
\pgfsetroundjoin%
\definecolor{currentfill}{rgb}{0.273006,0.204520,0.501721}%
\pgfsetfillcolor{currentfill}%
\pgfsetfillopacity{0.700000}%
\pgfsetlinewidth{0.501875pt}%
\definecolor{currentstroke}{rgb}{1.000000,1.000000,1.000000}%
\pgfsetstrokecolor{currentstroke}%
\pgfsetstrokeopacity{0.700000}%
\pgfsetdash{}{0pt}%
\pgfpathmoveto{\pgfqpoint{2.640678in}{2.442397in}}%
\pgfpathcurveto{\pgfqpoint{2.653701in}{2.442397in}}{\pgfqpoint{2.666192in}{2.447571in}}{\pgfqpoint{2.675400in}{2.456779in}}%
\pgfpathcurveto{\pgfqpoint{2.684609in}{2.465988in}}{\pgfqpoint{2.689783in}{2.478479in}}{\pgfqpoint{2.689783in}{2.491502in}}%
\pgfpathcurveto{\pgfqpoint{2.689783in}{2.504524in}}{\pgfqpoint{2.684609in}{2.517015in}}{\pgfqpoint{2.675400in}{2.526224in}}%
\pgfpathcurveto{\pgfqpoint{2.666192in}{2.535432in}}{\pgfqpoint{2.653701in}{2.540606in}}{\pgfqpoint{2.640678in}{2.540606in}}%
\pgfpathcurveto{\pgfqpoint{2.627655in}{2.540606in}}{\pgfqpoint{2.615164in}{2.535432in}}{\pgfqpoint{2.605956in}{2.526224in}}%
\pgfpathcurveto{\pgfqpoint{2.596747in}{2.517015in}}{\pgfqpoint{2.591573in}{2.504524in}}{\pgfqpoint{2.591573in}{2.491502in}}%
\pgfpathcurveto{\pgfqpoint{2.591573in}{2.478479in}}{\pgfqpoint{2.596747in}{2.465988in}}{\pgfqpoint{2.605956in}{2.456779in}}%
\pgfpathcurveto{\pgfqpoint{2.615164in}{2.447571in}}{\pgfqpoint{2.627655in}{2.442397in}}{\pgfqpoint{2.640678in}{2.442397in}}%
\pgfpathlineto{\pgfqpoint{2.640678in}{2.442397in}}%
\pgfpathclose%
\pgfusepath{stroke,fill}%
\end{pgfscope}%
\begin{pgfscope}%
\pgfpathrectangle{\pgfqpoint{0.786164in}{0.768110in}}{\pgfqpoint{8.851069in}{7.081890in}}%
\pgfusepath{clip}%
\pgfsetbuttcap%
\pgfsetroundjoin%
\definecolor{currentfill}{rgb}{0.263663,0.237631,0.518762}%
\pgfsetfillcolor{currentfill}%
\pgfsetfillopacity{0.700000}%
\pgfsetlinewidth{0.501875pt}%
\definecolor{currentstroke}{rgb}{1.000000,1.000000,1.000000}%
\pgfsetstrokecolor{currentstroke}%
\pgfsetstrokeopacity{0.700000}%
\pgfsetdash{}{0pt}%
\pgfpathmoveto{\pgfqpoint{2.439746in}{2.442397in}}%
\pgfpathcurveto{\pgfqpoint{2.452768in}{2.442397in}}{\pgfqpoint{2.465259in}{2.447571in}}{\pgfqpoint{2.474468in}{2.456779in}}%
\pgfpathcurveto{\pgfqpoint{2.483676in}{2.465988in}}{\pgfqpoint{2.488850in}{2.478479in}}{\pgfqpoint{2.488850in}{2.491502in}}%
\pgfpathcurveto{\pgfqpoint{2.488850in}{2.504524in}}{\pgfqpoint{2.483676in}{2.517015in}}{\pgfqpoint{2.474468in}{2.526224in}}%
\pgfpathcurveto{\pgfqpoint{2.465259in}{2.535432in}}{\pgfqpoint{2.452768in}{2.540606in}}{\pgfqpoint{2.439746in}{2.540606in}}%
\pgfpathcurveto{\pgfqpoint{2.426723in}{2.540606in}}{\pgfqpoint{2.414232in}{2.535432in}}{\pgfqpoint{2.405023in}{2.526224in}}%
\pgfpathcurveto{\pgfqpoint{2.395815in}{2.517015in}}{\pgfqpoint{2.390641in}{2.504524in}}{\pgfqpoint{2.390641in}{2.491502in}}%
\pgfpathcurveto{\pgfqpoint{2.390641in}{2.478479in}}{\pgfqpoint{2.395815in}{2.465988in}}{\pgfqpoint{2.405023in}{2.456779in}}%
\pgfpathcurveto{\pgfqpoint{2.414232in}{2.447571in}}{\pgfqpoint{2.426723in}{2.442397in}}{\pgfqpoint{2.439746in}{2.442397in}}%
\pgfpathlineto{\pgfqpoint{2.439746in}{2.442397in}}%
\pgfpathclose%
\pgfusepath{stroke,fill}%
\end{pgfscope}%
\begin{pgfscope}%
\pgfpathrectangle{\pgfqpoint{0.786164in}{0.768110in}}{\pgfqpoint{8.851069in}{7.081890in}}%
\pgfusepath{clip}%
\pgfsetbuttcap%
\pgfsetroundjoin%
\definecolor{currentfill}{rgb}{0.257322,0.256130,0.526563}%
\pgfsetfillcolor{currentfill}%
\pgfsetfillopacity{0.700000}%
\pgfsetlinewidth{0.501875pt}%
\definecolor{currentstroke}{rgb}{1.000000,1.000000,1.000000}%
\pgfsetstrokecolor{currentstroke}%
\pgfsetstrokeopacity{0.700000}%
\pgfsetdash{}{0pt}%
\pgfpathmoveto{\pgfqpoint{2.531078in}{2.464295in}}%
\pgfpathcurveto{\pgfqpoint{2.544101in}{2.464295in}}{\pgfqpoint{2.556592in}{2.469469in}}{\pgfqpoint{2.565801in}{2.478678in}}%
\pgfpathcurveto{\pgfqpoint{2.575009in}{2.487886in}}{\pgfqpoint{2.580183in}{2.500377in}}{\pgfqpoint{2.580183in}{2.513400in}}%
\pgfpathcurveto{\pgfqpoint{2.580183in}{2.526423in}}{\pgfqpoint{2.575009in}{2.538914in}}{\pgfqpoint{2.565801in}{2.548122in}}%
\pgfpathcurveto{\pgfqpoint{2.556592in}{2.557331in}}{\pgfqpoint{2.544101in}{2.562504in}}{\pgfqpoint{2.531078in}{2.562504in}}%
\pgfpathcurveto{\pgfqpoint{2.518056in}{2.562504in}}{\pgfqpoint{2.505565in}{2.557331in}}{\pgfqpoint{2.496356in}{2.548122in}}%
\pgfpathcurveto{\pgfqpoint{2.487148in}{2.538914in}}{\pgfqpoint{2.481974in}{2.526423in}}{\pgfqpoint{2.481974in}{2.513400in}}%
\pgfpathcurveto{\pgfqpoint{2.481974in}{2.500377in}}{\pgfqpoint{2.487148in}{2.487886in}}{\pgfqpoint{2.496356in}{2.478678in}}%
\pgfpathcurveto{\pgfqpoint{2.505565in}{2.469469in}}{\pgfqpoint{2.518056in}{2.464295in}}{\pgfqpoint{2.531078in}{2.464295in}}%
\pgfpathlineto{\pgfqpoint{2.531078in}{2.464295in}}%
\pgfpathclose%
\pgfusepath{stroke,fill}%
\end{pgfscope}%
\begin{pgfscope}%
\pgfpathrectangle{\pgfqpoint{0.786164in}{0.768110in}}{\pgfqpoint{8.851069in}{7.081890in}}%
\pgfusepath{clip}%
\pgfsetbuttcap%
\pgfsetroundjoin%
\definecolor{currentfill}{rgb}{0.262138,0.242286,0.520837}%
\pgfsetfillcolor{currentfill}%
\pgfsetfillopacity{0.700000}%
\pgfsetlinewidth{0.501875pt}%
\definecolor{currentstroke}{rgb}{1.000000,1.000000,1.000000}%
\pgfsetstrokecolor{currentstroke}%
\pgfsetstrokeopacity{0.700000}%
\pgfsetdash{}{0pt}%
\pgfpathmoveto{\pgfqpoint{2.567612in}{2.486193in}}%
\pgfpathcurveto{\pgfqpoint{2.580634in}{2.486193in}}{\pgfqpoint{2.593125in}{2.491367in}}{\pgfqpoint{2.602334in}{2.500576in}}%
\pgfpathcurveto{\pgfqpoint{2.611542in}{2.509784in}}{\pgfqpoint{2.616716in}{2.522275in}}{\pgfqpoint{2.616716in}{2.535298in}}%
\pgfpathcurveto{\pgfqpoint{2.616716in}{2.548321in}}{\pgfqpoint{2.611542in}{2.560812in}}{\pgfqpoint{2.602334in}{2.570020in}}%
\pgfpathcurveto{\pgfqpoint{2.593125in}{2.579229in}}{\pgfqpoint{2.580634in}{2.584403in}}{\pgfqpoint{2.567612in}{2.584403in}}%
\pgfpathcurveto{\pgfqpoint{2.554589in}{2.584403in}}{\pgfqpoint{2.542098in}{2.579229in}}{\pgfqpoint{2.532889in}{2.570020in}}%
\pgfpathcurveto{\pgfqpoint{2.523681in}{2.560812in}}{\pgfqpoint{2.518507in}{2.548321in}}{\pgfqpoint{2.518507in}{2.535298in}}%
\pgfpathcurveto{\pgfqpoint{2.518507in}{2.522275in}}{\pgfqpoint{2.523681in}{2.509784in}}{\pgfqpoint{2.532889in}{2.500576in}}%
\pgfpathcurveto{\pgfqpoint{2.542098in}{2.491367in}}{\pgfqpoint{2.554589in}{2.486193in}}{\pgfqpoint{2.567612in}{2.486193in}}%
\pgfpathlineto{\pgfqpoint{2.567612in}{2.486193in}}%
\pgfpathclose%
\pgfusepath{stroke,fill}%
\end{pgfscope}%
\begin{pgfscope}%
\pgfpathrectangle{\pgfqpoint{0.786164in}{0.768110in}}{\pgfqpoint{8.851069in}{7.081890in}}%
\pgfusepath{clip}%
\pgfsetbuttcap%
\pgfsetroundjoin%
\definecolor{currentfill}{rgb}{0.265145,0.232956,0.516599}%
\pgfsetfillcolor{currentfill}%
\pgfsetfillopacity{0.700000}%
\pgfsetlinewidth{0.501875pt}%
\definecolor{currentstroke}{rgb}{1.000000,1.000000,1.000000}%
\pgfsetstrokecolor{currentstroke}%
\pgfsetstrokeopacity{0.700000}%
\pgfsetdash{}{0pt}%
\pgfpathmoveto{\pgfqpoint{2.686344in}{2.529990in}}%
\pgfpathcurveto{\pgfqpoint{2.699367in}{2.529990in}}{\pgfqpoint{2.711858in}{2.535164in}}{\pgfqpoint{2.721067in}{2.544372in}}%
\pgfpathcurveto{\pgfqpoint{2.730275in}{2.553581in}}{\pgfqpoint{2.735449in}{2.566072in}}{\pgfqpoint{2.735449in}{2.579095in}}%
\pgfpathcurveto{\pgfqpoint{2.735449in}{2.592117in}}{\pgfqpoint{2.730275in}{2.604608in}}{\pgfqpoint{2.721067in}{2.613817in}}%
\pgfpathcurveto{\pgfqpoint{2.711858in}{2.623025in}}{\pgfqpoint{2.699367in}{2.628199in}}{\pgfqpoint{2.686344in}{2.628199in}}%
\pgfpathcurveto{\pgfqpoint{2.673322in}{2.628199in}}{\pgfqpoint{2.660831in}{2.623025in}}{\pgfqpoint{2.651622in}{2.613817in}}%
\pgfpathcurveto{\pgfqpoint{2.642414in}{2.604608in}}{\pgfqpoint{2.637240in}{2.592117in}}{\pgfqpoint{2.637240in}{2.579095in}}%
\pgfpathcurveto{\pgfqpoint{2.637240in}{2.566072in}}{\pgfqpoint{2.642414in}{2.553581in}}{\pgfqpoint{2.651622in}{2.544372in}}%
\pgfpathcurveto{\pgfqpoint{2.660831in}{2.535164in}}{\pgfqpoint{2.673322in}{2.529990in}}{\pgfqpoint{2.686344in}{2.529990in}}%
\pgfpathlineto{\pgfqpoint{2.686344in}{2.529990in}}%
\pgfpathclose%
\pgfusepath{stroke,fill}%
\end{pgfscope}%
\begin{pgfscope}%
\pgfpathrectangle{\pgfqpoint{0.786164in}{0.768110in}}{\pgfqpoint{8.851069in}{7.081890in}}%
\pgfusepath{clip}%
\pgfsetbuttcap%
\pgfsetroundjoin%
\definecolor{currentfill}{rgb}{0.263663,0.237631,0.518762}%
\pgfsetfillcolor{currentfill}%
\pgfsetfillopacity{0.700000}%
\pgfsetlinewidth{0.501875pt}%
\definecolor{currentstroke}{rgb}{1.000000,1.000000,1.000000}%
\pgfsetstrokecolor{currentstroke}%
\pgfsetstrokeopacity{0.700000}%
\pgfsetdash{}{0pt}%
\pgfpathmoveto{\pgfqpoint{2.604145in}{2.551888in}}%
\pgfpathcurveto{\pgfqpoint{2.617167in}{2.551888in}}{\pgfqpoint{2.629659in}{2.557062in}}{\pgfqpoint{2.638867in}{2.566271in}}%
\pgfpathcurveto{\pgfqpoint{2.648075in}{2.575479in}}{\pgfqpoint{2.653249in}{2.587970in}}{\pgfqpoint{2.653249in}{2.600993in}}%
\pgfpathcurveto{\pgfqpoint{2.653249in}{2.614015in}}{\pgfqpoint{2.648075in}{2.626507in}}{\pgfqpoint{2.638867in}{2.635715in}}%
\pgfpathcurveto{\pgfqpoint{2.629659in}{2.644923in}}{\pgfqpoint{2.617167in}{2.650097in}}{\pgfqpoint{2.604145in}{2.650097in}}%
\pgfpathcurveto{\pgfqpoint{2.591122in}{2.650097in}}{\pgfqpoint{2.578631in}{2.644923in}}{\pgfqpoint{2.569423in}{2.635715in}}%
\pgfpathcurveto{\pgfqpoint{2.560214in}{2.626507in}}{\pgfqpoint{2.555040in}{2.614015in}}{\pgfqpoint{2.555040in}{2.600993in}}%
\pgfpathcurveto{\pgfqpoint{2.555040in}{2.587970in}}{\pgfqpoint{2.560214in}{2.575479in}}{\pgfqpoint{2.569423in}{2.566271in}}%
\pgfpathcurveto{\pgfqpoint{2.578631in}{2.557062in}}{\pgfqpoint{2.591122in}{2.551888in}}{\pgfqpoint{2.604145in}{2.551888in}}%
\pgfpathlineto{\pgfqpoint{2.604145in}{2.551888in}}%
\pgfpathclose%
\pgfusepath{stroke,fill}%
\end{pgfscope}%
\begin{pgfscope}%
\pgfpathrectangle{\pgfqpoint{0.786164in}{0.768110in}}{\pgfqpoint{8.851069in}{7.081890in}}%
\pgfusepath{clip}%
\pgfsetbuttcap%
\pgfsetroundjoin%
\definecolor{currentfill}{rgb}{0.227802,0.326594,0.546532}%
\pgfsetfillcolor{currentfill}%
\pgfsetfillopacity{0.700000}%
\pgfsetlinewidth{0.501875pt}%
\definecolor{currentstroke}{rgb}{1.000000,1.000000,1.000000}%
\pgfsetstrokecolor{currentstroke}%
\pgfsetstrokeopacity{0.700000}%
\pgfsetdash{}{0pt}%
\pgfpathmoveto{\pgfqpoint{2.549345in}{2.420499in}}%
\pgfpathcurveto{\pgfqpoint{2.562368in}{2.420499in}}{\pgfqpoint{2.574859in}{2.425673in}}{\pgfqpoint{2.584067in}{2.434881in}}%
\pgfpathcurveto{\pgfqpoint{2.593276in}{2.444090in}}{\pgfqpoint{2.598450in}{2.456581in}}{\pgfqpoint{2.598450in}{2.469603in}}%
\pgfpathcurveto{\pgfqpoint{2.598450in}{2.482626in}}{\pgfqpoint{2.593276in}{2.495117in}}{\pgfqpoint{2.584067in}{2.504326in}}%
\pgfpathcurveto{\pgfqpoint{2.574859in}{2.513534in}}{\pgfqpoint{2.562368in}{2.518708in}}{\pgfqpoint{2.549345in}{2.518708in}}%
\pgfpathcurveto{\pgfqpoint{2.536322in}{2.518708in}}{\pgfqpoint{2.523831in}{2.513534in}}{\pgfqpoint{2.514623in}{2.504326in}}%
\pgfpathcurveto{\pgfqpoint{2.505414in}{2.495117in}}{\pgfqpoint{2.500240in}{2.482626in}}{\pgfqpoint{2.500240in}{2.469603in}}%
\pgfpathcurveto{\pgfqpoint{2.500240in}{2.456581in}}{\pgfqpoint{2.505414in}{2.444090in}}{\pgfqpoint{2.514623in}{2.434881in}}%
\pgfpathcurveto{\pgfqpoint{2.523831in}{2.425673in}}{\pgfqpoint{2.536322in}{2.420499in}}{\pgfqpoint{2.549345in}{2.420499in}}%
\pgfpathlineto{\pgfqpoint{2.549345in}{2.420499in}}%
\pgfpathclose%
\pgfusepath{stroke,fill}%
\end{pgfscope}%
\begin{pgfscope}%
\pgfpathrectangle{\pgfqpoint{0.786164in}{0.768110in}}{\pgfqpoint{8.851069in}{7.081890in}}%
\pgfusepath{clip}%
\pgfsetbuttcap%
\pgfsetroundjoin%
\definecolor{currentfill}{rgb}{0.225863,0.330805,0.547314}%
\pgfsetfillcolor{currentfill}%
\pgfsetfillopacity{0.700000}%
\pgfsetlinewidth{0.501875pt}%
\definecolor{currentstroke}{rgb}{1.000000,1.000000,1.000000}%
\pgfsetstrokecolor{currentstroke}%
\pgfsetstrokeopacity{0.700000}%
\pgfsetdash{}{0pt}%
\pgfpathmoveto{\pgfqpoint{2.412346in}{2.223415in}}%
\pgfpathcurveto{\pgfqpoint{2.425368in}{2.223415in}}{\pgfqpoint{2.437860in}{2.228589in}}{\pgfqpoint{2.447068in}{2.237797in}}%
\pgfpathcurveto{\pgfqpoint{2.456276in}{2.247005in}}{\pgfqpoint{2.461450in}{2.259497in}}{\pgfqpoint{2.461450in}{2.272519in}}%
\pgfpathcurveto{\pgfqpoint{2.461450in}{2.285542in}}{\pgfqpoint{2.456276in}{2.298033in}}{\pgfqpoint{2.447068in}{2.307241in}}%
\pgfpathcurveto{\pgfqpoint{2.437860in}{2.316450in}}{\pgfqpoint{2.425368in}{2.321624in}}{\pgfqpoint{2.412346in}{2.321624in}}%
\pgfpathcurveto{\pgfqpoint{2.399323in}{2.321624in}}{\pgfqpoint{2.386832in}{2.316450in}}{\pgfqpoint{2.377624in}{2.307241in}}%
\pgfpathcurveto{\pgfqpoint{2.368415in}{2.298033in}}{\pgfqpoint{2.363241in}{2.285542in}}{\pgfqpoint{2.363241in}{2.272519in}}%
\pgfpathcurveto{\pgfqpoint{2.363241in}{2.259497in}}{\pgfqpoint{2.368415in}{2.247005in}}{\pgfqpoint{2.377624in}{2.237797in}}%
\pgfpathcurveto{\pgfqpoint{2.386832in}{2.228589in}}{\pgfqpoint{2.399323in}{2.223415in}}{\pgfqpoint{2.412346in}{2.223415in}}%
\pgfpathlineto{\pgfqpoint{2.412346in}{2.223415in}}%
\pgfpathclose%
\pgfusepath{stroke,fill}%
\end{pgfscope}%
\begin{pgfscope}%
\pgfpathrectangle{\pgfqpoint{0.786164in}{0.768110in}}{\pgfqpoint{8.851069in}{7.081890in}}%
\pgfusepath{clip}%
\pgfsetbuttcap%
\pgfsetroundjoin%
\definecolor{currentfill}{rgb}{0.223925,0.334994,0.548053}%
\pgfsetfillcolor{currentfill}%
\pgfsetfillopacity{0.700000}%
\pgfsetlinewidth{0.501875pt}%
\definecolor{currentstroke}{rgb}{1.000000,1.000000,1.000000}%
\pgfsetstrokecolor{currentstroke}%
\pgfsetstrokeopacity{0.700000}%
\pgfsetdash{}{0pt}%
\pgfpathmoveto{\pgfqpoint{2.631545in}{2.398600in}}%
\pgfpathcurveto{\pgfqpoint{2.644567in}{2.398600in}}{\pgfqpoint{2.657058in}{2.403774in}}{\pgfqpoint{2.666267in}{2.412983in}}%
\pgfpathcurveto{\pgfqpoint{2.675475in}{2.422191in}}{\pgfqpoint{2.680649in}{2.434682in}}{\pgfqpoint{2.680649in}{2.447705in}}%
\pgfpathcurveto{\pgfqpoint{2.680649in}{2.460728in}}{\pgfqpoint{2.675475in}{2.473219in}}{\pgfqpoint{2.666267in}{2.482427in}}%
\pgfpathcurveto{\pgfqpoint{2.657058in}{2.491636in}}{\pgfqpoint{2.644567in}{2.496810in}}{\pgfqpoint{2.631545in}{2.496810in}}%
\pgfpathcurveto{\pgfqpoint{2.618522in}{2.496810in}}{\pgfqpoint{2.606031in}{2.491636in}}{\pgfqpoint{2.596822in}{2.482427in}}%
\pgfpathcurveto{\pgfqpoint{2.587614in}{2.473219in}}{\pgfqpoint{2.582440in}{2.460728in}}{\pgfqpoint{2.582440in}{2.447705in}}%
\pgfpathcurveto{\pgfqpoint{2.582440in}{2.434682in}}{\pgfqpoint{2.587614in}{2.422191in}}{\pgfqpoint{2.596822in}{2.412983in}}%
\pgfpathcurveto{\pgfqpoint{2.606031in}{2.403774in}}{\pgfqpoint{2.618522in}{2.398600in}}{\pgfqpoint{2.631545in}{2.398600in}}%
\pgfpathlineto{\pgfqpoint{2.631545in}{2.398600in}}%
\pgfpathclose%
\pgfusepath{stroke,fill}%
\end{pgfscope}%
\begin{pgfscope}%
\pgfpathrectangle{\pgfqpoint{0.786164in}{0.768110in}}{\pgfqpoint{8.851069in}{7.081890in}}%
\pgfusepath{clip}%
\pgfsetbuttcap%
\pgfsetroundjoin%
\definecolor{currentfill}{rgb}{0.223925,0.334994,0.548053}%
\pgfsetfillcolor{currentfill}%
\pgfsetfillopacity{0.700000}%
\pgfsetlinewidth{0.501875pt}%
\definecolor{currentstroke}{rgb}{1.000000,1.000000,1.000000}%
\pgfsetstrokecolor{currentstroke}%
\pgfsetstrokeopacity{0.700000}%
\pgfsetdash{}{0pt}%
\pgfpathmoveto{\pgfqpoint{2.476279in}{2.223415in}}%
\pgfpathcurveto{\pgfqpoint{2.489301in}{2.223415in}}{\pgfqpoint{2.501793in}{2.228589in}}{\pgfqpoint{2.511001in}{2.237797in}}%
\pgfpathcurveto{\pgfqpoint{2.520209in}{2.247005in}}{\pgfqpoint{2.525383in}{2.259497in}}{\pgfqpoint{2.525383in}{2.272519in}}%
\pgfpathcurveto{\pgfqpoint{2.525383in}{2.285542in}}{\pgfqpoint{2.520209in}{2.298033in}}{\pgfqpoint{2.511001in}{2.307241in}}%
\pgfpathcurveto{\pgfqpoint{2.501793in}{2.316450in}}{\pgfqpoint{2.489301in}{2.321624in}}{\pgfqpoint{2.476279in}{2.321624in}}%
\pgfpathcurveto{\pgfqpoint{2.463256in}{2.321624in}}{\pgfqpoint{2.450765in}{2.316450in}}{\pgfqpoint{2.441557in}{2.307241in}}%
\pgfpathcurveto{\pgfqpoint{2.432348in}{2.298033in}}{\pgfqpoint{2.427174in}{2.285542in}}{\pgfqpoint{2.427174in}{2.272519in}}%
\pgfpathcurveto{\pgfqpoint{2.427174in}{2.259497in}}{\pgfqpoint{2.432348in}{2.247005in}}{\pgfqpoint{2.441557in}{2.237797in}}%
\pgfpathcurveto{\pgfqpoint{2.450765in}{2.228589in}}{\pgfqpoint{2.463256in}{2.223415in}}{\pgfqpoint{2.476279in}{2.223415in}}%
\pgfpathlineto{\pgfqpoint{2.476279in}{2.223415in}}%
\pgfpathclose%
\pgfusepath{stroke,fill}%
\end{pgfscope}%
\begin{pgfscope}%
\pgfpathrectangle{\pgfqpoint{0.786164in}{0.768110in}}{\pgfqpoint{8.851069in}{7.081890in}}%
\pgfusepath{clip}%
\pgfsetrectcap%
\pgfsetroundjoin%
\pgfsetlinewidth{0.803000pt}%
\definecolor{currentstroke}{rgb}{0.690196,0.690196,0.690196}%
\pgfsetstrokecolor{currentstroke}%
\pgfsetdash{}{0pt}%
\pgfpathmoveto{\pgfqpoint{1.699949in}{0.768110in}}%
\pgfpathlineto{\pgfqpoint{1.699949in}{7.850000in}}%
\pgfusepath{stroke}%
\end{pgfscope}%
\begin{pgfscope}%
\pgfsetbuttcap%
\pgfsetroundjoin%
\definecolor{currentfill}{rgb}{0.000000,0.000000,0.000000}%
\pgfsetfillcolor{currentfill}%
\pgfsetlinewidth{0.803000pt}%
\definecolor{currentstroke}{rgb}{0.000000,0.000000,0.000000}%
\pgfsetstrokecolor{currentstroke}%
\pgfsetdash{}{0pt}%
\pgfsys@defobject{currentmarker}{\pgfqpoint{0.000000in}{-0.048611in}}{\pgfqpoint{0.000000in}{0.000000in}}{%
\pgfpathmoveto{\pgfqpoint{0.000000in}{0.000000in}}%
\pgfpathlineto{\pgfqpoint{0.000000in}{-0.048611in}}%
\pgfusepath{stroke,fill}%
}%
\begin{pgfscope}%
\pgfsys@transformshift{1.699949in}{0.768110in}%
\pgfsys@useobject{currentmarker}{}%
\end{pgfscope}%
\end{pgfscope}%
\begin{pgfscope}%
\definecolor{textcolor}{rgb}{0.000000,0.000000,0.000000}%
\pgfsetstrokecolor{textcolor}%
\pgfsetfillcolor{textcolor}%
\pgftext[x=1.699949in,y=0.670888in,,top]{\color{textcolor}{\sffamily\fontsize{15.000000}{18.000000}\selectfont\catcode`\^=\active\def^{\ifmmode\sp\else\^{}\fi}\catcode`\%=\active\def%{\%}2}}%
\end{pgfscope}%
\begin{pgfscope}%
\pgfpathrectangle{\pgfqpoint{0.786164in}{0.768110in}}{\pgfqpoint{8.851069in}{7.081890in}}%
\pgfusepath{clip}%
\pgfsetrectcap%
\pgfsetroundjoin%
\pgfsetlinewidth{0.803000pt}%
\definecolor{currentstroke}{rgb}{0.690196,0.690196,0.690196}%
\pgfsetstrokecolor{currentstroke}%
\pgfsetdash{}{0pt}%
\pgfpathmoveto{\pgfqpoint{3.526607in}{0.768110in}}%
\pgfpathlineto{\pgfqpoint{3.526607in}{7.850000in}}%
\pgfusepath{stroke}%
\end{pgfscope}%
\begin{pgfscope}%
\pgfsetbuttcap%
\pgfsetroundjoin%
\definecolor{currentfill}{rgb}{0.000000,0.000000,0.000000}%
\pgfsetfillcolor{currentfill}%
\pgfsetlinewidth{0.803000pt}%
\definecolor{currentstroke}{rgb}{0.000000,0.000000,0.000000}%
\pgfsetstrokecolor{currentstroke}%
\pgfsetdash{}{0pt}%
\pgfsys@defobject{currentmarker}{\pgfqpoint{0.000000in}{-0.048611in}}{\pgfqpoint{0.000000in}{0.000000in}}{%
\pgfpathmoveto{\pgfqpoint{0.000000in}{0.000000in}}%
\pgfpathlineto{\pgfqpoint{0.000000in}{-0.048611in}}%
\pgfusepath{stroke,fill}%
}%
\begin{pgfscope}%
\pgfsys@transformshift{3.526607in}{0.768110in}%
\pgfsys@useobject{currentmarker}{}%
\end{pgfscope}%
\end{pgfscope}%
\begin{pgfscope}%
\definecolor{textcolor}{rgb}{0.000000,0.000000,0.000000}%
\pgfsetstrokecolor{textcolor}%
\pgfsetfillcolor{textcolor}%
\pgftext[x=3.526607in,y=0.670888in,,top]{\color{textcolor}{\sffamily\fontsize{15.000000}{18.000000}\selectfont\catcode`\^=\active\def^{\ifmmode\sp\else\^{}\fi}\catcode`\%=\active\def%{\%}4}}%
\end{pgfscope}%
\begin{pgfscope}%
\pgfpathrectangle{\pgfqpoint{0.786164in}{0.768110in}}{\pgfqpoint{8.851069in}{7.081890in}}%
\pgfusepath{clip}%
\pgfsetrectcap%
\pgfsetroundjoin%
\pgfsetlinewidth{0.803000pt}%
\definecolor{currentstroke}{rgb}{0.690196,0.690196,0.690196}%
\pgfsetstrokecolor{currentstroke}%
\pgfsetdash{}{0pt}%
\pgfpathmoveto{\pgfqpoint{5.353264in}{0.768110in}}%
\pgfpathlineto{\pgfqpoint{5.353264in}{7.850000in}}%
\pgfusepath{stroke}%
\end{pgfscope}%
\begin{pgfscope}%
\pgfsetbuttcap%
\pgfsetroundjoin%
\definecolor{currentfill}{rgb}{0.000000,0.000000,0.000000}%
\pgfsetfillcolor{currentfill}%
\pgfsetlinewidth{0.803000pt}%
\definecolor{currentstroke}{rgb}{0.000000,0.000000,0.000000}%
\pgfsetstrokecolor{currentstroke}%
\pgfsetdash{}{0pt}%
\pgfsys@defobject{currentmarker}{\pgfqpoint{0.000000in}{-0.048611in}}{\pgfqpoint{0.000000in}{0.000000in}}{%
\pgfpathmoveto{\pgfqpoint{0.000000in}{0.000000in}}%
\pgfpathlineto{\pgfqpoint{0.000000in}{-0.048611in}}%
\pgfusepath{stroke,fill}%
}%
\begin{pgfscope}%
\pgfsys@transformshift{5.353264in}{0.768110in}%
\pgfsys@useobject{currentmarker}{}%
\end{pgfscope}%
\end{pgfscope}%
\begin{pgfscope}%
\definecolor{textcolor}{rgb}{0.000000,0.000000,0.000000}%
\pgfsetstrokecolor{textcolor}%
\pgfsetfillcolor{textcolor}%
\pgftext[x=5.353264in,y=0.670888in,,top]{\color{textcolor}{\sffamily\fontsize{15.000000}{18.000000}\selectfont\catcode`\^=\active\def^{\ifmmode\sp\else\^{}\fi}\catcode`\%=\active\def%{\%}6}}%
\end{pgfscope}%
\begin{pgfscope}%
\pgfpathrectangle{\pgfqpoint{0.786164in}{0.768110in}}{\pgfqpoint{8.851069in}{7.081890in}}%
\pgfusepath{clip}%
\pgfsetrectcap%
\pgfsetroundjoin%
\pgfsetlinewidth{0.803000pt}%
\definecolor{currentstroke}{rgb}{0.690196,0.690196,0.690196}%
\pgfsetstrokecolor{currentstroke}%
\pgfsetdash{}{0pt}%
\pgfpathmoveto{\pgfqpoint{7.179922in}{0.768110in}}%
\pgfpathlineto{\pgfqpoint{7.179922in}{7.850000in}}%
\pgfusepath{stroke}%
\end{pgfscope}%
\begin{pgfscope}%
\pgfsetbuttcap%
\pgfsetroundjoin%
\definecolor{currentfill}{rgb}{0.000000,0.000000,0.000000}%
\pgfsetfillcolor{currentfill}%
\pgfsetlinewidth{0.803000pt}%
\definecolor{currentstroke}{rgb}{0.000000,0.000000,0.000000}%
\pgfsetstrokecolor{currentstroke}%
\pgfsetdash{}{0pt}%
\pgfsys@defobject{currentmarker}{\pgfqpoint{0.000000in}{-0.048611in}}{\pgfqpoint{0.000000in}{0.000000in}}{%
\pgfpathmoveto{\pgfqpoint{0.000000in}{0.000000in}}%
\pgfpathlineto{\pgfqpoint{0.000000in}{-0.048611in}}%
\pgfusepath{stroke,fill}%
}%
\begin{pgfscope}%
\pgfsys@transformshift{7.179922in}{0.768110in}%
\pgfsys@useobject{currentmarker}{}%
\end{pgfscope}%
\end{pgfscope}%
\begin{pgfscope}%
\definecolor{textcolor}{rgb}{0.000000,0.000000,0.000000}%
\pgfsetstrokecolor{textcolor}%
\pgfsetfillcolor{textcolor}%
\pgftext[x=7.179922in,y=0.670888in,,top]{\color{textcolor}{\sffamily\fontsize{15.000000}{18.000000}\selectfont\catcode`\^=\active\def^{\ifmmode\sp\else\^{}\fi}\catcode`\%=\active\def%{\%}8}}%
\end{pgfscope}%
\begin{pgfscope}%
\pgfpathrectangle{\pgfqpoint{0.786164in}{0.768110in}}{\pgfqpoint{8.851069in}{7.081890in}}%
\pgfusepath{clip}%
\pgfsetrectcap%
\pgfsetroundjoin%
\pgfsetlinewidth{0.803000pt}%
\definecolor{currentstroke}{rgb}{0.690196,0.690196,0.690196}%
\pgfsetstrokecolor{currentstroke}%
\pgfsetdash{}{0pt}%
\pgfpathmoveto{\pgfqpoint{9.006579in}{0.768110in}}%
\pgfpathlineto{\pgfqpoint{9.006579in}{7.850000in}}%
\pgfusepath{stroke}%
\end{pgfscope}%
\begin{pgfscope}%
\pgfsetbuttcap%
\pgfsetroundjoin%
\definecolor{currentfill}{rgb}{0.000000,0.000000,0.000000}%
\pgfsetfillcolor{currentfill}%
\pgfsetlinewidth{0.803000pt}%
\definecolor{currentstroke}{rgb}{0.000000,0.000000,0.000000}%
\pgfsetstrokecolor{currentstroke}%
\pgfsetdash{}{0pt}%
\pgfsys@defobject{currentmarker}{\pgfqpoint{0.000000in}{-0.048611in}}{\pgfqpoint{0.000000in}{0.000000in}}{%
\pgfpathmoveto{\pgfqpoint{0.000000in}{0.000000in}}%
\pgfpathlineto{\pgfqpoint{0.000000in}{-0.048611in}}%
\pgfusepath{stroke,fill}%
}%
\begin{pgfscope}%
\pgfsys@transformshift{9.006579in}{0.768110in}%
\pgfsys@useobject{currentmarker}{}%
\end{pgfscope}%
\end{pgfscope}%
\begin{pgfscope}%
\definecolor{textcolor}{rgb}{0.000000,0.000000,0.000000}%
\pgfsetstrokecolor{textcolor}%
\pgfsetfillcolor{textcolor}%
\pgftext[x=9.006579in,y=0.670888in,,top]{\color{textcolor}{\sffamily\fontsize{15.000000}{18.000000}\selectfont\catcode`\^=\active\def^{\ifmmode\sp\else\^{}\fi}\catcode`\%=\active\def%{\%}10}}%
\end{pgfscope}%
\begin{pgfscope}%
\definecolor{textcolor}{rgb}{0.000000,0.000000,0.000000}%
\pgfsetstrokecolor{textcolor}%
\pgfsetfillcolor{textcolor}%
\pgftext[x=5.211698in,y=0.437555in,,top]{\color{textcolor}{\sffamily\fontsize{20.000000}{24.000000}\selectfont\catcode`\^=\active\def^{\ifmmode\sp\else\^{}\fi}\catcode`\%=\active\def%{\%}Energy Consumption (tonnes of oil equivalent per capita)}}%
\end{pgfscope}%
\begin{pgfscope}%
\pgfpathrectangle{\pgfqpoint{0.786164in}{0.768110in}}{\pgfqpoint{8.851069in}{7.081890in}}%
\pgfusepath{clip}%
\pgfsetrectcap%
\pgfsetroundjoin%
\pgfsetlinewidth{0.803000pt}%
\definecolor{currentstroke}{rgb}{0.690196,0.690196,0.690196}%
\pgfsetstrokecolor{currentstroke}%
\pgfsetdash{}{0pt}%
\pgfpathmoveto{\pgfqpoint{0.786164in}{1.068116in}}%
\pgfpathlineto{\pgfqpoint{9.637233in}{1.068116in}}%
\pgfusepath{stroke}%
\end{pgfscope}%
\begin{pgfscope}%
\pgfsetbuttcap%
\pgfsetroundjoin%
\definecolor{currentfill}{rgb}{0.000000,0.000000,0.000000}%
\pgfsetfillcolor{currentfill}%
\pgfsetlinewidth{0.803000pt}%
\definecolor{currentstroke}{rgb}{0.000000,0.000000,0.000000}%
\pgfsetstrokecolor{currentstroke}%
\pgfsetdash{}{0pt}%
\pgfsys@defobject{currentmarker}{\pgfqpoint{-0.048611in}{0.000000in}}{\pgfqpoint{-0.000000in}{0.000000in}}{%
\pgfpathmoveto{\pgfqpoint{-0.000000in}{0.000000in}}%
\pgfpathlineto{\pgfqpoint{-0.048611in}{0.000000in}}%
\pgfusepath{stroke,fill}%
}%
\begin{pgfscope}%
\pgfsys@transformshift{0.786164in}{1.068116in}%
\pgfsys@useobject{currentmarker}{}%
\end{pgfscope}%
\end{pgfscope}%
\begin{pgfscope}%
\definecolor{textcolor}{rgb}{0.000000,0.000000,0.000000}%
\pgfsetstrokecolor{textcolor}%
\pgfsetfillcolor{textcolor}%
\pgftext[x=0.591026in, y=0.998672in, left, base]{\color{textcolor}{\sffamily\fontsize{15.000000}{18.000000}\selectfont\catcode`\^=\active\def^{\ifmmode\sp\else\^{}\fi}\catcode`\%=\active\def%{\%}0}}%
\end{pgfscope}%
\begin{pgfscope}%
\pgfpathrectangle{\pgfqpoint{0.786164in}{0.768110in}}{\pgfqpoint{8.851069in}{7.081890in}}%
\pgfusepath{clip}%
\pgfsetrectcap%
\pgfsetroundjoin%
\pgfsetlinewidth{0.803000pt}%
\definecolor{currentstroke}{rgb}{0.690196,0.690196,0.690196}%
\pgfsetstrokecolor{currentstroke}%
\pgfsetdash{}{0pt}%
\pgfpathmoveto{\pgfqpoint{0.786164in}{2.163028in}}%
\pgfpathlineto{\pgfqpoint{9.637233in}{2.163028in}}%
\pgfusepath{stroke}%
\end{pgfscope}%
\begin{pgfscope}%
\pgfsetbuttcap%
\pgfsetroundjoin%
\definecolor{currentfill}{rgb}{0.000000,0.000000,0.000000}%
\pgfsetfillcolor{currentfill}%
\pgfsetlinewidth{0.803000pt}%
\definecolor{currentstroke}{rgb}{0.000000,0.000000,0.000000}%
\pgfsetstrokecolor{currentstroke}%
\pgfsetdash{}{0pt}%
\pgfsys@defobject{currentmarker}{\pgfqpoint{-0.048611in}{0.000000in}}{\pgfqpoint{-0.000000in}{0.000000in}}{%
\pgfpathmoveto{\pgfqpoint{-0.000000in}{0.000000in}}%
\pgfpathlineto{\pgfqpoint{-0.048611in}{0.000000in}}%
\pgfusepath{stroke,fill}%
}%
\begin{pgfscope}%
\pgfsys@transformshift{0.786164in}{2.163028in}%
\pgfsys@useobject{currentmarker}{}%
\end{pgfscope}%
\end{pgfscope}%
\begin{pgfscope}%
\definecolor{textcolor}{rgb}{0.000000,0.000000,0.000000}%
\pgfsetstrokecolor{textcolor}%
\pgfsetfillcolor{textcolor}%
\pgftext[x=0.591026in, y=2.093584in, left, base]{\color{textcolor}{\sffamily\fontsize{15.000000}{18.000000}\selectfont\catcode`\^=\active\def^{\ifmmode\sp\else\^{}\fi}\catcode`\%=\active\def%{\%}5}}%
\end{pgfscope}%
\begin{pgfscope}%
\pgfpathrectangle{\pgfqpoint{0.786164in}{0.768110in}}{\pgfqpoint{8.851069in}{7.081890in}}%
\pgfusepath{clip}%
\pgfsetrectcap%
\pgfsetroundjoin%
\pgfsetlinewidth{0.803000pt}%
\definecolor{currentstroke}{rgb}{0.690196,0.690196,0.690196}%
\pgfsetstrokecolor{currentstroke}%
\pgfsetdash{}{0pt}%
\pgfpathmoveto{\pgfqpoint{0.786164in}{3.257940in}}%
\pgfpathlineto{\pgfqpoint{9.637233in}{3.257940in}}%
\pgfusepath{stroke}%
\end{pgfscope}%
\begin{pgfscope}%
\pgfsetbuttcap%
\pgfsetroundjoin%
\definecolor{currentfill}{rgb}{0.000000,0.000000,0.000000}%
\pgfsetfillcolor{currentfill}%
\pgfsetlinewidth{0.803000pt}%
\definecolor{currentstroke}{rgb}{0.000000,0.000000,0.000000}%
\pgfsetstrokecolor{currentstroke}%
\pgfsetdash{}{0pt}%
\pgfsys@defobject{currentmarker}{\pgfqpoint{-0.048611in}{0.000000in}}{\pgfqpoint{-0.000000in}{0.000000in}}{%
\pgfpathmoveto{\pgfqpoint{-0.000000in}{0.000000in}}%
\pgfpathlineto{\pgfqpoint{-0.048611in}{0.000000in}}%
\pgfusepath{stroke,fill}%
}%
\begin{pgfscope}%
\pgfsys@transformshift{0.786164in}{3.257940in}%
\pgfsys@useobject{currentmarker}{}%
\end{pgfscope}%
\end{pgfscope}%
\begin{pgfscope}%
\definecolor{textcolor}{rgb}{0.000000,0.000000,0.000000}%
\pgfsetstrokecolor{textcolor}%
\pgfsetfillcolor{textcolor}%
\pgftext[x=0.493111in, y=3.188496in, left, base]{\color{textcolor}{\sffamily\fontsize{15.000000}{18.000000}\selectfont\catcode`\^=\active\def^{\ifmmode\sp\else\^{}\fi}\catcode`\%=\active\def%{\%}10}}%
\end{pgfscope}%
\begin{pgfscope}%
\pgfpathrectangle{\pgfqpoint{0.786164in}{0.768110in}}{\pgfqpoint{8.851069in}{7.081890in}}%
\pgfusepath{clip}%
\pgfsetrectcap%
\pgfsetroundjoin%
\pgfsetlinewidth{0.803000pt}%
\definecolor{currentstroke}{rgb}{0.690196,0.690196,0.690196}%
\pgfsetstrokecolor{currentstroke}%
\pgfsetdash{}{0pt}%
\pgfpathmoveto{\pgfqpoint{0.786164in}{4.352852in}}%
\pgfpathlineto{\pgfqpoint{9.637233in}{4.352852in}}%
\pgfusepath{stroke}%
\end{pgfscope}%
\begin{pgfscope}%
\pgfsetbuttcap%
\pgfsetroundjoin%
\definecolor{currentfill}{rgb}{0.000000,0.000000,0.000000}%
\pgfsetfillcolor{currentfill}%
\pgfsetlinewidth{0.803000pt}%
\definecolor{currentstroke}{rgb}{0.000000,0.000000,0.000000}%
\pgfsetstrokecolor{currentstroke}%
\pgfsetdash{}{0pt}%
\pgfsys@defobject{currentmarker}{\pgfqpoint{-0.048611in}{0.000000in}}{\pgfqpoint{-0.000000in}{0.000000in}}{%
\pgfpathmoveto{\pgfqpoint{-0.000000in}{0.000000in}}%
\pgfpathlineto{\pgfqpoint{-0.048611in}{0.000000in}}%
\pgfusepath{stroke,fill}%
}%
\begin{pgfscope}%
\pgfsys@transformshift{0.786164in}{4.352852in}%
\pgfsys@useobject{currentmarker}{}%
\end{pgfscope}%
\end{pgfscope}%
\begin{pgfscope}%
\definecolor{textcolor}{rgb}{0.000000,0.000000,0.000000}%
\pgfsetstrokecolor{textcolor}%
\pgfsetfillcolor{textcolor}%
\pgftext[x=0.493111in, y=4.283407in, left, base]{\color{textcolor}{\sffamily\fontsize{15.000000}{18.000000}\selectfont\catcode`\^=\active\def^{\ifmmode\sp\else\^{}\fi}\catcode`\%=\active\def%{\%}15}}%
\end{pgfscope}%
\begin{pgfscope}%
\pgfpathrectangle{\pgfqpoint{0.786164in}{0.768110in}}{\pgfqpoint{8.851069in}{7.081890in}}%
\pgfusepath{clip}%
\pgfsetrectcap%
\pgfsetroundjoin%
\pgfsetlinewidth{0.803000pt}%
\definecolor{currentstroke}{rgb}{0.690196,0.690196,0.690196}%
\pgfsetstrokecolor{currentstroke}%
\pgfsetdash{}{0pt}%
\pgfpathmoveto{\pgfqpoint{0.786164in}{5.447763in}}%
\pgfpathlineto{\pgfqpoint{9.637233in}{5.447763in}}%
\pgfusepath{stroke}%
\end{pgfscope}%
\begin{pgfscope}%
\pgfsetbuttcap%
\pgfsetroundjoin%
\definecolor{currentfill}{rgb}{0.000000,0.000000,0.000000}%
\pgfsetfillcolor{currentfill}%
\pgfsetlinewidth{0.803000pt}%
\definecolor{currentstroke}{rgb}{0.000000,0.000000,0.000000}%
\pgfsetstrokecolor{currentstroke}%
\pgfsetdash{}{0pt}%
\pgfsys@defobject{currentmarker}{\pgfqpoint{-0.048611in}{0.000000in}}{\pgfqpoint{-0.000000in}{0.000000in}}{%
\pgfpathmoveto{\pgfqpoint{-0.000000in}{0.000000in}}%
\pgfpathlineto{\pgfqpoint{-0.048611in}{0.000000in}}%
\pgfusepath{stroke,fill}%
}%
\begin{pgfscope}%
\pgfsys@transformshift{0.786164in}{5.447763in}%
\pgfsys@useobject{currentmarker}{}%
\end{pgfscope}%
\end{pgfscope}%
\begin{pgfscope}%
\definecolor{textcolor}{rgb}{0.000000,0.000000,0.000000}%
\pgfsetstrokecolor{textcolor}%
\pgfsetfillcolor{textcolor}%
\pgftext[x=0.493111in, y=5.378319in, left, base]{\color{textcolor}{\sffamily\fontsize{15.000000}{18.000000}\selectfont\catcode`\^=\active\def^{\ifmmode\sp\else\^{}\fi}\catcode`\%=\active\def%{\%}20}}%
\end{pgfscope}%
\begin{pgfscope}%
\pgfpathrectangle{\pgfqpoint{0.786164in}{0.768110in}}{\pgfqpoint{8.851069in}{7.081890in}}%
\pgfusepath{clip}%
\pgfsetrectcap%
\pgfsetroundjoin%
\pgfsetlinewidth{0.803000pt}%
\definecolor{currentstroke}{rgb}{0.690196,0.690196,0.690196}%
\pgfsetstrokecolor{currentstroke}%
\pgfsetdash{}{0pt}%
\pgfpathmoveto{\pgfqpoint{0.786164in}{6.542675in}}%
\pgfpathlineto{\pgfqpoint{9.637233in}{6.542675in}}%
\pgfusepath{stroke}%
\end{pgfscope}%
\begin{pgfscope}%
\pgfsetbuttcap%
\pgfsetroundjoin%
\definecolor{currentfill}{rgb}{0.000000,0.000000,0.000000}%
\pgfsetfillcolor{currentfill}%
\pgfsetlinewidth{0.803000pt}%
\definecolor{currentstroke}{rgb}{0.000000,0.000000,0.000000}%
\pgfsetstrokecolor{currentstroke}%
\pgfsetdash{}{0pt}%
\pgfsys@defobject{currentmarker}{\pgfqpoint{-0.048611in}{0.000000in}}{\pgfqpoint{-0.000000in}{0.000000in}}{%
\pgfpathmoveto{\pgfqpoint{-0.000000in}{0.000000in}}%
\pgfpathlineto{\pgfqpoint{-0.048611in}{0.000000in}}%
\pgfusepath{stroke,fill}%
}%
\begin{pgfscope}%
\pgfsys@transformshift{0.786164in}{6.542675in}%
\pgfsys@useobject{currentmarker}{}%
\end{pgfscope}%
\end{pgfscope}%
\begin{pgfscope}%
\definecolor{textcolor}{rgb}{0.000000,0.000000,0.000000}%
\pgfsetstrokecolor{textcolor}%
\pgfsetfillcolor{textcolor}%
\pgftext[x=0.493111in, y=6.473231in, left, base]{\color{textcolor}{\sffamily\fontsize{15.000000}{18.000000}\selectfont\catcode`\^=\active\def^{\ifmmode\sp\else\^{}\fi}\catcode`\%=\active\def%{\%}25}}%
\end{pgfscope}%
\begin{pgfscope}%
\pgfpathrectangle{\pgfqpoint{0.786164in}{0.768110in}}{\pgfqpoint{8.851069in}{7.081890in}}%
\pgfusepath{clip}%
\pgfsetrectcap%
\pgfsetroundjoin%
\pgfsetlinewidth{0.803000pt}%
\definecolor{currentstroke}{rgb}{0.690196,0.690196,0.690196}%
\pgfsetstrokecolor{currentstroke}%
\pgfsetdash{}{0pt}%
\pgfpathmoveto{\pgfqpoint{0.786164in}{7.637587in}}%
\pgfpathlineto{\pgfqpoint{9.637233in}{7.637587in}}%
\pgfusepath{stroke}%
\end{pgfscope}%
\begin{pgfscope}%
\pgfsetbuttcap%
\pgfsetroundjoin%
\definecolor{currentfill}{rgb}{0.000000,0.000000,0.000000}%
\pgfsetfillcolor{currentfill}%
\pgfsetlinewidth{0.803000pt}%
\definecolor{currentstroke}{rgb}{0.000000,0.000000,0.000000}%
\pgfsetstrokecolor{currentstroke}%
\pgfsetdash{}{0pt}%
\pgfsys@defobject{currentmarker}{\pgfqpoint{-0.048611in}{0.000000in}}{\pgfqpoint{-0.000000in}{0.000000in}}{%
\pgfpathmoveto{\pgfqpoint{-0.000000in}{0.000000in}}%
\pgfpathlineto{\pgfqpoint{-0.048611in}{0.000000in}}%
\pgfusepath{stroke,fill}%
}%
\begin{pgfscope}%
\pgfsys@transformshift{0.786164in}{7.637587in}%
\pgfsys@useobject{currentmarker}{}%
\end{pgfscope}%
\end{pgfscope}%
\begin{pgfscope}%
\definecolor{textcolor}{rgb}{0.000000,0.000000,0.000000}%
\pgfsetstrokecolor{textcolor}%
\pgfsetfillcolor{textcolor}%
\pgftext[x=0.493111in, y=7.568143in, left, base]{\color{textcolor}{\sffamily\fontsize{15.000000}{18.000000}\selectfont\catcode`\^=\active\def^{\ifmmode\sp\else\^{}\fi}\catcode`\%=\active\def%{\%}30}}%
\end{pgfscope}%
\begin{pgfscope}%
\definecolor{textcolor}{rgb}{0.000000,0.000000,0.000000}%
\pgfsetstrokecolor{textcolor}%
\pgfsetfillcolor{textcolor}%
\pgftext[x=0.437555in,y=4.309055in,,bottom,rotate=90.000000]{\color{textcolor}{\sffamily\fontsize{20.000000}{24.000000}\selectfont\catcode`\^=\active\def^{\ifmmode\sp\else\^{}\fi}\catcode`\%=\active\def%{\%}Greenhouse Gas Emissions (tonnes per capita)}}%
\end{pgfscope}%
\begin{pgfscope}%
\pgfpathrectangle{\pgfqpoint{0.786164in}{0.768110in}}{\pgfqpoint{8.851069in}{7.081890in}}%
\pgfusepath{clip}%
\pgfsetrectcap%
\pgfsetroundjoin%
\pgfsetlinewidth{1.505625pt}%
\definecolor{currentstroke}{rgb}{1.000000,0.000000,0.000000}%
\pgfsetstrokecolor{currentstroke}%
\pgfsetdash{}{0pt}%
\pgfpathmoveto{\pgfqpoint{3.362208in}{3.250135in}}%
\pgfpathlineto{\pgfqpoint{9.234911in}{6.078857in}}%
\pgfpathlineto{\pgfqpoint{1.188485in}{2.203111in}}%
\pgfpathlineto{\pgfqpoint{2.476279in}{2.823407in}}%
\pgfusepath{stroke}%
\end{pgfscope}%
\begin{pgfscope}%
\pgfsetrectcap%
\pgfsetmiterjoin%
\pgfsetlinewidth{0.803000pt}%
\definecolor{currentstroke}{rgb}{0.000000,0.000000,0.000000}%
\pgfsetstrokecolor{currentstroke}%
\pgfsetdash{}{0pt}%
\pgfpathmoveto{\pgfqpoint{0.786164in}{0.768110in}}%
\pgfpathlineto{\pgfqpoint{0.786164in}{7.850000in}}%
\pgfusepath{stroke}%
\end{pgfscope}%
\begin{pgfscope}%
\pgfsetrectcap%
\pgfsetmiterjoin%
\pgfsetlinewidth{0.803000pt}%
\definecolor{currentstroke}{rgb}{0.000000,0.000000,0.000000}%
\pgfsetstrokecolor{currentstroke}%
\pgfsetdash{}{0pt}%
\pgfpathmoveto{\pgfqpoint{9.637233in}{0.768110in}}%
\pgfpathlineto{\pgfqpoint{9.637233in}{7.850000in}}%
\pgfusepath{stroke}%
\end{pgfscope}%
\begin{pgfscope}%
\pgfsetrectcap%
\pgfsetmiterjoin%
\pgfsetlinewidth{0.803000pt}%
\definecolor{currentstroke}{rgb}{0.000000,0.000000,0.000000}%
\pgfsetstrokecolor{currentstroke}%
\pgfsetdash{}{0pt}%
\pgfpathmoveto{\pgfqpoint{0.786164in}{0.768110in}}%
\pgfpathlineto{\pgfqpoint{9.637233in}{0.768110in}}%
\pgfusepath{stroke}%
\end{pgfscope}%
\begin{pgfscope}%
\pgfsetrectcap%
\pgfsetmiterjoin%
\pgfsetlinewidth{0.803000pt}%
\definecolor{currentstroke}{rgb}{0.000000,0.000000,0.000000}%
\pgfsetstrokecolor{currentstroke}%
\pgfsetdash{}{0pt}%
\pgfpathmoveto{\pgfqpoint{0.786164in}{7.850000in}}%
\pgfpathlineto{\pgfqpoint{9.637233in}{7.850000in}}%
\pgfusepath{stroke}%
\end{pgfscope}%
\begin{pgfscope}%
\pgfsetbuttcap%
\pgfsetmiterjoin%
\definecolor{currentfill}{rgb}{1.000000,1.000000,1.000000}%
\pgfsetfillcolor{currentfill}%
\pgfsetfillopacity{0.800000}%
\pgfsetlinewidth{1.003750pt}%
\definecolor{currentstroke}{rgb}{0.800000,0.800000,0.800000}%
\pgfsetstrokecolor{currentstroke}%
\pgfsetstrokeopacity{0.800000}%
\pgfsetdash{}{0pt}%
\pgfpathmoveto{\pgfqpoint{0.980608in}{7.232821in}}%
\pgfpathlineto{\pgfqpoint{2.834606in}{7.232821in}}%
\pgfpathquadraticcurveto{\pgfqpoint{2.890162in}{7.232821in}}{\pgfqpoint{2.890162in}{7.288377in}}%
\pgfpathlineto{\pgfqpoint{2.890162in}{7.655556in}}%
\pgfpathquadraticcurveto{\pgfqpoint{2.890162in}{7.711111in}}{\pgfqpoint{2.834606in}{7.711111in}}%
\pgfpathlineto{\pgfqpoint{0.980608in}{7.711111in}}%
\pgfpathquadraticcurveto{\pgfqpoint{0.925053in}{7.711111in}}{\pgfqpoint{0.925053in}{7.655556in}}%
\pgfpathlineto{\pgfqpoint{0.925053in}{7.288377in}}%
\pgfpathquadraticcurveto{\pgfqpoint{0.925053in}{7.232821in}}{\pgfqpoint{0.980608in}{7.232821in}}%
\pgfpathlineto{\pgfqpoint{0.980608in}{7.232821in}}%
\pgfpathclose%
\pgfusepath{stroke,fill}%
\end{pgfscope}%
\begin{pgfscope}%
\pgfsetrectcap%
\pgfsetroundjoin%
\pgfsetlinewidth{1.505625pt}%
\definecolor{currentstroke}{rgb}{1.000000,0.000000,0.000000}%
\pgfsetstrokecolor{currentstroke}%
\pgfsetdash{}{0pt}%
\pgfpathmoveto{\pgfqpoint{1.036164in}{7.497184in}}%
\pgfpathlineto{\pgfqpoint{1.313942in}{7.497184in}}%
\pgfpathlineto{\pgfqpoint{1.591719in}{7.497184in}}%
\pgfusepath{stroke}%
\end{pgfscope}%
\begin{pgfscope}%
\definecolor{textcolor}{rgb}{0.000000,0.000000,0.000000}%
\pgfsetstrokecolor{textcolor}%
\pgfsetfillcolor{textcolor}%
\pgftext[x=1.813942in,y=7.399962in,left,base]{\color{textcolor}{\sffamily\fontsize{20.000000}{24.000000}\selectfont\catcode`\^=\active\def^{\ifmmode\sp\else\^{}\fi}\catcode`\%=\active\def%{\%}r = 0.64}}%
\end{pgfscope}%
\begin{pgfscope}%
\pgfsetbuttcap%
\pgfsetmiterjoin%
\definecolor{currentfill}{rgb}{1.000000,1.000000,1.000000}%
\pgfsetfillcolor{currentfill}%
\pgfsetlinewidth{0.000000pt}%
\definecolor{currentstroke}{rgb}{0.000000,0.000000,0.000000}%
\pgfsetstrokecolor{currentstroke}%
\pgfsetstrokeopacity{0.000000}%
\pgfsetdash{}{0pt}%
\pgfpathmoveto{\pgfqpoint{10.190425in}{0.768110in}}%
\pgfpathlineto{\pgfqpoint{10.544519in}{0.768110in}}%
\pgfpathlineto{\pgfqpoint{10.544519in}{7.850000in}}%
\pgfpathlineto{\pgfqpoint{10.190425in}{7.850000in}}%
\pgfpathlineto{\pgfqpoint{10.190425in}{0.768110in}}%
\pgfpathclose%
\pgfusepath{fill}%
\end{pgfscope}%
\begin{pgfscope}%
\pgfsys@transformshift{10.190000in}{0.770000in}%
\pgftext[left,bottom]{\includegraphics[interpolate=true,width=0.350000in,height=7.080000in]{plot_global_data_with_renewables-img0.png}}%
\end{pgfscope}%
\begin{pgfscope}%
\pgfsetbuttcap%
\pgfsetroundjoin%
\definecolor{currentfill}{rgb}{0.000000,0.000000,0.000000}%
\pgfsetfillcolor{currentfill}%
\pgfsetlinewidth{0.803000pt}%
\definecolor{currentstroke}{rgb}{0.000000,0.000000,0.000000}%
\pgfsetstrokecolor{currentstroke}%
\pgfsetdash{}{0pt}%
\pgfsys@defobject{currentmarker}{\pgfqpoint{0.000000in}{0.000000in}}{\pgfqpoint{0.048611in}{0.000000in}}{%
\pgfpathmoveto{\pgfqpoint{0.000000in}{0.000000in}}%
\pgfpathlineto{\pgfqpoint{0.048611in}{0.000000in}}%
\pgfusepath{stroke,fill}%
}%
\begin{pgfscope}%
\pgfsys@transformshift{10.544519in}{1.831791in}%
\pgfsys@useobject{currentmarker}{}%
\end{pgfscope}%
\end{pgfscope}%
\begin{pgfscope}%
\definecolor{textcolor}{rgb}{0.000000,0.000000,0.000000}%
\pgfsetstrokecolor{textcolor}%
\pgfsetfillcolor{textcolor}%
\pgftext[x=10.641741in, y=1.762347in, left, base]{\color{textcolor}{\sffamily\fontsize{15.000000}{18.000000}\selectfont\catcode`\^=\active\def^{\ifmmode\sp\else\^{}\fi}\catcode`\%=\active\def%{\%}10}}%
\end{pgfscope}%
\begin{pgfscope}%
\pgfsetbuttcap%
\pgfsetroundjoin%
\definecolor{currentfill}{rgb}{0.000000,0.000000,0.000000}%
\pgfsetfillcolor{currentfill}%
\pgfsetlinewidth{0.803000pt}%
\definecolor{currentstroke}{rgb}{0.000000,0.000000,0.000000}%
\pgfsetstrokecolor{currentstroke}%
\pgfsetdash{}{0pt}%
\pgfsys@defobject{currentmarker}{\pgfqpoint{0.000000in}{0.000000in}}{\pgfqpoint{0.048611in}{0.000000in}}{%
\pgfpathmoveto{\pgfqpoint{0.000000in}{0.000000in}}%
\pgfpathlineto{\pgfqpoint{0.048611in}{0.000000in}}%
\pgfusepath{stroke,fill}%
}%
\begin{pgfscope}%
\pgfsys@transformshift{10.544519in}{2.906433in}%
\pgfsys@useobject{currentmarker}{}%
\end{pgfscope}%
\end{pgfscope}%
\begin{pgfscope}%
\definecolor{textcolor}{rgb}{0.000000,0.000000,0.000000}%
\pgfsetstrokecolor{textcolor}%
\pgfsetfillcolor{textcolor}%
\pgftext[x=10.641741in, y=2.836988in, left, base]{\color{textcolor}{\sffamily\fontsize{15.000000}{18.000000}\selectfont\catcode`\^=\active\def^{\ifmmode\sp\else\^{}\fi}\catcode`\%=\active\def%{\%}20}}%
\end{pgfscope}%
\begin{pgfscope}%
\pgfsetbuttcap%
\pgfsetroundjoin%
\definecolor{currentfill}{rgb}{0.000000,0.000000,0.000000}%
\pgfsetfillcolor{currentfill}%
\pgfsetlinewidth{0.803000pt}%
\definecolor{currentstroke}{rgb}{0.000000,0.000000,0.000000}%
\pgfsetstrokecolor{currentstroke}%
\pgfsetdash{}{0pt}%
\pgfsys@defobject{currentmarker}{\pgfqpoint{0.000000in}{0.000000in}}{\pgfqpoint{0.048611in}{0.000000in}}{%
\pgfpathmoveto{\pgfqpoint{0.000000in}{0.000000in}}%
\pgfpathlineto{\pgfqpoint{0.048611in}{0.000000in}}%
\pgfusepath{stroke,fill}%
}%
\begin{pgfscope}%
\pgfsys@transformshift{10.544519in}{3.981075in}%
\pgfsys@useobject{currentmarker}{}%
\end{pgfscope}%
\end{pgfscope}%
\begin{pgfscope}%
\definecolor{textcolor}{rgb}{0.000000,0.000000,0.000000}%
\pgfsetstrokecolor{textcolor}%
\pgfsetfillcolor{textcolor}%
\pgftext[x=10.641741in, y=3.911630in, left, base]{\color{textcolor}{\sffamily\fontsize{15.000000}{18.000000}\selectfont\catcode`\^=\active\def^{\ifmmode\sp\else\^{}\fi}\catcode`\%=\active\def%{\%}30}}%
\end{pgfscope}%
\begin{pgfscope}%
\pgfsetbuttcap%
\pgfsetroundjoin%
\definecolor{currentfill}{rgb}{0.000000,0.000000,0.000000}%
\pgfsetfillcolor{currentfill}%
\pgfsetlinewidth{0.803000pt}%
\definecolor{currentstroke}{rgb}{0.000000,0.000000,0.000000}%
\pgfsetstrokecolor{currentstroke}%
\pgfsetdash{}{0pt}%
\pgfsys@defobject{currentmarker}{\pgfqpoint{0.000000in}{0.000000in}}{\pgfqpoint{0.048611in}{0.000000in}}{%
\pgfpathmoveto{\pgfqpoint{0.000000in}{0.000000in}}%
\pgfpathlineto{\pgfqpoint{0.048611in}{0.000000in}}%
\pgfusepath{stroke,fill}%
}%
\begin{pgfscope}%
\pgfsys@transformshift{10.544519in}{5.055716in}%
\pgfsys@useobject{currentmarker}{}%
\end{pgfscope}%
\end{pgfscope}%
\begin{pgfscope}%
\definecolor{textcolor}{rgb}{0.000000,0.000000,0.000000}%
\pgfsetstrokecolor{textcolor}%
\pgfsetfillcolor{textcolor}%
\pgftext[x=10.641741in, y=4.986272in, left, base]{\color{textcolor}{\sffamily\fontsize{15.000000}{18.000000}\selectfont\catcode`\^=\active\def^{\ifmmode\sp\else\^{}\fi}\catcode`\%=\active\def%{\%}40}}%
\end{pgfscope}%
\begin{pgfscope}%
\pgfsetbuttcap%
\pgfsetroundjoin%
\definecolor{currentfill}{rgb}{0.000000,0.000000,0.000000}%
\pgfsetfillcolor{currentfill}%
\pgfsetlinewidth{0.803000pt}%
\definecolor{currentstroke}{rgb}{0.000000,0.000000,0.000000}%
\pgfsetstrokecolor{currentstroke}%
\pgfsetdash{}{0pt}%
\pgfsys@defobject{currentmarker}{\pgfqpoint{0.000000in}{0.000000in}}{\pgfqpoint{0.048611in}{0.000000in}}{%
\pgfpathmoveto{\pgfqpoint{0.000000in}{0.000000in}}%
\pgfpathlineto{\pgfqpoint{0.048611in}{0.000000in}}%
\pgfusepath{stroke,fill}%
}%
\begin{pgfscope}%
\pgfsys@transformshift{10.544519in}{6.130358in}%
\pgfsys@useobject{currentmarker}{}%
\end{pgfscope}%
\end{pgfscope}%
\begin{pgfscope}%
\definecolor{textcolor}{rgb}{0.000000,0.000000,0.000000}%
\pgfsetstrokecolor{textcolor}%
\pgfsetfillcolor{textcolor}%
\pgftext[x=10.641741in, y=6.060914in, left, base]{\color{textcolor}{\sffamily\fontsize{15.000000}{18.000000}\selectfont\catcode`\^=\active\def^{\ifmmode\sp\else\^{}\fi}\catcode`\%=\active\def%{\%}50}}%
\end{pgfscope}%
\begin{pgfscope}%
\pgfsetbuttcap%
\pgfsetroundjoin%
\definecolor{currentfill}{rgb}{0.000000,0.000000,0.000000}%
\pgfsetfillcolor{currentfill}%
\pgfsetlinewidth{0.803000pt}%
\definecolor{currentstroke}{rgb}{0.000000,0.000000,0.000000}%
\pgfsetstrokecolor{currentstroke}%
\pgfsetdash{}{0pt}%
\pgfsys@defobject{currentmarker}{\pgfqpoint{0.000000in}{0.000000in}}{\pgfqpoint{0.048611in}{0.000000in}}{%
\pgfpathmoveto{\pgfqpoint{0.000000in}{0.000000in}}%
\pgfpathlineto{\pgfqpoint{0.048611in}{0.000000in}}%
\pgfusepath{stroke,fill}%
}%
\begin{pgfscope}%
\pgfsys@transformshift{10.544519in}{7.205000in}%
\pgfsys@useobject{currentmarker}{}%
\end{pgfscope}%
\end{pgfscope}%
\begin{pgfscope}%
\definecolor{textcolor}{rgb}{0.000000,0.000000,0.000000}%
\pgfsetstrokecolor{textcolor}%
\pgfsetfillcolor{textcolor}%
\pgftext[x=10.641741in, y=7.135556in, left, base]{\color{textcolor}{\sffamily\fontsize{15.000000}{18.000000}\selectfont\catcode`\^=\active\def^{\ifmmode\sp\else\^{}\fi}\catcode`\%=\active\def%{\%}60}}%
\end{pgfscope}%
\begin{pgfscope}%
\definecolor{textcolor}{rgb}{0.000000,0.000000,0.000000}%
\pgfsetstrokecolor{textcolor}%
\pgfsetfillcolor{textcolor}%
\pgftext[x=10.893128in,y=4.309055in,,top,rotate=90.000000]{\color{textcolor}{\sffamily\fontsize{20.000000}{24.000000}\selectfont\catcode`\^=\active\def^{\ifmmode\sp\else\^{}\fi}\catcode`\%=\active\def%{\%}Share of Renewable Energy (%)}}%
\end{pgfscope}%
\begin{pgfscope}%
\pgfsetrectcap%
\pgfsetmiterjoin%
\pgfsetlinewidth{0.803000pt}%
\definecolor{currentstroke}{rgb}{0.000000,0.000000,0.000000}%
\pgfsetstrokecolor{currentstroke}%
\pgfsetdash{}{0pt}%
\pgfpathmoveto{\pgfqpoint{10.190425in}{0.768110in}}%
\pgfpathlineto{\pgfqpoint{10.367472in}{0.768110in}}%
\pgfpathlineto{\pgfqpoint{10.544519in}{0.768110in}}%
\pgfpathlineto{\pgfqpoint{10.544519in}{7.850000in}}%
\pgfpathlineto{\pgfqpoint{10.367472in}{7.850000in}}%
\pgfpathlineto{\pgfqpoint{10.190425in}{7.850000in}}%
\pgfpathlineto{\pgfqpoint{10.190425in}{0.768110in}}%
\pgfpathclose%
\pgfusepath{stroke}%
\end{pgfscope}%
\end{pgfpicture}%
\makeatother%
\endgroup%
}
    \caption{Global Correlation between Energy Consumption and Greenhouse Gas Emissions}
    \label{plt:global_consumption_vs_emissions}
\end{figure}

\begin{figure}
    \centering
    \begin{subfigure}[b]{0.49\textwidth}
        \centering
        \resizebox{\textwidth}{!}{\input{plot_country_data_with_renewables_LU.pgf}}
        \caption{Luxembourg}
        \label{plt:LU_consumption_vs_emissions}
    \end{subfigure}
    \hfill
    \begin{subfigure}[b]{0.49\textwidth}
        \centering
        \resizebox{\textwidth}{!}{\input{plot_country_data_with_renewables_AT.pgf}}
        \caption{Austria}
        \label{plt:AT_consumption_vs_emissions}
    \end{subfigure}
    \caption{National Correlation between Energy Consumption and Greenhouse Gas Emissions}
\end{figure}

\subsection*{Interpretation of the Findings}
The findings align with the understanding, that ...

\section*{Conclusions}

\end{document}
