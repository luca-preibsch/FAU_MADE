\documentclass{article}
    % General document formatting
    \usepackage[margin=0.7in]{geometry}
    \usepackage[parfill]{parskip}
    \usepackage[utf8]{inputenc}

    \usepackage[colorlinks]{hyperref}
    \usepackage{caption}
    \usepackage{subcaption}
    \usepackage{array}
    \usepackage{amsmath}
    \usepackage{cleveref}

\begin{document}

\title{Final Report on the Topic of Climate Change
for the Cource Methods of Advanced Data Engineering at FAU}
\author{Luca Preibsch}
\date{\today}

% \begin{center}
% Luca Preibsch - \today
% \end{center}

\maketitle
\section*{Introduction}

Climate change is a huge challenge worldide and has gained more attention in recent.
The global cause to stop climate change or at least slow it down often focusses on global warming, like the
\href{https://unfccc.int/process-and-meetings/the-paris-agreement}{paris climate agreement},
which legally binds the 196 parties "to limit the temperatute increase to 1.5°C above pre-industrial levels".

Among various other contributors to global warming, burning fossil fuels for energy consumption stands out as
a primary factor to this day.
When burning fossil fuels, substantial amounts of greenhouse gases are released into the atmosphere,
which increase the so called greenhouse effect, in turn accelerating global warming and climate change.
Facing the amount of greenhouse gases produced by the energy sector, many countries - including countries of
the European Union (EU) - intensified their efforts to transition towards more sustainable energy sources.

This report aims to explore the aforementioned relationship between energy consumption and net greenhouse gas
emissions across countries of the EU, in order to show how they are interconnected.
Furthermore this report is set to discover how this connection is influenced by the share of renewable energy sources
in the energy pool of a country. This might help to indicate, how effective the adoption of renewable energy sources is
in fighting climate change.

The resulting core questions are:
\begin{enumerate}
    \item How does the amount of energy consumed influence the net greenhouse gas emissions of European countries?
    \item And how is this influenced by the share of renewables in total energy?
\end{enumerate}

\section*{Used Data}
This report uses a merged dataset produces by a data pipeline, derived from three data sources,
all originating from \href{https://ec.europa.eu/eurostat}{Eurostat}.
The resulting dataset provides a comprehensive view of net greenhouse gas emissions, primary energy consumption
and the share of renewable energy sources across countries of the EU.

\subsection*{Structure and Meaning of the Dataset}
The dataset is structured as a SQLite database containing three tables, each covering one data source from Eurostat.
Each table includes the following three columns:
\begin{itemize}
    \item \textbf{geo}: The \href{https://www.destatis.de/Europa/EN/Country/Country-Codes.html}{ALPHA-2 country codes} representing each EU country.
    \item \textbf{year}: The year in which the data was recorded.
    \item \textbf{value}: The observed value for the specific metric in each table.
\end{itemize}

\subsubsection*{Net Greenhouse Gas Emissions}
This table contains the amount of greenhouse gases emitted per person in each country for the specified years.
The emissions data is measured in tonnes per capita.
An example of how the data is structured can be seen in table \ref{tab:emissions}.

\subsubsection*{Primary Energy Consumption}
This table shows the energy consumption per person in each country, measured in Tonnes of Oil Equivalent (TOE) per capita.
An example of how the data is structured can be seen in table \ref{tab:consumption}.

\subsubsection*{Share of Energy from Renewable Sources}
This table covers the proportion of energy consumed, which originates from renewable sources as declared by the EU,
highlighting the adoption of renewable energy sources in each country.
The data is measured in percentage of the total consumed energy.
An example of how the data is structured can be seen in table \ref{tab:share}.

\begin{figure}[h!]
    \centering
    \begin{subfigure}[b]{0.3\textwidth}
        \centering
        \begin{tabular}{c c c}
            geo & year & value \\
            \hline\hline
            AT & 1990 & 8.9 \\
            AT & 1991 & 8.4 \\
            AT & 1992 & 8.7 \\
            \dots & \dots & \dots
        \end{tabular}
        \caption{Net Greenhouse Gas Emissions}
        \label{tab:emissions}
    \end{subfigure}
    \hfill
    \begin{subfigure}[b]{0.3\textwidth}
        \centering
        \begin{tabular}{c c c}
            geo & year & value \\
            \hline\hline
            AT & 2000 & 3.43 \\
            AT & 2001 & 3.62 \\
            AT & 2002 & 3.62 \\
            \dots & \dots & \dots
        \end{tabular}
        \caption{Primary Energy Consumption}
        \label{tab:consumption}
    \end{subfigure}
    \hfill
    \begin{subfigure}[b]{0.3\textwidth}
        \centering
        \begin{tabular}{c c c}
            geo & year & value \\
            \hline\hline
            AT & 2004 & 22.553 \\
            AT & 2005 & 24.353 \\
            AT & 2006 & 26.276 \\
            \dots & \dots & \dots
        \end{tabular}
        \caption{Share of Renewable Sources}
        \label{tab:share}
    \end{subfigure}
    \caption{Dataset Table Structure}
    \label{fig:three_tables}
\end{figure}

\subsection*{Compliance with Data Licenses}
All data used in this analysis is sourced from Eurostat and is subject to the
\href{https://ec.europa.eu/eurostat/about-us/policies/copyright}{Eurostat copyright notice},
which allows free re-use of data under obligations.

In order to comply, this report attributes Eurostat as the source of all data.
Furthermore, during the processing of the data sources by the pipeline, the source data was cleaned with certain
rows and columns being removed for relevance and consistency.

In detail the following modifications to the source data were performed:
\begin{itemize}
    \item Table Net Greenhouse Gas Emissions: all rows were removed, which did not cover the unit of measure "Total (Tonnes per capita)";
    all rows that covered other source sectors than "Total (excluding memo items, including international aviation)" were removed.
    \item Table Primary Energy Consumption: all rows were removed, that did not cover the unit of measure "Tonnes of oil equivalent (TOE) per capita".
    \item Table Share of Energy from Renewable Sources: all rows were removed, that did not cover the topic "Renewable energy sources".
    \item For all tables all columns except "geo", "TIME\_PERIOD" and "OBS\_VALUE" were removed.
    \item For all tables, the headers for "TIME\_PERIOD" and "OBS\_VALUE" were renamed to "year" and "value" respectively.
    \item For all tables, all rows were removed that did cover countries, which are not currently part of the \href{https://www.destatis.de/Europa/EN/Country/Country-Codes.html}{27 EU member states}.
\end{itemize}

\section*{Analysis}

\section*{Conclusions}

\end{document}
