\documentclass{article}
    % General document formatting
    \usepackage[margin=0.7in]{geometry}
    \usepackage[parfill]{parskip}
    \usepackage[utf8]{inputenc}

    \usepackage[colorlinks]{hyperref}
    \usepackage{caption}
    \usepackage{subcaption}
    \usepackage{array}
    \usepackage{amsmath}
    \usepackage{cleveref}
    \usepackage{pgf}
    \usepackage{import}

\begin{document}

\title{Final Report on the Topic of Climate Change
for the Cource Methods of Advanced Data Engineering at FAU}
\author{Luca Preibsch}
\date{\today}

\begin{center}
Final Report - Luca Preibsch - \today
\end{center}

% \maketitle

\section*{Introduction}
Climate change is a huge challenge worldide and has gained more attention in recent.
The global cause to stop climate change or at least slow it down often focusses on global warming, like the
\href{https://unfccc.int/process-and-meetings/the-paris-agreement}{paris climate agreement},
which legally binds the 196 parties "to limit the temperatute increase to 1.5°C above pre-industrial levels".

Among various other contributors to global warming, burning fossil fuels for energy consumption stands out as
a primary factor to this day.
When burning fossil fuels, substantial amounts of greenhouse gases are released into the atmosphere,
which increase the so called greenhouse effect, in turn accelerating global warming and climate change.
Facing the amount of greenhouse gases produced by the energy sector, many countries - including countries of
the European Union (EU) - intensified their efforts to transition towards more sustainable energy sources.

This report aims to explore the aforementioned relationship between energy consumption and net greenhouse gas
emissions across countries of the EU, in order to show how they are interconnected.
Furthermore this report is set to discover how this connection is influenced by the share of renewable energy sources
in the energy pool of a country. This might help to indicate, how effective the adoption of renewable energy sources is
in fighting climate change.

The resulting core questions are:
\begin{enumerate}
    \item How does the amount of energy consumed influence the net greenhouse gas emissions of European countries?
    \item And how is this influenced by the share of renewables in total energy?
\end{enumerate}

\section*{Used Data}
This report uses a merged dataset produces by a data pipeline, derived from three data sources,
all originating from \href{https://ec.europa.eu/eurostat}{Eurostat}.
The resulting dataset provides a comprehensive view of net greenhouse gas emissions, primary energy consumption
and the share of renewable energy sources across countries of the EU.

\subsection*{Structure and Meaning of the Dataset}
The dataset is structured as a SQLite database containing three tables, each covering one data source from Eurostat.
Each table includes the following three columns:
\begin{itemize}
    \item \textbf{geo}: The \href{https://www.destatis.de/Europa/EN/Country/Country-Codes.html}{ALPHA-2 country codes} representing each EU country.
    \item \textbf{year}: The year in which the data was recorded.
    \item \textbf{value}: The observed value for the specific metric in each table.
\end{itemize}

\textbf{Net Greenhouse Gas Emissions}:
This table contains the amount of greenhouse gases emitted per person in each country for the specified years.
The emissions data is measured in tonnes per capita.
% An example of how the data is structured can be seen in table \ref{tab:emissions}.

\textbf{Primary Energy Consumption}:
This table shows the energy consumption per person in each country, measured in Tonnes of Oil Equivalent (TOE) per capita.
% An example of how the data is structured can be seen in table \ref{tab:consumption}.

\textbf{Share of Energy from Renewable Sources}:
This table covers the proportion of energy consumed, which originates from renewable sources as declared by the EU,
highlighting the adoption of renewable energy sources in each country.
The data is measured in percentage of the total consumed energy.
% An example of how the data is structured can be seen in table \ref{tab:share}.

% \begin{figure}[h!]
%     \centering
%     \begin{subfigure}[b]{0.3\textwidth}
%         \centering
%         \begin{tabular}{c c c}
%             geo & year & value \\
%             \hline\hline
%             AT & 1990 & 8.9 \\
%             AT & 1991 & 8.4 \\
%             AT & 1992 & 8.7 \\
%             \dots & \dots & \dots
%         \end{tabular}
%         \caption{Net Greenhouse Gas Emissions}
%         \label{tab:emissions}
%     \end{subfigure}
%     \hfill
%     \begin{subfigure}[b]{0.3\textwidth}
%         \centering
%         \begin{tabular}{c c c}
%             geo & year & value \\
%             \hline\hline
%             AT & 2000 & 3.43 \\
%             AT & 2001 & 3.62 \\
%             AT & 2002 & 3.62 \\
%             \dots & \dots & \dots
%         \end{tabular}
%         \caption{Primary Energy Consumption}
%         \label{tab:consumption}
%     \end{subfigure}
%     \hfill
%     \begin{subfigure}[b]{0.3\textwidth}
%         \centering
%         \begin{tabular}{c c c}
%             geo & year & value \\
%             \hline\hline
%             AT & 2004 & 22.553 \\
%             AT & 2005 & 24.353 \\
%             AT & 2006 & 26.276 \\
%             \dots & \dots & \dots
%         \end{tabular}
%         \caption{Share of Renewable Sources}
%         \label{tab:share}
%     \end{subfigure}
%     \caption{Dataset Table Structure}
% \end{figure}

\subsection*{Compliance with Data Licenses}
All data used in this analysis is sourced from Eurostat and is subject to the
\href{https://ec.europa.eu/eurostat/about-us/policies/copyright}{Eurostat copyright notice},
which allows free re-use of data under obligations.

In order to comply, this report attributes Eurostat as the source of all data.
Furthermore, during the processing of the data sources by the pipeline, the source data was cleaned with certain
rows and columns being removed for relevance and consistency.

In detail the following modifications to the source data were performed:
\begin{itemize}
    \item Table Net Greenhouse Gas Emissions: all rows were removed, which did not cover the unit of measure "Total (Tonnes per capita)";
    all rows that covered other source sectors than "Total (excluding memo items, including international aviation)" were removed.
    \item Table Primary Energy Consumption: all rows were removed, that did not cover the unit of measure "Tonnes of oil equivalent (TOE) per capita".
    \item Table Share of Energy from Renewable Sources: all rows were removed, that did not cover the topic "Renewable energy sources".
    \item For all tables all columns except "geo", "TIME\_PERIOD" and "OBS\_VALUE" were removed.
    \item For all tables, the headers for "TIME\_PERIOD" and "OBS\_VALUE" were renamed to "year" and "value" respectively.
    \item For all tables, all rows were removed that did cover countries, which are not currently part of the \href{https://www.destatis.de/Europa/EN/Country/Country-Codes.html}{27 EU member states}.
\end{itemize}

\section*{Analysis of the Data}
\subsection*{Correlation between Energy Consumption and Emissions}
\subsubsection*{Method Used}
In order to answer the first question considering the correlation between emissions and energy consumption, the data first had to be prepared:
The data on emissions, energy consumption and the share of renewables was extracted from the SQLite database and the datasets were merged on
country codes and years to create a unified dataset for analysis.
The approach then was to create a scatter plot for visualizing the relationship between energy consumption and the
greenhouse gas emissions. The plot includes data points for all countries and all years in order to give an overview
over the EU as a whole. The single dots are colored appropriate to a scale.
This scale was then included in the plot in the form of a colorbar as a legend.
In order to make the plot more readable, linear regression was used to create a linear function describing the data.
Also, to assess the strength of the two datasets correlation, the Pearson Correlation Coefficient (PCC, $r$-value) was calculated.

Since the global plot shadows differences between single countries, also plots for all countries were created in order to gain
insights into the workings of the relationship on a national level.

\subsubsection*{Results}
The resulting global plot can be seen in \cref{plt:global_consumption_vs_emissions}.
The scatter plot backed by the regression line point to a positive correlation between energy consumption and greenhouse gas emissions.
Furthermore, the $r$-value is calculated to be .64, indicating a moderate to strong positive correlation.

Although, when observing plots of single countries, huge differences of the analyzed correlations for different countries arise.
Luxembourg shows a very strong PCC of $r=1$, while Austria has a moderate correlation with an $r$-value of 0.56.

Luxembourg represents a positive extreme, whereas another example, Slovenia represents the opposite extreme,
showing no positive correlation with an $r$-value of -0.46. Slovenia is a notable exception,
as it is the only country with a negative correlation.

Overall, the median $r$-value across all countries is 0.86, which is significantly higher than the PCC for the EU as a whole.
This high median value indicates a very strong positive correlation in the median, suggesting that most individual countries exhibit a stronger
relationship between energy consumption and greenhouse gas emissions than the aggregated EU data.

\begin{figure}
    \centering
    \resizebox{.7\textwidth}{!}{%% Creator: Matplotlib, PGF backend
%%
%% To include the figure in your LaTeX document, write
%%   \input{<filename>.pgf}
%%
%% Make sure the required packages are loaded in your preamble
%%   \usepackage{pgf}
%%
%% Also ensure that all the required font packages are loaded; for instance,
%% the lmodern package is sometimes necessary when using math font.
%%   \usepackage{lmodern}
%%
%% Figures using additional raster images can only be included by \input if
%% they are in the same directory as the main LaTeX file. For loading figures
%% from other directories you can use the `import` package
%%   \usepackage{import}
%%
%% and then include the figures with
%%   \import{<path to file>}{<filename>.pgf}
%%
%% Matplotlib used the following preamble
%%   \def\mathdefault#1{#1}
%%   \everymath=\expandafter{\the\everymath\displaystyle}
%%   
%%   \makeatletter\@ifpackageloaded{underscore}{}{\usepackage[strings]{underscore}}\makeatother
%%
\begingroup%
\makeatletter%
\begin{pgfpicture}%
\pgfpathrectangle{\pgfpointorigin}{\pgfqpoint{12.000000in}{8.000000in}}%
\pgfusepath{use as bounding box, clip}%
\begin{pgfscope}%
\pgfsetbuttcap%
\pgfsetmiterjoin%
\definecolor{currentfill}{rgb}{1.000000,1.000000,1.000000}%
\pgfsetfillcolor{currentfill}%
\pgfsetlinewidth{0.000000pt}%
\definecolor{currentstroke}{rgb}{1.000000,1.000000,1.000000}%
\pgfsetstrokecolor{currentstroke}%
\pgfsetdash{}{0pt}%
\pgfpathmoveto{\pgfqpoint{0.000000in}{0.000000in}}%
\pgfpathlineto{\pgfqpoint{12.000000in}{0.000000in}}%
\pgfpathlineto{\pgfqpoint{12.000000in}{8.000000in}}%
\pgfpathlineto{\pgfqpoint{0.000000in}{8.000000in}}%
\pgfpathlineto{\pgfqpoint{0.000000in}{0.000000in}}%
\pgfpathclose%
\pgfusepath{fill}%
\end{pgfscope}%
\begin{pgfscope}%
\pgfsetbuttcap%
\pgfsetmiterjoin%
\definecolor{currentfill}{rgb}{1.000000,1.000000,1.000000}%
\pgfsetfillcolor{currentfill}%
\pgfsetlinewidth{0.000000pt}%
\definecolor{currentstroke}{rgb}{0.000000,0.000000,0.000000}%
\pgfsetstrokecolor{currentstroke}%
\pgfsetstrokeopacity{0.000000}%
\pgfsetdash{}{0pt}%
\pgfpathmoveto{\pgfqpoint{0.786164in}{0.768110in}}%
\pgfpathlineto{\pgfqpoint{9.637233in}{0.768110in}}%
\pgfpathlineto{\pgfqpoint{9.637233in}{7.850000in}}%
\pgfpathlineto{\pgfqpoint{0.786164in}{7.850000in}}%
\pgfpathlineto{\pgfqpoint{0.786164in}{0.768110in}}%
\pgfpathclose%
\pgfusepath{fill}%
\end{pgfscope}%
\begin{pgfscope}%
\pgfpathrectangle{\pgfqpoint{0.786164in}{0.768110in}}{\pgfqpoint{8.851069in}{7.081890in}}%
\pgfusepath{clip}%
\pgfsetbuttcap%
\pgfsetroundjoin%
\definecolor{currentfill}{rgb}{0.187231,0.414746,0.556547}%
\pgfsetfillcolor{currentfill}%
\pgfsetfillopacity{0.700000}%
\pgfsetlinewidth{0.501875pt}%
\definecolor{currentstroke}{rgb}{1.000000,1.000000,1.000000}%
\pgfsetstrokecolor{currentstroke}%
\pgfsetstrokeopacity{0.700000}%
\pgfsetdash{}{0pt}%
\pgfpathmoveto{\pgfqpoint{3.362208in}{2.924158in}}%
\pgfpathcurveto{\pgfqpoint{3.375230in}{2.924158in}}{\pgfqpoint{3.387721in}{2.929332in}}{\pgfqpoint{3.396930in}{2.938541in}}%
\pgfpathcurveto{\pgfqpoint{3.406138in}{2.947749in}}{\pgfqpoint{3.411312in}{2.960240in}}{\pgfqpoint{3.411312in}{2.973263in}}%
\pgfpathcurveto{\pgfqpoint{3.411312in}{2.986286in}}{\pgfqpoint{3.406138in}{2.998777in}}{\pgfqpoint{3.396930in}{3.007985in}}%
\pgfpathcurveto{\pgfqpoint{3.387721in}{3.017193in}}{\pgfqpoint{3.375230in}{3.022367in}}{\pgfqpoint{3.362208in}{3.022367in}}%
\pgfpathcurveto{\pgfqpoint{3.349185in}{3.022367in}}{\pgfqpoint{3.336694in}{3.017193in}}{\pgfqpoint{3.327485in}{3.007985in}}%
\pgfpathcurveto{\pgfqpoint{3.318277in}{2.998777in}}{\pgfqpoint{3.313103in}{2.986286in}}{\pgfqpoint{3.313103in}{2.973263in}}%
\pgfpathcurveto{\pgfqpoint{3.313103in}{2.960240in}}{\pgfqpoint{3.318277in}{2.947749in}}{\pgfqpoint{3.327485in}{2.938541in}}%
\pgfpathcurveto{\pgfqpoint{3.336694in}{2.929332in}}{\pgfqpoint{3.349185in}{2.924158in}}{\pgfqpoint{3.362208in}{2.924158in}}%
\pgfpathlineto{\pgfqpoint{3.362208in}{2.924158in}}%
\pgfpathclose%
\pgfusepath{stroke,fill}%
\end{pgfscope}%
\begin{pgfscope}%
\pgfpathrectangle{\pgfqpoint{0.786164in}{0.768110in}}{\pgfqpoint{8.851069in}{7.081890in}}%
\pgfusepath{clip}%
\pgfsetbuttcap%
\pgfsetroundjoin%
\definecolor{currentfill}{rgb}{0.175841,0.441290,0.557685}%
\pgfsetfillcolor{currentfill}%
\pgfsetfillopacity{0.700000}%
\pgfsetlinewidth{0.501875pt}%
\definecolor{currentstroke}{rgb}{1.000000,1.000000,1.000000}%
\pgfsetstrokecolor{currentstroke}%
\pgfsetstrokeopacity{0.700000}%
\pgfsetdash{}{0pt}%
\pgfpathmoveto{\pgfqpoint{3.508340in}{3.055548in}}%
\pgfpathcurveto{\pgfqpoint{3.521363in}{3.055548in}}{\pgfqpoint{3.533854in}{3.060722in}}{\pgfqpoint{3.543062in}{3.069930in}}%
\pgfpathcurveto{\pgfqpoint{3.552271in}{3.079138in}}{\pgfqpoint{3.557445in}{3.091630in}}{\pgfqpoint{3.557445in}{3.104652in}}%
\pgfpathcurveto{\pgfqpoint{3.557445in}{3.117675in}}{\pgfqpoint{3.552271in}{3.130166in}}{\pgfqpoint{3.543062in}{3.139374in}}%
\pgfpathcurveto{\pgfqpoint{3.533854in}{3.148583in}}{\pgfqpoint{3.521363in}{3.153757in}}{\pgfqpoint{3.508340in}{3.153757in}}%
\pgfpathcurveto{\pgfqpoint{3.495318in}{3.153757in}}{\pgfqpoint{3.482826in}{3.148583in}}{\pgfqpoint{3.473618in}{3.139374in}}%
\pgfpathcurveto{\pgfqpoint{3.464410in}{3.130166in}}{\pgfqpoint{3.459236in}{3.117675in}}{\pgfqpoint{3.459236in}{3.104652in}}%
\pgfpathcurveto{\pgfqpoint{3.459236in}{3.091630in}}{\pgfqpoint{3.464410in}{3.079138in}}{\pgfqpoint{3.473618in}{3.069930in}}%
\pgfpathcurveto{\pgfqpoint{3.482826in}{3.060722in}}{\pgfqpoint{3.495318in}{3.055548in}}{\pgfqpoint{3.508340in}{3.055548in}}%
\pgfpathlineto{\pgfqpoint{3.508340in}{3.055548in}}%
\pgfpathclose%
\pgfusepath{stroke,fill}%
\end{pgfscope}%
\begin{pgfscope}%
\pgfpathrectangle{\pgfqpoint{0.786164in}{0.768110in}}{\pgfqpoint{8.851069in}{7.081890in}}%
\pgfusepath{clip}%
\pgfsetbuttcap%
\pgfsetroundjoin%
\definecolor{currentfill}{rgb}{0.165117,0.467423,0.558141}%
\pgfsetfillcolor{currentfill}%
\pgfsetfillopacity{0.700000}%
\pgfsetlinewidth{0.501875pt}%
\definecolor{currentstroke}{rgb}{1.000000,1.000000,1.000000}%
\pgfsetstrokecolor{currentstroke}%
\pgfsetstrokeopacity{0.700000}%
\pgfsetdash{}{0pt}%
\pgfpathmoveto{\pgfqpoint{3.480940in}{3.230733in}}%
\pgfpathcurveto{\pgfqpoint{3.493963in}{3.230733in}}{\pgfqpoint{3.506454in}{3.235907in}}{\pgfqpoint{3.515663in}{3.245116in}}%
\pgfpathcurveto{\pgfqpoint{3.524871in}{3.254324in}}{\pgfqpoint{3.530045in}{3.266815in}}{\pgfqpoint{3.530045in}{3.279838in}}%
\pgfpathcurveto{\pgfqpoint{3.530045in}{3.292861in}}{\pgfqpoint{3.524871in}{3.305352in}}{\pgfqpoint{3.515663in}{3.314560in}}%
\pgfpathcurveto{\pgfqpoint{3.506454in}{3.323769in}}{\pgfqpoint{3.493963in}{3.328943in}}{\pgfqpoint{3.480940in}{3.328943in}}%
\pgfpathcurveto{\pgfqpoint{3.467918in}{3.328943in}}{\pgfqpoint{3.455427in}{3.323769in}}{\pgfqpoint{3.446218in}{3.314560in}}%
\pgfpathcurveto{\pgfqpoint{3.437010in}{3.305352in}}{\pgfqpoint{3.431836in}{3.292861in}}{\pgfqpoint{3.431836in}{3.279838in}}%
\pgfpathcurveto{\pgfqpoint{3.431836in}{3.266815in}}{\pgfqpoint{3.437010in}{3.254324in}}{\pgfqpoint{3.446218in}{3.245116in}}%
\pgfpathcurveto{\pgfqpoint{3.455427in}{3.235907in}}{\pgfqpoint{3.467918in}{3.230733in}}{\pgfqpoint{3.480940in}{3.230733in}}%
\pgfpathlineto{\pgfqpoint{3.480940in}{3.230733in}}%
\pgfpathclose%
\pgfusepath{stroke,fill}%
\end{pgfscope}%
\begin{pgfscope}%
\pgfpathrectangle{\pgfqpoint{0.786164in}{0.768110in}}{\pgfqpoint{8.851069in}{7.081890in}}%
\pgfusepath{clip}%
\pgfsetbuttcap%
\pgfsetroundjoin%
\definecolor{currentfill}{rgb}{0.154815,0.493313,0.557840}%
\pgfsetfillcolor{currentfill}%
\pgfsetfillopacity{0.700000}%
\pgfsetlinewidth{0.501875pt}%
\definecolor{currentstroke}{rgb}{1.000000,1.000000,1.000000}%
\pgfsetstrokecolor{currentstroke}%
\pgfsetstrokeopacity{0.700000}%
\pgfsetdash{}{0pt}%
\pgfpathmoveto{\pgfqpoint{3.417007in}{3.252632in}}%
\pgfpathcurveto{\pgfqpoint{3.430030in}{3.252632in}}{\pgfqpoint{3.442521in}{3.257806in}}{\pgfqpoint{3.451730in}{3.267014in}}%
\pgfpathcurveto{\pgfqpoint{3.460938in}{3.276223in}}{\pgfqpoint{3.466112in}{3.288714in}}{\pgfqpoint{3.466112in}{3.301736in}}%
\pgfpathcurveto{\pgfqpoint{3.466112in}{3.314759in}}{\pgfqpoint{3.460938in}{3.327250in}}{\pgfqpoint{3.451730in}{3.336459in}}%
\pgfpathcurveto{\pgfqpoint{3.442521in}{3.345667in}}{\pgfqpoint{3.430030in}{3.350841in}}{\pgfqpoint{3.417007in}{3.350841in}}%
\pgfpathcurveto{\pgfqpoint{3.403985in}{3.350841in}}{\pgfqpoint{3.391494in}{3.345667in}}{\pgfqpoint{3.382285in}{3.336459in}}%
\pgfpathcurveto{\pgfqpoint{3.373077in}{3.327250in}}{\pgfqpoint{3.367903in}{3.314759in}}{\pgfqpoint{3.367903in}{3.301736in}}%
\pgfpathcurveto{\pgfqpoint{3.367903in}{3.288714in}}{\pgfqpoint{3.373077in}{3.276223in}}{\pgfqpoint{3.382285in}{3.267014in}}%
\pgfpathcurveto{\pgfqpoint{3.391494in}{3.257806in}}{\pgfqpoint{3.403985in}{3.252632in}}{\pgfqpoint{3.417007in}{3.252632in}}%
\pgfpathlineto{\pgfqpoint{3.417007in}{3.252632in}}%
\pgfpathclose%
\pgfusepath{stroke,fill}%
\end{pgfscope}%
\begin{pgfscope}%
\pgfpathrectangle{\pgfqpoint{0.786164in}{0.768110in}}{\pgfqpoint{8.851069in}{7.081890in}}%
\pgfusepath{clip}%
\pgfsetbuttcap%
\pgfsetroundjoin%
\definecolor{currentfill}{rgb}{0.150476,0.504369,0.557430}%
\pgfsetfillcolor{currentfill}%
\pgfsetfillopacity{0.700000}%
\pgfsetlinewidth{0.501875pt}%
\definecolor{currentstroke}{rgb}{1.000000,1.000000,1.000000}%
\pgfsetstrokecolor{currentstroke}%
\pgfsetstrokeopacity{0.700000}%
\pgfsetdash{}{0pt}%
\pgfpathmoveto{\pgfqpoint{3.435274in}{3.055548in}}%
\pgfpathcurveto{\pgfqpoint{3.448297in}{3.055548in}}{\pgfqpoint{3.460788in}{3.060722in}}{\pgfqpoint{3.469996in}{3.069930in}}%
\pgfpathcurveto{\pgfqpoint{3.479205in}{3.079138in}}{\pgfqpoint{3.484379in}{3.091630in}}{\pgfqpoint{3.484379in}{3.104652in}}%
\pgfpathcurveto{\pgfqpoint{3.484379in}{3.117675in}}{\pgfqpoint{3.479205in}{3.130166in}}{\pgfqpoint{3.469996in}{3.139374in}}%
\pgfpathcurveto{\pgfqpoint{3.460788in}{3.148583in}}{\pgfqpoint{3.448297in}{3.153757in}}{\pgfqpoint{3.435274in}{3.153757in}}%
\pgfpathcurveto{\pgfqpoint{3.422251in}{3.153757in}}{\pgfqpoint{3.409760in}{3.148583in}}{\pgfqpoint{3.400552in}{3.139374in}}%
\pgfpathcurveto{\pgfqpoint{3.391343in}{3.130166in}}{\pgfqpoint{3.386169in}{3.117675in}}{\pgfqpoint{3.386169in}{3.104652in}}%
\pgfpathcurveto{\pgfqpoint{3.386169in}{3.091630in}}{\pgfqpoint{3.391343in}{3.079138in}}{\pgfqpoint{3.400552in}{3.069930in}}%
\pgfpathcurveto{\pgfqpoint{3.409760in}{3.060722in}}{\pgfqpoint{3.422251in}{3.055548in}}{\pgfqpoint{3.435274in}{3.055548in}}%
\pgfpathlineto{\pgfqpoint{3.435274in}{3.055548in}}%
\pgfpathclose%
\pgfusepath{stroke,fill}%
\end{pgfscope}%
\begin{pgfscope}%
\pgfpathrectangle{\pgfqpoint{0.786164in}{0.768110in}}{\pgfqpoint{8.851069in}{7.081890in}}%
\pgfusepath{clip}%
\pgfsetbuttcap%
\pgfsetroundjoin%
\definecolor{currentfill}{rgb}{0.137770,0.537492,0.554906}%
\pgfsetfillcolor{currentfill}%
\pgfsetfillopacity{0.700000}%
\pgfsetlinewidth{0.501875pt}%
\definecolor{currentstroke}{rgb}{1.000000,1.000000,1.000000}%
\pgfsetstrokecolor{currentstroke}%
\pgfsetstrokeopacity{0.700000}%
\pgfsetdash{}{0pt}%
\pgfpathmoveto{\pgfqpoint{3.225208in}{2.967955in}}%
\pgfpathcurveto{\pgfqpoint{3.238231in}{2.967955in}}{\pgfqpoint{3.250722in}{2.973129in}}{\pgfqpoint{3.259931in}{2.982337in}}%
\pgfpathcurveto{\pgfqpoint{3.269139in}{2.991545in}}{\pgfqpoint{3.274313in}{3.004037in}}{\pgfqpoint{3.274313in}{3.017059in}}%
\pgfpathcurveto{\pgfqpoint{3.274313in}{3.030082in}}{\pgfqpoint{3.269139in}{3.042573in}}{\pgfqpoint{3.259931in}{3.051781in}}%
\pgfpathcurveto{\pgfqpoint{3.250722in}{3.060990in}}{\pgfqpoint{3.238231in}{3.066164in}}{\pgfqpoint{3.225208in}{3.066164in}}%
\pgfpathcurveto{\pgfqpoint{3.212186in}{3.066164in}}{\pgfqpoint{3.199695in}{3.060990in}}{\pgfqpoint{3.190486in}{3.051781in}}%
\pgfpathcurveto{\pgfqpoint{3.181278in}{3.042573in}}{\pgfqpoint{3.176104in}{3.030082in}}{\pgfqpoint{3.176104in}{3.017059in}}%
\pgfpathcurveto{\pgfqpoint{3.176104in}{3.004037in}}{\pgfqpoint{3.181278in}{2.991545in}}{\pgfqpoint{3.190486in}{2.982337in}}%
\pgfpathcurveto{\pgfqpoint{3.199695in}{2.973129in}}{\pgfqpoint{3.212186in}{2.967955in}}{\pgfqpoint{3.225208in}{2.967955in}}%
\pgfpathlineto{\pgfqpoint{3.225208in}{2.967955in}}%
\pgfpathclose%
\pgfusepath{stroke,fill}%
\end{pgfscope}%
\begin{pgfscope}%
\pgfpathrectangle{\pgfqpoint{0.786164in}{0.768110in}}{\pgfqpoint{8.851069in}{7.081890in}}%
\pgfusepath{clip}%
\pgfsetbuttcap%
\pgfsetroundjoin%
\definecolor{currentfill}{rgb}{0.137770,0.537492,0.554906}%
\pgfsetfillcolor{currentfill}%
\pgfsetfillopacity{0.700000}%
\pgfsetlinewidth{0.501875pt}%
\definecolor{currentstroke}{rgb}{1.000000,1.000000,1.000000}%
\pgfsetstrokecolor{currentstroke}%
\pgfsetstrokeopacity{0.700000}%
\pgfsetdash{}{0pt}%
\pgfpathmoveto{\pgfqpoint{3.462674in}{2.792769in}}%
\pgfpathcurveto{\pgfqpoint{3.475696in}{2.792769in}}{\pgfqpoint{3.488188in}{2.797943in}}{\pgfqpoint{3.497396in}{2.807151in}}%
\pgfpathcurveto{\pgfqpoint{3.506604in}{2.816360in}}{\pgfqpoint{3.511778in}{2.828851in}}{\pgfqpoint{3.511778in}{2.841873in}}%
\pgfpathcurveto{\pgfqpoint{3.511778in}{2.854896in}}{\pgfqpoint{3.506604in}{2.867387in}}{\pgfqpoint{3.497396in}{2.876596in}}%
\pgfpathcurveto{\pgfqpoint{3.488188in}{2.885804in}}{\pgfqpoint{3.475696in}{2.890978in}}{\pgfqpoint{3.462674in}{2.890978in}}%
\pgfpathcurveto{\pgfqpoint{3.449651in}{2.890978in}}{\pgfqpoint{3.437160in}{2.885804in}}{\pgfqpoint{3.427952in}{2.876596in}}%
\pgfpathcurveto{\pgfqpoint{3.418743in}{2.867387in}}{\pgfqpoint{3.413569in}{2.854896in}}{\pgfqpoint{3.413569in}{2.841873in}}%
\pgfpathcurveto{\pgfqpoint{3.413569in}{2.828851in}}{\pgfqpoint{3.418743in}{2.816360in}}{\pgfqpoint{3.427952in}{2.807151in}}%
\pgfpathcurveto{\pgfqpoint{3.437160in}{2.797943in}}{\pgfqpoint{3.449651in}{2.792769in}}{\pgfqpoint{3.462674in}{2.792769in}}%
\pgfpathlineto{\pgfqpoint{3.462674in}{2.792769in}}%
\pgfpathclose%
\pgfusepath{stroke,fill}%
\end{pgfscope}%
\begin{pgfscope}%
\pgfpathrectangle{\pgfqpoint{0.786164in}{0.768110in}}{\pgfqpoint{8.851069in}{7.081890in}}%
\pgfusepath{clip}%
\pgfsetbuttcap%
\pgfsetroundjoin%
\definecolor{currentfill}{rgb}{0.135066,0.544853,0.554029}%
\pgfsetfillcolor{currentfill}%
\pgfsetfillopacity{0.700000}%
\pgfsetlinewidth{0.501875pt}%
\definecolor{currentstroke}{rgb}{1.000000,1.000000,1.000000}%
\pgfsetstrokecolor{currentstroke}%
\pgfsetstrokeopacity{0.700000}%
\pgfsetdash{}{0pt}%
\pgfpathmoveto{\pgfqpoint{3.353074in}{2.836565in}}%
\pgfpathcurveto{\pgfqpoint{3.366097in}{2.836565in}}{\pgfqpoint{3.378588in}{2.841739in}}{\pgfqpoint{3.387797in}{2.850948in}}%
\pgfpathcurveto{\pgfqpoint{3.397005in}{2.860156in}}{\pgfqpoint{3.402179in}{2.872647in}}{\pgfqpoint{3.402179in}{2.885670in}}%
\pgfpathcurveto{\pgfqpoint{3.402179in}{2.898693in}}{\pgfqpoint{3.397005in}{2.911184in}}{\pgfqpoint{3.387797in}{2.920392in}}%
\pgfpathcurveto{\pgfqpoint{3.378588in}{2.929601in}}{\pgfqpoint{3.366097in}{2.934774in}}{\pgfqpoint{3.353074in}{2.934774in}}%
\pgfpathcurveto{\pgfqpoint{3.340052in}{2.934774in}}{\pgfqpoint{3.327561in}{2.929601in}}{\pgfqpoint{3.318352in}{2.920392in}}%
\pgfpathcurveto{\pgfqpoint{3.309144in}{2.911184in}}{\pgfqpoint{3.303970in}{2.898693in}}{\pgfqpoint{3.303970in}{2.885670in}}%
\pgfpathcurveto{\pgfqpoint{3.303970in}{2.872647in}}{\pgfqpoint{3.309144in}{2.860156in}}{\pgfqpoint{3.318352in}{2.850948in}}%
\pgfpathcurveto{\pgfqpoint{3.327561in}{2.841739in}}{\pgfqpoint{3.340052in}{2.836565in}}{\pgfqpoint{3.353074in}{2.836565in}}%
\pgfpathlineto{\pgfqpoint{3.353074in}{2.836565in}}%
\pgfpathclose%
\pgfusepath{stroke,fill}%
\end{pgfscope}%
\begin{pgfscope}%
\pgfpathrectangle{\pgfqpoint{0.786164in}{0.768110in}}{\pgfqpoint{8.851069in}{7.081890in}}%
\pgfusepath{clip}%
\pgfsetbuttcap%
\pgfsetroundjoin%
\definecolor{currentfill}{rgb}{0.129933,0.559582,0.551864}%
\pgfsetfillcolor{currentfill}%
\pgfsetfillopacity{0.700000}%
\pgfsetlinewidth{0.501875pt}%
\definecolor{currentstroke}{rgb}{1.000000,1.000000,1.000000}%
\pgfsetstrokecolor{currentstroke}%
\pgfsetstrokeopacity{0.700000}%
\pgfsetdash{}{0pt}%
\pgfpathmoveto{\pgfqpoint{3.307408in}{3.011751in}}%
\pgfpathcurveto{\pgfqpoint{3.320431in}{3.011751in}}{\pgfqpoint{3.332922in}{3.016925in}}{\pgfqpoint{3.342130in}{3.026134in}}%
\pgfpathcurveto{\pgfqpoint{3.351339in}{3.035342in}}{\pgfqpoint{3.356513in}{3.047833in}}{\pgfqpoint{3.356513in}{3.060856in}}%
\pgfpathcurveto{\pgfqpoint{3.356513in}{3.073878in}}{\pgfqpoint{3.351339in}{3.086370in}}{\pgfqpoint{3.342130in}{3.095578in}}%
\pgfpathcurveto{\pgfqpoint{3.332922in}{3.104786in}}{\pgfqpoint{3.320431in}{3.109960in}}{\pgfqpoint{3.307408in}{3.109960in}}%
\pgfpathcurveto{\pgfqpoint{3.294385in}{3.109960in}}{\pgfqpoint{3.281894in}{3.104786in}}{\pgfqpoint{3.272686in}{3.095578in}}%
\pgfpathcurveto{\pgfqpoint{3.263477in}{3.086370in}}{\pgfqpoint{3.258303in}{3.073878in}}{\pgfqpoint{3.258303in}{3.060856in}}%
\pgfpathcurveto{\pgfqpoint{3.258303in}{3.047833in}}{\pgfqpoint{3.263477in}{3.035342in}}{\pgfqpoint{3.272686in}{3.026134in}}%
\pgfpathcurveto{\pgfqpoint{3.281894in}{3.016925in}}{\pgfqpoint{3.294385in}{3.011751in}}{\pgfqpoint{3.307408in}{3.011751in}}%
\pgfpathlineto{\pgfqpoint{3.307408in}{3.011751in}}%
\pgfpathclose%
\pgfusepath{stroke,fill}%
\end{pgfscope}%
\begin{pgfscope}%
\pgfpathrectangle{\pgfqpoint{0.786164in}{0.768110in}}{\pgfqpoint{8.851069in}{7.081890in}}%
\pgfusepath{clip}%
\pgfsetbuttcap%
\pgfsetroundjoin%
\definecolor{currentfill}{rgb}{0.129933,0.559582,0.551864}%
\pgfsetfillcolor{currentfill}%
\pgfsetfillopacity{0.700000}%
\pgfsetlinewidth{0.501875pt}%
\definecolor{currentstroke}{rgb}{1.000000,1.000000,1.000000}%
\pgfsetstrokecolor{currentstroke}%
\pgfsetstrokeopacity{0.700000}%
\pgfsetdash{}{0pt}%
\pgfpathmoveto{\pgfqpoint{3.325674in}{2.989853in}}%
\pgfpathcurveto{\pgfqpoint{3.338697in}{2.989853in}}{\pgfqpoint{3.351188in}{2.995027in}}{\pgfqpoint{3.360397in}{3.004235in}}%
\pgfpathcurveto{\pgfqpoint{3.369605in}{3.013444in}}{\pgfqpoint{3.374779in}{3.025935in}}{\pgfqpoint{3.374779in}{3.038958in}}%
\pgfpathcurveto{\pgfqpoint{3.374779in}{3.051980in}}{\pgfqpoint{3.369605in}{3.064471in}}{\pgfqpoint{3.360397in}{3.073680in}}%
\pgfpathcurveto{\pgfqpoint{3.351188in}{3.082888in}}{\pgfqpoint{3.338697in}{3.088062in}}{\pgfqpoint{3.325674in}{3.088062in}}%
\pgfpathcurveto{\pgfqpoint{3.312652in}{3.088062in}}{\pgfqpoint{3.300161in}{3.082888in}}{\pgfqpoint{3.290952in}{3.073680in}}%
\pgfpathcurveto{\pgfqpoint{3.281744in}{3.064471in}}{\pgfqpoint{3.276570in}{3.051980in}}{\pgfqpoint{3.276570in}{3.038958in}}%
\pgfpathcurveto{\pgfqpoint{3.276570in}{3.025935in}}{\pgfqpoint{3.281744in}{3.013444in}}{\pgfqpoint{3.290952in}{3.004235in}}%
\pgfpathcurveto{\pgfqpoint{3.300161in}{2.995027in}}{\pgfqpoint{3.312652in}{2.989853in}}{\pgfqpoint{3.325674in}{2.989853in}}%
\pgfpathlineto{\pgfqpoint{3.325674in}{2.989853in}}%
\pgfpathclose%
\pgfusepath{stroke,fill}%
\end{pgfscope}%
\begin{pgfscope}%
\pgfpathrectangle{\pgfqpoint{0.786164in}{0.768110in}}{\pgfqpoint{8.851069in}{7.081890in}}%
\pgfusepath{clip}%
\pgfsetbuttcap%
\pgfsetroundjoin%
\definecolor{currentfill}{rgb}{0.126453,0.570633,0.549841}%
\pgfsetfillcolor{currentfill}%
\pgfsetfillopacity{0.700000}%
\pgfsetlinewidth{0.501875pt}%
\definecolor{currentstroke}{rgb}{1.000000,1.000000,1.000000}%
\pgfsetstrokecolor{currentstroke}%
\pgfsetstrokeopacity{0.700000}%
\pgfsetdash{}{0pt}%
\pgfpathmoveto{\pgfqpoint{3.161275in}{2.858463in}}%
\pgfpathcurveto{\pgfqpoint{3.174298in}{2.858463in}}{\pgfqpoint{3.186789in}{2.863637in}}{\pgfqpoint{3.195998in}{2.872846in}}%
\pgfpathcurveto{\pgfqpoint{3.205206in}{2.882054in}}{\pgfqpoint{3.210380in}{2.894545in}}{\pgfqpoint{3.210380in}{2.907568in}}%
\pgfpathcurveto{\pgfqpoint{3.210380in}{2.920591in}}{\pgfqpoint{3.205206in}{2.933082in}}{\pgfqpoint{3.195998in}{2.942290in}}%
\pgfpathcurveto{\pgfqpoint{3.186789in}{2.951499in}}{\pgfqpoint{3.174298in}{2.956673in}}{\pgfqpoint{3.161275in}{2.956673in}}%
\pgfpathcurveto{\pgfqpoint{3.148253in}{2.956673in}}{\pgfqpoint{3.135762in}{2.951499in}}{\pgfqpoint{3.126553in}{2.942290in}}%
\pgfpathcurveto{\pgfqpoint{3.117345in}{2.933082in}}{\pgfqpoint{3.112171in}{2.920591in}}{\pgfqpoint{3.112171in}{2.907568in}}%
\pgfpathcurveto{\pgfqpoint{3.112171in}{2.894545in}}{\pgfqpoint{3.117345in}{2.882054in}}{\pgfqpoint{3.126553in}{2.872846in}}%
\pgfpathcurveto{\pgfqpoint{3.135762in}{2.863637in}}{\pgfqpoint{3.148253in}{2.858463in}}{\pgfqpoint{3.161275in}{2.858463in}}%
\pgfpathlineto{\pgfqpoint{3.161275in}{2.858463in}}%
\pgfpathclose%
\pgfusepath{stroke,fill}%
\end{pgfscope}%
\begin{pgfscope}%
\pgfpathrectangle{\pgfqpoint{0.786164in}{0.768110in}}{\pgfqpoint{8.851069in}{7.081890in}}%
\pgfusepath{clip}%
\pgfsetbuttcap%
\pgfsetroundjoin%
\definecolor{currentfill}{rgb}{0.126453,0.570633,0.549841}%
\pgfsetfillcolor{currentfill}%
\pgfsetfillopacity{0.700000}%
\pgfsetlinewidth{0.501875pt}%
\definecolor{currentstroke}{rgb}{1.000000,1.000000,1.000000}%
\pgfsetstrokecolor{currentstroke}%
\pgfsetstrokeopacity{0.700000}%
\pgfsetdash{}{0pt}%
\pgfpathmoveto{\pgfqpoint{3.216075in}{2.924158in}}%
\pgfpathcurveto{\pgfqpoint{3.229098in}{2.924158in}}{\pgfqpoint{3.241589in}{2.929332in}}{\pgfqpoint{3.250797in}{2.938541in}}%
\pgfpathcurveto{\pgfqpoint{3.260006in}{2.947749in}}{\pgfqpoint{3.265180in}{2.960240in}}{\pgfqpoint{3.265180in}{2.973263in}}%
\pgfpathcurveto{\pgfqpoint{3.265180in}{2.986286in}}{\pgfqpoint{3.260006in}{2.998777in}}{\pgfqpoint{3.250797in}{3.007985in}}%
\pgfpathcurveto{\pgfqpoint{3.241589in}{3.017193in}}{\pgfqpoint{3.229098in}{3.022367in}}{\pgfqpoint{3.216075in}{3.022367in}}%
\pgfpathcurveto{\pgfqpoint{3.203052in}{3.022367in}}{\pgfqpoint{3.190561in}{3.017193in}}{\pgfqpoint{3.181353in}{3.007985in}}%
\pgfpathcurveto{\pgfqpoint{3.172144in}{2.998777in}}{\pgfqpoint{3.166970in}{2.986286in}}{\pgfqpoint{3.166970in}{2.973263in}}%
\pgfpathcurveto{\pgfqpoint{3.166970in}{2.960240in}}{\pgfqpoint{3.172144in}{2.947749in}}{\pgfqpoint{3.181353in}{2.938541in}}%
\pgfpathcurveto{\pgfqpoint{3.190561in}{2.929332in}}{\pgfqpoint{3.203052in}{2.924158in}}{\pgfqpoint{3.216075in}{2.924158in}}%
\pgfpathlineto{\pgfqpoint{3.216075in}{2.924158in}}%
\pgfpathclose%
\pgfusepath{stroke,fill}%
\end{pgfscope}%
\begin{pgfscope}%
\pgfpathrectangle{\pgfqpoint{0.786164in}{0.768110in}}{\pgfqpoint{8.851069in}{7.081890in}}%
\pgfusepath{clip}%
\pgfsetbuttcap%
\pgfsetroundjoin%
\definecolor{currentfill}{rgb}{0.126453,0.570633,0.549841}%
\pgfsetfillcolor{currentfill}%
\pgfsetfillopacity{0.700000}%
\pgfsetlinewidth{0.501875pt}%
\definecolor{currentstroke}{rgb}{1.000000,1.000000,1.000000}%
\pgfsetstrokecolor{currentstroke}%
\pgfsetstrokeopacity{0.700000}%
\pgfsetdash{}{0pt}%
\pgfpathmoveto{\pgfqpoint{3.225208in}{2.902260in}}%
\pgfpathcurveto{\pgfqpoint{3.238231in}{2.902260in}}{\pgfqpoint{3.250722in}{2.907434in}}{\pgfqpoint{3.259931in}{2.916642in}}%
\pgfpathcurveto{\pgfqpoint{3.269139in}{2.925851in}}{\pgfqpoint{3.274313in}{2.938342in}}{\pgfqpoint{3.274313in}{2.951365in}}%
\pgfpathcurveto{\pgfqpoint{3.274313in}{2.964387in}}{\pgfqpoint{3.269139in}{2.976878in}}{\pgfqpoint{3.259931in}{2.986087in}}%
\pgfpathcurveto{\pgfqpoint{3.250722in}{2.995295in}}{\pgfqpoint{3.238231in}{3.000469in}}{\pgfqpoint{3.225208in}{3.000469in}}%
\pgfpathcurveto{\pgfqpoint{3.212186in}{3.000469in}}{\pgfqpoint{3.199695in}{2.995295in}}{\pgfqpoint{3.190486in}{2.986087in}}%
\pgfpathcurveto{\pgfqpoint{3.181278in}{2.976878in}}{\pgfqpoint{3.176104in}{2.964387in}}{\pgfqpoint{3.176104in}{2.951365in}}%
\pgfpathcurveto{\pgfqpoint{3.176104in}{2.938342in}}{\pgfqpoint{3.181278in}{2.925851in}}{\pgfqpoint{3.190486in}{2.916642in}}%
\pgfpathcurveto{\pgfqpoint{3.199695in}{2.907434in}}{\pgfqpoint{3.212186in}{2.902260in}}{\pgfqpoint{3.225208in}{2.902260in}}%
\pgfpathlineto{\pgfqpoint{3.225208in}{2.902260in}}%
\pgfpathclose%
\pgfusepath{stroke,fill}%
\end{pgfscope}%
\begin{pgfscope}%
\pgfpathrectangle{\pgfqpoint{0.786164in}{0.768110in}}{\pgfqpoint{8.851069in}{7.081890in}}%
\pgfusepath{clip}%
\pgfsetbuttcap%
\pgfsetroundjoin%
\definecolor{currentfill}{rgb}{0.127568,0.566949,0.550556}%
\pgfsetfillcolor{currentfill}%
\pgfsetfillopacity{0.700000}%
\pgfsetlinewidth{0.501875pt}%
\definecolor{currentstroke}{rgb}{1.000000,1.000000,1.000000}%
\pgfsetstrokecolor{currentstroke}%
\pgfsetstrokeopacity{0.700000}%
\pgfsetdash{}{0pt}%
\pgfpathmoveto{\pgfqpoint{3.280008in}{3.055548in}}%
\pgfpathcurveto{\pgfqpoint{3.293031in}{3.055548in}}{\pgfqpoint{3.305522in}{3.060722in}}{\pgfqpoint{3.314730in}{3.069930in}}%
\pgfpathcurveto{\pgfqpoint{3.323939in}{3.079138in}}{\pgfqpoint{3.329113in}{3.091630in}}{\pgfqpoint{3.329113in}{3.104652in}}%
\pgfpathcurveto{\pgfqpoint{3.329113in}{3.117675in}}{\pgfqpoint{3.323939in}{3.130166in}}{\pgfqpoint{3.314730in}{3.139374in}}%
\pgfpathcurveto{\pgfqpoint{3.305522in}{3.148583in}}{\pgfqpoint{3.293031in}{3.153757in}}{\pgfqpoint{3.280008in}{3.153757in}}%
\pgfpathcurveto{\pgfqpoint{3.266985in}{3.153757in}}{\pgfqpoint{3.254494in}{3.148583in}}{\pgfqpoint{3.245286in}{3.139374in}}%
\pgfpathcurveto{\pgfqpoint{3.236077in}{3.130166in}}{\pgfqpoint{3.230903in}{3.117675in}}{\pgfqpoint{3.230903in}{3.104652in}}%
\pgfpathcurveto{\pgfqpoint{3.230903in}{3.091630in}}{\pgfqpoint{3.236077in}{3.079138in}}{\pgfqpoint{3.245286in}{3.069930in}}%
\pgfpathcurveto{\pgfqpoint{3.254494in}{3.060722in}}{\pgfqpoint{3.266985in}{3.055548in}}{\pgfqpoint{3.280008in}{3.055548in}}%
\pgfpathlineto{\pgfqpoint{3.280008in}{3.055548in}}%
\pgfpathclose%
\pgfusepath{stroke,fill}%
\end{pgfscope}%
\begin{pgfscope}%
\pgfpathrectangle{\pgfqpoint{0.786164in}{0.768110in}}{\pgfqpoint{8.851069in}{7.081890in}}%
\pgfusepath{clip}%
\pgfsetbuttcap%
\pgfsetroundjoin%
\definecolor{currentfill}{rgb}{0.125394,0.574318,0.549086}%
\pgfsetfillcolor{currentfill}%
\pgfsetfillopacity{0.700000}%
\pgfsetlinewidth{0.501875pt}%
\definecolor{currentstroke}{rgb}{1.000000,1.000000,1.000000}%
\pgfsetstrokecolor{currentstroke}%
\pgfsetstrokeopacity{0.700000}%
\pgfsetdash{}{0pt}%
\pgfpathmoveto{\pgfqpoint{3.161275in}{3.165039in}}%
\pgfpathcurveto{\pgfqpoint{3.174298in}{3.165039in}}{\pgfqpoint{3.186789in}{3.170213in}}{\pgfqpoint{3.195998in}{3.179421in}}%
\pgfpathcurveto{\pgfqpoint{3.205206in}{3.188630in}}{\pgfqpoint{3.210380in}{3.201121in}}{\pgfqpoint{3.210380in}{3.214143in}}%
\pgfpathcurveto{\pgfqpoint{3.210380in}{3.227166in}}{\pgfqpoint{3.205206in}{3.239657in}}{\pgfqpoint{3.195998in}{3.248866in}}%
\pgfpathcurveto{\pgfqpoint{3.186789in}{3.258074in}}{\pgfqpoint{3.174298in}{3.263248in}}{\pgfqpoint{3.161275in}{3.263248in}}%
\pgfpathcurveto{\pgfqpoint{3.148253in}{3.263248in}}{\pgfqpoint{3.135762in}{3.258074in}}{\pgfqpoint{3.126553in}{3.248866in}}%
\pgfpathcurveto{\pgfqpoint{3.117345in}{3.239657in}}{\pgfqpoint{3.112171in}{3.227166in}}{\pgfqpoint{3.112171in}{3.214143in}}%
\pgfpathcurveto{\pgfqpoint{3.112171in}{3.201121in}}{\pgfqpoint{3.117345in}{3.188630in}}{\pgfqpoint{3.126553in}{3.179421in}}%
\pgfpathcurveto{\pgfqpoint{3.135762in}{3.170213in}}{\pgfqpoint{3.148253in}{3.165039in}}{\pgfqpoint{3.161275in}{3.165039in}}%
\pgfpathlineto{\pgfqpoint{3.161275in}{3.165039in}}%
\pgfpathclose%
\pgfusepath{stroke,fill}%
\end{pgfscope}%
\begin{pgfscope}%
\pgfpathrectangle{\pgfqpoint{0.786164in}{0.768110in}}{\pgfqpoint{8.851069in}{7.081890in}}%
\pgfusepath{clip}%
\pgfsetbuttcap%
\pgfsetroundjoin%
\definecolor{currentfill}{rgb}{0.125394,0.574318,0.549086}%
\pgfsetfillcolor{currentfill}%
\pgfsetfillopacity{0.700000}%
\pgfsetlinewidth{0.501875pt}%
\definecolor{currentstroke}{rgb}{1.000000,1.000000,1.000000}%
\pgfsetstrokecolor{currentstroke}%
\pgfsetstrokeopacity{0.700000}%
\pgfsetdash{}{0pt}%
\pgfpathmoveto{\pgfqpoint{3.188675in}{3.121242in}}%
\pgfpathcurveto{\pgfqpoint{3.201698in}{3.121242in}}{\pgfqpoint{3.214189in}{3.126416in}}{\pgfqpoint{3.223397in}{3.135625in}}%
\pgfpathcurveto{\pgfqpoint{3.232606in}{3.144833in}}{\pgfqpoint{3.237780in}{3.157324in}}{\pgfqpoint{3.237780in}{3.170347in}}%
\pgfpathcurveto{\pgfqpoint{3.237780in}{3.183370in}}{\pgfqpoint{3.232606in}{3.195861in}}{\pgfqpoint{3.223397in}{3.205069in}}%
\pgfpathcurveto{\pgfqpoint{3.214189in}{3.214278in}}{\pgfqpoint{3.201698in}{3.219452in}}{\pgfqpoint{3.188675in}{3.219452in}}%
\pgfpathcurveto{\pgfqpoint{3.175652in}{3.219452in}}{\pgfqpoint{3.163161in}{3.214278in}}{\pgfqpoint{3.153953in}{3.205069in}}%
\pgfpathcurveto{\pgfqpoint{3.144744in}{3.195861in}}{\pgfqpoint{3.139571in}{3.183370in}}{\pgfqpoint{3.139571in}{3.170347in}}%
\pgfpathcurveto{\pgfqpoint{3.139571in}{3.157324in}}{\pgfqpoint{3.144744in}{3.144833in}}{\pgfqpoint{3.153953in}{3.135625in}}%
\pgfpathcurveto{\pgfqpoint{3.163161in}{3.126416in}}{\pgfqpoint{3.175652in}{3.121242in}}{\pgfqpoint{3.188675in}{3.121242in}}%
\pgfpathlineto{\pgfqpoint{3.188675in}{3.121242in}}%
\pgfpathclose%
\pgfusepath{stroke,fill}%
\end{pgfscope}%
\begin{pgfscope}%
\pgfpathrectangle{\pgfqpoint{0.786164in}{0.768110in}}{\pgfqpoint{8.851069in}{7.081890in}}%
\pgfusepath{clip}%
\pgfsetbuttcap%
\pgfsetroundjoin%
\definecolor{currentfill}{rgb}{0.119483,0.614817,0.537692}%
\pgfsetfillcolor{currentfill}%
\pgfsetfillopacity{0.700000}%
\pgfsetlinewidth{0.501875pt}%
\definecolor{currentstroke}{rgb}{1.000000,1.000000,1.000000}%
\pgfsetstrokecolor{currentstroke}%
\pgfsetstrokeopacity{0.700000}%
\pgfsetdash{}{0pt}%
\pgfpathmoveto{\pgfqpoint{2.932943in}{2.727074in}}%
\pgfpathcurveto{\pgfqpoint{2.945966in}{2.727074in}}{\pgfqpoint{2.958457in}{2.732248in}}{\pgfqpoint{2.967665in}{2.741456in}}%
\pgfpathcurveto{\pgfqpoint{2.976874in}{2.750665in}}{\pgfqpoint{2.982048in}{2.763156in}}{\pgfqpoint{2.982048in}{2.776179in}}%
\pgfpathcurveto{\pgfqpoint{2.982048in}{2.789201in}}{\pgfqpoint{2.976874in}{2.801692in}}{\pgfqpoint{2.967665in}{2.810901in}}%
\pgfpathcurveto{\pgfqpoint{2.958457in}{2.820109in}}{\pgfqpoint{2.945966in}{2.825283in}}{\pgfqpoint{2.932943in}{2.825283in}}%
\pgfpathcurveto{\pgfqpoint{2.919920in}{2.825283in}}{\pgfqpoint{2.907429in}{2.820109in}}{\pgfqpoint{2.898221in}{2.810901in}}%
\pgfpathcurveto{\pgfqpoint{2.889012in}{2.801692in}}{\pgfqpoint{2.883838in}{2.789201in}}{\pgfqpoint{2.883838in}{2.776179in}}%
\pgfpathcurveto{\pgfqpoint{2.883838in}{2.763156in}}{\pgfqpoint{2.889012in}{2.750665in}}{\pgfqpoint{2.898221in}{2.741456in}}%
\pgfpathcurveto{\pgfqpoint{2.907429in}{2.732248in}}{\pgfqpoint{2.919920in}{2.727074in}}{\pgfqpoint{2.932943in}{2.727074in}}%
\pgfpathlineto{\pgfqpoint{2.932943in}{2.727074in}}%
\pgfpathclose%
\pgfusepath{stroke,fill}%
\end{pgfscope}%
\begin{pgfscope}%
\pgfpathrectangle{\pgfqpoint{0.786164in}{0.768110in}}{\pgfqpoint{8.851069in}{7.081890in}}%
\pgfusepath{clip}%
\pgfsetbuttcap%
\pgfsetroundjoin%
\definecolor{currentfill}{rgb}{0.122606,0.585371,0.546557}%
\pgfsetfillcolor{currentfill}%
\pgfsetfillopacity{0.700000}%
\pgfsetlinewidth{0.501875pt}%
\definecolor{currentstroke}{rgb}{1.000000,1.000000,1.000000}%
\pgfsetstrokecolor{currentstroke}%
\pgfsetstrokeopacity{0.700000}%
\pgfsetdash{}{0pt}%
\pgfpathmoveto{\pgfqpoint{3.097342in}{2.661379in}}%
\pgfpathcurveto{\pgfqpoint{3.110365in}{2.661379in}}{\pgfqpoint{3.122856in}{2.666553in}}{\pgfqpoint{3.132065in}{2.675762in}}%
\pgfpathcurveto{\pgfqpoint{3.141273in}{2.684970in}}{\pgfqpoint{3.146447in}{2.697461in}}{\pgfqpoint{3.146447in}{2.710484in}}%
\pgfpathcurveto{\pgfqpoint{3.146447in}{2.723507in}}{\pgfqpoint{3.141273in}{2.735998in}}{\pgfqpoint{3.132065in}{2.745206in}}%
\pgfpathcurveto{\pgfqpoint{3.122856in}{2.754415in}}{\pgfqpoint{3.110365in}{2.759589in}}{\pgfqpoint{3.097342in}{2.759589in}}%
\pgfpathcurveto{\pgfqpoint{3.084320in}{2.759589in}}{\pgfqpoint{3.071829in}{2.754415in}}{\pgfqpoint{3.062620in}{2.745206in}}%
\pgfpathcurveto{\pgfqpoint{3.053412in}{2.735998in}}{\pgfqpoint{3.048238in}{2.723507in}}{\pgfqpoint{3.048238in}{2.710484in}}%
\pgfpathcurveto{\pgfqpoint{3.048238in}{2.697461in}}{\pgfqpoint{3.053412in}{2.684970in}}{\pgfqpoint{3.062620in}{2.675762in}}%
\pgfpathcurveto{\pgfqpoint{3.071829in}{2.666553in}}{\pgfqpoint{3.084320in}{2.661379in}}{\pgfqpoint{3.097342in}{2.661379in}}%
\pgfpathlineto{\pgfqpoint{3.097342in}{2.661379in}}%
\pgfpathclose%
\pgfusepath{stroke,fill}%
\end{pgfscope}%
\begin{pgfscope}%
\pgfpathrectangle{\pgfqpoint{0.786164in}{0.768110in}}{\pgfqpoint{8.851069in}{7.081890in}}%
\pgfusepath{clip}%
\pgfsetbuttcap%
\pgfsetroundjoin%
\definecolor{currentfill}{rgb}{0.125394,0.574318,0.549086}%
\pgfsetfillcolor{currentfill}%
\pgfsetfillopacity{0.700000}%
\pgfsetlinewidth{0.501875pt}%
\definecolor{currentstroke}{rgb}{1.000000,1.000000,1.000000}%
\pgfsetstrokecolor{currentstroke}%
\pgfsetstrokeopacity{0.700000}%
\pgfsetdash{}{0pt}%
\pgfpathmoveto{\pgfqpoint{2.923810in}{2.727074in}}%
\pgfpathcurveto{\pgfqpoint{2.936833in}{2.727074in}}{\pgfqpoint{2.949324in}{2.732248in}}{\pgfqpoint{2.958532in}{2.741456in}}%
\pgfpathcurveto{\pgfqpoint{2.967740in}{2.750665in}}{\pgfqpoint{2.972914in}{2.763156in}}{\pgfqpoint{2.972914in}{2.776179in}}%
\pgfpathcurveto{\pgfqpoint{2.972914in}{2.789201in}}{\pgfqpoint{2.967740in}{2.801692in}}{\pgfqpoint{2.958532in}{2.810901in}}%
\pgfpathcurveto{\pgfqpoint{2.949324in}{2.820109in}}{\pgfqpoint{2.936833in}{2.825283in}}{\pgfqpoint{2.923810in}{2.825283in}}%
\pgfpathcurveto{\pgfqpoint{2.910787in}{2.825283in}}{\pgfqpoint{2.898296in}{2.820109in}}{\pgfqpoint{2.889088in}{2.810901in}}%
\pgfpathcurveto{\pgfqpoint{2.879879in}{2.801692in}}{\pgfqpoint{2.874705in}{2.789201in}}{\pgfqpoint{2.874705in}{2.776179in}}%
\pgfpathcurveto{\pgfqpoint{2.874705in}{2.763156in}}{\pgfqpoint{2.879879in}{2.750665in}}{\pgfqpoint{2.889088in}{2.741456in}}%
\pgfpathcurveto{\pgfqpoint{2.898296in}{2.732248in}}{\pgfqpoint{2.910787in}{2.727074in}}{\pgfqpoint{2.923810in}{2.727074in}}%
\pgfpathlineto{\pgfqpoint{2.923810in}{2.727074in}}%
\pgfpathclose%
\pgfusepath{stroke,fill}%
\end{pgfscope}%
\begin{pgfscope}%
\pgfpathrectangle{\pgfqpoint{0.786164in}{0.768110in}}{\pgfqpoint{8.851069in}{7.081890in}}%
\pgfusepath{clip}%
\pgfsetbuttcap%
\pgfsetroundjoin%
\definecolor{currentfill}{rgb}{0.276022,0.044167,0.370164}%
\pgfsetfillcolor{currentfill}%
\pgfsetfillopacity{0.700000}%
\pgfsetlinewidth{0.501875pt}%
\definecolor{currentstroke}{rgb}{1.000000,1.000000,1.000000}%
\pgfsetstrokecolor{currentstroke}%
\pgfsetstrokeopacity{0.700000}%
\pgfsetdash{}{0pt}%
\pgfpathmoveto{\pgfqpoint{4.467335in}{4.194256in}}%
\pgfpathcurveto{\pgfqpoint{4.480358in}{4.194256in}}{\pgfqpoint{4.492849in}{4.199430in}}{\pgfqpoint{4.502058in}{4.208638in}}%
\pgfpathcurveto{\pgfqpoint{4.511266in}{4.217847in}}{\pgfqpoint{4.516440in}{4.230338in}}{\pgfqpoint{4.516440in}{4.243360in}}%
\pgfpathcurveto{\pgfqpoint{4.516440in}{4.256383in}}{\pgfqpoint{4.511266in}{4.268874in}}{\pgfqpoint{4.502058in}{4.278083in}}%
\pgfpathcurveto{\pgfqpoint{4.492849in}{4.287291in}}{\pgfqpoint{4.480358in}{4.292465in}}{\pgfqpoint{4.467335in}{4.292465in}}%
\pgfpathcurveto{\pgfqpoint{4.454313in}{4.292465in}}{\pgfqpoint{4.441822in}{4.287291in}}{\pgfqpoint{4.432613in}{4.278083in}}%
\pgfpathcurveto{\pgfqpoint{4.423405in}{4.268874in}}{\pgfqpoint{4.418231in}{4.256383in}}{\pgfqpoint{4.418231in}{4.243360in}}%
\pgfpathcurveto{\pgfqpoint{4.418231in}{4.230338in}}{\pgfqpoint{4.423405in}{4.217847in}}{\pgfqpoint{4.432613in}{4.208638in}}%
\pgfpathcurveto{\pgfqpoint{4.441822in}{4.199430in}}{\pgfqpoint{4.454313in}{4.194256in}}{\pgfqpoint{4.467335in}{4.194256in}}%
\pgfpathlineto{\pgfqpoint{4.467335in}{4.194256in}}%
\pgfpathclose%
\pgfusepath{stroke,fill}%
\end{pgfscope}%
\begin{pgfscope}%
\pgfpathrectangle{\pgfqpoint{0.786164in}{0.768110in}}{\pgfqpoint{8.851069in}{7.081890in}}%
\pgfusepath{clip}%
\pgfsetbuttcap%
\pgfsetroundjoin%
\definecolor{currentfill}{rgb}{0.277018,0.050344,0.375715}%
\pgfsetfillcolor{currentfill}%
\pgfsetfillopacity{0.700000}%
\pgfsetlinewidth{0.501875pt}%
\definecolor{currentstroke}{rgb}{1.000000,1.000000,1.000000}%
\pgfsetstrokecolor{currentstroke}%
\pgfsetstrokeopacity{0.700000}%
\pgfsetdash{}{0pt}%
\pgfpathmoveto{\pgfqpoint{4.376003in}{4.106663in}}%
\pgfpathcurveto{\pgfqpoint{4.389025in}{4.106663in}}{\pgfqpoint{4.401516in}{4.111837in}}{\pgfqpoint{4.410725in}{4.121045in}}%
\pgfpathcurveto{\pgfqpoint{4.419933in}{4.130254in}}{\pgfqpoint{4.425107in}{4.142745in}}{\pgfqpoint{4.425107in}{4.155768in}}%
\pgfpathcurveto{\pgfqpoint{4.425107in}{4.168790in}}{\pgfqpoint{4.419933in}{4.181281in}}{\pgfqpoint{4.410725in}{4.190490in}}%
\pgfpathcurveto{\pgfqpoint{4.401516in}{4.199698in}}{\pgfqpoint{4.389025in}{4.204872in}}{\pgfqpoint{4.376003in}{4.204872in}}%
\pgfpathcurveto{\pgfqpoint{4.362980in}{4.204872in}}{\pgfqpoint{4.350489in}{4.199698in}}{\pgfqpoint{4.341280in}{4.190490in}}%
\pgfpathcurveto{\pgfqpoint{4.332072in}{4.181281in}}{\pgfqpoint{4.326898in}{4.168790in}}{\pgfqpoint{4.326898in}{4.155768in}}%
\pgfpathcurveto{\pgfqpoint{4.326898in}{4.142745in}}{\pgfqpoint{4.332072in}{4.130254in}}{\pgfqpoint{4.341280in}{4.121045in}}%
\pgfpathcurveto{\pgfqpoint{4.350489in}{4.111837in}}{\pgfqpoint{4.362980in}{4.106663in}}{\pgfqpoint{4.376003in}{4.106663in}}%
\pgfpathlineto{\pgfqpoint{4.376003in}{4.106663in}}%
\pgfpathclose%
\pgfusepath{stroke,fill}%
\end{pgfscope}%
\begin{pgfscope}%
\pgfpathrectangle{\pgfqpoint{0.786164in}{0.768110in}}{\pgfqpoint{8.851069in}{7.081890in}}%
\pgfusepath{clip}%
\pgfsetbuttcap%
\pgfsetroundjoin%
\definecolor{currentfill}{rgb}{0.277941,0.056324,0.381191}%
\pgfsetfillcolor{currentfill}%
\pgfsetfillopacity{0.700000}%
\pgfsetlinewidth{0.501875pt}%
\definecolor{currentstroke}{rgb}{1.000000,1.000000,1.000000}%
\pgfsetstrokecolor{currentstroke}%
\pgfsetstrokeopacity{0.700000}%
\pgfsetdash{}{0pt}%
\pgfpathmoveto{\pgfqpoint{4.330336in}{4.019070in}}%
\pgfpathcurveto{\pgfqpoint{4.343359in}{4.019070in}}{\pgfqpoint{4.355850in}{4.024244in}}{\pgfqpoint{4.365058in}{4.033452in}}%
\pgfpathcurveto{\pgfqpoint{4.374267in}{4.042661in}}{\pgfqpoint{4.379441in}{4.055152in}}{\pgfqpoint{4.379441in}{4.068175in}}%
\pgfpathcurveto{\pgfqpoint{4.379441in}{4.081197in}}{\pgfqpoint{4.374267in}{4.093688in}}{\pgfqpoint{4.365058in}{4.102897in}}%
\pgfpathcurveto{\pgfqpoint{4.355850in}{4.112105in}}{\pgfqpoint{4.343359in}{4.117279in}}{\pgfqpoint{4.330336in}{4.117279in}}%
\pgfpathcurveto{\pgfqpoint{4.317313in}{4.117279in}}{\pgfqpoint{4.304822in}{4.112105in}}{\pgfqpoint{4.295614in}{4.102897in}}%
\pgfpathcurveto{\pgfqpoint{4.286405in}{4.093688in}}{\pgfqpoint{4.281231in}{4.081197in}}{\pgfqpoint{4.281231in}{4.068175in}}%
\pgfpathcurveto{\pgfqpoint{4.281231in}{4.055152in}}{\pgfqpoint{4.286405in}{4.042661in}}{\pgfqpoint{4.295614in}{4.033452in}}%
\pgfpathcurveto{\pgfqpoint{4.304822in}{4.024244in}}{\pgfqpoint{4.317313in}{4.019070in}}{\pgfqpoint{4.330336in}{4.019070in}}%
\pgfpathlineto{\pgfqpoint{4.330336in}{4.019070in}}%
\pgfpathclose%
\pgfusepath{stroke,fill}%
\end{pgfscope}%
\begin{pgfscope}%
\pgfpathrectangle{\pgfqpoint{0.786164in}{0.768110in}}{\pgfqpoint{8.851069in}{7.081890in}}%
\pgfusepath{clip}%
\pgfsetbuttcap%
\pgfsetroundjoin%
\definecolor{currentfill}{rgb}{0.279566,0.067836,0.391917}%
\pgfsetfillcolor{currentfill}%
\pgfsetfillopacity{0.700000}%
\pgfsetlinewidth{0.501875pt}%
\definecolor{currentstroke}{rgb}{1.000000,1.000000,1.000000}%
\pgfsetstrokecolor{currentstroke}%
\pgfsetstrokeopacity{0.700000}%
\pgfsetdash{}{0pt}%
\pgfpathmoveto{\pgfqpoint{4.202470in}{3.931477in}}%
\pgfpathcurveto{\pgfqpoint{4.215493in}{3.931477in}}{\pgfqpoint{4.227984in}{3.936651in}}{\pgfqpoint{4.237192in}{3.945859in}}%
\pgfpathcurveto{\pgfqpoint{4.246401in}{3.955068in}}{\pgfqpoint{4.251575in}{3.967559in}}{\pgfqpoint{4.251575in}{3.980582in}}%
\pgfpathcurveto{\pgfqpoint{4.251575in}{3.993604in}}{\pgfqpoint{4.246401in}{4.006095in}}{\pgfqpoint{4.237192in}{4.015304in}}%
\pgfpathcurveto{\pgfqpoint{4.227984in}{4.024512in}}{\pgfqpoint{4.215493in}{4.029686in}}{\pgfqpoint{4.202470in}{4.029686in}}%
\pgfpathcurveto{\pgfqpoint{4.189447in}{4.029686in}}{\pgfqpoint{4.176956in}{4.024512in}}{\pgfqpoint{4.167748in}{4.015304in}}%
\pgfpathcurveto{\pgfqpoint{4.158539in}{4.006095in}}{\pgfqpoint{4.153365in}{3.993604in}}{\pgfqpoint{4.153365in}{3.980582in}}%
\pgfpathcurveto{\pgfqpoint{4.153365in}{3.967559in}}{\pgfqpoint{4.158539in}{3.955068in}}{\pgfqpoint{4.167748in}{3.945859in}}%
\pgfpathcurveto{\pgfqpoint{4.176956in}{3.936651in}}{\pgfqpoint{4.189447in}{3.931477in}}{\pgfqpoint{4.202470in}{3.931477in}}%
\pgfpathlineto{\pgfqpoint{4.202470in}{3.931477in}}%
\pgfpathclose%
\pgfusepath{stroke,fill}%
\end{pgfscope}%
\begin{pgfscope}%
\pgfpathrectangle{\pgfqpoint{0.786164in}{0.768110in}}{\pgfqpoint{8.851069in}{7.081890in}}%
\pgfusepath{clip}%
\pgfsetbuttcap%
\pgfsetroundjoin%
\definecolor{currentfill}{rgb}{0.280894,0.078907,0.402329}%
\pgfsetfillcolor{currentfill}%
\pgfsetfillopacity{0.700000}%
\pgfsetlinewidth{0.501875pt}%
\definecolor{currentstroke}{rgb}{1.000000,1.000000,1.000000}%
\pgfsetstrokecolor{currentstroke}%
\pgfsetstrokeopacity{0.700000}%
\pgfsetdash{}{0pt}%
\pgfpathmoveto{\pgfqpoint{4.239003in}{3.931477in}}%
\pgfpathcurveto{\pgfqpoint{4.252026in}{3.931477in}}{\pgfqpoint{4.264517in}{3.936651in}}{\pgfqpoint{4.273725in}{3.945859in}}%
\pgfpathcurveto{\pgfqpoint{4.282934in}{3.955068in}}{\pgfqpoint{4.288108in}{3.967559in}}{\pgfqpoint{4.288108in}{3.980582in}}%
\pgfpathcurveto{\pgfqpoint{4.288108in}{3.993604in}}{\pgfqpoint{4.282934in}{4.006095in}}{\pgfqpoint{4.273725in}{4.015304in}}%
\pgfpathcurveto{\pgfqpoint{4.264517in}{4.024512in}}{\pgfqpoint{4.252026in}{4.029686in}}{\pgfqpoint{4.239003in}{4.029686in}}%
\pgfpathcurveto{\pgfqpoint{4.225981in}{4.029686in}}{\pgfqpoint{4.213489in}{4.024512in}}{\pgfqpoint{4.204281in}{4.015304in}}%
\pgfpathcurveto{\pgfqpoint{4.195073in}{4.006095in}}{\pgfqpoint{4.189899in}{3.993604in}}{\pgfqpoint{4.189899in}{3.980582in}}%
\pgfpathcurveto{\pgfqpoint{4.189899in}{3.967559in}}{\pgfqpoint{4.195073in}{3.955068in}}{\pgfqpoint{4.204281in}{3.945859in}}%
\pgfpathcurveto{\pgfqpoint{4.213489in}{3.936651in}}{\pgfqpoint{4.225981in}{3.931477in}}{\pgfqpoint{4.239003in}{3.931477in}}%
\pgfpathlineto{\pgfqpoint{4.239003in}{3.931477in}}%
\pgfpathclose%
\pgfusepath{stroke,fill}%
\end{pgfscope}%
\begin{pgfscope}%
\pgfpathrectangle{\pgfqpoint{0.786164in}{0.768110in}}{\pgfqpoint{8.851069in}{7.081890in}}%
\pgfusepath{clip}%
\pgfsetbuttcap%
\pgfsetroundjoin%
\definecolor{currentfill}{rgb}{0.282910,0.105393,0.426902}%
\pgfsetfillcolor{currentfill}%
\pgfsetfillopacity{0.700000}%
\pgfsetlinewidth{0.501875pt}%
\definecolor{currentstroke}{rgb}{1.000000,1.000000,1.000000}%
\pgfsetstrokecolor{currentstroke}%
\pgfsetstrokeopacity{0.700000}%
\pgfsetdash{}{0pt}%
\pgfpathmoveto{\pgfqpoint{4.111137in}{3.646800in}}%
\pgfpathcurveto{\pgfqpoint{4.124160in}{3.646800in}}{\pgfqpoint{4.136651in}{3.651974in}}{\pgfqpoint{4.145859in}{3.661182in}}%
\pgfpathcurveto{\pgfqpoint{4.155068in}{3.670391in}}{\pgfqpoint{4.160242in}{3.682882in}}{\pgfqpoint{4.160242in}{3.695905in}}%
\pgfpathcurveto{\pgfqpoint{4.160242in}{3.708927in}}{\pgfqpoint{4.155068in}{3.721418in}}{\pgfqpoint{4.145859in}{3.730627in}}%
\pgfpathcurveto{\pgfqpoint{4.136651in}{3.739835in}}{\pgfqpoint{4.124160in}{3.745009in}}{\pgfqpoint{4.111137in}{3.745009in}}%
\pgfpathcurveto{\pgfqpoint{4.098114in}{3.745009in}}{\pgfqpoint{4.085623in}{3.739835in}}{\pgfqpoint{4.076415in}{3.730627in}}%
\pgfpathcurveto{\pgfqpoint{4.067207in}{3.721418in}}{\pgfqpoint{4.062033in}{3.708927in}}{\pgfqpoint{4.062033in}{3.695905in}}%
\pgfpathcurveto{\pgfqpoint{4.062033in}{3.682882in}}{\pgfqpoint{4.067207in}{3.670391in}}{\pgfqpoint{4.076415in}{3.661182in}}%
\pgfpathcurveto{\pgfqpoint{4.085623in}{3.651974in}}{\pgfqpoint{4.098114in}{3.646800in}}{\pgfqpoint{4.111137in}{3.646800in}}%
\pgfpathlineto{\pgfqpoint{4.111137in}{3.646800in}}%
\pgfpathclose%
\pgfusepath{stroke,fill}%
\end{pgfscope}%
\begin{pgfscope}%
\pgfpathrectangle{\pgfqpoint{0.786164in}{0.768110in}}{\pgfqpoint{8.851069in}{7.081890in}}%
\pgfusepath{clip}%
\pgfsetbuttcap%
\pgfsetroundjoin%
\definecolor{currentfill}{rgb}{0.283187,0.125848,0.444960}%
\pgfsetfillcolor{currentfill}%
\pgfsetfillopacity{0.700000}%
\pgfsetlinewidth{0.501875pt}%
\definecolor{currentstroke}{rgb}{1.000000,1.000000,1.000000}%
\pgfsetstrokecolor{currentstroke}%
\pgfsetstrokeopacity{0.700000}%
\pgfsetdash{}{0pt}%
\pgfpathmoveto{\pgfqpoint{4.348603in}{3.778189in}}%
\pgfpathcurveto{\pgfqpoint{4.361625in}{3.778189in}}{\pgfqpoint{4.374116in}{3.783363in}}{\pgfqpoint{4.383325in}{3.792572in}}%
\pgfpathcurveto{\pgfqpoint{4.392533in}{3.801780in}}{\pgfqpoint{4.397707in}{3.814271in}}{\pgfqpoint{4.397707in}{3.827294in}}%
\pgfpathcurveto{\pgfqpoint{4.397707in}{3.840317in}}{\pgfqpoint{4.392533in}{3.852808in}}{\pgfqpoint{4.383325in}{3.862016in}}%
\pgfpathcurveto{\pgfqpoint{4.374116in}{3.871225in}}{\pgfqpoint{4.361625in}{3.876399in}}{\pgfqpoint{4.348603in}{3.876399in}}%
\pgfpathcurveto{\pgfqpoint{4.335580in}{3.876399in}}{\pgfqpoint{4.323089in}{3.871225in}}{\pgfqpoint{4.313880in}{3.862016in}}%
\pgfpathcurveto{\pgfqpoint{4.304672in}{3.852808in}}{\pgfqpoint{4.299498in}{3.840317in}}{\pgfqpoint{4.299498in}{3.827294in}}%
\pgfpathcurveto{\pgfqpoint{4.299498in}{3.814271in}}{\pgfqpoint{4.304672in}{3.801780in}}{\pgfqpoint{4.313880in}{3.792572in}}%
\pgfpathcurveto{\pgfqpoint{4.323089in}{3.783363in}}{\pgfqpoint{4.335580in}{3.778189in}}{\pgfqpoint{4.348603in}{3.778189in}}%
\pgfpathlineto{\pgfqpoint{4.348603in}{3.778189in}}%
\pgfpathclose%
\pgfusepath{stroke,fill}%
\end{pgfscope}%
\begin{pgfscope}%
\pgfpathrectangle{\pgfqpoint{0.786164in}{0.768110in}}{\pgfqpoint{8.851069in}{7.081890in}}%
\pgfusepath{clip}%
\pgfsetbuttcap%
\pgfsetroundjoin%
\definecolor{currentfill}{rgb}{0.282884,0.135920,0.453427}%
\pgfsetfillcolor{currentfill}%
\pgfsetfillopacity{0.700000}%
\pgfsetlinewidth{0.501875pt}%
\definecolor{currentstroke}{rgb}{1.000000,1.000000,1.000000}%
\pgfsetstrokecolor{currentstroke}%
\pgfsetstrokeopacity{0.700000}%
\pgfsetdash{}{0pt}%
\pgfpathmoveto{\pgfqpoint{3.965005in}{3.537309in}}%
\pgfpathcurveto{\pgfqpoint{3.978027in}{3.537309in}}{\pgfqpoint{3.990518in}{3.542483in}}{\pgfqpoint{3.999727in}{3.551691in}}%
\pgfpathcurveto{\pgfqpoint{4.008935in}{3.560900in}}{\pgfqpoint{4.014109in}{3.573391in}}{\pgfqpoint{4.014109in}{3.586413in}}%
\pgfpathcurveto{\pgfqpoint{4.014109in}{3.599436in}}{\pgfqpoint{4.008935in}{3.611927in}}{\pgfqpoint{3.999727in}{3.621136in}}%
\pgfpathcurveto{\pgfqpoint{3.990518in}{3.630344in}}{\pgfqpoint{3.978027in}{3.635518in}}{\pgfqpoint{3.965005in}{3.635518in}}%
\pgfpathcurveto{\pgfqpoint{3.951982in}{3.635518in}}{\pgfqpoint{3.939491in}{3.630344in}}{\pgfqpoint{3.930282in}{3.621136in}}%
\pgfpathcurveto{\pgfqpoint{3.921074in}{3.611927in}}{\pgfqpoint{3.915900in}{3.599436in}}{\pgfqpoint{3.915900in}{3.586413in}}%
\pgfpathcurveto{\pgfqpoint{3.915900in}{3.573391in}}{\pgfqpoint{3.921074in}{3.560900in}}{\pgfqpoint{3.930282in}{3.551691in}}%
\pgfpathcurveto{\pgfqpoint{3.939491in}{3.542483in}}{\pgfqpoint{3.951982in}{3.537309in}}{\pgfqpoint{3.965005in}{3.537309in}}%
\pgfpathlineto{\pgfqpoint{3.965005in}{3.537309in}}%
\pgfpathclose%
\pgfusepath{stroke,fill}%
\end{pgfscope}%
\begin{pgfscope}%
\pgfpathrectangle{\pgfqpoint{0.786164in}{0.768110in}}{\pgfqpoint{8.851069in}{7.081890in}}%
\pgfusepath{clip}%
\pgfsetbuttcap%
\pgfsetroundjoin%
\definecolor{currentfill}{rgb}{0.281887,0.150881,0.465405}%
\pgfsetfillcolor{currentfill}%
\pgfsetfillopacity{0.700000}%
\pgfsetlinewidth{0.501875pt}%
\definecolor{currentstroke}{rgb}{1.000000,1.000000,1.000000}%
\pgfsetstrokecolor{currentstroke}%
\pgfsetstrokeopacity{0.700000}%
\pgfsetdash{}{0pt}%
\pgfpathmoveto{\pgfqpoint{3.745806in}{3.471614in}}%
\pgfpathcurveto{\pgfqpoint{3.758828in}{3.471614in}}{\pgfqpoint{3.771319in}{3.476788in}}{\pgfqpoint{3.780528in}{3.485996in}}%
\pgfpathcurveto{\pgfqpoint{3.789736in}{3.495205in}}{\pgfqpoint{3.794910in}{3.507696in}}{\pgfqpoint{3.794910in}{3.520719in}}%
\pgfpathcurveto{\pgfqpoint{3.794910in}{3.533741in}}{\pgfqpoint{3.789736in}{3.546232in}}{\pgfqpoint{3.780528in}{3.555441in}}%
\pgfpathcurveto{\pgfqpoint{3.771319in}{3.564649in}}{\pgfqpoint{3.758828in}{3.569823in}}{\pgfqpoint{3.745806in}{3.569823in}}%
\pgfpathcurveto{\pgfqpoint{3.732783in}{3.569823in}}{\pgfqpoint{3.720292in}{3.564649in}}{\pgfqpoint{3.711083in}{3.555441in}}%
\pgfpathcurveto{\pgfqpoint{3.701875in}{3.546232in}}{\pgfqpoint{3.696701in}{3.533741in}}{\pgfqpoint{3.696701in}{3.520719in}}%
\pgfpathcurveto{\pgfqpoint{3.696701in}{3.507696in}}{\pgfqpoint{3.701875in}{3.495205in}}{\pgfqpoint{3.711083in}{3.485996in}}%
\pgfpathcurveto{\pgfqpoint{3.720292in}{3.476788in}}{\pgfqpoint{3.732783in}{3.471614in}}{\pgfqpoint{3.745806in}{3.471614in}}%
\pgfpathlineto{\pgfqpoint{3.745806in}{3.471614in}}%
\pgfpathclose%
\pgfusepath{stroke,fill}%
\end{pgfscope}%
\begin{pgfscope}%
\pgfpathrectangle{\pgfqpoint{0.786164in}{0.768110in}}{\pgfqpoint{8.851069in}{7.081890in}}%
\pgfusepath{clip}%
\pgfsetbuttcap%
\pgfsetroundjoin%
\definecolor{currentfill}{rgb}{0.280868,0.160771,0.472899}%
\pgfsetfillcolor{currentfill}%
\pgfsetfillopacity{0.700000}%
\pgfsetlinewidth{0.501875pt}%
\definecolor{currentstroke}{rgb}{1.000000,1.000000,1.000000}%
\pgfsetstrokecolor{currentstroke}%
\pgfsetstrokeopacity{0.700000}%
\pgfsetdash{}{0pt}%
\pgfpathmoveto{\pgfqpoint{3.855405in}{3.449716in}}%
\pgfpathcurveto{\pgfqpoint{3.868428in}{3.449716in}}{\pgfqpoint{3.880919in}{3.454890in}}{\pgfqpoint{3.890127in}{3.464098in}}%
\pgfpathcurveto{\pgfqpoint{3.899336in}{3.473307in}}{\pgfqpoint{3.904510in}{3.485798in}}{\pgfqpoint{3.904510in}{3.498820in}}%
\pgfpathcurveto{\pgfqpoint{3.904510in}{3.511843in}}{\pgfqpoint{3.899336in}{3.524334in}}{\pgfqpoint{3.890127in}{3.533543in}}%
\pgfpathcurveto{\pgfqpoint{3.880919in}{3.542751in}}{\pgfqpoint{3.868428in}{3.547925in}}{\pgfqpoint{3.855405in}{3.547925in}}%
\pgfpathcurveto{\pgfqpoint{3.842382in}{3.547925in}}{\pgfqpoint{3.829891in}{3.542751in}}{\pgfqpoint{3.820683in}{3.533543in}}%
\pgfpathcurveto{\pgfqpoint{3.811474in}{3.524334in}}{\pgfqpoint{3.806301in}{3.511843in}}{\pgfqpoint{3.806301in}{3.498820in}}%
\pgfpathcurveto{\pgfqpoint{3.806301in}{3.485798in}}{\pgfqpoint{3.811474in}{3.473307in}}{\pgfqpoint{3.820683in}{3.464098in}}%
\pgfpathcurveto{\pgfqpoint{3.829891in}{3.454890in}}{\pgfqpoint{3.842382in}{3.449716in}}{\pgfqpoint{3.855405in}{3.449716in}}%
\pgfpathlineto{\pgfqpoint{3.855405in}{3.449716in}}%
\pgfpathclose%
\pgfusepath{stroke,fill}%
\end{pgfscope}%
\begin{pgfscope}%
\pgfpathrectangle{\pgfqpoint{0.786164in}{0.768110in}}{\pgfqpoint{8.851069in}{7.081890in}}%
\pgfusepath{clip}%
\pgfsetbuttcap%
\pgfsetroundjoin%
\definecolor{currentfill}{rgb}{0.280255,0.165693,0.476498}%
\pgfsetfillcolor{currentfill}%
\pgfsetfillopacity{0.700000}%
\pgfsetlinewidth{0.501875pt}%
\definecolor{currentstroke}{rgb}{1.000000,1.000000,1.000000}%
\pgfsetstrokecolor{currentstroke}%
\pgfsetstrokeopacity{0.700000}%
\pgfsetdash{}{0pt}%
\pgfpathmoveto{\pgfqpoint{3.563140in}{3.318326in}}%
\pgfpathcurveto{\pgfqpoint{3.576163in}{3.318326in}}{\pgfqpoint{3.588654in}{3.323500in}}{\pgfqpoint{3.597862in}{3.332709in}}%
\pgfpathcurveto{\pgfqpoint{3.607071in}{3.341917in}}{\pgfqpoint{3.612245in}{3.354408in}}{\pgfqpoint{3.612245in}{3.367431in}}%
\pgfpathcurveto{\pgfqpoint{3.612245in}{3.380454in}}{\pgfqpoint{3.607071in}{3.392945in}}{\pgfqpoint{3.597862in}{3.402153in}}%
\pgfpathcurveto{\pgfqpoint{3.588654in}{3.411362in}}{\pgfqpoint{3.576163in}{3.416536in}}{\pgfqpoint{3.563140in}{3.416536in}}%
\pgfpathcurveto{\pgfqpoint{3.550117in}{3.416536in}}{\pgfqpoint{3.537626in}{3.411362in}}{\pgfqpoint{3.528418in}{3.402153in}}%
\pgfpathcurveto{\pgfqpoint{3.519209in}{3.392945in}}{\pgfqpoint{3.514035in}{3.380454in}}{\pgfqpoint{3.514035in}{3.367431in}}%
\pgfpathcurveto{\pgfqpoint{3.514035in}{3.354408in}}{\pgfqpoint{3.519209in}{3.341917in}}{\pgfqpoint{3.528418in}{3.332709in}}%
\pgfpathcurveto{\pgfqpoint{3.537626in}{3.323500in}}{\pgfqpoint{3.550117in}{3.318326in}}{\pgfqpoint{3.563140in}{3.318326in}}%
\pgfpathlineto{\pgfqpoint{3.563140in}{3.318326in}}%
\pgfpathclose%
\pgfusepath{stroke,fill}%
\end{pgfscope}%
\begin{pgfscope}%
\pgfpathrectangle{\pgfqpoint{0.786164in}{0.768110in}}{\pgfqpoint{8.851069in}{7.081890in}}%
\pgfusepath{clip}%
\pgfsetbuttcap%
\pgfsetroundjoin%
\definecolor{currentfill}{rgb}{0.280255,0.165693,0.476498}%
\pgfsetfillcolor{currentfill}%
\pgfsetfillopacity{0.700000}%
\pgfsetlinewidth{0.501875pt}%
\definecolor{currentstroke}{rgb}{1.000000,1.000000,1.000000}%
\pgfsetstrokecolor{currentstroke}%
\pgfsetstrokeopacity{0.700000}%
\pgfsetdash{}{0pt}%
\pgfpathmoveto{\pgfqpoint{3.572273in}{3.405919in}}%
\pgfpathcurveto{\pgfqpoint{3.585296in}{3.405919in}}{\pgfqpoint{3.597787in}{3.411093in}}{\pgfqpoint{3.606995in}{3.420302in}}%
\pgfpathcurveto{\pgfqpoint{3.616204in}{3.429510in}}{\pgfqpoint{3.621378in}{3.442001in}}{\pgfqpoint{3.621378in}{3.455024in}}%
\pgfpathcurveto{\pgfqpoint{3.621378in}{3.468047in}}{\pgfqpoint{3.616204in}{3.480538in}}{\pgfqpoint{3.606995in}{3.489746in}}%
\pgfpathcurveto{\pgfqpoint{3.597787in}{3.498955in}}{\pgfqpoint{3.585296in}{3.504129in}}{\pgfqpoint{3.572273in}{3.504129in}}%
\pgfpathcurveto{\pgfqpoint{3.559251in}{3.504129in}}{\pgfqpoint{3.546759in}{3.498955in}}{\pgfqpoint{3.537551in}{3.489746in}}%
\pgfpathcurveto{\pgfqpoint{3.528343in}{3.480538in}}{\pgfqpoint{3.523169in}{3.468047in}}{\pgfqpoint{3.523169in}{3.455024in}}%
\pgfpathcurveto{\pgfqpoint{3.523169in}{3.442001in}}{\pgfqpoint{3.528343in}{3.429510in}}{\pgfqpoint{3.537551in}{3.420302in}}%
\pgfpathcurveto{\pgfqpoint{3.546759in}{3.411093in}}{\pgfqpoint{3.559251in}{3.405919in}}{\pgfqpoint{3.572273in}{3.405919in}}%
\pgfpathlineto{\pgfqpoint{3.572273in}{3.405919in}}%
\pgfpathclose%
\pgfusepath{stroke,fill}%
\end{pgfscope}%
\begin{pgfscope}%
\pgfpathrectangle{\pgfqpoint{0.786164in}{0.768110in}}{\pgfqpoint{8.851069in}{7.081890in}}%
\pgfusepath{clip}%
\pgfsetbuttcap%
\pgfsetroundjoin%
\definecolor{currentfill}{rgb}{0.278012,0.180367,0.486697}%
\pgfsetfillcolor{currentfill}%
\pgfsetfillopacity{0.700000}%
\pgfsetlinewidth{0.501875pt}%
\definecolor{currentstroke}{rgb}{1.000000,1.000000,1.000000}%
\pgfsetstrokecolor{currentstroke}%
\pgfsetstrokeopacity{0.700000}%
\pgfsetdash{}{0pt}%
\pgfpathmoveto{\pgfqpoint{3.782339in}{3.362123in}}%
\pgfpathcurveto{\pgfqpoint{3.795362in}{3.362123in}}{\pgfqpoint{3.807853in}{3.367297in}}{\pgfqpoint{3.817061in}{3.376505in}}%
\pgfpathcurveto{\pgfqpoint{3.826270in}{3.385714in}}{\pgfqpoint{3.831443in}{3.398205in}}{\pgfqpoint{3.831443in}{3.411228in}}%
\pgfpathcurveto{\pgfqpoint{3.831443in}{3.424250in}}{\pgfqpoint{3.826270in}{3.436741in}}{\pgfqpoint{3.817061in}{3.445950in}}%
\pgfpathcurveto{\pgfqpoint{3.807853in}{3.455158in}}{\pgfqpoint{3.795362in}{3.460332in}}{\pgfqpoint{3.782339in}{3.460332in}}%
\pgfpathcurveto{\pgfqpoint{3.769316in}{3.460332in}}{\pgfqpoint{3.756825in}{3.455158in}}{\pgfqpoint{3.747617in}{3.445950in}}%
\pgfpathcurveto{\pgfqpoint{3.738408in}{3.436741in}}{\pgfqpoint{3.733234in}{3.424250in}}{\pgfqpoint{3.733234in}{3.411228in}}%
\pgfpathcurveto{\pgfqpoint{3.733234in}{3.398205in}}{\pgfqpoint{3.738408in}{3.385714in}}{\pgfqpoint{3.747617in}{3.376505in}}%
\pgfpathcurveto{\pgfqpoint{3.756825in}{3.367297in}}{\pgfqpoint{3.769316in}{3.362123in}}{\pgfqpoint{3.782339in}{3.362123in}}%
\pgfpathlineto{\pgfqpoint{3.782339in}{3.362123in}}%
\pgfpathclose%
\pgfusepath{stroke,fill}%
\end{pgfscope}%
\begin{pgfscope}%
\pgfpathrectangle{\pgfqpoint{0.786164in}{0.768110in}}{\pgfqpoint{8.851069in}{7.081890in}}%
\pgfusepath{clip}%
\pgfsetbuttcap%
\pgfsetroundjoin%
\definecolor{currentfill}{rgb}{0.276194,0.190074,0.493001}%
\pgfsetfillcolor{currentfill}%
\pgfsetfillopacity{0.700000}%
\pgfsetlinewidth{0.501875pt}%
\definecolor{currentstroke}{rgb}{1.000000,1.000000,1.000000}%
\pgfsetstrokecolor{currentstroke}%
\pgfsetstrokeopacity{0.700000}%
\pgfsetdash{}{0pt}%
\pgfpathmoveto{\pgfqpoint{3.764072in}{3.362123in}}%
\pgfpathcurveto{\pgfqpoint{3.777095in}{3.362123in}}{\pgfqpoint{3.789586in}{3.367297in}}{\pgfqpoint{3.798794in}{3.376505in}}%
\pgfpathcurveto{\pgfqpoint{3.808003in}{3.385714in}}{\pgfqpoint{3.813177in}{3.398205in}}{\pgfqpoint{3.813177in}{3.411228in}}%
\pgfpathcurveto{\pgfqpoint{3.813177in}{3.424250in}}{\pgfqpoint{3.808003in}{3.436741in}}{\pgfqpoint{3.798794in}{3.445950in}}%
\pgfpathcurveto{\pgfqpoint{3.789586in}{3.455158in}}{\pgfqpoint{3.777095in}{3.460332in}}{\pgfqpoint{3.764072in}{3.460332in}}%
\pgfpathcurveto{\pgfqpoint{3.751050in}{3.460332in}}{\pgfqpoint{3.738558in}{3.455158in}}{\pgfqpoint{3.729350in}{3.445950in}}%
\pgfpathcurveto{\pgfqpoint{3.720142in}{3.436741in}}{\pgfqpoint{3.714968in}{3.424250in}}{\pgfqpoint{3.714968in}{3.411228in}}%
\pgfpathcurveto{\pgfqpoint{3.714968in}{3.398205in}}{\pgfqpoint{3.720142in}{3.385714in}}{\pgfqpoint{3.729350in}{3.376505in}}%
\pgfpathcurveto{\pgfqpoint{3.738558in}{3.367297in}}{\pgfqpoint{3.751050in}{3.362123in}}{\pgfqpoint{3.764072in}{3.362123in}}%
\pgfpathlineto{\pgfqpoint{3.764072in}{3.362123in}}%
\pgfpathclose%
\pgfusepath{stroke,fill}%
\end{pgfscope}%
\begin{pgfscope}%
\pgfpathrectangle{\pgfqpoint{0.786164in}{0.768110in}}{\pgfqpoint{8.851069in}{7.081890in}}%
\pgfusepath{clip}%
\pgfsetbuttcap%
\pgfsetroundjoin%
\definecolor{currentfill}{rgb}{0.275191,0.194905,0.496005}%
\pgfsetfillcolor{currentfill}%
\pgfsetfillopacity{0.700000}%
\pgfsetlinewidth{0.501875pt}%
\definecolor{currentstroke}{rgb}{1.000000,1.000000,1.000000}%
\pgfsetstrokecolor{currentstroke}%
\pgfsetstrokeopacity{0.700000}%
\pgfsetdash{}{0pt}%
\pgfpathmoveto{\pgfqpoint{3.590540in}{3.362123in}}%
\pgfpathcurveto{\pgfqpoint{3.603563in}{3.362123in}}{\pgfqpoint{3.616054in}{3.367297in}}{\pgfqpoint{3.625262in}{3.376505in}}%
\pgfpathcurveto{\pgfqpoint{3.634470in}{3.385714in}}{\pgfqpoint{3.639644in}{3.398205in}}{\pgfqpoint{3.639644in}{3.411228in}}%
\pgfpathcurveto{\pgfqpoint{3.639644in}{3.424250in}}{\pgfqpoint{3.634470in}{3.436741in}}{\pgfqpoint{3.625262in}{3.445950in}}%
\pgfpathcurveto{\pgfqpoint{3.616054in}{3.455158in}}{\pgfqpoint{3.603563in}{3.460332in}}{\pgfqpoint{3.590540in}{3.460332in}}%
\pgfpathcurveto{\pgfqpoint{3.577517in}{3.460332in}}{\pgfqpoint{3.565026in}{3.455158in}}{\pgfqpoint{3.555818in}{3.445950in}}%
\pgfpathcurveto{\pgfqpoint{3.546609in}{3.436741in}}{\pgfqpoint{3.541435in}{3.424250in}}{\pgfqpoint{3.541435in}{3.411228in}}%
\pgfpathcurveto{\pgfqpoint{3.541435in}{3.398205in}}{\pgfqpoint{3.546609in}{3.385714in}}{\pgfqpoint{3.555818in}{3.376505in}}%
\pgfpathcurveto{\pgfqpoint{3.565026in}{3.367297in}}{\pgfqpoint{3.577517in}{3.362123in}}{\pgfqpoint{3.590540in}{3.362123in}}%
\pgfpathlineto{\pgfqpoint{3.590540in}{3.362123in}}%
\pgfpathclose%
\pgfusepath{stroke,fill}%
\end{pgfscope}%
\begin{pgfscope}%
\pgfpathrectangle{\pgfqpoint{0.786164in}{0.768110in}}{\pgfqpoint{8.851069in}{7.081890in}}%
\pgfusepath{clip}%
\pgfsetbuttcap%
\pgfsetroundjoin%
\definecolor{currentfill}{rgb}{0.273006,0.204520,0.501721}%
\pgfsetfillcolor{currentfill}%
\pgfsetfillopacity{0.700000}%
\pgfsetlinewidth{0.501875pt}%
\definecolor{currentstroke}{rgb}{1.000000,1.000000,1.000000}%
\pgfsetstrokecolor{currentstroke}%
\pgfsetstrokeopacity{0.700000}%
\pgfsetdash{}{0pt}%
\pgfpathmoveto{\pgfqpoint{3.718406in}{3.340225in}}%
\pgfpathcurveto{\pgfqpoint{3.731429in}{3.340225in}}{\pgfqpoint{3.743920in}{3.345399in}}{\pgfqpoint{3.753128in}{3.354607in}}%
\pgfpathcurveto{\pgfqpoint{3.762336in}{3.363816in}}{\pgfqpoint{3.767510in}{3.376307in}}{\pgfqpoint{3.767510in}{3.389329in}}%
\pgfpathcurveto{\pgfqpoint{3.767510in}{3.402352in}}{\pgfqpoint{3.762336in}{3.414843in}}{\pgfqpoint{3.753128in}{3.424052in}}%
\pgfpathcurveto{\pgfqpoint{3.743920in}{3.433260in}}{\pgfqpoint{3.731429in}{3.438434in}}{\pgfqpoint{3.718406in}{3.438434in}}%
\pgfpathcurveto{\pgfqpoint{3.705383in}{3.438434in}}{\pgfqpoint{3.692892in}{3.433260in}}{\pgfqpoint{3.683684in}{3.424052in}}%
\pgfpathcurveto{\pgfqpoint{3.674475in}{3.414843in}}{\pgfqpoint{3.669301in}{3.402352in}}{\pgfqpoint{3.669301in}{3.389329in}}%
\pgfpathcurveto{\pgfqpoint{3.669301in}{3.376307in}}{\pgfqpoint{3.674475in}{3.363816in}}{\pgfqpoint{3.683684in}{3.354607in}}%
\pgfpathcurveto{\pgfqpoint{3.692892in}{3.345399in}}{\pgfqpoint{3.705383in}{3.340225in}}{\pgfqpoint{3.718406in}{3.340225in}}%
\pgfpathlineto{\pgfqpoint{3.718406in}{3.340225in}}%
\pgfpathclose%
\pgfusepath{stroke,fill}%
\end{pgfscope}%
\begin{pgfscope}%
\pgfpathrectangle{\pgfqpoint{0.786164in}{0.768110in}}{\pgfqpoint{8.851069in}{7.081890in}}%
\pgfusepath{clip}%
\pgfsetbuttcap%
\pgfsetroundjoin%
\definecolor{currentfill}{rgb}{0.255645,0.260703,0.528312}%
\pgfsetfillcolor{currentfill}%
\pgfsetfillopacity{0.700000}%
\pgfsetlinewidth{0.501875pt}%
\definecolor{currentstroke}{rgb}{1.000000,1.000000,1.000000}%
\pgfsetstrokecolor{currentstroke}%
\pgfsetstrokeopacity{0.700000}%
\pgfsetdash{}{0pt}%
\pgfpathmoveto{\pgfqpoint{3.343941in}{3.121242in}}%
\pgfpathcurveto{\pgfqpoint{3.356964in}{3.121242in}}{\pgfqpoint{3.369455in}{3.126416in}}{\pgfqpoint{3.378663in}{3.135625in}}%
\pgfpathcurveto{\pgfqpoint{3.387872in}{3.144833in}}{\pgfqpoint{3.393046in}{3.157324in}}{\pgfqpoint{3.393046in}{3.170347in}}%
\pgfpathcurveto{\pgfqpoint{3.393046in}{3.183370in}}{\pgfqpoint{3.387872in}{3.195861in}}{\pgfqpoint{3.378663in}{3.205069in}}%
\pgfpathcurveto{\pgfqpoint{3.369455in}{3.214278in}}{\pgfqpoint{3.356964in}{3.219452in}}{\pgfqpoint{3.343941in}{3.219452in}}%
\pgfpathcurveto{\pgfqpoint{3.330918in}{3.219452in}}{\pgfqpoint{3.318427in}{3.214278in}}{\pgfqpoint{3.309219in}{3.205069in}}%
\pgfpathcurveto{\pgfqpoint{3.300010in}{3.195861in}}{\pgfqpoint{3.294836in}{3.183370in}}{\pgfqpoint{3.294836in}{3.170347in}}%
\pgfpathcurveto{\pgfqpoint{3.294836in}{3.157324in}}{\pgfqpoint{3.300010in}{3.144833in}}{\pgfqpoint{3.309219in}{3.135625in}}%
\pgfpathcurveto{\pgfqpoint{3.318427in}{3.126416in}}{\pgfqpoint{3.330918in}{3.121242in}}{\pgfqpoint{3.343941in}{3.121242in}}%
\pgfpathlineto{\pgfqpoint{3.343941in}{3.121242in}}%
\pgfpathclose%
\pgfusepath{stroke,fill}%
\end{pgfscope}%
\begin{pgfscope}%
\pgfpathrectangle{\pgfqpoint{0.786164in}{0.768110in}}{\pgfqpoint{8.851069in}{7.081890in}}%
\pgfusepath{clip}%
\pgfsetbuttcap%
\pgfsetroundjoin%
\definecolor{currentfill}{rgb}{0.255645,0.260703,0.528312}%
\pgfsetfillcolor{currentfill}%
\pgfsetfillopacity{0.700000}%
\pgfsetlinewidth{0.501875pt}%
\definecolor{currentstroke}{rgb}{1.000000,1.000000,1.000000}%
\pgfsetstrokecolor{currentstroke}%
\pgfsetstrokeopacity{0.700000}%
\pgfsetdash{}{0pt}%
\pgfpathmoveto{\pgfqpoint{3.718406in}{3.186937in}}%
\pgfpathcurveto{\pgfqpoint{3.731429in}{3.186937in}}{\pgfqpoint{3.743920in}{3.192111in}}{\pgfqpoint{3.753128in}{3.201319in}}%
\pgfpathcurveto{\pgfqpoint{3.762336in}{3.210528in}}{\pgfqpoint{3.767510in}{3.223019in}}{\pgfqpoint{3.767510in}{3.236042in}}%
\pgfpathcurveto{\pgfqpoint{3.767510in}{3.249064in}}{\pgfqpoint{3.762336in}{3.261555in}}{\pgfqpoint{3.753128in}{3.270764in}}%
\pgfpathcurveto{\pgfqpoint{3.743920in}{3.279972in}}{\pgfqpoint{3.731429in}{3.285146in}}{\pgfqpoint{3.718406in}{3.285146in}}%
\pgfpathcurveto{\pgfqpoint{3.705383in}{3.285146in}}{\pgfqpoint{3.692892in}{3.279972in}}{\pgfqpoint{3.683684in}{3.270764in}}%
\pgfpathcurveto{\pgfqpoint{3.674475in}{3.261555in}}{\pgfqpoint{3.669301in}{3.249064in}}{\pgfqpoint{3.669301in}{3.236042in}}%
\pgfpathcurveto{\pgfqpoint{3.669301in}{3.223019in}}{\pgfqpoint{3.674475in}{3.210528in}}{\pgfqpoint{3.683684in}{3.201319in}}%
\pgfpathcurveto{\pgfqpoint{3.692892in}{3.192111in}}{\pgfqpoint{3.705383in}{3.186937in}}{\pgfqpoint{3.718406in}{3.186937in}}%
\pgfpathlineto{\pgfqpoint{3.718406in}{3.186937in}}%
\pgfpathclose%
\pgfusepath{stroke,fill}%
\end{pgfscope}%
\begin{pgfscope}%
\pgfpathrectangle{\pgfqpoint{0.786164in}{0.768110in}}{\pgfqpoint{8.851069in}{7.081890in}}%
\pgfusepath{clip}%
\pgfsetbuttcap%
\pgfsetroundjoin%
\definecolor{currentfill}{rgb}{0.250425,0.274290,0.533103}%
\pgfsetfillcolor{currentfill}%
\pgfsetfillopacity{0.700000}%
\pgfsetlinewidth{0.501875pt}%
\definecolor{currentstroke}{rgb}{1.000000,1.000000,1.000000}%
\pgfsetstrokecolor{currentstroke}%
\pgfsetstrokeopacity{0.700000}%
\pgfsetdash{}{0pt}%
\pgfpathmoveto{\pgfqpoint{3.407874in}{3.055548in}}%
\pgfpathcurveto{\pgfqpoint{3.420897in}{3.055548in}}{\pgfqpoint{3.433388in}{3.060722in}}{\pgfqpoint{3.442596in}{3.069930in}}%
\pgfpathcurveto{\pgfqpoint{3.451805in}{3.079138in}}{\pgfqpoint{3.456979in}{3.091630in}}{\pgfqpoint{3.456979in}{3.104652in}}%
\pgfpathcurveto{\pgfqpoint{3.456979in}{3.117675in}}{\pgfqpoint{3.451805in}{3.130166in}}{\pgfqpoint{3.442596in}{3.139374in}}%
\pgfpathcurveto{\pgfqpoint{3.433388in}{3.148583in}}{\pgfqpoint{3.420897in}{3.153757in}}{\pgfqpoint{3.407874in}{3.153757in}}%
\pgfpathcurveto{\pgfqpoint{3.394851in}{3.153757in}}{\pgfqpoint{3.382360in}{3.148583in}}{\pgfqpoint{3.373152in}{3.139374in}}%
\pgfpathcurveto{\pgfqpoint{3.363943in}{3.130166in}}{\pgfqpoint{3.358769in}{3.117675in}}{\pgfqpoint{3.358769in}{3.104652in}}%
\pgfpathcurveto{\pgfqpoint{3.358769in}{3.091630in}}{\pgfqpoint{3.363943in}{3.079138in}}{\pgfqpoint{3.373152in}{3.069930in}}%
\pgfpathcurveto{\pgfqpoint{3.382360in}{3.060722in}}{\pgfqpoint{3.394851in}{3.055548in}}{\pgfqpoint{3.407874in}{3.055548in}}%
\pgfpathlineto{\pgfqpoint{3.407874in}{3.055548in}}%
\pgfpathclose%
\pgfusepath{stroke,fill}%
\end{pgfscope}%
\begin{pgfscope}%
\pgfpathrectangle{\pgfqpoint{0.786164in}{0.768110in}}{\pgfqpoint{8.851069in}{7.081890in}}%
\pgfusepath{clip}%
\pgfsetbuttcap%
\pgfsetroundjoin%
\definecolor{currentfill}{rgb}{0.276194,0.190074,0.493001}%
\pgfsetfillcolor{currentfill}%
\pgfsetfillopacity{0.700000}%
\pgfsetlinewidth{0.501875pt}%
\definecolor{currentstroke}{rgb}{1.000000,1.000000,1.000000}%
\pgfsetstrokecolor{currentstroke}%
\pgfsetstrokeopacity{0.700000}%
\pgfsetdash{}{0pt}%
\pgfpathmoveto{\pgfqpoint{2.028748in}{2.354804in}}%
\pgfpathcurveto{\pgfqpoint{2.041770in}{2.354804in}}{\pgfqpoint{2.054261in}{2.359978in}}{\pgfqpoint{2.063470in}{2.369186in}}%
\pgfpathcurveto{\pgfqpoint{2.072678in}{2.378395in}}{\pgfqpoint{2.077852in}{2.390886in}}{\pgfqpoint{2.077852in}{2.403909in}}%
\pgfpathcurveto{\pgfqpoint{2.077852in}{2.416931in}}{\pgfqpoint{2.072678in}{2.429422in}}{\pgfqpoint{2.063470in}{2.438631in}}%
\pgfpathcurveto{\pgfqpoint{2.054261in}{2.447839in}}{\pgfqpoint{2.041770in}{2.453013in}}{\pgfqpoint{2.028748in}{2.453013in}}%
\pgfpathcurveto{\pgfqpoint{2.015725in}{2.453013in}}{\pgfqpoint{2.003234in}{2.447839in}}{\pgfqpoint{1.994025in}{2.438631in}}%
\pgfpathcurveto{\pgfqpoint{1.984817in}{2.429422in}}{\pgfqpoint{1.979643in}{2.416931in}}{\pgfqpoint{1.979643in}{2.403909in}}%
\pgfpathcurveto{\pgfqpoint{1.979643in}{2.390886in}}{\pgfqpoint{1.984817in}{2.378395in}}{\pgfqpoint{1.994025in}{2.369186in}}%
\pgfpathcurveto{\pgfqpoint{2.003234in}{2.359978in}}{\pgfqpoint{2.015725in}{2.354804in}}{\pgfqpoint{2.028748in}{2.354804in}}%
\pgfpathlineto{\pgfqpoint{2.028748in}{2.354804in}}%
\pgfpathclose%
\pgfusepath{stroke,fill}%
\end{pgfscope}%
\begin{pgfscope}%
\pgfpathrectangle{\pgfqpoint{0.786164in}{0.768110in}}{\pgfqpoint{8.851069in}{7.081890in}}%
\pgfusepath{clip}%
\pgfsetbuttcap%
\pgfsetroundjoin%
\definecolor{currentfill}{rgb}{0.276194,0.190074,0.493001}%
\pgfsetfillcolor{currentfill}%
\pgfsetfillopacity{0.700000}%
\pgfsetlinewidth{0.501875pt}%
\definecolor{currentstroke}{rgb}{1.000000,1.000000,1.000000}%
\pgfsetstrokecolor{currentstroke}%
\pgfsetstrokeopacity{0.700000}%
\pgfsetdash{}{0pt}%
\pgfpathmoveto{\pgfqpoint{2.165747in}{2.354804in}}%
\pgfpathcurveto{\pgfqpoint{2.178770in}{2.354804in}}{\pgfqpoint{2.191261in}{2.359978in}}{\pgfqpoint{2.200469in}{2.369186in}}%
\pgfpathcurveto{\pgfqpoint{2.209678in}{2.378395in}}{\pgfqpoint{2.214852in}{2.390886in}}{\pgfqpoint{2.214852in}{2.403909in}}%
\pgfpathcurveto{\pgfqpoint{2.214852in}{2.416931in}}{\pgfqpoint{2.209678in}{2.429422in}}{\pgfqpoint{2.200469in}{2.438631in}}%
\pgfpathcurveto{\pgfqpoint{2.191261in}{2.447839in}}{\pgfqpoint{2.178770in}{2.453013in}}{\pgfqpoint{2.165747in}{2.453013in}}%
\pgfpathcurveto{\pgfqpoint{2.152724in}{2.453013in}}{\pgfqpoint{2.140233in}{2.447839in}}{\pgfqpoint{2.131025in}{2.438631in}}%
\pgfpathcurveto{\pgfqpoint{2.121816in}{2.429422in}}{\pgfqpoint{2.116642in}{2.416931in}}{\pgfqpoint{2.116642in}{2.403909in}}%
\pgfpathcurveto{\pgfqpoint{2.116642in}{2.390886in}}{\pgfqpoint{2.121816in}{2.378395in}}{\pgfqpoint{2.131025in}{2.369186in}}%
\pgfpathcurveto{\pgfqpoint{2.140233in}{2.359978in}}{\pgfqpoint{2.152724in}{2.354804in}}{\pgfqpoint{2.165747in}{2.354804in}}%
\pgfpathlineto{\pgfqpoint{2.165747in}{2.354804in}}%
\pgfpathclose%
\pgfusepath{stroke,fill}%
\end{pgfscope}%
\begin{pgfscope}%
\pgfpathrectangle{\pgfqpoint{0.786164in}{0.768110in}}{\pgfqpoint{8.851069in}{7.081890in}}%
\pgfusepath{clip}%
\pgfsetbuttcap%
\pgfsetroundjoin%
\definecolor{currentfill}{rgb}{0.275191,0.194905,0.496005}%
\pgfsetfillcolor{currentfill}%
\pgfsetfillopacity{0.700000}%
\pgfsetlinewidth{0.501875pt}%
\definecolor{currentstroke}{rgb}{1.000000,1.000000,1.000000}%
\pgfsetstrokecolor{currentstroke}%
\pgfsetstrokeopacity{0.700000}%
\pgfsetdash{}{0pt}%
\pgfpathmoveto{\pgfqpoint{2.257080in}{2.486193in}}%
\pgfpathcurveto{\pgfqpoint{2.270103in}{2.486193in}}{\pgfqpoint{2.282594in}{2.491367in}}{\pgfqpoint{2.291802in}{2.500576in}}%
\pgfpathcurveto{\pgfqpoint{2.301010in}{2.509784in}}{\pgfqpoint{2.306184in}{2.522275in}}{\pgfqpoint{2.306184in}{2.535298in}}%
\pgfpathcurveto{\pgfqpoint{2.306184in}{2.548321in}}{\pgfqpoint{2.301010in}{2.560812in}}{\pgfqpoint{2.291802in}{2.570020in}}%
\pgfpathcurveto{\pgfqpoint{2.282594in}{2.579229in}}{\pgfqpoint{2.270103in}{2.584403in}}{\pgfqpoint{2.257080in}{2.584403in}}%
\pgfpathcurveto{\pgfqpoint{2.244057in}{2.584403in}}{\pgfqpoint{2.231566in}{2.579229in}}{\pgfqpoint{2.222358in}{2.570020in}}%
\pgfpathcurveto{\pgfqpoint{2.213149in}{2.560812in}}{\pgfqpoint{2.207975in}{2.548321in}}{\pgfqpoint{2.207975in}{2.535298in}}%
\pgfpathcurveto{\pgfqpoint{2.207975in}{2.522275in}}{\pgfqpoint{2.213149in}{2.509784in}}{\pgfqpoint{2.222358in}{2.500576in}}%
\pgfpathcurveto{\pgfqpoint{2.231566in}{2.491367in}}{\pgfqpoint{2.244057in}{2.486193in}}{\pgfqpoint{2.257080in}{2.486193in}}%
\pgfpathlineto{\pgfqpoint{2.257080in}{2.486193in}}%
\pgfpathclose%
\pgfusepath{stroke,fill}%
\end{pgfscope}%
\begin{pgfscope}%
\pgfpathrectangle{\pgfqpoint{0.786164in}{0.768110in}}{\pgfqpoint{8.851069in}{7.081890in}}%
\pgfusepath{clip}%
\pgfsetbuttcap%
\pgfsetroundjoin%
\definecolor{currentfill}{rgb}{0.277134,0.185228,0.489898}%
\pgfsetfillcolor{currentfill}%
\pgfsetfillopacity{0.700000}%
\pgfsetlinewidth{0.501875pt}%
\definecolor{currentstroke}{rgb}{1.000000,1.000000,1.000000}%
\pgfsetstrokecolor{currentstroke}%
\pgfsetstrokeopacity{0.700000}%
\pgfsetdash{}{0pt}%
\pgfpathmoveto{\pgfqpoint{2.238813in}{2.573786in}}%
\pgfpathcurveto{\pgfqpoint{2.251836in}{2.573786in}}{\pgfqpoint{2.264327in}{2.578960in}}{\pgfqpoint{2.273535in}{2.588169in}}%
\pgfpathcurveto{\pgfqpoint{2.282744in}{2.597377in}}{\pgfqpoint{2.287918in}{2.609868in}}{\pgfqpoint{2.287918in}{2.622891in}}%
\pgfpathcurveto{\pgfqpoint{2.287918in}{2.635914in}}{\pgfqpoint{2.282744in}{2.648405in}}{\pgfqpoint{2.273535in}{2.657613in}}%
\pgfpathcurveto{\pgfqpoint{2.264327in}{2.666822in}}{\pgfqpoint{2.251836in}{2.671996in}}{\pgfqpoint{2.238813in}{2.671996in}}%
\pgfpathcurveto{\pgfqpoint{2.225791in}{2.671996in}}{\pgfqpoint{2.213299in}{2.666822in}}{\pgfqpoint{2.204091in}{2.657613in}}%
\pgfpathcurveto{\pgfqpoint{2.194883in}{2.648405in}}{\pgfqpoint{2.189709in}{2.635914in}}{\pgfqpoint{2.189709in}{2.622891in}}%
\pgfpathcurveto{\pgfqpoint{2.189709in}{2.609868in}}{\pgfqpoint{2.194883in}{2.597377in}}{\pgfqpoint{2.204091in}{2.588169in}}%
\pgfpathcurveto{\pgfqpoint{2.213299in}{2.578960in}}{\pgfqpoint{2.225791in}{2.573786in}}{\pgfqpoint{2.238813in}{2.573786in}}%
\pgfpathlineto{\pgfqpoint{2.238813in}{2.573786in}}%
\pgfpathclose%
\pgfusepath{stroke,fill}%
\end{pgfscope}%
\begin{pgfscope}%
\pgfpathrectangle{\pgfqpoint{0.786164in}{0.768110in}}{\pgfqpoint{8.851069in}{7.081890in}}%
\pgfusepath{clip}%
\pgfsetbuttcap%
\pgfsetroundjoin%
\definecolor{currentfill}{rgb}{0.271828,0.209303,0.504434}%
\pgfsetfillcolor{currentfill}%
\pgfsetfillopacity{0.700000}%
\pgfsetlinewidth{0.501875pt}%
\definecolor{currentstroke}{rgb}{1.000000,1.000000,1.000000}%
\pgfsetstrokecolor{currentstroke}%
\pgfsetstrokeopacity{0.700000}%
\pgfsetdash{}{0pt}%
\pgfpathmoveto{\pgfqpoint{2.193147in}{2.595685in}}%
\pgfpathcurveto{\pgfqpoint{2.206170in}{2.595685in}}{\pgfqpoint{2.218661in}{2.600859in}}{\pgfqpoint{2.227869in}{2.610067in}}%
\pgfpathcurveto{\pgfqpoint{2.237077in}{2.619275in}}{\pgfqpoint{2.242251in}{2.631767in}}{\pgfqpoint{2.242251in}{2.644789in}}%
\pgfpathcurveto{\pgfqpoint{2.242251in}{2.657812in}}{\pgfqpoint{2.237077in}{2.670303in}}{\pgfqpoint{2.227869in}{2.679511in}}%
\pgfpathcurveto{\pgfqpoint{2.218661in}{2.688720in}}{\pgfqpoint{2.206170in}{2.693894in}}{\pgfqpoint{2.193147in}{2.693894in}}%
\pgfpathcurveto{\pgfqpoint{2.180124in}{2.693894in}}{\pgfqpoint{2.167633in}{2.688720in}}{\pgfqpoint{2.158425in}{2.679511in}}%
\pgfpathcurveto{\pgfqpoint{2.149216in}{2.670303in}}{\pgfqpoint{2.144042in}{2.657812in}}{\pgfqpoint{2.144042in}{2.644789in}}%
\pgfpathcurveto{\pgfqpoint{2.144042in}{2.631767in}}{\pgfqpoint{2.149216in}{2.619275in}}{\pgfqpoint{2.158425in}{2.610067in}}%
\pgfpathcurveto{\pgfqpoint{2.167633in}{2.600859in}}{\pgfqpoint{2.180124in}{2.595685in}}{\pgfqpoint{2.193147in}{2.595685in}}%
\pgfpathlineto{\pgfqpoint{2.193147in}{2.595685in}}%
\pgfpathclose%
\pgfusepath{stroke,fill}%
\end{pgfscope}%
\begin{pgfscope}%
\pgfpathrectangle{\pgfqpoint{0.786164in}{0.768110in}}{\pgfqpoint{8.851069in}{7.081890in}}%
\pgfusepath{clip}%
\pgfsetbuttcap%
\pgfsetroundjoin%
\definecolor{currentfill}{rgb}{0.262138,0.242286,0.520837}%
\pgfsetfillcolor{currentfill}%
\pgfsetfillopacity{0.700000}%
\pgfsetlinewidth{0.501875pt}%
\definecolor{currentstroke}{rgb}{1.000000,1.000000,1.000000}%
\pgfsetstrokecolor{currentstroke}%
\pgfsetstrokeopacity{0.700000}%
\pgfsetdash{}{0pt}%
\pgfpathmoveto{\pgfqpoint{1.946548in}{2.332906in}}%
\pgfpathcurveto{\pgfqpoint{1.959571in}{2.332906in}}{\pgfqpoint{1.972062in}{2.338080in}}{\pgfqpoint{1.981270in}{2.347288in}}%
\pgfpathcurveto{\pgfqpoint{1.990479in}{2.356497in}}{\pgfqpoint{1.995653in}{2.368988in}}{\pgfqpoint{1.995653in}{2.382010in}}%
\pgfpathcurveto{\pgfqpoint{1.995653in}{2.395033in}}{\pgfqpoint{1.990479in}{2.407524in}}{\pgfqpoint{1.981270in}{2.416733in}}%
\pgfpathcurveto{\pgfqpoint{1.972062in}{2.425941in}}{\pgfqpoint{1.959571in}{2.431115in}}{\pgfqpoint{1.946548in}{2.431115in}}%
\pgfpathcurveto{\pgfqpoint{1.933525in}{2.431115in}}{\pgfqpoint{1.921034in}{2.425941in}}{\pgfqpoint{1.911826in}{2.416733in}}%
\pgfpathcurveto{\pgfqpoint{1.902617in}{2.407524in}}{\pgfqpoint{1.897443in}{2.395033in}}{\pgfqpoint{1.897443in}{2.382010in}}%
\pgfpathcurveto{\pgfqpoint{1.897443in}{2.368988in}}{\pgfqpoint{1.902617in}{2.356497in}}{\pgfqpoint{1.911826in}{2.347288in}}%
\pgfpathcurveto{\pgfqpoint{1.921034in}{2.338080in}}{\pgfqpoint{1.933525in}{2.332906in}}{\pgfqpoint{1.946548in}{2.332906in}}%
\pgfpathlineto{\pgfqpoint{1.946548in}{2.332906in}}%
\pgfpathclose%
\pgfusepath{stroke,fill}%
\end{pgfscope}%
\begin{pgfscope}%
\pgfpathrectangle{\pgfqpoint{0.786164in}{0.768110in}}{\pgfqpoint{8.851069in}{7.081890in}}%
\pgfusepath{clip}%
\pgfsetbuttcap%
\pgfsetroundjoin%
\definecolor{currentfill}{rgb}{0.250425,0.274290,0.533103}%
\pgfsetfillcolor{currentfill}%
\pgfsetfillopacity{0.700000}%
\pgfsetlinewidth{0.501875pt}%
\definecolor{currentstroke}{rgb}{1.000000,1.000000,1.000000}%
\pgfsetstrokecolor{currentstroke}%
\pgfsetstrokeopacity{0.700000}%
\pgfsetdash{}{0pt}%
\pgfpathmoveto{\pgfqpoint{2.019614in}{2.442397in}}%
\pgfpathcurveto{\pgfqpoint{2.032637in}{2.442397in}}{\pgfqpoint{2.045128in}{2.447571in}}{\pgfqpoint{2.054337in}{2.456779in}}%
\pgfpathcurveto{\pgfqpoint{2.063545in}{2.465988in}}{\pgfqpoint{2.068719in}{2.478479in}}{\pgfqpoint{2.068719in}{2.491502in}}%
\pgfpathcurveto{\pgfqpoint{2.068719in}{2.504524in}}{\pgfqpoint{2.063545in}{2.517015in}}{\pgfqpoint{2.054337in}{2.526224in}}%
\pgfpathcurveto{\pgfqpoint{2.045128in}{2.535432in}}{\pgfqpoint{2.032637in}{2.540606in}}{\pgfqpoint{2.019614in}{2.540606in}}%
\pgfpathcurveto{\pgfqpoint{2.006592in}{2.540606in}}{\pgfqpoint{1.994101in}{2.535432in}}{\pgfqpoint{1.984892in}{2.526224in}}%
\pgfpathcurveto{\pgfqpoint{1.975684in}{2.517015in}}{\pgfqpoint{1.970510in}{2.504524in}}{\pgfqpoint{1.970510in}{2.491502in}}%
\pgfpathcurveto{\pgfqpoint{1.970510in}{2.478479in}}{\pgfqpoint{1.975684in}{2.465988in}}{\pgfqpoint{1.984892in}{2.456779in}}%
\pgfpathcurveto{\pgfqpoint{1.994101in}{2.447571in}}{\pgfqpoint{2.006592in}{2.442397in}}{\pgfqpoint{2.019614in}{2.442397in}}%
\pgfpathlineto{\pgfqpoint{2.019614in}{2.442397in}}%
\pgfpathclose%
\pgfusepath{stroke,fill}%
\end{pgfscope}%
\begin{pgfscope}%
\pgfpathrectangle{\pgfqpoint{0.786164in}{0.768110in}}{\pgfqpoint{8.851069in}{7.081890in}}%
\pgfusepath{clip}%
\pgfsetbuttcap%
\pgfsetroundjoin%
\definecolor{currentfill}{rgb}{0.248629,0.278775,0.534556}%
\pgfsetfillcolor{currentfill}%
\pgfsetfillopacity{0.700000}%
\pgfsetlinewidth{0.501875pt}%
\definecolor{currentstroke}{rgb}{1.000000,1.000000,1.000000}%
\pgfsetstrokecolor{currentstroke}%
\pgfsetstrokeopacity{0.700000}%
\pgfsetdash{}{0pt}%
\pgfpathmoveto{\pgfqpoint{2.184014in}{2.705176in}}%
\pgfpathcurveto{\pgfqpoint{2.197036in}{2.705176in}}{\pgfqpoint{2.209527in}{2.710350in}}{\pgfqpoint{2.218736in}{2.719558in}}%
\pgfpathcurveto{\pgfqpoint{2.227944in}{2.728767in}}{\pgfqpoint{2.233118in}{2.741258in}}{\pgfqpoint{2.233118in}{2.754280in}}%
\pgfpathcurveto{\pgfqpoint{2.233118in}{2.767303in}}{\pgfqpoint{2.227944in}{2.779794in}}{\pgfqpoint{2.218736in}{2.789003in}}%
\pgfpathcurveto{\pgfqpoint{2.209527in}{2.798211in}}{\pgfqpoint{2.197036in}{2.803385in}}{\pgfqpoint{2.184014in}{2.803385in}}%
\pgfpathcurveto{\pgfqpoint{2.170991in}{2.803385in}}{\pgfqpoint{2.158500in}{2.798211in}}{\pgfqpoint{2.149291in}{2.789003in}}%
\pgfpathcurveto{\pgfqpoint{2.140083in}{2.779794in}}{\pgfqpoint{2.134909in}{2.767303in}}{\pgfqpoint{2.134909in}{2.754280in}}%
\pgfpathcurveto{\pgfqpoint{2.134909in}{2.741258in}}{\pgfqpoint{2.140083in}{2.728767in}}{\pgfqpoint{2.149291in}{2.719558in}}%
\pgfpathcurveto{\pgfqpoint{2.158500in}{2.710350in}}{\pgfqpoint{2.170991in}{2.705176in}}{\pgfqpoint{2.184014in}{2.705176in}}%
\pgfpathlineto{\pgfqpoint{2.184014in}{2.705176in}}%
\pgfpathclose%
\pgfusepath{stroke,fill}%
\end{pgfscope}%
\begin{pgfscope}%
\pgfpathrectangle{\pgfqpoint{0.786164in}{0.768110in}}{\pgfqpoint{8.851069in}{7.081890in}}%
\pgfusepath{clip}%
\pgfsetbuttcap%
\pgfsetroundjoin%
\definecolor{currentfill}{rgb}{0.235526,0.309527,0.542944}%
\pgfsetfillcolor{currentfill}%
\pgfsetfillopacity{0.700000}%
\pgfsetlinewidth{0.501875pt}%
\definecolor{currentstroke}{rgb}{1.000000,1.000000,1.000000}%
\pgfsetstrokecolor{currentstroke}%
\pgfsetstrokeopacity{0.700000}%
\pgfsetdash{}{0pt}%
\pgfpathmoveto{\pgfqpoint{2.101814in}{2.573786in}}%
\pgfpathcurveto{\pgfqpoint{2.114837in}{2.573786in}}{\pgfqpoint{2.127328in}{2.578960in}}{\pgfqpoint{2.136536in}{2.588169in}}%
\pgfpathcurveto{\pgfqpoint{2.145745in}{2.597377in}}{\pgfqpoint{2.150919in}{2.609868in}}{\pgfqpoint{2.150919in}{2.622891in}}%
\pgfpathcurveto{\pgfqpoint{2.150919in}{2.635914in}}{\pgfqpoint{2.145745in}{2.648405in}}{\pgfqpoint{2.136536in}{2.657613in}}%
\pgfpathcurveto{\pgfqpoint{2.127328in}{2.666822in}}{\pgfqpoint{2.114837in}{2.671996in}}{\pgfqpoint{2.101814in}{2.671996in}}%
\pgfpathcurveto{\pgfqpoint{2.088791in}{2.671996in}}{\pgfqpoint{2.076300in}{2.666822in}}{\pgfqpoint{2.067092in}{2.657613in}}%
\pgfpathcurveto{\pgfqpoint{2.057883in}{2.648405in}}{\pgfqpoint{2.052709in}{2.635914in}}{\pgfqpoint{2.052709in}{2.622891in}}%
\pgfpathcurveto{\pgfqpoint{2.052709in}{2.609868in}}{\pgfqpoint{2.057883in}{2.597377in}}{\pgfqpoint{2.067092in}{2.588169in}}%
\pgfpathcurveto{\pgfqpoint{2.076300in}{2.578960in}}{\pgfqpoint{2.088791in}{2.573786in}}{\pgfqpoint{2.101814in}{2.573786in}}%
\pgfpathlineto{\pgfqpoint{2.101814in}{2.573786in}}%
\pgfpathclose%
\pgfusepath{stroke,fill}%
\end{pgfscope}%
\begin{pgfscope}%
\pgfpathrectangle{\pgfqpoint{0.786164in}{0.768110in}}{\pgfqpoint{8.851069in}{7.081890in}}%
\pgfusepath{clip}%
\pgfsetbuttcap%
\pgfsetroundjoin%
\definecolor{currentfill}{rgb}{0.212395,0.359683,0.551710}%
\pgfsetfillcolor{currentfill}%
\pgfsetfillopacity{0.700000}%
\pgfsetlinewidth{0.501875pt}%
\definecolor{currentstroke}{rgb}{1.000000,1.000000,1.000000}%
\pgfsetstrokecolor{currentstroke}%
\pgfsetstrokeopacity{0.700000}%
\pgfsetdash{}{0pt}%
\pgfpathmoveto{\pgfqpoint{1.946548in}{2.464295in}}%
\pgfpathcurveto{\pgfqpoint{1.959571in}{2.464295in}}{\pgfqpoint{1.972062in}{2.469469in}}{\pgfqpoint{1.981270in}{2.478678in}}%
\pgfpathcurveto{\pgfqpoint{1.990479in}{2.487886in}}{\pgfqpoint{1.995653in}{2.500377in}}{\pgfqpoint{1.995653in}{2.513400in}}%
\pgfpathcurveto{\pgfqpoint{1.995653in}{2.526423in}}{\pgfqpoint{1.990479in}{2.538914in}}{\pgfqpoint{1.981270in}{2.548122in}}%
\pgfpathcurveto{\pgfqpoint{1.972062in}{2.557331in}}{\pgfqpoint{1.959571in}{2.562504in}}{\pgfqpoint{1.946548in}{2.562504in}}%
\pgfpathcurveto{\pgfqpoint{1.933525in}{2.562504in}}{\pgfqpoint{1.921034in}{2.557331in}}{\pgfqpoint{1.911826in}{2.548122in}}%
\pgfpathcurveto{\pgfqpoint{1.902617in}{2.538914in}}{\pgfqpoint{1.897443in}{2.526423in}}{\pgfqpoint{1.897443in}{2.513400in}}%
\pgfpathcurveto{\pgfqpoint{1.897443in}{2.500377in}}{\pgfqpoint{1.902617in}{2.487886in}}{\pgfqpoint{1.911826in}{2.478678in}}%
\pgfpathcurveto{\pgfqpoint{1.921034in}{2.469469in}}{\pgfqpoint{1.933525in}{2.464295in}}{\pgfqpoint{1.946548in}{2.464295in}}%
\pgfpathlineto{\pgfqpoint{1.946548in}{2.464295in}}%
\pgfpathclose%
\pgfusepath{stroke,fill}%
\end{pgfscope}%
\begin{pgfscope}%
\pgfpathrectangle{\pgfqpoint{0.786164in}{0.768110in}}{\pgfqpoint{8.851069in}{7.081890in}}%
\pgfusepath{clip}%
\pgfsetbuttcap%
\pgfsetroundjoin%
\definecolor{currentfill}{rgb}{0.220057,0.343307,0.549413}%
\pgfsetfillcolor{currentfill}%
\pgfsetfillopacity{0.700000}%
\pgfsetlinewidth{0.501875pt}%
\definecolor{currentstroke}{rgb}{1.000000,1.000000,1.000000}%
\pgfsetstrokecolor{currentstroke}%
\pgfsetstrokeopacity{0.700000}%
\pgfsetdash{}{0pt}%
\pgfpathmoveto{\pgfqpoint{2.056148in}{2.508092in}}%
\pgfpathcurveto{\pgfqpoint{2.069170in}{2.508092in}}{\pgfqpoint{2.081661in}{2.513266in}}{\pgfqpoint{2.090870in}{2.522474in}}%
\pgfpathcurveto{\pgfqpoint{2.100078in}{2.531683in}}{\pgfqpoint{2.105252in}{2.544174in}}{\pgfqpoint{2.105252in}{2.557196in}}%
\pgfpathcurveto{\pgfqpoint{2.105252in}{2.570219in}}{\pgfqpoint{2.100078in}{2.582710in}}{\pgfqpoint{2.090870in}{2.591919in}}%
\pgfpathcurveto{\pgfqpoint{2.081661in}{2.601127in}}{\pgfqpoint{2.069170in}{2.606301in}}{\pgfqpoint{2.056148in}{2.606301in}}%
\pgfpathcurveto{\pgfqpoint{2.043125in}{2.606301in}}{\pgfqpoint{2.030634in}{2.601127in}}{\pgfqpoint{2.021425in}{2.591919in}}%
\pgfpathcurveto{\pgfqpoint{2.012217in}{2.582710in}}{\pgfqpoint{2.007043in}{2.570219in}}{\pgfqpoint{2.007043in}{2.557196in}}%
\pgfpathcurveto{\pgfqpoint{2.007043in}{2.544174in}}{\pgfqpoint{2.012217in}{2.531683in}}{\pgfqpoint{2.021425in}{2.522474in}}%
\pgfpathcurveto{\pgfqpoint{2.030634in}{2.513266in}}{\pgfqpoint{2.043125in}{2.508092in}}{\pgfqpoint{2.056148in}{2.508092in}}%
\pgfpathlineto{\pgfqpoint{2.056148in}{2.508092in}}%
\pgfpathclose%
\pgfusepath{stroke,fill}%
\end{pgfscope}%
\begin{pgfscope}%
\pgfpathrectangle{\pgfqpoint{0.786164in}{0.768110in}}{\pgfqpoint{8.851069in}{7.081890in}}%
\pgfusepath{clip}%
\pgfsetbuttcap%
\pgfsetroundjoin%
\definecolor{currentfill}{rgb}{0.218130,0.347432,0.550038}%
\pgfsetfillcolor{currentfill}%
\pgfsetfillopacity{0.700000}%
\pgfsetlinewidth{0.501875pt}%
\definecolor{currentstroke}{rgb}{1.000000,1.000000,1.000000}%
\pgfsetstrokecolor{currentstroke}%
\pgfsetstrokeopacity{0.700000}%
\pgfsetdash{}{0pt}%
\pgfpathmoveto{\pgfqpoint{2.156614in}{2.639481in}}%
\pgfpathcurveto{\pgfqpoint{2.169636in}{2.639481in}}{\pgfqpoint{2.182127in}{2.644655in}}{\pgfqpoint{2.191336in}{2.653864in}}%
\pgfpathcurveto{\pgfqpoint{2.200544in}{2.663072in}}{\pgfqpoint{2.205718in}{2.675563in}}{\pgfqpoint{2.205718in}{2.688586in}}%
\pgfpathcurveto{\pgfqpoint{2.205718in}{2.701608in}}{\pgfqpoint{2.200544in}{2.714100in}}{\pgfqpoint{2.191336in}{2.723308in}}%
\pgfpathcurveto{\pgfqpoint{2.182127in}{2.732516in}}{\pgfqpoint{2.169636in}{2.737690in}}{\pgfqpoint{2.156614in}{2.737690in}}%
\pgfpathcurveto{\pgfqpoint{2.143591in}{2.737690in}}{\pgfqpoint{2.131100in}{2.732516in}}{\pgfqpoint{2.121891in}{2.723308in}}%
\pgfpathcurveto{\pgfqpoint{2.112683in}{2.714100in}}{\pgfqpoint{2.107509in}{2.701608in}}{\pgfqpoint{2.107509in}{2.688586in}}%
\pgfpathcurveto{\pgfqpoint{2.107509in}{2.675563in}}{\pgfqpoint{2.112683in}{2.663072in}}{\pgfqpoint{2.121891in}{2.653864in}}%
\pgfpathcurveto{\pgfqpoint{2.131100in}{2.644655in}}{\pgfqpoint{2.143591in}{2.639481in}}{\pgfqpoint{2.156614in}{2.639481in}}%
\pgfpathlineto{\pgfqpoint{2.156614in}{2.639481in}}%
\pgfpathclose%
\pgfusepath{stroke,fill}%
\end{pgfscope}%
\begin{pgfscope}%
\pgfpathrectangle{\pgfqpoint{0.786164in}{0.768110in}}{\pgfqpoint{8.851069in}{7.081890in}}%
\pgfusepath{clip}%
\pgfsetbuttcap%
\pgfsetroundjoin%
\definecolor{currentfill}{rgb}{0.214298,0.355619,0.551184}%
\pgfsetfillcolor{currentfill}%
\pgfsetfillopacity{0.700000}%
\pgfsetlinewidth{0.501875pt}%
\definecolor{currentstroke}{rgb}{1.000000,1.000000,1.000000}%
\pgfsetstrokecolor{currentstroke}%
\pgfsetstrokeopacity{0.700000}%
\pgfsetdash{}{0pt}%
\pgfpathmoveto{\pgfqpoint{2.138347in}{2.508092in}}%
\pgfpathcurveto{\pgfqpoint{2.151370in}{2.508092in}}{\pgfqpoint{2.163861in}{2.513266in}}{\pgfqpoint{2.173069in}{2.522474in}}%
\pgfpathcurveto{\pgfqpoint{2.182278in}{2.531683in}}{\pgfqpoint{2.187452in}{2.544174in}}{\pgfqpoint{2.187452in}{2.557196in}}%
\pgfpathcurveto{\pgfqpoint{2.187452in}{2.570219in}}{\pgfqpoint{2.182278in}{2.582710in}}{\pgfqpoint{2.173069in}{2.591919in}}%
\pgfpathcurveto{\pgfqpoint{2.163861in}{2.601127in}}{\pgfqpoint{2.151370in}{2.606301in}}{\pgfqpoint{2.138347in}{2.606301in}}%
\pgfpathcurveto{\pgfqpoint{2.125324in}{2.606301in}}{\pgfqpoint{2.112833in}{2.601127in}}{\pgfqpoint{2.103625in}{2.591919in}}%
\pgfpathcurveto{\pgfqpoint{2.094416in}{2.582710in}}{\pgfqpoint{2.089242in}{2.570219in}}{\pgfqpoint{2.089242in}{2.557196in}}%
\pgfpathcurveto{\pgfqpoint{2.089242in}{2.544174in}}{\pgfqpoint{2.094416in}{2.531683in}}{\pgfqpoint{2.103625in}{2.522474in}}%
\pgfpathcurveto{\pgfqpoint{2.112833in}{2.513266in}}{\pgfqpoint{2.125324in}{2.508092in}}{\pgfqpoint{2.138347in}{2.508092in}}%
\pgfpathlineto{\pgfqpoint{2.138347in}{2.508092in}}%
\pgfpathclose%
\pgfusepath{stroke,fill}%
\end{pgfscope}%
\begin{pgfscope}%
\pgfpathrectangle{\pgfqpoint{0.786164in}{0.768110in}}{\pgfqpoint{8.851069in}{7.081890in}}%
\pgfusepath{clip}%
\pgfsetbuttcap%
\pgfsetroundjoin%
\definecolor{currentfill}{rgb}{0.214298,0.355619,0.551184}%
\pgfsetfillcolor{currentfill}%
\pgfsetfillopacity{0.700000}%
\pgfsetlinewidth{0.501875pt}%
\definecolor{currentstroke}{rgb}{1.000000,1.000000,1.000000}%
\pgfsetstrokecolor{currentstroke}%
\pgfsetstrokeopacity{0.700000}%
\pgfsetdash{}{0pt}%
\pgfpathmoveto{\pgfqpoint{2.238813in}{2.595685in}}%
\pgfpathcurveto{\pgfqpoint{2.251836in}{2.595685in}}{\pgfqpoint{2.264327in}{2.600859in}}{\pgfqpoint{2.273535in}{2.610067in}}%
\pgfpathcurveto{\pgfqpoint{2.282744in}{2.619275in}}{\pgfqpoint{2.287918in}{2.631767in}}{\pgfqpoint{2.287918in}{2.644789in}}%
\pgfpathcurveto{\pgfqpoint{2.287918in}{2.657812in}}{\pgfqpoint{2.282744in}{2.670303in}}{\pgfqpoint{2.273535in}{2.679511in}}%
\pgfpathcurveto{\pgfqpoint{2.264327in}{2.688720in}}{\pgfqpoint{2.251836in}{2.693894in}}{\pgfqpoint{2.238813in}{2.693894in}}%
\pgfpathcurveto{\pgfqpoint{2.225791in}{2.693894in}}{\pgfqpoint{2.213299in}{2.688720in}}{\pgfqpoint{2.204091in}{2.679511in}}%
\pgfpathcurveto{\pgfqpoint{2.194883in}{2.670303in}}{\pgfqpoint{2.189709in}{2.657812in}}{\pgfqpoint{2.189709in}{2.644789in}}%
\pgfpathcurveto{\pgfqpoint{2.189709in}{2.631767in}}{\pgfqpoint{2.194883in}{2.619275in}}{\pgfqpoint{2.204091in}{2.610067in}}%
\pgfpathcurveto{\pgfqpoint{2.213299in}{2.600859in}}{\pgfqpoint{2.225791in}{2.595685in}}{\pgfqpoint{2.238813in}{2.595685in}}%
\pgfpathlineto{\pgfqpoint{2.238813in}{2.595685in}}%
\pgfpathclose%
\pgfusepath{stroke,fill}%
\end{pgfscope}%
\begin{pgfscope}%
\pgfpathrectangle{\pgfqpoint{0.786164in}{0.768110in}}{\pgfqpoint{8.851069in}{7.081890in}}%
\pgfusepath{clip}%
\pgfsetbuttcap%
\pgfsetroundjoin%
\definecolor{currentfill}{rgb}{0.201239,0.383670,0.554294}%
\pgfsetfillcolor{currentfill}%
\pgfsetfillopacity{0.700000}%
\pgfsetlinewidth{0.501875pt}%
\definecolor{currentstroke}{rgb}{1.000000,1.000000,1.000000}%
\pgfsetstrokecolor{currentstroke}%
\pgfsetstrokeopacity{0.700000}%
\pgfsetdash{}{0pt}%
\pgfpathmoveto{\pgfqpoint{2.247947in}{2.464295in}}%
\pgfpathcurveto{\pgfqpoint{2.260969in}{2.464295in}}{\pgfqpoint{2.273460in}{2.469469in}}{\pgfqpoint{2.282669in}{2.478678in}}%
\pgfpathcurveto{\pgfqpoint{2.291877in}{2.487886in}}{\pgfqpoint{2.297051in}{2.500377in}}{\pgfqpoint{2.297051in}{2.513400in}}%
\pgfpathcurveto{\pgfqpoint{2.297051in}{2.526423in}}{\pgfqpoint{2.291877in}{2.538914in}}{\pgfqpoint{2.282669in}{2.548122in}}%
\pgfpathcurveto{\pgfqpoint{2.273460in}{2.557331in}}{\pgfqpoint{2.260969in}{2.562504in}}{\pgfqpoint{2.247947in}{2.562504in}}%
\pgfpathcurveto{\pgfqpoint{2.234924in}{2.562504in}}{\pgfqpoint{2.222433in}{2.557331in}}{\pgfqpoint{2.213224in}{2.548122in}}%
\pgfpathcurveto{\pgfqpoint{2.204016in}{2.538914in}}{\pgfqpoint{2.198842in}{2.526423in}}{\pgfqpoint{2.198842in}{2.513400in}}%
\pgfpathcurveto{\pgfqpoint{2.198842in}{2.500377in}}{\pgfqpoint{2.204016in}{2.487886in}}{\pgfqpoint{2.213224in}{2.478678in}}%
\pgfpathcurveto{\pgfqpoint{2.222433in}{2.469469in}}{\pgfqpoint{2.234924in}{2.464295in}}{\pgfqpoint{2.247947in}{2.464295in}}%
\pgfpathlineto{\pgfqpoint{2.247947in}{2.464295in}}%
\pgfpathclose%
\pgfusepath{stroke,fill}%
\end{pgfscope}%
\begin{pgfscope}%
\pgfpathrectangle{\pgfqpoint{0.786164in}{0.768110in}}{\pgfqpoint{8.851069in}{7.081890in}}%
\pgfusepath{clip}%
\pgfsetbuttcap%
\pgfsetroundjoin%
\definecolor{currentfill}{rgb}{0.194100,0.399323,0.555565}%
\pgfsetfillcolor{currentfill}%
\pgfsetfillopacity{0.700000}%
\pgfsetlinewidth{0.501875pt}%
\definecolor{currentstroke}{rgb}{1.000000,1.000000,1.000000}%
\pgfsetstrokecolor{currentstroke}%
\pgfsetstrokeopacity{0.700000}%
\pgfsetdash{}{0pt}%
\pgfpathmoveto{\pgfqpoint{2.238813in}{2.442397in}}%
\pgfpathcurveto{\pgfqpoint{2.251836in}{2.442397in}}{\pgfqpoint{2.264327in}{2.447571in}}{\pgfqpoint{2.273535in}{2.456779in}}%
\pgfpathcurveto{\pgfqpoint{2.282744in}{2.465988in}}{\pgfqpoint{2.287918in}{2.478479in}}{\pgfqpoint{2.287918in}{2.491502in}}%
\pgfpathcurveto{\pgfqpoint{2.287918in}{2.504524in}}{\pgfqpoint{2.282744in}{2.517015in}}{\pgfqpoint{2.273535in}{2.526224in}}%
\pgfpathcurveto{\pgfqpoint{2.264327in}{2.535432in}}{\pgfqpoint{2.251836in}{2.540606in}}{\pgfqpoint{2.238813in}{2.540606in}}%
\pgfpathcurveto{\pgfqpoint{2.225791in}{2.540606in}}{\pgfqpoint{2.213299in}{2.535432in}}{\pgfqpoint{2.204091in}{2.526224in}}%
\pgfpathcurveto{\pgfqpoint{2.194883in}{2.517015in}}{\pgfqpoint{2.189709in}{2.504524in}}{\pgfqpoint{2.189709in}{2.491502in}}%
\pgfpathcurveto{\pgfqpoint{2.189709in}{2.478479in}}{\pgfqpoint{2.194883in}{2.465988in}}{\pgfqpoint{2.204091in}{2.456779in}}%
\pgfpathcurveto{\pgfqpoint{2.213299in}{2.447571in}}{\pgfqpoint{2.225791in}{2.442397in}}{\pgfqpoint{2.238813in}{2.442397in}}%
\pgfpathlineto{\pgfqpoint{2.238813in}{2.442397in}}%
\pgfpathclose%
\pgfusepath{stroke,fill}%
\end{pgfscope}%
\begin{pgfscope}%
\pgfpathrectangle{\pgfqpoint{0.786164in}{0.768110in}}{\pgfqpoint{8.851069in}{7.081890in}}%
\pgfusepath{clip}%
\pgfsetbuttcap%
\pgfsetroundjoin%
\definecolor{currentfill}{rgb}{0.182256,0.426184,0.557120}%
\pgfsetfillcolor{currentfill}%
\pgfsetfillopacity{0.700000}%
\pgfsetlinewidth{0.501875pt}%
\definecolor{currentstroke}{rgb}{1.000000,1.000000,1.000000}%
\pgfsetstrokecolor{currentstroke}%
\pgfsetstrokeopacity{0.700000}%
\pgfsetdash{}{0pt}%
\pgfpathmoveto{\pgfqpoint{2.120081in}{2.245313in}}%
\pgfpathcurveto{\pgfqpoint{2.133103in}{2.245313in}}{\pgfqpoint{2.145594in}{2.250487in}}{\pgfqpoint{2.154803in}{2.259695in}}%
\pgfpathcurveto{\pgfqpoint{2.164011in}{2.268904in}}{\pgfqpoint{2.169185in}{2.281395in}}{\pgfqpoint{2.169185in}{2.294417in}}%
\pgfpathcurveto{\pgfqpoint{2.169185in}{2.307440in}}{\pgfqpoint{2.164011in}{2.319931in}}{\pgfqpoint{2.154803in}{2.329140in}}%
\pgfpathcurveto{\pgfqpoint{2.145594in}{2.338348in}}{\pgfqpoint{2.133103in}{2.343522in}}{\pgfqpoint{2.120081in}{2.343522in}}%
\pgfpathcurveto{\pgfqpoint{2.107058in}{2.343522in}}{\pgfqpoint{2.094567in}{2.338348in}}{\pgfqpoint{2.085358in}{2.329140in}}%
\pgfpathcurveto{\pgfqpoint{2.076150in}{2.319931in}}{\pgfqpoint{2.070976in}{2.307440in}}{\pgfqpoint{2.070976in}{2.294417in}}%
\pgfpathcurveto{\pgfqpoint{2.070976in}{2.281395in}}{\pgfqpoint{2.076150in}{2.268904in}}{\pgfqpoint{2.085358in}{2.259695in}}%
\pgfpathcurveto{\pgfqpoint{2.094567in}{2.250487in}}{\pgfqpoint{2.107058in}{2.245313in}}{\pgfqpoint{2.120081in}{2.245313in}}%
\pgfpathlineto{\pgfqpoint{2.120081in}{2.245313in}}%
\pgfpathclose%
\pgfusepath{stroke,fill}%
\end{pgfscope}%
\begin{pgfscope}%
\pgfpathrectangle{\pgfqpoint{0.786164in}{0.768110in}}{\pgfqpoint{8.851069in}{7.081890in}}%
\pgfusepath{clip}%
\pgfsetbuttcap%
\pgfsetroundjoin%
\definecolor{currentfill}{rgb}{0.208623,0.367752,0.552675}%
\pgfsetfillcolor{currentfill}%
\pgfsetfillopacity{0.700000}%
\pgfsetlinewidth{0.501875pt}%
\definecolor{currentstroke}{rgb}{1.000000,1.000000,1.000000}%
\pgfsetstrokecolor{currentstroke}%
\pgfsetstrokeopacity{0.700000}%
\pgfsetdash{}{0pt}%
\pgfpathmoveto{\pgfqpoint{2.339279in}{2.442397in}}%
\pgfpathcurveto{\pgfqpoint{2.352302in}{2.442397in}}{\pgfqpoint{2.364793in}{2.447571in}}{\pgfqpoint{2.374002in}{2.456779in}}%
\pgfpathcurveto{\pgfqpoint{2.383210in}{2.465988in}}{\pgfqpoint{2.388384in}{2.478479in}}{\pgfqpoint{2.388384in}{2.491502in}}%
\pgfpathcurveto{\pgfqpoint{2.388384in}{2.504524in}}{\pgfqpoint{2.383210in}{2.517015in}}{\pgfqpoint{2.374002in}{2.526224in}}%
\pgfpathcurveto{\pgfqpoint{2.364793in}{2.535432in}}{\pgfqpoint{2.352302in}{2.540606in}}{\pgfqpoint{2.339279in}{2.540606in}}%
\pgfpathcurveto{\pgfqpoint{2.326257in}{2.540606in}}{\pgfqpoint{2.313766in}{2.535432in}}{\pgfqpoint{2.304557in}{2.526224in}}%
\pgfpathcurveto{\pgfqpoint{2.295349in}{2.517015in}}{\pgfqpoint{2.290175in}{2.504524in}}{\pgfqpoint{2.290175in}{2.491502in}}%
\pgfpathcurveto{\pgfqpoint{2.290175in}{2.478479in}}{\pgfqpoint{2.295349in}{2.465988in}}{\pgfqpoint{2.304557in}{2.456779in}}%
\pgfpathcurveto{\pgfqpoint{2.313766in}{2.447571in}}{\pgfqpoint{2.326257in}{2.442397in}}{\pgfqpoint{2.339279in}{2.442397in}}%
\pgfpathlineto{\pgfqpoint{2.339279in}{2.442397in}}%
\pgfpathclose%
\pgfusepath{stroke,fill}%
\end{pgfscope}%
\begin{pgfscope}%
\pgfpathrectangle{\pgfqpoint{0.786164in}{0.768110in}}{\pgfqpoint{8.851069in}{7.081890in}}%
\pgfusepath{clip}%
\pgfsetbuttcap%
\pgfsetroundjoin%
\definecolor{currentfill}{rgb}{0.212395,0.359683,0.551710}%
\pgfsetfillcolor{currentfill}%
\pgfsetfillopacity{0.700000}%
\pgfsetlinewidth{0.501875pt}%
\definecolor{currentstroke}{rgb}{1.000000,1.000000,1.000000}%
\pgfsetstrokecolor{currentstroke}%
\pgfsetstrokeopacity{0.700000}%
\pgfsetdash{}{0pt}%
\pgfpathmoveto{\pgfqpoint{2.476279in}{2.661379in}}%
\pgfpathcurveto{\pgfqpoint{2.489301in}{2.661379in}}{\pgfqpoint{2.501793in}{2.666553in}}{\pgfqpoint{2.511001in}{2.675762in}}%
\pgfpathcurveto{\pgfqpoint{2.520209in}{2.684970in}}{\pgfqpoint{2.525383in}{2.697461in}}{\pgfqpoint{2.525383in}{2.710484in}}%
\pgfpathcurveto{\pgfqpoint{2.525383in}{2.723507in}}{\pgfqpoint{2.520209in}{2.735998in}}{\pgfqpoint{2.511001in}{2.745206in}}%
\pgfpathcurveto{\pgfqpoint{2.501793in}{2.754415in}}{\pgfqpoint{2.489301in}{2.759589in}}{\pgfqpoint{2.476279in}{2.759589in}}%
\pgfpathcurveto{\pgfqpoint{2.463256in}{2.759589in}}{\pgfqpoint{2.450765in}{2.754415in}}{\pgfqpoint{2.441557in}{2.745206in}}%
\pgfpathcurveto{\pgfqpoint{2.432348in}{2.735998in}}{\pgfqpoint{2.427174in}{2.723507in}}{\pgfqpoint{2.427174in}{2.710484in}}%
\pgfpathcurveto{\pgfqpoint{2.427174in}{2.697461in}}{\pgfqpoint{2.432348in}{2.684970in}}{\pgfqpoint{2.441557in}{2.675762in}}%
\pgfpathcurveto{\pgfqpoint{2.450765in}{2.666553in}}{\pgfqpoint{2.463256in}{2.661379in}}{\pgfqpoint{2.476279in}{2.661379in}}%
\pgfpathlineto{\pgfqpoint{2.476279in}{2.661379in}}%
\pgfpathclose%
\pgfusepath{stroke,fill}%
\end{pgfscope}%
\begin{pgfscope}%
\pgfpathrectangle{\pgfqpoint{0.786164in}{0.768110in}}{\pgfqpoint{8.851069in}{7.081890in}}%
\pgfusepath{clip}%
\pgfsetbuttcap%
\pgfsetroundjoin%
\definecolor{currentfill}{rgb}{0.279566,0.067836,0.391917}%
\pgfsetfillcolor{currentfill}%
\pgfsetfillopacity{0.700000}%
\pgfsetlinewidth{0.501875pt}%
\definecolor{currentstroke}{rgb}{1.000000,1.000000,1.000000}%
\pgfsetstrokecolor{currentstroke}%
\pgfsetstrokeopacity{0.700000}%
\pgfsetdash{}{0pt}%
\pgfpathmoveto{\pgfqpoint{2.932943in}{3.975274in}}%
\pgfpathcurveto{\pgfqpoint{2.945966in}{3.975274in}}{\pgfqpoint{2.958457in}{3.980447in}}{\pgfqpoint{2.967665in}{3.989656in}}%
\pgfpathcurveto{\pgfqpoint{2.976874in}{3.998864in}}{\pgfqpoint{2.982048in}{4.011355in}}{\pgfqpoint{2.982048in}{4.024378in}}%
\pgfpathcurveto{\pgfqpoint{2.982048in}{4.037401in}}{\pgfqpoint{2.976874in}{4.049892in}}{\pgfqpoint{2.967665in}{4.059100in}}%
\pgfpathcurveto{\pgfqpoint{2.958457in}{4.068309in}}{\pgfqpoint{2.945966in}{4.073483in}}{\pgfqpoint{2.932943in}{4.073483in}}%
\pgfpathcurveto{\pgfqpoint{2.919920in}{4.073483in}}{\pgfqpoint{2.907429in}{4.068309in}}{\pgfqpoint{2.898221in}{4.059100in}}%
\pgfpathcurveto{\pgfqpoint{2.889012in}{4.049892in}}{\pgfqpoint{2.883838in}{4.037401in}}{\pgfqpoint{2.883838in}{4.024378in}}%
\pgfpathcurveto{\pgfqpoint{2.883838in}{4.011355in}}{\pgfqpoint{2.889012in}{3.998864in}}{\pgfqpoint{2.898221in}{3.989656in}}%
\pgfpathcurveto{\pgfqpoint{2.907429in}{3.980447in}}{\pgfqpoint{2.919920in}{3.975274in}}{\pgfqpoint{2.932943in}{3.975274in}}%
\pgfpathlineto{\pgfqpoint{2.932943in}{3.975274in}}%
\pgfpathclose%
\pgfusepath{stroke,fill}%
\end{pgfscope}%
\begin{pgfscope}%
\pgfpathrectangle{\pgfqpoint{0.786164in}{0.768110in}}{\pgfqpoint{8.851069in}{7.081890in}}%
\pgfusepath{clip}%
\pgfsetbuttcap%
\pgfsetroundjoin%
\definecolor{currentfill}{rgb}{0.279566,0.067836,0.391917}%
\pgfsetfillcolor{currentfill}%
\pgfsetfillopacity{0.700000}%
\pgfsetlinewidth{0.501875pt}%
\definecolor{currentstroke}{rgb}{1.000000,1.000000,1.000000}%
\pgfsetstrokecolor{currentstroke}%
\pgfsetstrokeopacity{0.700000}%
\pgfsetdash{}{0pt}%
\pgfpathmoveto{\pgfqpoint{2.942076in}{3.931477in}}%
\pgfpathcurveto{\pgfqpoint{2.955099in}{3.931477in}}{\pgfqpoint{2.967590in}{3.936651in}}{\pgfqpoint{2.976799in}{3.945859in}}%
\pgfpathcurveto{\pgfqpoint{2.986007in}{3.955068in}}{\pgfqpoint{2.991181in}{3.967559in}}{\pgfqpoint{2.991181in}{3.980582in}}%
\pgfpathcurveto{\pgfqpoint{2.991181in}{3.993604in}}{\pgfqpoint{2.986007in}{4.006095in}}{\pgfqpoint{2.976799in}{4.015304in}}%
\pgfpathcurveto{\pgfqpoint{2.967590in}{4.024512in}}{\pgfqpoint{2.955099in}{4.029686in}}{\pgfqpoint{2.942076in}{4.029686in}}%
\pgfpathcurveto{\pgfqpoint{2.929054in}{4.029686in}}{\pgfqpoint{2.916563in}{4.024512in}}{\pgfqpoint{2.907354in}{4.015304in}}%
\pgfpathcurveto{\pgfqpoint{2.898146in}{4.006095in}}{\pgfqpoint{2.892972in}{3.993604in}}{\pgfqpoint{2.892972in}{3.980582in}}%
\pgfpathcurveto{\pgfqpoint{2.892972in}{3.967559in}}{\pgfqpoint{2.898146in}{3.955068in}}{\pgfqpoint{2.907354in}{3.945859in}}%
\pgfpathcurveto{\pgfqpoint{2.916563in}{3.936651in}}{\pgfqpoint{2.929054in}{3.931477in}}{\pgfqpoint{2.942076in}{3.931477in}}%
\pgfpathlineto{\pgfqpoint{2.942076in}{3.931477in}}%
\pgfpathclose%
\pgfusepath{stroke,fill}%
\end{pgfscope}%
\begin{pgfscope}%
\pgfpathrectangle{\pgfqpoint{0.786164in}{0.768110in}}{\pgfqpoint{8.851069in}{7.081890in}}%
\pgfusepath{clip}%
\pgfsetbuttcap%
\pgfsetroundjoin%
\definecolor{currentfill}{rgb}{0.280267,0.073417,0.397163}%
\pgfsetfillcolor{currentfill}%
\pgfsetfillopacity{0.700000}%
\pgfsetlinewidth{0.501875pt}%
\definecolor{currentstroke}{rgb}{1.000000,1.000000,1.000000}%
\pgfsetstrokecolor{currentstroke}%
\pgfsetstrokeopacity{0.700000}%
\pgfsetdash{}{0pt}%
\pgfpathmoveto{\pgfqpoint{2.996876in}{3.975274in}}%
\pgfpathcurveto{\pgfqpoint{3.009899in}{3.975274in}}{\pgfqpoint{3.022390in}{3.980447in}}{\pgfqpoint{3.031598in}{3.989656in}}%
\pgfpathcurveto{\pgfqpoint{3.040807in}{3.998864in}}{\pgfqpoint{3.045981in}{4.011355in}}{\pgfqpoint{3.045981in}{4.024378in}}%
\pgfpathcurveto{\pgfqpoint{3.045981in}{4.037401in}}{\pgfqpoint{3.040807in}{4.049892in}}{\pgfqpoint{3.031598in}{4.059100in}}%
\pgfpathcurveto{\pgfqpoint{3.022390in}{4.068309in}}{\pgfqpoint{3.009899in}{4.073483in}}{\pgfqpoint{2.996876in}{4.073483in}}%
\pgfpathcurveto{\pgfqpoint{2.983853in}{4.073483in}}{\pgfqpoint{2.971362in}{4.068309in}}{\pgfqpoint{2.962154in}{4.059100in}}%
\pgfpathcurveto{\pgfqpoint{2.952945in}{4.049892in}}{\pgfqpoint{2.947771in}{4.037401in}}{\pgfqpoint{2.947771in}{4.024378in}}%
\pgfpathcurveto{\pgfqpoint{2.947771in}{4.011355in}}{\pgfqpoint{2.952945in}{3.998864in}}{\pgfqpoint{2.962154in}{3.989656in}}%
\pgfpathcurveto{\pgfqpoint{2.971362in}{3.980447in}}{\pgfqpoint{2.983853in}{3.975274in}}{\pgfqpoint{2.996876in}{3.975274in}}%
\pgfpathlineto{\pgfqpoint{2.996876in}{3.975274in}}%
\pgfpathclose%
\pgfusepath{stroke,fill}%
\end{pgfscope}%
\begin{pgfscope}%
\pgfpathrectangle{\pgfqpoint{0.786164in}{0.768110in}}{\pgfqpoint{8.851069in}{7.081890in}}%
\pgfusepath{clip}%
\pgfsetbuttcap%
\pgfsetroundjoin%
\definecolor{currentfill}{rgb}{0.281924,0.089666,0.412415}%
\pgfsetfillcolor{currentfill}%
\pgfsetfillopacity{0.700000}%
\pgfsetlinewidth{0.501875pt}%
\definecolor{currentstroke}{rgb}{1.000000,1.000000,1.000000}%
\pgfsetstrokecolor{currentstroke}%
\pgfsetstrokeopacity{0.700000}%
\pgfsetdash{}{0pt}%
\pgfpathmoveto{\pgfqpoint{3.088209in}{4.019070in}}%
\pgfpathcurveto{\pgfqpoint{3.101232in}{4.019070in}}{\pgfqpoint{3.113723in}{4.024244in}}{\pgfqpoint{3.122931in}{4.033452in}}%
\pgfpathcurveto{\pgfqpoint{3.132140in}{4.042661in}}{\pgfqpoint{3.137314in}{4.055152in}}{\pgfqpoint{3.137314in}{4.068175in}}%
\pgfpathcurveto{\pgfqpoint{3.137314in}{4.081197in}}{\pgfqpoint{3.132140in}{4.093688in}}{\pgfqpoint{3.122931in}{4.102897in}}%
\pgfpathcurveto{\pgfqpoint{3.113723in}{4.112105in}}{\pgfqpoint{3.101232in}{4.117279in}}{\pgfqpoint{3.088209in}{4.117279in}}%
\pgfpathcurveto{\pgfqpoint{3.075186in}{4.117279in}}{\pgfqpoint{3.062695in}{4.112105in}}{\pgfqpoint{3.053487in}{4.102897in}}%
\pgfpathcurveto{\pgfqpoint{3.044278in}{4.093688in}}{\pgfqpoint{3.039104in}{4.081197in}}{\pgfqpoint{3.039104in}{4.068175in}}%
\pgfpathcurveto{\pgfqpoint{3.039104in}{4.055152in}}{\pgfqpoint{3.044278in}{4.042661in}}{\pgfqpoint{3.053487in}{4.033452in}}%
\pgfpathcurveto{\pgfqpoint{3.062695in}{4.024244in}}{\pgfqpoint{3.075186in}{4.019070in}}{\pgfqpoint{3.088209in}{4.019070in}}%
\pgfpathlineto{\pgfqpoint{3.088209in}{4.019070in}}%
\pgfpathclose%
\pgfusepath{stroke,fill}%
\end{pgfscope}%
\begin{pgfscope}%
\pgfpathrectangle{\pgfqpoint{0.786164in}{0.768110in}}{\pgfqpoint{8.851069in}{7.081890in}}%
\pgfusepath{clip}%
\pgfsetbuttcap%
\pgfsetroundjoin%
\definecolor{currentfill}{rgb}{0.283091,0.110553,0.431554}%
\pgfsetfillcolor{currentfill}%
\pgfsetfillopacity{0.700000}%
\pgfsetlinewidth{0.501875pt}%
\definecolor{currentstroke}{rgb}{1.000000,1.000000,1.000000}%
\pgfsetstrokecolor{currentstroke}%
\pgfsetstrokeopacity{0.700000}%
\pgfsetdash{}{0pt}%
\pgfpathmoveto{\pgfqpoint{3.179542in}{3.975274in}}%
\pgfpathcurveto{\pgfqpoint{3.192565in}{3.975274in}}{\pgfqpoint{3.205056in}{3.980447in}}{\pgfqpoint{3.214264in}{3.989656in}}%
\pgfpathcurveto{\pgfqpoint{3.223473in}{3.998864in}}{\pgfqpoint{3.228647in}{4.011355in}}{\pgfqpoint{3.228647in}{4.024378in}}%
\pgfpathcurveto{\pgfqpoint{3.228647in}{4.037401in}}{\pgfqpoint{3.223473in}{4.049892in}}{\pgfqpoint{3.214264in}{4.059100in}}%
\pgfpathcurveto{\pgfqpoint{3.205056in}{4.068309in}}{\pgfqpoint{3.192565in}{4.073483in}}{\pgfqpoint{3.179542in}{4.073483in}}%
\pgfpathcurveto{\pgfqpoint{3.166519in}{4.073483in}}{\pgfqpoint{3.154028in}{4.068309in}}{\pgfqpoint{3.144820in}{4.059100in}}%
\pgfpathcurveto{\pgfqpoint{3.135611in}{4.049892in}}{\pgfqpoint{3.130437in}{4.037401in}}{\pgfqpoint{3.130437in}{4.024378in}}%
\pgfpathcurveto{\pgfqpoint{3.130437in}{4.011355in}}{\pgfqpoint{3.135611in}{3.998864in}}{\pgfqpoint{3.144820in}{3.989656in}}%
\pgfpathcurveto{\pgfqpoint{3.154028in}{3.980447in}}{\pgfqpoint{3.166519in}{3.975274in}}{\pgfqpoint{3.179542in}{3.975274in}}%
\pgfpathlineto{\pgfqpoint{3.179542in}{3.975274in}}%
\pgfpathclose%
\pgfusepath{stroke,fill}%
\end{pgfscope}%
\begin{pgfscope}%
\pgfpathrectangle{\pgfqpoint{0.786164in}{0.768110in}}{\pgfqpoint{8.851069in}{7.081890in}}%
\pgfusepath{clip}%
\pgfsetbuttcap%
\pgfsetroundjoin%
\definecolor{currentfill}{rgb}{0.283187,0.125848,0.444960}%
\pgfsetfillcolor{currentfill}%
\pgfsetfillopacity{0.700000}%
\pgfsetlinewidth{0.501875pt}%
\definecolor{currentstroke}{rgb}{1.000000,1.000000,1.000000}%
\pgfsetstrokecolor{currentstroke}%
\pgfsetstrokeopacity{0.700000}%
\pgfsetdash{}{0pt}%
\pgfpathmoveto{\pgfqpoint{3.006009in}{3.821986in}}%
\pgfpathcurveto{\pgfqpoint{3.019032in}{3.821986in}}{\pgfqpoint{3.031523in}{3.827160in}}{\pgfqpoint{3.040732in}{3.836368in}}%
\pgfpathcurveto{\pgfqpoint{3.049940in}{3.845577in}}{\pgfqpoint{3.055114in}{3.858068in}}{\pgfqpoint{3.055114in}{3.871090in}}%
\pgfpathcurveto{\pgfqpoint{3.055114in}{3.884113in}}{\pgfqpoint{3.049940in}{3.896604in}}{\pgfqpoint{3.040732in}{3.905813in}}%
\pgfpathcurveto{\pgfqpoint{3.031523in}{3.915021in}}{\pgfqpoint{3.019032in}{3.920195in}}{\pgfqpoint{3.006009in}{3.920195in}}%
\pgfpathcurveto{\pgfqpoint{2.992987in}{3.920195in}}{\pgfqpoint{2.980496in}{3.915021in}}{\pgfqpoint{2.971287in}{3.905813in}}%
\pgfpathcurveto{\pgfqpoint{2.962079in}{3.896604in}}{\pgfqpoint{2.956905in}{3.884113in}}{\pgfqpoint{2.956905in}{3.871090in}}%
\pgfpathcurveto{\pgfqpoint{2.956905in}{3.858068in}}{\pgfqpoint{2.962079in}{3.845577in}}{\pgfqpoint{2.971287in}{3.836368in}}%
\pgfpathcurveto{\pgfqpoint{2.980496in}{3.827160in}}{\pgfqpoint{2.992987in}{3.821986in}}{\pgfqpoint{3.006009in}{3.821986in}}%
\pgfpathlineto{\pgfqpoint{3.006009in}{3.821986in}}%
\pgfpathclose%
\pgfusepath{stroke,fill}%
\end{pgfscope}%
\begin{pgfscope}%
\pgfpathrectangle{\pgfqpoint{0.786164in}{0.768110in}}{\pgfqpoint{8.851069in}{7.081890in}}%
\pgfusepath{clip}%
\pgfsetbuttcap%
\pgfsetroundjoin%
\definecolor{currentfill}{rgb}{0.283072,0.130895,0.449241}%
\pgfsetfillcolor{currentfill}%
\pgfsetfillopacity{0.700000}%
\pgfsetlinewidth{0.501875pt}%
\definecolor{currentstroke}{rgb}{1.000000,1.000000,1.000000}%
\pgfsetstrokecolor{currentstroke}%
\pgfsetstrokeopacity{0.700000}%
\pgfsetdash{}{0pt}%
\pgfpathmoveto{\pgfqpoint{2.814210in}{3.668698in}}%
\pgfpathcurveto{\pgfqpoint{2.827233in}{3.668698in}}{\pgfqpoint{2.839724in}{3.673872in}}{\pgfqpoint{2.848933in}{3.683081in}}%
\pgfpathcurveto{\pgfqpoint{2.858141in}{3.692289in}}{\pgfqpoint{2.863315in}{3.704780in}}{\pgfqpoint{2.863315in}{3.717803in}}%
\pgfpathcurveto{\pgfqpoint{2.863315in}{3.730826in}}{\pgfqpoint{2.858141in}{3.743317in}}{\pgfqpoint{2.848933in}{3.752525in}}%
\pgfpathcurveto{\pgfqpoint{2.839724in}{3.761733in}}{\pgfqpoint{2.827233in}{3.766907in}}{\pgfqpoint{2.814210in}{3.766907in}}%
\pgfpathcurveto{\pgfqpoint{2.801188in}{3.766907in}}{\pgfqpoint{2.788697in}{3.761733in}}{\pgfqpoint{2.779488in}{3.752525in}}%
\pgfpathcurveto{\pgfqpoint{2.770280in}{3.743317in}}{\pgfqpoint{2.765106in}{3.730826in}}{\pgfqpoint{2.765106in}{3.717803in}}%
\pgfpathcurveto{\pgfqpoint{2.765106in}{3.704780in}}{\pgfqpoint{2.770280in}{3.692289in}}{\pgfqpoint{2.779488in}{3.683081in}}%
\pgfpathcurveto{\pgfqpoint{2.788697in}{3.673872in}}{\pgfqpoint{2.801188in}{3.668698in}}{\pgfqpoint{2.814210in}{3.668698in}}%
\pgfpathlineto{\pgfqpoint{2.814210in}{3.668698in}}%
\pgfpathclose%
\pgfusepath{stroke,fill}%
\end{pgfscope}%
\begin{pgfscope}%
\pgfpathrectangle{\pgfqpoint{0.786164in}{0.768110in}}{\pgfqpoint{8.851069in}{7.081890in}}%
\pgfusepath{clip}%
\pgfsetbuttcap%
\pgfsetroundjoin%
\definecolor{currentfill}{rgb}{0.283072,0.130895,0.449241}%
\pgfsetfillcolor{currentfill}%
\pgfsetfillopacity{0.700000}%
\pgfsetlinewidth{0.501875pt}%
\definecolor{currentstroke}{rgb}{1.000000,1.000000,1.000000}%
\pgfsetstrokecolor{currentstroke}%
\pgfsetstrokeopacity{0.700000}%
\pgfsetdash{}{0pt}%
\pgfpathmoveto{\pgfqpoint{2.713744in}{3.515411in}}%
\pgfpathcurveto{\pgfqpoint{2.726767in}{3.515411in}}{\pgfqpoint{2.739258in}{3.520585in}}{\pgfqpoint{2.748466in}{3.529793in}}%
\pgfpathcurveto{\pgfqpoint{2.757675in}{3.539001in}}{\pgfqpoint{2.762849in}{3.551492in}}{\pgfqpoint{2.762849in}{3.564515in}}%
\pgfpathcurveto{\pgfqpoint{2.762849in}{3.577538in}}{\pgfqpoint{2.757675in}{3.590029in}}{\pgfqpoint{2.748466in}{3.599237in}}%
\pgfpathcurveto{\pgfqpoint{2.739258in}{3.608446in}}{\pgfqpoint{2.726767in}{3.613620in}}{\pgfqpoint{2.713744in}{3.613620in}}%
\pgfpathcurveto{\pgfqpoint{2.700722in}{3.613620in}}{\pgfqpoint{2.688230in}{3.608446in}}{\pgfqpoint{2.679022in}{3.599237in}}%
\pgfpathcurveto{\pgfqpoint{2.669814in}{3.590029in}}{\pgfqpoint{2.664640in}{3.577538in}}{\pgfqpoint{2.664640in}{3.564515in}}%
\pgfpathcurveto{\pgfqpoint{2.664640in}{3.551492in}}{\pgfqpoint{2.669814in}{3.539001in}}{\pgfqpoint{2.679022in}{3.529793in}}%
\pgfpathcurveto{\pgfqpoint{2.688230in}{3.520585in}}{\pgfqpoint{2.700722in}{3.515411in}}{\pgfqpoint{2.713744in}{3.515411in}}%
\pgfpathlineto{\pgfqpoint{2.713744in}{3.515411in}}%
\pgfpathclose%
\pgfusepath{stroke,fill}%
\end{pgfscope}%
\begin{pgfscope}%
\pgfpathrectangle{\pgfqpoint{0.786164in}{0.768110in}}{\pgfqpoint{8.851069in}{7.081890in}}%
\pgfusepath{clip}%
\pgfsetbuttcap%
\pgfsetroundjoin%
\definecolor{currentfill}{rgb}{0.281887,0.150881,0.465405}%
\pgfsetfillcolor{currentfill}%
\pgfsetfillopacity{0.700000}%
\pgfsetlinewidth{0.501875pt}%
\definecolor{currentstroke}{rgb}{1.000000,1.000000,1.000000}%
\pgfsetstrokecolor{currentstroke}%
\pgfsetstrokeopacity{0.700000}%
\pgfsetdash{}{0pt}%
\pgfpathmoveto{\pgfqpoint{2.512812in}{3.340225in}}%
\pgfpathcurveto{\pgfqpoint{2.525835in}{3.340225in}}{\pgfqpoint{2.538326in}{3.345399in}}{\pgfqpoint{2.547534in}{3.354607in}}%
\pgfpathcurveto{\pgfqpoint{2.556743in}{3.363816in}}{\pgfqpoint{2.561917in}{3.376307in}}{\pgfqpoint{2.561917in}{3.389329in}}%
\pgfpathcurveto{\pgfqpoint{2.561917in}{3.402352in}}{\pgfqpoint{2.556743in}{3.414843in}}{\pgfqpoint{2.547534in}{3.424052in}}%
\pgfpathcurveto{\pgfqpoint{2.538326in}{3.433260in}}{\pgfqpoint{2.525835in}{3.438434in}}{\pgfqpoint{2.512812in}{3.438434in}}%
\pgfpathcurveto{\pgfqpoint{2.499789in}{3.438434in}}{\pgfqpoint{2.487298in}{3.433260in}}{\pgfqpoint{2.478090in}{3.424052in}}%
\pgfpathcurveto{\pgfqpoint{2.468881in}{3.414843in}}{\pgfqpoint{2.463707in}{3.402352in}}{\pgfqpoint{2.463707in}{3.389329in}}%
\pgfpathcurveto{\pgfqpoint{2.463707in}{3.376307in}}{\pgfqpoint{2.468881in}{3.363816in}}{\pgfqpoint{2.478090in}{3.354607in}}%
\pgfpathcurveto{\pgfqpoint{2.487298in}{3.345399in}}{\pgfqpoint{2.499789in}{3.340225in}}{\pgfqpoint{2.512812in}{3.340225in}}%
\pgfpathlineto{\pgfqpoint{2.512812in}{3.340225in}}%
\pgfpathclose%
\pgfusepath{stroke,fill}%
\end{pgfscope}%
\begin{pgfscope}%
\pgfpathrectangle{\pgfqpoint{0.786164in}{0.768110in}}{\pgfqpoint{8.851069in}{7.081890in}}%
\pgfusepath{clip}%
\pgfsetbuttcap%
\pgfsetroundjoin%
\definecolor{currentfill}{rgb}{0.278826,0.175490,0.483397}%
\pgfsetfillcolor{currentfill}%
\pgfsetfillopacity{0.700000}%
\pgfsetlinewidth{0.501875pt}%
\definecolor{currentstroke}{rgb}{1.000000,1.000000,1.000000}%
\pgfsetstrokecolor{currentstroke}%
\pgfsetstrokeopacity{0.700000}%
\pgfsetdash{}{0pt}%
\pgfpathmoveto{\pgfqpoint{2.184014in}{3.165039in}}%
\pgfpathcurveto{\pgfqpoint{2.197036in}{3.165039in}}{\pgfqpoint{2.209527in}{3.170213in}}{\pgfqpoint{2.218736in}{3.179421in}}%
\pgfpathcurveto{\pgfqpoint{2.227944in}{3.188630in}}{\pgfqpoint{2.233118in}{3.201121in}}{\pgfqpoint{2.233118in}{3.214143in}}%
\pgfpathcurveto{\pgfqpoint{2.233118in}{3.227166in}}{\pgfqpoint{2.227944in}{3.239657in}}{\pgfqpoint{2.218736in}{3.248866in}}%
\pgfpathcurveto{\pgfqpoint{2.209527in}{3.258074in}}{\pgfqpoint{2.197036in}{3.263248in}}{\pgfqpoint{2.184014in}{3.263248in}}%
\pgfpathcurveto{\pgfqpoint{2.170991in}{3.263248in}}{\pgfqpoint{2.158500in}{3.258074in}}{\pgfqpoint{2.149291in}{3.248866in}}%
\pgfpathcurveto{\pgfqpoint{2.140083in}{3.239657in}}{\pgfqpoint{2.134909in}{3.227166in}}{\pgfqpoint{2.134909in}{3.214143in}}%
\pgfpathcurveto{\pgfqpoint{2.134909in}{3.201121in}}{\pgfqpoint{2.140083in}{3.188630in}}{\pgfqpoint{2.149291in}{3.179421in}}%
\pgfpathcurveto{\pgfqpoint{2.158500in}{3.170213in}}{\pgfqpoint{2.170991in}{3.165039in}}{\pgfqpoint{2.184014in}{3.165039in}}%
\pgfpathlineto{\pgfqpoint{2.184014in}{3.165039in}}%
\pgfpathclose%
\pgfusepath{stroke,fill}%
\end{pgfscope}%
\begin{pgfscope}%
\pgfpathrectangle{\pgfqpoint{0.786164in}{0.768110in}}{\pgfqpoint{8.851069in}{7.081890in}}%
\pgfusepath{clip}%
\pgfsetbuttcap%
\pgfsetroundjoin%
\definecolor{currentfill}{rgb}{0.276194,0.190074,0.493001}%
\pgfsetfillcolor{currentfill}%
\pgfsetfillopacity{0.700000}%
\pgfsetlinewidth{0.501875pt}%
\definecolor{currentstroke}{rgb}{1.000000,1.000000,1.000000}%
\pgfsetstrokecolor{currentstroke}%
\pgfsetstrokeopacity{0.700000}%
\pgfsetdash{}{0pt}%
\pgfpathmoveto{\pgfqpoint{2.247947in}{3.274530in}}%
\pgfpathcurveto{\pgfqpoint{2.260969in}{3.274530in}}{\pgfqpoint{2.273460in}{3.279704in}}{\pgfqpoint{2.282669in}{3.288912in}}%
\pgfpathcurveto{\pgfqpoint{2.291877in}{3.298121in}}{\pgfqpoint{2.297051in}{3.310612in}}{\pgfqpoint{2.297051in}{3.323635in}}%
\pgfpathcurveto{\pgfqpoint{2.297051in}{3.336657in}}{\pgfqpoint{2.291877in}{3.349148in}}{\pgfqpoint{2.282669in}{3.358357in}}%
\pgfpathcurveto{\pgfqpoint{2.273460in}{3.367565in}}{\pgfqpoint{2.260969in}{3.372739in}}{\pgfqpoint{2.247947in}{3.372739in}}%
\pgfpathcurveto{\pgfqpoint{2.234924in}{3.372739in}}{\pgfqpoint{2.222433in}{3.367565in}}{\pgfqpoint{2.213224in}{3.358357in}}%
\pgfpathcurveto{\pgfqpoint{2.204016in}{3.349148in}}{\pgfqpoint{2.198842in}{3.336657in}}{\pgfqpoint{2.198842in}{3.323635in}}%
\pgfpathcurveto{\pgfqpoint{2.198842in}{3.310612in}}{\pgfqpoint{2.204016in}{3.298121in}}{\pgfqpoint{2.213224in}{3.288912in}}%
\pgfpathcurveto{\pgfqpoint{2.222433in}{3.279704in}}{\pgfqpoint{2.234924in}{3.274530in}}{\pgfqpoint{2.247947in}{3.274530in}}%
\pgfpathlineto{\pgfqpoint{2.247947in}{3.274530in}}%
\pgfpathclose%
\pgfusepath{stroke,fill}%
\end{pgfscope}%
\begin{pgfscope}%
\pgfpathrectangle{\pgfqpoint{0.786164in}{0.768110in}}{\pgfqpoint{8.851069in}{7.081890in}}%
\pgfusepath{clip}%
\pgfsetbuttcap%
\pgfsetroundjoin%
\definecolor{currentfill}{rgb}{0.273006,0.204520,0.501721}%
\pgfsetfillcolor{currentfill}%
\pgfsetfillopacity{0.700000}%
\pgfsetlinewidth{0.501875pt}%
\definecolor{currentstroke}{rgb}{1.000000,1.000000,1.000000}%
\pgfsetstrokecolor{currentstroke}%
\pgfsetstrokeopacity{0.700000}%
\pgfsetdash{}{0pt}%
\pgfpathmoveto{\pgfqpoint{2.321013in}{3.296428in}}%
\pgfpathcurveto{\pgfqpoint{2.334036in}{3.296428in}}{\pgfqpoint{2.346527in}{3.301602in}}{\pgfqpoint{2.355735in}{3.310811in}}%
\pgfpathcurveto{\pgfqpoint{2.364944in}{3.320019in}}{\pgfqpoint{2.370117in}{3.332510in}}{\pgfqpoint{2.370117in}{3.345533in}}%
\pgfpathcurveto{\pgfqpoint{2.370117in}{3.358556in}}{\pgfqpoint{2.364944in}{3.371047in}}{\pgfqpoint{2.355735in}{3.380255in}}%
\pgfpathcurveto{\pgfqpoint{2.346527in}{3.389463in}}{\pgfqpoint{2.334036in}{3.394637in}}{\pgfqpoint{2.321013in}{3.394637in}}%
\pgfpathcurveto{\pgfqpoint{2.307990in}{3.394637in}}{\pgfqpoint{2.295499in}{3.389463in}}{\pgfqpoint{2.286291in}{3.380255in}}%
\pgfpathcurveto{\pgfqpoint{2.277082in}{3.371047in}}{\pgfqpoint{2.271908in}{3.358556in}}{\pgfqpoint{2.271908in}{3.345533in}}%
\pgfpathcurveto{\pgfqpoint{2.271908in}{3.332510in}}{\pgfqpoint{2.277082in}{3.320019in}}{\pgfqpoint{2.286291in}{3.310811in}}%
\pgfpathcurveto{\pgfqpoint{2.295499in}{3.301602in}}{\pgfqpoint{2.307990in}{3.296428in}}{\pgfqpoint{2.321013in}{3.296428in}}%
\pgfpathlineto{\pgfqpoint{2.321013in}{3.296428in}}%
\pgfpathclose%
\pgfusepath{stroke,fill}%
\end{pgfscope}%
\begin{pgfscope}%
\pgfpathrectangle{\pgfqpoint{0.786164in}{0.768110in}}{\pgfqpoint{8.851069in}{7.081890in}}%
\pgfusepath{clip}%
\pgfsetbuttcap%
\pgfsetroundjoin%
\definecolor{currentfill}{rgb}{0.274128,0.199721,0.498911}%
\pgfsetfillcolor{currentfill}%
\pgfsetfillopacity{0.700000}%
\pgfsetlinewidth{0.501875pt}%
\definecolor{currentstroke}{rgb}{1.000000,1.000000,1.000000}%
\pgfsetstrokecolor{currentstroke}%
\pgfsetstrokeopacity{0.700000}%
\pgfsetdash{}{0pt}%
\pgfpathmoveto{\pgfqpoint{2.467145in}{3.449716in}}%
\pgfpathcurveto{\pgfqpoint{2.480168in}{3.449716in}}{\pgfqpoint{2.492659in}{3.454890in}}{\pgfqpoint{2.501868in}{3.464098in}}%
\pgfpathcurveto{\pgfqpoint{2.511076in}{3.473307in}}{\pgfqpoint{2.516250in}{3.485798in}}{\pgfqpoint{2.516250in}{3.498820in}}%
\pgfpathcurveto{\pgfqpoint{2.516250in}{3.511843in}}{\pgfqpoint{2.511076in}{3.524334in}}{\pgfqpoint{2.501868in}{3.533543in}}%
\pgfpathcurveto{\pgfqpoint{2.492659in}{3.542751in}}{\pgfqpoint{2.480168in}{3.547925in}}{\pgfqpoint{2.467145in}{3.547925in}}%
\pgfpathcurveto{\pgfqpoint{2.454123in}{3.547925in}}{\pgfqpoint{2.441632in}{3.542751in}}{\pgfqpoint{2.432423in}{3.533543in}}%
\pgfpathcurveto{\pgfqpoint{2.423215in}{3.524334in}}{\pgfqpoint{2.418041in}{3.511843in}}{\pgfqpoint{2.418041in}{3.498820in}}%
\pgfpathcurveto{\pgfqpoint{2.418041in}{3.485798in}}{\pgfqpoint{2.423215in}{3.473307in}}{\pgfqpoint{2.432423in}{3.464098in}}%
\pgfpathcurveto{\pgfqpoint{2.441632in}{3.454890in}}{\pgfqpoint{2.454123in}{3.449716in}}{\pgfqpoint{2.467145in}{3.449716in}}%
\pgfpathlineto{\pgfqpoint{2.467145in}{3.449716in}}%
\pgfpathclose%
\pgfusepath{stroke,fill}%
\end{pgfscope}%
\begin{pgfscope}%
\pgfpathrectangle{\pgfqpoint{0.786164in}{0.768110in}}{\pgfqpoint{8.851069in}{7.081890in}}%
\pgfusepath{clip}%
\pgfsetbuttcap%
\pgfsetroundjoin%
\definecolor{currentfill}{rgb}{0.270595,0.214069,0.507052}%
\pgfsetfillcolor{currentfill}%
\pgfsetfillopacity{0.700000}%
\pgfsetlinewidth{0.501875pt}%
\definecolor{currentstroke}{rgb}{1.000000,1.000000,1.000000}%
\pgfsetstrokecolor{currentstroke}%
\pgfsetstrokeopacity{0.700000}%
\pgfsetdash{}{0pt}%
\pgfpathmoveto{\pgfqpoint{2.558478in}{3.471614in}}%
\pgfpathcurveto{\pgfqpoint{2.571501in}{3.471614in}}{\pgfqpoint{2.583992in}{3.476788in}}{\pgfqpoint{2.593201in}{3.485996in}}%
\pgfpathcurveto{\pgfqpoint{2.602409in}{3.495205in}}{\pgfqpoint{2.607583in}{3.507696in}}{\pgfqpoint{2.607583in}{3.520719in}}%
\pgfpathcurveto{\pgfqpoint{2.607583in}{3.533741in}}{\pgfqpoint{2.602409in}{3.546232in}}{\pgfqpoint{2.593201in}{3.555441in}}%
\pgfpathcurveto{\pgfqpoint{2.583992in}{3.564649in}}{\pgfqpoint{2.571501in}{3.569823in}}{\pgfqpoint{2.558478in}{3.569823in}}%
\pgfpathcurveto{\pgfqpoint{2.545456in}{3.569823in}}{\pgfqpoint{2.532965in}{3.564649in}}{\pgfqpoint{2.523756in}{3.555441in}}%
\pgfpathcurveto{\pgfqpoint{2.514548in}{3.546232in}}{\pgfqpoint{2.509374in}{3.533741in}}{\pgfqpoint{2.509374in}{3.520719in}}%
\pgfpathcurveto{\pgfqpoint{2.509374in}{3.507696in}}{\pgfqpoint{2.514548in}{3.495205in}}{\pgfqpoint{2.523756in}{3.485996in}}%
\pgfpathcurveto{\pgfqpoint{2.532965in}{3.476788in}}{\pgfqpoint{2.545456in}{3.471614in}}{\pgfqpoint{2.558478in}{3.471614in}}%
\pgfpathlineto{\pgfqpoint{2.558478in}{3.471614in}}%
\pgfpathclose%
\pgfusepath{stroke,fill}%
\end{pgfscope}%
\begin{pgfscope}%
\pgfpathrectangle{\pgfqpoint{0.786164in}{0.768110in}}{\pgfqpoint{8.851069in}{7.081890in}}%
\pgfusepath{clip}%
\pgfsetbuttcap%
\pgfsetroundjoin%
\definecolor{currentfill}{rgb}{0.250425,0.274290,0.533103}%
\pgfsetfillcolor{currentfill}%
\pgfsetfillopacity{0.700000}%
\pgfsetlinewidth{0.501875pt}%
\definecolor{currentstroke}{rgb}{1.000000,1.000000,1.000000}%
\pgfsetstrokecolor{currentstroke}%
\pgfsetstrokeopacity{0.700000}%
\pgfsetdash{}{0pt}%
\pgfpathmoveto{\pgfqpoint{2.549345in}{3.427818in}}%
\pgfpathcurveto{\pgfqpoint{2.562368in}{3.427818in}}{\pgfqpoint{2.574859in}{3.432992in}}{\pgfqpoint{2.584067in}{3.442200in}}%
\pgfpathcurveto{\pgfqpoint{2.593276in}{3.451408in}}{\pgfqpoint{2.598450in}{3.463900in}}{\pgfqpoint{2.598450in}{3.476922in}}%
\pgfpathcurveto{\pgfqpoint{2.598450in}{3.489945in}}{\pgfqpoint{2.593276in}{3.502436in}}{\pgfqpoint{2.584067in}{3.511644in}}%
\pgfpathcurveto{\pgfqpoint{2.574859in}{3.520853in}}{\pgfqpoint{2.562368in}{3.526027in}}{\pgfqpoint{2.549345in}{3.526027in}}%
\pgfpathcurveto{\pgfqpoint{2.536322in}{3.526027in}}{\pgfqpoint{2.523831in}{3.520853in}}{\pgfqpoint{2.514623in}{3.511644in}}%
\pgfpathcurveto{\pgfqpoint{2.505414in}{3.502436in}}{\pgfqpoint{2.500240in}{3.489945in}}{\pgfqpoint{2.500240in}{3.476922in}}%
\pgfpathcurveto{\pgfqpoint{2.500240in}{3.463900in}}{\pgfqpoint{2.505414in}{3.451408in}}{\pgfqpoint{2.514623in}{3.442200in}}%
\pgfpathcurveto{\pgfqpoint{2.523831in}{3.432992in}}{\pgfqpoint{2.536322in}{3.427818in}}{\pgfqpoint{2.549345in}{3.427818in}}%
\pgfpathlineto{\pgfqpoint{2.549345in}{3.427818in}}%
\pgfpathclose%
\pgfusepath{stroke,fill}%
\end{pgfscope}%
\begin{pgfscope}%
\pgfpathrectangle{\pgfqpoint{0.786164in}{0.768110in}}{\pgfqpoint{8.851069in}{7.081890in}}%
\pgfusepath{clip}%
\pgfsetbuttcap%
\pgfsetroundjoin%
\definecolor{currentfill}{rgb}{0.250425,0.274290,0.533103}%
\pgfsetfillcolor{currentfill}%
\pgfsetfillopacity{0.700000}%
\pgfsetlinewidth{0.501875pt}%
\definecolor{currentstroke}{rgb}{1.000000,1.000000,1.000000}%
\pgfsetstrokecolor{currentstroke}%
\pgfsetstrokeopacity{0.700000}%
\pgfsetdash{}{0pt}%
\pgfpathmoveto{\pgfqpoint{2.503679in}{3.405919in}}%
\pgfpathcurveto{\pgfqpoint{2.516701in}{3.405919in}}{\pgfqpoint{2.529192in}{3.411093in}}{\pgfqpoint{2.538401in}{3.420302in}}%
\pgfpathcurveto{\pgfqpoint{2.547609in}{3.429510in}}{\pgfqpoint{2.552783in}{3.442001in}}{\pgfqpoint{2.552783in}{3.455024in}}%
\pgfpathcurveto{\pgfqpoint{2.552783in}{3.468047in}}{\pgfqpoint{2.547609in}{3.480538in}}{\pgfqpoint{2.538401in}{3.489746in}}%
\pgfpathcurveto{\pgfqpoint{2.529192in}{3.498955in}}{\pgfqpoint{2.516701in}{3.504129in}}{\pgfqpoint{2.503679in}{3.504129in}}%
\pgfpathcurveto{\pgfqpoint{2.490656in}{3.504129in}}{\pgfqpoint{2.478165in}{3.498955in}}{\pgfqpoint{2.468956in}{3.489746in}}%
\pgfpathcurveto{\pgfqpoint{2.459748in}{3.480538in}}{\pgfqpoint{2.454574in}{3.468047in}}{\pgfqpoint{2.454574in}{3.455024in}}%
\pgfpathcurveto{\pgfqpoint{2.454574in}{3.442001in}}{\pgfqpoint{2.459748in}{3.429510in}}{\pgfqpoint{2.468956in}{3.420302in}}%
\pgfpathcurveto{\pgfqpoint{2.478165in}{3.411093in}}{\pgfqpoint{2.490656in}{3.405919in}}{\pgfqpoint{2.503679in}{3.405919in}}%
\pgfpathlineto{\pgfqpoint{2.503679in}{3.405919in}}%
\pgfpathclose%
\pgfusepath{stroke,fill}%
\end{pgfscope}%
\begin{pgfscope}%
\pgfpathrectangle{\pgfqpoint{0.786164in}{0.768110in}}{\pgfqpoint{8.851069in}{7.081890in}}%
\pgfusepath{clip}%
\pgfsetbuttcap%
\pgfsetroundjoin%
\definecolor{currentfill}{rgb}{0.227802,0.326594,0.546532}%
\pgfsetfillcolor{currentfill}%
\pgfsetfillopacity{0.700000}%
\pgfsetlinewidth{0.501875pt}%
\definecolor{currentstroke}{rgb}{1.000000,1.000000,1.000000}%
\pgfsetstrokecolor{currentstroke}%
\pgfsetstrokeopacity{0.700000}%
\pgfsetdash{}{0pt}%
\pgfpathmoveto{\pgfqpoint{2.129214in}{3.121242in}}%
\pgfpathcurveto{\pgfqpoint{2.142237in}{3.121242in}}{\pgfqpoint{2.154728in}{3.126416in}}{\pgfqpoint{2.163936in}{3.135625in}}%
\pgfpathcurveto{\pgfqpoint{2.173144in}{3.144833in}}{\pgfqpoint{2.178318in}{3.157324in}}{\pgfqpoint{2.178318in}{3.170347in}}%
\pgfpathcurveto{\pgfqpoint{2.178318in}{3.183370in}}{\pgfqpoint{2.173144in}{3.195861in}}{\pgfqpoint{2.163936in}{3.205069in}}%
\pgfpathcurveto{\pgfqpoint{2.154728in}{3.214278in}}{\pgfqpoint{2.142237in}{3.219452in}}{\pgfqpoint{2.129214in}{3.219452in}}%
\pgfpathcurveto{\pgfqpoint{2.116191in}{3.219452in}}{\pgfqpoint{2.103700in}{3.214278in}}{\pgfqpoint{2.094492in}{3.205069in}}%
\pgfpathcurveto{\pgfqpoint{2.085283in}{3.195861in}}{\pgfqpoint{2.080109in}{3.183370in}}{\pgfqpoint{2.080109in}{3.170347in}}%
\pgfpathcurveto{\pgfqpoint{2.080109in}{3.157324in}}{\pgfqpoint{2.085283in}{3.144833in}}{\pgfqpoint{2.094492in}{3.135625in}}%
\pgfpathcurveto{\pgfqpoint{2.103700in}{3.126416in}}{\pgfqpoint{2.116191in}{3.121242in}}{\pgfqpoint{2.129214in}{3.121242in}}%
\pgfpathlineto{\pgfqpoint{2.129214in}{3.121242in}}%
\pgfpathclose%
\pgfusepath{stroke,fill}%
\end{pgfscope}%
\begin{pgfscope}%
\pgfpathrectangle{\pgfqpoint{0.786164in}{0.768110in}}{\pgfqpoint{8.851069in}{7.081890in}}%
\pgfusepath{clip}%
\pgfsetbuttcap%
\pgfsetroundjoin%
\definecolor{currentfill}{rgb}{0.212395,0.359683,0.551710}%
\pgfsetfillcolor{currentfill}%
\pgfsetfillopacity{0.700000}%
\pgfsetlinewidth{0.501875pt}%
\definecolor{currentstroke}{rgb}{1.000000,1.000000,1.000000}%
\pgfsetstrokecolor{currentstroke}%
\pgfsetstrokeopacity{0.700000}%
\pgfsetdash{}{0pt}%
\pgfpathmoveto{\pgfqpoint{2.220547in}{3.230733in}}%
\pgfpathcurveto{\pgfqpoint{2.233569in}{3.230733in}}{\pgfqpoint{2.246060in}{3.235907in}}{\pgfqpoint{2.255269in}{3.245116in}}%
\pgfpathcurveto{\pgfqpoint{2.264477in}{3.254324in}}{\pgfqpoint{2.269651in}{3.266815in}}{\pgfqpoint{2.269651in}{3.279838in}}%
\pgfpathcurveto{\pgfqpoint{2.269651in}{3.292861in}}{\pgfqpoint{2.264477in}{3.305352in}}{\pgfqpoint{2.255269in}{3.314560in}}%
\pgfpathcurveto{\pgfqpoint{2.246060in}{3.323769in}}{\pgfqpoint{2.233569in}{3.328943in}}{\pgfqpoint{2.220547in}{3.328943in}}%
\pgfpathcurveto{\pgfqpoint{2.207524in}{3.328943in}}{\pgfqpoint{2.195033in}{3.323769in}}{\pgfqpoint{2.185824in}{3.314560in}}%
\pgfpathcurveto{\pgfqpoint{2.176616in}{3.305352in}}{\pgfqpoint{2.171442in}{3.292861in}}{\pgfqpoint{2.171442in}{3.279838in}}%
\pgfpathcurveto{\pgfqpoint{2.171442in}{3.266815in}}{\pgfqpoint{2.176616in}{3.254324in}}{\pgfqpoint{2.185824in}{3.245116in}}%
\pgfpathcurveto{\pgfqpoint{2.195033in}{3.235907in}}{\pgfqpoint{2.207524in}{3.230733in}}{\pgfqpoint{2.220547in}{3.230733in}}%
\pgfpathlineto{\pgfqpoint{2.220547in}{3.230733in}}%
\pgfpathclose%
\pgfusepath{stroke,fill}%
\end{pgfscope}%
\begin{pgfscope}%
\pgfpathrectangle{\pgfqpoint{0.786164in}{0.768110in}}{\pgfqpoint{8.851069in}{7.081890in}}%
\pgfusepath{clip}%
\pgfsetbuttcap%
\pgfsetroundjoin%
\definecolor{currentfill}{rgb}{0.208623,0.367752,0.552675}%
\pgfsetfillcolor{currentfill}%
\pgfsetfillopacity{0.700000}%
\pgfsetlinewidth{0.501875pt}%
\definecolor{currentstroke}{rgb}{1.000000,1.000000,1.000000}%
\pgfsetstrokecolor{currentstroke}%
\pgfsetstrokeopacity{0.700000}%
\pgfsetdash{}{0pt}%
\pgfpathmoveto{\pgfqpoint{2.357546in}{3.252632in}}%
\pgfpathcurveto{\pgfqpoint{2.370569in}{3.252632in}}{\pgfqpoint{2.383060in}{3.257806in}}{\pgfqpoint{2.392268in}{3.267014in}}%
\pgfpathcurveto{\pgfqpoint{2.401477in}{3.276223in}}{\pgfqpoint{2.406651in}{3.288714in}}{\pgfqpoint{2.406651in}{3.301736in}}%
\pgfpathcurveto{\pgfqpoint{2.406651in}{3.314759in}}{\pgfqpoint{2.401477in}{3.327250in}}{\pgfqpoint{2.392268in}{3.336459in}}%
\pgfpathcurveto{\pgfqpoint{2.383060in}{3.345667in}}{\pgfqpoint{2.370569in}{3.350841in}}{\pgfqpoint{2.357546in}{3.350841in}}%
\pgfpathcurveto{\pgfqpoint{2.344523in}{3.350841in}}{\pgfqpoint{2.332032in}{3.345667in}}{\pgfqpoint{2.322824in}{3.336459in}}%
\pgfpathcurveto{\pgfqpoint{2.313615in}{3.327250in}}{\pgfqpoint{2.308441in}{3.314759in}}{\pgfqpoint{2.308441in}{3.301736in}}%
\pgfpathcurveto{\pgfqpoint{2.308441in}{3.288714in}}{\pgfqpoint{2.313615in}{3.276223in}}{\pgfqpoint{2.322824in}{3.267014in}}%
\pgfpathcurveto{\pgfqpoint{2.332032in}{3.257806in}}{\pgfqpoint{2.344523in}{3.252632in}}{\pgfqpoint{2.357546in}{3.252632in}}%
\pgfpathlineto{\pgfqpoint{2.357546in}{3.252632in}}%
\pgfpathclose%
\pgfusepath{stroke,fill}%
\end{pgfscope}%
\begin{pgfscope}%
\pgfpathrectangle{\pgfqpoint{0.786164in}{0.768110in}}{\pgfqpoint{8.851069in}{7.081890in}}%
\pgfusepath{clip}%
\pgfsetbuttcap%
\pgfsetroundjoin%
\definecolor{currentfill}{rgb}{0.282623,0.140926,0.457517}%
\pgfsetfillcolor{currentfill}%
\pgfsetfillopacity{0.700000}%
\pgfsetlinewidth{0.501875pt}%
\definecolor{currentstroke}{rgb}{1.000000,1.000000,1.000000}%
\pgfsetstrokecolor{currentstroke}%
\pgfsetstrokeopacity{0.700000}%
\pgfsetdash{}{0pt}%
\pgfpathmoveto{\pgfqpoint{3.727539in}{4.128561in}}%
\pgfpathcurveto{\pgfqpoint{3.740562in}{4.128561in}}{\pgfqpoint{3.753053in}{4.133735in}}{\pgfqpoint{3.762261in}{4.142944in}}%
\pgfpathcurveto{\pgfqpoint{3.771470in}{4.152152in}}{\pgfqpoint{3.776644in}{4.164643in}}{\pgfqpoint{3.776644in}{4.177666in}}%
\pgfpathcurveto{\pgfqpoint{3.776644in}{4.190688in}}{\pgfqpoint{3.771470in}{4.203180in}}{\pgfqpoint{3.762261in}{4.212388in}}%
\pgfpathcurveto{\pgfqpoint{3.753053in}{4.221596in}}{\pgfqpoint{3.740562in}{4.226770in}}{\pgfqpoint{3.727539in}{4.226770in}}%
\pgfpathcurveto{\pgfqpoint{3.714516in}{4.226770in}}{\pgfqpoint{3.702025in}{4.221596in}}{\pgfqpoint{3.692817in}{4.212388in}}%
\pgfpathcurveto{\pgfqpoint{3.683608in}{4.203180in}}{\pgfqpoint{3.678434in}{4.190688in}}{\pgfqpoint{3.678434in}{4.177666in}}%
\pgfpathcurveto{\pgfqpoint{3.678434in}{4.164643in}}{\pgfqpoint{3.683608in}{4.152152in}}{\pgfqpoint{3.692817in}{4.142944in}}%
\pgfpathcurveto{\pgfqpoint{3.702025in}{4.133735in}}{\pgfqpoint{3.714516in}{4.128561in}}{\pgfqpoint{3.727539in}{4.128561in}}%
\pgfpathlineto{\pgfqpoint{3.727539in}{4.128561in}}%
\pgfpathclose%
\pgfusepath{stroke,fill}%
\end{pgfscope}%
\begin{pgfscope}%
\pgfpathrectangle{\pgfqpoint{0.786164in}{0.768110in}}{\pgfqpoint{8.851069in}{7.081890in}}%
\pgfusepath{clip}%
\pgfsetbuttcap%
\pgfsetroundjoin%
\definecolor{currentfill}{rgb}{0.281887,0.150881,0.465405}%
\pgfsetfillcolor{currentfill}%
\pgfsetfillopacity{0.700000}%
\pgfsetlinewidth{0.501875pt}%
\definecolor{currentstroke}{rgb}{1.000000,1.000000,1.000000}%
\pgfsetstrokecolor{currentstroke}%
\pgfsetstrokeopacity{0.700000}%
\pgfsetdash{}{0pt}%
\pgfpathmoveto{\pgfqpoint{3.672739in}{4.084765in}}%
\pgfpathcurveto{\pgfqpoint{3.685762in}{4.084765in}}{\pgfqpoint{3.698253in}{4.089939in}}{\pgfqpoint{3.707462in}{4.099147in}}%
\pgfpathcurveto{\pgfqpoint{3.716670in}{4.108356in}}{\pgfqpoint{3.721844in}{4.120847in}}{\pgfqpoint{3.721844in}{4.133869in}}%
\pgfpathcurveto{\pgfqpoint{3.721844in}{4.146892in}}{\pgfqpoint{3.716670in}{4.159383in}}{\pgfqpoint{3.707462in}{4.168592in}}%
\pgfpathcurveto{\pgfqpoint{3.698253in}{4.177800in}}{\pgfqpoint{3.685762in}{4.182974in}}{\pgfqpoint{3.672739in}{4.182974in}}%
\pgfpathcurveto{\pgfqpoint{3.659717in}{4.182974in}}{\pgfqpoint{3.647226in}{4.177800in}}{\pgfqpoint{3.638017in}{4.168592in}}%
\pgfpathcurveto{\pgfqpoint{3.628809in}{4.159383in}}{\pgfqpoint{3.623635in}{4.146892in}}{\pgfqpoint{3.623635in}{4.133869in}}%
\pgfpathcurveto{\pgfqpoint{3.623635in}{4.120847in}}{\pgfqpoint{3.628809in}{4.108356in}}{\pgfqpoint{3.638017in}{4.099147in}}%
\pgfpathcurveto{\pgfqpoint{3.647226in}{4.089939in}}{\pgfqpoint{3.659717in}{4.084765in}}{\pgfqpoint{3.672739in}{4.084765in}}%
\pgfpathlineto{\pgfqpoint{3.672739in}{4.084765in}}%
\pgfpathclose%
\pgfusepath{stroke,fill}%
\end{pgfscope}%
\begin{pgfscope}%
\pgfpathrectangle{\pgfqpoint{0.786164in}{0.768110in}}{\pgfqpoint{8.851069in}{7.081890in}}%
\pgfusepath{clip}%
\pgfsetbuttcap%
\pgfsetroundjoin%
\definecolor{currentfill}{rgb}{0.281412,0.155834,0.469201}%
\pgfsetfillcolor{currentfill}%
\pgfsetfillopacity{0.700000}%
\pgfsetlinewidth{0.501875pt}%
\definecolor{currentstroke}{rgb}{1.000000,1.000000,1.000000}%
\pgfsetstrokecolor{currentstroke}%
\pgfsetstrokeopacity{0.700000}%
\pgfsetdash{}{0pt}%
\pgfpathmoveto{\pgfqpoint{3.754939in}{4.128561in}}%
\pgfpathcurveto{\pgfqpoint{3.767962in}{4.128561in}}{\pgfqpoint{3.780453in}{4.133735in}}{\pgfqpoint{3.789661in}{4.142944in}}%
\pgfpathcurveto{\pgfqpoint{3.798870in}{4.152152in}}{\pgfqpoint{3.804044in}{4.164643in}}{\pgfqpoint{3.804044in}{4.177666in}}%
\pgfpathcurveto{\pgfqpoint{3.804044in}{4.190688in}}{\pgfqpoint{3.798870in}{4.203180in}}{\pgfqpoint{3.789661in}{4.212388in}}%
\pgfpathcurveto{\pgfqpoint{3.780453in}{4.221596in}}{\pgfqpoint{3.767962in}{4.226770in}}{\pgfqpoint{3.754939in}{4.226770in}}%
\pgfpathcurveto{\pgfqpoint{3.741916in}{4.226770in}}{\pgfqpoint{3.729425in}{4.221596in}}{\pgfqpoint{3.720217in}{4.212388in}}%
\pgfpathcurveto{\pgfqpoint{3.711008in}{4.203180in}}{\pgfqpoint{3.705834in}{4.190688in}}{\pgfqpoint{3.705834in}{4.177666in}}%
\pgfpathcurveto{\pgfqpoint{3.705834in}{4.164643in}}{\pgfqpoint{3.711008in}{4.152152in}}{\pgfqpoint{3.720217in}{4.142944in}}%
\pgfpathcurveto{\pgfqpoint{3.729425in}{4.133735in}}{\pgfqpoint{3.741916in}{4.128561in}}{\pgfqpoint{3.754939in}{4.128561in}}%
\pgfpathlineto{\pgfqpoint{3.754939in}{4.128561in}}%
\pgfpathclose%
\pgfusepath{stroke,fill}%
\end{pgfscope}%
\begin{pgfscope}%
\pgfpathrectangle{\pgfqpoint{0.786164in}{0.768110in}}{\pgfqpoint{8.851069in}{7.081890in}}%
\pgfusepath{clip}%
\pgfsetbuttcap%
\pgfsetroundjoin%
\definecolor{currentfill}{rgb}{0.280255,0.165693,0.476498}%
\pgfsetfillcolor{currentfill}%
\pgfsetfillopacity{0.700000}%
\pgfsetlinewidth{0.501875pt}%
\definecolor{currentstroke}{rgb}{1.000000,1.000000,1.000000}%
\pgfsetstrokecolor{currentstroke}%
\pgfsetstrokeopacity{0.700000}%
\pgfsetdash{}{0pt}%
\pgfpathmoveto{\pgfqpoint{3.745806in}{4.150459in}}%
\pgfpathcurveto{\pgfqpoint{3.758828in}{4.150459in}}{\pgfqpoint{3.771319in}{4.155633in}}{\pgfqpoint{3.780528in}{4.164842in}}%
\pgfpathcurveto{\pgfqpoint{3.789736in}{4.174050in}}{\pgfqpoint{3.794910in}{4.186541in}}{\pgfqpoint{3.794910in}{4.199564in}}%
\pgfpathcurveto{\pgfqpoint{3.794910in}{4.212587in}}{\pgfqpoint{3.789736in}{4.225078in}}{\pgfqpoint{3.780528in}{4.234286in}}%
\pgfpathcurveto{\pgfqpoint{3.771319in}{4.243495in}}{\pgfqpoint{3.758828in}{4.248669in}}{\pgfqpoint{3.745806in}{4.248669in}}%
\pgfpathcurveto{\pgfqpoint{3.732783in}{4.248669in}}{\pgfqpoint{3.720292in}{4.243495in}}{\pgfqpoint{3.711083in}{4.234286in}}%
\pgfpathcurveto{\pgfqpoint{3.701875in}{4.225078in}}{\pgfqpoint{3.696701in}{4.212587in}}{\pgfqpoint{3.696701in}{4.199564in}}%
\pgfpathcurveto{\pgfqpoint{3.696701in}{4.186541in}}{\pgfqpoint{3.701875in}{4.174050in}}{\pgfqpoint{3.711083in}{4.164842in}}%
\pgfpathcurveto{\pgfqpoint{3.720292in}{4.155633in}}{\pgfqpoint{3.732783in}{4.150459in}}{\pgfqpoint{3.745806in}{4.150459in}}%
\pgfpathlineto{\pgfqpoint{3.745806in}{4.150459in}}%
\pgfpathclose%
\pgfusepath{stroke,fill}%
\end{pgfscope}%
\begin{pgfscope}%
\pgfpathrectangle{\pgfqpoint{0.786164in}{0.768110in}}{\pgfqpoint{8.851069in}{7.081890in}}%
\pgfusepath{clip}%
\pgfsetbuttcap%
\pgfsetroundjoin%
\definecolor{currentfill}{rgb}{0.278012,0.180367,0.486697}%
\pgfsetfillcolor{currentfill}%
\pgfsetfillopacity{0.700000}%
\pgfsetlinewidth{0.501875pt}%
\definecolor{currentstroke}{rgb}{1.000000,1.000000,1.000000}%
\pgfsetstrokecolor{currentstroke}%
\pgfsetstrokeopacity{0.700000}%
\pgfsetdash{}{0pt}%
\pgfpathmoveto{\pgfqpoint{3.608806in}{3.997172in}}%
\pgfpathcurveto{\pgfqpoint{3.621829in}{3.997172in}}{\pgfqpoint{3.634320in}{4.002346in}}{\pgfqpoint{3.643529in}{4.011554in}}%
\pgfpathcurveto{\pgfqpoint{3.652737in}{4.020763in}}{\pgfqpoint{3.657911in}{4.033254in}}{\pgfqpoint{3.657911in}{4.046276in}}%
\pgfpathcurveto{\pgfqpoint{3.657911in}{4.059299in}}{\pgfqpoint{3.652737in}{4.071790in}}{\pgfqpoint{3.643529in}{4.080999in}}%
\pgfpathcurveto{\pgfqpoint{3.634320in}{4.090207in}}{\pgfqpoint{3.621829in}{4.095381in}}{\pgfqpoint{3.608806in}{4.095381in}}%
\pgfpathcurveto{\pgfqpoint{3.595784in}{4.095381in}}{\pgfqpoint{3.583293in}{4.090207in}}{\pgfqpoint{3.574084in}{4.080999in}}%
\pgfpathcurveto{\pgfqpoint{3.564876in}{4.071790in}}{\pgfqpoint{3.559702in}{4.059299in}}{\pgfqpoint{3.559702in}{4.046276in}}%
\pgfpathcurveto{\pgfqpoint{3.559702in}{4.033254in}}{\pgfqpoint{3.564876in}{4.020763in}}{\pgfqpoint{3.574084in}{4.011554in}}%
\pgfpathcurveto{\pgfqpoint{3.583293in}{4.002346in}}{\pgfqpoint{3.595784in}{3.997172in}}{\pgfqpoint{3.608806in}{3.997172in}}%
\pgfpathlineto{\pgfqpoint{3.608806in}{3.997172in}}%
\pgfpathclose%
\pgfusepath{stroke,fill}%
\end{pgfscope}%
\begin{pgfscope}%
\pgfpathrectangle{\pgfqpoint{0.786164in}{0.768110in}}{\pgfqpoint{8.851069in}{7.081890in}}%
\pgfusepath{clip}%
\pgfsetbuttcap%
\pgfsetroundjoin%
\definecolor{currentfill}{rgb}{0.273006,0.204520,0.501721}%
\pgfsetfillcolor{currentfill}%
\pgfsetfillopacity{0.700000}%
\pgfsetlinewidth{0.501875pt}%
\definecolor{currentstroke}{rgb}{1.000000,1.000000,1.000000}%
\pgfsetstrokecolor{currentstroke}%
\pgfsetstrokeopacity{0.700000}%
\pgfsetdash{}{0pt}%
\pgfpathmoveto{\pgfqpoint{3.389607in}{3.778189in}}%
\pgfpathcurveto{\pgfqpoint{3.402630in}{3.778189in}}{\pgfqpoint{3.415121in}{3.783363in}}{\pgfqpoint{3.424330in}{3.792572in}}%
\pgfpathcurveto{\pgfqpoint{3.433538in}{3.801780in}}{\pgfqpoint{3.438712in}{3.814271in}}{\pgfqpoint{3.438712in}{3.827294in}}%
\pgfpathcurveto{\pgfqpoint{3.438712in}{3.840317in}}{\pgfqpoint{3.433538in}{3.852808in}}{\pgfqpoint{3.424330in}{3.862016in}}%
\pgfpathcurveto{\pgfqpoint{3.415121in}{3.871225in}}{\pgfqpoint{3.402630in}{3.876399in}}{\pgfqpoint{3.389607in}{3.876399in}}%
\pgfpathcurveto{\pgfqpoint{3.376585in}{3.876399in}}{\pgfqpoint{3.364094in}{3.871225in}}{\pgfqpoint{3.354885in}{3.862016in}}%
\pgfpathcurveto{\pgfqpoint{3.345677in}{3.852808in}}{\pgfqpoint{3.340503in}{3.840317in}}{\pgfqpoint{3.340503in}{3.827294in}}%
\pgfpathcurveto{\pgfqpoint{3.340503in}{3.814271in}}{\pgfqpoint{3.345677in}{3.801780in}}{\pgfqpoint{3.354885in}{3.792572in}}%
\pgfpathcurveto{\pgfqpoint{3.364094in}{3.783363in}}{\pgfqpoint{3.376585in}{3.778189in}}{\pgfqpoint{3.389607in}{3.778189in}}%
\pgfpathlineto{\pgfqpoint{3.389607in}{3.778189in}}%
\pgfpathclose%
\pgfusepath{stroke,fill}%
\end{pgfscope}%
\begin{pgfscope}%
\pgfpathrectangle{\pgfqpoint{0.786164in}{0.768110in}}{\pgfqpoint{8.851069in}{7.081890in}}%
\pgfusepath{clip}%
\pgfsetbuttcap%
\pgfsetroundjoin%
\definecolor{currentfill}{rgb}{0.270595,0.214069,0.507052}%
\pgfsetfillcolor{currentfill}%
\pgfsetfillopacity{0.700000}%
\pgfsetlinewidth{0.501875pt}%
\definecolor{currentstroke}{rgb}{1.000000,1.000000,1.000000}%
\pgfsetstrokecolor{currentstroke}%
\pgfsetstrokeopacity{0.700000}%
\pgfsetdash{}{0pt}%
\pgfpathmoveto{\pgfqpoint{3.581407in}{3.843884in}}%
\pgfpathcurveto{\pgfqpoint{3.594429in}{3.843884in}}{\pgfqpoint{3.606920in}{3.849058in}}{\pgfqpoint{3.616129in}{3.858266in}}%
\pgfpathcurveto{\pgfqpoint{3.625337in}{3.867475in}}{\pgfqpoint{3.630511in}{3.879966in}}{\pgfqpoint{3.630511in}{3.892989in}}%
\pgfpathcurveto{\pgfqpoint{3.630511in}{3.906011in}}{\pgfqpoint{3.625337in}{3.918503in}}{\pgfqpoint{3.616129in}{3.927711in}}%
\pgfpathcurveto{\pgfqpoint{3.606920in}{3.936919in}}{\pgfqpoint{3.594429in}{3.942093in}}{\pgfqpoint{3.581407in}{3.942093in}}%
\pgfpathcurveto{\pgfqpoint{3.568384in}{3.942093in}}{\pgfqpoint{3.555893in}{3.936919in}}{\pgfqpoint{3.546684in}{3.927711in}}%
\pgfpathcurveto{\pgfqpoint{3.537476in}{3.918503in}}{\pgfqpoint{3.532302in}{3.906011in}}{\pgfqpoint{3.532302in}{3.892989in}}%
\pgfpathcurveto{\pgfqpoint{3.532302in}{3.879966in}}{\pgfqpoint{3.537476in}{3.867475in}}{\pgfqpoint{3.546684in}{3.858266in}}%
\pgfpathcurveto{\pgfqpoint{3.555893in}{3.849058in}}{\pgfqpoint{3.568384in}{3.843884in}}{\pgfqpoint{3.581407in}{3.843884in}}%
\pgfpathlineto{\pgfqpoint{3.581407in}{3.843884in}}%
\pgfpathclose%
\pgfusepath{stroke,fill}%
\end{pgfscope}%
\begin{pgfscope}%
\pgfpathrectangle{\pgfqpoint{0.786164in}{0.768110in}}{\pgfqpoint{8.851069in}{7.081890in}}%
\pgfusepath{clip}%
\pgfsetbuttcap%
\pgfsetroundjoin%
\definecolor{currentfill}{rgb}{0.267968,0.223549,0.512008}%
\pgfsetfillcolor{currentfill}%
\pgfsetfillopacity{0.700000}%
\pgfsetlinewidth{0.501875pt}%
\definecolor{currentstroke}{rgb}{1.000000,1.000000,1.000000}%
\pgfsetstrokecolor{currentstroke}%
\pgfsetstrokeopacity{0.700000}%
\pgfsetdash{}{0pt}%
\pgfpathmoveto{\pgfqpoint{3.426141in}{3.778189in}}%
\pgfpathcurveto{\pgfqpoint{3.439163in}{3.778189in}}{\pgfqpoint{3.451654in}{3.783363in}}{\pgfqpoint{3.460863in}{3.792572in}}%
\pgfpathcurveto{\pgfqpoint{3.470071in}{3.801780in}}{\pgfqpoint{3.475245in}{3.814271in}}{\pgfqpoint{3.475245in}{3.827294in}}%
\pgfpathcurveto{\pgfqpoint{3.475245in}{3.840317in}}{\pgfqpoint{3.470071in}{3.852808in}}{\pgfqpoint{3.460863in}{3.862016in}}%
\pgfpathcurveto{\pgfqpoint{3.451654in}{3.871225in}}{\pgfqpoint{3.439163in}{3.876399in}}{\pgfqpoint{3.426141in}{3.876399in}}%
\pgfpathcurveto{\pgfqpoint{3.413118in}{3.876399in}}{\pgfqpoint{3.400627in}{3.871225in}}{\pgfqpoint{3.391418in}{3.862016in}}%
\pgfpathcurveto{\pgfqpoint{3.382210in}{3.852808in}}{\pgfqpoint{3.377036in}{3.840317in}}{\pgfqpoint{3.377036in}{3.827294in}}%
\pgfpathcurveto{\pgfqpoint{3.377036in}{3.814271in}}{\pgfqpoint{3.382210in}{3.801780in}}{\pgfqpoint{3.391418in}{3.792572in}}%
\pgfpathcurveto{\pgfqpoint{3.400627in}{3.783363in}}{\pgfqpoint{3.413118in}{3.778189in}}{\pgfqpoint{3.426141in}{3.778189in}}%
\pgfpathlineto{\pgfqpoint{3.426141in}{3.778189in}}%
\pgfpathclose%
\pgfusepath{stroke,fill}%
\end{pgfscope}%
\begin{pgfscope}%
\pgfpathrectangle{\pgfqpoint{0.786164in}{0.768110in}}{\pgfqpoint{8.851069in}{7.081890in}}%
\pgfusepath{clip}%
\pgfsetbuttcap%
\pgfsetroundjoin%
\definecolor{currentfill}{rgb}{0.257322,0.256130,0.526563}%
\pgfsetfillcolor{currentfill}%
\pgfsetfillopacity{0.700000}%
\pgfsetlinewidth{0.501875pt}%
\definecolor{currentstroke}{rgb}{1.000000,1.000000,1.000000}%
\pgfsetstrokecolor{currentstroke}%
\pgfsetstrokeopacity{0.700000}%
\pgfsetdash{}{0pt}%
\pgfpathmoveto{\pgfqpoint{3.380474in}{3.690596in}}%
\pgfpathcurveto{\pgfqpoint{3.393497in}{3.690596in}}{\pgfqpoint{3.405988in}{3.695770in}}{\pgfqpoint{3.415196in}{3.704979in}}%
\pgfpathcurveto{\pgfqpoint{3.424405in}{3.714187in}}{\pgfqpoint{3.429579in}{3.726678in}}{\pgfqpoint{3.429579in}{3.739701in}}%
\pgfpathcurveto{\pgfqpoint{3.429579in}{3.752724in}}{\pgfqpoint{3.424405in}{3.765215in}}{\pgfqpoint{3.415196in}{3.774423in}}%
\pgfpathcurveto{\pgfqpoint{3.405988in}{3.783632in}}{\pgfqpoint{3.393497in}{3.788806in}}{\pgfqpoint{3.380474in}{3.788806in}}%
\pgfpathcurveto{\pgfqpoint{3.367451in}{3.788806in}}{\pgfqpoint{3.354960in}{3.783632in}}{\pgfqpoint{3.345752in}{3.774423in}}%
\pgfpathcurveto{\pgfqpoint{3.336544in}{3.765215in}}{\pgfqpoint{3.331370in}{3.752724in}}{\pgfqpoint{3.331370in}{3.739701in}}%
\pgfpathcurveto{\pgfqpoint{3.331370in}{3.726678in}}{\pgfqpoint{3.336544in}{3.714187in}}{\pgfqpoint{3.345752in}{3.704979in}}%
\pgfpathcurveto{\pgfqpoint{3.354960in}{3.695770in}}{\pgfqpoint{3.367451in}{3.690596in}}{\pgfqpoint{3.380474in}{3.690596in}}%
\pgfpathlineto{\pgfqpoint{3.380474in}{3.690596in}}%
\pgfpathclose%
\pgfusepath{stroke,fill}%
\end{pgfscope}%
\begin{pgfscope}%
\pgfpathrectangle{\pgfqpoint{0.786164in}{0.768110in}}{\pgfqpoint{8.851069in}{7.081890in}}%
\pgfusepath{clip}%
\pgfsetbuttcap%
\pgfsetroundjoin%
\definecolor{currentfill}{rgb}{0.250425,0.274290,0.533103}%
\pgfsetfillcolor{currentfill}%
\pgfsetfillopacity{0.700000}%
\pgfsetlinewidth{0.501875pt}%
\definecolor{currentstroke}{rgb}{1.000000,1.000000,1.000000}%
\pgfsetstrokecolor{currentstroke}%
\pgfsetstrokeopacity{0.700000}%
\pgfsetdash{}{0pt}%
\pgfpathmoveto{\pgfqpoint{3.407874in}{3.603003in}}%
\pgfpathcurveto{\pgfqpoint{3.420897in}{3.603003in}}{\pgfqpoint{3.433388in}{3.608177in}}{\pgfqpoint{3.442596in}{3.617386in}}%
\pgfpathcurveto{\pgfqpoint{3.451805in}{3.626594in}}{\pgfqpoint{3.456979in}{3.639085in}}{\pgfqpoint{3.456979in}{3.652108in}}%
\pgfpathcurveto{\pgfqpoint{3.456979in}{3.665131in}}{\pgfqpoint{3.451805in}{3.677622in}}{\pgfqpoint{3.442596in}{3.686830in}}%
\pgfpathcurveto{\pgfqpoint{3.433388in}{3.696039in}}{\pgfqpoint{3.420897in}{3.701213in}}{\pgfqpoint{3.407874in}{3.701213in}}%
\pgfpathcurveto{\pgfqpoint{3.394851in}{3.701213in}}{\pgfqpoint{3.382360in}{3.696039in}}{\pgfqpoint{3.373152in}{3.686830in}}%
\pgfpathcurveto{\pgfqpoint{3.363943in}{3.677622in}}{\pgfqpoint{3.358769in}{3.665131in}}{\pgfqpoint{3.358769in}{3.652108in}}%
\pgfpathcurveto{\pgfqpoint{3.358769in}{3.639085in}}{\pgfqpoint{3.363943in}{3.626594in}}{\pgfqpoint{3.373152in}{3.617386in}}%
\pgfpathcurveto{\pgfqpoint{3.382360in}{3.608177in}}{\pgfqpoint{3.394851in}{3.603003in}}{\pgfqpoint{3.407874in}{3.603003in}}%
\pgfpathlineto{\pgfqpoint{3.407874in}{3.603003in}}%
\pgfpathclose%
\pgfusepath{stroke,fill}%
\end{pgfscope}%
\begin{pgfscope}%
\pgfpathrectangle{\pgfqpoint{0.786164in}{0.768110in}}{\pgfqpoint{8.851069in}{7.081890in}}%
\pgfusepath{clip}%
\pgfsetbuttcap%
\pgfsetroundjoin%
\definecolor{currentfill}{rgb}{0.241237,0.296485,0.539709}%
\pgfsetfillcolor{currentfill}%
\pgfsetfillopacity{0.700000}%
\pgfsetlinewidth{0.501875pt}%
\definecolor{currentstroke}{rgb}{1.000000,1.000000,1.000000}%
\pgfsetstrokecolor{currentstroke}%
\pgfsetstrokeopacity{0.700000}%
\pgfsetdash{}{0pt}%
\pgfpathmoveto{\pgfqpoint{3.252608in}{3.537309in}}%
\pgfpathcurveto{\pgfqpoint{3.265631in}{3.537309in}}{\pgfqpoint{3.278122in}{3.542483in}}{\pgfqpoint{3.287330in}{3.551691in}}%
\pgfpathcurveto{\pgfqpoint{3.296539in}{3.560900in}}{\pgfqpoint{3.301713in}{3.573391in}}{\pgfqpoint{3.301713in}{3.586413in}}%
\pgfpathcurveto{\pgfqpoint{3.301713in}{3.599436in}}{\pgfqpoint{3.296539in}{3.611927in}}{\pgfqpoint{3.287330in}{3.621136in}}%
\pgfpathcurveto{\pgfqpoint{3.278122in}{3.630344in}}{\pgfqpoint{3.265631in}{3.635518in}}{\pgfqpoint{3.252608in}{3.635518in}}%
\pgfpathcurveto{\pgfqpoint{3.239585in}{3.635518in}}{\pgfqpoint{3.227094in}{3.630344in}}{\pgfqpoint{3.217886in}{3.621136in}}%
\pgfpathcurveto{\pgfqpoint{3.208678in}{3.611927in}}{\pgfqpoint{3.203504in}{3.599436in}}{\pgfqpoint{3.203504in}{3.586413in}}%
\pgfpathcurveto{\pgfqpoint{3.203504in}{3.573391in}}{\pgfqpoint{3.208678in}{3.560900in}}{\pgfqpoint{3.217886in}{3.551691in}}%
\pgfpathcurveto{\pgfqpoint{3.227094in}{3.542483in}}{\pgfqpoint{3.239585in}{3.537309in}}{\pgfqpoint{3.252608in}{3.537309in}}%
\pgfpathlineto{\pgfqpoint{3.252608in}{3.537309in}}%
\pgfpathclose%
\pgfusepath{stroke,fill}%
\end{pgfscope}%
\begin{pgfscope}%
\pgfpathrectangle{\pgfqpoint{0.786164in}{0.768110in}}{\pgfqpoint{8.851069in}{7.081890in}}%
\pgfusepath{clip}%
\pgfsetbuttcap%
\pgfsetroundjoin%
\definecolor{currentfill}{rgb}{0.241237,0.296485,0.539709}%
\pgfsetfillcolor{currentfill}%
\pgfsetfillopacity{0.700000}%
\pgfsetlinewidth{0.501875pt}%
\definecolor{currentstroke}{rgb}{1.000000,1.000000,1.000000}%
\pgfsetstrokecolor{currentstroke}%
\pgfsetstrokeopacity{0.700000}%
\pgfsetdash{}{0pt}%
\pgfpathmoveto{\pgfqpoint{3.289141in}{3.581105in}}%
\pgfpathcurveto{\pgfqpoint{3.302164in}{3.581105in}}{\pgfqpoint{3.314655in}{3.586279in}}{\pgfqpoint{3.323864in}{3.595488in}}%
\pgfpathcurveto{\pgfqpoint{3.333072in}{3.604696in}}{\pgfqpoint{3.338246in}{3.617187in}}{\pgfqpoint{3.338246in}{3.630210in}}%
\pgfpathcurveto{\pgfqpoint{3.338246in}{3.643233in}}{\pgfqpoint{3.333072in}{3.655724in}}{\pgfqpoint{3.323864in}{3.664932in}}%
\pgfpathcurveto{\pgfqpoint{3.314655in}{3.674141in}}{\pgfqpoint{3.302164in}{3.679315in}}{\pgfqpoint{3.289141in}{3.679315in}}%
\pgfpathcurveto{\pgfqpoint{3.276119in}{3.679315in}}{\pgfqpoint{3.263628in}{3.674141in}}{\pgfqpoint{3.254419in}{3.664932in}}%
\pgfpathcurveto{\pgfqpoint{3.245211in}{3.655724in}}{\pgfqpoint{3.240037in}{3.643233in}}{\pgfqpoint{3.240037in}{3.630210in}}%
\pgfpathcurveto{\pgfqpoint{3.240037in}{3.617187in}}{\pgfqpoint{3.245211in}{3.604696in}}{\pgfqpoint{3.254419in}{3.595488in}}%
\pgfpathcurveto{\pgfqpoint{3.263628in}{3.586279in}}{\pgfqpoint{3.276119in}{3.581105in}}{\pgfqpoint{3.289141in}{3.581105in}}%
\pgfpathlineto{\pgfqpoint{3.289141in}{3.581105in}}%
\pgfpathclose%
\pgfusepath{stroke,fill}%
\end{pgfscope}%
\begin{pgfscope}%
\pgfpathrectangle{\pgfqpoint{0.786164in}{0.768110in}}{\pgfqpoint{8.851069in}{7.081890in}}%
\pgfusepath{clip}%
\pgfsetbuttcap%
\pgfsetroundjoin%
\definecolor{currentfill}{rgb}{0.243113,0.292092,0.538516}%
\pgfsetfillcolor{currentfill}%
\pgfsetfillopacity{0.700000}%
\pgfsetlinewidth{0.501875pt}%
\definecolor{currentstroke}{rgb}{1.000000,1.000000,1.000000}%
\pgfsetstrokecolor{currentstroke}%
\pgfsetstrokeopacity{0.700000}%
\pgfsetdash{}{0pt}%
\pgfpathmoveto{\pgfqpoint{3.307408in}{3.624902in}}%
\pgfpathcurveto{\pgfqpoint{3.320431in}{3.624902in}}{\pgfqpoint{3.332922in}{3.630076in}}{\pgfqpoint{3.342130in}{3.639284in}}%
\pgfpathcurveto{\pgfqpoint{3.351339in}{3.648493in}}{\pgfqpoint{3.356513in}{3.660984in}}{\pgfqpoint{3.356513in}{3.674006in}}%
\pgfpathcurveto{\pgfqpoint{3.356513in}{3.687029in}}{\pgfqpoint{3.351339in}{3.699520in}}{\pgfqpoint{3.342130in}{3.708729in}}%
\pgfpathcurveto{\pgfqpoint{3.332922in}{3.717937in}}{\pgfqpoint{3.320431in}{3.723111in}}{\pgfqpoint{3.307408in}{3.723111in}}%
\pgfpathcurveto{\pgfqpoint{3.294385in}{3.723111in}}{\pgfqpoint{3.281894in}{3.717937in}}{\pgfqpoint{3.272686in}{3.708729in}}%
\pgfpathcurveto{\pgfqpoint{3.263477in}{3.699520in}}{\pgfqpoint{3.258303in}{3.687029in}}{\pgfqpoint{3.258303in}{3.674006in}}%
\pgfpathcurveto{\pgfqpoint{3.258303in}{3.660984in}}{\pgfqpoint{3.263477in}{3.648493in}}{\pgfqpoint{3.272686in}{3.639284in}}%
\pgfpathcurveto{\pgfqpoint{3.281894in}{3.630076in}}{\pgfqpoint{3.294385in}{3.624902in}}{\pgfqpoint{3.307408in}{3.624902in}}%
\pgfpathlineto{\pgfqpoint{3.307408in}{3.624902in}}%
\pgfpathclose%
\pgfusepath{stroke,fill}%
\end{pgfscope}%
\begin{pgfscope}%
\pgfpathrectangle{\pgfqpoint{0.786164in}{0.768110in}}{\pgfqpoint{8.851069in}{7.081890in}}%
\pgfusepath{clip}%
\pgfsetbuttcap%
\pgfsetroundjoin%
\definecolor{currentfill}{rgb}{0.243113,0.292092,0.538516}%
\pgfsetfillcolor{currentfill}%
\pgfsetfillopacity{0.700000}%
\pgfsetlinewidth{0.501875pt}%
\definecolor{currentstroke}{rgb}{1.000000,1.000000,1.000000}%
\pgfsetstrokecolor{currentstroke}%
\pgfsetstrokeopacity{0.700000}%
\pgfsetdash{}{0pt}%
\pgfpathmoveto{\pgfqpoint{3.353074in}{3.668698in}}%
\pgfpathcurveto{\pgfqpoint{3.366097in}{3.668698in}}{\pgfqpoint{3.378588in}{3.673872in}}{\pgfqpoint{3.387797in}{3.683081in}}%
\pgfpathcurveto{\pgfqpoint{3.397005in}{3.692289in}}{\pgfqpoint{3.402179in}{3.704780in}}{\pgfqpoint{3.402179in}{3.717803in}}%
\pgfpathcurveto{\pgfqpoint{3.402179in}{3.730826in}}{\pgfqpoint{3.397005in}{3.743317in}}{\pgfqpoint{3.387797in}{3.752525in}}%
\pgfpathcurveto{\pgfqpoint{3.378588in}{3.761733in}}{\pgfqpoint{3.366097in}{3.766907in}}{\pgfqpoint{3.353074in}{3.766907in}}%
\pgfpathcurveto{\pgfqpoint{3.340052in}{3.766907in}}{\pgfqpoint{3.327561in}{3.761733in}}{\pgfqpoint{3.318352in}{3.752525in}}%
\pgfpathcurveto{\pgfqpoint{3.309144in}{3.743317in}}{\pgfqpoint{3.303970in}{3.730826in}}{\pgfqpoint{3.303970in}{3.717803in}}%
\pgfpathcurveto{\pgfqpoint{3.303970in}{3.704780in}}{\pgfqpoint{3.309144in}{3.692289in}}{\pgfqpoint{3.318352in}{3.683081in}}%
\pgfpathcurveto{\pgfqpoint{3.327561in}{3.673872in}}{\pgfqpoint{3.340052in}{3.668698in}}{\pgfqpoint{3.353074in}{3.668698in}}%
\pgfpathlineto{\pgfqpoint{3.353074in}{3.668698in}}%
\pgfpathclose%
\pgfusepath{stroke,fill}%
\end{pgfscope}%
\begin{pgfscope}%
\pgfpathrectangle{\pgfqpoint{0.786164in}{0.768110in}}{\pgfqpoint{8.851069in}{7.081890in}}%
\pgfusepath{clip}%
\pgfsetbuttcap%
\pgfsetroundjoin%
\definecolor{currentfill}{rgb}{0.241237,0.296485,0.539709}%
\pgfsetfillcolor{currentfill}%
\pgfsetfillopacity{0.700000}%
\pgfsetlinewidth{0.501875pt}%
\definecolor{currentstroke}{rgb}{1.000000,1.000000,1.000000}%
\pgfsetstrokecolor{currentstroke}%
\pgfsetstrokeopacity{0.700000}%
\pgfsetdash{}{0pt}%
\pgfpathmoveto{\pgfqpoint{3.353074in}{3.734393in}}%
\pgfpathcurveto{\pgfqpoint{3.366097in}{3.734393in}}{\pgfqpoint{3.378588in}{3.739567in}}{\pgfqpoint{3.387797in}{3.748775in}}%
\pgfpathcurveto{\pgfqpoint{3.397005in}{3.757984in}}{\pgfqpoint{3.402179in}{3.770475in}}{\pgfqpoint{3.402179in}{3.783498in}}%
\pgfpathcurveto{\pgfqpoint{3.402179in}{3.796520in}}{\pgfqpoint{3.397005in}{3.809011in}}{\pgfqpoint{3.387797in}{3.818220in}}%
\pgfpathcurveto{\pgfqpoint{3.378588in}{3.827428in}}{\pgfqpoint{3.366097in}{3.832602in}}{\pgfqpoint{3.353074in}{3.832602in}}%
\pgfpathcurveto{\pgfqpoint{3.340052in}{3.832602in}}{\pgfqpoint{3.327561in}{3.827428in}}{\pgfqpoint{3.318352in}{3.818220in}}%
\pgfpathcurveto{\pgfqpoint{3.309144in}{3.809011in}}{\pgfqpoint{3.303970in}{3.796520in}}{\pgfqpoint{3.303970in}{3.783498in}}%
\pgfpathcurveto{\pgfqpoint{3.303970in}{3.770475in}}{\pgfqpoint{3.309144in}{3.757984in}}{\pgfqpoint{3.318352in}{3.748775in}}%
\pgfpathcurveto{\pgfqpoint{3.327561in}{3.739567in}}{\pgfqpoint{3.340052in}{3.734393in}}{\pgfqpoint{3.353074in}{3.734393in}}%
\pgfpathlineto{\pgfqpoint{3.353074in}{3.734393in}}%
\pgfpathclose%
\pgfusepath{stroke,fill}%
\end{pgfscope}%
\begin{pgfscope}%
\pgfpathrectangle{\pgfqpoint{0.786164in}{0.768110in}}{\pgfqpoint{8.851069in}{7.081890in}}%
\pgfusepath{clip}%
\pgfsetbuttcap%
\pgfsetroundjoin%
\definecolor{currentfill}{rgb}{0.233603,0.313828,0.543914}%
\pgfsetfillcolor{currentfill}%
\pgfsetfillopacity{0.700000}%
\pgfsetlinewidth{0.501875pt}%
\definecolor{currentstroke}{rgb}{1.000000,1.000000,1.000000}%
\pgfsetstrokecolor{currentstroke}%
\pgfsetstrokeopacity{0.700000}%
\pgfsetdash{}{0pt}%
\pgfpathmoveto{\pgfqpoint{3.270875in}{3.734393in}}%
\pgfpathcurveto{\pgfqpoint{3.283897in}{3.734393in}}{\pgfqpoint{3.296389in}{3.739567in}}{\pgfqpoint{3.305597in}{3.748775in}}%
\pgfpathcurveto{\pgfqpoint{3.314805in}{3.757984in}}{\pgfqpoint{3.319979in}{3.770475in}}{\pgfqpoint{3.319979in}{3.783498in}}%
\pgfpathcurveto{\pgfqpoint{3.319979in}{3.796520in}}{\pgfqpoint{3.314805in}{3.809011in}}{\pgfqpoint{3.305597in}{3.818220in}}%
\pgfpathcurveto{\pgfqpoint{3.296389in}{3.827428in}}{\pgfqpoint{3.283897in}{3.832602in}}{\pgfqpoint{3.270875in}{3.832602in}}%
\pgfpathcurveto{\pgfqpoint{3.257852in}{3.832602in}}{\pgfqpoint{3.245361in}{3.827428in}}{\pgfqpoint{3.236153in}{3.818220in}}%
\pgfpathcurveto{\pgfqpoint{3.226944in}{3.809011in}}{\pgfqpoint{3.221770in}{3.796520in}}{\pgfqpoint{3.221770in}{3.783498in}}%
\pgfpathcurveto{\pgfqpoint{3.221770in}{3.770475in}}{\pgfqpoint{3.226944in}{3.757984in}}{\pgfqpoint{3.236153in}{3.748775in}}%
\pgfpathcurveto{\pgfqpoint{3.245361in}{3.739567in}}{\pgfqpoint{3.257852in}{3.734393in}}{\pgfqpoint{3.270875in}{3.734393in}}%
\pgfpathlineto{\pgfqpoint{3.270875in}{3.734393in}}%
\pgfpathclose%
\pgfusepath{stroke,fill}%
\end{pgfscope}%
\begin{pgfscope}%
\pgfpathrectangle{\pgfqpoint{0.786164in}{0.768110in}}{\pgfqpoint{8.851069in}{7.081890in}}%
\pgfusepath{clip}%
\pgfsetbuttcap%
\pgfsetroundjoin%
\definecolor{currentfill}{rgb}{0.225863,0.330805,0.547314}%
\pgfsetfillcolor{currentfill}%
\pgfsetfillopacity{0.700000}%
\pgfsetlinewidth{0.501875pt}%
\definecolor{currentstroke}{rgb}{1.000000,1.000000,1.000000}%
\pgfsetstrokecolor{currentstroke}%
\pgfsetstrokeopacity{0.700000}%
\pgfsetdash{}{0pt}%
\pgfpathmoveto{\pgfqpoint{3.079076in}{3.559207in}}%
\pgfpathcurveto{\pgfqpoint{3.092098in}{3.559207in}}{\pgfqpoint{3.104589in}{3.564381in}}{\pgfqpoint{3.113798in}{3.573589in}}%
\pgfpathcurveto{\pgfqpoint{3.123006in}{3.582798in}}{\pgfqpoint{3.128180in}{3.595289in}}{\pgfqpoint{3.128180in}{3.608312in}}%
\pgfpathcurveto{\pgfqpoint{3.128180in}{3.621334in}}{\pgfqpoint{3.123006in}{3.633825in}}{\pgfqpoint{3.113798in}{3.643034in}}%
\pgfpathcurveto{\pgfqpoint{3.104589in}{3.652242in}}{\pgfqpoint{3.092098in}{3.657416in}}{\pgfqpoint{3.079076in}{3.657416in}}%
\pgfpathcurveto{\pgfqpoint{3.066053in}{3.657416in}}{\pgfqpoint{3.053562in}{3.652242in}}{\pgfqpoint{3.044353in}{3.643034in}}%
\pgfpathcurveto{\pgfqpoint{3.035145in}{3.633825in}}{\pgfqpoint{3.029971in}{3.621334in}}{\pgfqpoint{3.029971in}{3.608312in}}%
\pgfpathcurveto{\pgfqpoint{3.029971in}{3.595289in}}{\pgfqpoint{3.035145in}{3.582798in}}{\pgfqpoint{3.044353in}{3.573589in}}%
\pgfpathcurveto{\pgfqpoint{3.053562in}{3.564381in}}{\pgfqpoint{3.066053in}{3.559207in}}{\pgfqpoint{3.079076in}{3.559207in}}%
\pgfpathlineto{\pgfqpoint{3.079076in}{3.559207in}}%
\pgfpathclose%
\pgfusepath{stroke,fill}%
\end{pgfscope}%
\begin{pgfscope}%
\pgfpathrectangle{\pgfqpoint{0.786164in}{0.768110in}}{\pgfqpoint{8.851069in}{7.081890in}}%
\pgfusepath{clip}%
\pgfsetbuttcap%
\pgfsetroundjoin%
\definecolor{currentfill}{rgb}{0.221989,0.339161,0.548752}%
\pgfsetfillcolor{currentfill}%
\pgfsetfillopacity{0.700000}%
\pgfsetlinewidth{0.501875pt}%
\definecolor{currentstroke}{rgb}{1.000000,1.000000,1.000000}%
\pgfsetstrokecolor{currentstroke}%
\pgfsetstrokeopacity{0.700000}%
\pgfsetdash{}{0pt}%
\pgfpathmoveto{\pgfqpoint{3.307408in}{3.646800in}}%
\pgfpathcurveto{\pgfqpoint{3.320431in}{3.646800in}}{\pgfqpoint{3.332922in}{3.651974in}}{\pgfqpoint{3.342130in}{3.661182in}}%
\pgfpathcurveto{\pgfqpoint{3.351339in}{3.670391in}}{\pgfqpoint{3.356513in}{3.682882in}}{\pgfqpoint{3.356513in}{3.695905in}}%
\pgfpathcurveto{\pgfqpoint{3.356513in}{3.708927in}}{\pgfqpoint{3.351339in}{3.721418in}}{\pgfqpoint{3.342130in}{3.730627in}}%
\pgfpathcurveto{\pgfqpoint{3.332922in}{3.739835in}}{\pgfqpoint{3.320431in}{3.745009in}}{\pgfqpoint{3.307408in}{3.745009in}}%
\pgfpathcurveto{\pgfqpoint{3.294385in}{3.745009in}}{\pgfqpoint{3.281894in}{3.739835in}}{\pgfqpoint{3.272686in}{3.730627in}}%
\pgfpathcurveto{\pgfqpoint{3.263477in}{3.721418in}}{\pgfqpoint{3.258303in}{3.708927in}}{\pgfqpoint{3.258303in}{3.695905in}}%
\pgfpathcurveto{\pgfqpoint{3.258303in}{3.682882in}}{\pgfqpoint{3.263477in}{3.670391in}}{\pgfqpoint{3.272686in}{3.661182in}}%
\pgfpathcurveto{\pgfqpoint{3.281894in}{3.651974in}}{\pgfqpoint{3.294385in}{3.646800in}}{\pgfqpoint{3.307408in}{3.646800in}}%
\pgfpathlineto{\pgfqpoint{3.307408in}{3.646800in}}%
\pgfpathclose%
\pgfusepath{stroke,fill}%
\end{pgfscope}%
\begin{pgfscope}%
\pgfpathrectangle{\pgfqpoint{0.786164in}{0.768110in}}{\pgfqpoint{8.851069in}{7.081890in}}%
\pgfusepath{clip}%
\pgfsetbuttcap%
\pgfsetroundjoin%
\definecolor{currentfill}{rgb}{0.218130,0.347432,0.550038}%
\pgfsetfillcolor{currentfill}%
\pgfsetfillopacity{0.700000}%
\pgfsetlinewidth{0.501875pt}%
\definecolor{currentstroke}{rgb}{1.000000,1.000000,1.000000}%
\pgfsetstrokecolor{currentstroke}%
\pgfsetstrokeopacity{0.700000}%
\pgfsetdash{}{0pt}%
\pgfpathmoveto{\pgfqpoint{3.179542in}{3.515411in}}%
\pgfpathcurveto{\pgfqpoint{3.192565in}{3.515411in}}{\pgfqpoint{3.205056in}{3.520585in}}{\pgfqpoint{3.214264in}{3.529793in}}%
\pgfpathcurveto{\pgfqpoint{3.223473in}{3.539001in}}{\pgfqpoint{3.228647in}{3.551492in}}{\pgfqpoint{3.228647in}{3.564515in}}%
\pgfpathcurveto{\pgfqpoint{3.228647in}{3.577538in}}{\pgfqpoint{3.223473in}{3.590029in}}{\pgfqpoint{3.214264in}{3.599237in}}%
\pgfpathcurveto{\pgfqpoint{3.205056in}{3.608446in}}{\pgfqpoint{3.192565in}{3.613620in}}{\pgfqpoint{3.179542in}{3.613620in}}%
\pgfpathcurveto{\pgfqpoint{3.166519in}{3.613620in}}{\pgfqpoint{3.154028in}{3.608446in}}{\pgfqpoint{3.144820in}{3.599237in}}%
\pgfpathcurveto{\pgfqpoint{3.135611in}{3.590029in}}{\pgfqpoint{3.130437in}{3.577538in}}{\pgfqpoint{3.130437in}{3.564515in}}%
\pgfpathcurveto{\pgfqpoint{3.130437in}{3.551492in}}{\pgfqpoint{3.135611in}{3.539001in}}{\pgfqpoint{3.144820in}{3.529793in}}%
\pgfpathcurveto{\pgfqpoint{3.154028in}{3.520585in}}{\pgfqpoint{3.166519in}{3.515411in}}{\pgfqpoint{3.179542in}{3.515411in}}%
\pgfpathlineto{\pgfqpoint{3.179542in}{3.515411in}}%
\pgfpathclose%
\pgfusepath{stroke,fill}%
\end{pgfscope}%
\begin{pgfscope}%
\pgfpathrectangle{\pgfqpoint{0.786164in}{0.768110in}}{\pgfqpoint{8.851069in}{7.081890in}}%
\pgfusepath{clip}%
\pgfsetbuttcap%
\pgfsetroundjoin%
\definecolor{currentfill}{rgb}{0.283072,0.130895,0.449241}%
\pgfsetfillcolor{currentfill}%
\pgfsetfillopacity{0.700000}%
\pgfsetlinewidth{0.501875pt}%
\definecolor{currentstroke}{rgb}{1.000000,1.000000,1.000000}%
\pgfsetstrokecolor{currentstroke}%
\pgfsetstrokeopacity{0.700000}%
\pgfsetdash{}{0pt}%
\pgfpathmoveto{\pgfqpoint{3.453540in}{3.756291in}}%
\pgfpathcurveto{\pgfqpoint{3.466563in}{3.756291in}}{\pgfqpoint{3.479054in}{3.761465in}}{\pgfqpoint{3.488263in}{3.770674in}}%
\pgfpathcurveto{\pgfqpoint{3.497471in}{3.779882in}}{\pgfqpoint{3.502645in}{3.792373in}}{\pgfqpoint{3.502645in}{3.805396in}}%
\pgfpathcurveto{\pgfqpoint{3.502645in}{3.818418in}}{\pgfqpoint{3.497471in}{3.830910in}}{\pgfqpoint{3.488263in}{3.840118in}}%
\pgfpathcurveto{\pgfqpoint{3.479054in}{3.849326in}}{\pgfqpoint{3.466563in}{3.854500in}}{\pgfqpoint{3.453540in}{3.854500in}}%
\pgfpathcurveto{\pgfqpoint{3.440518in}{3.854500in}}{\pgfqpoint{3.428027in}{3.849326in}}{\pgfqpoint{3.418818in}{3.840118in}}%
\pgfpathcurveto{\pgfqpoint{3.409610in}{3.830910in}}{\pgfqpoint{3.404436in}{3.818418in}}{\pgfqpoint{3.404436in}{3.805396in}}%
\pgfpathcurveto{\pgfqpoint{3.404436in}{3.792373in}}{\pgfqpoint{3.409610in}{3.779882in}}{\pgfqpoint{3.418818in}{3.770674in}}%
\pgfpathcurveto{\pgfqpoint{3.428027in}{3.761465in}}{\pgfqpoint{3.440518in}{3.756291in}}{\pgfqpoint{3.453540in}{3.756291in}}%
\pgfpathlineto{\pgfqpoint{3.453540in}{3.756291in}}%
\pgfpathclose%
\pgfusepath{stroke,fill}%
\end{pgfscope}%
\begin{pgfscope}%
\pgfpathrectangle{\pgfqpoint{0.786164in}{0.768110in}}{\pgfqpoint{8.851069in}{7.081890in}}%
\pgfusepath{clip}%
\pgfsetbuttcap%
\pgfsetroundjoin%
\definecolor{currentfill}{rgb}{0.281887,0.150881,0.465405}%
\pgfsetfillcolor{currentfill}%
\pgfsetfillopacity{0.700000}%
\pgfsetlinewidth{0.501875pt}%
\definecolor{currentstroke}{rgb}{1.000000,1.000000,1.000000}%
\pgfsetstrokecolor{currentstroke}%
\pgfsetstrokeopacity{0.700000}%
\pgfsetdash{}{0pt}%
\pgfpathmoveto{\pgfqpoint{3.435274in}{3.734393in}}%
\pgfpathcurveto{\pgfqpoint{3.448297in}{3.734393in}}{\pgfqpoint{3.460788in}{3.739567in}}{\pgfqpoint{3.469996in}{3.748775in}}%
\pgfpathcurveto{\pgfqpoint{3.479205in}{3.757984in}}{\pgfqpoint{3.484379in}{3.770475in}}{\pgfqpoint{3.484379in}{3.783498in}}%
\pgfpathcurveto{\pgfqpoint{3.484379in}{3.796520in}}{\pgfqpoint{3.479205in}{3.809011in}}{\pgfqpoint{3.469996in}{3.818220in}}%
\pgfpathcurveto{\pgfqpoint{3.460788in}{3.827428in}}{\pgfqpoint{3.448297in}{3.832602in}}{\pgfqpoint{3.435274in}{3.832602in}}%
\pgfpathcurveto{\pgfqpoint{3.422251in}{3.832602in}}{\pgfqpoint{3.409760in}{3.827428in}}{\pgfqpoint{3.400552in}{3.818220in}}%
\pgfpathcurveto{\pgfqpoint{3.391343in}{3.809011in}}{\pgfqpoint{3.386169in}{3.796520in}}{\pgfqpoint{3.386169in}{3.783498in}}%
\pgfpathcurveto{\pgfqpoint{3.386169in}{3.770475in}}{\pgfqpoint{3.391343in}{3.757984in}}{\pgfqpoint{3.400552in}{3.748775in}}%
\pgfpathcurveto{\pgfqpoint{3.409760in}{3.739567in}}{\pgfqpoint{3.422251in}{3.734393in}}{\pgfqpoint{3.435274in}{3.734393in}}%
\pgfpathlineto{\pgfqpoint{3.435274in}{3.734393in}}%
\pgfpathclose%
\pgfusepath{stroke,fill}%
\end{pgfscope}%
\begin{pgfscope}%
\pgfpathrectangle{\pgfqpoint{0.786164in}{0.768110in}}{\pgfqpoint{8.851069in}{7.081890in}}%
\pgfusepath{clip}%
\pgfsetbuttcap%
\pgfsetroundjoin%
\definecolor{currentfill}{rgb}{0.278826,0.175490,0.483397}%
\pgfsetfillcolor{currentfill}%
\pgfsetfillopacity{0.700000}%
\pgfsetlinewidth{0.501875pt}%
\definecolor{currentstroke}{rgb}{1.000000,1.000000,1.000000}%
\pgfsetstrokecolor{currentstroke}%
\pgfsetstrokeopacity{0.700000}%
\pgfsetdash{}{0pt}%
\pgfpathmoveto{\pgfqpoint{3.563140in}{3.756291in}}%
\pgfpathcurveto{\pgfqpoint{3.576163in}{3.756291in}}{\pgfqpoint{3.588654in}{3.761465in}}{\pgfqpoint{3.597862in}{3.770674in}}%
\pgfpathcurveto{\pgfqpoint{3.607071in}{3.779882in}}{\pgfqpoint{3.612245in}{3.792373in}}{\pgfqpoint{3.612245in}{3.805396in}}%
\pgfpathcurveto{\pgfqpoint{3.612245in}{3.818418in}}{\pgfqpoint{3.607071in}{3.830910in}}{\pgfqpoint{3.597862in}{3.840118in}}%
\pgfpathcurveto{\pgfqpoint{3.588654in}{3.849326in}}{\pgfqpoint{3.576163in}{3.854500in}}{\pgfqpoint{3.563140in}{3.854500in}}%
\pgfpathcurveto{\pgfqpoint{3.550117in}{3.854500in}}{\pgfqpoint{3.537626in}{3.849326in}}{\pgfqpoint{3.528418in}{3.840118in}}%
\pgfpathcurveto{\pgfqpoint{3.519209in}{3.830910in}}{\pgfqpoint{3.514035in}{3.818418in}}{\pgfqpoint{3.514035in}{3.805396in}}%
\pgfpathcurveto{\pgfqpoint{3.514035in}{3.792373in}}{\pgfqpoint{3.519209in}{3.779882in}}{\pgfqpoint{3.528418in}{3.770674in}}%
\pgfpathcurveto{\pgfqpoint{3.537626in}{3.761465in}}{\pgfqpoint{3.550117in}{3.756291in}}{\pgfqpoint{3.563140in}{3.756291in}}%
\pgfpathlineto{\pgfqpoint{3.563140in}{3.756291in}}%
\pgfpathclose%
\pgfusepath{stroke,fill}%
\end{pgfscope}%
\begin{pgfscope}%
\pgfpathrectangle{\pgfqpoint{0.786164in}{0.768110in}}{\pgfqpoint{8.851069in}{7.081890in}}%
\pgfusepath{clip}%
\pgfsetbuttcap%
\pgfsetroundjoin%
\definecolor{currentfill}{rgb}{0.273006,0.204520,0.501721}%
\pgfsetfillcolor{currentfill}%
\pgfsetfillopacity{0.700000}%
\pgfsetlinewidth{0.501875pt}%
\definecolor{currentstroke}{rgb}{1.000000,1.000000,1.000000}%
\pgfsetstrokecolor{currentstroke}%
\pgfsetstrokeopacity{0.700000}%
\pgfsetdash{}{0pt}%
\pgfpathmoveto{\pgfqpoint{3.380474in}{3.668698in}}%
\pgfpathcurveto{\pgfqpoint{3.393497in}{3.668698in}}{\pgfqpoint{3.405988in}{3.673872in}}{\pgfqpoint{3.415196in}{3.683081in}}%
\pgfpathcurveto{\pgfqpoint{3.424405in}{3.692289in}}{\pgfqpoint{3.429579in}{3.704780in}}{\pgfqpoint{3.429579in}{3.717803in}}%
\pgfpathcurveto{\pgfqpoint{3.429579in}{3.730826in}}{\pgfqpoint{3.424405in}{3.743317in}}{\pgfqpoint{3.415196in}{3.752525in}}%
\pgfpathcurveto{\pgfqpoint{3.405988in}{3.761733in}}{\pgfqpoint{3.393497in}{3.766907in}}{\pgfqpoint{3.380474in}{3.766907in}}%
\pgfpathcurveto{\pgfqpoint{3.367451in}{3.766907in}}{\pgfqpoint{3.354960in}{3.761733in}}{\pgfqpoint{3.345752in}{3.752525in}}%
\pgfpathcurveto{\pgfqpoint{3.336544in}{3.743317in}}{\pgfqpoint{3.331370in}{3.730826in}}{\pgfqpoint{3.331370in}{3.717803in}}%
\pgfpathcurveto{\pgfqpoint{3.331370in}{3.704780in}}{\pgfqpoint{3.336544in}{3.692289in}}{\pgfqpoint{3.345752in}{3.683081in}}%
\pgfpathcurveto{\pgfqpoint{3.354960in}{3.673872in}}{\pgfqpoint{3.367451in}{3.668698in}}{\pgfqpoint{3.380474in}{3.668698in}}%
\pgfpathlineto{\pgfqpoint{3.380474in}{3.668698in}}%
\pgfpathclose%
\pgfusepath{stroke,fill}%
\end{pgfscope}%
\begin{pgfscope}%
\pgfpathrectangle{\pgfqpoint{0.786164in}{0.768110in}}{\pgfqpoint{8.851069in}{7.081890in}}%
\pgfusepath{clip}%
\pgfsetbuttcap%
\pgfsetroundjoin%
\definecolor{currentfill}{rgb}{0.273006,0.204520,0.501721}%
\pgfsetfillcolor{currentfill}%
\pgfsetfillopacity{0.700000}%
\pgfsetlinewidth{0.501875pt}%
\definecolor{currentstroke}{rgb}{1.000000,1.000000,1.000000}%
\pgfsetstrokecolor{currentstroke}%
\pgfsetstrokeopacity{0.700000}%
\pgfsetdash{}{0pt}%
\pgfpathmoveto{\pgfqpoint{3.444407in}{3.646800in}}%
\pgfpathcurveto{\pgfqpoint{3.457430in}{3.646800in}}{\pgfqpoint{3.469921in}{3.651974in}}{\pgfqpoint{3.479129in}{3.661182in}}%
\pgfpathcurveto{\pgfqpoint{3.488338in}{3.670391in}}{\pgfqpoint{3.493512in}{3.682882in}}{\pgfqpoint{3.493512in}{3.695905in}}%
\pgfpathcurveto{\pgfqpoint{3.493512in}{3.708927in}}{\pgfqpoint{3.488338in}{3.721418in}}{\pgfqpoint{3.479129in}{3.730627in}}%
\pgfpathcurveto{\pgfqpoint{3.469921in}{3.739835in}}{\pgfqpoint{3.457430in}{3.745009in}}{\pgfqpoint{3.444407in}{3.745009in}}%
\pgfpathcurveto{\pgfqpoint{3.431385in}{3.745009in}}{\pgfqpoint{3.418893in}{3.739835in}}{\pgfqpoint{3.409685in}{3.730627in}}%
\pgfpathcurveto{\pgfqpoint{3.400477in}{3.721418in}}{\pgfqpoint{3.395303in}{3.708927in}}{\pgfqpoint{3.395303in}{3.695905in}}%
\pgfpathcurveto{\pgfqpoint{3.395303in}{3.682882in}}{\pgfqpoint{3.400477in}{3.670391in}}{\pgfqpoint{3.409685in}{3.661182in}}%
\pgfpathcurveto{\pgfqpoint{3.418893in}{3.651974in}}{\pgfqpoint{3.431385in}{3.646800in}}{\pgfqpoint{3.444407in}{3.646800in}}%
\pgfpathlineto{\pgfqpoint{3.444407in}{3.646800in}}%
\pgfpathclose%
\pgfusepath{stroke,fill}%
\end{pgfscope}%
\begin{pgfscope}%
\pgfpathrectangle{\pgfqpoint{0.786164in}{0.768110in}}{\pgfqpoint{8.851069in}{7.081890in}}%
\pgfusepath{clip}%
\pgfsetbuttcap%
\pgfsetroundjoin%
\definecolor{currentfill}{rgb}{0.269308,0.218818,0.509577}%
\pgfsetfillcolor{currentfill}%
\pgfsetfillopacity{0.700000}%
\pgfsetlinewidth{0.501875pt}%
\definecolor{currentstroke}{rgb}{1.000000,1.000000,1.000000}%
\pgfsetstrokecolor{currentstroke}%
\pgfsetstrokeopacity{0.700000}%
\pgfsetdash{}{0pt}%
\pgfpathmoveto{\pgfqpoint{3.216075in}{3.471614in}}%
\pgfpathcurveto{\pgfqpoint{3.229098in}{3.471614in}}{\pgfqpoint{3.241589in}{3.476788in}}{\pgfqpoint{3.250797in}{3.485996in}}%
\pgfpathcurveto{\pgfqpoint{3.260006in}{3.495205in}}{\pgfqpoint{3.265180in}{3.507696in}}{\pgfqpoint{3.265180in}{3.520719in}}%
\pgfpathcurveto{\pgfqpoint{3.265180in}{3.533741in}}{\pgfqpoint{3.260006in}{3.546232in}}{\pgfqpoint{3.250797in}{3.555441in}}%
\pgfpathcurveto{\pgfqpoint{3.241589in}{3.564649in}}{\pgfqpoint{3.229098in}{3.569823in}}{\pgfqpoint{3.216075in}{3.569823in}}%
\pgfpathcurveto{\pgfqpoint{3.203052in}{3.569823in}}{\pgfqpoint{3.190561in}{3.564649in}}{\pgfqpoint{3.181353in}{3.555441in}}%
\pgfpathcurveto{\pgfqpoint{3.172144in}{3.546232in}}{\pgfqpoint{3.166970in}{3.533741in}}{\pgfqpoint{3.166970in}{3.520719in}}%
\pgfpathcurveto{\pgfqpoint{3.166970in}{3.507696in}}{\pgfqpoint{3.172144in}{3.495205in}}{\pgfqpoint{3.181353in}{3.485996in}}%
\pgfpathcurveto{\pgfqpoint{3.190561in}{3.476788in}}{\pgfqpoint{3.203052in}{3.471614in}}{\pgfqpoint{3.216075in}{3.471614in}}%
\pgfpathlineto{\pgfqpoint{3.216075in}{3.471614in}}%
\pgfpathclose%
\pgfusepath{stroke,fill}%
\end{pgfscope}%
\begin{pgfscope}%
\pgfpathrectangle{\pgfqpoint{0.786164in}{0.768110in}}{\pgfqpoint{8.851069in}{7.081890in}}%
\pgfusepath{clip}%
\pgfsetbuttcap%
\pgfsetroundjoin%
\definecolor{currentfill}{rgb}{0.265145,0.232956,0.516599}%
\pgfsetfillcolor{currentfill}%
\pgfsetfillopacity{0.700000}%
\pgfsetlinewidth{0.501875pt}%
\definecolor{currentstroke}{rgb}{1.000000,1.000000,1.000000}%
\pgfsetstrokecolor{currentstroke}%
\pgfsetstrokeopacity{0.700000}%
\pgfsetdash{}{0pt}%
\pgfpathmoveto{\pgfqpoint{3.389607in}{3.559207in}}%
\pgfpathcurveto{\pgfqpoint{3.402630in}{3.559207in}}{\pgfqpoint{3.415121in}{3.564381in}}{\pgfqpoint{3.424330in}{3.573589in}}%
\pgfpathcurveto{\pgfqpoint{3.433538in}{3.582798in}}{\pgfqpoint{3.438712in}{3.595289in}}{\pgfqpoint{3.438712in}{3.608312in}}%
\pgfpathcurveto{\pgfqpoint{3.438712in}{3.621334in}}{\pgfqpoint{3.433538in}{3.633825in}}{\pgfqpoint{3.424330in}{3.643034in}}%
\pgfpathcurveto{\pgfqpoint{3.415121in}{3.652242in}}{\pgfqpoint{3.402630in}{3.657416in}}{\pgfqpoint{3.389607in}{3.657416in}}%
\pgfpathcurveto{\pgfqpoint{3.376585in}{3.657416in}}{\pgfqpoint{3.364094in}{3.652242in}}{\pgfqpoint{3.354885in}{3.643034in}}%
\pgfpathcurveto{\pgfqpoint{3.345677in}{3.633825in}}{\pgfqpoint{3.340503in}{3.621334in}}{\pgfqpoint{3.340503in}{3.608312in}}%
\pgfpathcurveto{\pgfqpoint{3.340503in}{3.595289in}}{\pgfqpoint{3.345677in}{3.582798in}}{\pgfqpoint{3.354885in}{3.573589in}}%
\pgfpathcurveto{\pgfqpoint{3.364094in}{3.564381in}}{\pgfqpoint{3.376585in}{3.559207in}}{\pgfqpoint{3.389607in}{3.559207in}}%
\pgfpathlineto{\pgfqpoint{3.389607in}{3.559207in}}%
\pgfpathclose%
\pgfusepath{stroke,fill}%
\end{pgfscope}%
\begin{pgfscope}%
\pgfpathrectangle{\pgfqpoint{0.786164in}{0.768110in}}{\pgfqpoint{8.851069in}{7.081890in}}%
\pgfusepath{clip}%
\pgfsetbuttcap%
\pgfsetroundjoin%
\definecolor{currentfill}{rgb}{0.258965,0.251537,0.524736}%
\pgfsetfillcolor{currentfill}%
\pgfsetfillopacity{0.700000}%
\pgfsetlinewidth{0.501875pt}%
\definecolor{currentstroke}{rgb}{1.000000,1.000000,1.000000}%
\pgfsetstrokecolor{currentstroke}%
\pgfsetstrokeopacity{0.700000}%
\pgfsetdash{}{0pt}%
\pgfpathmoveto{\pgfqpoint{3.261741in}{3.537309in}}%
\pgfpathcurveto{\pgfqpoint{3.274764in}{3.537309in}}{\pgfqpoint{3.287255in}{3.542483in}}{\pgfqpoint{3.296464in}{3.551691in}}%
\pgfpathcurveto{\pgfqpoint{3.305672in}{3.560900in}}{\pgfqpoint{3.310846in}{3.573391in}}{\pgfqpoint{3.310846in}{3.586413in}}%
\pgfpathcurveto{\pgfqpoint{3.310846in}{3.599436in}}{\pgfqpoint{3.305672in}{3.611927in}}{\pgfqpoint{3.296464in}{3.621136in}}%
\pgfpathcurveto{\pgfqpoint{3.287255in}{3.630344in}}{\pgfqpoint{3.274764in}{3.635518in}}{\pgfqpoint{3.261741in}{3.635518in}}%
\pgfpathcurveto{\pgfqpoint{3.248719in}{3.635518in}}{\pgfqpoint{3.236228in}{3.630344in}}{\pgfqpoint{3.227019in}{3.621136in}}%
\pgfpathcurveto{\pgfqpoint{3.217811in}{3.611927in}}{\pgfqpoint{3.212637in}{3.599436in}}{\pgfqpoint{3.212637in}{3.586413in}}%
\pgfpathcurveto{\pgfqpoint{3.212637in}{3.573391in}}{\pgfqpoint{3.217811in}{3.560900in}}{\pgfqpoint{3.227019in}{3.551691in}}%
\pgfpathcurveto{\pgfqpoint{3.236228in}{3.542483in}}{\pgfqpoint{3.248719in}{3.537309in}}{\pgfqpoint{3.261741in}{3.537309in}}%
\pgfpathlineto{\pgfqpoint{3.261741in}{3.537309in}}%
\pgfpathclose%
\pgfusepath{stroke,fill}%
\end{pgfscope}%
\begin{pgfscope}%
\pgfpathrectangle{\pgfqpoint{0.786164in}{0.768110in}}{\pgfqpoint{8.851069in}{7.081890in}}%
\pgfusepath{clip}%
\pgfsetbuttcap%
\pgfsetroundjoin%
\definecolor{currentfill}{rgb}{0.252194,0.269783,0.531579}%
\pgfsetfillcolor{currentfill}%
\pgfsetfillopacity{0.700000}%
\pgfsetlinewidth{0.501875pt}%
\definecolor{currentstroke}{rgb}{1.000000,1.000000,1.000000}%
\pgfsetstrokecolor{currentstroke}%
\pgfsetstrokeopacity{0.700000}%
\pgfsetdash{}{0pt}%
\pgfpathmoveto{\pgfqpoint{3.289141in}{3.537309in}}%
\pgfpathcurveto{\pgfqpoint{3.302164in}{3.537309in}}{\pgfqpoint{3.314655in}{3.542483in}}{\pgfqpoint{3.323864in}{3.551691in}}%
\pgfpathcurveto{\pgfqpoint{3.333072in}{3.560900in}}{\pgfqpoint{3.338246in}{3.573391in}}{\pgfqpoint{3.338246in}{3.586413in}}%
\pgfpathcurveto{\pgfqpoint{3.338246in}{3.599436in}}{\pgfqpoint{3.333072in}{3.611927in}}{\pgfqpoint{3.323864in}{3.621136in}}%
\pgfpathcurveto{\pgfqpoint{3.314655in}{3.630344in}}{\pgfqpoint{3.302164in}{3.635518in}}{\pgfqpoint{3.289141in}{3.635518in}}%
\pgfpathcurveto{\pgfqpoint{3.276119in}{3.635518in}}{\pgfqpoint{3.263628in}{3.630344in}}{\pgfqpoint{3.254419in}{3.621136in}}%
\pgfpathcurveto{\pgfqpoint{3.245211in}{3.611927in}}{\pgfqpoint{3.240037in}{3.599436in}}{\pgfqpoint{3.240037in}{3.586413in}}%
\pgfpathcurveto{\pgfqpoint{3.240037in}{3.573391in}}{\pgfqpoint{3.245211in}{3.560900in}}{\pgfqpoint{3.254419in}{3.551691in}}%
\pgfpathcurveto{\pgfqpoint{3.263628in}{3.542483in}}{\pgfqpoint{3.276119in}{3.537309in}}{\pgfqpoint{3.289141in}{3.537309in}}%
\pgfpathlineto{\pgfqpoint{3.289141in}{3.537309in}}%
\pgfpathclose%
\pgfusepath{stroke,fill}%
\end{pgfscope}%
\begin{pgfscope}%
\pgfpathrectangle{\pgfqpoint{0.786164in}{0.768110in}}{\pgfqpoint{8.851069in}{7.081890in}}%
\pgfusepath{clip}%
\pgfsetbuttcap%
\pgfsetroundjoin%
\definecolor{currentfill}{rgb}{0.250425,0.274290,0.533103}%
\pgfsetfillcolor{currentfill}%
\pgfsetfillopacity{0.700000}%
\pgfsetlinewidth{0.501875pt}%
\definecolor{currentstroke}{rgb}{1.000000,1.000000,1.000000}%
\pgfsetstrokecolor{currentstroke}%
\pgfsetstrokeopacity{0.700000}%
\pgfsetdash{}{0pt}%
\pgfpathmoveto{\pgfqpoint{3.362208in}{3.581105in}}%
\pgfpathcurveto{\pgfqpoint{3.375230in}{3.581105in}}{\pgfqpoint{3.387721in}{3.586279in}}{\pgfqpoint{3.396930in}{3.595488in}}%
\pgfpathcurveto{\pgfqpoint{3.406138in}{3.604696in}}{\pgfqpoint{3.411312in}{3.617187in}}{\pgfqpoint{3.411312in}{3.630210in}}%
\pgfpathcurveto{\pgfqpoint{3.411312in}{3.643233in}}{\pgfqpoint{3.406138in}{3.655724in}}{\pgfqpoint{3.396930in}{3.664932in}}%
\pgfpathcurveto{\pgfqpoint{3.387721in}{3.674141in}}{\pgfqpoint{3.375230in}{3.679315in}}{\pgfqpoint{3.362208in}{3.679315in}}%
\pgfpathcurveto{\pgfqpoint{3.349185in}{3.679315in}}{\pgfqpoint{3.336694in}{3.674141in}}{\pgfqpoint{3.327485in}{3.664932in}}%
\pgfpathcurveto{\pgfqpoint{3.318277in}{3.655724in}}{\pgfqpoint{3.313103in}{3.643233in}}{\pgfqpoint{3.313103in}{3.630210in}}%
\pgfpathcurveto{\pgfqpoint{3.313103in}{3.617187in}}{\pgfqpoint{3.318277in}{3.604696in}}{\pgfqpoint{3.327485in}{3.595488in}}%
\pgfpathcurveto{\pgfqpoint{3.336694in}{3.586279in}}{\pgfqpoint{3.349185in}{3.581105in}}{\pgfqpoint{3.362208in}{3.581105in}}%
\pgfpathlineto{\pgfqpoint{3.362208in}{3.581105in}}%
\pgfpathclose%
\pgfusepath{stroke,fill}%
\end{pgfscope}%
\begin{pgfscope}%
\pgfpathrectangle{\pgfqpoint{0.786164in}{0.768110in}}{\pgfqpoint{8.851069in}{7.081890in}}%
\pgfusepath{clip}%
\pgfsetbuttcap%
\pgfsetroundjoin%
\definecolor{currentfill}{rgb}{0.246811,0.283237,0.535941}%
\pgfsetfillcolor{currentfill}%
\pgfsetfillopacity{0.700000}%
\pgfsetlinewidth{0.501875pt}%
\definecolor{currentstroke}{rgb}{1.000000,1.000000,1.000000}%
\pgfsetstrokecolor{currentstroke}%
\pgfsetstrokeopacity{0.700000}%
\pgfsetdash{}{0pt}%
\pgfpathmoveto{\pgfqpoint{3.188675in}{3.471614in}}%
\pgfpathcurveto{\pgfqpoint{3.201698in}{3.471614in}}{\pgfqpoint{3.214189in}{3.476788in}}{\pgfqpoint{3.223397in}{3.485996in}}%
\pgfpathcurveto{\pgfqpoint{3.232606in}{3.495205in}}{\pgfqpoint{3.237780in}{3.507696in}}{\pgfqpoint{3.237780in}{3.520719in}}%
\pgfpathcurveto{\pgfqpoint{3.237780in}{3.533741in}}{\pgfqpoint{3.232606in}{3.546232in}}{\pgfqpoint{3.223397in}{3.555441in}}%
\pgfpathcurveto{\pgfqpoint{3.214189in}{3.564649in}}{\pgfqpoint{3.201698in}{3.569823in}}{\pgfqpoint{3.188675in}{3.569823in}}%
\pgfpathcurveto{\pgfqpoint{3.175652in}{3.569823in}}{\pgfqpoint{3.163161in}{3.564649in}}{\pgfqpoint{3.153953in}{3.555441in}}%
\pgfpathcurveto{\pgfqpoint{3.144744in}{3.546232in}}{\pgfqpoint{3.139571in}{3.533741in}}{\pgfqpoint{3.139571in}{3.520719in}}%
\pgfpathcurveto{\pgfqpoint{3.139571in}{3.507696in}}{\pgfqpoint{3.144744in}{3.495205in}}{\pgfqpoint{3.153953in}{3.485996in}}%
\pgfpathcurveto{\pgfqpoint{3.163161in}{3.476788in}}{\pgfqpoint{3.175652in}{3.471614in}}{\pgfqpoint{3.188675in}{3.471614in}}%
\pgfpathlineto{\pgfqpoint{3.188675in}{3.471614in}}%
\pgfpathclose%
\pgfusepath{stroke,fill}%
\end{pgfscope}%
\begin{pgfscope}%
\pgfpathrectangle{\pgfqpoint{0.786164in}{0.768110in}}{\pgfqpoint{8.851069in}{7.081890in}}%
\pgfusepath{clip}%
\pgfsetbuttcap%
\pgfsetroundjoin%
\definecolor{currentfill}{rgb}{0.243113,0.292092,0.538516}%
\pgfsetfillcolor{currentfill}%
\pgfsetfillopacity{0.700000}%
\pgfsetlinewidth{0.501875pt}%
\definecolor{currentstroke}{rgb}{1.000000,1.000000,1.000000}%
\pgfsetstrokecolor{currentstroke}%
\pgfsetstrokeopacity{0.700000}%
\pgfsetdash{}{0pt}%
\pgfpathmoveto{\pgfqpoint{3.179542in}{3.471614in}}%
\pgfpathcurveto{\pgfqpoint{3.192565in}{3.471614in}}{\pgfqpoint{3.205056in}{3.476788in}}{\pgfqpoint{3.214264in}{3.485996in}}%
\pgfpathcurveto{\pgfqpoint{3.223473in}{3.495205in}}{\pgfqpoint{3.228647in}{3.507696in}}{\pgfqpoint{3.228647in}{3.520719in}}%
\pgfpathcurveto{\pgfqpoint{3.228647in}{3.533741in}}{\pgfqpoint{3.223473in}{3.546232in}}{\pgfqpoint{3.214264in}{3.555441in}}%
\pgfpathcurveto{\pgfqpoint{3.205056in}{3.564649in}}{\pgfqpoint{3.192565in}{3.569823in}}{\pgfqpoint{3.179542in}{3.569823in}}%
\pgfpathcurveto{\pgfqpoint{3.166519in}{3.569823in}}{\pgfqpoint{3.154028in}{3.564649in}}{\pgfqpoint{3.144820in}{3.555441in}}%
\pgfpathcurveto{\pgfqpoint{3.135611in}{3.546232in}}{\pgfqpoint{3.130437in}{3.533741in}}{\pgfqpoint{3.130437in}{3.520719in}}%
\pgfpathcurveto{\pgfqpoint{3.130437in}{3.507696in}}{\pgfqpoint{3.135611in}{3.495205in}}{\pgfqpoint{3.144820in}{3.485996in}}%
\pgfpathcurveto{\pgfqpoint{3.154028in}{3.476788in}}{\pgfqpoint{3.166519in}{3.471614in}}{\pgfqpoint{3.179542in}{3.471614in}}%
\pgfpathlineto{\pgfqpoint{3.179542in}{3.471614in}}%
\pgfpathclose%
\pgfusepath{stroke,fill}%
\end{pgfscope}%
\begin{pgfscope}%
\pgfpathrectangle{\pgfqpoint{0.786164in}{0.768110in}}{\pgfqpoint{8.851069in}{7.081890in}}%
\pgfusepath{clip}%
\pgfsetbuttcap%
\pgfsetroundjoin%
\definecolor{currentfill}{rgb}{0.243113,0.292092,0.538516}%
\pgfsetfillcolor{currentfill}%
\pgfsetfillopacity{0.700000}%
\pgfsetlinewidth{0.501875pt}%
\definecolor{currentstroke}{rgb}{1.000000,1.000000,1.000000}%
\pgfsetstrokecolor{currentstroke}%
\pgfsetstrokeopacity{0.700000}%
\pgfsetdash{}{0pt}%
\pgfpathmoveto{\pgfqpoint{3.170409in}{3.449716in}}%
\pgfpathcurveto{\pgfqpoint{3.183431in}{3.449716in}}{\pgfqpoint{3.195922in}{3.454890in}}{\pgfqpoint{3.205131in}{3.464098in}}%
\pgfpathcurveto{\pgfqpoint{3.214339in}{3.473307in}}{\pgfqpoint{3.219513in}{3.485798in}}{\pgfqpoint{3.219513in}{3.498820in}}%
\pgfpathcurveto{\pgfqpoint{3.219513in}{3.511843in}}{\pgfqpoint{3.214339in}{3.524334in}}{\pgfqpoint{3.205131in}{3.533543in}}%
\pgfpathcurveto{\pgfqpoint{3.195922in}{3.542751in}}{\pgfqpoint{3.183431in}{3.547925in}}{\pgfqpoint{3.170409in}{3.547925in}}%
\pgfpathcurveto{\pgfqpoint{3.157386in}{3.547925in}}{\pgfqpoint{3.144895in}{3.542751in}}{\pgfqpoint{3.135686in}{3.533543in}}%
\pgfpathcurveto{\pgfqpoint{3.126478in}{3.524334in}}{\pgfqpoint{3.121304in}{3.511843in}}{\pgfqpoint{3.121304in}{3.498820in}}%
\pgfpathcurveto{\pgfqpoint{3.121304in}{3.485798in}}{\pgfqpoint{3.126478in}{3.473307in}}{\pgfqpoint{3.135686in}{3.464098in}}%
\pgfpathcurveto{\pgfqpoint{3.144895in}{3.454890in}}{\pgfqpoint{3.157386in}{3.449716in}}{\pgfqpoint{3.170409in}{3.449716in}}%
\pgfpathlineto{\pgfqpoint{3.170409in}{3.449716in}}%
\pgfpathclose%
\pgfusepath{stroke,fill}%
\end{pgfscope}%
\begin{pgfscope}%
\pgfpathrectangle{\pgfqpoint{0.786164in}{0.768110in}}{\pgfqpoint{8.851069in}{7.081890in}}%
\pgfusepath{clip}%
\pgfsetbuttcap%
\pgfsetroundjoin%
\definecolor{currentfill}{rgb}{0.239346,0.300855,0.540844}%
\pgfsetfillcolor{currentfill}%
\pgfsetfillopacity{0.700000}%
\pgfsetlinewidth{0.501875pt}%
\definecolor{currentstroke}{rgb}{1.000000,1.000000,1.000000}%
\pgfsetstrokecolor{currentstroke}%
\pgfsetstrokeopacity{0.700000}%
\pgfsetdash{}{0pt}%
\pgfpathmoveto{\pgfqpoint{3.170409in}{3.405919in}}%
\pgfpathcurveto{\pgfqpoint{3.183431in}{3.405919in}}{\pgfqpoint{3.195922in}{3.411093in}}{\pgfqpoint{3.205131in}{3.420302in}}%
\pgfpathcurveto{\pgfqpoint{3.214339in}{3.429510in}}{\pgfqpoint{3.219513in}{3.442001in}}{\pgfqpoint{3.219513in}{3.455024in}}%
\pgfpathcurveto{\pgfqpoint{3.219513in}{3.468047in}}{\pgfqpoint{3.214339in}{3.480538in}}{\pgfqpoint{3.205131in}{3.489746in}}%
\pgfpathcurveto{\pgfqpoint{3.195922in}{3.498955in}}{\pgfqpoint{3.183431in}{3.504129in}}{\pgfqpoint{3.170409in}{3.504129in}}%
\pgfpathcurveto{\pgfqpoint{3.157386in}{3.504129in}}{\pgfqpoint{3.144895in}{3.498955in}}{\pgfqpoint{3.135686in}{3.489746in}}%
\pgfpathcurveto{\pgfqpoint{3.126478in}{3.480538in}}{\pgfqpoint{3.121304in}{3.468047in}}{\pgfqpoint{3.121304in}{3.455024in}}%
\pgfpathcurveto{\pgfqpoint{3.121304in}{3.442001in}}{\pgfqpoint{3.126478in}{3.429510in}}{\pgfqpoint{3.135686in}{3.420302in}}%
\pgfpathcurveto{\pgfqpoint{3.144895in}{3.411093in}}{\pgfqpoint{3.157386in}{3.405919in}}{\pgfqpoint{3.170409in}{3.405919in}}%
\pgfpathlineto{\pgfqpoint{3.170409in}{3.405919in}}%
\pgfpathclose%
\pgfusepath{stroke,fill}%
\end{pgfscope}%
\begin{pgfscope}%
\pgfpathrectangle{\pgfqpoint{0.786164in}{0.768110in}}{\pgfqpoint{8.851069in}{7.081890in}}%
\pgfusepath{clip}%
\pgfsetbuttcap%
\pgfsetroundjoin%
\definecolor{currentfill}{rgb}{0.229739,0.322361,0.545706}%
\pgfsetfillcolor{currentfill}%
\pgfsetfillopacity{0.700000}%
\pgfsetlinewidth{0.501875pt}%
\definecolor{currentstroke}{rgb}{1.000000,1.000000,1.000000}%
\pgfsetstrokecolor{currentstroke}%
\pgfsetstrokeopacity{0.700000}%
\pgfsetdash{}{0pt}%
\pgfpathmoveto{\pgfqpoint{3.088209in}{3.340225in}}%
\pgfpathcurveto{\pgfqpoint{3.101232in}{3.340225in}}{\pgfqpoint{3.113723in}{3.345399in}}{\pgfqpoint{3.122931in}{3.354607in}}%
\pgfpathcurveto{\pgfqpoint{3.132140in}{3.363816in}}{\pgfqpoint{3.137314in}{3.376307in}}{\pgfqpoint{3.137314in}{3.389329in}}%
\pgfpathcurveto{\pgfqpoint{3.137314in}{3.402352in}}{\pgfqpoint{3.132140in}{3.414843in}}{\pgfqpoint{3.122931in}{3.424052in}}%
\pgfpathcurveto{\pgfqpoint{3.113723in}{3.433260in}}{\pgfqpoint{3.101232in}{3.438434in}}{\pgfqpoint{3.088209in}{3.438434in}}%
\pgfpathcurveto{\pgfqpoint{3.075186in}{3.438434in}}{\pgfqpoint{3.062695in}{3.433260in}}{\pgfqpoint{3.053487in}{3.424052in}}%
\pgfpathcurveto{\pgfqpoint{3.044278in}{3.414843in}}{\pgfqpoint{3.039104in}{3.402352in}}{\pgfqpoint{3.039104in}{3.389329in}}%
\pgfpathcurveto{\pgfqpoint{3.039104in}{3.376307in}}{\pgfqpoint{3.044278in}{3.363816in}}{\pgfqpoint{3.053487in}{3.354607in}}%
\pgfpathcurveto{\pgfqpoint{3.062695in}{3.345399in}}{\pgfqpoint{3.075186in}{3.340225in}}{\pgfqpoint{3.088209in}{3.340225in}}%
\pgfpathlineto{\pgfqpoint{3.088209in}{3.340225in}}%
\pgfpathclose%
\pgfusepath{stroke,fill}%
\end{pgfscope}%
\begin{pgfscope}%
\pgfpathrectangle{\pgfqpoint{0.786164in}{0.768110in}}{\pgfqpoint{8.851069in}{7.081890in}}%
\pgfusepath{clip}%
\pgfsetbuttcap%
\pgfsetroundjoin%
\definecolor{currentfill}{rgb}{0.225863,0.330805,0.547314}%
\pgfsetfillcolor{currentfill}%
\pgfsetfillopacity{0.700000}%
\pgfsetlinewidth{0.501875pt}%
\definecolor{currentstroke}{rgb}{1.000000,1.000000,1.000000}%
\pgfsetstrokecolor{currentstroke}%
\pgfsetstrokeopacity{0.700000}%
\pgfsetdash{}{0pt}%
\pgfpathmoveto{\pgfqpoint{3.006009in}{3.186937in}}%
\pgfpathcurveto{\pgfqpoint{3.019032in}{3.186937in}}{\pgfqpoint{3.031523in}{3.192111in}}{\pgfqpoint{3.040732in}{3.201319in}}%
\pgfpathcurveto{\pgfqpoint{3.049940in}{3.210528in}}{\pgfqpoint{3.055114in}{3.223019in}}{\pgfqpoint{3.055114in}{3.236042in}}%
\pgfpathcurveto{\pgfqpoint{3.055114in}{3.249064in}}{\pgfqpoint{3.049940in}{3.261555in}}{\pgfqpoint{3.040732in}{3.270764in}}%
\pgfpathcurveto{\pgfqpoint{3.031523in}{3.279972in}}{\pgfqpoint{3.019032in}{3.285146in}}{\pgfqpoint{3.006009in}{3.285146in}}%
\pgfpathcurveto{\pgfqpoint{2.992987in}{3.285146in}}{\pgfqpoint{2.980496in}{3.279972in}}{\pgfqpoint{2.971287in}{3.270764in}}%
\pgfpathcurveto{\pgfqpoint{2.962079in}{3.261555in}}{\pgfqpoint{2.956905in}{3.249064in}}{\pgfqpoint{2.956905in}{3.236042in}}%
\pgfpathcurveto{\pgfqpoint{2.956905in}{3.223019in}}{\pgfqpoint{2.962079in}{3.210528in}}{\pgfqpoint{2.971287in}{3.201319in}}%
\pgfpathcurveto{\pgfqpoint{2.980496in}{3.192111in}}{\pgfqpoint{2.992987in}{3.186937in}}{\pgfqpoint{3.006009in}{3.186937in}}%
\pgfpathlineto{\pgfqpoint{3.006009in}{3.186937in}}%
\pgfpathclose%
\pgfusepath{stroke,fill}%
\end{pgfscope}%
\begin{pgfscope}%
\pgfpathrectangle{\pgfqpoint{0.786164in}{0.768110in}}{\pgfqpoint{8.851069in}{7.081890in}}%
\pgfusepath{clip}%
\pgfsetbuttcap%
\pgfsetroundjoin%
\definecolor{currentfill}{rgb}{0.212395,0.359683,0.551710}%
\pgfsetfillcolor{currentfill}%
\pgfsetfillopacity{0.700000}%
\pgfsetlinewidth{0.501875pt}%
\definecolor{currentstroke}{rgb}{1.000000,1.000000,1.000000}%
\pgfsetstrokecolor{currentstroke}%
\pgfsetstrokeopacity{0.700000}%
\pgfsetdash{}{0pt}%
\pgfpathmoveto{\pgfqpoint{2.750277in}{2.989853in}}%
\pgfpathcurveto{\pgfqpoint{2.763300in}{2.989853in}}{\pgfqpoint{2.775791in}{2.995027in}}{\pgfqpoint{2.785000in}{3.004235in}}%
\pgfpathcurveto{\pgfqpoint{2.794208in}{3.013444in}}{\pgfqpoint{2.799382in}{3.025935in}}{\pgfqpoint{2.799382in}{3.038958in}}%
\pgfpathcurveto{\pgfqpoint{2.799382in}{3.051980in}}{\pgfqpoint{2.794208in}{3.064471in}}{\pgfqpoint{2.785000in}{3.073680in}}%
\pgfpathcurveto{\pgfqpoint{2.775791in}{3.082888in}}{\pgfqpoint{2.763300in}{3.088062in}}{\pgfqpoint{2.750277in}{3.088062in}}%
\pgfpathcurveto{\pgfqpoint{2.737255in}{3.088062in}}{\pgfqpoint{2.724764in}{3.082888in}}{\pgfqpoint{2.715555in}{3.073680in}}%
\pgfpathcurveto{\pgfqpoint{2.706347in}{3.064471in}}{\pgfqpoint{2.701173in}{3.051980in}}{\pgfqpoint{2.701173in}{3.038958in}}%
\pgfpathcurveto{\pgfqpoint{2.701173in}{3.025935in}}{\pgfqpoint{2.706347in}{3.013444in}}{\pgfqpoint{2.715555in}{3.004235in}}%
\pgfpathcurveto{\pgfqpoint{2.724764in}{2.995027in}}{\pgfqpoint{2.737255in}{2.989853in}}{\pgfqpoint{2.750277in}{2.989853in}}%
\pgfpathlineto{\pgfqpoint{2.750277in}{2.989853in}}%
\pgfpathclose%
\pgfusepath{stroke,fill}%
\end{pgfscope}%
\begin{pgfscope}%
\pgfpathrectangle{\pgfqpoint{0.786164in}{0.768110in}}{\pgfqpoint{8.851069in}{7.081890in}}%
\pgfusepath{clip}%
\pgfsetbuttcap%
\pgfsetroundjoin%
\definecolor{currentfill}{rgb}{0.210503,0.363727,0.552206}%
\pgfsetfillcolor{currentfill}%
\pgfsetfillopacity{0.700000}%
\pgfsetlinewidth{0.501875pt}%
\definecolor{currentstroke}{rgb}{1.000000,1.000000,1.000000}%
\pgfsetstrokecolor{currentstroke}%
\pgfsetstrokeopacity{0.700000}%
\pgfsetdash{}{0pt}%
\pgfpathmoveto{\pgfqpoint{2.850744in}{3.077446in}}%
\pgfpathcurveto{\pgfqpoint{2.863766in}{3.077446in}}{\pgfqpoint{2.876257in}{3.082620in}}{\pgfqpoint{2.885466in}{3.091828in}}%
\pgfpathcurveto{\pgfqpoint{2.894674in}{3.101037in}}{\pgfqpoint{2.899848in}{3.113528in}}{\pgfqpoint{2.899848in}{3.126550in}}%
\pgfpathcurveto{\pgfqpoint{2.899848in}{3.139573in}}{\pgfqpoint{2.894674in}{3.152064in}}{\pgfqpoint{2.885466in}{3.161273in}}%
\pgfpathcurveto{\pgfqpoint{2.876257in}{3.170481in}}{\pgfqpoint{2.863766in}{3.175655in}}{\pgfqpoint{2.850744in}{3.175655in}}%
\pgfpathcurveto{\pgfqpoint{2.837721in}{3.175655in}}{\pgfqpoint{2.825230in}{3.170481in}}{\pgfqpoint{2.816021in}{3.161273in}}%
\pgfpathcurveto{\pgfqpoint{2.806813in}{3.152064in}}{\pgfqpoint{2.801639in}{3.139573in}}{\pgfqpoint{2.801639in}{3.126550in}}%
\pgfpathcurveto{\pgfqpoint{2.801639in}{3.113528in}}{\pgfqpoint{2.806813in}{3.101037in}}{\pgfqpoint{2.816021in}{3.091828in}}%
\pgfpathcurveto{\pgfqpoint{2.825230in}{3.082620in}}{\pgfqpoint{2.837721in}{3.077446in}}{\pgfqpoint{2.850744in}{3.077446in}}%
\pgfpathlineto{\pgfqpoint{2.850744in}{3.077446in}}%
\pgfpathclose%
\pgfusepath{stroke,fill}%
\end{pgfscope}%
\begin{pgfscope}%
\pgfpathrectangle{\pgfqpoint{0.786164in}{0.768110in}}{\pgfqpoint{8.851069in}{7.081890in}}%
\pgfusepath{clip}%
\pgfsetbuttcap%
\pgfsetroundjoin%
\definecolor{currentfill}{rgb}{0.199430,0.387607,0.554642}%
\pgfsetfillcolor{currentfill}%
\pgfsetfillopacity{0.700000}%
\pgfsetlinewidth{0.501875pt}%
\definecolor{currentstroke}{rgb}{1.000000,1.000000,1.000000}%
\pgfsetstrokecolor{currentstroke}%
\pgfsetstrokeopacity{0.700000}%
\pgfsetdash{}{0pt}%
\pgfpathmoveto{\pgfqpoint{2.704611in}{3.055548in}}%
\pgfpathcurveto{\pgfqpoint{2.717634in}{3.055548in}}{\pgfqpoint{2.730125in}{3.060722in}}{\pgfqpoint{2.739333in}{3.069930in}}%
\pgfpathcurveto{\pgfqpoint{2.748542in}{3.079138in}}{\pgfqpoint{2.753716in}{3.091630in}}{\pgfqpoint{2.753716in}{3.104652in}}%
\pgfpathcurveto{\pgfqpoint{2.753716in}{3.117675in}}{\pgfqpoint{2.748542in}{3.130166in}}{\pgfqpoint{2.739333in}{3.139374in}}%
\pgfpathcurveto{\pgfqpoint{2.730125in}{3.148583in}}{\pgfqpoint{2.717634in}{3.153757in}}{\pgfqpoint{2.704611in}{3.153757in}}%
\pgfpathcurveto{\pgfqpoint{2.691588in}{3.153757in}}{\pgfqpoint{2.679097in}{3.148583in}}{\pgfqpoint{2.669889in}{3.139374in}}%
\pgfpathcurveto{\pgfqpoint{2.660680in}{3.130166in}}{\pgfqpoint{2.655506in}{3.117675in}}{\pgfqpoint{2.655506in}{3.104652in}}%
\pgfpathcurveto{\pgfqpoint{2.655506in}{3.091630in}}{\pgfqpoint{2.660680in}{3.079138in}}{\pgfqpoint{2.669889in}{3.069930in}}%
\pgfpathcurveto{\pgfqpoint{2.679097in}{3.060722in}}{\pgfqpoint{2.691588in}{3.055548in}}{\pgfqpoint{2.704611in}{3.055548in}}%
\pgfpathlineto{\pgfqpoint{2.704611in}{3.055548in}}%
\pgfpathclose%
\pgfusepath{stroke,fill}%
\end{pgfscope}%
\begin{pgfscope}%
\pgfpathrectangle{\pgfqpoint{0.786164in}{0.768110in}}{\pgfqpoint{8.851069in}{7.081890in}}%
\pgfusepath{clip}%
\pgfsetbuttcap%
\pgfsetroundjoin%
\definecolor{currentfill}{rgb}{0.243113,0.292092,0.538516}%
\pgfsetfillcolor{currentfill}%
\pgfsetfillopacity{0.700000}%
\pgfsetlinewidth{0.501875pt}%
\definecolor{currentstroke}{rgb}{1.000000,1.000000,1.000000}%
\pgfsetstrokecolor{currentstroke}%
\pgfsetstrokeopacity{0.700000}%
\pgfsetdash{}{0pt}%
\pgfpathmoveto{\pgfqpoint{3.243475in}{4.259951in}}%
\pgfpathcurveto{\pgfqpoint{3.256498in}{4.259951in}}{\pgfqpoint{3.268989in}{4.265125in}}{\pgfqpoint{3.278197in}{4.274333in}}%
\pgfpathcurveto{\pgfqpoint{3.287406in}{4.283541in}}{\pgfqpoint{3.292580in}{4.296033in}}{\pgfqpoint{3.292580in}{4.309055in}}%
\pgfpathcurveto{\pgfqpoint{3.292580in}{4.322078in}}{\pgfqpoint{3.287406in}{4.334569in}}{\pgfqpoint{3.278197in}{4.343777in}}%
\pgfpathcurveto{\pgfqpoint{3.268989in}{4.352986in}}{\pgfqpoint{3.256498in}{4.358160in}}{\pgfqpoint{3.243475in}{4.358160in}}%
\pgfpathcurveto{\pgfqpoint{3.230452in}{4.358160in}}{\pgfqpoint{3.217961in}{4.352986in}}{\pgfqpoint{3.208753in}{4.343777in}}%
\pgfpathcurveto{\pgfqpoint{3.199544in}{4.334569in}}{\pgfqpoint{3.194370in}{4.322078in}}{\pgfqpoint{3.194370in}{4.309055in}}%
\pgfpathcurveto{\pgfqpoint{3.194370in}{4.296033in}}{\pgfqpoint{3.199544in}{4.283541in}}{\pgfqpoint{3.208753in}{4.274333in}}%
\pgfpathcurveto{\pgfqpoint{3.217961in}{4.265125in}}{\pgfqpoint{3.230452in}{4.259951in}}{\pgfqpoint{3.243475in}{4.259951in}}%
\pgfpathlineto{\pgfqpoint{3.243475in}{4.259951in}}%
\pgfpathclose%
\pgfusepath{stroke,fill}%
\end{pgfscope}%
\begin{pgfscope}%
\pgfpathrectangle{\pgfqpoint{0.786164in}{0.768110in}}{\pgfqpoint{8.851069in}{7.081890in}}%
\pgfusepath{clip}%
\pgfsetbuttcap%
\pgfsetroundjoin%
\definecolor{currentfill}{rgb}{0.235526,0.309527,0.542944}%
\pgfsetfillcolor{currentfill}%
\pgfsetfillopacity{0.700000}%
\pgfsetlinewidth{0.501875pt}%
\definecolor{currentstroke}{rgb}{1.000000,1.000000,1.000000}%
\pgfsetstrokecolor{currentstroke}%
\pgfsetstrokeopacity{0.700000}%
\pgfsetdash{}{0pt}%
\pgfpathmoveto{\pgfqpoint{3.152142in}{4.062866in}}%
\pgfpathcurveto{\pgfqpoint{3.165165in}{4.062866in}}{\pgfqpoint{3.177656in}{4.068040in}}{\pgfqpoint{3.186864in}{4.077249in}}%
\pgfpathcurveto{\pgfqpoint{3.196073in}{4.086457in}}{\pgfqpoint{3.201247in}{4.098948in}}{\pgfqpoint{3.201247in}{4.111971in}}%
\pgfpathcurveto{\pgfqpoint{3.201247in}{4.124994in}}{\pgfqpoint{3.196073in}{4.137485in}}{\pgfqpoint{3.186864in}{4.146693in}}%
\pgfpathcurveto{\pgfqpoint{3.177656in}{4.155902in}}{\pgfqpoint{3.165165in}{4.161076in}}{\pgfqpoint{3.152142in}{4.161076in}}%
\pgfpathcurveto{\pgfqpoint{3.139119in}{4.161076in}}{\pgfqpoint{3.126628in}{4.155902in}}{\pgfqpoint{3.117420in}{4.146693in}}%
\pgfpathcurveto{\pgfqpoint{3.108211in}{4.137485in}}{\pgfqpoint{3.103037in}{4.124994in}}{\pgfqpoint{3.103037in}{4.111971in}}%
\pgfpathcurveto{\pgfqpoint{3.103037in}{4.098948in}}{\pgfqpoint{3.108211in}{4.086457in}}{\pgfqpoint{3.117420in}{4.077249in}}%
\pgfpathcurveto{\pgfqpoint{3.126628in}{4.068040in}}{\pgfqpoint{3.139119in}{4.062866in}}{\pgfqpoint{3.152142in}{4.062866in}}%
\pgfpathlineto{\pgfqpoint{3.152142in}{4.062866in}}%
\pgfpathclose%
\pgfusepath{stroke,fill}%
\end{pgfscope}%
\begin{pgfscope}%
\pgfpathrectangle{\pgfqpoint{0.786164in}{0.768110in}}{\pgfqpoint{8.851069in}{7.081890in}}%
\pgfusepath{clip}%
\pgfsetbuttcap%
\pgfsetroundjoin%
\definecolor{currentfill}{rgb}{0.231674,0.318106,0.544834}%
\pgfsetfillcolor{currentfill}%
\pgfsetfillopacity{0.700000}%
\pgfsetlinewidth{0.501875pt}%
\definecolor{currentstroke}{rgb}{1.000000,1.000000,1.000000}%
\pgfsetstrokecolor{currentstroke}%
\pgfsetstrokeopacity{0.700000}%
\pgfsetdash{}{0pt}%
\pgfpathmoveto{\pgfqpoint{3.371341in}{4.391340in}}%
\pgfpathcurveto{\pgfqpoint{3.384364in}{4.391340in}}{\pgfqpoint{3.396855in}{4.396514in}}{\pgfqpoint{3.406063in}{4.405722in}}%
\pgfpathcurveto{\pgfqpoint{3.415272in}{4.414931in}}{\pgfqpoint{3.420446in}{4.427422in}}{\pgfqpoint{3.420446in}{4.440445in}}%
\pgfpathcurveto{\pgfqpoint{3.420446in}{4.453467in}}{\pgfqpoint{3.415272in}{4.465958in}}{\pgfqpoint{3.406063in}{4.475167in}}%
\pgfpathcurveto{\pgfqpoint{3.396855in}{4.484375in}}{\pgfqpoint{3.384364in}{4.489549in}}{\pgfqpoint{3.371341in}{4.489549in}}%
\pgfpathcurveto{\pgfqpoint{3.358318in}{4.489549in}}{\pgfqpoint{3.345827in}{4.484375in}}{\pgfqpoint{3.336619in}{4.475167in}}%
\pgfpathcurveto{\pgfqpoint{3.327410in}{4.465958in}}{\pgfqpoint{3.322236in}{4.453467in}}{\pgfqpoint{3.322236in}{4.440445in}}%
\pgfpathcurveto{\pgfqpoint{3.322236in}{4.427422in}}{\pgfqpoint{3.327410in}{4.414931in}}{\pgfqpoint{3.336619in}{4.405722in}}%
\pgfpathcurveto{\pgfqpoint{3.345827in}{4.396514in}}{\pgfqpoint{3.358318in}{4.391340in}}{\pgfqpoint{3.371341in}{4.391340in}}%
\pgfpathlineto{\pgfqpoint{3.371341in}{4.391340in}}%
\pgfpathclose%
\pgfusepath{stroke,fill}%
\end{pgfscope}%
\begin{pgfscope}%
\pgfpathrectangle{\pgfqpoint{0.786164in}{0.768110in}}{\pgfqpoint{8.851069in}{7.081890in}}%
\pgfusepath{clip}%
\pgfsetbuttcap%
\pgfsetroundjoin%
\definecolor{currentfill}{rgb}{0.221989,0.339161,0.548752}%
\pgfsetfillcolor{currentfill}%
\pgfsetfillopacity{0.700000}%
\pgfsetlinewidth{0.501875pt}%
\definecolor{currentstroke}{rgb}{1.000000,1.000000,1.000000}%
\pgfsetstrokecolor{currentstroke}%
\pgfsetstrokeopacity{0.700000}%
\pgfsetdash{}{0pt}%
\pgfpathmoveto{\pgfqpoint{3.280008in}{4.194256in}}%
\pgfpathcurveto{\pgfqpoint{3.293031in}{4.194256in}}{\pgfqpoint{3.305522in}{4.199430in}}{\pgfqpoint{3.314730in}{4.208638in}}%
\pgfpathcurveto{\pgfqpoint{3.323939in}{4.217847in}}{\pgfqpoint{3.329113in}{4.230338in}}{\pgfqpoint{3.329113in}{4.243360in}}%
\pgfpathcurveto{\pgfqpoint{3.329113in}{4.256383in}}{\pgfqpoint{3.323939in}{4.268874in}}{\pgfqpoint{3.314730in}{4.278083in}}%
\pgfpathcurveto{\pgfqpoint{3.305522in}{4.287291in}}{\pgfqpoint{3.293031in}{4.292465in}}{\pgfqpoint{3.280008in}{4.292465in}}%
\pgfpathcurveto{\pgfqpoint{3.266985in}{4.292465in}}{\pgfqpoint{3.254494in}{4.287291in}}{\pgfqpoint{3.245286in}{4.278083in}}%
\pgfpathcurveto{\pgfqpoint{3.236077in}{4.268874in}}{\pgfqpoint{3.230903in}{4.256383in}}{\pgfqpoint{3.230903in}{4.243360in}}%
\pgfpathcurveto{\pgfqpoint{3.230903in}{4.230338in}}{\pgfqpoint{3.236077in}{4.217847in}}{\pgfqpoint{3.245286in}{4.208638in}}%
\pgfpathcurveto{\pgfqpoint{3.254494in}{4.199430in}}{\pgfqpoint{3.266985in}{4.194256in}}{\pgfqpoint{3.280008in}{4.194256in}}%
\pgfpathlineto{\pgfqpoint{3.280008in}{4.194256in}}%
\pgfpathclose%
\pgfusepath{stroke,fill}%
\end{pgfscope}%
\begin{pgfscope}%
\pgfpathrectangle{\pgfqpoint{0.786164in}{0.768110in}}{\pgfqpoint{8.851069in}{7.081890in}}%
\pgfusepath{clip}%
\pgfsetbuttcap%
\pgfsetroundjoin%
\definecolor{currentfill}{rgb}{0.216210,0.351535,0.550627}%
\pgfsetfillcolor{currentfill}%
\pgfsetfillopacity{0.700000}%
\pgfsetlinewidth{0.501875pt}%
\definecolor{currentstroke}{rgb}{1.000000,1.000000,1.000000}%
\pgfsetstrokecolor{currentstroke}%
\pgfsetstrokeopacity{0.700000}%
\pgfsetdash{}{0pt}%
\pgfpathmoveto{\pgfqpoint{3.170409in}{3.975274in}}%
\pgfpathcurveto{\pgfqpoint{3.183431in}{3.975274in}}{\pgfqpoint{3.195922in}{3.980447in}}{\pgfqpoint{3.205131in}{3.989656in}}%
\pgfpathcurveto{\pgfqpoint{3.214339in}{3.998864in}}{\pgfqpoint{3.219513in}{4.011355in}}{\pgfqpoint{3.219513in}{4.024378in}}%
\pgfpathcurveto{\pgfqpoint{3.219513in}{4.037401in}}{\pgfqpoint{3.214339in}{4.049892in}}{\pgfqpoint{3.205131in}{4.059100in}}%
\pgfpathcurveto{\pgfqpoint{3.195922in}{4.068309in}}{\pgfqpoint{3.183431in}{4.073483in}}{\pgfqpoint{3.170409in}{4.073483in}}%
\pgfpathcurveto{\pgfqpoint{3.157386in}{4.073483in}}{\pgfqpoint{3.144895in}{4.068309in}}{\pgfqpoint{3.135686in}{4.059100in}}%
\pgfpathcurveto{\pgfqpoint{3.126478in}{4.049892in}}{\pgfqpoint{3.121304in}{4.037401in}}{\pgfqpoint{3.121304in}{4.024378in}}%
\pgfpathcurveto{\pgfqpoint{3.121304in}{4.011355in}}{\pgfqpoint{3.126478in}{3.998864in}}{\pgfqpoint{3.135686in}{3.989656in}}%
\pgfpathcurveto{\pgfqpoint{3.144895in}{3.980447in}}{\pgfqpoint{3.157386in}{3.975274in}}{\pgfqpoint{3.170409in}{3.975274in}}%
\pgfpathlineto{\pgfqpoint{3.170409in}{3.975274in}}%
\pgfpathclose%
\pgfusepath{stroke,fill}%
\end{pgfscope}%
\begin{pgfscope}%
\pgfpathrectangle{\pgfqpoint{0.786164in}{0.768110in}}{\pgfqpoint{8.851069in}{7.081890in}}%
\pgfusepath{clip}%
\pgfsetbuttcap%
\pgfsetroundjoin%
\definecolor{currentfill}{rgb}{0.204903,0.375746,0.553533}%
\pgfsetfillcolor{currentfill}%
\pgfsetfillopacity{0.700000}%
\pgfsetlinewidth{0.501875pt}%
\definecolor{currentstroke}{rgb}{1.000000,1.000000,1.000000}%
\pgfsetstrokecolor{currentstroke}%
\pgfsetstrokeopacity{0.700000}%
\pgfsetdash{}{0pt}%
\pgfpathmoveto{\pgfqpoint{3.024276in}{3.800088in}}%
\pgfpathcurveto{\pgfqpoint{3.037299in}{3.800088in}}{\pgfqpoint{3.049790in}{3.805262in}}{\pgfqpoint{3.058998in}{3.814470in}}%
\pgfpathcurveto{\pgfqpoint{3.068207in}{3.823678in}}{\pgfqpoint{3.073381in}{3.836170in}}{\pgfqpoint{3.073381in}{3.849192in}}%
\pgfpathcurveto{\pgfqpoint{3.073381in}{3.862215in}}{\pgfqpoint{3.068207in}{3.874706in}}{\pgfqpoint{3.058998in}{3.883914in}}%
\pgfpathcurveto{\pgfqpoint{3.049790in}{3.893123in}}{\pgfqpoint{3.037299in}{3.898297in}}{\pgfqpoint{3.024276in}{3.898297in}}%
\pgfpathcurveto{\pgfqpoint{3.011253in}{3.898297in}}{\pgfqpoint{2.998762in}{3.893123in}}{\pgfqpoint{2.989554in}{3.883914in}}%
\pgfpathcurveto{\pgfqpoint{2.980345in}{3.874706in}}{\pgfqpoint{2.975171in}{3.862215in}}{\pgfqpoint{2.975171in}{3.849192in}}%
\pgfpathcurveto{\pgfqpoint{2.975171in}{3.836170in}}{\pgfqpoint{2.980345in}{3.823678in}}{\pgfqpoint{2.989554in}{3.814470in}}%
\pgfpathcurveto{\pgfqpoint{2.998762in}{3.805262in}}{\pgfqpoint{3.011253in}{3.800088in}}{\pgfqpoint{3.024276in}{3.800088in}}%
\pgfpathlineto{\pgfqpoint{3.024276in}{3.800088in}}%
\pgfpathclose%
\pgfusepath{stroke,fill}%
\end{pgfscope}%
\begin{pgfscope}%
\pgfpathrectangle{\pgfqpoint{0.786164in}{0.768110in}}{\pgfqpoint{8.851069in}{7.081890in}}%
\pgfusepath{clip}%
\pgfsetbuttcap%
\pgfsetroundjoin%
\definecolor{currentfill}{rgb}{0.192357,0.403199,0.555836}%
\pgfsetfillcolor{currentfill}%
\pgfsetfillopacity{0.700000}%
\pgfsetlinewidth{0.501875pt}%
\definecolor{currentstroke}{rgb}{1.000000,1.000000,1.000000}%
\pgfsetstrokecolor{currentstroke}%
\pgfsetstrokeopacity{0.700000}%
\pgfsetdash{}{0pt}%
\pgfpathmoveto{\pgfqpoint{3.152142in}{3.756291in}}%
\pgfpathcurveto{\pgfqpoint{3.165165in}{3.756291in}}{\pgfqpoint{3.177656in}{3.761465in}}{\pgfqpoint{3.186864in}{3.770674in}}%
\pgfpathcurveto{\pgfqpoint{3.196073in}{3.779882in}}{\pgfqpoint{3.201247in}{3.792373in}}{\pgfqpoint{3.201247in}{3.805396in}}%
\pgfpathcurveto{\pgfqpoint{3.201247in}{3.818418in}}{\pgfqpoint{3.196073in}{3.830910in}}{\pgfqpoint{3.186864in}{3.840118in}}%
\pgfpathcurveto{\pgfqpoint{3.177656in}{3.849326in}}{\pgfqpoint{3.165165in}{3.854500in}}{\pgfqpoint{3.152142in}{3.854500in}}%
\pgfpathcurveto{\pgfqpoint{3.139119in}{3.854500in}}{\pgfqpoint{3.126628in}{3.849326in}}{\pgfqpoint{3.117420in}{3.840118in}}%
\pgfpathcurveto{\pgfqpoint{3.108211in}{3.830910in}}{\pgfqpoint{3.103037in}{3.818418in}}{\pgfqpoint{3.103037in}{3.805396in}}%
\pgfpathcurveto{\pgfqpoint{3.103037in}{3.792373in}}{\pgfqpoint{3.108211in}{3.779882in}}{\pgfqpoint{3.117420in}{3.770674in}}%
\pgfpathcurveto{\pgfqpoint{3.126628in}{3.761465in}}{\pgfqpoint{3.139119in}{3.756291in}}{\pgfqpoint{3.152142in}{3.756291in}}%
\pgfpathlineto{\pgfqpoint{3.152142in}{3.756291in}}%
\pgfpathclose%
\pgfusepath{stroke,fill}%
\end{pgfscope}%
\begin{pgfscope}%
\pgfpathrectangle{\pgfqpoint{0.786164in}{0.768110in}}{\pgfqpoint{8.851069in}{7.081890in}}%
\pgfusepath{clip}%
\pgfsetbuttcap%
\pgfsetroundjoin%
\definecolor{currentfill}{rgb}{0.182256,0.426184,0.557120}%
\pgfsetfillcolor{currentfill}%
\pgfsetfillopacity{0.700000}%
\pgfsetlinewidth{0.501875pt}%
\definecolor{currentstroke}{rgb}{1.000000,1.000000,1.000000}%
\pgfsetstrokecolor{currentstroke}%
\pgfsetstrokeopacity{0.700000}%
\pgfsetdash{}{0pt}%
\pgfpathmoveto{\pgfqpoint{2.905543in}{3.515411in}}%
\pgfpathcurveto{\pgfqpoint{2.918566in}{3.515411in}}{\pgfqpoint{2.931057in}{3.520585in}}{\pgfqpoint{2.940265in}{3.529793in}}%
\pgfpathcurveto{\pgfqpoint{2.949474in}{3.539001in}}{\pgfqpoint{2.954648in}{3.551492in}}{\pgfqpoint{2.954648in}{3.564515in}}%
\pgfpathcurveto{\pgfqpoint{2.954648in}{3.577538in}}{\pgfqpoint{2.949474in}{3.590029in}}{\pgfqpoint{2.940265in}{3.599237in}}%
\pgfpathcurveto{\pgfqpoint{2.931057in}{3.608446in}}{\pgfqpoint{2.918566in}{3.613620in}}{\pgfqpoint{2.905543in}{3.613620in}}%
\pgfpathcurveto{\pgfqpoint{2.892521in}{3.613620in}}{\pgfqpoint{2.880029in}{3.608446in}}{\pgfqpoint{2.870821in}{3.599237in}}%
\pgfpathcurveto{\pgfqpoint{2.861613in}{3.590029in}}{\pgfqpoint{2.856439in}{3.577538in}}{\pgfqpoint{2.856439in}{3.564515in}}%
\pgfpathcurveto{\pgfqpoint{2.856439in}{3.551492in}}{\pgfqpoint{2.861613in}{3.539001in}}{\pgfqpoint{2.870821in}{3.529793in}}%
\pgfpathcurveto{\pgfqpoint{2.880029in}{3.520585in}}{\pgfqpoint{2.892521in}{3.515411in}}{\pgfqpoint{2.905543in}{3.515411in}}%
\pgfpathlineto{\pgfqpoint{2.905543in}{3.515411in}}%
\pgfpathclose%
\pgfusepath{stroke,fill}%
\end{pgfscope}%
\begin{pgfscope}%
\pgfpathrectangle{\pgfqpoint{0.786164in}{0.768110in}}{\pgfqpoint{8.851069in}{7.081890in}}%
\pgfusepath{clip}%
\pgfsetbuttcap%
\pgfsetroundjoin%
\definecolor{currentfill}{rgb}{0.169646,0.456262,0.558030}%
\pgfsetfillcolor{currentfill}%
\pgfsetfillopacity{0.700000}%
\pgfsetlinewidth{0.501875pt}%
\definecolor{currentstroke}{rgb}{1.000000,1.000000,1.000000}%
\pgfsetstrokecolor{currentstroke}%
\pgfsetstrokeopacity{0.700000}%
\pgfsetdash{}{0pt}%
\pgfpathmoveto{\pgfqpoint{2.768544in}{3.296428in}}%
\pgfpathcurveto{\pgfqpoint{2.781567in}{3.296428in}}{\pgfqpoint{2.794058in}{3.301602in}}{\pgfqpoint{2.803266in}{3.310811in}}%
\pgfpathcurveto{\pgfqpoint{2.812475in}{3.320019in}}{\pgfqpoint{2.817649in}{3.332510in}}{\pgfqpoint{2.817649in}{3.345533in}}%
\pgfpathcurveto{\pgfqpoint{2.817649in}{3.358556in}}{\pgfqpoint{2.812475in}{3.371047in}}{\pgfqpoint{2.803266in}{3.380255in}}%
\pgfpathcurveto{\pgfqpoint{2.794058in}{3.389463in}}{\pgfqpoint{2.781567in}{3.394637in}}{\pgfqpoint{2.768544in}{3.394637in}}%
\pgfpathcurveto{\pgfqpoint{2.755521in}{3.394637in}}{\pgfqpoint{2.743030in}{3.389463in}}{\pgfqpoint{2.733822in}{3.380255in}}%
\pgfpathcurveto{\pgfqpoint{2.724613in}{3.371047in}}{\pgfqpoint{2.719439in}{3.358556in}}{\pgfqpoint{2.719439in}{3.345533in}}%
\pgfpathcurveto{\pgfqpoint{2.719439in}{3.332510in}}{\pgfqpoint{2.724613in}{3.320019in}}{\pgfqpoint{2.733822in}{3.310811in}}%
\pgfpathcurveto{\pgfqpoint{2.743030in}{3.301602in}}{\pgfqpoint{2.755521in}{3.296428in}}{\pgfqpoint{2.768544in}{3.296428in}}%
\pgfpathlineto{\pgfqpoint{2.768544in}{3.296428in}}%
\pgfpathclose%
\pgfusepath{stroke,fill}%
\end{pgfscope}%
\begin{pgfscope}%
\pgfpathrectangle{\pgfqpoint{0.786164in}{0.768110in}}{\pgfqpoint{8.851069in}{7.081890in}}%
\pgfusepath{clip}%
\pgfsetbuttcap%
\pgfsetroundjoin%
\definecolor{currentfill}{rgb}{0.159194,0.482237,0.558073}%
\pgfsetfillcolor{currentfill}%
\pgfsetfillopacity{0.700000}%
\pgfsetlinewidth{0.501875pt}%
\definecolor{currentstroke}{rgb}{1.000000,1.000000,1.000000}%
\pgfsetstrokecolor{currentstroke}%
\pgfsetstrokeopacity{0.700000}%
\pgfsetdash{}{0pt}%
\pgfpathmoveto{\pgfqpoint{2.768544in}{3.340225in}}%
\pgfpathcurveto{\pgfqpoint{2.781567in}{3.340225in}}{\pgfqpoint{2.794058in}{3.345399in}}{\pgfqpoint{2.803266in}{3.354607in}}%
\pgfpathcurveto{\pgfqpoint{2.812475in}{3.363816in}}{\pgfqpoint{2.817649in}{3.376307in}}{\pgfqpoint{2.817649in}{3.389329in}}%
\pgfpathcurveto{\pgfqpoint{2.817649in}{3.402352in}}{\pgfqpoint{2.812475in}{3.414843in}}{\pgfqpoint{2.803266in}{3.424052in}}%
\pgfpathcurveto{\pgfqpoint{2.794058in}{3.433260in}}{\pgfqpoint{2.781567in}{3.438434in}}{\pgfqpoint{2.768544in}{3.438434in}}%
\pgfpathcurveto{\pgfqpoint{2.755521in}{3.438434in}}{\pgfqpoint{2.743030in}{3.433260in}}{\pgfqpoint{2.733822in}{3.424052in}}%
\pgfpathcurveto{\pgfqpoint{2.724613in}{3.414843in}}{\pgfqpoint{2.719439in}{3.402352in}}{\pgfqpoint{2.719439in}{3.389329in}}%
\pgfpathcurveto{\pgfqpoint{2.719439in}{3.376307in}}{\pgfqpoint{2.724613in}{3.363816in}}{\pgfqpoint{2.733822in}{3.354607in}}%
\pgfpathcurveto{\pgfqpoint{2.743030in}{3.345399in}}{\pgfqpoint{2.755521in}{3.340225in}}{\pgfqpoint{2.768544in}{3.340225in}}%
\pgfpathlineto{\pgfqpoint{2.768544in}{3.340225in}}%
\pgfpathclose%
\pgfusepath{stroke,fill}%
\end{pgfscope}%
\begin{pgfscope}%
\pgfpathrectangle{\pgfqpoint{0.786164in}{0.768110in}}{\pgfqpoint{8.851069in}{7.081890in}}%
\pgfusepath{clip}%
\pgfsetbuttcap%
\pgfsetroundjoin%
\definecolor{currentfill}{rgb}{0.147607,0.511733,0.557049}%
\pgfsetfillcolor{currentfill}%
\pgfsetfillopacity{0.700000}%
\pgfsetlinewidth{0.501875pt}%
\definecolor{currentstroke}{rgb}{1.000000,1.000000,1.000000}%
\pgfsetstrokecolor{currentstroke}%
\pgfsetstrokeopacity{0.700000}%
\pgfsetdash{}{0pt}%
\pgfpathmoveto{\pgfqpoint{2.613278in}{3.186937in}}%
\pgfpathcurveto{\pgfqpoint{2.626301in}{3.186937in}}{\pgfqpoint{2.638792in}{3.192111in}}{\pgfqpoint{2.648000in}{3.201319in}}%
\pgfpathcurveto{\pgfqpoint{2.657209in}{3.210528in}}{\pgfqpoint{2.662383in}{3.223019in}}{\pgfqpoint{2.662383in}{3.236042in}}%
\pgfpathcurveto{\pgfqpoint{2.662383in}{3.249064in}}{\pgfqpoint{2.657209in}{3.261555in}}{\pgfqpoint{2.648000in}{3.270764in}}%
\pgfpathcurveto{\pgfqpoint{2.638792in}{3.279972in}}{\pgfqpoint{2.626301in}{3.285146in}}{\pgfqpoint{2.613278in}{3.285146in}}%
\pgfpathcurveto{\pgfqpoint{2.600255in}{3.285146in}}{\pgfqpoint{2.587764in}{3.279972in}}{\pgfqpoint{2.578556in}{3.270764in}}%
\pgfpathcurveto{\pgfqpoint{2.569347in}{3.261555in}}{\pgfqpoint{2.564173in}{3.249064in}}{\pgfqpoint{2.564173in}{3.236042in}}%
\pgfpathcurveto{\pgfqpoint{2.564173in}{3.223019in}}{\pgfqpoint{2.569347in}{3.210528in}}{\pgfqpoint{2.578556in}{3.201319in}}%
\pgfpathcurveto{\pgfqpoint{2.587764in}{3.192111in}}{\pgfqpoint{2.600255in}{3.186937in}}{\pgfqpoint{2.613278in}{3.186937in}}%
\pgfpathlineto{\pgfqpoint{2.613278in}{3.186937in}}%
\pgfpathclose%
\pgfusepath{stroke,fill}%
\end{pgfscope}%
\begin{pgfscope}%
\pgfpathrectangle{\pgfqpoint{0.786164in}{0.768110in}}{\pgfqpoint{8.851069in}{7.081890in}}%
\pgfusepath{clip}%
\pgfsetbuttcap%
\pgfsetroundjoin%
\definecolor{currentfill}{rgb}{0.141935,0.526453,0.555991}%
\pgfsetfillcolor{currentfill}%
\pgfsetfillopacity{0.700000}%
\pgfsetlinewidth{0.501875pt}%
\definecolor{currentstroke}{rgb}{1.000000,1.000000,1.000000}%
\pgfsetstrokecolor{currentstroke}%
\pgfsetstrokeopacity{0.700000}%
\pgfsetdash{}{0pt}%
\pgfpathmoveto{\pgfqpoint{2.576745in}{3.033649in}}%
\pgfpathcurveto{\pgfqpoint{2.589768in}{3.033649in}}{\pgfqpoint{2.602259in}{3.038823in}}{\pgfqpoint{2.611467in}{3.048032in}}%
\pgfpathcurveto{\pgfqpoint{2.620676in}{3.057240in}}{\pgfqpoint{2.625850in}{3.069731in}}{\pgfqpoint{2.625850in}{3.082754in}}%
\pgfpathcurveto{\pgfqpoint{2.625850in}{3.095777in}}{\pgfqpoint{2.620676in}{3.108268in}}{\pgfqpoint{2.611467in}{3.117476in}}%
\pgfpathcurveto{\pgfqpoint{2.602259in}{3.126685in}}{\pgfqpoint{2.589768in}{3.131859in}}{\pgfqpoint{2.576745in}{3.131859in}}%
\pgfpathcurveto{\pgfqpoint{2.563722in}{3.131859in}}{\pgfqpoint{2.551231in}{3.126685in}}{\pgfqpoint{2.542023in}{3.117476in}}%
\pgfpathcurveto{\pgfqpoint{2.532814in}{3.108268in}}{\pgfqpoint{2.527640in}{3.095777in}}{\pgfqpoint{2.527640in}{3.082754in}}%
\pgfpathcurveto{\pgfqpoint{2.527640in}{3.069731in}}{\pgfqpoint{2.532814in}{3.057240in}}{\pgfqpoint{2.542023in}{3.048032in}}%
\pgfpathcurveto{\pgfqpoint{2.551231in}{3.038823in}}{\pgfqpoint{2.563722in}{3.033649in}}{\pgfqpoint{2.576745in}{3.033649in}}%
\pgfpathlineto{\pgfqpoint{2.576745in}{3.033649in}}%
\pgfpathclose%
\pgfusepath{stroke,fill}%
\end{pgfscope}%
\begin{pgfscope}%
\pgfpathrectangle{\pgfqpoint{0.786164in}{0.768110in}}{\pgfqpoint{8.851069in}{7.081890in}}%
\pgfusepath{clip}%
\pgfsetbuttcap%
\pgfsetroundjoin%
\definecolor{currentfill}{rgb}{0.135066,0.544853,0.554029}%
\pgfsetfillcolor{currentfill}%
\pgfsetfillopacity{0.700000}%
\pgfsetlinewidth{0.501875pt}%
\definecolor{currentstroke}{rgb}{1.000000,1.000000,1.000000}%
\pgfsetstrokecolor{currentstroke}%
\pgfsetstrokeopacity{0.700000}%
\pgfsetdash{}{0pt}%
\pgfpathmoveto{\pgfqpoint{2.631545in}{3.143141in}}%
\pgfpathcurveto{\pgfqpoint{2.644567in}{3.143141in}}{\pgfqpoint{2.657058in}{3.148314in}}{\pgfqpoint{2.666267in}{3.157523in}}%
\pgfpathcurveto{\pgfqpoint{2.675475in}{3.166731in}}{\pgfqpoint{2.680649in}{3.179222in}}{\pgfqpoint{2.680649in}{3.192245in}}%
\pgfpathcurveto{\pgfqpoint{2.680649in}{3.205268in}}{\pgfqpoint{2.675475in}{3.217759in}}{\pgfqpoint{2.666267in}{3.226967in}}%
\pgfpathcurveto{\pgfqpoint{2.657058in}{3.236176in}}{\pgfqpoint{2.644567in}{3.241350in}}{\pgfqpoint{2.631545in}{3.241350in}}%
\pgfpathcurveto{\pgfqpoint{2.618522in}{3.241350in}}{\pgfqpoint{2.606031in}{3.236176in}}{\pgfqpoint{2.596822in}{3.226967in}}%
\pgfpathcurveto{\pgfqpoint{2.587614in}{3.217759in}}{\pgfqpoint{2.582440in}{3.205268in}}{\pgfqpoint{2.582440in}{3.192245in}}%
\pgfpathcurveto{\pgfqpoint{2.582440in}{3.179222in}}{\pgfqpoint{2.587614in}{3.166731in}}{\pgfqpoint{2.596822in}{3.157523in}}%
\pgfpathcurveto{\pgfqpoint{2.606031in}{3.148314in}}{\pgfqpoint{2.618522in}{3.143141in}}{\pgfqpoint{2.631545in}{3.143141in}}%
\pgfpathlineto{\pgfqpoint{2.631545in}{3.143141in}}%
\pgfpathclose%
\pgfusepath{stroke,fill}%
\end{pgfscope}%
\begin{pgfscope}%
\pgfpathrectangle{\pgfqpoint{0.786164in}{0.768110in}}{\pgfqpoint{8.851069in}{7.081890in}}%
\pgfusepath{clip}%
\pgfsetbuttcap%
\pgfsetroundjoin%
\definecolor{currentfill}{rgb}{0.122606,0.585371,0.546557}%
\pgfsetfillcolor{currentfill}%
\pgfsetfillopacity{0.700000}%
\pgfsetlinewidth{0.501875pt}%
\definecolor{currentstroke}{rgb}{1.000000,1.000000,1.000000}%
\pgfsetstrokecolor{currentstroke}%
\pgfsetstrokeopacity{0.700000}%
\pgfsetdash{}{0pt}%
\pgfpathmoveto{\pgfqpoint{2.631545in}{3.033649in}}%
\pgfpathcurveto{\pgfqpoint{2.644567in}{3.033649in}}{\pgfqpoint{2.657058in}{3.038823in}}{\pgfqpoint{2.666267in}{3.048032in}}%
\pgfpathcurveto{\pgfqpoint{2.675475in}{3.057240in}}{\pgfqpoint{2.680649in}{3.069731in}}{\pgfqpoint{2.680649in}{3.082754in}}%
\pgfpathcurveto{\pgfqpoint{2.680649in}{3.095777in}}{\pgfqpoint{2.675475in}{3.108268in}}{\pgfqpoint{2.666267in}{3.117476in}}%
\pgfpathcurveto{\pgfqpoint{2.657058in}{3.126685in}}{\pgfqpoint{2.644567in}{3.131859in}}{\pgfqpoint{2.631545in}{3.131859in}}%
\pgfpathcurveto{\pgfqpoint{2.618522in}{3.131859in}}{\pgfqpoint{2.606031in}{3.126685in}}{\pgfqpoint{2.596822in}{3.117476in}}%
\pgfpathcurveto{\pgfqpoint{2.587614in}{3.108268in}}{\pgfqpoint{2.582440in}{3.095777in}}{\pgfqpoint{2.582440in}{3.082754in}}%
\pgfpathcurveto{\pgfqpoint{2.582440in}{3.069731in}}{\pgfqpoint{2.587614in}{3.057240in}}{\pgfqpoint{2.596822in}{3.048032in}}%
\pgfpathcurveto{\pgfqpoint{2.606031in}{3.038823in}}{\pgfqpoint{2.618522in}{3.033649in}}{\pgfqpoint{2.631545in}{3.033649in}}%
\pgfpathlineto{\pgfqpoint{2.631545in}{3.033649in}}%
\pgfpathclose%
\pgfusepath{stroke,fill}%
\end{pgfscope}%
\begin{pgfscope}%
\pgfpathrectangle{\pgfqpoint{0.786164in}{0.768110in}}{\pgfqpoint{8.851069in}{7.081890in}}%
\pgfusepath{clip}%
\pgfsetbuttcap%
\pgfsetroundjoin%
\definecolor{currentfill}{rgb}{0.120565,0.596422,0.543611}%
\pgfsetfillcolor{currentfill}%
\pgfsetfillopacity{0.700000}%
\pgfsetlinewidth{0.501875pt}%
\definecolor{currentstroke}{rgb}{1.000000,1.000000,1.000000}%
\pgfsetstrokecolor{currentstroke}%
\pgfsetstrokeopacity{0.700000}%
\pgfsetdash{}{0pt}%
\pgfpathmoveto{\pgfqpoint{2.622411in}{3.077446in}}%
\pgfpathcurveto{\pgfqpoint{2.635434in}{3.077446in}}{\pgfqpoint{2.647925in}{3.082620in}}{\pgfqpoint{2.657134in}{3.091828in}}%
\pgfpathcurveto{\pgfqpoint{2.666342in}{3.101037in}}{\pgfqpoint{2.671516in}{3.113528in}}{\pgfqpoint{2.671516in}{3.126550in}}%
\pgfpathcurveto{\pgfqpoint{2.671516in}{3.139573in}}{\pgfqpoint{2.666342in}{3.152064in}}{\pgfqpoint{2.657134in}{3.161273in}}%
\pgfpathcurveto{\pgfqpoint{2.647925in}{3.170481in}}{\pgfqpoint{2.635434in}{3.175655in}}{\pgfqpoint{2.622411in}{3.175655in}}%
\pgfpathcurveto{\pgfqpoint{2.609389in}{3.175655in}}{\pgfqpoint{2.596898in}{3.170481in}}{\pgfqpoint{2.587689in}{3.161273in}}%
\pgfpathcurveto{\pgfqpoint{2.578481in}{3.152064in}}{\pgfqpoint{2.573307in}{3.139573in}}{\pgfqpoint{2.573307in}{3.126550in}}%
\pgfpathcurveto{\pgfqpoint{2.573307in}{3.113528in}}{\pgfqpoint{2.578481in}{3.101037in}}{\pgfqpoint{2.587689in}{3.091828in}}%
\pgfpathcurveto{\pgfqpoint{2.596898in}{3.082620in}}{\pgfqpoint{2.609389in}{3.077446in}}{\pgfqpoint{2.622411in}{3.077446in}}%
\pgfpathlineto{\pgfqpoint{2.622411in}{3.077446in}}%
\pgfpathclose%
\pgfusepath{stroke,fill}%
\end{pgfscope}%
\begin{pgfscope}%
\pgfpathrectangle{\pgfqpoint{0.786164in}{0.768110in}}{\pgfqpoint{8.851069in}{7.081890in}}%
\pgfusepath{clip}%
\pgfsetbuttcap%
\pgfsetroundjoin%
\definecolor{currentfill}{rgb}{0.120081,0.622161,0.534946}%
\pgfsetfillcolor{currentfill}%
\pgfsetfillopacity{0.700000}%
\pgfsetlinewidth{0.501875pt}%
\definecolor{currentstroke}{rgb}{1.000000,1.000000,1.000000}%
\pgfsetstrokecolor{currentstroke}%
\pgfsetstrokeopacity{0.700000}%
\pgfsetdash{}{0pt}%
\pgfpathmoveto{\pgfqpoint{2.521945in}{2.902260in}}%
\pgfpathcurveto{\pgfqpoint{2.534968in}{2.902260in}}{\pgfqpoint{2.547459in}{2.907434in}}{\pgfqpoint{2.556667in}{2.916642in}}%
\pgfpathcurveto{\pgfqpoint{2.565876in}{2.925851in}}{\pgfqpoint{2.571050in}{2.938342in}}{\pgfqpoint{2.571050in}{2.951365in}}%
\pgfpathcurveto{\pgfqpoint{2.571050in}{2.964387in}}{\pgfqpoint{2.565876in}{2.976878in}}{\pgfqpoint{2.556667in}{2.986087in}}%
\pgfpathcurveto{\pgfqpoint{2.547459in}{2.995295in}}{\pgfqpoint{2.534968in}{3.000469in}}{\pgfqpoint{2.521945in}{3.000469in}}%
\pgfpathcurveto{\pgfqpoint{2.508922in}{3.000469in}}{\pgfqpoint{2.496431in}{2.995295in}}{\pgfqpoint{2.487223in}{2.986087in}}%
\pgfpathcurveto{\pgfqpoint{2.478015in}{2.976878in}}{\pgfqpoint{2.472841in}{2.964387in}}{\pgfqpoint{2.472841in}{2.951365in}}%
\pgfpathcurveto{\pgfqpoint{2.472841in}{2.938342in}}{\pgfqpoint{2.478015in}{2.925851in}}{\pgfqpoint{2.487223in}{2.916642in}}%
\pgfpathcurveto{\pgfqpoint{2.496431in}{2.907434in}}{\pgfqpoint{2.508922in}{2.902260in}}{\pgfqpoint{2.521945in}{2.902260in}}%
\pgfpathlineto{\pgfqpoint{2.521945in}{2.902260in}}%
\pgfpathclose%
\pgfusepath{stroke,fill}%
\end{pgfscope}%
\begin{pgfscope}%
\pgfpathrectangle{\pgfqpoint{0.786164in}{0.768110in}}{\pgfqpoint{8.851069in}{7.081890in}}%
\pgfusepath{clip}%
\pgfsetbuttcap%
\pgfsetroundjoin%
\definecolor{currentfill}{rgb}{0.135066,0.544853,0.554029}%
\pgfsetfillcolor{currentfill}%
\pgfsetfillopacity{0.700000}%
\pgfsetlinewidth{0.501875pt}%
\definecolor{currentstroke}{rgb}{1.000000,1.000000,1.000000}%
\pgfsetstrokecolor{currentstroke}%
\pgfsetstrokeopacity{0.700000}%
\pgfsetdash{}{0pt}%
\pgfpathmoveto{\pgfqpoint{2.293613in}{2.705176in}}%
\pgfpathcurveto{\pgfqpoint{2.306636in}{2.705176in}}{\pgfqpoint{2.319127in}{2.710350in}}{\pgfqpoint{2.328335in}{2.719558in}}%
\pgfpathcurveto{\pgfqpoint{2.337544in}{2.728767in}}{\pgfqpoint{2.342718in}{2.741258in}}{\pgfqpoint{2.342718in}{2.754280in}}%
\pgfpathcurveto{\pgfqpoint{2.342718in}{2.767303in}}{\pgfqpoint{2.337544in}{2.779794in}}{\pgfqpoint{2.328335in}{2.789003in}}%
\pgfpathcurveto{\pgfqpoint{2.319127in}{2.798211in}}{\pgfqpoint{2.306636in}{2.803385in}}{\pgfqpoint{2.293613in}{2.803385in}}%
\pgfpathcurveto{\pgfqpoint{2.280590in}{2.803385in}}{\pgfqpoint{2.268099in}{2.798211in}}{\pgfqpoint{2.258891in}{2.789003in}}%
\pgfpathcurveto{\pgfqpoint{2.249682in}{2.779794in}}{\pgfqpoint{2.244508in}{2.767303in}}{\pgfqpoint{2.244508in}{2.754280in}}%
\pgfpathcurveto{\pgfqpoint{2.244508in}{2.741258in}}{\pgfqpoint{2.249682in}{2.728767in}}{\pgfqpoint{2.258891in}{2.719558in}}%
\pgfpathcurveto{\pgfqpoint{2.268099in}{2.710350in}}{\pgfqpoint{2.280590in}{2.705176in}}{\pgfqpoint{2.293613in}{2.705176in}}%
\pgfpathlineto{\pgfqpoint{2.293613in}{2.705176in}}%
\pgfpathclose%
\pgfusepath{stroke,fill}%
\end{pgfscope}%
\begin{pgfscope}%
\pgfpathrectangle{\pgfqpoint{0.786164in}{0.768110in}}{\pgfqpoint{8.851069in}{7.081890in}}%
\pgfusepath{clip}%
\pgfsetbuttcap%
\pgfsetroundjoin%
\definecolor{currentfill}{rgb}{0.150148,0.676631,0.506589}%
\pgfsetfillcolor{currentfill}%
\pgfsetfillopacity{0.700000}%
\pgfsetlinewidth{0.501875pt}%
\definecolor{currentstroke}{rgb}{1.000000,1.000000,1.000000}%
\pgfsetstrokecolor{currentstroke}%
\pgfsetstrokeopacity{0.700000}%
\pgfsetdash{}{0pt}%
\pgfpathmoveto{\pgfqpoint{2.421479in}{2.705176in}}%
\pgfpathcurveto{\pgfqpoint{2.434502in}{2.705176in}}{\pgfqpoint{2.446993in}{2.710350in}}{\pgfqpoint{2.456201in}{2.719558in}}%
\pgfpathcurveto{\pgfqpoint{2.465410in}{2.728767in}}{\pgfqpoint{2.470584in}{2.741258in}}{\pgfqpoint{2.470584in}{2.754280in}}%
\pgfpathcurveto{\pgfqpoint{2.470584in}{2.767303in}}{\pgfqpoint{2.465410in}{2.779794in}}{\pgfqpoint{2.456201in}{2.789003in}}%
\pgfpathcurveto{\pgfqpoint{2.446993in}{2.798211in}}{\pgfqpoint{2.434502in}{2.803385in}}{\pgfqpoint{2.421479in}{2.803385in}}%
\pgfpathcurveto{\pgfqpoint{2.408456in}{2.803385in}}{\pgfqpoint{2.395965in}{2.798211in}}{\pgfqpoint{2.386757in}{2.789003in}}%
\pgfpathcurveto{\pgfqpoint{2.377548in}{2.779794in}}{\pgfqpoint{2.372374in}{2.767303in}}{\pgfqpoint{2.372374in}{2.754280in}}%
\pgfpathcurveto{\pgfqpoint{2.372374in}{2.741258in}}{\pgfqpoint{2.377548in}{2.728767in}}{\pgfqpoint{2.386757in}{2.719558in}}%
\pgfpathcurveto{\pgfqpoint{2.395965in}{2.710350in}}{\pgfqpoint{2.408456in}{2.705176in}}{\pgfqpoint{2.421479in}{2.705176in}}%
\pgfpathlineto{\pgfqpoint{2.421479in}{2.705176in}}%
\pgfpathclose%
\pgfusepath{stroke,fill}%
\end{pgfscope}%
\begin{pgfscope}%
\pgfpathrectangle{\pgfqpoint{0.786164in}{0.768110in}}{\pgfqpoint{8.851069in}{7.081890in}}%
\pgfusepath{clip}%
\pgfsetbuttcap%
\pgfsetroundjoin%
\definecolor{currentfill}{rgb}{0.162016,0.687316,0.499129}%
\pgfsetfillcolor{currentfill}%
\pgfsetfillopacity{0.700000}%
\pgfsetlinewidth{0.501875pt}%
\definecolor{currentstroke}{rgb}{1.000000,1.000000,1.000000}%
\pgfsetstrokecolor{currentstroke}%
\pgfsetstrokeopacity{0.700000}%
\pgfsetdash{}{0pt}%
\pgfpathmoveto{\pgfqpoint{2.348413in}{2.639481in}}%
\pgfpathcurveto{\pgfqpoint{2.361435in}{2.639481in}}{\pgfqpoint{2.373926in}{2.644655in}}{\pgfqpoint{2.383135in}{2.653864in}}%
\pgfpathcurveto{\pgfqpoint{2.392343in}{2.663072in}}{\pgfqpoint{2.397517in}{2.675563in}}{\pgfqpoint{2.397517in}{2.688586in}}%
\pgfpathcurveto{\pgfqpoint{2.397517in}{2.701608in}}{\pgfqpoint{2.392343in}{2.714100in}}{\pgfqpoint{2.383135in}{2.723308in}}%
\pgfpathcurveto{\pgfqpoint{2.373926in}{2.732516in}}{\pgfqpoint{2.361435in}{2.737690in}}{\pgfqpoint{2.348413in}{2.737690in}}%
\pgfpathcurveto{\pgfqpoint{2.335390in}{2.737690in}}{\pgfqpoint{2.322899in}{2.732516in}}{\pgfqpoint{2.313690in}{2.723308in}}%
\pgfpathcurveto{\pgfqpoint{2.304482in}{2.714100in}}{\pgfqpoint{2.299308in}{2.701608in}}{\pgfqpoint{2.299308in}{2.688586in}}%
\pgfpathcurveto{\pgfqpoint{2.299308in}{2.675563in}}{\pgfqpoint{2.304482in}{2.663072in}}{\pgfqpoint{2.313690in}{2.653864in}}%
\pgfpathcurveto{\pgfqpoint{2.322899in}{2.644655in}}{\pgfqpoint{2.335390in}{2.639481in}}{\pgfqpoint{2.348413in}{2.639481in}}%
\pgfpathlineto{\pgfqpoint{2.348413in}{2.639481in}}%
\pgfpathclose%
\pgfusepath{stroke,fill}%
\end{pgfscope}%
\begin{pgfscope}%
\pgfpathrectangle{\pgfqpoint{0.786164in}{0.768110in}}{\pgfqpoint{8.851069in}{7.081890in}}%
\pgfusepath{clip}%
\pgfsetbuttcap%
\pgfsetroundjoin%
\definecolor{currentfill}{rgb}{0.216210,0.351535,0.550627}%
\pgfsetfillcolor{currentfill}%
\pgfsetfillopacity{0.700000}%
\pgfsetlinewidth{0.501875pt}%
\definecolor{currentstroke}{rgb}{1.000000,1.000000,1.000000}%
\pgfsetstrokecolor{currentstroke}%
\pgfsetstrokeopacity{0.700000}%
\pgfsetdash{}{0pt}%
\pgfpathmoveto{\pgfqpoint{3.362208in}{3.887681in}}%
\pgfpathcurveto{\pgfqpoint{3.375230in}{3.887681in}}{\pgfqpoint{3.387721in}{3.892855in}}{\pgfqpoint{3.396930in}{3.902063in}}%
\pgfpathcurveto{\pgfqpoint{3.406138in}{3.911271in}}{\pgfqpoint{3.411312in}{3.923762in}}{\pgfqpoint{3.411312in}{3.936785in}}%
\pgfpathcurveto{\pgfqpoint{3.411312in}{3.949808in}}{\pgfqpoint{3.406138in}{3.962299in}}{\pgfqpoint{3.396930in}{3.971507in}}%
\pgfpathcurveto{\pgfqpoint{3.387721in}{3.980716in}}{\pgfqpoint{3.375230in}{3.985890in}}{\pgfqpoint{3.362208in}{3.985890in}}%
\pgfpathcurveto{\pgfqpoint{3.349185in}{3.985890in}}{\pgfqpoint{3.336694in}{3.980716in}}{\pgfqpoint{3.327485in}{3.971507in}}%
\pgfpathcurveto{\pgfqpoint{3.318277in}{3.962299in}}{\pgfqpoint{3.313103in}{3.949808in}}{\pgfqpoint{3.313103in}{3.936785in}}%
\pgfpathcurveto{\pgfqpoint{3.313103in}{3.923762in}}{\pgfqpoint{3.318277in}{3.911271in}}{\pgfqpoint{3.327485in}{3.902063in}}%
\pgfpathcurveto{\pgfqpoint{3.336694in}{3.892855in}}{\pgfqpoint{3.349185in}{3.887681in}}{\pgfqpoint{3.362208in}{3.887681in}}%
\pgfpathlineto{\pgfqpoint{3.362208in}{3.887681in}}%
\pgfpathclose%
\pgfusepath{stroke,fill}%
\end{pgfscope}%
\begin{pgfscope}%
\pgfpathrectangle{\pgfqpoint{0.786164in}{0.768110in}}{\pgfqpoint{8.851069in}{7.081890in}}%
\pgfusepath{clip}%
\pgfsetbuttcap%
\pgfsetroundjoin%
\definecolor{currentfill}{rgb}{0.223925,0.334994,0.548053}%
\pgfsetfillcolor{currentfill}%
\pgfsetfillopacity{0.700000}%
\pgfsetlinewidth{0.501875pt}%
\definecolor{currentstroke}{rgb}{1.000000,1.000000,1.000000}%
\pgfsetstrokecolor{currentstroke}%
\pgfsetstrokeopacity{0.700000}%
\pgfsetdash{}{0pt}%
\pgfpathmoveto{\pgfqpoint{3.435274in}{3.668698in}}%
\pgfpathcurveto{\pgfqpoint{3.448297in}{3.668698in}}{\pgfqpoint{3.460788in}{3.673872in}}{\pgfqpoint{3.469996in}{3.683081in}}%
\pgfpathcurveto{\pgfqpoint{3.479205in}{3.692289in}}{\pgfqpoint{3.484379in}{3.704780in}}{\pgfqpoint{3.484379in}{3.717803in}}%
\pgfpathcurveto{\pgfqpoint{3.484379in}{3.730826in}}{\pgfqpoint{3.479205in}{3.743317in}}{\pgfqpoint{3.469996in}{3.752525in}}%
\pgfpathcurveto{\pgfqpoint{3.460788in}{3.761733in}}{\pgfqpoint{3.448297in}{3.766907in}}{\pgfqpoint{3.435274in}{3.766907in}}%
\pgfpathcurveto{\pgfqpoint{3.422251in}{3.766907in}}{\pgfqpoint{3.409760in}{3.761733in}}{\pgfqpoint{3.400552in}{3.752525in}}%
\pgfpathcurveto{\pgfqpoint{3.391343in}{3.743317in}}{\pgfqpoint{3.386169in}{3.730826in}}{\pgfqpoint{3.386169in}{3.717803in}}%
\pgfpathcurveto{\pgfqpoint{3.386169in}{3.704780in}}{\pgfqpoint{3.391343in}{3.692289in}}{\pgfqpoint{3.400552in}{3.683081in}}%
\pgfpathcurveto{\pgfqpoint{3.409760in}{3.673872in}}{\pgfqpoint{3.422251in}{3.668698in}}{\pgfqpoint{3.435274in}{3.668698in}}%
\pgfpathlineto{\pgfqpoint{3.435274in}{3.668698in}}%
\pgfpathclose%
\pgfusepath{stroke,fill}%
\end{pgfscope}%
\begin{pgfscope}%
\pgfpathrectangle{\pgfqpoint{0.786164in}{0.768110in}}{\pgfqpoint{8.851069in}{7.081890in}}%
\pgfusepath{clip}%
\pgfsetbuttcap%
\pgfsetroundjoin%
\definecolor{currentfill}{rgb}{0.235526,0.309527,0.542944}%
\pgfsetfillcolor{currentfill}%
\pgfsetfillopacity{0.700000}%
\pgfsetlinewidth{0.501875pt}%
\definecolor{currentstroke}{rgb}{1.000000,1.000000,1.000000}%
\pgfsetstrokecolor{currentstroke}%
\pgfsetstrokeopacity{0.700000}%
\pgfsetdash{}{0pt}%
\pgfpathmoveto{\pgfqpoint{3.417007in}{3.405919in}}%
\pgfpathcurveto{\pgfqpoint{3.430030in}{3.405919in}}{\pgfqpoint{3.442521in}{3.411093in}}{\pgfqpoint{3.451730in}{3.420302in}}%
\pgfpathcurveto{\pgfqpoint{3.460938in}{3.429510in}}{\pgfqpoint{3.466112in}{3.442001in}}{\pgfqpoint{3.466112in}{3.455024in}}%
\pgfpathcurveto{\pgfqpoint{3.466112in}{3.468047in}}{\pgfqpoint{3.460938in}{3.480538in}}{\pgfqpoint{3.451730in}{3.489746in}}%
\pgfpathcurveto{\pgfqpoint{3.442521in}{3.498955in}}{\pgfqpoint{3.430030in}{3.504129in}}{\pgfqpoint{3.417007in}{3.504129in}}%
\pgfpathcurveto{\pgfqpoint{3.403985in}{3.504129in}}{\pgfqpoint{3.391494in}{3.498955in}}{\pgfqpoint{3.382285in}{3.489746in}}%
\pgfpathcurveto{\pgfqpoint{3.373077in}{3.480538in}}{\pgfqpoint{3.367903in}{3.468047in}}{\pgfqpoint{3.367903in}{3.455024in}}%
\pgfpathcurveto{\pgfqpoint{3.367903in}{3.442001in}}{\pgfqpoint{3.373077in}{3.429510in}}{\pgfqpoint{3.382285in}{3.420302in}}%
\pgfpathcurveto{\pgfqpoint{3.391494in}{3.411093in}}{\pgfqpoint{3.403985in}{3.405919in}}{\pgfqpoint{3.417007in}{3.405919in}}%
\pgfpathlineto{\pgfqpoint{3.417007in}{3.405919in}}%
\pgfpathclose%
\pgfusepath{stroke,fill}%
\end{pgfscope}%
\begin{pgfscope}%
\pgfpathrectangle{\pgfqpoint{0.786164in}{0.768110in}}{\pgfqpoint{8.851069in}{7.081890in}}%
\pgfusepath{clip}%
\pgfsetbuttcap%
\pgfsetroundjoin%
\definecolor{currentfill}{rgb}{0.225863,0.330805,0.547314}%
\pgfsetfillcolor{currentfill}%
\pgfsetfillopacity{0.700000}%
\pgfsetlinewidth{0.501875pt}%
\definecolor{currentstroke}{rgb}{1.000000,1.000000,1.000000}%
\pgfsetstrokecolor{currentstroke}%
\pgfsetstrokeopacity{0.700000}%
\pgfsetdash{}{0pt}%
\pgfpathmoveto{\pgfqpoint{4.065471in}{3.778189in}}%
\pgfpathcurveto{\pgfqpoint{4.078493in}{3.778189in}}{\pgfqpoint{4.090985in}{3.783363in}}{\pgfqpoint{4.100193in}{3.792572in}}%
\pgfpathcurveto{\pgfqpoint{4.109401in}{3.801780in}}{\pgfqpoint{4.114575in}{3.814271in}}{\pgfqpoint{4.114575in}{3.827294in}}%
\pgfpathcurveto{\pgfqpoint{4.114575in}{3.840317in}}{\pgfqpoint{4.109401in}{3.852808in}}{\pgfqpoint{4.100193in}{3.862016in}}%
\pgfpathcurveto{\pgfqpoint{4.090985in}{3.871225in}}{\pgfqpoint{4.078493in}{3.876399in}}{\pgfqpoint{4.065471in}{3.876399in}}%
\pgfpathcurveto{\pgfqpoint{4.052448in}{3.876399in}}{\pgfqpoint{4.039957in}{3.871225in}}{\pgfqpoint{4.030749in}{3.862016in}}%
\pgfpathcurveto{\pgfqpoint{4.021540in}{3.852808in}}{\pgfqpoint{4.016366in}{3.840317in}}{\pgfqpoint{4.016366in}{3.827294in}}%
\pgfpathcurveto{\pgfqpoint{4.016366in}{3.814271in}}{\pgfqpoint{4.021540in}{3.801780in}}{\pgfqpoint{4.030749in}{3.792572in}}%
\pgfpathcurveto{\pgfqpoint{4.039957in}{3.783363in}}{\pgfqpoint{4.052448in}{3.778189in}}{\pgfqpoint{4.065471in}{3.778189in}}%
\pgfpathlineto{\pgfqpoint{4.065471in}{3.778189in}}%
\pgfpathclose%
\pgfusepath{stroke,fill}%
\end{pgfscope}%
\begin{pgfscope}%
\pgfpathrectangle{\pgfqpoint{0.786164in}{0.768110in}}{\pgfqpoint{8.851069in}{7.081890in}}%
\pgfusepath{clip}%
\pgfsetbuttcap%
\pgfsetroundjoin%
\definecolor{currentfill}{rgb}{0.214298,0.355619,0.551184}%
\pgfsetfillcolor{currentfill}%
\pgfsetfillopacity{0.700000}%
\pgfsetlinewidth{0.501875pt}%
\definecolor{currentstroke}{rgb}{1.000000,1.000000,1.000000}%
\pgfsetstrokecolor{currentstroke}%
\pgfsetstrokeopacity{0.700000}%
\pgfsetdash{}{0pt}%
\pgfpathmoveto{\pgfqpoint{3.544873in}{3.405919in}}%
\pgfpathcurveto{\pgfqpoint{3.557896in}{3.405919in}}{\pgfqpoint{3.570387in}{3.411093in}}{\pgfqpoint{3.579596in}{3.420302in}}%
\pgfpathcurveto{\pgfqpoint{3.588804in}{3.429510in}}{\pgfqpoint{3.593978in}{3.442001in}}{\pgfqpoint{3.593978in}{3.455024in}}%
\pgfpathcurveto{\pgfqpoint{3.593978in}{3.468047in}}{\pgfqpoint{3.588804in}{3.480538in}}{\pgfqpoint{3.579596in}{3.489746in}}%
\pgfpathcurveto{\pgfqpoint{3.570387in}{3.498955in}}{\pgfqpoint{3.557896in}{3.504129in}}{\pgfqpoint{3.544873in}{3.504129in}}%
\pgfpathcurveto{\pgfqpoint{3.531851in}{3.504129in}}{\pgfqpoint{3.519360in}{3.498955in}}{\pgfqpoint{3.510151in}{3.489746in}}%
\pgfpathcurveto{\pgfqpoint{3.500943in}{3.480538in}}{\pgfqpoint{3.495769in}{3.468047in}}{\pgfqpoint{3.495769in}{3.455024in}}%
\pgfpathcurveto{\pgfqpoint{3.495769in}{3.442001in}}{\pgfqpoint{3.500943in}{3.429510in}}{\pgfqpoint{3.510151in}{3.420302in}}%
\pgfpathcurveto{\pgfqpoint{3.519360in}{3.411093in}}{\pgfqpoint{3.531851in}{3.405919in}}{\pgfqpoint{3.544873in}{3.405919in}}%
\pgfpathlineto{\pgfqpoint{3.544873in}{3.405919in}}%
\pgfpathclose%
\pgfusepath{stroke,fill}%
\end{pgfscope}%
\begin{pgfscope}%
\pgfpathrectangle{\pgfqpoint{0.786164in}{0.768110in}}{\pgfqpoint{8.851069in}{7.081890in}}%
\pgfusepath{clip}%
\pgfsetbuttcap%
\pgfsetroundjoin%
\definecolor{currentfill}{rgb}{0.185556,0.418570,0.556753}%
\pgfsetfillcolor{currentfill}%
\pgfsetfillopacity{0.700000}%
\pgfsetlinewidth{0.501875pt}%
\definecolor{currentstroke}{rgb}{1.000000,1.000000,1.000000}%
\pgfsetstrokecolor{currentstroke}%
\pgfsetstrokeopacity{0.700000}%
\pgfsetdash{}{0pt}%
\pgfpathmoveto{\pgfqpoint{2.832477in}{2.705176in}}%
\pgfpathcurveto{\pgfqpoint{2.845500in}{2.705176in}}{\pgfqpoint{2.857991in}{2.710350in}}{\pgfqpoint{2.867199in}{2.719558in}}%
\pgfpathcurveto{\pgfqpoint{2.876408in}{2.728767in}}{\pgfqpoint{2.881582in}{2.741258in}}{\pgfqpoint{2.881582in}{2.754280in}}%
\pgfpathcurveto{\pgfqpoint{2.881582in}{2.767303in}}{\pgfqpoint{2.876408in}{2.779794in}}{\pgfqpoint{2.867199in}{2.789003in}}%
\pgfpathcurveto{\pgfqpoint{2.857991in}{2.798211in}}{\pgfqpoint{2.845500in}{2.803385in}}{\pgfqpoint{2.832477in}{2.803385in}}%
\pgfpathcurveto{\pgfqpoint{2.819454in}{2.803385in}}{\pgfqpoint{2.806963in}{2.798211in}}{\pgfqpoint{2.797755in}{2.789003in}}%
\pgfpathcurveto{\pgfqpoint{2.788546in}{2.779794in}}{\pgfqpoint{2.783372in}{2.767303in}}{\pgfqpoint{2.783372in}{2.754280in}}%
\pgfpathcurveto{\pgfqpoint{2.783372in}{2.741258in}}{\pgfqpoint{2.788546in}{2.728767in}}{\pgfqpoint{2.797755in}{2.719558in}}%
\pgfpathcurveto{\pgfqpoint{2.806963in}{2.710350in}}{\pgfqpoint{2.819454in}{2.705176in}}{\pgfqpoint{2.832477in}{2.705176in}}%
\pgfpathlineto{\pgfqpoint{2.832477in}{2.705176in}}%
\pgfpathclose%
\pgfusepath{stroke,fill}%
\end{pgfscope}%
\begin{pgfscope}%
\pgfpathrectangle{\pgfqpoint{0.786164in}{0.768110in}}{\pgfqpoint{8.851069in}{7.081890in}}%
\pgfusepath{clip}%
\pgfsetbuttcap%
\pgfsetroundjoin%
\definecolor{currentfill}{rgb}{0.174274,0.445044,0.557792}%
\pgfsetfillcolor{currentfill}%
\pgfsetfillopacity{0.700000}%
\pgfsetlinewidth{0.501875pt}%
\definecolor{currentstroke}{rgb}{1.000000,1.000000,1.000000}%
\pgfsetstrokecolor{currentstroke}%
\pgfsetstrokeopacity{0.700000}%
\pgfsetdash{}{0pt}%
\pgfpathmoveto{\pgfqpoint{3.873672in}{3.734393in}}%
\pgfpathcurveto{\pgfqpoint{3.886694in}{3.734393in}}{\pgfqpoint{3.899185in}{3.739567in}}{\pgfqpoint{3.908394in}{3.748775in}}%
\pgfpathcurveto{\pgfqpoint{3.917602in}{3.757984in}}{\pgfqpoint{3.922776in}{3.770475in}}{\pgfqpoint{3.922776in}{3.783498in}}%
\pgfpathcurveto{\pgfqpoint{3.922776in}{3.796520in}}{\pgfqpoint{3.917602in}{3.809011in}}{\pgfqpoint{3.908394in}{3.818220in}}%
\pgfpathcurveto{\pgfqpoint{3.899185in}{3.827428in}}{\pgfqpoint{3.886694in}{3.832602in}}{\pgfqpoint{3.873672in}{3.832602in}}%
\pgfpathcurveto{\pgfqpoint{3.860649in}{3.832602in}}{\pgfqpoint{3.848158in}{3.827428in}}{\pgfqpoint{3.838949in}{3.818220in}}%
\pgfpathcurveto{\pgfqpoint{3.829741in}{3.809011in}}{\pgfqpoint{3.824567in}{3.796520in}}{\pgfqpoint{3.824567in}{3.783498in}}%
\pgfpathcurveto{\pgfqpoint{3.824567in}{3.770475in}}{\pgfqpoint{3.829741in}{3.757984in}}{\pgfqpoint{3.838949in}{3.748775in}}%
\pgfpathcurveto{\pgfqpoint{3.848158in}{3.739567in}}{\pgfqpoint{3.860649in}{3.734393in}}{\pgfqpoint{3.873672in}{3.734393in}}%
\pgfpathlineto{\pgfqpoint{3.873672in}{3.734393in}}%
\pgfpathclose%
\pgfusepath{stroke,fill}%
\end{pgfscope}%
\begin{pgfscope}%
\pgfpathrectangle{\pgfqpoint{0.786164in}{0.768110in}}{\pgfqpoint{8.851069in}{7.081890in}}%
\pgfusepath{clip}%
\pgfsetbuttcap%
\pgfsetroundjoin%
\definecolor{currentfill}{rgb}{0.169646,0.456262,0.558030}%
\pgfsetfillcolor{currentfill}%
\pgfsetfillopacity{0.700000}%
\pgfsetlinewidth{0.501875pt}%
\definecolor{currentstroke}{rgb}{1.000000,1.000000,1.000000}%
\pgfsetstrokecolor{currentstroke}%
\pgfsetstrokeopacity{0.700000}%
\pgfsetdash{}{0pt}%
\pgfpathmoveto{\pgfqpoint{3.791472in}{3.997172in}}%
\pgfpathcurveto{\pgfqpoint{3.804495in}{3.997172in}}{\pgfqpoint{3.816986in}{4.002346in}}{\pgfqpoint{3.826194in}{4.011554in}}%
\pgfpathcurveto{\pgfqpoint{3.835403in}{4.020763in}}{\pgfqpoint{3.840577in}{4.033254in}}{\pgfqpoint{3.840577in}{4.046276in}}%
\pgfpathcurveto{\pgfqpoint{3.840577in}{4.059299in}}{\pgfqpoint{3.835403in}{4.071790in}}{\pgfqpoint{3.826194in}{4.080999in}}%
\pgfpathcurveto{\pgfqpoint{3.816986in}{4.090207in}}{\pgfqpoint{3.804495in}{4.095381in}}{\pgfqpoint{3.791472in}{4.095381in}}%
\pgfpathcurveto{\pgfqpoint{3.778449in}{4.095381in}}{\pgfqpoint{3.765958in}{4.090207in}}{\pgfqpoint{3.756750in}{4.080999in}}%
\pgfpathcurveto{\pgfqpoint{3.747541in}{4.071790in}}{\pgfqpoint{3.742367in}{4.059299in}}{\pgfqpoint{3.742367in}{4.046276in}}%
\pgfpathcurveto{\pgfqpoint{3.742367in}{4.033254in}}{\pgfqpoint{3.747541in}{4.020763in}}{\pgfqpoint{3.756750in}{4.011554in}}%
\pgfpathcurveto{\pgfqpoint{3.765958in}{4.002346in}}{\pgfqpoint{3.778449in}{3.997172in}}{\pgfqpoint{3.791472in}{3.997172in}}%
\pgfpathlineto{\pgfqpoint{3.791472in}{3.997172in}}%
\pgfpathclose%
\pgfusepath{stroke,fill}%
\end{pgfscope}%
\begin{pgfscope}%
\pgfpathrectangle{\pgfqpoint{0.786164in}{0.768110in}}{\pgfqpoint{8.851069in}{7.081890in}}%
\pgfusepath{clip}%
\pgfsetbuttcap%
\pgfsetroundjoin%
\definecolor{currentfill}{rgb}{0.169646,0.456262,0.558030}%
\pgfsetfillcolor{currentfill}%
\pgfsetfillopacity{0.700000}%
\pgfsetlinewidth{0.501875pt}%
\definecolor{currentstroke}{rgb}{1.000000,1.000000,1.000000}%
\pgfsetstrokecolor{currentstroke}%
\pgfsetstrokeopacity{0.700000}%
\pgfsetdash{}{0pt}%
\pgfpathmoveto{\pgfqpoint{3.462674in}{4.084765in}}%
\pgfpathcurveto{\pgfqpoint{3.475696in}{4.084765in}}{\pgfqpoint{3.488188in}{4.089939in}}{\pgfqpoint{3.497396in}{4.099147in}}%
\pgfpathcurveto{\pgfqpoint{3.506604in}{4.108356in}}{\pgfqpoint{3.511778in}{4.120847in}}{\pgfqpoint{3.511778in}{4.133869in}}%
\pgfpathcurveto{\pgfqpoint{3.511778in}{4.146892in}}{\pgfqpoint{3.506604in}{4.159383in}}{\pgfqpoint{3.497396in}{4.168592in}}%
\pgfpathcurveto{\pgfqpoint{3.488188in}{4.177800in}}{\pgfqpoint{3.475696in}{4.182974in}}{\pgfqpoint{3.462674in}{4.182974in}}%
\pgfpathcurveto{\pgfqpoint{3.449651in}{4.182974in}}{\pgfqpoint{3.437160in}{4.177800in}}{\pgfqpoint{3.427952in}{4.168592in}}%
\pgfpathcurveto{\pgfqpoint{3.418743in}{4.159383in}}{\pgfqpoint{3.413569in}{4.146892in}}{\pgfqpoint{3.413569in}{4.133869in}}%
\pgfpathcurveto{\pgfqpoint{3.413569in}{4.120847in}}{\pgfqpoint{3.418743in}{4.108356in}}{\pgfqpoint{3.427952in}{4.099147in}}%
\pgfpathcurveto{\pgfqpoint{3.437160in}{4.089939in}}{\pgfqpoint{3.449651in}{4.084765in}}{\pgfqpoint{3.462674in}{4.084765in}}%
\pgfpathlineto{\pgfqpoint{3.462674in}{4.084765in}}%
\pgfpathclose%
\pgfusepath{stroke,fill}%
\end{pgfscope}%
\begin{pgfscope}%
\pgfpathrectangle{\pgfqpoint{0.786164in}{0.768110in}}{\pgfqpoint{8.851069in}{7.081890in}}%
\pgfusepath{clip}%
\pgfsetbuttcap%
\pgfsetroundjoin%
\definecolor{currentfill}{rgb}{0.169646,0.456262,0.558030}%
\pgfsetfillcolor{currentfill}%
\pgfsetfillopacity{0.700000}%
\pgfsetlinewidth{0.501875pt}%
\definecolor{currentstroke}{rgb}{1.000000,1.000000,1.000000}%
\pgfsetstrokecolor{currentstroke}%
\pgfsetstrokeopacity{0.700000}%
\pgfsetdash{}{0pt}%
\pgfpathmoveto{\pgfqpoint{3.837139in}{4.588424in}}%
\pgfpathcurveto{\pgfqpoint{3.850161in}{4.588424in}}{\pgfqpoint{3.862652in}{4.593598in}}{\pgfqpoint{3.871861in}{4.602807in}}%
\pgfpathcurveto{\pgfqpoint{3.881069in}{4.612015in}}{\pgfqpoint{3.886243in}{4.624506in}}{\pgfqpoint{3.886243in}{4.637529in}}%
\pgfpathcurveto{\pgfqpoint{3.886243in}{4.650551in}}{\pgfqpoint{3.881069in}{4.663043in}}{\pgfqpoint{3.871861in}{4.672251in}}%
\pgfpathcurveto{\pgfqpoint{3.862652in}{4.681459in}}{\pgfqpoint{3.850161in}{4.686633in}}{\pgfqpoint{3.837139in}{4.686633in}}%
\pgfpathcurveto{\pgfqpoint{3.824116in}{4.686633in}}{\pgfqpoint{3.811625in}{4.681459in}}{\pgfqpoint{3.802416in}{4.672251in}}%
\pgfpathcurveto{\pgfqpoint{3.793208in}{4.663043in}}{\pgfqpoint{3.788034in}{4.650551in}}{\pgfqpoint{3.788034in}{4.637529in}}%
\pgfpathcurveto{\pgfqpoint{3.788034in}{4.624506in}}{\pgfqpoint{3.793208in}{4.612015in}}{\pgfqpoint{3.802416in}{4.602807in}}%
\pgfpathcurveto{\pgfqpoint{3.811625in}{4.593598in}}{\pgfqpoint{3.824116in}{4.588424in}}{\pgfqpoint{3.837139in}{4.588424in}}%
\pgfpathlineto{\pgfqpoint{3.837139in}{4.588424in}}%
\pgfpathclose%
\pgfusepath{stroke,fill}%
\end{pgfscope}%
\begin{pgfscope}%
\pgfpathrectangle{\pgfqpoint{0.786164in}{0.768110in}}{\pgfqpoint{8.851069in}{7.081890in}}%
\pgfusepath{clip}%
\pgfsetbuttcap%
\pgfsetroundjoin%
\definecolor{currentfill}{rgb}{0.165117,0.467423,0.558141}%
\pgfsetfillcolor{currentfill}%
\pgfsetfillopacity{0.700000}%
\pgfsetlinewidth{0.501875pt}%
\definecolor{currentstroke}{rgb}{1.000000,1.000000,1.000000}%
\pgfsetstrokecolor{currentstroke}%
\pgfsetstrokeopacity{0.700000}%
\pgfsetdash{}{0pt}%
\pgfpathmoveto{\pgfqpoint{3.681873in}{4.588424in}}%
\pgfpathcurveto{\pgfqpoint{3.694895in}{4.588424in}}{\pgfqpoint{3.707386in}{4.593598in}}{\pgfqpoint{3.716595in}{4.602807in}}%
\pgfpathcurveto{\pgfqpoint{3.725803in}{4.612015in}}{\pgfqpoint{3.730977in}{4.624506in}}{\pgfqpoint{3.730977in}{4.637529in}}%
\pgfpathcurveto{\pgfqpoint{3.730977in}{4.650551in}}{\pgfqpoint{3.725803in}{4.663043in}}{\pgfqpoint{3.716595in}{4.672251in}}%
\pgfpathcurveto{\pgfqpoint{3.707386in}{4.681459in}}{\pgfqpoint{3.694895in}{4.686633in}}{\pgfqpoint{3.681873in}{4.686633in}}%
\pgfpathcurveto{\pgfqpoint{3.668850in}{4.686633in}}{\pgfqpoint{3.656359in}{4.681459in}}{\pgfqpoint{3.647150in}{4.672251in}}%
\pgfpathcurveto{\pgfqpoint{3.637942in}{4.663043in}}{\pgfqpoint{3.632768in}{4.650551in}}{\pgfqpoint{3.632768in}{4.637529in}}%
\pgfpathcurveto{\pgfqpoint{3.632768in}{4.624506in}}{\pgfqpoint{3.637942in}{4.612015in}}{\pgfqpoint{3.647150in}{4.602807in}}%
\pgfpathcurveto{\pgfqpoint{3.656359in}{4.593598in}}{\pgfqpoint{3.668850in}{4.588424in}}{\pgfqpoint{3.681873in}{4.588424in}}%
\pgfpathlineto{\pgfqpoint{3.681873in}{4.588424in}}%
\pgfpathclose%
\pgfusepath{stroke,fill}%
\end{pgfscope}%
\begin{pgfscope}%
\pgfpathrectangle{\pgfqpoint{0.786164in}{0.768110in}}{\pgfqpoint{8.851069in}{7.081890in}}%
\pgfusepath{clip}%
\pgfsetbuttcap%
\pgfsetroundjoin%
\definecolor{currentfill}{rgb}{0.149039,0.508051,0.557250}%
\pgfsetfillcolor{currentfill}%
\pgfsetfillopacity{0.700000}%
\pgfsetlinewidth{0.501875pt}%
\definecolor{currentstroke}{rgb}{1.000000,1.000000,1.000000}%
\pgfsetstrokecolor{currentstroke}%
\pgfsetstrokeopacity{0.700000}%
\pgfsetdash{}{0pt}%
\pgfpathmoveto{\pgfqpoint{3.179542in}{4.172358in}}%
\pgfpathcurveto{\pgfqpoint{3.192565in}{4.172358in}}{\pgfqpoint{3.205056in}{4.177532in}}{\pgfqpoint{3.214264in}{4.186740in}}%
\pgfpathcurveto{\pgfqpoint{3.223473in}{4.195948in}}{\pgfqpoint{3.228647in}{4.208440in}}{\pgfqpoint{3.228647in}{4.221462in}}%
\pgfpathcurveto{\pgfqpoint{3.228647in}{4.234485in}}{\pgfqpoint{3.223473in}{4.246976in}}{\pgfqpoint{3.214264in}{4.256184in}}%
\pgfpathcurveto{\pgfqpoint{3.205056in}{4.265393in}}{\pgfqpoint{3.192565in}{4.270567in}}{\pgfqpoint{3.179542in}{4.270567in}}%
\pgfpathcurveto{\pgfqpoint{3.166519in}{4.270567in}}{\pgfqpoint{3.154028in}{4.265393in}}{\pgfqpoint{3.144820in}{4.256184in}}%
\pgfpathcurveto{\pgfqpoint{3.135611in}{4.246976in}}{\pgfqpoint{3.130437in}{4.234485in}}{\pgfqpoint{3.130437in}{4.221462in}}%
\pgfpathcurveto{\pgfqpoint{3.130437in}{4.208440in}}{\pgfqpoint{3.135611in}{4.195948in}}{\pgfqpoint{3.144820in}{4.186740in}}%
\pgfpathcurveto{\pgfqpoint{3.154028in}{4.177532in}}{\pgfqpoint{3.166519in}{4.172358in}}{\pgfqpoint{3.179542in}{4.172358in}}%
\pgfpathlineto{\pgfqpoint{3.179542in}{4.172358in}}%
\pgfpathclose%
\pgfusepath{stroke,fill}%
\end{pgfscope}%
\begin{pgfscope}%
\pgfpathrectangle{\pgfqpoint{0.786164in}{0.768110in}}{\pgfqpoint{8.851069in}{7.081890in}}%
\pgfusepath{clip}%
\pgfsetbuttcap%
\pgfsetroundjoin%
\definecolor{currentfill}{rgb}{0.147607,0.511733,0.557049}%
\pgfsetfillcolor{currentfill}%
\pgfsetfillopacity{0.700000}%
\pgfsetlinewidth{0.501875pt}%
\definecolor{currentstroke}{rgb}{1.000000,1.000000,1.000000}%
\pgfsetstrokecolor{currentstroke}%
\pgfsetstrokeopacity{0.700000}%
\pgfsetdash{}{0pt}%
\pgfpathmoveto{\pgfqpoint{4.010671in}{4.610322in}}%
\pgfpathcurveto{\pgfqpoint{4.023694in}{4.610322in}}{\pgfqpoint{4.036185in}{4.615496in}}{\pgfqpoint{4.045393in}{4.624705in}}%
\pgfpathcurveto{\pgfqpoint{4.054602in}{4.633913in}}{\pgfqpoint{4.059776in}{4.646404in}}{\pgfqpoint{4.059776in}{4.659427in}}%
\pgfpathcurveto{\pgfqpoint{4.059776in}{4.672450in}}{\pgfqpoint{4.054602in}{4.684941in}}{\pgfqpoint{4.045393in}{4.694149in}}%
\pgfpathcurveto{\pgfqpoint{4.036185in}{4.703358in}}{\pgfqpoint{4.023694in}{4.708532in}}{\pgfqpoint{4.010671in}{4.708532in}}%
\pgfpathcurveto{\pgfqpoint{3.997648in}{4.708532in}}{\pgfqpoint{3.985157in}{4.703358in}}{\pgfqpoint{3.975949in}{4.694149in}}%
\pgfpathcurveto{\pgfqpoint{3.966740in}{4.684941in}}{\pgfqpoint{3.961566in}{4.672450in}}{\pgfqpoint{3.961566in}{4.659427in}}%
\pgfpathcurveto{\pgfqpoint{3.961566in}{4.646404in}}{\pgfqpoint{3.966740in}{4.633913in}}{\pgfqpoint{3.975949in}{4.624705in}}%
\pgfpathcurveto{\pgfqpoint{3.985157in}{4.615496in}}{\pgfqpoint{3.997648in}{4.610322in}}{\pgfqpoint{4.010671in}{4.610322in}}%
\pgfpathlineto{\pgfqpoint{4.010671in}{4.610322in}}%
\pgfpathclose%
\pgfusepath{stroke,fill}%
\end{pgfscope}%
\begin{pgfscope}%
\pgfpathrectangle{\pgfqpoint{0.786164in}{0.768110in}}{\pgfqpoint{8.851069in}{7.081890in}}%
\pgfusepath{clip}%
\pgfsetbuttcap%
\pgfsetroundjoin%
\definecolor{currentfill}{rgb}{0.146180,0.515413,0.556823}%
\pgfsetfillcolor{currentfill}%
\pgfsetfillopacity{0.700000}%
\pgfsetlinewidth{0.501875pt}%
\definecolor{currentstroke}{rgb}{1.000000,1.000000,1.000000}%
\pgfsetstrokecolor{currentstroke}%
\pgfsetstrokeopacity{0.700000}%
\pgfsetdash{}{0pt}%
\pgfpathmoveto{\pgfqpoint{3.882805in}{4.785508in}}%
\pgfpathcurveto{\pgfqpoint{3.895828in}{4.785508in}}{\pgfqpoint{3.908319in}{4.790682in}}{\pgfqpoint{3.917527in}{4.799891in}}%
\pgfpathcurveto{\pgfqpoint{3.926736in}{4.809099in}}{\pgfqpoint{3.931910in}{4.821590in}}{\pgfqpoint{3.931910in}{4.834613in}}%
\pgfpathcurveto{\pgfqpoint{3.931910in}{4.847636in}}{\pgfqpoint{3.926736in}{4.860127in}}{\pgfqpoint{3.917527in}{4.869335in}}%
\pgfpathcurveto{\pgfqpoint{3.908319in}{4.878544in}}{\pgfqpoint{3.895828in}{4.883718in}}{\pgfqpoint{3.882805in}{4.883718in}}%
\pgfpathcurveto{\pgfqpoint{3.869782in}{4.883718in}}{\pgfqpoint{3.857291in}{4.878544in}}{\pgfqpoint{3.848083in}{4.869335in}}%
\pgfpathcurveto{\pgfqpoint{3.838874in}{4.860127in}}{\pgfqpoint{3.833700in}{4.847636in}}{\pgfqpoint{3.833700in}{4.834613in}}%
\pgfpathcurveto{\pgfqpoint{3.833700in}{4.821590in}}{\pgfqpoint{3.838874in}{4.809099in}}{\pgfqpoint{3.848083in}{4.799891in}}%
\pgfpathcurveto{\pgfqpoint{3.857291in}{4.790682in}}{\pgfqpoint{3.869782in}{4.785508in}}{\pgfqpoint{3.882805in}{4.785508in}}%
\pgfpathlineto{\pgfqpoint{3.882805in}{4.785508in}}%
\pgfpathclose%
\pgfusepath{stroke,fill}%
\end{pgfscope}%
\begin{pgfscope}%
\pgfpathrectangle{\pgfqpoint{0.786164in}{0.768110in}}{\pgfqpoint{8.851069in}{7.081890in}}%
\pgfusepath{clip}%
\pgfsetbuttcap%
\pgfsetroundjoin%
\definecolor{currentfill}{rgb}{0.143343,0.522773,0.556295}%
\pgfsetfillcolor{currentfill}%
\pgfsetfillopacity{0.700000}%
\pgfsetlinewidth{0.501875pt}%
\definecolor{currentstroke}{rgb}{1.000000,1.000000,1.000000}%
\pgfsetstrokecolor{currentstroke}%
\pgfsetstrokeopacity{0.700000}%
\pgfsetdash{}{0pt}%
\pgfpathmoveto{\pgfqpoint{3.745806in}{4.938796in}}%
\pgfpathcurveto{\pgfqpoint{3.758828in}{4.938796in}}{\pgfqpoint{3.771319in}{4.943970in}}{\pgfqpoint{3.780528in}{4.953178in}}%
\pgfpathcurveto{\pgfqpoint{3.789736in}{4.962387in}}{\pgfqpoint{3.794910in}{4.974878in}}{\pgfqpoint{3.794910in}{4.987901in}}%
\pgfpathcurveto{\pgfqpoint{3.794910in}{5.000923in}}{\pgfqpoint{3.789736in}{5.013414in}}{\pgfqpoint{3.780528in}{5.022623in}}%
\pgfpathcurveto{\pgfqpoint{3.771319in}{5.031831in}}{\pgfqpoint{3.758828in}{5.037005in}}{\pgfqpoint{3.745806in}{5.037005in}}%
\pgfpathcurveto{\pgfqpoint{3.732783in}{5.037005in}}{\pgfqpoint{3.720292in}{5.031831in}}{\pgfqpoint{3.711083in}{5.022623in}}%
\pgfpathcurveto{\pgfqpoint{3.701875in}{5.013414in}}{\pgfqpoint{3.696701in}{5.000923in}}{\pgfqpoint{3.696701in}{4.987901in}}%
\pgfpathcurveto{\pgfqpoint{3.696701in}{4.974878in}}{\pgfqpoint{3.701875in}{4.962387in}}{\pgfqpoint{3.711083in}{4.953178in}}%
\pgfpathcurveto{\pgfqpoint{3.720292in}{4.943970in}}{\pgfqpoint{3.732783in}{4.938796in}}{\pgfqpoint{3.745806in}{4.938796in}}%
\pgfpathlineto{\pgfqpoint{3.745806in}{4.938796in}}%
\pgfpathclose%
\pgfusepath{stroke,fill}%
\end{pgfscope}%
\begin{pgfscope}%
\pgfpathrectangle{\pgfqpoint{0.786164in}{0.768110in}}{\pgfqpoint{8.851069in}{7.081890in}}%
\pgfusepath{clip}%
\pgfsetbuttcap%
\pgfsetroundjoin%
\definecolor{currentfill}{rgb}{0.135066,0.544853,0.554029}%
\pgfsetfillcolor{currentfill}%
\pgfsetfillopacity{0.700000}%
\pgfsetlinewidth{0.501875pt}%
\definecolor{currentstroke}{rgb}{1.000000,1.000000,1.000000}%
\pgfsetstrokecolor{currentstroke}%
\pgfsetstrokeopacity{0.700000}%
\pgfsetdash{}{0pt}%
\pgfpathmoveto{\pgfqpoint{3.170409in}{3.997172in}}%
\pgfpathcurveto{\pgfqpoint{3.183431in}{3.997172in}}{\pgfqpoint{3.195922in}{4.002346in}}{\pgfqpoint{3.205131in}{4.011554in}}%
\pgfpathcurveto{\pgfqpoint{3.214339in}{4.020763in}}{\pgfqpoint{3.219513in}{4.033254in}}{\pgfqpoint{3.219513in}{4.046276in}}%
\pgfpathcurveto{\pgfqpoint{3.219513in}{4.059299in}}{\pgfqpoint{3.214339in}{4.071790in}}{\pgfqpoint{3.205131in}{4.080999in}}%
\pgfpathcurveto{\pgfqpoint{3.195922in}{4.090207in}}{\pgfqpoint{3.183431in}{4.095381in}}{\pgfqpoint{3.170409in}{4.095381in}}%
\pgfpathcurveto{\pgfqpoint{3.157386in}{4.095381in}}{\pgfqpoint{3.144895in}{4.090207in}}{\pgfqpoint{3.135686in}{4.080999in}}%
\pgfpathcurveto{\pgfqpoint{3.126478in}{4.071790in}}{\pgfqpoint{3.121304in}{4.059299in}}{\pgfqpoint{3.121304in}{4.046276in}}%
\pgfpathcurveto{\pgfqpoint{3.121304in}{4.033254in}}{\pgfqpoint{3.126478in}{4.020763in}}{\pgfqpoint{3.135686in}{4.011554in}}%
\pgfpathcurveto{\pgfqpoint{3.144895in}{4.002346in}}{\pgfqpoint{3.157386in}{3.997172in}}{\pgfqpoint{3.170409in}{3.997172in}}%
\pgfpathlineto{\pgfqpoint{3.170409in}{3.997172in}}%
\pgfpathclose%
\pgfusepath{stroke,fill}%
\end{pgfscope}%
\begin{pgfscope}%
\pgfpathrectangle{\pgfqpoint{0.786164in}{0.768110in}}{\pgfqpoint{8.851069in}{7.081890in}}%
\pgfusepath{clip}%
\pgfsetbuttcap%
\pgfsetroundjoin%
\definecolor{currentfill}{rgb}{0.143343,0.522773,0.556295}%
\pgfsetfillcolor{currentfill}%
\pgfsetfillopacity{0.700000}%
\pgfsetlinewidth{0.501875pt}%
\definecolor{currentstroke}{rgb}{1.000000,1.000000,1.000000}%
\pgfsetstrokecolor{currentstroke}%
\pgfsetstrokeopacity{0.700000}%
\pgfsetdash{}{0pt}%
\pgfpathmoveto{\pgfqpoint{2.841610in}{3.099344in}}%
\pgfpathcurveto{\pgfqpoint{2.854633in}{3.099344in}}{\pgfqpoint{2.867124in}{3.104518in}}{\pgfqpoint{2.876332in}{3.113726in}}%
\pgfpathcurveto{\pgfqpoint{2.885541in}{3.122935in}}{\pgfqpoint{2.890715in}{3.135426in}}{\pgfqpoint{2.890715in}{3.148449in}}%
\pgfpathcurveto{\pgfqpoint{2.890715in}{3.161471in}}{\pgfqpoint{2.885541in}{3.173962in}}{\pgfqpoint{2.876332in}{3.183171in}}%
\pgfpathcurveto{\pgfqpoint{2.867124in}{3.192379in}}{\pgfqpoint{2.854633in}{3.197553in}}{\pgfqpoint{2.841610in}{3.197553in}}%
\pgfpathcurveto{\pgfqpoint{2.828588in}{3.197553in}}{\pgfqpoint{2.816096in}{3.192379in}}{\pgfqpoint{2.806888in}{3.183171in}}%
\pgfpathcurveto{\pgfqpoint{2.797680in}{3.173962in}}{\pgfqpoint{2.792506in}{3.161471in}}{\pgfqpoint{2.792506in}{3.148449in}}%
\pgfpathcurveto{\pgfqpoint{2.792506in}{3.135426in}}{\pgfqpoint{2.797680in}{3.122935in}}{\pgfqpoint{2.806888in}{3.113726in}}%
\pgfpathcurveto{\pgfqpoint{2.816096in}{3.104518in}}{\pgfqpoint{2.828588in}{3.099344in}}{\pgfqpoint{2.841610in}{3.099344in}}%
\pgfpathlineto{\pgfqpoint{2.841610in}{3.099344in}}%
\pgfpathclose%
\pgfusepath{stroke,fill}%
\end{pgfscope}%
\begin{pgfscope}%
\pgfpathrectangle{\pgfqpoint{0.786164in}{0.768110in}}{\pgfqpoint{8.851069in}{7.081890in}}%
\pgfusepath{clip}%
\pgfsetbuttcap%
\pgfsetroundjoin%
\definecolor{currentfill}{rgb}{0.121380,0.629492,0.531973}%
\pgfsetfillcolor{currentfill}%
\pgfsetfillopacity{0.700000}%
\pgfsetlinewidth{0.501875pt}%
\definecolor{currentstroke}{rgb}{1.000000,1.000000,1.000000}%
\pgfsetstrokecolor{currentstroke}%
\pgfsetstrokeopacity{0.700000}%
\pgfsetdash{}{0pt}%
\pgfpathmoveto{\pgfqpoint{2.923810in}{3.252632in}}%
\pgfpathcurveto{\pgfqpoint{2.936833in}{3.252632in}}{\pgfqpoint{2.949324in}{3.257806in}}{\pgfqpoint{2.958532in}{3.267014in}}%
\pgfpathcurveto{\pgfqpoint{2.967740in}{3.276223in}}{\pgfqpoint{2.972914in}{3.288714in}}{\pgfqpoint{2.972914in}{3.301736in}}%
\pgfpathcurveto{\pgfqpoint{2.972914in}{3.314759in}}{\pgfqpoint{2.967740in}{3.327250in}}{\pgfqpoint{2.958532in}{3.336459in}}%
\pgfpathcurveto{\pgfqpoint{2.949324in}{3.345667in}}{\pgfqpoint{2.936833in}{3.350841in}}{\pgfqpoint{2.923810in}{3.350841in}}%
\pgfpathcurveto{\pgfqpoint{2.910787in}{3.350841in}}{\pgfqpoint{2.898296in}{3.345667in}}{\pgfqpoint{2.889088in}{3.336459in}}%
\pgfpathcurveto{\pgfqpoint{2.879879in}{3.327250in}}{\pgfqpoint{2.874705in}{3.314759in}}{\pgfqpoint{2.874705in}{3.301736in}}%
\pgfpathcurveto{\pgfqpoint{2.874705in}{3.288714in}}{\pgfqpoint{2.879879in}{3.276223in}}{\pgfqpoint{2.889088in}{3.267014in}}%
\pgfpathcurveto{\pgfqpoint{2.898296in}{3.257806in}}{\pgfqpoint{2.910787in}{3.252632in}}{\pgfqpoint{2.923810in}{3.252632in}}%
\pgfpathlineto{\pgfqpoint{2.923810in}{3.252632in}}%
\pgfpathclose%
\pgfusepath{stroke,fill}%
\end{pgfscope}%
\begin{pgfscope}%
\pgfpathrectangle{\pgfqpoint{0.786164in}{0.768110in}}{\pgfqpoint{8.851069in}{7.081890in}}%
\pgfusepath{clip}%
\pgfsetbuttcap%
\pgfsetroundjoin%
\definecolor{currentfill}{rgb}{0.126326,0.644107,0.525311}%
\pgfsetfillcolor{currentfill}%
\pgfsetfillopacity{0.700000}%
\pgfsetlinewidth{0.501875pt}%
\definecolor{currentstroke}{rgb}{1.000000,1.000000,1.000000}%
\pgfsetstrokecolor{currentstroke}%
\pgfsetstrokeopacity{0.700000}%
\pgfsetdash{}{0pt}%
\pgfpathmoveto{\pgfqpoint{3.069942in}{3.362123in}}%
\pgfpathcurveto{\pgfqpoint{3.082965in}{3.362123in}}{\pgfqpoint{3.095456in}{3.367297in}}{\pgfqpoint{3.104665in}{3.376505in}}%
\pgfpathcurveto{\pgfqpoint{3.113873in}{3.385714in}}{\pgfqpoint{3.119047in}{3.398205in}}{\pgfqpoint{3.119047in}{3.411228in}}%
\pgfpathcurveto{\pgfqpoint{3.119047in}{3.424250in}}{\pgfqpoint{3.113873in}{3.436741in}}{\pgfqpoint{3.104665in}{3.445950in}}%
\pgfpathcurveto{\pgfqpoint{3.095456in}{3.455158in}}{\pgfqpoint{3.082965in}{3.460332in}}{\pgfqpoint{3.069942in}{3.460332in}}%
\pgfpathcurveto{\pgfqpoint{3.056920in}{3.460332in}}{\pgfqpoint{3.044429in}{3.455158in}}{\pgfqpoint{3.035220in}{3.445950in}}%
\pgfpathcurveto{\pgfqpoint{3.026012in}{3.436741in}}{\pgfqpoint{3.020838in}{3.424250in}}{\pgfqpoint{3.020838in}{3.411228in}}%
\pgfpathcurveto{\pgfqpoint{3.020838in}{3.398205in}}{\pgfqpoint{3.026012in}{3.385714in}}{\pgfqpoint{3.035220in}{3.376505in}}%
\pgfpathcurveto{\pgfqpoint{3.044429in}{3.367297in}}{\pgfqpoint{3.056920in}{3.362123in}}{\pgfqpoint{3.069942in}{3.362123in}}%
\pgfpathlineto{\pgfqpoint{3.069942in}{3.362123in}}%
\pgfpathclose%
\pgfusepath{stroke,fill}%
\end{pgfscope}%
\begin{pgfscope}%
\pgfpathrectangle{\pgfqpoint{0.786164in}{0.768110in}}{\pgfqpoint{8.851069in}{7.081890in}}%
\pgfusepath{clip}%
\pgfsetbuttcap%
\pgfsetroundjoin%
\definecolor{currentfill}{rgb}{0.281887,0.150881,0.465405}%
\pgfsetfillcolor{currentfill}%
\pgfsetfillopacity{0.700000}%
\pgfsetlinewidth{0.501875pt}%
\definecolor{currentstroke}{rgb}{1.000000,1.000000,1.000000}%
\pgfsetstrokecolor{currentstroke}%
\pgfsetstrokeopacity{0.700000}%
\pgfsetdash{}{0pt}%
\pgfpathmoveto{\pgfqpoint{2.348413in}{3.668698in}}%
\pgfpathcurveto{\pgfqpoint{2.361435in}{3.668698in}}{\pgfqpoint{2.373926in}{3.673872in}}{\pgfqpoint{2.383135in}{3.683081in}}%
\pgfpathcurveto{\pgfqpoint{2.392343in}{3.692289in}}{\pgfqpoint{2.397517in}{3.704780in}}{\pgfqpoint{2.397517in}{3.717803in}}%
\pgfpathcurveto{\pgfqpoint{2.397517in}{3.730826in}}{\pgfqpoint{2.392343in}{3.743317in}}{\pgfqpoint{2.383135in}{3.752525in}}%
\pgfpathcurveto{\pgfqpoint{2.373926in}{3.761733in}}{\pgfqpoint{2.361435in}{3.766907in}}{\pgfqpoint{2.348413in}{3.766907in}}%
\pgfpathcurveto{\pgfqpoint{2.335390in}{3.766907in}}{\pgfqpoint{2.322899in}{3.761733in}}{\pgfqpoint{2.313690in}{3.752525in}}%
\pgfpathcurveto{\pgfqpoint{2.304482in}{3.743317in}}{\pgfqpoint{2.299308in}{3.730826in}}{\pgfqpoint{2.299308in}{3.717803in}}%
\pgfpathcurveto{\pgfqpoint{2.299308in}{3.704780in}}{\pgfqpoint{2.304482in}{3.692289in}}{\pgfqpoint{2.313690in}{3.683081in}}%
\pgfpathcurveto{\pgfqpoint{2.322899in}{3.673872in}}{\pgfqpoint{2.335390in}{3.668698in}}{\pgfqpoint{2.348413in}{3.668698in}}%
\pgfpathlineto{\pgfqpoint{2.348413in}{3.668698in}}%
\pgfpathclose%
\pgfusepath{stroke,fill}%
\end{pgfscope}%
\begin{pgfscope}%
\pgfpathrectangle{\pgfqpoint{0.786164in}{0.768110in}}{\pgfqpoint{8.851069in}{7.081890in}}%
\pgfusepath{clip}%
\pgfsetbuttcap%
\pgfsetroundjoin%
\definecolor{currentfill}{rgb}{0.281887,0.150881,0.465405}%
\pgfsetfillcolor{currentfill}%
\pgfsetfillopacity{0.700000}%
\pgfsetlinewidth{0.501875pt}%
\definecolor{currentstroke}{rgb}{1.000000,1.000000,1.000000}%
\pgfsetstrokecolor{currentstroke}%
\pgfsetstrokeopacity{0.700000}%
\pgfsetdash{}{0pt}%
\pgfpathmoveto{\pgfqpoint{2.394079in}{3.734393in}}%
\pgfpathcurveto{\pgfqpoint{2.407102in}{3.734393in}}{\pgfqpoint{2.419593in}{3.739567in}}{\pgfqpoint{2.428801in}{3.748775in}}%
\pgfpathcurveto{\pgfqpoint{2.438010in}{3.757984in}}{\pgfqpoint{2.443184in}{3.770475in}}{\pgfqpoint{2.443184in}{3.783498in}}%
\pgfpathcurveto{\pgfqpoint{2.443184in}{3.796520in}}{\pgfqpoint{2.438010in}{3.809011in}}{\pgfqpoint{2.428801in}{3.818220in}}%
\pgfpathcurveto{\pgfqpoint{2.419593in}{3.827428in}}{\pgfqpoint{2.407102in}{3.832602in}}{\pgfqpoint{2.394079in}{3.832602in}}%
\pgfpathcurveto{\pgfqpoint{2.381056in}{3.832602in}}{\pgfqpoint{2.368565in}{3.827428in}}{\pgfqpoint{2.359357in}{3.818220in}}%
\pgfpathcurveto{\pgfqpoint{2.350148in}{3.809011in}}{\pgfqpoint{2.344975in}{3.796520in}}{\pgfqpoint{2.344975in}{3.783498in}}%
\pgfpathcurveto{\pgfqpoint{2.344975in}{3.770475in}}{\pgfqpoint{2.350148in}{3.757984in}}{\pgfqpoint{2.359357in}{3.748775in}}%
\pgfpathcurveto{\pgfqpoint{2.368565in}{3.739567in}}{\pgfqpoint{2.381056in}{3.734393in}}{\pgfqpoint{2.394079in}{3.734393in}}%
\pgfpathlineto{\pgfqpoint{2.394079in}{3.734393in}}%
\pgfpathclose%
\pgfusepath{stroke,fill}%
\end{pgfscope}%
\begin{pgfscope}%
\pgfpathrectangle{\pgfqpoint{0.786164in}{0.768110in}}{\pgfqpoint{8.851069in}{7.081890in}}%
\pgfusepath{clip}%
\pgfsetbuttcap%
\pgfsetroundjoin%
\definecolor{currentfill}{rgb}{0.281412,0.155834,0.469201}%
\pgfsetfillcolor{currentfill}%
\pgfsetfillopacity{0.700000}%
\pgfsetlinewidth{0.501875pt}%
\definecolor{currentstroke}{rgb}{1.000000,1.000000,1.000000}%
\pgfsetstrokecolor{currentstroke}%
\pgfsetstrokeopacity{0.700000}%
\pgfsetdash{}{0pt}%
\pgfpathmoveto{\pgfqpoint{2.384946in}{3.646800in}}%
\pgfpathcurveto{\pgfqpoint{2.397969in}{3.646800in}}{\pgfqpoint{2.410460in}{3.651974in}}{\pgfqpoint{2.419668in}{3.661182in}}%
\pgfpathcurveto{\pgfqpoint{2.428877in}{3.670391in}}{\pgfqpoint{2.434050in}{3.682882in}}{\pgfqpoint{2.434050in}{3.695905in}}%
\pgfpathcurveto{\pgfqpoint{2.434050in}{3.708927in}}{\pgfqpoint{2.428877in}{3.721418in}}{\pgfqpoint{2.419668in}{3.730627in}}%
\pgfpathcurveto{\pgfqpoint{2.410460in}{3.739835in}}{\pgfqpoint{2.397969in}{3.745009in}}{\pgfqpoint{2.384946in}{3.745009in}}%
\pgfpathcurveto{\pgfqpoint{2.371923in}{3.745009in}}{\pgfqpoint{2.359432in}{3.739835in}}{\pgfqpoint{2.350224in}{3.730627in}}%
\pgfpathcurveto{\pgfqpoint{2.341015in}{3.721418in}}{\pgfqpoint{2.335841in}{3.708927in}}{\pgfqpoint{2.335841in}{3.695905in}}%
\pgfpathcurveto{\pgfqpoint{2.335841in}{3.682882in}}{\pgfqpoint{2.341015in}{3.670391in}}{\pgfqpoint{2.350224in}{3.661182in}}%
\pgfpathcurveto{\pgfqpoint{2.359432in}{3.651974in}}{\pgfqpoint{2.371923in}{3.646800in}}{\pgfqpoint{2.384946in}{3.646800in}}%
\pgfpathlineto{\pgfqpoint{2.384946in}{3.646800in}}%
\pgfpathclose%
\pgfusepath{stroke,fill}%
\end{pgfscope}%
\begin{pgfscope}%
\pgfpathrectangle{\pgfqpoint{0.786164in}{0.768110in}}{\pgfqpoint{8.851069in}{7.081890in}}%
\pgfusepath{clip}%
\pgfsetbuttcap%
\pgfsetroundjoin%
\definecolor{currentfill}{rgb}{0.279574,0.170599,0.479997}%
\pgfsetfillcolor{currentfill}%
\pgfsetfillopacity{0.700000}%
\pgfsetlinewidth{0.501875pt}%
\definecolor{currentstroke}{rgb}{1.000000,1.000000,1.000000}%
\pgfsetstrokecolor{currentstroke}%
\pgfsetstrokeopacity{0.700000}%
\pgfsetdash{}{0pt}%
\pgfpathmoveto{\pgfqpoint{2.375813in}{3.734393in}}%
\pgfpathcurveto{\pgfqpoint{2.388835in}{3.734393in}}{\pgfqpoint{2.401326in}{3.739567in}}{\pgfqpoint{2.410535in}{3.748775in}}%
\pgfpathcurveto{\pgfqpoint{2.419743in}{3.757984in}}{\pgfqpoint{2.424917in}{3.770475in}}{\pgfqpoint{2.424917in}{3.783498in}}%
\pgfpathcurveto{\pgfqpoint{2.424917in}{3.796520in}}{\pgfqpoint{2.419743in}{3.809011in}}{\pgfqpoint{2.410535in}{3.818220in}}%
\pgfpathcurveto{\pgfqpoint{2.401326in}{3.827428in}}{\pgfqpoint{2.388835in}{3.832602in}}{\pgfqpoint{2.375813in}{3.832602in}}%
\pgfpathcurveto{\pgfqpoint{2.362790in}{3.832602in}}{\pgfqpoint{2.350299in}{3.827428in}}{\pgfqpoint{2.341090in}{3.818220in}}%
\pgfpathcurveto{\pgfqpoint{2.331882in}{3.809011in}}{\pgfqpoint{2.326708in}{3.796520in}}{\pgfqpoint{2.326708in}{3.783498in}}%
\pgfpathcurveto{\pgfqpoint{2.326708in}{3.770475in}}{\pgfqpoint{2.331882in}{3.757984in}}{\pgfqpoint{2.341090in}{3.748775in}}%
\pgfpathcurveto{\pgfqpoint{2.350299in}{3.739567in}}{\pgfqpoint{2.362790in}{3.734393in}}{\pgfqpoint{2.375813in}{3.734393in}}%
\pgfpathlineto{\pgfqpoint{2.375813in}{3.734393in}}%
\pgfpathclose%
\pgfusepath{stroke,fill}%
\end{pgfscope}%
\begin{pgfscope}%
\pgfpathrectangle{\pgfqpoint{0.786164in}{0.768110in}}{\pgfqpoint{8.851069in}{7.081890in}}%
\pgfusepath{clip}%
\pgfsetbuttcap%
\pgfsetroundjoin%
\definecolor{currentfill}{rgb}{0.279574,0.170599,0.479997}%
\pgfsetfillcolor{currentfill}%
\pgfsetfillopacity{0.700000}%
\pgfsetlinewidth{0.501875pt}%
\definecolor{currentstroke}{rgb}{1.000000,1.000000,1.000000}%
\pgfsetstrokecolor{currentstroke}%
\pgfsetstrokeopacity{0.700000}%
\pgfsetdash{}{0pt}%
\pgfpathmoveto{\pgfqpoint{2.384946in}{3.624902in}}%
\pgfpathcurveto{\pgfqpoint{2.397969in}{3.624902in}}{\pgfqpoint{2.410460in}{3.630076in}}{\pgfqpoint{2.419668in}{3.639284in}}%
\pgfpathcurveto{\pgfqpoint{2.428877in}{3.648493in}}{\pgfqpoint{2.434050in}{3.660984in}}{\pgfqpoint{2.434050in}{3.674006in}}%
\pgfpathcurveto{\pgfqpoint{2.434050in}{3.687029in}}{\pgfqpoint{2.428877in}{3.699520in}}{\pgfqpoint{2.419668in}{3.708729in}}%
\pgfpathcurveto{\pgfqpoint{2.410460in}{3.717937in}}{\pgfqpoint{2.397969in}{3.723111in}}{\pgfqpoint{2.384946in}{3.723111in}}%
\pgfpathcurveto{\pgfqpoint{2.371923in}{3.723111in}}{\pgfqpoint{2.359432in}{3.717937in}}{\pgfqpoint{2.350224in}{3.708729in}}%
\pgfpathcurveto{\pgfqpoint{2.341015in}{3.699520in}}{\pgfqpoint{2.335841in}{3.687029in}}{\pgfqpoint{2.335841in}{3.674006in}}%
\pgfpathcurveto{\pgfqpoint{2.335841in}{3.660984in}}{\pgfqpoint{2.341015in}{3.648493in}}{\pgfqpoint{2.350224in}{3.639284in}}%
\pgfpathcurveto{\pgfqpoint{2.359432in}{3.630076in}}{\pgfqpoint{2.371923in}{3.624902in}}{\pgfqpoint{2.384946in}{3.624902in}}%
\pgfpathlineto{\pgfqpoint{2.384946in}{3.624902in}}%
\pgfpathclose%
\pgfusepath{stroke,fill}%
\end{pgfscope}%
\begin{pgfscope}%
\pgfpathrectangle{\pgfqpoint{0.786164in}{0.768110in}}{\pgfqpoint{8.851069in}{7.081890in}}%
\pgfusepath{clip}%
\pgfsetbuttcap%
\pgfsetroundjoin%
\definecolor{currentfill}{rgb}{0.278012,0.180367,0.486697}%
\pgfsetfillcolor{currentfill}%
\pgfsetfillopacity{0.700000}%
\pgfsetlinewidth{0.501875pt}%
\definecolor{currentstroke}{rgb}{1.000000,1.000000,1.000000}%
\pgfsetstrokecolor{currentstroke}%
\pgfsetstrokeopacity{0.700000}%
\pgfsetdash{}{0pt}%
\pgfpathmoveto{\pgfqpoint{2.293613in}{3.471614in}}%
\pgfpathcurveto{\pgfqpoint{2.306636in}{3.471614in}}{\pgfqpoint{2.319127in}{3.476788in}}{\pgfqpoint{2.328335in}{3.485996in}}%
\pgfpathcurveto{\pgfqpoint{2.337544in}{3.495205in}}{\pgfqpoint{2.342718in}{3.507696in}}{\pgfqpoint{2.342718in}{3.520719in}}%
\pgfpathcurveto{\pgfqpoint{2.342718in}{3.533741in}}{\pgfqpoint{2.337544in}{3.546232in}}{\pgfqpoint{2.328335in}{3.555441in}}%
\pgfpathcurveto{\pgfqpoint{2.319127in}{3.564649in}}{\pgfqpoint{2.306636in}{3.569823in}}{\pgfqpoint{2.293613in}{3.569823in}}%
\pgfpathcurveto{\pgfqpoint{2.280590in}{3.569823in}}{\pgfqpoint{2.268099in}{3.564649in}}{\pgfqpoint{2.258891in}{3.555441in}}%
\pgfpathcurveto{\pgfqpoint{2.249682in}{3.546232in}}{\pgfqpoint{2.244508in}{3.533741in}}{\pgfqpoint{2.244508in}{3.520719in}}%
\pgfpathcurveto{\pgfqpoint{2.244508in}{3.507696in}}{\pgfqpoint{2.249682in}{3.495205in}}{\pgfqpoint{2.258891in}{3.485996in}}%
\pgfpathcurveto{\pgfqpoint{2.268099in}{3.476788in}}{\pgfqpoint{2.280590in}{3.471614in}}{\pgfqpoint{2.293613in}{3.471614in}}%
\pgfpathlineto{\pgfqpoint{2.293613in}{3.471614in}}%
\pgfpathclose%
\pgfusepath{stroke,fill}%
\end{pgfscope}%
\begin{pgfscope}%
\pgfpathrectangle{\pgfqpoint{0.786164in}{0.768110in}}{\pgfqpoint{8.851069in}{7.081890in}}%
\pgfusepath{clip}%
\pgfsetbuttcap%
\pgfsetroundjoin%
\definecolor{currentfill}{rgb}{0.273006,0.204520,0.501721}%
\pgfsetfillcolor{currentfill}%
\pgfsetfillopacity{0.700000}%
\pgfsetlinewidth{0.501875pt}%
\definecolor{currentstroke}{rgb}{1.000000,1.000000,1.000000}%
\pgfsetstrokecolor{currentstroke}%
\pgfsetstrokeopacity{0.700000}%
\pgfsetdash{}{0pt}%
\pgfpathmoveto{\pgfqpoint{2.110947in}{3.340225in}}%
\pgfpathcurveto{\pgfqpoint{2.123970in}{3.340225in}}{\pgfqpoint{2.136461in}{3.345399in}}{\pgfqpoint{2.145669in}{3.354607in}}%
\pgfpathcurveto{\pgfqpoint{2.154878in}{3.363816in}}{\pgfqpoint{2.160052in}{3.376307in}}{\pgfqpoint{2.160052in}{3.389329in}}%
\pgfpathcurveto{\pgfqpoint{2.160052in}{3.402352in}}{\pgfqpoint{2.154878in}{3.414843in}}{\pgfqpoint{2.145669in}{3.424052in}}%
\pgfpathcurveto{\pgfqpoint{2.136461in}{3.433260in}}{\pgfqpoint{2.123970in}{3.438434in}}{\pgfqpoint{2.110947in}{3.438434in}}%
\pgfpathcurveto{\pgfqpoint{2.097925in}{3.438434in}}{\pgfqpoint{2.085433in}{3.433260in}}{\pgfqpoint{2.076225in}{3.424052in}}%
\pgfpathcurveto{\pgfqpoint{2.067017in}{3.414843in}}{\pgfqpoint{2.061843in}{3.402352in}}{\pgfqpoint{2.061843in}{3.389329in}}%
\pgfpathcurveto{\pgfqpoint{2.061843in}{3.376307in}}{\pgfqpoint{2.067017in}{3.363816in}}{\pgfqpoint{2.076225in}{3.354607in}}%
\pgfpathcurveto{\pgfqpoint{2.085433in}{3.345399in}}{\pgfqpoint{2.097925in}{3.340225in}}{\pgfqpoint{2.110947in}{3.340225in}}%
\pgfpathlineto{\pgfqpoint{2.110947in}{3.340225in}}%
\pgfpathclose%
\pgfusepath{stroke,fill}%
\end{pgfscope}%
\begin{pgfscope}%
\pgfpathrectangle{\pgfqpoint{0.786164in}{0.768110in}}{\pgfqpoint{8.851069in}{7.081890in}}%
\pgfusepath{clip}%
\pgfsetbuttcap%
\pgfsetroundjoin%
\definecolor{currentfill}{rgb}{0.267968,0.223549,0.512008}%
\pgfsetfillcolor{currentfill}%
\pgfsetfillopacity{0.700000}%
\pgfsetlinewidth{0.501875pt}%
\definecolor{currentstroke}{rgb}{1.000000,1.000000,1.000000}%
\pgfsetstrokecolor{currentstroke}%
\pgfsetstrokeopacity{0.700000}%
\pgfsetdash{}{0pt}%
\pgfpathmoveto{\pgfqpoint{2.065281in}{3.296428in}}%
\pgfpathcurveto{\pgfqpoint{2.078304in}{3.296428in}}{\pgfqpoint{2.090795in}{3.301602in}}{\pgfqpoint{2.100003in}{3.310811in}}%
\pgfpathcurveto{\pgfqpoint{2.109211in}{3.320019in}}{\pgfqpoint{2.114385in}{3.332510in}}{\pgfqpoint{2.114385in}{3.345533in}}%
\pgfpathcurveto{\pgfqpoint{2.114385in}{3.358556in}}{\pgfqpoint{2.109211in}{3.371047in}}{\pgfqpoint{2.100003in}{3.380255in}}%
\pgfpathcurveto{\pgfqpoint{2.090795in}{3.389463in}}{\pgfqpoint{2.078304in}{3.394637in}}{\pgfqpoint{2.065281in}{3.394637in}}%
\pgfpathcurveto{\pgfqpoint{2.052258in}{3.394637in}}{\pgfqpoint{2.039767in}{3.389463in}}{\pgfqpoint{2.030559in}{3.380255in}}%
\pgfpathcurveto{\pgfqpoint{2.021350in}{3.371047in}}{\pgfqpoint{2.016176in}{3.358556in}}{\pgfqpoint{2.016176in}{3.345533in}}%
\pgfpathcurveto{\pgfqpoint{2.016176in}{3.332510in}}{\pgfqpoint{2.021350in}{3.320019in}}{\pgfqpoint{2.030559in}{3.310811in}}%
\pgfpathcurveto{\pgfqpoint{2.039767in}{3.301602in}}{\pgfqpoint{2.052258in}{3.296428in}}{\pgfqpoint{2.065281in}{3.296428in}}%
\pgfpathlineto{\pgfqpoint{2.065281in}{3.296428in}}%
\pgfpathclose%
\pgfusepath{stroke,fill}%
\end{pgfscope}%
\begin{pgfscope}%
\pgfpathrectangle{\pgfqpoint{0.786164in}{0.768110in}}{\pgfqpoint{8.851069in}{7.081890in}}%
\pgfusepath{clip}%
\pgfsetbuttcap%
\pgfsetroundjoin%
\definecolor{currentfill}{rgb}{0.252194,0.269783,0.531579}%
\pgfsetfillcolor{currentfill}%
\pgfsetfillopacity{0.700000}%
\pgfsetlinewidth{0.501875pt}%
\definecolor{currentstroke}{rgb}{1.000000,1.000000,1.000000}%
\pgfsetstrokecolor{currentstroke}%
\pgfsetstrokeopacity{0.700000}%
\pgfsetdash{}{0pt}%
\pgfpathmoveto{\pgfqpoint{2.065281in}{3.230733in}}%
\pgfpathcurveto{\pgfqpoint{2.078304in}{3.230733in}}{\pgfqpoint{2.090795in}{3.235907in}}{\pgfqpoint{2.100003in}{3.245116in}}%
\pgfpathcurveto{\pgfqpoint{2.109211in}{3.254324in}}{\pgfqpoint{2.114385in}{3.266815in}}{\pgfqpoint{2.114385in}{3.279838in}}%
\pgfpathcurveto{\pgfqpoint{2.114385in}{3.292861in}}{\pgfqpoint{2.109211in}{3.305352in}}{\pgfqpoint{2.100003in}{3.314560in}}%
\pgfpathcurveto{\pgfqpoint{2.090795in}{3.323769in}}{\pgfqpoint{2.078304in}{3.328943in}}{\pgfqpoint{2.065281in}{3.328943in}}%
\pgfpathcurveto{\pgfqpoint{2.052258in}{3.328943in}}{\pgfqpoint{2.039767in}{3.323769in}}{\pgfqpoint{2.030559in}{3.314560in}}%
\pgfpathcurveto{\pgfqpoint{2.021350in}{3.305352in}}{\pgfqpoint{2.016176in}{3.292861in}}{\pgfqpoint{2.016176in}{3.279838in}}%
\pgfpathcurveto{\pgfqpoint{2.016176in}{3.266815in}}{\pgfqpoint{2.021350in}{3.254324in}}{\pgfqpoint{2.030559in}{3.245116in}}%
\pgfpathcurveto{\pgfqpoint{2.039767in}{3.235907in}}{\pgfqpoint{2.052258in}{3.230733in}}{\pgfqpoint{2.065281in}{3.230733in}}%
\pgfpathlineto{\pgfqpoint{2.065281in}{3.230733in}}%
\pgfpathclose%
\pgfusepath{stroke,fill}%
\end{pgfscope}%
\begin{pgfscope}%
\pgfpathrectangle{\pgfqpoint{0.786164in}{0.768110in}}{\pgfqpoint{8.851069in}{7.081890in}}%
\pgfusepath{clip}%
\pgfsetbuttcap%
\pgfsetroundjoin%
\definecolor{currentfill}{rgb}{0.239346,0.300855,0.540844}%
\pgfsetfillcolor{currentfill}%
\pgfsetfillopacity{0.700000}%
\pgfsetlinewidth{0.501875pt}%
\definecolor{currentstroke}{rgb}{1.000000,1.000000,1.000000}%
\pgfsetstrokecolor{currentstroke}%
\pgfsetstrokeopacity{0.700000}%
\pgfsetdash{}{0pt}%
\pgfpathmoveto{\pgfqpoint{1.827815in}{3.099344in}}%
\pgfpathcurveto{\pgfqpoint{1.840838in}{3.099344in}}{\pgfqpoint{1.853329in}{3.104518in}}{\pgfqpoint{1.862538in}{3.113726in}}%
\pgfpathcurveto{\pgfqpoint{1.871746in}{3.122935in}}{\pgfqpoint{1.876920in}{3.135426in}}{\pgfqpoint{1.876920in}{3.148449in}}%
\pgfpathcurveto{\pgfqpoint{1.876920in}{3.161471in}}{\pgfqpoint{1.871746in}{3.173962in}}{\pgfqpoint{1.862538in}{3.183171in}}%
\pgfpathcurveto{\pgfqpoint{1.853329in}{3.192379in}}{\pgfqpoint{1.840838in}{3.197553in}}{\pgfqpoint{1.827815in}{3.197553in}}%
\pgfpathcurveto{\pgfqpoint{1.814793in}{3.197553in}}{\pgfqpoint{1.802302in}{3.192379in}}{\pgfqpoint{1.793093in}{3.183171in}}%
\pgfpathcurveto{\pgfqpoint{1.783885in}{3.173962in}}{\pgfqpoint{1.778711in}{3.161471in}}{\pgfqpoint{1.778711in}{3.148449in}}%
\pgfpathcurveto{\pgfqpoint{1.778711in}{3.135426in}}{\pgfqpoint{1.783885in}{3.122935in}}{\pgfqpoint{1.793093in}{3.113726in}}%
\pgfpathcurveto{\pgfqpoint{1.802302in}{3.104518in}}{\pgfqpoint{1.814793in}{3.099344in}}{\pgfqpoint{1.827815in}{3.099344in}}%
\pgfpathlineto{\pgfqpoint{1.827815in}{3.099344in}}%
\pgfpathclose%
\pgfusepath{stroke,fill}%
\end{pgfscope}%
\begin{pgfscope}%
\pgfpathrectangle{\pgfqpoint{0.786164in}{0.768110in}}{\pgfqpoint{8.851069in}{7.081890in}}%
\pgfusepath{clip}%
\pgfsetbuttcap%
\pgfsetroundjoin%
\definecolor{currentfill}{rgb}{0.237441,0.305202,0.541921}%
\pgfsetfillcolor{currentfill}%
\pgfsetfillopacity{0.700000}%
\pgfsetlinewidth{0.501875pt}%
\definecolor{currentstroke}{rgb}{1.000000,1.000000,1.000000}%
\pgfsetstrokecolor{currentstroke}%
\pgfsetstrokeopacity{0.700000}%
\pgfsetdash{}{0pt}%
\pgfpathmoveto{\pgfqpoint{1.827815in}{3.077446in}}%
\pgfpathcurveto{\pgfqpoint{1.840838in}{3.077446in}}{\pgfqpoint{1.853329in}{3.082620in}}{\pgfqpoint{1.862538in}{3.091828in}}%
\pgfpathcurveto{\pgfqpoint{1.871746in}{3.101037in}}{\pgfqpoint{1.876920in}{3.113528in}}{\pgfqpoint{1.876920in}{3.126550in}}%
\pgfpathcurveto{\pgfqpoint{1.876920in}{3.139573in}}{\pgfqpoint{1.871746in}{3.152064in}}{\pgfqpoint{1.862538in}{3.161273in}}%
\pgfpathcurveto{\pgfqpoint{1.853329in}{3.170481in}}{\pgfqpoint{1.840838in}{3.175655in}}{\pgfqpoint{1.827815in}{3.175655in}}%
\pgfpathcurveto{\pgfqpoint{1.814793in}{3.175655in}}{\pgfqpoint{1.802302in}{3.170481in}}{\pgfqpoint{1.793093in}{3.161273in}}%
\pgfpathcurveto{\pgfqpoint{1.783885in}{3.152064in}}{\pgfqpoint{1.778711in}{3.139573in}}{\pgfqpoint{1.778711in}{3.126550in}}%
\pgfpathcurveto{\pgfqpoint{1.778711in}{3.113528in}}{\pgfqpoint{1.783885in}{3.101037in}}{\pgfqpoint{1.793093in}{3.091828in}}%
\pgfpathcurveto{\pgfqpoint{1.802302in}{3.082620in}}{\pgfqpoint{1.814793in}{3.077446in}}{\pgfqpoint{1.827815in}{3.077446in}}%
\pgfpathlineto{\pgfqpoint{1.827815in}{3.077446in}}%
\pgfpathclose%
\pgfusepath{stroke,fill}%
\end{pgfscope}%
\begin{pgfscope}%
\pgfpathrectangle{\pgfqpoint{0.786164in}{0.768110in}}{\pgfqpoint{8.851069in}{7.081890in}}%
\pgfusepath{clip}%
\pgfsetbuttcap%
\pgfsetroundjoin%
\definecolor{currentfill}{rgb}{0.237441,0.305202,0.541921}%
\pgfsetfillcolor{currentfill}%
\pgfsetfillopacity{0.700000}%
\pgfsetlinewidth{0.501875pt}%
\definecolor{currentstroke}{rgb}{1.000000,1.000000,1.000000}%
\pgfsetstrokecolor{currentstroke}%
\pgfsetstrokeopacity{0.700000}%
\pgfsetdash{}{0pt}%
\pgfpathmoveto{\pgfqpoint{1.846082in}{2.946056in}}%
\pgfpathcurveto{\pgfqpoint{1.859105in}{2.946056in}}{\pgfqpoint{1.871596in}{2.951230in}}{\pgfqpoint{1.880804in}{2.960439in}}%
\pgfpathcurveto{\pgfqpoint{1.890013in}{2.969647in}}{\pgfqpoint{1.895187in}{2.982138in}}{\pgfqpoint{1.895187in}{2.995161in}}%
\pgfpathcurveto{\pgfqpoint{1.895187in}{3.008184in}}{\pgfqpoint{1.890013in}{3.020675in}}{\pgfqpoint{1.880804in}{3.029883in}}%
\pgfpathcurveto{\pgfqpoint{1.871596in}{3.039092in}}{\pgfqpoint{1.859105in}{3.044266in}}{\pgfqpoint{1.846082in}{3.044266in}}%
\pgfpathcurveto{\pgfqpoint{1.833059in}{3.044266in}}{\pgfqpoint{1.820568in}{3.039092in}}{\pgfqpoint{1.811360in}{3.029883in}}%
\pgfpathcurveto{\pgfqpoint{1.802151in}{3.020675in}}{\pgfqpoint{1.796977in}{3.008184in}}{\pgfqpoint{1.796977in}{2.995161in}}%
\pgfpathcurveto{\pgfqpoint{1.796977in}{2.982138in}}{\pgfqpoint{1.802151in}{2.969647in}}{\pgfqpoint{1.811360in}{2.960439in}}%
\pgfpathcurveto{\pgfqpoint{1.820568in}{2.951230in}}{\pgfqpoint{1.833059in}{2.946056in}}{\pgfqpoint{1.846082in}{2.946056in}}%
\pgfpathlineto{\pgfqpoint{1.846082in}{2.946056in}}%
\pgfpathclose%
\pgfusepath{stroke,fill}%
\end{pgfscope}%
\begin{pgfscope}%
\pgfpathrectangle{\pgfqpoint{0.786164in}{0.768110in}}{\pgfqpoint{8.851069in}{7.081890in}}%
\pgfusepath{clip}%
\pgfsetbuttcap%
\pgfsetroundjoin%
\definecolor{currentfill}{rgb}{0.239346,0.300855,0.540844}%
\pgfsetfillcolor{currentfill}%
\pgfsetfillopacity{0.700000}%
\pgfsetlinewidth{0.501875pt}%
\definecolor{currentstroke}{rgb}{1.000000,1.000000,1.000000}%
\pgfsetstrokecolor{currentstroke}%
\pgfsetstrokeopacity{0.700000}%
\pgfsetdash{}{0pt}%
\pgfpathmoveto{\pgfqpoint{1.827815in}{2.880362in}}%
\pgfpathcurveto{\pgfqpoint{1.840838in}{2.880362in}}{\pgfqpoint{1.853329in}{2.885536in}}{\pgfqpoint{1.862538in}{2.894744in}}%
\pgfpathcurveto{\pgfqpoint{1.871746in}{2.903953in}}{\pgfqpoint{1.876920in}{2.916444in}}{\pgfqpoint{1.876920in}{2.929466in}}%
\pgfpathcurveto{\pgfqpoint{1.876920in}{2.942489in}}{\pgfqpoint{1.871746in}{2.954980in}}{\pgfqpoint{1.862538in}{2.964189in}}%
\pgfpathcurveto{\pgfqpoint{1.853329in}{2.973397in}}{\pgfqpoint{1.840838in}{2.978571in}}{\pgfqpoint{1.827815in}{2.978571in}}%
\pgfpathcurveto{\pgfqpoint{1.814793in}{2.978571in}}{\pgfqpoint{1.802302in}{2.973397in}}{\pgfqpoint{1.793093in}{2.964189in}}%
\pgfpathcurveto{\pgfqpoint{1.783885in}{2.954980in}}{\pgfqpoint{1.778711in}{2.942489in}}{\pgfqpoint{1.778711in}{2.929466in}}%
\pgfpathcurveto{\pgfqpoint{1.778711in}{2.916444in}}{\pgfqpoint{1.783885in}{2.903953in}}{\pgfqpoint{1.793093in}{2.894744in}}%
\pgfpathcurveto{\pgfqpoint{1.802302in}{2.885536in}}{\pgfqpoint{1.814793in}{2.880362in}}{\pgfqpoint{1.827815in}{2.880362in}}%
\pgfpathlineto{\pgfqpoint{1.827815in}{2.880362in}}%
\pgfpathclose%
\pgfusepath{stroke,fill}%
\end{pgfscope}%
\begin{pgfscope}%
\pgfpathrectangle{\pgfqpoint{0.786164in}{0.768110in}}{\pgfqpoint{8.851069in}{7.081890in}}%
\pgfusepath{clip}%
\pgfsetbuttcap%
\pgfsetroundjoin%
\definecolor{currentfill}{rgb}{0.225863,0.330805,0.547314}%
\pgfsetfillcolor{currentfill}%
\pgfsetfillopacity{0.700000}%
\pgfsetlinewidth{0.501875pt}%
\definecolor{currentstroke}{rgb}{1.000000,1.000000,1.000000}%
\pgfsetstrokecolor{currentstroke}%
\pgfsetstrokeopacity{0.700000}%
\pgfsetdash{}{0pt}%
\pgfpathmoveto{\pgfqpoint{1.846082in}{2.967955in}}%
\pgfpathcurveto{\pgfqpoint{1.859105in}{2.967955in}}{\pgfqpoint{1.871596in}{2.973129in}}{\pgfqpoint{1.880804in}{2.982337in}}%
\pgfpathcurveto{\pgfqpoint{1.890013in}{2.991545in}}{\pgfqpoint{1.895187in}{3.004037in}}{\pgfqpoint{1.895187in}{3.017059in}}%
\pgfpathcurveto{\pgfqpoint{1.895187in}{3.030082in}}{\pgfqpoint{1.890013in}{3.042573in}}{\pgfqpoint{1.880804in}{3.051781in}}%
\pgfpathcurveto{\pgfqpoint{1.871596in}{3.060990in}}{\pgfqpoint{1.859105in}{3.066164in}}{\pgfqpoint{1.846082in}{3.066164in}}%
\pgfpathcurveto{\pgfqpoint{1.833059in}{3.066164in}}{\pgfqpoint{1.820568in}{3.060990in}}{\pgfqpoint{1.811360in}{3.051781in}}%
\pgfpathcurveto{\pgfqpoint{1.802151in}{3.042573in}}{\pgfqpoint{1.796977in}{3.030082in}}{\pgfqpoint{1.796977in}{3.017059in}}%
\pgfpathcurveto{\pgfqpoint{1.796977in}{3.004037in}}{\pgfqpoint{1.802151in}{2.991545in}}{\pgfqpoint{1.811360in}{2.982337in}}%
\pgfpathcurveto{\pgfqpoint{1.820568in}{2.973129in}}{\pgfqpoint{1.833059in}{2.967955in}}{\pgfqpoint{1.846082in}{2.967955in}}%
\pgfpathlineto{\pgfqpoint{1.846082in}{2.967955in}}%
\pgfpathclose%
\pgfusepath{stroke,fill}%
\end{pgfscope}%
\begin{pgfscope}%
\pgfpathrectangle{\pgfqpoint{0.786164in}{0.768110in}}{\pgfqpoint{8.851069in}{7.081890in}}%
\pgfusepath{clip}%
\pgfsetbuttcap%
\pgfsetroundjoin%
\definecolor{currentfill}{rgb}{0.220057,0.343307,0.549413}%
\pgfsetfillcolor{currentfill}%
\pgfsetfillopacity{0.700000}%
\pgfsetlinewidth{0.501875pt}%
\definecolor{currentstroke}{rgb}{1.000000,1.000000,1.000000}%
\pgfsetstrokecolor{currentstroke}%
\pgfsetstrokeopacity{0.700000}%
\pgfsetdash{}{0pt}%
\pgfpathmoveto{\pgfqpoint{1.800415in}{2.902260in}}%
\pgfpathcurveto{\pgfqpoint{1.813438in}{2.902260in}}{\pgfqpoint{1.825929in}{2.907434in}}{\pgfqpoint{1.835138in}{2.916642in}}%
\pgfpathcurveto{\pgfqpoint{1.844346in}{2.925851in}}{\pgfqpoint{1.849520in}{2.938342in}}{\pgfqpoint{1.849520in}{2.951365in}}%
\pgfpathcurveto{\pgfqpoint{1.849520in}{2.964387in}}{\pgfqpoint{1.844346in}{2.976878in}}{\pgfqpoint{1.835138in}{2.986087in}}%
\pgfpathcurveto{\pgfqpoint{1.825929in}{2.995295in}}{\pgfqpoint{1.813438in}{3.000469in}}{\pgfqpoint{1.800415in}{3.000469in}}%
\pgfpathcurveto{\pgfqpoint{1.787393in}{3.000469in}}{\pgfqpoint{1.774902in}{2.995295in}}{\pgfqpoint{1.765693in}{2.986087in}}%
\pgfpathcurveto{\pgfqpoint{1.756485in}{2.976878in}}{\pgfqpoint{1.751311in}{2.964387in}}{\pgfqpoint{1.751311in}{2.951365in}}%
\pgfpathcurveto{\pgfqpoint{1.751311in}{2.938342in}}{\pgfqpoint{1.756485in}{2.925851in}}{\pgfqpoint{1.765693in}{2.916642in}}%
\pgfpathcurveto{\pgfqpoint{1.774902in}{2.907434in}}{\pgfqpoint{1.787393in}{2.902260in}}{\pgfqpoint{1.800415in}{2.902260in}}%
\pgfpathlineto{\pgfqpoint{1.800415in}{2.902260in}}%
\pgfpathclose%
\pgfusepath{stroke,fill}%
\end{pgfscope}%
\begin{pgfscope}%
\pgfpathrectangle{\pgfqpoint{0.786164in}{0.768110in}}{\pgfqpoint{8.851069in}{7.081890in}}%
\pgfusepath{clip}%
\pgfsetbuttcap%
\pgfsetroundjoin%
\definecolor{currentfill}{rgb}{0.208623,0.367752,0.552675}%
\pgfsetfillcolor{currentfill}%
\pgfsetfillopacity{0.700000}%
\pgfsetlinewidth{0.501875pt}%
\definecolor{currentstroke}{rgb}{1.000000,1.000000,1.000000}%
\pgfsetstrokecolor{currentstroke}%
\pgfsetstrokeopacity{0.700000}%
\pgfsetdash{}{0pt}%
\pgfpathmoveto{\pgfqpoint{1.773016in}{2.748972in}}%
\pgfpathcurveto{\pgfqpoint{1.786038in}{2.748972in}}{\pgfqpoint{1.798529in}{2.754146in}}{\pgfqpoint{1.807738in}{2.763355in}}%
\pgfpathcurveto{\pgfqpoint{1.816946in}{2.772563in}}{\pgfqpoint{1.822120in}{2.785054in}}{\pgfqpoint{1.822120in}{2.798077in}}%
\pgfpathcurveto{\pgfqpoint{1.822120in}{2.811100in}}{\pgfqpoint{1.816946in}{2.823591in}}{\pgfqpoint{1.807738in}{2.832799in}}%
\pgfpathcurveto{\pgfqpoint{1.798529in}{2.842008in}}{\pgfqpoint{1.786038in}{2.847182in}}{\pgfqpoint{1.773016in}{2.847182in}}%
\pgfpathcurveto{\pgfqpoint{1.759993in}{2.847182in}}{\pgfqpoint{1.747502in}{2.842008in}}{\pgfqpoint{1.738293in}{2.832799in}}%
\pgfpathcurveto{\pgfqpoint{1.729085in}{2.823591in}}{\pgfqpoint{1.723911in}{2.811100in}}{\pgfqpoint{1.723911in}{2.798077in}}%
\pgfpathcurveto{\pgfqpoint{1.723911in}{2.785054in}}{\pgfqpoint{1.729085in}{2.772563in}}{\pgfqpoint{1.738293in}{2.763355in}}%
\pgfpathcurveto{\pgfqpoint{1.747502in}{2.754146in}}{\pgfqpoint{1.759993in}{2.748972in}}{\pgfqpoint{1.773016in}{2.748972in}}%
\pgfpathlineto{\pgfqpoint{1.773016in}{2.748972in}}%
\pgfpathclose%
\pgfusepath{stroke,fill}%
\end{pgfscope}%
\begin{pgfscope}%
\pgfpathrectangle{\pgfqpoint{0.786164in}{0.768110in}}{\pgfqpoint{8.851069in}{7.081890in}}%
\pgfusepath{clip}%
\pgfsetbuttcap%
\pgfsetroundjoin%
\definecolor{currentfill}{rgb}{0.192357,0.403199,0.555836}%
\pgfsetfillcolor{currentfill}%
\pgfsetfillopacity{0.700000}%
\pgfsetlinewidth{0.501875pt}%
\definecolor{currentstroke}{rgb}{1.000000,1.000000,1.000000}%
\pgfsetstrokecolor{currentstroke}%
\pgfsetstrokeopacity{0.700000}%
\pgfsetdash{}{0pt}%
\pgfpathmoveto{\pgfqpoint{1.517284in}{2.486193in}}%
\pgfpathcurveto{\pgfqpoint{1.530306in}{2.486193in}}{\pgfqpoint{1.542797in}{2.491367in}}{\pgfqpoint{1.552006in}{2.500576in}}%
\pgfpathcurveto{\pgfqpoint{1.561214in}{2.509784in}}{\pgfqpoint{1.566388in}{2.522275in}}{\pgfqpoint{1.566388in}{2.535298in}}%
\pgfpathcurveto{\pgfqpoint{1.566388in}{2.548321in}}{\pgfqpoint{1.561214in}{2.560812in}}{\pgfqpoint{1.552006in}{2.570020in}}%
\pgfpathcurveto{\pgfqpoint{1.542797in}{2.579229in}}{\pgfqpoint{1.530306in}{2.584403in}}{\pgfqpoint{1.517284in}{2.584403in}}%
\pgfpathcurveto{\pgfqpoint{1.504261in}{2.584403in}}{\pgfqpoint{1.491770in}{2.579229in}}{\pgfqpoint{1.482561in}{2.570020in}}%
\pgfpathcurveto{\pgfqpoint{1.473353in}{2.560812in}}{\pgfqpoint{1.468179in}{2.548321in}}{\pgfqpoint{1.468179in}{2.535298in}}%
\pgfpathcurveto{\pgfqpoint{1.468179in}{2.522275in}}{\pgfqpoint{1.473353in}{2.509784in}}{\pgfqpoint{1.482561in}{2.500576in}}%
\pgfpathcurveto{\pgfqpoint{1.491770in}{2.491367in}}{\pgfqpoint{1.504261in}{2.486193in}}{\pgfqpoint{1.517284in}{2.486193in}}%
\pgfpathlineto{\pgfqpoint{1.517284in}{2.486193in}}%
\pgfpathclose%
\pgfusepath{stroke,fill}%
\end{pgfscope}%
\begin{pgfscope}%
\pgfpathrectangle{\pgfqpoint{0.786164in}{0.768110in}}{\pgfqpoint{8.851069in}{7.081890in}}%
\pgfusepath{clip}%
\pgfsetbuttcap%
\pgfsetroundjoin%
\definecolor{currentfill}{rgb}{0.190631,0.407061,0.556089}%
\pgfsetfillcolor{currentfill}%
\pgfsetfillopacity{0.700000}%
\pgfsetlinewidth{0.501875pt}%
\definecolor{currentstroke}{rgb}{1.000000,1.000000,1.000000}%
\pgfsetstrokecolor{currentstroke}%
\pgfsetstrokeopacity{0.700000}%
\pgfsetdash{}{0pt}%
\pgfpathmoveto{\pgfqpoint{1.626883in}{2.573786in}}%
\pgfpathcurveto{\pgfqpoint{1.639906in}{2.573786in}}{\pgfqpoint{1.652397in}{2.578960in}}{\pgfqpoint{1.661605in}{2.588169in}}%
\pgfpathcurveto{\pgfqpoint{1.670814in}{2.597377in}}{\pgfqpoint{1.675988in}{2.609868in}}{\pgfqpoint{1.675988in}{2.622891in}}%
\pgfpathcurveto{\pgfqpoint{1.675988in}{2.635914in}}{\pgfqpoint{1.670814in}{2.648405in}}{\pgfqpoint{1.661605in}{2.657613in}}%
\pgfpathcurveto{\pgfqpoint{1.652397in}{2.666822in}}{\pgfqpoint{1.639906in}{2.671996in}}{\pgfqpoint{1.626883in}{2.671996in}}%
\pgfpathcurveto{\pgfqpoint{1.613860in}{2.671996in}}{\pgfqpoint{1.601369in}{2.666822in}}{\pgfqpoint{1.592161in}{2.657613in}}%
\pgfpathcurveto{\pgfqpoint{1.582952in}{2.648405in}}{\pgfqpoint{1.577778in}{2.635914in}}{\pgfqpoint{1.577778in}{2.622891in}}%
\pgfpathcurveto{\pgfqpoint{1.577778in}{2.609868in}}{\pgfqpoint{1.582952in}{2.597377in}}{\pgfqpoint{1.592161in}{2.588169in}}%
\pgfpathcurveto{\pgfqpoint{1.601369in}{2.578960in}}{\pgfqpoint{1.613860in}{2.573786in}}{\pgfqpoint{1.626883in}{2.573786in}}%
\pgfpathlineto{\pgfqpoint{1.626883in}{2.573786in}}%
\pgfpathclose%
\pgfusepath{stroke,fill}%
\end{pgfscope}%
\begin{pgfscope}%
\pgfpathrectangle{\pgfqpoint{0.786164in}{0.768110in}}{\pgfqpoint{8.851069in}{7.081890in}}%
\pgfusepath{clip}%
\pgfsetbuttcap%
\pgfsetroundjoin%
\definecolor{currentfill}{rgb}{0.187231,0.414746,0.556547}%
\pgfsetfillcolor{currentfill}%
\pgfsetfillopacity{0.700000}%
\pgfsetlinewidth{0.501875pt}%
\definecolor{currentstroke}{rgb}{1.000000,1.000000,1.000000}%
\pgfsetstrokecolor{currentstroke}%
\pgfsetstrokeopacity{0.700000}%
\pgfsetdash{}{0pt}%
\pgfpathmoveto{\pgfqpoint{1.699949in}{2.639481in}}%
\pgfpathcurveto{\pgfqpoint{1.712972in}{2.639481in}}{\pgfqpoint{1.725463in}{2.644655in}}{\pgfqpoint{1.734672in}{2.653864in}}%
\pgfpathcurveto{\pgfqpoint{1.743880in}{2.663072in}}{\pgfqpoint{1.749054in}{2.675563in}}{\pgfqpoint{1.749054in}{2.688586in}}%
\pgfpathcurveto{\pgfqpoint{1.749054in}{2.701608in}}{\pgfqpoint{1.743880in}{2.714100in}}{\pgfqpoint{1.734672in}{2.723308in}}%
\pgfpathcurveto{\pgfqpoint{1.725463in}{2.732516in}}{\pgfqpoint{1.712972in}{2.737690in}}{\pgfqpoint{1.699949in}{2.737690in}}%
\pgfpathcurveto{\pgfqpoint{1.686927in}{2.737690in}}{\pgfqpoint{1.674436in}{2.732516in}}{\pgfqpoint{1.665227in}{2.723308in}}%
\pgfpathcurveto{\pgfqpoint{1.656019in}{2.714100in}}{\pgfqpoint{1.650845in}{2.701608in}}{\pgfqpoint{1.650845in}{2.688586in}}%
\pgfpathcurveto{\pgfqpoint{1.650845in}{2.675563in}}{\pgfqpoint{1.656019in}{2.663072in}}{\pgfqpoint{1.665227in}{2.653864in}}%
\pgfpathcurveto{\pgfqpoint{1.674436in}{2.644655in}}{\pgfqpoint{1.686927in}{2.639481in}}{\pgfqpoint{1.699949in}{2.639481in}}%
\pgfpathlineto{\pgfqpoint{1.699949in}{2.639481in}}%
\pgfpathclose%
\pgfusepath{stroke,fill}%
\end{pgfscope}%
\begin{pgfscope}%
\pgfpathrectangle{\pgfqpoint{0.786164in}{0.768110in}}{\pgfqpoint{8.851069in}{7.081890in}}%
\pgfusepath{clip}%
\pgfsetbuttcap%
\pgfsetroundjoin%
\definecolor{currentfill}{rgb}{0.278826,0.175490,0.483397}%
\pgfsetfillcolor{currentfill}%
\pgfsetfillopacity{0.700000}%
\pgfsetlinewidth{0.501875pt}%
\definecolor{currentstroke}{rgb}{1.000000,1.000000,1.000000}%
\pgfsetstrokecolor{currentstroke}%
\pgfsetstrokeopacity{0.700000}%
\pgfsetdash{}{0pt}%
\pgfpathmoveto{\pgfqpoint{2.695478in}{3.011751in}}%
\pgfpathcurveto{\pgfqpoint{2.708500in}{3.011751in}}{\pgfqpoint{2.720991in}{3.016925in}}{\pgfqpoint{2.730200in}{3.026134in}}%
\pgfpathcurveto{\pgfqpoint{2.739408in}{3.035342in}}{\pgfqpoint{2.744582in}{3.047833in}}{\pgfqpoint{2.744582in}{3.060856in}}%
\pgfpathcurveto{\pgfqpoint{2.744582in}{3.073878in}}{\pgfqpoint{2.739408in}{3.086370in}}{\pgfqpoint{2.730200in}{3.095578in}}%
\pgfpathcurveto{\pgfqpoint{2.720991in}{3.104786in}}{\pgfqpoint{2.708500in}{3.109960in}}{\pgfqpoint{2.695478in}{3.109960in}}%
\pgfpathcurveto{\pgfqpoint{2.682455in}{3.109960in}}{\pgfqpoint{2.669964in}{3.104786in}}{\pgfqpoint{2.660755in}{3.095578in}}%
\pgfpathcurveto{\pgfqpoint{2.651547in}{3.086370in}}{\pgfqpoint{2.646373in}{3.073878in}}{\pgfqpoint{2.646373in}{3.060856in}}%
\pgfpathcurveto{\pgfqpoint{2.646373in}{3.047833in}}{\pgfqpoint{2.651547in}{3.035342in}}{\pgfqpoint{2.660755in}{3.026134in}}%
\pgfpathcurveto{\pgfqpoint{2.669964in}{3.016925in}}{\pgfqpoint{2.682455in}{3.011751in}}{\pgfqpoint{2.695478in}{3.011751in}}%
\pgfpathlineto{\pgfqpoint{2.695478in}{3.011751in}}%
\pgfpathclose%
\pgfusepath{stroke,fill}%
\end{pgfscope}%
\begin{pgfscope}%
\pgfpathrectangle{\pgfqpoint{0.786164in}{0.768110in}}{\pgfqpoint{8.851069in}{7.081890in}}%
\pgfusepath{clip}%
\pgfsetbuttcap%
\pgfsetroundjoin%
\definecolor{currentfill}{rgb}{0.278826,0.175490,0.483397}%
\pgfsetfillcolor{currentfill}%
\pgfsetfillopacity{0.700000}%
\pgfsetlinewidth{0.501875pt}%
\definecolor{currentstroke}{rgb}{1.000000,1.000000,1.000000}%
\pgfsetstrokecolor{currentstroke}%
\pgfsetstrokeopacity{0.700000}%
\pgfsetdash{}{0pt}%
\pgfpathmoveto{\pgfqpoint{2.722877in}{3.033649in}}%
\pgfpathcurveto{\pgfqpoint{2.735900in}{3.033649in}}{\pgfqpoint{2.748391in}{3.038823in}}{\pgfqpoint{2.757600in}{3.048032in}}%
\pgfpathcurveto{\pgfqpoint{2.766808in}{3.057240in}}{\pgfqpoint{2.771982in}{3.069731in}}{\pgfqpoint{2.771982in}{3.082754in}}%
\pgfpathcurveto{\pgfqpoint{2.771982in}{3.095777in}}{\pgfqpoint{2.766808in}{3.108268in}}{\pgfqpoint{2.757600in}{3.117476in}}%
\pgfpathcurveto{\pgfqpoint{2.748391in}{3.126685in}}{\pgfqpoint{2.735900in}{3.131859in}}{\pgfqpoint{2.722877in}{3.131859in}}%
\pgfpathcurveto{\pgfqpoint{2.709855in}{3.131859in}}{\pgfqpoint{2.697364in}{3.126685in}}{\pgfqpoint{2.688155in}{3.117476in}}%
\pgfpathcurveto{\pgfqpoint{2.678947in}{3.108268in}}{\pgfqpoint{2.673773in}{3.095777in}}{\pgfqpoint{2.673773in}{3.082754in}}%
\pgfpathcurveto{\pgfqpoint{2.673773in}{3.069731in}}{\pgfqpoint{2.678947in}{3.057240in}}{\pgfqpoint{2.688155in}{3.048032in}}%
\pgfpathcurveto{\pgfqpoint{2.697364in}{3.038823in}}{\pgfqpoint{2.709855in}{3.033649in}}{\pgfqpoint{2.722877in}{3.033649in}}%
\pgfpathlineto{\pgfqpoint{2.722877in}{3.033649in}}%
\pgfpathclose%
\pgfusepath{stroke,fill}%
\end{pgfscope}%
\begin{pgfscope}%
\pgfpathrectangle{\pgfqpoint{0.786164in}{0.768110in}}{\pgfqpoint{8.851069in}{7.081890in}}%
\pgfusepath{clip}%
\pgfsetbuttcap%
\pgfsetroundjoin%
\definecolor{currentfill}{rgb}{0.276194,0.190074,0.493001}%
\pgfsetfillcolor{currentfill}%
\pgfsetfillopacity{0.700000}%
\pgfsetlinewidth{0.501875pt}%
\definecolor{currentstroke}{rgb}{1.000000,1.000000,1.000000}%
\pgfsetstrokecolor{currentstroke}%
\pgfsetstrokeopacity{0.700000}%
\pgfsetdash{}{0pt}%
\pgfpathmoveto{\pgfqpoint{2.686344in}{2.967955in}}%
\pgfpathcurveto{\pgfqpoint{2.699367in}{2.967955in}}{\pgfqpoint{2.711858in}{2.973129in}}{\pgfqpoint{2.721067in}{2.982337in}}%
\pgfpathcurveto{\pgfqpoint{2.730275in}{2.991545in}}{\pgfqpoint{2.735449in}{3.004037in}}{\pgfqpoint{2.735449in}{3.017059in}}%
\pgfpathcurveto{\pgfqpoint{2.735449in}{3.030082in}}{\pgfqpoint{2.730275in}{3.042573in}}{\pgfqpoint{2.721067in}{3.051781in}}%
\pgfpathcurveto{\pgfqpoint{2.711858in}{3.060990in}}{\pgfqpoint{2.699367in}{3.066164in}}{\pgfqpoint{2.686344in}{3.066164in}}%
\pgfpathcurveto{\pgfqpoint{2.673322in}{3.066164in}}{\pgfqpoint{2.660831in}{3.060990in}}{\pgfqpoint{2.651622in}{3.051781in}}%
\pgfpathcurveto{\pgfqpoint{2.642414in}{3.042573in}}{\pgfqpoint{2.637240in}{3.030082in}}{\pgfqpoint{2.637240in}{3.017059in}}%
\pgfpathcurveto{\pgfqpoint{2.637240in}{3.004037in}}{\pgfqpoint{2.642414in}{2.991545in}}{\pgfqpoint{2.651622in}{2.982337in}}%
\pgfpathcurveto{\pgfqpoint{2.660831in}{2.973129in}}{\pgfqpoint{2.673322in}{2.967955in}}{\pgfqpoint{2.686344in}{2.967955in}}%
\pgfpathlineto{\pgfqpoint{2.686344in}{2.967955in}}%
\pgfpathclose%
\pgfusepath{stroke,fill}%
\end{pgfscope}%
\begin{pgfscope}%
\pgfpathrectangle{\pgfqpoint{0.786164in}{0.768110in}}{\pgfqpoint{8.851069in}{7.081890in}}%
\pgfusepath{clip}%
\pgfsetbuttcap%
\pgfsetroundjoin%
\definecolor{currentfill}{rgb}{0.274128,0.199721,0.498911}%
\pgfsetfillcolor{currentfill}%
\pgfsetfillopacity{0.700000}%
\pgfsetlinewidth{0.501875pt}%
\definecolor{currentstroke}{rgb}{1.000000,1.000000,1.000000}%
\pgfsetstrokecolor{currentstroke}%
\pgfsetstrokeopacity{0.700000}%
\pgfsetdash{}{0pt}%
\pgfpathmoveto{\pgfqpoint{2.677211in}{3.011751in}}%
\pgfpathcurveto{\pgfqpoint{2.690234in}{3.011751in}}{\pgfqpoint{2.702725in}{3.016925in}}{\pgfqpoint{2.711933in}{3.026134in}}%
\pgfpathcurveto{\pgfqpoint{2.721142in}{3.035342in}}{\pgfqpoint{2.726316in}{3.047833in}}{\pgfqpoint{2.726316in}{3.060856in}}%
\pgfpathcurveto{\pgfqpoint{2.726316in}{3.073878in}}{\pgfqpoint{2.721142in}{3.086370in}}{\pgfqpoint{2.711933in}{3.095578in}}%
\pgfpathcurveto{\pgfqpoint{2.702725in}{3.104786in}}{\pgfqpoint{2.690234in}{3.109960in}}{\pgfqpoint{2.677211in}{3.109960in}}%
\pgfpathcurveto{\pgfqpoint{2.664188in}{3.109960in}}{\pgfqpoint{2.651697in}{3.104786in}}{\pgfqpoint{2.642489in}{3.095578in}}%
\pgfpathcurveto{\pgfqpoint{2.633280in}{3.086370in}}{\pgfqpoint{2.628106in}{3.073878in}}{\pgfqpoint{2.628106in}{3.060856in}}%
\pgfpathcurveto{\pgfqpoint{2.628106in}{3.047833in}}{\pgfqpoint{2.633280in}{3.035342in}}{\pgfqpoint{2.642489in}{3.026134in}}%
\pgfpathcurveto{\pgfqpoint{2.651697in}{3.016925in}}{\pgfqpoint{2.664188in}{3.011751in}}{\pgfqpoint{2.677211in}{3.011751in}}%
\pgfpathlineto{\pgfqpoint{2.677211in}{3.011751in}}%
\pgfpathclose%
\pgfusepath{stroke,fill}%
\end{pgfscope}%
\begin{pgfscope}%
\pgfpathrectangle{\pgfqpoint{0.786164in}{0.768110in}}{\pgfqpoint{8.851069in}{7.081890in}}%
\pgfusepath{clip}%
\pgfsetbuttcap%
\pgfsetroundjoin%
\definecolor{currentfill}{rgb}{0.269308,0.218818,0.509577}%
\pgfsetfillcolor{currentfill}%
\pgfsetfillopacity{0.700000}%
\pgfsetlinewidth{0.501875pt}%
\definecolor{currentstroke}{rgb}{1.000000,1.000000,1.000000}%
\pgfsetstrokecolor{currentstroke}%
\pgfsetstrokeopacity{0.700000}%
\pgfsetdash{}{0pt}%
\pgfpathmoveto{\pgfqpoint{2.531078in}{2.814667in}}%
\pgfpathcurveto{\pgfqpoint{2.544101in}{2.814667in}}{\pgfqpoint{2.556592in}{2.819841in}}{\pgfqpoint{2.565801in}{2.829049in}}%
\pgfpathcurveto{\pgfqpoint{2.575009in}{2.838258in}}{\pgfqpoint{2.580183in}{2.850749in}}{\pgfqpoint{2.580183in}{2.863772in}}%
\pgfpathcurveto{\pgfqpoint{2.580183in}{2.876794in}}{\pgfqpoint{2.575009in}{2.889285in}}{\pgfqpoint{2.565801in}{2.898494in}}%
\pgfpathcurveto{\pgfqpoint{2.556592in}{2.907702in}}{\pgfqpoint{2.544101in}{2.912876in}}{\pgfqpoint{2.531078in}{2.912876in}}%
\pgfpathcurveto{\pgfqpoint{2.518056in}{2.912876in}}{\pgfqpoint{2.505565in}{2.907702in}}{\pgfqpoint{2.496356in}{2.898494in}}%
\pgfpathcurveto{\pgfqpoint{2.487148in}{2.889285in}}{\pgfqpoint{2.481974in}{2.876794in}}{\pgfqpoint{2.481974in}{2.863772in}}%
\pgfpathcurveto{\pgfqpoint{2.481974in}{2.850749in}}{\pgfqpoint{2.487148in}{2.838258in}}{\pgfqpoint{2.496356in}{2.829049in}}%
\pgfpathcurveto{\pgfqpoint{2.505565in}{2.819841in}}{\pgfqpoint{2.518056in}{2.814667in}}{\pgfqpoint{2.531078in}{2.814667in}}%
\pgfpathlineto{\pgfqpoint{2.531078in}{2.814667in}}%
\pgfpathclose%
\pgfusepath{stroke,fill}%
\end{pgfscope}%
\begin{pgfscope}%
\pgfpathrectangle{\pgfqpoint{0.786164in}{0.768110in}}{\pgfqpoint{8.851069in}{7.081890in}}%
\pgfusepath{clip}%
\pgfsetbuttcap%
\pgfsetroundjoin%
\definecolor{currentfill}{rgb}{0.257322,0.256130,0.526563}%
\pgfsetfillcolor{currentfill}%
\pgfsetfillopacity{0.700000}%
\pgfsetlinewidth{0.501875pt}%
\definecolor{currentstroke}{rgb}{1.000000,1.000000,1.000000}%
\pgfsetstrokecolor{currentstroke}%
\pgfsetstrokeopacity{0.700000}%
\pgfsetdash{}{0pt}%
\pgfpathmoveto{\pgfqpoint{2.293613in}{2.617583in}}%
\pgfpathcurveto{\pgfqpoint{2.306636in}{2.617583in}}{\pgfqpoint{2.319127in}{2.622757in}}{\pgfqpoint{2.328335in}{2.631965in}}%
\pgfpathcurveto{\pgfqpoint{2.337544in}{2.641174in}}{\pgfqpoint{2.342718in}{2.653665in}}{\pgfqpoint{2.342718in}{2.666687in}}%
\pgfpathcurveto{\pgfqpoint{2.342718in}{2.679710in}}{\pgfqpoint{2.337544in}{2.692201in}}{\pgfqpoint{2.328335in}{2.701410in}}%
\pgfpathcurveto{\pgfqpoint{2.319127in}{2.710618in}}{\pgfqpoint{2.306636in}{2.715792in}}{\pgfqpoint{2.293613in}{2.715792in}}%
\pgfpathcurveto{\pgfqpoint{2.280590in}{2.715792in}}{\pgfqpoint{2.268099in}{2.710618in}}{\pgfqpoint{2.258891in}{2.701410in}}%
\pgfpathcurveto{\pgfqpoint{2.249682in}{2.692201in}}{\pgfqpoint{2.244508in}{2.679710in}}{\pgfqpoint{2.244508in}{2.666687in}}%
\pgfpathcurveto{\pgfqpoint{2.244508in}{2.653665in}}{\pgfqpoint{2.249682in}{2.641174in}}{\pgfqpoint{2.258891in}{2.631965in}}%
\pgfpathcurveto{\pgfqpoint{2.268099in}{2.622757in}}{\pgfqpoint{2.280590in}{2.617583in}}{\pgfqpoint{2.293613in}{2.617583in}}%
\pgfpathlineto{\pgfqpoint{2.293613in}{2.617583in}}%
\pgfpathclose%
\pgfusepath{stroke,fill}%
\end{pgfscope}%
\begin{pgfscope}%
\pgfpathrectangle{\pgfqpoint{0.786164in}{0.768110in}}{\pgfqpoint{8.851069in}{7.081890in}}%
\pgfusepath{clip}%
\pgfsetbuttcap%
\pgfsetroundjoin%
\definecolor{currentfill}{rgb}{0.250425,0.274290,0.533103}%
\pgfsetfillcolor{currentfill}%
\pgfsetfillopacity{0.700000}%
\pgfsetlinewidth{0.501875pt}%
\definecolor{currentstroke}{rgb}{1.000000,1.000000,1.000000}%
\pgfsetstrokecolor{currentstroke}%
\pgfsetstrokeopacity{0.700000}%
\pgfsetdash{}{0pt}%
\pgfpathmoveto{\pgfqpoint{2.284480in}{2.529990in}}%
\pgfpathcurveto{\pgfqpoint{2.297502in}{2.529990in}}{\pgfqpoint{2.309993in}{2.535164in}}{\pgfqpoint{2.319202in}{2.544372in}}%
\pgfpathcurveto{\pgfqpoint{2.328410in}{2.553581in}}{\pgfqpoint{2.333584in}{2.566072in}}{\pgfqpoint{2.333584in}{2.579095in}}%
\pgfpathcurveto{\pgfqpoint{2.333584in}{2.592117in}}{\pgfqpoint{2.328410in}{2.604608in}}{\pgfqpoint{2.319202in}{2.613817in}}%
\pgfpathcurveto{\pgfqpoint{2.309993in}{2.623025in}}{\pgfqpoint{2.297502in}{2.628199in}}{\pgfqpoint{2.284480in}{2.628199in}}%
\pgfpathcurveto{\pgfqpoint{2.271457in}{2.628199in}}{\pgfqpoint{2.258966in}{2.623025in}}{\pgfqpoint{2.249757in}{2.613817in}}%
\pgfpathcurveto{\pgfqpoint{2.240549in}{2.604608in}}{\pgfqpoint{2.235375in}{2.592117in}}{\pgfqpoint{2.235375in}{2.579095in}}%
\pgfpathcurveto{\pgfqpoint{2.235375in}{2.566072in}}{\pgfqpoint{2.240549in}{2.553581in}}{\pgfqpoint{2.249757in}{2.544372in}}%
\pgfpathcurveto{\pgfqpoint{2.258966in}{2.535164in}}{\pgfqpoint{2.271457in}{2.529990in}}{\pgfqpoint{2.284480in}{2.529990in}}%
\pgfpathlineto{\pgfqpoint{2.284480in}{2.529990in}}%
\pgfpathclose%
\pgfusepath{stroke,fill}%
\end{pgfscope}%
\begin{pgfscope}%
\pgfpathrectangle{\pgfqpoint{0.786164in}{0.768110in}}{\pgfqpoint{8.851069in}{7.081890in}}%
\pgfusepath{clip}%
\pgfsetbuttcap%
\pgfsetroundjoin%
\definecolor{currentfill}{rgb}{0.255645,0.260703,0.528312}%
\pgfsetfillcolor{currentfill}%
\pgfsetfillopacity{0.700000}%
\pgfsetlinewidth{0.501875pt}%
\definecolor{currentstroke}{rgb}{1.000000,1.000000,1.000000}%
\pgfsetstrokecolor{currentstroke}%
\pgfsetstrokeopacity{0.700000}%
\pgfsetdash{}{0pt}%
\pgfpathmoveto{\pgfqpoint{2.266213in}{2.529990in}}%
\pgfpathcurveto{\pgfqpoint{2.279236in}{2.529990in}}{\pgfqpoint{2.291727in}{2.535164in}}{\pgfqpoint{2.300935in}{2.544372in}}%
\pgfpathcurveto{\pgfqpoint{2.310144in}{2.553581in}}{\pgfqpoint{2.315318in}{2.566072in}}{\pgfqpoint{2.315318in}{2.579095in}}%
\pgfpathcurveto{\pgfqpoint{2.315318in}{2.592117in}}{\pgfqpoint{2.310144in}{2.604608in}}{\pgfqpoint{2.300935in}{2.613817in}}%
\pgfpathcurveto{\pgfqpoint{2.291727in}{2.623025in}}{\pgfqpoint{2.279236in}{2.628199in}}{\pgfqpoint{2.266213in}{2.628199in}}%
\pgfpathcurveto{\pgfqpoint{2.253190in}{2.628199in}}{\pgfqpoint{2.240699in}{2.623025in}}{\pgfqpoint{2.231491in}{2.613817in}}%
\pgfpathcurveto{\pgfqpoint{2.222282in}{2.604608in}}{\pgfqpoint{2.217108in}{2.592117in}}{\pgfqpoint{2.217108in}{2.579095in}}%
\pgfpathcurveto{\pgfqpoint{2.217108in}{2.566072in}}{\pgfqpoint{2.222282in}{2.553581in}}{\pgfqpoint{2.231491in}{2.544372in}}%
\pgfpathcurveto{\pgfqpoint{2.240699in}{2.535164in}}{\pgfqpoint{2.253190in}{2.529990in}}{\pgfqpoint{2.266213in}{2.529990in}}%
\pgfpathlineto{\pgfqpoint{2.266213in}{2.529990in}}%
\pgfpathclose%
\pgfusepath{stroke,fill}%
\end{pgfscope}%
\begin{pgfscope}%
\pgfpathrectangle{\pgfqpoint{0.786164in}{0.768110in}}{\pgfqpoint{8.851069in}{7.081890in}}%
\pgfusepath{clip}%
\pgfsetbuttcap%
\pgfsetroundjoin%
\definecolor{currentfill}{rgb}{0.248629,0.278775,0.534556}%
\pgfsetfillcolor{currentfill}%
\pgfsetfillopacity{0.700000}%
\pgfsetlinewidth{0.501875pt}%
\definecolor{currentstroke}{rgb}{1.000000,1.000000,1.000000}%
\pgfsetstrokecolor{currentstroke}%
\pgfsetstrokeopacity{0.700000}%
\pgfsetdash{}{0pt}%
\pgfpathmoveto{\pgfqpoint{2.275346in}{2.508092in}}%
\pgfpathcurveto{\pgfqpoint{2.288369in}{2.508092in}}{\pgfqpoint{2.300860in}{2.513266in}}{\pgfqpoint{2.310069in}{2.522474in}}%
\pgfpathcurveto{\pgfqpoint{2.319277in}{2.531683in}}{\pgfqpoint{2.324451in}{2.544174in}}{\pgfqpoint{2.324451in}{2.557196in}}%
\pgfpathcurveto{\pgfqpoint{2.324451in}{2.570219in}}{\pgfqpoint{2.319277in}{2.582710in}}{\pgfqpoint{2.310069in}{2.591919in}}%
\pgfpathcurveto{\pgfqpoint{2.300860in}{2.601127in}}{\pgfqpoint{2.288369in}{2.606301in}}{\pgfqpoint{2.275346in}{2.606301in}}%
\pgfpathcurveto{\pgfqpoint{2.262324in}{2.606301in}}{\pgfqpoint{2.249833in}{2.601127in}}{\pgfqpoint{2.240624in}{2.591919in}}%
\pgfpathcurveto{\pgfqpoint{2.231416in}{2.582710in}}{\pgfqpoint{2.226242in}{2.570219in}}{\pgfqpoint{2.226242in}{2.557196in}}%
\pgfpathcurveto{\pgfqpoint{2.226242in}{2.544174in}}{\pgfqpoint{2.231416in}{2.531683in}}{\pgfqpoint{2.240624in}{2.522474in}}%
\pgfpathcurveto{\pgfqpoint{2.249833in}{2.513266in}}{\pgfqpoint{2.262324in}{2.508092in}}{\pgfqpoint{2.275346in}{2.508092in}}%
\pgfpathlineto{\pgfqpoint{2.275346in}{2.508092in}}%
\pgfpathclose%
\pgfusepath{stroke,fill}%
\end{pgfscope}%
\begin{pgfscope}%
\pgfpathrectangle{\pgfqpoint{0.786164in}{0.768110in}}{\pgfqpoint{8.851069in}{7.081890in}}%
\pgfusepath{clip}%
\pgfsetbuttcap%
\pgfsetroundjoin%
\definecolor{currentfill}{rgb}{0.241237,0.296485,0.539709}%
\pgfsetfillcolor{currentfill}%
\pgfsetfillopacity{0.700000}%
\pgfsetlinewidth{0.501875pt}%
\definecolor{currentstroke}{rgb}{1.000000,1.000000,1.000000}%
\pgfsetstrokecolor{currentstroke}%
\pgfsetstrokeopacity{0.700000}%
\pgfsetdash{}{0pt}%
\pgfpathmoveto{\pgfqpoint{2.138347in}{2.398600in}}%
\pgfpathcurveto{\pgfqpoint{2.151370in}{2.398600in}}{\pgfqpoint{2.163861in}{2.403774in}}{\pgfqpoint{2.173069in}{2.412983in}}%
\pgfpathcurveto{\pgfqpoint{2.182278in}{2.422191in}}{\pgfqpoint{2.187452in}{2.434682in}}{\pgfqpoint{2.187452in}{2.447705in}}%
\pgfpathcurveto{\pgfqpoint{2.187452in}{2.460728in}}{\pgfqpoint{2.182278in}{2.473219in}}{\pgfqpoint{2.173069in}{2.482427in}}%
\pgfpathcurveto{\pgfqpoint{2.163861in}{2.491636in}}{\pgfqpoint{2.151370in}{2.496810in}}{\pgfqpoint{2.138347in}{2.496810in}}%
\pgfpathcurveto{\pgfqpoint{2.125324in}{2.496810in}}{\pgfqpoint{2.112833in}{2.491636in}}{\pgfqpoint{2.103625in}{2.482427in}}%
\pgfpathcurveto{\pgfqpoint{2.094416in}{2.473219in}}{\pgfqpoint{2.089242in}{2.460728in}}{\pgfqpoint{2.089242in}{2.447705in}}%
\pgfpathcurveto{\pgfqpoint{2.089242in}{2.434682in}}{\pgfqpoint{2.094416in}{2.422191in}}{\pgfqpoint{2.103625in}{2.412983in}}%
\pgfpathcurveto{\pgfqpoint{2.112833in}{2.403774in}}{\pgfqpoint{2.125324in}{2.398600in}}{\pgfqpoint{2.138347in}{2.398600in}}%
\pgfpathlineto{\pgfqpoint{2.138347in}{2.398600in}}%
\pgfpathclose%
\pgfusepath{stroke,fill}%
\end{pgfscope}%
\begin{pgfscope}%
\pgfpathrectangle{\pgfqpoint{0.786164in}{0.768110in}}{\pgfqpoint{8.851069in}{7.081890in}}%
\pgfusepath{clip}%
\pgfsetbuttcap%
\pgfsetroundjoin%
\definecolor{currentfill}{rgb}{0.235526,0.309527,0.542944}%
\pgfsetfillcolor{currentfill}%
\pgfsetfillopacity{0.700000}%
\pgfsetlinewidth{0.501875pt}%
\definecolor{currentstroke}{rgb}{1.000000,1.000000,1.000000}%
\pgfsetstrokecolor{currentstroke}%
\pgfsetstrokeopacity{0.700000}%
\pgfsetdash{}{0pt}%
\pgfpathmoveto{\pgfqpoint{2.110947in}{2.398600in}}%
\pgfpathcurveto{\pgfqpoint{2.123970in}{2.398600in}}{\pgfqpoint{2.136461in}{2.403774in}}{\pgfqpoint{2.145669in}{2.412983in}}%
\pgfpathcurveto{\pgfqpoint{2.154878in}{2.422191in}}{\pgfqpoint{2.160052in}{2.434682in}}{\pgfqpoint{2.160052in}{2.447705in}}%
\pgfpathcurveto{\pgfqpoint{2.160052in}{2.460728in}}{\pgfqpoint{2.154878in}{2.473219in}}{\pgfqpoint{2.145669in}{2.482427in}}%
\pgfpathcurveto{\pgfqpoint{2.136461in}{2.491636in}}{\pgfqpoint{2.123970in}{2.496810in}}{\pgfqpoint{2.110947in}{2.496810in}}%
\pgfpathcurveto{\pgfqpoint{2.097925in}{2.496810in}}{\pgfqpoint{2.085433in}{2.491636in}}{\pgfqpoint{2.076225in}{2.482427in}}%
\pgfpathcurveto{\pgfqpoint{2.067017in}{2.473219in}}{\pgfqpoint{2.061843in}{2.460728in}}{\pgfqpoint{2.061843in}{2.447705in}}%
\pgfpathcurveto{\pgfqpoint{2.061843in}{2.434682in}}{\pgfqpoint{2.067017in}{2.422191in}}{\pgfqpoint{2.076225in}{2.412983in}}%
\pgfpathcurveto{\pgfqpoint{2.085433in}{2.403774in}}{\pgfqpoint{2.097925in}{2.398600in}}{\pgfqpoint{2.110947in}{2.398600in}}%
\pgfpathlineto{\pgfqpoint{2.110947in}{2.398600in}}%
\pgfpathclose%
\pgfusepath{stroke,fill}%
\end{pgfscope}%
\begin{pgfscope}%
\pgfpathrectangle{\pgfqpoint{0.786164in}{0.768110in}}{\pgfqpoint{8.851069in}{7.081890in}}%
\pgfusepath{clip}%
\pgfsetbuttcap%
\pgfsetroundjoin%
\definecolor{currentfill}{rgb}{0.233603,0.313828,0.543914}%
\pgfsetfillcolor{currentfill}%
\pgfsetfillopacity{0.700000}%
\pgfsetlinewidth{0.501875pt}%
\definecolor{currentstroke}{rgb}{1.000000,1.000000,1.000000}%
\pgfsetstrokecolor{currentstroke}%
\pgfsetstrokeopacity{0.700000}%
\pgfsetdash{}{0pt}%
\pgfpathmoveto{\pgfqpoint{2.193147in}{2.442397in}}%
\pgfpathcurveto{\pgfqpoint{2.206170in}{2.442397in}}{\pgfqpoint{2.218661in}{2.447571in}}{\pgfqpoint{2.227869in}{2.456779in}}%
\pgfpathcurveto{\pgfqpoint{2.237077in}{2.465988in}}{\pgfqpoint{2.242251in}{2.478479in}}{\pgfqpoint{2.242251in}{2.491502in}}%
\pgfpathcurveto{\pgfqpoint{2.242251in}{2.504524in}}{\pgfqpoint{2.237077in}{2.517015in}}{\pgfqpoint{2.227869in}{2.526224in}}%
\pgfpathcurveto{\pgfqpoint{2.218661in}{2.535432in}}{\pgfqpoint{2.206170in}{2.540606in}}{\pgfqpoint{2.193147in}{2.540606in}}%
\pgfpathcurveto{\pgfqpoint{2.180124in}{2.540606in}}{\pgfqpoint{2.167633in}{2.535432in}}{\pgfqpoint{2.158425in}{2.526224in}}%
\pgfpathcurveto{\pgfqpoint{2.149216in}{2.517015in}}{\pgfqpoint{2.144042in}{2.504524in}}{\pgfqpoint{2.144042in}{2.491502in}}%
\pgfpathcurveto{\pgfqpoint{2.144042in}{2.478479in}}{\pgfqpoint{2.149216in}{2.465988in}}{\pgfqpoint{2.158425in}{2.456779in}}%
\pgfpathcurveto{\pgfqpoint{2.167633in}{2.447571in}}{\pgfqpoint{2.180124in}{2.442397in}}{\pgfqpoint{2.193147in}{2.442397in}}%
\pgfpathlineto{\pgfqpoint{2.193147in}{2.442397in}}%
\pgfpathclose%
\pgfusepath{stroke,fill}%
\end{pgfscope}%
\begin{pgfscope}%
\pgfpathrectangle{\pgfqpoint{0.786164in}{0.768110in}}{\pgfqpoint{8.851069in}{7.081890in}}%
\pgfusepath{clip}%
\pgfsetbuttcap%
\pgfsetroundjoin%
\definecolor{currentfill}{rgb}{0.227802,0.326594,0.546532}%
\pgfsetfillcolor{currentfill}%
\pgfsetfillopacity{0.700000}%
\pgfsetlinewidth{0.501875pt}%
\definecolor{currentstroke}{rgb}{1.000000,1.000000,1.000000}%
\pgfsetstrokecolor{currentstroke}%
\pgfsetstrokeopacity{0.700000}%
\pgfsetdash{}{0pt}%
\pgfpathmoveto{\pgfqpoint{2.202280in}{2.376702in}}%
\pgfpathcurveto{\pgfqpoint{2.215303in}{2.376702in}}{\pgfqpoint{2.227794in}{2.381876in}}{\pgfqpoint{2.237002in}{2.391085in}}%
\pgfpathcurveto{\pgfqpoint{2.246211in}{2.400293in}}{\pgfqpoint{2.251385in}{2.412784in}}{\pgfqpoint{2.251385in}{2.425807in}}%
\pgfpathcurveto{\pgfqpoint{2.251385in}{2.438830in}}{\pgfqpoint{2.246211in}{2.451321in}}{\pgfqpoint{2.237002in}{2.460529in}}%
\pgfpathcurveto{\pgfqpoint{2.227794in}{2.469738in}}{\pgfqpoint{2.215303in}{2.474912in}}{\pgfqpoint{2.202280in}{2.474912in}}%
\pgfpathcurveto{\pgfqpoint{2.189257in}{2.474912in}}{\pgfqpoint{2.176766in}{2.469738in}}{\pgfqpoint{2.167558in}{2.460529in}}%
\pgfpathcurveto{\pgfqpoint{2.158349in}{2.451321in}}{\pgfqpoint{2.153175in}{2.438830in}}{\pgfqpoint{2.153175in}{2.425807in}}%
\pgfpathcurveto{\pgfqpoint{2.153175in}{2.412784in}}{\pgfqpoint{2.158349in}{2.400293in}}{\pgfqpoint{2.167558in}{2.391085in}}%
\pgfpathcurveto{\pgfqpoint{2.176766in}{2.381876in}}{\pgfqpoint{2.189257in}{2.376702in}}{\pgfqpoint{2.202280in}{2.376702in}}%
\pgfpathlineto{\pgfqpoint{2.202280in}{2.376702in}}%
\pgfpathclose%
\pgfusepath{stroke,fill}%
\end{pgfscope}%
\begin{pgfscope}%
\pgfpathrectangle{\pgfqpoint{0.786164in}{0.768110in}}{\pgfqpoint{8.851069in}{7.081890in}}%
\pgfusepath{clip}%
\pgfsetbuttcap%
\pgfsetroundjoin%
\definecolor{currentfill}{rgb}{0.225863,0.330805,0.547314}%
\pgfsetfillcolor{currentfill}%
\pgfsetfillopacity{0.700000}%
\pgfsetlinewidth{0.501875pt}%
\definecolor{currentstroke}{rgb}{1.000000,1.000000,1.000000}%
\pgfsetstrokecolor{currentstroke}%
\pgfsetstrokeopacity{0.700000}%
\pgfsetdash{}{0pt}%
\pgfpathmoveto{\pgfqpoint{2.321013in}{2.442397in}}%
\pgfpathcurveto{\pgfqpoint{2.334036in}{2.442397in}}{\pgfqpoint{2.346527in}{2.447571in}}{\pgfqpoint{2.355735in}{2.456779in}}%
\pgfpathcurveto{\pgfqpoint{2.364944in}{2.465988in}}{\pgfqpoint{2.370117in}{2.478479in}}{\pgfqpoint{2.370117in}{2.491502in}}%
\pgfpathcurveto{\pgfqpoint{2.370117in}{2.504524in}}{\pgfqpoint{2.364944in}{2.517015in}}{\pgfqpoint{2.355735in}{2.526224in}}%
\pgfpathcurveto{\pgfqpoint{2.346527in}{2.535432in}}{\pgfqpoint{2.334036in}{2.540606in}}{\pgfqpoint{2.321013in}{2.540606in}}%
\pgfpathcurveto{\pgfqpoint{2.307990in}{2.540606in}}{\pgfqpoint{2.295499in}{2.535432in}}{\pgfqpoint{2.286291in}{2.526224in}}%
\pgfpathcurveto{\pgfqpoint{2.277082in}{2.517015in}}{\pgfqpoint{2.271908in}{2.504524in}}{\pgfqpoint{2.271908in}{2.491502in}}%
\pgfpathcurveto{\pgfqpoint{2.271908in}{2.478479in}}{\pgfqpoint{2.277082in}{2.465988in}}{\pgfqpoint{2.286291in}{2.456779in}}%
\pgfpathcurveto{\pgfqpoint{2.295499in}{2.447571in}}{\pgfqpoint{2.307990in}{2.442397in}}{\pgfqpoint{2.321013in}{2.442397in}}%
\pgfpathlineto{\pgfqpoint{2.321013in}{2.442397in}}%
\pgfpathclose%
\pgfusepath{stroke,fill}%
\end{pgfscope}%
\begin{pgfscope}%
\pgfpathrectangle{\pgfqpoint{0.786164in}{0.768110in}}{\pgfqpoint{8.851069in}{7.081890in}}%
\pgfusepath{clip}%
\pgfsetbuttcap%
\pgfsetroundjoin%
\definecolor{currentfill}{rgb}{0.227802,0.326594,0.546532}%
\pgfsetfillcolor{currentfill}%
\pgfsetfillopacity{0.700000}%
\pgfsetlinewidth{0.501875pt}%
\definecolor{currentstroke}{rgb}{1.000000,1.000000,1.000000}%
\pgfsetstrokecolor{currentstroke}%
\pgfsetstrokeopacity{0.700000}%
\pgfsetdash{}{0pt}%
\pgfpathmoveto{\pgfqpoint{2.302746in}{2.420499in}}%
\pgfpathcurveto{\pgfqpoint{2.315769in}{2.420499in}}{\pgfqpoint{2.328260in}{2.425673in}}{\pgfqpoint{2.337468in}{2.434881in}}%
\pgfpathcurveto{\pgfqpoint{2.346677in}{2.444090in}}{\pgfqpoint{2.351851in}{2.456581in}}{\pgfqpoint{2.351851in}{2.469603in}}%
\pgfpathcurveto{\pgfqpoint{2.351851in}{2.482626in}}{\pgfqpoint{2.346677in}{2.495117in}}{\pgfqpoint{2.337468in}{2.504326in}}%
\pgfpathcurveto{\pgfqpoint{2.328260in}{2.513534in}}{\pgfqpoint{2.315769in}{2.518708in}}{\pgfqpoint{2.302746in}{2.518708in}}%
\pgfpathcurveto{\pgfqpoint{2.289724in}{2.518708in}}{\pgfqpoint{2.277232in}{2.513534in}}{\pgfqpoint{2.268024in}{2.504326in}}%
\pgfpathcurveto{\pgfqpoint{2.258816in}{2.495117in}}{\pgfqpoint{2.253642in}{2.482626in}}{\pgfqpoint{2.253642in}{2.469603in}}%
\pgfpathcurveto{\pgfqpoint{2.253642in}{2.456581in}}{\pgfqpoint{2.258816in}{2.444090in}}{\pgfqpoint{2.268024in}{2.434881in}}%
\pgfpathcurveto{\pgfqpoint{2.277232in}{2.425673in}}{\pgfqpoint{2.289724in}{2.420499in}}{\pgfqpoint{2.302746in}{2.420499in}}%
\pgfpathlineto{\pgfqpoint{2.302746in}{2.420499in}}%
\pgfpathclose%
\pgfusepath{stroke,fill}%
\end{pgfscope}%
\begin{pgfscope}%
\pgfpathrectangle{\pgfqpoint{0.786164in}{0.768110in}}{\pgfqpoint{8.851069in}{7.081890in}}%
\pgfusepath{clip}%
\pgfsetbuttcap%
\pgfsetroundjoin%
\definecolor{currentfill}{rgb}{0.221989,0.339161,0.548752}%
\pgfsetfillcolor{currentfill}%
\pgfsetfillopacity{0.700000}%
\pgfsetlinewidth{0.501875pt}%
\definecolor{currentstroke}{rgb}{1.000000,1.000000,1.000000}%
\pgfsetstrokecolor{currentstroke}%
\pgfsetstrokeopacity{0.700000}%
\pgfsetdash{}{0pt}%
\pgfpathmoveto{\pgfqpoint{2.211413in}{2.332906in}}%
\pgfpathcurveto{\pgfqpoint{2.224436in}{2.332906in}}{\pgfqpoint{2.236927in}{2.338080in}}{\pgfqpoint{2.246136in}{2.347288in}}%
\pgfpathcurveto{\pgfqpoint{2.255344in}{2.356497in}}{\pgfqpoint{2.260518in}{2.368988in}}{\pgfqpoint{2.260518in}{2.382010in}}%
\pgfpathcurveto{\pgfqpoint{2.260518in}{2.395033in}}{\pgfqpoint{2.255344in}{2.407524in}}{\pgfqpoint{2.246136in}{2.416733in}}%
\pgfpathcurveto{\pgfqpoint{2.236927in}{2.425941in}}{\pgfqpoint{2.224436in}{2.431115in}}{\pgfqpoint{2.211413in}{2.431115in}}%
\pgfpathcurveto{\pgfqpoint{2.198391in}{2.431115in}}{\pgfqpoint{2.185900in}{2.425941in}}{\pgfqpoint{2.176691in}{2.416733in}}%
\pgfpathcurveto{\pgfqpoint{2.167483in}{2.407524in}}{\pgfqpoint{2.162309in}{2.395033in}}{\pgfqpoint{2.162309in}{2.382010in}}%
\pgfpathcurveto{\pgfqpoint{2.162309in}{2.368988in}}{\pgfqpoint{2.167483in}{2.356497in}}{\pgfqpoint{2.176691in}{2.347288in}}%
\pgfpathcurveto{\pgfqpoint{2.185900in}{2.338080in}}{\pgfqpoint{2.198391in}{2.332906in}}{\pgfqpoint{2.211413in}{2.332906in}}%
\pgfpathlineto{\pgfqpoint{2.211413in}{2.332906in}}%
\pgfpathclose%
\pgfusepath{stroke,fill}%
\end{pgfscope}%
\begin{pgfscope}%
\pgfpathrectangle{\pgfqpoint{0.786164in}{0.768110in}}{\pgfqpoint{8.851069in}{7.081890in}}%
\pgfusepath{clip}%
\pgfsetbuttcap%
\pgfsetroundjoin%
\definecolor{currentfill}{rgb}{0.195860,0.395433,0.555276}%
\pgfsetfillcolor{currentfill}%
\pgfsetfillopacity{0.700000}%
\pgfsetlinewidth{0.501875pt}%
\definecolor{currentstroke}{rgb}{1.000000,1.000000,1.000000}%
\pgfsetstrokecolor{currentstroke}%
\pgfsetstrokeopacity{0.700000}%
\pgfsetdash{}{0pt}%
\pgfpathmoveto{\pgfqpoint{1.900882in}{2.092025in}}%
\pgfpathcurveto{\pgfqpoint{1.913904in}{2.092025in}}{\pgfqpoint{1.926395in}{2.097199in}}{\pgfqpoint{1.935604in}{2.106408in}}%
\pgfpathcurveto{\pgfqpoint{1.944812in}{2.115616in}}{\pgfqpoint{1.949986in}{2.128107in}}{\pgfqpoint{1.949986in}{2.141130in}}%
\pgfpathcurveto{\pgfqpoint{1.949986in}{2.154153in}}{\pgfqpoint{1.944812in}{2.166644in}}{\pgfqpoint{1.935604in}{2.175852in}}%
\pgfpathcurveto{\pgfqpoint{1.926395in}{2.185060in}}{\pgfqpoint{1.913904in}{2.190234in}}{\pgfqpoint{1.900882in}{2.190234in}}%
\pgfpathcurveto{\pgfqpoint{1.887859in}{2.190234in}}{\pgfqpoint{1.875368in}{2.185060in}}{\pgfqpoint{1.866159in}{2.175852in}}%
\pgfpathcurveto{\pgfqpoint{1.856951in}{2.166644in}}{\pgfqpoint{1.851777in}{2.154153in}}{\pgfqpoint{1.851777in}{2.141130in}}%
\pgfpathcurveto{\pgfqpoint{1.851777in}{2.128107in}}{\pgfqpoint{1.856951in}{2.115616in}}{\pgfqpoint{1.866159in}{2.106408in}}%
\pgfpathcurveto{\pgfqpoint{1.875368in}{2.097199in}}{\pgfqpoint{1.887859in}{2.092025in}}{\pgfqpoint{1.900882in}{2.092025in}}%
\pgfpathlineto{\pgfqpoint{1.900882in}{2.092025in}}%
\pgfpathclose%
\pgfusepath{stroke,fill}%
\end{pgfscope}%
\begin{pgfscope}%
\pgfpathrectangle{\pgfqpoint{0.786164in}{0.768110in}}{\pgfqpoint{8.851069in}{7.081890in}}%
\pgfusepath{clip}%
\pgfsetbuttcap%
\pgfsetroundjoin%
\definecolor{currentfill}{rgb}{0.199430,0.387607,0.554642}%
\pgfsetfillcolor{currentfill}%
\pgfsetfillopacity{0.700000}%
\pgfsetlinewidth{0.501875pt}%
\definecolor{currentstroke}{rgb}{1.000000,1.000000,1.000000}%
\pgfsetstrokecolor{currentstroke}%
\pgfsetstrokeopacity{0.700000}%
\pgfsetdash{}{0pt}%
\pgfpathmoveto{\pgfqpoint{2.019614in}{2.179618in}}%
\pgfpathcurveto{\pgfqpoint{2.032637in}{2.179618in}}{\pgfqpoint{2.045128in}{2.184792in}}{\pgfqpoint{2.054337in}{2.194001in}}%
\pgfpathcurveto{\pgfqpoint{2.063545in}{2.203209in}}{\pgfqpoint{2.068719in}{2.215700in}}{\pgfqpoint{2.068719in}{2.228723in}}%
\pgfpathcurveto{\pgfqpoint{2.068719in}{2.241745in}}{\pgfqpoint{2.063545in}{2.254237in}}{\pgfqpoint{2.054337in}{2.263445in}}%
\pgfpathcurveto{\pgfqpoint{2.045128in}{2.272653in}}{\pgfqpoint{2.032637in}{2.277827in}}{\pgfqpoint{2.019614in}{2.277827in}}%
\pgfpathcurveto{\pgfqpoint{2.006592in}{2.277827in}}{\pgfqpoint{1.994101in}{2.272653in}}{\pgfqpoint{1.984892in}{2.263445in}}%
\pgfpathcurveto{\pgfqpoint{1.975684in}{2.254237in}}{\pgfqpoint{1.970510in}{2.241745in}}{\pgfqpoint{1.970510in}{2.228723in}}%
\pgfpathcurveto{\pgfqpoint{1.970510in}{2.215700in}}{\pgfqpoint{1.975684in}{2.203209in}}{\pgfqpoint{1.984892in}{2.194001in}}%
\pgfpathcurveto{\pgfqpoint{1.994101in}{2.184792in}}{\pgfqpoint{2.006592in}{2.179618in}}{\pgfqpoint{2.019614in}{2.179618in}}%
\pgfpathlineto{\pgfqpoint{2.019614in}{2.179618in}}%
\pgfpathclose%
\pgfusepath{stroke,fill}%
\end{pgfscope}%
\begin{pgfscope}%
\pgfpathrectangle{\pgfqpoint{0.786164in}{0.768110in}}{\pgfqpoint{8.851069in}{7.081890in}}%
\pgfusepath{clip}%
\pgfsetbuttcap%
\pgfsetroundjoin%
\definecolor{currentfill}{rgb}{0.190631,0.407061,0.556089}%
\pgfsetfillcolor{currentfill}%
\pgfsetfillopacity{0.700000}%
\pgfsetlinewidth{0.501875pt}%
\definecolor{currentstroke}{rgb}{1.000000,1.000000,1.000000}%
\pgfsetstrokecolor{currentstroke}%
\pgfsetstrokeopacity{0.700000}%
\pgfsetdash{}{0pt}%
\pgfpathmoveto{\pgfqpoint{2.037881in}{2.223415in}}%
\pgfpathcurveto{\pgfqpoint{2.050904in}{2.223415in}}{\pgfqpoint{2.063395in}{2.228589in}}{\pgfqpoint{2.072603in}{2.237797in}}%
\pgfpathcurveto{\pgfqpoint{2.081812in}{2.247005in}}{\pgfqpoint{2.086986in}{2.259497in}}{\pgfqpoint{2.086986in}{2.272519in}}%
\pgfpathcurveto{\pgfqpoint{2.086986in}{2.285542in}}{\pgfqpoint{2.081812in}{2.298033in}}{\pgfqpoint{2.072603in}{2.307241in}}%
\pgfpathcurveto{\pgfqpoint{2.063395in}{2.316450in}}{\pgfqpoint{2.050904in}{2.321624in}}{\pgfqpoint{2.037881in}{2.321624in}}%
\pgfpathcurveto{\pgfqpoint{2.024858in}{2.321624in}}{\pgfqpoint{2.012367in}{2.316450in}}{\pgfqpoint{2.003159in}{2.307241in}}%
\pgfpathcurveto{\pgfqpoint{1.993950in}{2.298033in}}{\pgfqpoint{1.988776in}{2.285542in}}{\pgfqpoint{1.988776in}{2.272519in}}%
\pgfpathcurveto{\pgfqpoint{1.988776in}{2.259497in}}{\pgfqpoint{1.993950in}{2.247005in}}{\pgfqpoint{2.003159in}{2.237797in}}%
\pgfpathcurveto{\pgfqpoint{2.012367in}{2.228589in}}{\pgfqpoint{2.024858in}{2.223415in}}{\pgfqpoint{2.037881in}{2.223415in}}%
\pgfpathlineto{\pgfqpoint{2.037881in}{2.223415in}}%
\pgfpathclose%
\pgfusepath{stroke,fill}%
\end{pgfscope}%
\begin{pgfscope}%
\pgfpathrectangle{\pgfqpoint{0.786164in}{0.768110in}}{\pgfqpoint{8.851069in}{7.081890in}}%
\pgfusepath{clip}%
\pgfsetbuttcap%
\pgfsetroundjoin%
\definecolor{currentfill}{rgb}{0.147607,0.511733,0.557049}%
\pgfsetfillcolor{currentfill}%
\pgfsetfillopacity{0.700000}%
\pgfsetlinewidth{0.501875pt}%
\definecolor{currentstroke}{rgb}{1.000000,1.000000,1.000000}%
\pgfsetstrokecolor{currentstroke}%
\pgfsetstrokeopacity{0.700000}%
\pgfsetdash{}{0pt}%
\pgfpathmoveto{\pgfqpoint{6.257460in}{3.449716in}}%
\pgfpathcurveto{\pgfqpoint{6.270482in}{3.449716in}}{\pgfqpoint{6.282974in}{3.454890in}}{\pgfqpoint{6.292182in}{3.464098in}}%
\pgfpathcurveto{\pgfqpoint{6.301390in}{3.473307in}}{\pgfqpoint{6.306564in}{3.485798in}}{\pgfqpoint{6.306564in}{3.498820in}}%
\pgfpathcurveto{\pgfqpoint{6.306564in}{3.511843in}}{\pgfqpoint{6.301390in}{3.524334in}}{\pgfqpoint{6.292182in}{3.533543in}}%
\pgfpathcurveto{\pgfqpoint{6.282974in}{3.542751in}}{\pgfqpoint{6.270482in}{3.547925in}}{\pgfqpoint{6.257460in}{3.547925in}}%
\pgfpathcurveto{\pgfqpoint{6.244437in}{3.547925in}}{\pgfqpoint{6.231946in}{3.542751in}}{\pgfqpoint{6.222738in}{3.533543in}}%
\pgfpathcurveto{\pgfqpoint{6.213529in}{3.524334in}}{\pgfqpoint{6.208355in}{3.511843in}}{\pgfqpoint{6.208355in}{3.498820in}}%
\pgfpathcurveto{\pgfqpoint{6.208355in}{3.485798in}}{\pgfqpoint{6.213529in}{3.473307in}}{\pgfqpoint{6.222738in}{3.464098in}}%
\pgfpathcurveto{\pgfqpoint{6.231946in}{3.454890in}}{\pgfqpoint{6.244437in}{3.449716in}}{\pgfqpoint{6.257460in}{3.449716in}}%
\pgfpathlineto{\pgfqpoint{6.257460in}{3.449716in}}%
\pgfpathclose%
\pgfusepath{stroke,fill}%
\end{pgfscope}%
\begin{pgfscope}%
\pgfpathrectangle{\pgfqpoint{0.786164in}{0.768110in}}{\pgfqpoint{8.851069in}{7.081890in}}%
\pgfusepath{clip}%
\pgfsetbuttcap%
\pgfsetroundjoin%
\definecolor{currentfill}{rgb}{0.150476,0.504369,0.557430}%
\pgfsetfillcolor{currentfill}%
\pgfsetfillopacity{0.700000}%
\pgfsetlinewidth{0.501875pt}%
\definecolor{currentstroke}{rgb}{1.000000,1.000000,1.000000}%
\pgfsetstrokecolor{currentstroke}%
\pgfsetstrokeopacity{0.700000}%
\pgfsetdash{}{0pt}%
\pgfpathmoveto{\pgfqpoint{5.718596in}{2.946056in}}%
\pgfpathcurveto{\pgfqpoint{5.731618in}{2.946056in}}{\pgfqpoint{5.744110in}{2.951230in}}{\pgfqpoint{5.753318in}{2.960439in}}%
\pgfpathcurveto{\pgfqpoint{5.762526in}{2.969647in}}{\pgfqpoint{5.767700in}{2.982138in}}{\pgfqpoint{5.767700in}{2.995161in}}%
\pgfpathcurveto{\pgfqpoint{5.767700in}{3.008184in}}{\pgfqpoint{5.762526in}{3.020675in}}{\pgfqpoint{5.753318in}{3.029883in}}%
\pgfpathcurveto{\pgfqpoint{5.744110in}{3.039092in}}{\pgfqpoint{5.731618in}{3.044266in}}{\pgfqpoint{5.718596in}{3.044266in}}%
\pgfpathcurveto{\pgfqpoint{5.705573in}{3.044266in}}{\pgfqpoint{5.693082in}{3.039092in}}{\pgfqpoint{5.683874in}{3.029883in}}%
\pgfpathcurveto{\pgfqpoint{5.674665in}{3.020675in}}{\pgfqpoint{5.669491in}{3.008184in}}{\pgfqpoint{5.669491in}{2.995161in}}%
\pgfpathcurveto{\pgfqpoint{5.669491in}{2.982138in}}{\pgfqpoint{5.674665in}{2.969647in}}{\pgfqpoint{5.683874in}{2.960439in}}%
\pgfpathcurveto{\pgfqpoint{5.693082in}{2.951230in}}{\pgfqpoint{5.705573in}{2.946056in}}{\pgfqpoint{5.718596in}{2.946056in}}%
\pgfpathlineto{\pgfqpoint{5.718596in}{2.946056in}}%
\pgfpathclose%
\pgfusepath{stroke,fill}%
\end{pgfscope}%
\begin{pgfscope}%
\pgfpathrectangle{\pgfqpoint{0.786164in}{0.768110in}}{\pgfqpoint{8.851069in}{7.081890in}}%
\pgfusepath{clip}%
\pgfsetbuttcap%
\pgfsetroundjoin%
\definecolor{currentfill}{rgb}{0.143343,0.522773,0.556295}%
\pgfsetfillcolor{currentfill}%
\pgfsetfillopacity{0.700000}%
\pgfsetlinewidth{0.501875pt}%
\definecolor{currentstroke}{rgb}{1.000000,1.000000,1.000000}%
\pgfsetstrokecolor{currentstroke}%
\pgfsetstrokeopacity{0.700000}%
\pgfsetdash{}{0pt}%
\pgfpathmoveto{\pgfqpoint{6.230060in}{3.252632in}}%
\pgfpathcurveto{\pgfqpoint{6.243083in}{3.252632in}}{\pgfqpoint{6.255574in}{3.257806in}}{\pgfqpoint{6.264782in}{3.267014in}}%
\pgfpathcurveto{\pgfqpoint{6.273991in}{3.276223in}}{\pgfqpoint{6.279165in}{3.288714in}}{\pgfqpoint{6.279165in}{3.301736in}}%
\pgfpathcurveto{\pgfqpoint{6.279165in}{3.314759in}}{\pgfqpoint{6.273991in}{3.327250in}}{\pgfqpoint{6.264782in}{3.336459in}}%
\pgfpathcurveto{\pgfqpoint{6.255574in}{3.345667in}}{\pgfqpoint{6.243083in}{3.350841in}}{\pgfqpoint{6.230060in}{3.350841in}}%
\pgfpathcurveto{\pgfqpoint{6.217037in}{3.350841in}}{\pgfqpoint{6.204546in}{3.345667in}}{\pgfqpoint{6.195338in}{3.336459in}}%
\pgfpathcurveto{\pgfqpoint{6.186129in}{3.327250in}}{\pgfqpoint{6.180955in}{3.314759in}}{\pgfqpoint{6.180955in}{3.301736in}}%
\pgfpathcurveto{\pgfqpoint{6.180955in}{3.288714in}}{\pgfqpoint{6.186129in}{3.276223in}}{\pgfqpoint{6.195338in}{3.267014in}}%
\pgfpathcurveto{\pgfqpoint{6.204546in}{3.257806in}}{\pgfqpoint{6.217037in}{3.252632in}}{\pgfqpoint{6.230060in}{3.252632in}}%
\pgfpathlineto{\pgfqpoint{6.230060in}{3.252632in}}%
\pgfpathclose%
\pgfusepath{stroke,fill}%
\end{pgfscope}%
\begin{pgfscope}%
\pgfpathrectangle{\pgfqpoint{0.786164in}{0.768110in}}{\pgfqpoint{8.851069in}{7.081890in}}%
\pgfusepath{clip}%
\pgfsetbuttcap%
\pgfsetroundjoin%
\definecolor{currentfill}{rgb}{0.146180,0.515413,0.556823}%
\pgfsetfillcolor{currentfill}%
\pgfsetfillopacity{0.700000}%
\pgfsetlinewidth{0.501875pt}%
\definecolor{currentstroke}{rgb}{1.000000,1.000000,1.000000}%
\pgfsetstrokecolor{currentstroke}%
\pgfsetstrokeopacity{0.700000}%
\pgfsetdash{}{0pt}%
\pgfpathmoveto{\pgfqpoint{6.093061in}{3.581105in}}%
\pgfpathcurveto{\pgfqpoint{6.106083in}{3.581105in}}{\pgfqpoint{6.118574in}{3.586279in}}{\pgfqpoint{6.127783in}{3.595488in}}%
\pgfpathcurveto{\pgfqpoint{6.136991in}{3.604696in}}{\pgfqpoint{6.142165in}{3.617187in}}{\pgfqpoint{6.142165in}{3.630210in}}%
\pgfpathcurveto{\pgfqpoint{6.142165in}{3.643233in}}{\pgfqpoint{6.136991in}{3.655724in}}{\pgfqpoint{6.127783in}{3.664932in}}%
\pgfpathcurveto{\pgfqpoint{6.118574in}{3.674141in}}{\pgfqpoint{6.106083in}{3.679315in}}{\pgfqpoint{6.093061in}{3.679315in}}%
\pgfpathcurveto{\pgfqpoint{6.080038in}{3.679315in}}{\pgfqpoint{6.067547in}{3.674141in}}{\pgfqpoint{6.058338in}{3.664932in}}%
\pgfpathcurveto{\pgfqpoint{6.049130in}{3.655724in}}{\pgfqpoint{6.043956in}{3.643233in}}{\pgfqpoint{6.043956in}{3.630210in}}%
\pgfpathcurveto{\pgfqpoint{6.043956in}{3.617187in}}{\pgfqpoint{6.049130in}{3.604696in}}{\pgfqpoint{6.058338in}{3.595488in}}%
\pgfpathcurveto{\pgfqpoint{6.067547in}{3.586279in}}{\pgfqpoint{6.080038in}{3.581105in}}{\pgfqpoint{6.093061in}{3.581105in}}%
\pgfpathlineto{\pgfqpoint{6.093061in}{3.581105in}}%
\pgfpathclose%
\pgfusepath{stroke,fill}%
\end{pgfscope}%
\begin{pgfscope}%
\pgfpathrectangle{\pgfqpoint{0.786164in}{0.768110in}}{\pgfqpoint{8.851069in}{7.081890in}}%
\pgfusepath{clip}%
\pgfsetbuttcap%
\pgfsetroundjoin%
\definecolor{currentfill}{rgb}{0.137770,0.537492,0.554906}%
\pgfsetfillcolor{currentfill}%
\pgfsetfillopacity{0.700000}%
\pgfsetlinewidth{0.501875pt}%
\definecolor{currentstroke}{rgb}{1.000000,1.000000,1.000000}%
\pgfsetstrokecolor{currentstroke}%
\pgfsetstrokeopacity{0.700000}%
\pgfsetdash{}{0pt}%
\pgfpathmoveto{\pgfqpoint{5.809929in}{3.143141in}}%
\pgfpathcurveto{\pgfqpoint{5.822951in}{3.143141in}}{\pgfqpoint{5.835442in}{3.148314in}}{\pgfqpoint{5.844651in}{3.157523in}}%
\pgfpathcurveto{\pgfqpoint{5.853859in}{3.166731in}}{\pgfqpoint{5.859033in}{3.179222in}}{\pgfqpoint{5.859033in}{3.192245in}}%
\pgfpathcurveto{\pgfqpoint{5.859033in}{3.205268in}}{\pgfqpoint{5.853859in}{3.217759in}}{\pgfqpoint{5.844651in}{3.226967in}}%
\pgfpathcurveto{\pgfqpoint{5.835442in}{3.236176in}}{\pgfqpoint{5.822951in}{3.241350in}}{\pgfqpoint{5.809929in}{3.241350in}}%
\pgfpathcurveto{\pgfqpoint{5.796906in}{3.241350in}}{\pgfqpoint{5.784415in}{3.236176in}}{\pgfqpoint{5.775206in}{3.226967in}}%
\pgfpathcurveto{\pgfqpoint{5.765998in}{3.217759in}}{\pgfqpoint{5.760824in}{3.205268in}}{\pgfqpoint{5.760824in}{3.192245in}}%
\pgfpathcurveto{\pgfqpoint{5.760824in}{3.179222in}}{\pgfqpoint{5.765998in}{3.166731in}}{\pgfqpoint{5.775206in}{3.157523in}}%
\pgfpathcurveto{\pgfqpoint{5.784415in}{3.148314in}}{\pgfqpoint{5.796906in}{3.143141in}}{\pgfqpoint{5.809929in}{3.143141in}}%
\pgfpathlineto{\pgfqpoint{5.809929in}{3.143141in}}%
\pgfpathclose%
\pgfusepath{stroke,fill}%
\end{pgfscope}%
\begin{pgfscope}%
\pgfpathrectangle{\pgfqpoint{0.786164in}{0.768110in}}{\pgfqpoint{8.851069in}{7.081890in}}%
\pgfusepath{clip}%
\pgfsetbuttcap%
\pgfsetroundjoin%
\definecolor{currentfill}{rgb}{0.137770,0.537492,0.554906}%
\pgfsetfillcolor{currentfill}%
\pgfsetfillopacity{0.700000}%
\pgfsetlinewidth{0.501875pt}%
\definecolor{currentstroke}{rgb}{1.000000,1.000000,1.000000}%
\pgfsetstrokecolor{currentstroke}%
\pgfsetstrokeopacity{0.700000}%
\pgfsetdash{}{0pt}%
\pgfpathmoveto{\pgfqpoint{5.408064in}{2.464295in}}%
\pgfpathcurveto{\pgfqpoint{5.421087in}{2.464295in}}{\pgfqpoint{5.433578in}{2.469469in}}{\pgfqpoint{5.442786in}{2.478678in}}%
\pgfpathcurveto{\pgfqpoint{5.451995in}{2.487886in}}{\pgfqpoint{5.457169in}{2.500377in}}{\pgfqpoint{5.457169in}{2.513400in}}%
\pgfpathcurveto{\pgfqpoint{5.457169in}{2.526423in}}{\pgfqpoint{5.451995in}{2.538914in}}{\pgfqpoint{5.442786in}{2.548122in}}%
\pgfpathcurveto{\pgfqpoint{5.433578in}{2.557331in}}{\pgfqpoint{5.421087in}{2.562504in}}{\pgfqpoint{5.408064in}{2.562504in}}%
\pgfpathcurveto{\pgfqpoint{5.395041in}{2.562504in}}{\pgfqpoint{5.382550in}{2.557331in}}{\pgfqpoint{5.373342in}{2.548122in}}%
\pgfpathcurveto{\pgfqpoint{5.364133in}{2.538914in}}{\pgfqpoint{5.358959in}{2.526423in}}{\pgfqpoint{5.358959in}{2.513400in}}%
\pgfpathcurveto{\pgfqpoint{5.358959in}{2.500377in}}{\pgfqpoint{5.364133in}{2.487886in}}{\pgfqpoint{5.373342in}{2.478678in}}%
\pgfpathcurveto{\pgfqpoint{5.382550in}{2.469469in}}{\pgfqpoint{5.395041in}{2.464295in}}{\pgfqpoint{5.408064in}{2.464295in}}%
\pgfpathlineto{\pgfqpoint{5.408064in}{2.464295in}}%
\pgfpathclose%
\pgfusepath{stroke,fill}%
\end{pgfscope}%
\begin{pgfscope}%
\pgfpathrectangle{\pgfqpoint{0.786164in}{0.768110in}}{\pgfqpoint{8.851069in}{7.081890in}}%
\pgfusepath{clip}%
\pgfsetbuttcap%
\pgfsetroundjoin%
\definecolor{currentfill}{rgb}{0.132444,0.552216,0.553018}%
\pgfsetfillcolor{currentfill}%
\pgfsetfillopacity{0.700000}%
\pgfsetlinewidth{0.501875pt}%
\definecolor{currentstroke}{rgb}{1.000000,1.000000,1.000000}%
\pgfsetstrokecolor{currentstroke}%
\pgfsetstrokeopacity{0.700000}%
\pgfsetdash{}{0pt}%
\pgfpathmoveto{\pgfqpoint{5.910395in}{3.252632in}}%
\pgfpathcurveto{\pgfqpoint{5.923418in}{3.252632in}}{\pgfqpoint{5.935909in}{3.257806in}}{\pgfqpoint{5.945117in}{3.267014in}}%
\pgfpathcurveto{\pgfqpoint{5.954325in}{3.276223in}}{\pgfqpoint{5.959499in}{3.288714in}}{\pgfqpoint{5.959499in}{3.301736in}}%
\pgfpathcurveto{\pgfqpoint{5.959499in}{3.314759in}}{\pgfqpoint{5.954325in}{3.327250in}}{\pgfqpoint{5.945117in}{3.336459in}}%
\pgfpathcurveto{\pgfqpoint{5.935909in}{3.345667in}}{\pgfqpoint{5.923418in}{3.350841in}}{\pgfqpoint{5.910395in}{3.350841in}}%
\pgfpathcurveto{\pgfqpoint{5.897372in}{3.350841in}}{\pgfqpoint{5.884881in}{3.345667in}}{\pgfqpoint{5.875673in}{3.336459in}}%
\pgfpathcurveto{\pgfqpoint{5.866464in}{3.327250in}}{\pgfqpoint{5.861290in}{3.314759in}}{\pgfqpoint{5.861290in}{3.301736in}}%
\pgfpathcurveto{\pgfqpoint{5.861290in}{3.288714in}}{\pgfqpoint{5.866464in}{3.276223in}}{\pgfqpoint{5.875673in}{3.267014in}}%
\pgfpathcurveto{\pgfqpoint{5.884881in}{3.257806in}}{\pgfqpoint{5.897372in}{3.252632in}}{\pgfqpoint{5.910395in}{3.252632in}}%
\pgfpathlineto{\pgfqpoint{5.910395in}{3.252632in}}%
\pgfpathclose%
\pgfusepath{stroke,fill}%
\end{pgfscope}%
\begin{pgfscope}%
\pgfpathrectangle{\pgfqpoint{0.786164in}{0.768110in}}{\pgfqpoint{8.851069in}{7.081890in}}%
\pgfusepath{clip}%
\pgfsetbuttcap%
\pgfsetroundjoin%
\definecolor{currentfill}{rgb}{0.131172,0.555899,0.552459}%
\pgfsetfillcolor{currentfill}%
\pgfsetfillopacity{0.700000}%
\pgfsetlinewidth{0.501875pt}%
\definecolor{currentstroke}{rgb}{1.000000,1.000000,1.000000}%
\pgfsetstrokecolor{currentstroke}%
\pgfsetstrokeopacity{0.700000}%
\pgfsetdash{}{0pt}%
\pgfpathmoveto{\pgfqpoint{5.672929in}{2.967955in}}%
\pgfpathcurveto{\pgfqpoint{5.685952in}{2.967955in}}{\pgfqpoint{5.698443in}{2.973129in}}{\pgfqpoint{5.707652in}{2.982337in}}%
\pgfpathcurveto{\pgfqpoint{5.716860in}{2.991545in}}{\pgfqpoint{5.722034in}{3.004037in}}{\pgfqpoint{5.722034in}{3.017059in}}%
\pgfpathcurveto{\pgfqpoint{5.722034in}{3.030082in}}{\pgfqpoint{5.716860in}{3.042573in}}{\pgfqpoint{5.707652in}{3.051781in}}%
\pgfpathcurveto{\pgfqpoint{5.698443in}{3.060990in}}{\pgfqpoint{5.685952in}{3.066164in}}{\pgfqpoint{5.672929in}{3.066164in}}%
\pgfpathcurveto{\pgfqpoint{5.659907in}{3.066164in}}{\pgfqpoint{5.647416in}{3.060990in}}{\pgfqpoint{5.638207in}{3.051781in}}%
\pgfpathcurveto{\pgfqpoint{5.628999in}{3.042573in}}{\pgfqpoint{5.623825in}{3.030082in}}{\pgfqpoint{5.623825in}{3.017059in}}%
\pgfpathcurveto{\pgfqpoint{5.623825in}{3.004037in}}{\pgfqpoint{5.628999in}{2.991545in}}{\pgfqpoint{5.638207in}{2.982337in}}%
\pgfpathcurveto{\pgfqpoint{5.647416in}{2.973129in}}{\pgfqpoint{5.659907in}{2.967955in}}{\pgfqpoint{5.672929in}{2.967955in}}%
\pgfpathlineto{\pgfqpoint{5.672929in}{2.967955in}}%
\pgfpathclose%
\pgfusepath{stroke,fill}%
\end{pgfscope}%
\begin{pgfscope}%
\pgfpathrectangle{\pgfqpoint{0.786164in}{0.768110in}}{\pgfqpoint{8.851069in}{7.081890in}}%
\pgfusepath{clip}%
\pgfsetbuttcap%
\pgfsetroundjoin%
\definecolor{currentfill}{rgb}{0.123463,0.581687,0.547445}%
\pgfsetfillcolor{currentfill}%
\pgfsetfillopacity{0.700000}%
\pgfsetlinewidth{0.501875pt}%
\definecolor{currentstroke}{rgb}{1.000000,1.000000,1.000000}%
\pgfsetstrokecolor{currentstroke}%
\pgfsetstrokeopacity{0.700000}%
\pgfsetdash{}{0pt}%
\pgfpathmoveto{\pgfqpoint{5.435464in}{2.661379in}}%
\pgfpathcurveto{\pgfqpoint{5.448487in}{2.661379in}}{\pgfqpoint{5.460978in}{2.666553in}}{\pgfqpoint{5.470186in}{2.675762in}}%
\pgfpathcurveto{\pgfqpoint{5.479395in}{2.684970in}}{\pgfqpoint{5.484569in}{2.697461in}}{\pgfqpoint{5.484569in}{2.710484in}}%
\pgfpathcurveto{\pgfqpoint{5.484569in}{2.723507in}}{\pgfqpoint{5.479395in}{2.735998in}}{\pgfqpoint{5.470186in}{2.745206in}}%
\pgfpathcurveto{\pgfqpoint{5.460978in}{2.754415in}}{\pgfqpoint{5.448487in}{2.759589in}}{\pgfqpoint{5.435464in}{2.759589in}}%
\pgfpathcurveto{\pgfqpoint{5.422441in}{2.759589in}}{\pgfqpoint{5.409950in}{2.754415in}}{\pgfqpoint{5.400742in}{2.745206in}}%
\pgfpathcurveto{\pgfqpoint{5.391533in}{2.735998in}}{\pgfqpoint{5.386359in}{2.723507in}}{\pgfqpoint{5.386359in}{2.710484in}}%
\pgfpathcurveto{\pgfqpoint{5.386359in}{2.697461in}}{\pgfqpoint{5.391533in}{2.684970in}}{\pgfqpoint{5.400742in}{2.675762in}}%
\pgfpathcurveto{\pgfqpoint{5.409950in}{2.666553in}}{\pgfqpoint{5.422441in}{2.661379in}}{\pgfqpoint{5.435464in}{2.661379in}}%
\pgfpathlineto{\pgfqpoint{5.435464in}{2.661379in}}%
\pgfpathclose%
\pgfusepath{stroke,fill}%
\end{pgfscope}%
\begin{pgfscope}%
\pgfpathrectangle{\pgfqpoint{0.786164in}{0.768110in}}{\pgfqpoint{8.851069in}{7.081890in}}%
\pgfusepath{clip}%
\pgfsetbuttcap%
\pgfsetroundjoin%
\definecolor{currentfill}{rgb}{0.119483,0.614817,0.537692}%
\pgfsetfillcolor{currentfill}%
\pgfsetfillopacity{0.700000}%
\pgfsetlinewidth{0.501875pt}%
\definecolor{currentstroke}{rgb}{1.000000,1.000000,1.000000}%
\pgfsetstrokecolor{currentstroke}%
\pgfsetstrokeopacity{0.700000}%
\pgfsetdash{}{0pt}%
\pgfpathmoveto{\pgfqpoint{5.243665in}{2.946056in}}%
\pgfpathcurveto{\pgfqpoint{5.256688in}{2.946056in}}{\pgfqpoint{5.269179in}{2.951230in}}{\pgfqpoint{5.278387in}{2.960439in}}%
\pgfpathcurveto{\pgfqpoint{5.287596in}{2.969647in}}{\pgfqpoint{5.292769in}{2.982138in}}{\pgfqpoint{5.292769in}{2.995161in}}%
\pgfpathcurveto{\pgfqpoint{5.292769in}{3.008184in}}{\pgfqpoint{5.287596in}{3.020675in}}{\pgfqpoint{5.278387in}{3.029883in}}%
\pgfpathcurveto{\pgfqpoint{5.269179in}{3.039092in}}{\pgfqpoint{5.256688in}{3.044266in}}{\pgfqpoint{5.243665in}{3.044266in}}%
\pgfpathcurveto{\pgfqpoint{5.230642in}{3.044266in}}{\pgfqpoint{5.218151in}{3.039092in}}{\pgfqpoint{5.208943in}{3.029883in}}%
\pgfpathcurveto{\pgfqpoint{5.199734in}{3.020675in}}{\pgfqpoint{5.194560in}{3.008184in}}{\pgfqpoint{5.194560in}{2.995161in}}%
\pgfpathcurveto{\pgfqpoint{5.194560in}{2.982138in}}{\pgfqpoint{5.199734in}{2.969647in}}{\pgfqpoint{5.208943in}{2.960439in}}%
\pgfpathcurveto{\pgfqpoint{5.218151in}{2.951230in}}{\pgfqpoint{5.230642in}{2.946056in}}{\pgfqpoint{5.243665in}{2.946056in}}%
\pgfpathlineto{\pgfqpoint{5.243665in}{2.946056in}}%
\pgfpathclose%
\pgfusepath{stroke,fill}%
\end{pgfscope}%
\begin{pgfscope}%
\pgfpathrectangle{\pgfqpoint{0.786164in}{0.768110in}}{\pgfqpoint{8.851069in}{7.081890in}}%
\pgfusepath{clip}%
\pgfsetbuttcap%
\pgfsetroundjoin%
\definecolor{currentfill}{rgb}{0.126326,0.644107,0.525311}%
\pgfsetfillcolor{currentfill}%
\pgfsetfillopacity{0.700000}%
\pgfsetlinewidth{0.501875pt}%
\definecolor{currentstroke}{rgb}{1.000000,1.000000,1.000000}%
\pgfsetstrokecolor{currentstroke}%
\pgfsetstrokeopacity{0.700000}%
\pgfsetdash{}{0pt}%
\pgfpathmoveto{\pgfqpoint{5.334998in}{2.727074in}}%
\pgfpathcurveto{\pgfqpoint{5.348020in}{2.727074in}}{\pgfqpoint{5.360511in}{2.732248in}}{\pgfqpoint{5.369720in}{2.741456in}}%
\pgfpathcurveto{\pgfqpoint{5.378928in}{2.750665in}}{\pgfqpoint{5.384102in}{2.763156in}}{\pgfqpoint{5.384102in}{2.776179in}}%
\pgfpathcurveto{\pgfqpoint{5.384102in}{2.789201in}}{\pgfqpoint{5.378928in}{2.801692in}}{\pgfqpoint{5.369720in}{2.810901in}}%
\pgfpathcurveto{\pgfqpoint{5.360511in}{2.820109in}}{\pgfqpoint{5.348020in}{2.825283in}}{\pgfqpoint{5.334998in}{2.825283in}}%
\pgfpathcurveto{\pgfqpoint{5.321975in}{2.825283in}}{\pgfqpoint{5.309484in}{2.820109in}}{\pgfqpoint{5.300275in}{2.810901in}}%
\pgfpathcurveto{\pgfqpoint{5.291067in}{2.801692in}}{\pgfqpoint{5.285893in}{2.789201in}}{\pgfqpoint{5.285893in}{2.776179in}}%
\pgfpathcurveto{\pgfqpoint{5.285893in}{2.763156in}}{\pgfqpoint{5.291067in}{2.750665in}}{\pgfqpoint{5.300275in}{2.741456in}}%
\pgfpathcurveto{\pgfqpoint{5.309484in}{2.732248in}}{\pgfqpoint{5.321975in}{2.727074in}}{\pgfqpoint{5.334998in}{2.727074in}}%
\pgfpathlineto{\pgfqpoint{5.334998in}{2.727074in}}%
\pgfpathclose%
\pgfusepath{stroke,fill}%
\end{pgfscope}%
\begin{pgfscope}%
\pgfpathrectangle{\pgfqpoint{0.786164in}{0.768110in}}{\pgfqpoint{8.851069in}{7.081890in}}%
\pgfusepath{clip}%
\pgfsetbuttcap%
\pgfsetroundjoin%
\definecolor{currentfill}{rgb}{0.130067,0.651384,0.521608}%
\pgfsetfillcolor{currentfill}%
\pgfsetfillopacity{0.700000}%
\pgfsetlinewidth{0.501875pt}%
\definecolor{currentstroke}{rgb}{1.000000,1.000000,1.000000}%
\pgfsetstrokecolor{currentstroke}%
\pgfsetstrokeopacity{0.700000}%
\pgfsetdash{}{0pt}%
\pgfpathmoveto{\pgfqpoint{5.070132in}{2.770871in}}%
\pgfpathcurveto{\pgfqpoint{5.083155in}{2.770871in}}{\pgfqpoint{5.095646in}{2.776044in}}{\pgfqpoint{5.104855in}{2.785253in}}%
\pgfpathcurveto{\pgfqpoint{5.114063in}{2.794461in}}{\pgfqpoint{5.119237in}{2.806952in}}{\pgfqpoint{5.119237in}{2.819975in}}%
\pgfpathcurveto{\pgfqpoint{5.119237in}{2.832998in}}{\pgfqpoint{5.114063in}{2.845489in}}{\pgfqpoint{5.104855in}{2.854697in}}%
\pgfpathcurveto{\pgfqpoint{5.095646in}{2.863906in}}{\pgfqpoint{5.083155in}{2.869080in}}{\pgfqpoint{5.070132in}{2.869080in}}%
\pgfpathcurveto{\pgfqpoint{5.057110in}{2.869080in}}{\pgfqpoint{5.044619in}{2.863906in}}{\pgfqpoint{5.035410in}{2.854697in}}%
\pgfpathcurveto{\pgfqpoint{5.026202in}{2.845489in}}{\pgfqpoint{5.021028in}{2.832998in}}{\pgfqpoint{5.021028in}{2.819975in}}%
\pgfpathcurveto{\pgfqpoint{5.021028in}{2.806952in}}{\pgfqpoint{5.026202in}{2.794461in}}{\pgfqpoint{5.035410in}{2.785253in}}%
\pgfpathcurveto{\pgfqpoint{5.044619in}{2.776044in}}{\pgfqpoint{5.057110in}{2.770871in}}{\pgfqpoint{5.070132in}{2.770871in}}%
\pgfpathlineto{\pgfqpoint{5.070132in}{2.770871in}}%
\pgfpathclose%
\pgfusepath{stroke,fill}%
\end{pgfscope}%
\begin{pgfscope}%
\pgfpathrectangle{\pgfqpoint{0.786164in}{0.768110in}}{\pgfqpoint{8.851069in}{7.081890in}}%
\pgfusepath{clip}%
\pgfsetbuttcap%
\pgfsetroundjoin%
\definecolor{currentfill}{rgb}{0.128087,0.647749,0.523491}%
\pgfsetfillcolor{currentfill}%
\pgfsetfillopacity{0.700000}%
\pgfsetlinewidth{0.501875pt}%
\definecolor{currentstroke}{rgb}{1.000000,1.000000,1.000000}%
\pgfsetstrokecolor{currentstroke}%
\pgfsetstrokeopacity{0.700000}%
\pgfsetdash{}{0pt}%
\pgfpathmoveto{\pgfqpoint{5.234532in}{3.033649in}}%
\pgfpathcurveto{\pgfqpoint{5.247554in}{3.033649in}}{\pgfqpoint{5.260045in}{3.038823in}}{\pgfqpoint{5.269254in}{3.048032in}}%
\pgfpathcurveto{\pgfqpoint{5.278462in}{3.057240in}}{\pgfqpoint{5.283636in}{3.069731in}}{\pgfqpoint{5.283636in}{3.082754in}}%
\pgfpathcurveto{\pgfqpoint{5.283636in}{3.095777in}}{\pgfqpoint{5.278462in}{3.108268in}}{\pgfqpoint{5.269254in}{3.117476in}}%
\pgfpathcurveto{\pgfqpoint{5.260045in}{3.126685in}}{\pgfqpoint{5.247554in}{3.131859in}}{\pgfqpoint{5.234532in}{3.131859in}}%
\pgfpathcurveto{\pgfqpoint{5.221509in}{3.131859in}}{\pgfqpoint{5.209018in}{3.126685in}}{\pgfqpoint{5.199809in}{3.117476in}}%
\pgfpathcurveto{\pgfqpoint{5.190601in}{3.108268in}}{\pgfqpoint{5.185427in}{3.095777in}}{\pgfqpoint{5.185427in}{3.082754in}}%
\pgfpathcurveto{\pgfqpoint{5.185427in}{3.069731in}}{\pgfqpoint{5.190601in}{3.057240in}}{\pgfqpoint{5.199809in}{3.048032in}}%
\pgfpathcurveto{\pgfqpoint{5.209018in}{3.038823in}}{\pgfqpoint{5.221509in}{3.033649in}}{\pgfqpoint{5.234532in}{3.033649in}}%
\pgfpathlineto{\pgfqpoint{5.234532in}{3.033649in}}%
\pgfpathclose%
\pgfusepath{stroke,fill}%
\end{pgfscope}%
\begin{pgfscope}%
\pgfpathrectangle{\pgfqpoint{0.786164in}{0.768110in}}{\pgfqpoint{8.851069in}{7.081890in}}%
\pgfusepath{clip}%
\pgfsetbuttcap%
\pgfsetroundjoin%
\definecolor{currentfill}{rgb}{0.150148,0.676631,0.506589}%
\pgfsetfillcolor{currentfill}%
\pgfsetfillopacity{0.700000}%
\pgfsetlinewidth{0.501875pt}%
\definecolor{currentstroke}{rgb}{1.000000,1.000000,1.000000}%
\pgfsetstrokecolor{currentstroke}%
\pgfsetstrokeopacity{0.700000}%
\pgfsetdash{}{0pt}%
\pgfpathmoveto{\pgfqpoint{5.207132in}{3.011751in}}%
\pgfpathcurveto{\pgfqpoint{5.220154in}{3.011751in}}{\pgfqpoint{5.232645in}{3.016925in}}{\pgfqpoint{5.241854in}{3.026134in}}%
\pgfpathcurveto{\pgfqpoint{5.251062in}{3.035342in}}{\pgfqpoint{5.256236in}{3.047833in}}{\pgfqpoint{5.256236in}{3.060856in}}%
\pgfpathcurveto{\pgfqpoint{5.256236in}{3.073878in}}{\pgfqpoint{5.251062in}{3.086370in}}{\pgfqpoint{5.241854in}{3.095578in}}%
\pgfpathcurveto{\pgfqpoint{5.232645in}{3.104786in}}{\pgfqpoint{5.220154in}{3.109960in}}{\pgfqpoint{5.207132in}{3.109960in}}%
\pgfpathcurveto{\pgfqpoint{5.194109in}{3.109960in}}{\pgfqpoint{5.181618in}{3.104786in}}{\pgfqpoint{5.172409in}{3.095578in}}%
\pgfpathcurveto{\pgfqpoint{5.163201in}{3.086370in}}{\pgfqpoint{5.158027in}{3.073878in}}{\pgfqpoint{5.158027in}{3.060856in}}%
\pgfpathcurveto{\pgfqpoint{5.158027in}{3.047833in}}{\pgfqpoint{5.163201in}{3.035342in}}{\pgfqpoint{5.172409in}{3.026134in}}%
\pgfpathcurveto{\pgfqpoint{5.181618in}{3.016925in}}{\pgfqpoint{5.194109in}{3.011751in}}{\pgfqpoint{5.207132in}{3.011751in}}%
\pgfpathlineto{\pgfqpoint{5.207132in}{3.011751in}}%
\pgfpathclose%
\pgfusepath{stroke,fill}%
\end{pgfscope}%
\begin{pgfscope}%
\pgfpathrectangle{\pgfqpoint{0.786164in}{0.768110in}}{\pgfqpoint{8.851069in}{7.081890in}}%
\pgfusepath{clip}%
\pgfsetbuttcap%
\pgfsetroundjoin%
\definecolor{currentfill}{rgb}{0.153894,0.680203,0.504172}%
\pgfsetfillcolor{currentfill}%
\pgfsetfillopacity{0.700000}%
\pgfsetlinewidth{0.501875pt}%
\definecolor{currentstroke}{rgb}{1.000000,1.000000,1.000000}%
\pgfsetstrokecolor{currentstroke}%
\pgfsetstrokeopacity{0.700000}%
\pgfsetdash{}{0pt}%
\pgfpathmoveto{\pgfqpoint{5.307598in}{3.427818in}}%
\pgfpathcurveto{\pgfqpoint{5.320621in}{3.427818in}}{\pgfqpoint{5.333112in}{3.432992in}}{\pgfqpoint{5.342320in}{3.442200in}}%
\pgfpathcurveto{\pgfqpoint{5.351529in}{3.451408in}}{\pgfqpoint{5.356702in}{3.463900in}}{\pgfqpoint{5.356702in}{3.476922in}}%
\pgfpathcurveto{\pgfqpoint{5.356702in}{3.489945in}}{\pgfqpoint{5.351529in}{3.502436in}}{\pgfqpoint{5.342320in}{3.511644in}}%
\pgfpathcurveto{\pgfqpoint{5.333112in}{3.520853in}}{\pgfqpoint{5.320621in}{3.526027in}}{\pgfqpoint{5.307598in}{3.526027in}}%
\pgfpathcurveto{\pgfqpoint{5.294575in}{3.526027in}}{\pgfqpoint{5.282084in}{3.520853in}}{\pgfqpoint{5.272876in}{3.511644in}}%
\pgfpathcurveto{\pgfqpoint{5.263667in}{3.502436in}}{\pgfqpoint{5.258493in}{3.489945in}}{\pgfqpoint{5.258493in}{3.476922in}}%
\pgfpathcurveto{\pgfqpoint{5.258493in}{3.463900in}}{\pgfqpoint{5.263667in}{3.451408in}}{\pgfqpoint{5.272876in}{3.442200in}}%
\pgfpathcurveto{\pgfqpoint{5.282084in}{3.432992in}}{\pgfqpoint{5.294575in}{3.427818in}}{\pgfqpoint{5.307598in}{3.427818in}}%
\pgfpathlineto{\pgfqpoint{5.307598in}{3.427818in}}%
\pgfpathclose%
\pgfusepath{stroke,fill}%
\end{pgfscope}%
\begin{pgfscope}%
\pgfpathrectangle{\pgfqpoint{0.786164in}{0.768110in}}{\pgfqpoint{8.851069in}{7.081890in}}%
\pgfusepath{clip}%
\pgfsetbuttcap%
\pgfsetroundjoin%
\definecolor{currentfill}{rgb}{0.180653,0.701402,0.488189}%
\pgfsetfillcolor{currentfill}%
\pgfsetfillopacity{0.700000}%
\pgfsetlinewidth{0.501875pt}%
\definecolor{currentstroke}{rgb}{1.000000,1.000000,1.000000}%
\pgfsetstrokecolor{currentstroke}%
\pgfsetstrokeopacity{0.700000}%
\pgfsetdash{}{0pt}%
\pgfpathmoveto{\pgfqpoint{5.179732in}{3.077446in}}%
\pgfpathcurveto{\pgfqpoint{5.192755in}{3.077446in}}{\pgfqpoint{5.205246in}{3.082620in}}{\pgfqpoint{5.214454in}{3.091828in}}%
\pgfpathcurveto{\pgfqpoint{5.223662in}{3.101037in}}{\pgfqpoint{5.228836in}{3.113528in}}{\pgfqpoint{5.228836in}{3.126550in}}%
\pgfpathcurveto{\pgfqpoint{5.228836in}{3.139573in}}{\pgfqpoint{5.223662in}{3.152064in}}{\pgfqpoint{5.214454in}{3.161273in}}%
\pgfpathcurveto{\pgfqpoint{5.205246in}{3.170481in}}{\pgfqpoint{5.192755in}{3.175655in}}{\pgfqpoint{5.179732in}{3.175655in}}%
\pgfpathcurveto{\pgfqpoint{5.166709in}{3.175655in}}{\pgfqpoint{5.154218in}{3.170481in}}{\pgfqpoint{5.145010in}{3.161273in}}%
\pgfpathcurveto{\pgfqpoint{5.135801in}{3.152064in}}{\pgfqpoint{5.130627in}{3.139573in}}{\pgfqpoint{5.130627in}{3.126550in}}%
\pgfpathcurveto{\pgfqpoint{5.130627in}{3.113528in}}{\pgfqpoint{5.135801in}{3.101037in}}{\pgfqpoint{5.145010in}{3.091828in}}%
\pgfpathcurveto{\pgfqpoint{5.154218in}{3.082620in}}{\pgfqpoint{5.166709in}{3.077446in}}{\pgfqpoint{5.179732in}{3.077446in}}%
\pgfpathlineto{\pgfqpoint{5.179732in}{3.077446in}}%
\pgfpathclose%
\pgfusepath{stroke,fill}%
\end{pgfscope}%
\begin{pgfscope}%
\pgfpathrectangle{\pgfqpoint{0.786164in}{0.768110in}}{\pgfqpoint{8.851069in}{7.081890in}}%
\pgfusepath{clip}%
\pgfsetbuttcap%
\pgfsetroundjoin%
\definecolor{currentfill}{rgb}{0.208030,0.718701,0.472873}%
\pgfsetfillcolor{currentfill}%
\pgfsetfillopacity{0.700000}%
\pgfsetlinewidth{0.501875pt}%
\definecolor{currentstroke}{rgb}{1.000000,1.000000,1.000000}%
\pgfsetstrokecolor{currentstroke}%
\pgfsetstrokeopacity{0.700000}%
\pgfsetdash{}{0pt}%
\pgfpathmoveto{\pgfqpoint{4.814400in}{2.727074in}}%
\pgfpathcurveto{\pgfqpoint{4.827423in}{2.727074in}}{\pgfqpoint{4.839914in}{2.732248in}}{\pgfqpoint{4.849123in}{2.741456in}}%
\pgfpathcurveto{\pgfqpoint{4.858331in}{2.750665in}}{\pgfqpoint{4.863505in}{2.763156in}}{\pgfqpoint{4.863505in}{2.776179in}}%
\pgfpathcurveto{\pgfqpoint{4.863505in}{2.789201in}}{\pgfqpoint{4.858331in}{2.801692in}}{\pgfqpoint{4.849123in}{2.810901in}}%
\pgfpathcurveto{\pgfqpoint{4.839914in}{2.820109in}}{\pgfqpoint{4.827423in}{2.825283in}}{\pgfqpoint{4.814400in}{2.825283in}}%
\pgfpathcurveto{\pgfqpoint{4.801378in}{2.825283in}}{\pgfqpoint{4.788887in}{2.820109in}}{\pgfqpoint{4.779678in}{2.810901in}}%
\pgfpathcurveto{\pgfqpoint{4.770470in}{2.801692in}}{\pgfqpoint{4.765296in}{2.789201in}}{\pgfqpoint{4.765296in}{2.776179in}}%
\pgfpathcurveto{\pgfqpoint{4.765296in}{2.763156in}}{\pgfqpoint{4.770470in}{2.750665in}}{\pgfqpoint{4.779678in}{2.741456in}}%
\pgfpathcurveto{\pgfqpoint{4.788887in}{2.732248in}}{\pgfqpoint{4.801378in}{2.727074in}}{\pgfqpoint{4.814400in}{2.727074in}}%
\pgfpathlineto{\pgfqpoint{4.814400in}{2.727074in}}%
\pgfpathclose%
\pgfusepath{stroke,fill}%
\end{pgfscope}%
\begin{pgfscope}%
\pgfpathrectangle{\pgfqpoint{0.786164in}{0.768110in}}{\pgfqpoint{8.851069in}{7.081890in}}%
\pgfusepath{clip}%
\pgfsetbuttcap%
\pgfsetroundjoin%
\definecolor{currentfill}{rgb}{0.185783,0.704891,0.485273}%
\pgfsetfillcolor{currentfill}%
\pgfsetfillopacity{0.700000}%
\pgfsetlinewidth{0.501875pt}%
\definecolor{currentstroke}{rgb}{1.000000,1.000000,1.000000}%
\pgfsetstrokecolor{currentstroke}%
\pgfsetstrokeopacity{0.700000}%
\pgfsetdash{}{0pt}%
\pgfpathmoveto{\pgfqpoint{5.060999in}{3.077446in}}%
\pgfpathcurveto{\pgfqpoint{5.074022in}{3.077446in}}{\pgfqpoint{5.086513in}{3.082620in}}{\pgfqpoint{5.095721in}{3.091828in}}%
\pgfpathcurveto{\pgfqpoint{5.104930in}{3.101037in}}{\pgfqpoint{5.110104in}{3.113528in}}{\pgfqpoint{5.110104in}{3.126550in}}%
\pgfpathcurveto{\pgfqpoint{5.110104in}{3.139573in}}{\pgfqpoint{5.104930in}{3.152064in}}{\pgfqpoint{5.095721in}{3.161273in}}%
\pgfpathcurveto{\pgfqpoint{5.086513in}{3.170481in}}{\pgfqpoint{5.074022in}{3.175655in}}{\pgfqpoint{5.060999in}{3.175655in}}%
\pgfpathcurveto{\pgfqpoint{5.047976in}{3.175655in}}{\pgfqpoint{5.035485in}{3.170481in}}{\pgfqpoint{5.026277in}{3.161273in}}%
\pgfpathcurveto{\pgfqpoint{5.017068in}{3.152064in}}{\pgfqpoint{5.011894in}{3.139573in}}{\pgfqpoint{5.011894in}{3.126550in}}%
\pgfpathcurveto{\pgfqpoint{5.011894in}{3.113528in}}{\pgfqpoint{5.017068in}{3.101037in}}{\pgfqpoint{5.026277in}{3.091828in}}%
\pgfpathcurveto{\pgfqpoint{5.035485in}{3.082620in}}{\pgfqpoint{5.047976in}{3.077446in}}{\pgfqpoint{5.060999in}{3.077446in}}%
\pgfpathlineto{\pgfqpoint{5.060999in}{3.077446in}}%
\pgfpathclose%
\pgfusepath{stroke,fill}%
\end{pgfscope}%
\begin{pgfscope}%
\pgfpathrectangle{\pgfqpoint{0.786164in}{0.768110in}}{\pgfqpoint{8.851069in}{7.081890in}}%
\pgfusepath{clip}%
\pgfsetbuttcap%
\pgfsetroundjoin%
\definecolor{currentfill}{rgb}{0.311925,0.767822,0.415586}%
\pgfsetfillcolor{currentfill}%
\pgfsetfillopacity{0.700000}%
\pgfsetlinewidth{0.501875pt}%
\definecolor{currentstroke}{rgb}{1.000000,1.000000,1.000000}%
\pgfsetstrokecolor{currentstroke}%
\pgfsetstrokeopacity{0.700000}%
\pgfsetdash{}{0pt}%
\pgfpathmoveto{\pgfqpoint{4.832667in}{3.055548in}}%
\pgfpathcurveto{\pgfqpoint{4.845690in}{3.055548in}}{\pgfqpoint{4.858181in}{3.060722in}}{\pgfqpoint{4.867389in}{3.069930in}}%
\pgfpathcurveto{\pgfqpoint{4.876598in}{3.079138in}}{\pgfqpoint{4.881772in}{3.091630in}}{\pgfqpoint{4.881772in}{3.104652in}}%
\pgfpathcurveto{\pgfqpoint{4.881772in}{3.117675in}}{\pgfqpoint{4.876598in}{3.130166in}}{\pgfqpoint{4.867389in}{3.139374in}}%
\pgfpathcurveto{\pgfqpoint{4.858181in}{3.148583in}}{\pgfqpoint{4.845690in}{3.153757in}}{\pgfqpoint{4.832667in}{3.153757in}}%
\pgfpathcurveto{\pgfqpoint{4.819644in}{3.153757in}}{\pgfqpoint{4.807153in}{3.148583in}}{\pgfqpoint{4.797945in}{3.139374in}}%
\pgfpathcurveto{\pgfqpoint{4.788736in}{3.130166in}}{\pgfqpoint{4.783562in}{3.117675in}}{\pgfqpoint{4.783562in}{3.104652in}}%
\pgfpathcurveto{\pgfqpoint{4.783562in}{3.091630in}}{\pgfqpoint{4.788736in}{3.079138in}}{\pgfqpoint{4.797945in}{3.069930in}}%
\pgfpathcurveto{\pgfqpoint{4.807153in}{3.060722in}}{\pgfqpoint{4.819644in}{3.055548in}}{\pgfqpoint{4.832667in}{3.055548in}}%
\pgfpathlineto{\pgfqpoint{4.832667in}{3.055548in}}%
\pgfpathclose%
\pgfusepath{stroke,fill}%
\end{pgfscope}%
\begin{pgfscope}%
\pgfpathrectangle{\pgfqpoint{0.786164in}{0.768110in}}{\pgfqpoint{8.851069in}{7.081890in}}%
\pgfusepath{clip}%
\pgfsetbuttcap%
\pgfsetroundjoin%
\definecolor{currentfill}{rgb}{0.276194,0.190074,0.493001}%
\pgfsetfillcolor{currentfill}%
\pgfsetfillopacity{0.700000}%
\pgfsetlinewidth{0.501875pt}%
\definecolor{currentstroke}{rgb}{1.000000,1.000000,1.000000}%
\pgfsetstrokecolor{currentstroke}%
\pgfsetstrokeopacity{0.700000}%
\pgfsetdash{}{0pt}%
\pgfpathmoveto{\pgfqpoint{3.654473in}{2.814667in}}%
\pgfpathcurveto{\pgfqpoint{3.667496in}{2.814667in}}{\pgfqpoint{3.679987in}{2.819841in}}{\pgfqpoint{3.689195in}{2.829049in}}%
\pgfpathcurveto{\pgfqpoint{3.698403in}{2.838258in}}{\pgfqpoint{3.703577in}{2.850749in}}{\pgfqpoint{3.703577in}{2.863772in}}%
\pgfpathcurveto{\pgfqpoint{3.703577in}{2.876794in}}{\pgfqpoint{3.698403in}{2.889285in}}{\pgfqpoint{3.689195in}{2.898494in}}%
\pgfpathcurveto{\pgfqpoint{3.679987in}{2.907702in}}{\pgfqpoint{3.667496in}{2.912876in}}{\pgfqpoint{3.654473in}{2.912876in}}%
\pgfpathcurveto{\pgfqpoint{3.641450in}{2.912876in}}{\pgfqpoint{3.628959in}{2.907702in}}{\pgfqpoint{3.619751in}{2.898494in}}%
\pgfpathcurveto{\pgfqpoint{3.610542in}{2.889285in}}{\pgfqpoint{3.605368in}{2.876794in}}{\pgfqpoint{3.605368in}{2.863772in}}%
\pgfpathcurveto{\pgfqpoint{3.605368in}{2.850749in}}{\pgfqpoint{3.610542in}{2.838258in}}{\pgfqpoint{3.619751in}{2.829049in}}%
\pgfpathcurveto{\pgfqpoint{3.628959in}{2.819841in}}{\pgfqpoint{3.641450in}{2.814667in}}{\pgfqpoint{3.654473in}{2.814667in}}%
\pgfpathlineto{\pgfqpoint{3.654473in}{2.814667in}}%
\pgfpathclose%
\pgfusepath{stroke,fill}%
\end{pgfscope}%
\begin{pgfscope}%
\pgfpathrectangle{\pgfqpoint{0.786164in}{0.768110in}}{\pgfqpoint{8.851069in}{7.081890in}}%
\pgfusepath{clip}%
\pgfsetbuttcap%
\pgfsetroundjoin%
\definecolor{currentfill}{rgb}{0.276194,0.190074,0.493001}%
\pgfsetfillcolor{currentfill}%
\pgfsetfillopacity{0.700000}%
\pgfsetlinewidth{0.501875pt}%
\definecolor{currentstroke}{rgb}{1.000000,1.000000,1.000000}%
\pgfsetstrokecolor{currentstroke}%
\pgfsetstrokeopacity{0.700000}%
\pgfsetdash{}{0pt}%
\pgfpathmoveto{\pgfqpoint{3.654473in}{2.792769in}}%
\pgfpathcurveto{\pgfqpoint{3.667496in}{2.792769in}}{\pgfqpoint{3.679987in}{2.797943in}}{\pgfqpoint{3.689195in}{2.807151in}}%
\pgfpathcurveto{\pgfqpoint{3.698403in}{2.816360in}}{\pgfqpoint{3.703577in}{2.828851in}}{\pgfqpoint{3.703577in}{2.841873in}}%
\pgfpathcurveto{\pgfqpoint{3.703577in}{2.854896in}}{\pgfqpoint{3.698403in}{2.867387in}}{\pgfqpoint{3.689195in}{2.876596in}}%
\pgfpathcurveto{\pgfqpoint{3.679987in}{2.885804in}}{\pgfqpoint{3.667496in}{2.890978in}}{\pgfqpoint{3.654473in}{2.890978in}}%
\pgfpathcurveto{\pgfqpoint{3.641450in}{2.890978in}}{\pgfqpoint{3.628959in}{2.885804in}}{\pgfqpoint{3.619751in}{2.876596in}}%
\pgfpathcurveto{\pgfqpoint{3.610542in}{2.867387in}}{\pgfqpoint{3.605368in}{2.854896in}}{\pgfqpoint{3.605368in}{2.841873in}}%
\pgfpathcurveto{\pgfqpoint{3.605368in}{2.828851in}}{\pgfqpoint{3.610542in}{2.816360in}}{\pgfqpoint{3.619751in}{2.807151in}}%
\pgfpathcurveto{\pgfqpoint{3.628959in}{2.797943in}}{\pgfqpoint{3.641450in}{2.792769in}}{\pgfqpoint{3.654473in}{2.792769in}}%
\pgfpathlineto{\pgfqpoint{3.654473in}{2.792769in}}%
\pgfpathclose%
\pgfusepath{stroke,fill}%
\end{pgfscope}%
\begin{pgfscope}%
\pgfpathrectangle{\pgfqpoint{0.786164in}{0.768110in}}{\pgfqpoint{8.851069in}{7.081890in}}%
\pgfusepath{clip}%
\pgfsetbuttcap%
\pgfsetroundjoin%
\definecolor{currentfill}{rgb}{0.277134,0.185228,0.489898}%
\pgfsetfillcolor{currentfill}%
\pgfsetfillopacity{0.700000}%
\pgfsetlinewidth{0.501875pt}%
\definecolor{currentstroke}{rgb}{1.000000,1.000000,1.000000}%
\pgfsetstrokecolor{currentstroke}%
\pgfsetstrokeopacity{0.700000}%
\pgfsetdash{}{0pt}%
\pgfpathmoveto{\pgfqpoint{3.563140in}{2.748972in}}%
\pgfpathcurveto{\pgfqpoint{3.576163in}{2.748972in}}{\pgfqpoint{3.588654in}{2.754146in}}{\pgfqpoint{3.597862in}{2.763355in}}%
\pgfpathcurveto{\pgfqpoint{3.607071in}{2.772563in}}{\pgfqpoint{3.612245in}{2.785054in}}{\pgfqpoint{3.612245in}{2.798077in}}%
\pgfpathcurveto{\pgfqpoint{3.612245in}{2.811100in}}{\pgfqpoint{3.607071in}{2.823591in}}{\pgfqpoint{3.597862in}{2.832799in}}%
\pgfpathcurveto{\pgfqpoint{3.588654in}{2.842008in}}{\pgfqpoint{3.576163in}{2.847182in}}{\pgfqpoint{3.563140in}{2.847182in}}%
\pgfpathcurveto{\pgfqpoint{3.550117in}{2.847182in}}{\pgfqpoint{3.537626in}{2.842008in}}{\pgfqpoint{3.528418in}{2.832799in}}%
\pgfpathcurveto{\pgfqpoint{3.519209in}{2.823591in}}{\pgfqpoint{3.514035in}{2.811100in}}{\pgfqpoint{3.514035in}{2.798077in}}%
\pgfpathcurveto{\pgfqpoint{3.514035in}{2.785054in}}{\pgfqpoint{3.519209in}{2.772563in}}{\pgfqpoint{3.528418in}{2.763355in}}%
\pgfpathcurveto{\pgfqpoint{3.537626in}{2.754146in}}{\pgfqpoint{3.550117in}{2.748972in}}{\pgfqpoint{3.563140in}{2.748972in}}%
\pgfpathlineto{\pgfqpoint{3.563140in}{2.748972in}}%
\pgfpathclose%
\pgfusepath{stroke,fill}%
\end{pgfscope}%
\begin{pgfscope}%
\pgfpathrectangle{\pgfqpoint{0.786164in}{0.768110in}}{\pgfqpoint{8.851069in}{7.081890in}}%
\pgfusepath{clip}%
\pgfsetbuttcap%
\pgfsetroundjoin%
\definecolor{currentfill}{rgb}{0.275191,0.194905,0.496005}%
\pgfsetfillcolor{currentfill}%
\pgfsetfillopacity{0.700000}%
\pgfsetlinewidth{0.501875pt}%
\definecolor{currentstroke}{rgb}{1.000000,1.000000,1.000000}%
\pgfsetstrokecolor{currentstroke}%
\pgfsetstrokeopacity{0.700000}%
\pgfsetdash{}{0pt}%
\pgfpathmoveto{\pgfqpoint{3.490074in}{2.727074in}}%
\pgfpathcurveto{\pgfqpoint{3.503096in}{2.727074in}}{\pgfqpoint{3.515587in}{2.732248in}}{\pgfqpoint{3.524796in}{2.741456in}}%
\pgfpathcurveto{\pgfqpoint{3.534004in}{2.750665in}}{\pgfqpoint{3.539178in}{2.763156in}}{\pgfqpoint{3.539178in}{2.776179in}}%
\pgfpathcurveto{\pgfqpoint{3.539178in}{2.789201in}}{\pgfqpoint{3.534004in}{2.801692in}}{\pgfqpoint{3.524796in}{2.810901in}}%
\pgfpathcurveto{\pgfqpoint{3.515587in}{2.820109in}}{\pgfqpoint{3.503096in}{2.825283in}}{\pgfqpoint{3.490074in}{2.825283in}}%
\pgfpathcurveto{\pgfqpoint{3.477051in}{2.825283in}}{\pgfqpoint{3.464560in}{2.820109in}}{\pgfqpoint{3.455351in}{2.810901in}}%
\pgfpathcurveto{\pgfqpoint{3.446143in}{2.801692in}}{\pgfqpoint{3.440969in}{2.789201in}}{\pgfqpoint{3.440969in}{2.776179in}}%
\pgfpathcurveto{\pgfqpoint{3.440969in}{2.763156in}}{\pgfqpoint{3.446143in}{2.750665in}}{\pgfqpoint{3.455351in}{2.741456in}}%
\pgfpathcurveto{\pgfqpoint{3.464560in}{2.732248in}}{\pgfqpoint{3.477051in}{2.727074in}}{\pgfqpoint{3.490074in}{2.727074in}}%
\pgfpathlineto{\pgfqpoint{3.490074in}{2.727074in}}%
\pgfpathclose%
\pgfusepath{stroke,fill}%
\end{pgfscope}%
\begin{pgfscope}%
\pgfpathrectangle{\pgfqpoint{0.786164in}{0.768110in}}{\pgfqpoint{8.851069in}{7.081890in}}%
\pgfusepath{clip}%
\pgfsetbuttcap%
\pgfsetroundjoin%
\definecolor{currentfill}{rgb}{0.266580,0.228262,0.514349}%
\pgfsetfillcolor{currentfill}%
\pgfsetfillopacity{0.700000}%
\pgfsetlinewidth{0.501875pt}%
\definecolor{currentstroke}{rgb}{1.000000,1.000000,1.000000}%
\pgfsetstrokecolor{currentstroke}%
\pgfsetstrokeopacity{0.700000}%
\pgfsetdash{}{0pt}%
\pgfpathmoveto{\pgfqpoint{3.508340in}{2.705176in}}%
\pgfpathcurveto{\pgfqpoint{3.521363in}{2.705176in}}{\pgfqpoint{3.533854in}{2.710350in}}{\pgfqpoint{3.543062in}{2.719558in}}%
\pgfpathcurveto{\pgfqpoint{3.552271in}{2.728767in}}{\pgfqpoint{3.557445in}{2.741258in}}{\pgfqpoint{3.557445in}{2.754280in}}%
\pgfpathcurveto{\pgfqpoint{3.557445in}{2.767303in}}{\pgfqpoint{3.552271in}{2.779794in}}{\pgfqpoint{3.543062in}{2.789003in}}%
\pgfpathcurveto{\pgfqpoint{3.533854in}{2.798211in}}{\pgfqpoint{3.521363in}{2.803385in}}{\pgfqpoint{3.508340in}{2.803385in}}%
\pgfpathcurveto{\pgfqpoint{3.495318in}{2.803385in}}{\pgfqpoint{3.482826in}{2.798211in}}{\pgfqpoint{3.473618in}{2.789003in}}%
\pgfpathcurveto{\pgfqpoint{3.464410in}{2.779794in}}{\pgfqpoint{3.459236in}{2.767303in}}{\pgfqpoint{3.459236in}{2.754280in}}%
\pgfpathcurveto{\pgfqpoint{3.459236in}{2.741258in}}{\pgfqpoint{3.464410in}{2.728767in}}{\pgfqpoint{3.473618in}{2.719558in}}%
\pgfpathcurveto{\pgfqpoint{3.482826in}{2.710350in}}{\pgfqpoint{3.495318in}{2.705176in}}{\pgfqpoint{3.508340in}{2.705176in}}%
\pgfpathlineto{\pgfqpoint{3.508340in}{2.705176in}}%
\pgfpathclose%
\pgfusepath{stroke,fill}%
\end{pgfscope}%
\begin{pgfscope}%
\pgfpathrectangle{\pgfqpoint{0.786164in}{0.768110in}}{\pgfqpoint{8.851069in}{7.081890in}}%
\pgfusepath{clip}%
\pgfsetbuttcap%
\pgfsetroundjoin%
\definecolor{currentfill}{rgb}{0.260571,0.246922,0.522828}%
\pgfsetfillcolor{currentfill}%
\pgfsetfillopacity{0.700000}%
\pgfsetlinewidth{0.501875pt}%
\definecolor{currentstroke}{rgb}{1.000000,1.000000,1.000000}%
\pgfsetstrokecolor{currentstroke}%
\pgfsetstrokeopacity{0.700000}%
\pgfsetdash{}{0pt}%
\pgfpathmoveto{\pgfqpoint{3.362208in}{2.661379in}}%
\pgfpathcurveto{\pgfqpoint{3.375230in}{2.661379in}}{\pgfqpoint{3.387721in}{2.666553in}}{\pgfqpoint{3.396930in}{2.675762in}}%
\pgfpathcurveto{\pgfqpoint{3.406138in}{2.684970in}}{\pgfqpoint{3.411312in}{2.697461in}}{\pgfqpoint{3.411312in}{2.710484in}}%
\pgfpathcurveto{\pgfqpoint{3.411312in}{2.723507in}}{\pgfqpoint{3.406138in}{2.735998in}}{\pgfqpoint{3.396930in}{2.745206in}}%
\pgfpathcurveto{\pgfqpoint{3.387721in}{2.754415in}}{\pgfqpoint{3.375230in}{2.759589in}}{\pgfqpoint{3.362208in}{2.759589in}}%
\pgfpathcurveto{\pgfqpoint{3.349185in}{2.759589in}}{\pgfqpoint{3.336694in}{2.754415in}}{\pgfqpoint{3.327485in}{2.745206in}}%
\pgfpathcurveto{\pgfqpoint{3.318277in}{2.735998in}}{\pgfqpoint{3.313103in}{2.723507in}}{\pgfqpoint{3.313103in}{2.710484in}}%
\pgfpathcurveto{\pgfqpoint{3.313103in}{2.697461in}}{\pgfqpoint{3.318277in}{2.684970in}}{\pgfqpoint{3.327485in}{2.675762in}}%
\pgfpathcurveto{\pgfqpoint{3.336694in}{2.666553in}}{\pgfqpoint{3.349185in}{2.661379in}}{\pgfqpoint{3.362208in}{2.661379in}}%
\pgfpathlineto{\pgfqpoint{3.362208in}{2.661379in}}%
\pgfpathclose%
\pgfusepath{stroke,fill}%
\end{pgfscope}%
\begin{pgfscope}%
\pgfpathrectangle{\pgfqpoint{0.786164in}{0.768110in}}{\pgfqpoint{8.851069in}{7.081890in}}%
\pgfusepath{clip}%
\pgfsetbuttcap%
\pgfsetroundjoin%
\definecolor{currentfill}{rgb}{0.258965,0.251537,0.524736}%
\pgfsetfillcolor{currentfill}%
\pgfsetfillopacity{0.700000}%
\pgfsetlinewidth{0.501875pt}%
\definecolor{currentstroke}{rgb}{1.000000,1.000000,1.000000}%
\pgfsetstrokecolor{currentstroke}%
\pgfsetstrokeopacity{0.700000}%
\pgfsetdash{}{0pt}%
\pgfpathmoveto{\pgfqpoint{3.462674in}{2.661379in}}%
\pgfpathcurveto{\pgfqpoint{3.475696in}{2.661379in}}{\pgfqpoint{3.488188in}{2.666553in}}{\pgfqpoint{3.497396in}{2.675762in}}%
\pgfpathcurveto{\pgfqpoint{3.506604in}{2.684970in}}{\pgfqpoint{3.511778in}{2.697461in}}{\pgfqpoint{3.511778in}{2.710484in}}%
\pgfpathcurveto{\pgfqpoint{3.511778in}{2.723507in}}{\pgfqpoint{3.506604in}{2.735998in}}{\pgfqpoint{3.497396in}{2.745206in}}%
\pgfpathcurveto{\pgfqpoint{3.488188in}{2.754415in}}{\pgfqpoint{3.475696in}{2.759589in}}{\pgfqpoint{3.462674in}{2.759589in}}%
\pgfpathcurveto{\pgfqpoint{3.449651in}{2.759589in}}{\pgfqpoint{3.437160in}{2.754415in}}{\pgfqpoint{3.427952in}{2.745206in}}%
\pgfpathcurveto{\pgfqpoint{3.418743in}{2.735998in}}{\pgfqpoint{3.413569in}{2.723507in}}{\pgfqpoint{3.413569in}{2.710484in}}%
\pgfpathcurveto{\pgfqpoint{3.413569in}{2.697461in}}{\pgfqpoint{3.418743in}{2.684970in}}{\pgfqpoint{3.427952in}{2.675762in}}%
\pgfpathcurveto{\pgfqpoint{3.437160in}{2.666553in}}{\pgfqpoint{3.449651in}{2.661379in}}{\pgfqpoint{3.462674in}{2.661379in}}%
\pgfpathlineto{\pgfqpoint{3.462674in}{2.661379in}}%
\pgfpathclose%
\pgfusepath{stroke,fill}%
\end{pgfscope}%
\begin{pgfscope}%
\pgfpathrectangle{\pgfqpoint{0.786164in}{0.768110in}}{\pgfqpoint{8.851069in}{7.081890in}}%
\pgfusepath{clip}%
\pgfsetbuttcap%
\pgfsetroundjoin%
\definecolor{currentfill}{rgb}{0.269308,0.218818,0.509577}%
\pgfsetfillcolor{currentfill}%
\pgfsetfillopacity{0.700000}%
\pgfsetlinewidth{0.501875pt}%
\definecolor{currentstroke}{rgb}{1.000000,1.000000,1.000000}%
\pgfsetstrokecolor{currentstroke}%
\pgfsetstrokeopacity{0.700000}%
\pgfsetdash{}{0pt}%
\pgfpathmoveto{\pgfqpoint{3.362208in}{2.573786in}}%
\pgfpathcurveto{\pgfqpoint{3.375230in}{2.573786in}}{\pgfqpoint{3.387721in}{2.578960in}}{\pgfqpoint{3.396930in}{2.588169in}}%
\pgfpathcurveto{\pgfqpoint{3.406138in}{2.597377in}}{\pgfqpoint{3.411312in}{2.609868in}}{\pgfqpoint{3.411312in}{2.622891in}}%
\pgfpathcurveto{\pgfqpoint{3.411312in}{2.635914in}}{\pgfqpoint{3.406138in}{2.648405in}}{\pgfqpoint{3.396930in}{2.657613in}}%
\pgfpathcurveto{\pgfqpoint{3.387721in}{2.666822in}}{\pgfqpoint{3.375230in}{2.671996in}}{\pgfqpoint{3.362208in}{2.671996in}}%
\pgfpathcurveto{\pgfqpoint{3.349185in}{2.671996in}}{\pgfqpoint{3.336694in}{2.666822in}}{\pgfqpoint{3.327485in}{2.657613in}}%
\pgfpathcurveto{\pgfqpoint{3.318277in}{2.648405in}}{\pgfqpoint{3.313103in}{2.635914in}}{\pgfqpoint{3.313103in}{2.622891in}}%
\pgfpathcurveto{\pgfqpoint{3.313103in}{2.609868in}}{\pgfqpoint{3.318277in}{2.597377in}}{\pgfqpoint{3.327485in}{2.588169in}}%
\pgfpathcurveto{\pgfqpoint{3.336694in}{2.578960in}}{\pgfqpoint{3.349185in}{2.573786in}}{\pgfqpoint{3.362208in}{2.573786in}}%
\pgfpathlineto{\pgfqpoint{3.362208in}{2.573786in}}%
\pgfpathclose%
\pgfusepath{stroke,fill}%
\end{pgfscope}%
\begin{pgfscope}%
\pgfpathrectangle{\pgfqpoint{0.786164in}{0.768110in}}{\pgfqpoint{8.851069in}{7.081890in}}%
\pgfusepath{clip}%
\pgfsetbuttcap%
\pgfsetroundjoin%
\definecolor{currentfill}{rgb}{0.253935,0.265254,0.529983}%
\pgfsetfillcolor{currentfill}%
\pgfsetfillopacity{0.700000}%
\pgfsetlinewidth{0.501875pt}%
\definecolor{currentstroke}{rgb}{1.000000,1.000000,1.000000}%
\pgfsetstrokecolor{currentstroke}%
\pgfsetstrokeopacity{0.700000}%
\pgfsetdash{}{0pt}%
\pgfpathmoveto{\pgfqpoint{3.353074in}{2.551888in}}%
\pgfpathcurveto{\pgfqpoint{3.366097in}{2.551888in}}{\pgfqpoint{3.378588in}{2.557062in}}{\pgfqpoint{3.387797in}{2.566271in}}%
\pgfpathcurveto{\pgfqpoint{3.397005in}{2.575479in}}{\pgfqpoint{3.402179in}{2.587970in}}{\pgfqpoint{3.402179in}{2.600993in}}%
\pgfpathcurveto{\pgfqpoint{3.402179in}{2.614015in}}{\pgfqpoint{3.397005in}{2.626507in}}{\pgfqpoint{3.387797in}{2.635715in}}%
\pgfpathcurveto{\pgfqpoint{3.378588in}{2.644923in}}{\pgfqpoint{3.366097in}{2.650097in}}{\pgfqpoint{3.353074in}{2.650097in}}%
\pgfpathcurveto{\pgfqpoint{3.340052in}{2.650097in}}{\pgfqpoint{3.327561in}{2.644923in}}{\pgfqpoint{3.318352in}{2.635715in}}%
\pgfpathcurveto{\pgfqpoint{3.309144in}{2.626507in}}{\pgfqpoint{3.303970in}{2.614015in}}{\pgfqpoint{3.303970in}{2.600993in}}%
\pgfpathcurveto{\pgfqpoint{3.303970in}{2.587970in}}{\pgfqpoint{3.309144in}{2.575479in}}{\pgfqpoint{3.318352in}{2.566271in}}%
\pgfpathcurveto{\pgfqpoint{3.327561in}{2.557062in}}{\pgfqpoint{3.340052in}{2.551888in}}{\pgfqpoint{3.353074in}{2.551888in}}%
\pgfpathlineto{\pgfqpoint{3.353074in}{2.551888in}}%
\pgfpathclose%
\pgfusepath{stroke,fill}%
\end{pgfscope}%
\begin{pgfscope}%
\pgfpathrectangle{\pgfqpoint{0.786164in}{0.768110in}}{\pgfqpoint{8.851069in}{7.081890in}}%
\pgfusepath{clip}%
\pgfsetbuttcap%
\pgfsetroundjoin%
\definecolor{currentfill}{rgb}{0.250425,0.274290,0.533103}%
\pgfsetfillcolor{currentfill}%
\pgfsetfillopacity{0.700000}%
\pgfsetlinewidth{0.501875pt}%
\definecolor{currentstroke}{rgb}{1.000000,1.000000,1.000000}%
\pgfsetstrokecolor{currentstroke}%
\pgfsetstrokeopacity{0.700000}%
\pgfsetdash{}{0pt}%
\pgfpathmoveto{\pgfqpoint{3.353074in}{2.529990in}}%
\pgfpathcurveto{\pgfqpoint{3.366097in}{2.529990in}}{\pgfqpoint{3.378588in}{2.535164in}}{\pgfqpoint{3.387797in}{2.544372in}}%
\pgfpathcurveto{\pgfqpoint{3.397005in}{2.553581in}}{\pgfqpoint{3.402179in}{2.566072in}}{\pgfqpoint{3.402179in}{2.579095in}}%
\pgfpathcurveto{\pgfqpoint{3.402179in}{2.592117in}}{\pgfqpoint{3.397005in}{2.604608in}}{\pgfqpoint{3.387797in}{2.613817in}}%
\pgfpathcurveto{\pgfqpoint{3.378588in}{2.623025in}}{\pgfqpoint{3.366097in}{2.628199in}}{\pgfqpoint{3.353074in}{2.628199in}}%
\pgfpathcurveto{\pgfqpoint{3.340052in}{2.628199in}}{\pgfqpoint{3.327561in}{2.623025in}}{\pgfqpoint{3.318352in}{2.613817in}}%
\pgfpathcurveto{\pgfqpoint{3.309144in}{2.604608in}}{\pgfqpoint{3.303970in}{2.592117in}}{\pgfqpoint{3.303970in}{2.579095in}}%
\pgfpathcurveto{\pgfqpoint{3.303970in}{2.566072in}}{\pgfqpoint{3.309144in}{2.553581in}}{\pgfqpoint{3.318352in}{2.544372in}}%
\pgfpathcurveto{\pgfqpoint{3.327561in}{2.535164in}}{\pgfqpoint{3.340052in}{2.529990in}}{\pgfqpoint{3.353074in}{2.529990in}}%
\pgfpathlineto{\pgfqpoint{3.353074in}{2.529990in}}%
\pgfpathclose%
\pgfusepath{stroke,fill}%
\end{pgfscope}%
\begin{pgfscope}%
\pgfpathrectangle{\pgfqpoint{0.786164in}{0.768110in}}{\pgfqpoint{8.851069in}{7.081890in}}%
\pgfusepath{clip}%
\pgfsetbuttcap%
\pgfsetroundjoin%
\definecolor{currentfill}{rgb}{0.246811,0.283237,0.535941}%
\pgfsetfillcolor{currentfill}%
\pgfsetfillopacity{0.700000}%
\pgfsetlinewidth{0.501875pt}%
\definecolor{currentstroke}{rgb}{1.000000,1.000000,1.000000}%
\pgfsetstrokecolor{currentstroke}%
\pgfsetstrokeopacity{0.700000}%
\pgfsetdash{}{0pt}%
\pgfpathmoveto{\pgfqpoint{3.179542in}{2.420499in}}%
\pgfpathcurveto{\pgfqpoint{3.192565in}{2.420499in}}{\pgfqpoint{3.205056in}{2.425673in}}{\pgfqpoint{3.214264in}{2.434881in}}%
\pgfpathcurveto{\pgfqpoint{3.223473in}{2.444090in}}{\pgfqpoint{3.228647in}{2.456581in}}{\pgfqpoint{3.228647in}{2.469603in}}%
\pgfpathcurveto{\pgfqpoint{3.228647in}{2.482626in}}{\pgfqpoint{3.223473in}{2.495117in}}{\pgfqpoint{3.214264in}{2.504326in}}%
\pgfpathcurveto{\pgfqpoint{3.205056in}{2.513534in}}{\pgfqpoint{3.192565in}{2.518708in}}{\pgfqpoint{3.179542in}{2.518708in}}%
\pgfpathcurveto{\pgfqpoint{3.166519in}{2.518708in}}{\pgfqpoint{3.154028in}{2.513534in}}{\pgfqpoint{3.144820in}{2.504326in}}%
\pgfpathcurveto{\pgfqpoint{3.135611in}{2.495117in}}{\pgfqpoint{3.130437in}{2.482626in}}{\pgfqpoint{3.130437in}{2.469603in}}%
\pgfpathcurveto{\pgfqpoint{3.130437in}{2.456581in}}{\pgfqpoint{3.135611in}{2.444090in}}{\pgfqpoint{3.144820in}{2.434881in}}%
\pgfpathcurveto{\pgfqpoint{3.154028in}{2.425673in}}{\pgfqpoint{3.166519in}{2.420499in}}{\pgfqpoint{3.179542in}{2.420499in}}%
\pgfpathlineto{\pgfqpoint{3.179542in}{2.420499in}}%
\pgfpathclose%
\pgfusepath{stroke,fill}%
\end{pgfscope}%
\begin{pgfscope}%
\pgfpathrectangle{\pgfqpoint{0.786164in}{0.768110in}}{\pgfqpoint{8.851069in}{7.081890in}}%
\pgfusepath{clip}%
\pgfsetbuttcap%
\pgfsetroundjoin%
\definecolor{currentfill}{rgb}{0.243113,0.292092,0.538516}%
\pgfsetfillcolor{currentfill}%
\pgfsetfillopacity{0.700000}%
\pgfsetlinewidth{0.501875pt}%
\definecolor{currentstroke}{rgb}{1.000000,1.000000,1.000000}%
\pgfsetstrokecolor{currentstroke}%
\pgfsetstrokeopacity{0.700000}%
\pgfsetdash{}{0pt}%
\pgfpathmoveto{\pgfqpoint{3.225208in}{2.442397in}}%
\pgfpathcurveto{\pgfqpoint{3.238231in}{2.442397in}}{\pgfqpoint{3.250722in}{2.447571in}}{\pgfqpoint{3.259931in}{2.456779in}}%
\pgfpathcurveto{\pgfqpoint{3.269139in}{2.465988in}}{\pgfqpoint{3.274313in}{2.478479in}}{\pgfqpoint{3.274313in}{2.491502in}}%
\pgfpathcurveto{\pgfqpoint{3.274313in}{2.504524in}}{\pgfqpoint{3.269139in}{2.517015in}}{\pgfqpoint{3.259931in}{2.526224in}}%
\pgfpathcurveto{\pgfqpoint{3.250722in}{2.535432in}}{\pgfqpoint{3.238231in}{2.540606in}}{\pgfqpoint{3.225208in}{2.540606in}}%
\pgfpathcurveto{\pgfqpoint{3.212186in}{2.540606in}}{\pgfqpoint{3.199695in}{2.535432in}}{\pgfqpoint{3.190486in}{2.526224in}}%
\pgfpathcurveto{\pgfqpoint{3.181278in}{2.517015in}}{\pgfqpoint{3.176104in}{2.504524in}}{\pgfqpoint{3.176104in}{2.491502in}}%
\pgfpathcurveto{\pgfqpoint{3.176104in}{2.478479in}}{\pgfqpoint{3.181278in}{2.465988in}}{\pgfqpoint{3.190486in}{2.456779in}}%
\pgfpathcurveto{\pgfqpoint{3.199695in}{2.447571in}}{\pgfqpoint{3.212186in}{2.442397in}}{\pgfqpoint{3.225208in}{2.442397in}}%
\pgfpathlineto{\pgfqpoint{3.225208in}{2.442397in}}%
\pgfpathclose%
\pgfusepath{stroke,fill}%
\end{pgfscope}%
\begin{pgfscope}%
\pgfpathrectangle{\pgfqpoint{0.786164in}{0.768110in}}{\pgfqpoint{8.851069in}{7.081890in}}%
\pgfusepath{clip}%
\pgfsetbuttcap%
\pgfsetroundjoin%
\definecolor{currentfill}{rgb}{0.239346,0.300855,0.540844}%
\pgfsetfillcolor{currentfill}%
\pgfsetfillopacity{0.700000}%
\pgfsetlinewidth{0.501875pt}%
\definecolor{currentstroke}{rgb}{1.000000,1.000000,1.000000}%
\pgfsetstrokecolor{currentstroke}%
\pgfsetstrokeopacity{0.700000}%
\pgfsetdash{}{0pt}%
\pgfpathmoveto{\pgfqpoint{3.161275in}{2.464295in}}%
\pgfpathcurveto{\pgfqpoint{3.174298in}{2.464295in}}{\pgfqpoint{3.186789in}{2.469469in}}{\pgfqpoint{3.195998in}{2.478678in}}%
\pgfpathcurveto{\pgfqpoint{3.205206in}{2.487886in}}{\pgfqpoint{3.210380in}{2.500377in}}{\pgfqpoint{3.210380in}{2.513400in}}%
\pgfpathcurveto{\pgfqpoint{3.210380in}{2.526423in}}{\pgfqpoint{3.205206in}{2.538914in}}{\pgfqpoint{3.195998in}{2.548122in}}%
\pgfpathcurveto{\pgfqpoint{3.186789in}{2.557331in}}{\pgfqpoint{3.174298in}{2.562504in}}{\pgfqpoint{3.161275in}{2.562504in}}%
\pgfpathcurveto{\pgfqpoint{3.148253in}{2.562504in}}{\pgfqpoint{3.135762in}{2.557331in}}{\pgfqpoint{3.126553in}{2.548122in}}%
\pgfpathcurveto{\pgfqpoint{3.117345in}{2.538914in}}{\pgfqpoint{3.112171in}{2.526423in}}{\pgfqpoint{3.112171in}{2.513400in}}%
\pgfpathcurveto{\pgfqpoint{3.112171in}{2.500377in}}{\pgfqpoint{3.117345in}{2.487886in}}{\pgfqpoint{3.126553in}{2.478678in}}%
\pgfpathcurveto{\pgfqpoint{3.135762in}{2.469469in}}{\pgfqpoint{3.148253in}{2.464295in}}{\pgfqpoint{3.161275in}{2.464295in}}%
\pgfpathlineto{\pgfqpoint{3.161275in}{2.464295in}}%
\pgfpathclose%
\pgfusepath{stroke,fill}%
\end{pgfscope}%
\begin{pgfscope}%
\pgfpathrectangle{\pgfqpoint{0.786164in}{0.768110in}}{\pgfqpoint{8.851069in}{7.081890in}}%
\pgfusepath{clip}%
\pgfsetbuttcap%
\pgfsetroundjoin%
\definecolor{currentfill}{rgb}{0.235526,0.309527,0.542944}%
\pgfsetfillcolor{currentfill}%
\pgfsetfillopacity{0.700000}%
\pgfsetlinewidth{0.501875pt}%
\definecolor{currentstroke}{rgb}{1.000000,1.000000,1.000000}%
\pgfsetstrokecolor{currentstroke}%
\pgfsetstrokeopacity{0.700000}%
\pgfsetdash{}{0pt}%
\pgfpathmoveto{\pgfqpoint{3.143009in}{2.508092in}}%
\pgfpathcurveto{\pgfqpoint{3.156031in}{2.508092in}}{\pgfqpoint{3.168523in}{2.513266in}}{\pgfqpoint{3.177731in}{2.522474in}}%
\pgfpathcurveto{\pgfqpoint{3.186939in}{2.531683in}}{\pgfqpoint{3.192113in}{2.544174in}}{\pgfqpoint{3.192113in}{2.557196in}}%
\pgfpathcurveto{\pgfqpoint{3.192113in}{2.570219in}}{\pgfqpoint{3.186939in}{2.582710in}}{\pgfqpoint{3.177731in}{2.591919in}}%
\pgfpathcurveto{\pgfqpoint{3.168523in}{2.601127in}}{\pgfqpoint{3.156031in}{2.606301in}}{\pgfqpoint{3.143009in}{2.606301in}}%
\pgfpathcurveto{\pgfqpoint{3.129986in}{2.606301in}}{\pgfqpoint{3.117495in}{2.601127in}}{\pgfqpoint{3.108286in}{2.591919in}}%
\pgfpathcurveto{\pgfqpoint{3.099078in}{2.582710in}}{\pgfqpoint{3.093904in}{2.570219in}}{\pgfqpoint{3.093904in}{2.557196in}}%
\pgfpathcurveto{\pgfqpoint{3.093904in}{2.544174in}}{\pgfqpoint{3.099078in}{2.531683in}}{\pgfqpoint{3.108286in}{2.522474in}}%
\pgfpathcurveto{\pgfqpoint{3.117495in}{2.513266in}}{\pgfqpoint{3.129986in}{2.508092in}}{\pgfqpoint{3.143009in}{2.508092in}}%
\pgfpathlineto{\pgfqpoint{3.143009in}{2.508092in}}%
\pgfpathclose%
\pgfusepath{stroke,fill}%
\end{pgfscope}%
\begin{pgfscope}%
\pgfpathrectangle{\pgfqpoint{0.786164in}{0.768110in}}{\pgfqpoint{8.851069in}{7.081890in}}%
\pgfusepath{clip}%
\pgfsetbuttcap%
\pgfsetroundjoin%
\definecolor{currentfill}{rgb}{0.231674,0.318106,0.544834}%
\pgfsetfillcolor{currentfill}%
\pgfsetfillopacity{0.700000}%
\pgfsetlinewidth{0.501875pt}%
\definecolor{currentstroke}{rgb}{1.000000,1.000000,1.000000}%
\pgfsetstrokecolor{currentstroke}%
\pgfsetstrokeopacity{0.700000}%
\pgfsetdash{}{0pt}%
\pgfpathmoveto{\pgfqpoint{3.124742in}{2.420499in}}%
\pgfpathcurveto{\pgfqpoint{3.137765in}{2.420499in}}{\pgfqpoint{3.150256in}{2.425673in}}{\pgfqpoint{3.159464in}{2.434881in}}%
\pgfpathcurveto{\pgfqpoint{3.168673in}{2.444090in}}{\pgfqpoint{3.173847in}{2.456581in}}{\pgfqpoint{3.173847in}{2.469603in}}%
\pgfpathcurveto{\pgfqpoint{3.173847in}{2.482626in}}{\pgfqpoint{3.168673in}{2.495117in}}{\pgfqpoint{3.159464in}{2.504326in}}%
\pgfpathcurveto{\pgfqpoint{3.150256in}{2.513534in}}{\pgfqpoint{3.137765in}{2.518708in}}{\pgfqpoint{3.124742in}{2.518708in}}%
\pgfpathcurveto{\pgfqpoint{3.111719in}{2.518708in}}{\pgfqpoint{3.099228in}{2.513534in}}{\pgfqpoint{3.090020in}{2.504326in}}%
\pgfpathcurveto{\pgfqpoint{3.080811in}{2.495117in}}{\pgfqpoint{3.075638in}{2.482626in}}{\pgfqpoint{3.075638in}{2.469603in}}%
\pgfpathcurveto{\pgfqpoint{3.075638in}{2.456581in}}{\pgfqpoint{3.080811in}{2.444090in}}{\pgfqpoint{3.090020in}{2.434881in}}%
\pgfpathcurveto{\pgfqpoint{3.099228in}{2.425673in}}{\pgfqpoint{3.111719in}{2.420499in}}{\pgfqpoint{3.124742in}{2.420499in}}%
\pgfpathlineto{\pgfqpoint{3.124742in}{2.420499in}}%
\pgfpathclose%
\pgfusepath{stroke,fill}%
\end{pgfscope}%
\begin{pgfscope}%
\pgfpathrectangle{\pgfqpoint{0.786164in}{0.768110in}}{\pgfqpoint{8.851069in}{7.081890in}}%
\pgfusepath{clip}%
\pgfsetbuttcap%
\pgfsetroundjoin%
\definecolor{currentfill}{rgb}{0.225863,0.330805,0.547314}%
\pgfsetfillcolor{currentfill}%
\pgfsetfillopacity{0.700000}%
\pgfsetlinewidth{0.501875pt}%
\definecolor{currentstroke}{rgb}{1.000000,1.000000,1.000000}%
\pgfsetstrokecolor{currentstroke}%
\pgfsetstrokeopacity{0.700000}%
\pgfsetdash{}{0pt}%
\pgfpathmoveto{\pgfqpoint{3.060809in}{2.420499in}}%
\pgfpathcurveto{\pgfqpoint{3.073832in}{2.420499in}}{\pgfqpoint{3.086323in}{2.425673in}}{\pgfqpoint{3.095531in}{2.434881in}}%
\pgfpathcurveto{\pgfqpoint{3.104740in}{2.444090in}}{\pgfqpoint{3.109914in}{2.456581in}}{\pgfqpoint{3.109914in}{2.469603in}}%
\pgfpathcurveto{\pgfqpoint{3.109914in}{2.482626in}}{\pgfqpoint{3.104740in}{2.495117in}}{\pgfqpoint{3.095531in}{2.504326in}}%
\pgfpathcurveto{\pgfqpoint{3.086323in}{2.513534in}}{\pgfqpoint{3.073832in}{2.518708in}}{\pgfqpoint{3.060809in}{2.518708in}}%
\pgfpathcurveto{\pgfqpoint{3.047786in}{2.518708in}}{\pgfqpoint{3.035295in}{2.513534in}}{\pgfqpoint{3.026087in}{2.504326in}}%
\pgfpathcurveto{\pgfqpoint{3.016878in}{2.495117in}}{\pgfqpoint{3.011704in}{2.482626in}}{\pgfqpoint{3.011704in}{2.469603in}}%
\pgfpathcurveto{\pgfqpoint{3.011704in}{2.456581in}}{\pgfqpoint{3.016878in}{2.444090in}}{\pgfqpoint{3.026087in}{2.434881in}}%
\pgfpathcurveto{\pgfqpoint{3.035295in}{2.425673in}}{\pgfqpoint{3.047786in}{2.420499in}}{\pgfqpoint{3.060809in}{2.420499in}}%
\pgfpathlineto{\pgfqpoint{3.060809in}{2.420499in}}%
\pgfpathclose%
\pgfusepath{stroke,fill}%
\end{pgfscope}%
\begin{pgfscope}%
\pgfpathrectangle{\pgfqpoint{0.786164in}{0.768110in}}{\pgfqpoint{8.851069in}{7.081890in}}%
\pgfusepath{clip}%
\pgfsetbuttcap%
\pgfsetroundjoin%
\definecolor{currentfill}{rgb}{0.212395,0.359683,0.551710}%
\pgfsetfillcolor{currentfill}%
\pgfsetfillopacity{0.700000}%
\pgfsetlinewidth{0.501875pt}%
\definecolor{currentstroke}{rgb}{1.000000,1.000000,1.000000}%
\pgfsetstrokecolor{currentstroke}%
\pgfsetstrokeopacity{0.700000}%
\pgfsetdash{}{0pt}%
\pgfpathmoveto{\pgfqpoint{2.686344in}{2.245313in}}%
\pgfpathcurveto{\pgfqpoint{2.699367in}{2.245313in}}{\pgfqpoint{2.711858in}{2.250487in}}{\pgfqpoint{2.721067in}{2.259695in}}%
\pgfpathcurveto{\pgfqpoint{2.730275in}{2.268904in}}{\pgfqpoint{2.735449in}{2.281395in}}{\pgfqpoint{2.735449in}{2.294417in}}%
\pgfpathcurveto{\pgfqpoint{2.735449in}{2.307440in}}{\pgfqpoint{2.730275in}{2.319931in}}{\pgfqpoint{2.721067in}{2.329140in}}%
\pgfpathcurveto{\pgfqpoint{2.711858in}{2.338348in}}{\pgfqpoint{2.699367in}{2.343522in}}{\pgfqpoint{2.686344in}{2.343522in}}%
\pgfpathcurveto{\pgfqpoint{2.673322in}{2.343522in}}{\pgfqpoint{2.660831in}{2.338348in}}{\pgfqpoint{2.651622in}{2.329140in}}%
\pgfpathcurveto{\pgfqpoint{2.642414in}{2.319931in}}{\pgfqpoint{2.637240in}{2.307440in}}{\pgfqpoint{2.637240in}{2.294417in}}%
\pgfpathcurveto{\pgfqpoint{2.637240in}{2.281395in}}{\pgfqpoint{2.642414in}{2.268904in}}{\pgfqpoint{2.651622in}{2.259695in}}%
\pgfpathcurveto{\pgfqpoint{2.660831in}{2.250487in}}{\pgfqpoint{2.673322in}{2.245313in}}{\pgfqpoint{2.686344in}{2.245313in}}%
\pgfpathlineto{\pgfqpoint{2.686344in}{2.245313in}}%
\pgfpathclose%
\pgfusepath{stroke,fill}%
\end{pgfscope}%
\begin{pgfscope}%
\pgfpathrectangle{\pgfqpoint{0.786164in}{0.768110in}}{\pgfqpoint{8.851069in}{7.081890in}}%
\pgfusepath{clip}%
\pgfsetbuttcap%
\pgfsetroundjoin%
\definecolor{currentfill}{rgb}{0.210503,0.363727,0.552206}%
\pgfsetfillcolor{currentfill}%
\pgfsetfillopacity{0.700000}%
\pgfsetlinewidth{0.501875pt}%
\definecolor{currentstroke}{rgb}{1.000000,1.000000,1.000000}%
\pgfsetstrokecolor{currentstroke}%
\pgfsetstrokeopacity{0.700000}%
\pgfsetdash{}{0pt}%
\pgfpathmoveto{\pgfqpoint{2.869010in}{2.311008in}}%
\pgfpathcurveto{\pgfqpoint{2.882033in}{2.311008in}}{\pgfqpoint{2.894524in}{2.316182in}}{\pgfqpoint{2.903732in}{2.325390in}}%
\pgfpathcurveto{\pgfqpoint{2.912941in}{2.334598in}}{\pgfqpoint{2.918115in}{2.347089in}}{\pgfqpoint{2.918115in}{2.360112in}}%
\pgfpathcurveto{\pgfqpoint{2.918115in}{2.373135in}}{\pgfqpoint{2.912941in}{2.385626in}}{\pgfqpoint{2.903732in}{2.394834in}}%
\pgfpathcurveto{\pgfqpoint{2.894524in}{2.404043in}}{\pgfqpoint{2.882033in}{2.409217in}}{\pgfqpoint{2.869010in}{2.409217in}}%
\pgfpathcurveto{\pgfqpoint{2.855987in}{2.409217in}}{\pgfqpoint{2.843496in}{2.404043in}}{\pgfqpoint{2.834288in}{2.394834in}}%
\pgfpathcurveto{\pgfqpoint{2.825079in}{2.385626in}}{\pgfqpoint{2.819905in}{2.373135in}}{\pgfqpoint{2.819905in}{2.360112in}}%
\pgfpathcurveto{\pgfqpoint{2.819905in}{2.347089in}}{\pgfqpoint{2.825079in}{2.334598in}}{\pgfqpoint{2.834288in}{2.325390in}}%
\pgfpathcurveto{\pgfqpoint{2.843496in}{2.316182in}}{\pgfqpoint{2.855987in}{2.311008in}}{\pgfqpoint{2.869010in}{2.311008in}}%
\pgfpathlineto{\pgfqpoint{2.869010in}{2.311008in}}%
\pgfpathclose%
\pgfusepath{stroke,fill}%
\end{pgfscope}%
\begin{pgfscope}%
\pgfpathrectangle{\pgfqpoint{0.786164in}{0.768110in}}{\pgfqpoint{8.851069in}{7.081890in}}%
\pgfusepath{clip}%
\pgfsetbuttcap%
\pgfsetroundjoin%
\definecolor{currentfill}{rgb}{0.203063,0.379716,0.553925}%
\pgfsetfillcolor{currentfill}%
\pgfsetfillopacity{0.700000}%
\pgfsetlinewidth{0.501875pt}%
\definecolor{currentstroke}{rgb}{1.000000,1.000000,1.000000}%
\pgfsetstrokecolor{currentstroke}%
\pgfsetstrokeopacity{0.700000}%
\pgfsetdash{}{0pt}%
\pgfpathmoveto{\pgfqpoint{2.622411in}{2.267211in}}%
\pgfpathcurveto{\pgfqpoint{2.635434in}{2.267211in}}{\pgfqpoint{2.647925in}{2.272385in}}{\pgfqpoint{2.657134in}{2.281593in}}%
\pgfpathcurveto{\pgfqpoint{2.666342in}{2.290802in}}{\pgfqpoint{2.671516in}{2.303293in}}{\pgfqpoint{2.671516in}{2.316316in}}%
\pgfpathcurveto{\pgfqpoint{2.671516in}{2.329338in}}{\pgfqpoint{2.666342in}{2.341829in}}{\pgfqpoint{2.657134in}{2.351038in}}%
\pgfpathcurveto{\pgfqpoint{2.647925in}{2.360246in}}{\pgfqpoint{2.635434in}{2.365420in}}{\pgfqpoint{2.622411in}{2.365420in}}%
\pgfpathcurveto{\pgfqpoint{2.609389in}{2.365420in}}{\pgfqpoint{2.596898in}{2.360246in}}{\pgfqpoint{2.587689in}{2.351038in}}%
\pgfpathcurveto{\pgfqpoint{2.578481in}{2.341829in}}{\pgfqpoint{2.573307in}{2.329338in}}{\pgfqpoint{2.573307in}{2.316316in}}%
\pgfpathcurveto{\pgfqpoint{2.573307in}{2.303293in}}{\pgfqpoint{2.578481in}{2.290802in}}{\pgfqpoint{2.587689in}{2.281593in}}%
\pgfpathcurveto{\pgfqpoint{2.596898in}{2.272385in}}{\pgfqpoint{2.609389in}{2.267211in}}{\pgfqpoint{2.622411in}{2.267211in}}%
\pgfpathlineto{\pgfqpoint{2.622411in}{2.267211in}}%
\pgfpathclose%
\pgfusepath{stroke,fill}%
\end{pgfscope}%
\begin{pgfscope}%
\pgfpathrectangle{\pgfqpoint{0.786164in}{0.768110in}}{\pgfqpoint{8.851069in}{7.081890in}}%
\pgfusepath{clip}%
\pgfsetbuttcap%
\pgfsetroundjoin%
\definecolor{currentfill}{rgb}{0.182256,0.426184,0.557120}%
\pgfsetfillcolor{currentfill}%
\pgfsetfillopacity{0.700000}%
\pgfsetlinewidth{0.501875pt}%
\definecolor{currentstroke}{rgb}{1.000000,1.000000,1.000000}%
\pgfsetstrokecolor{currentstroke}%
\pgfsetstrokeopacity{0.700000}%
\pgfsetdash{}{0pt}%
\pgfpathmoveto{\pgfqpoint{1.773016in}{2.179618in}}%
\pgfpathcurveto{\pgfqpoint{1.786038in}{2.179618in}}{\pgfqpoint{1.798529in}{2.184792in}}{\pgfqpoint{1.807738in}{2.194001in}}%
\pgfpathcurveto{\pgfqpoint{1.816946in}{2.203209in}}{\pgfqpoint{1.822120in}{2.215700in}}{\pgfqpoint{1.822120in}{2.228723in}}%
\pgfpathcurveto{\pgfqpoint{1.822120in}{2.241745in}}{\pgfqpoint{1.816946in}{2.254237in}}{\pgfqpoint{1.807738in}{2.263445in}}%
\pgfpathcurveto{\pgfqpoint{1.798529in}{2.272653in}}{\pgfqpoint{1.786038in}{2.277827in}}{\pgfqpoint{1.773016in}{2.277827in}}%
\pgfpathcurveto{\pgfqpoint{1.759993in}{2.277827in}}{\pgfqpoint{1.747502in}{2.272653in}}{\pgfqpoint{1.738293in}{2.263445in}}%
\pgfpathcurveto{\pgfqpoint{1.729085in}{2.254237in}}{\pgfqpoint{1.723911in}{2.241745in}}{\pgfqpoint{1.723911in}{2.228723in}}%
\pgfpathcurveto{\pgfqpoint{1.723911in}{2.215700in}}{\pgfqpoint{1.729085in}{2.203209in}}{\pgfqpoint{1.738293in}{2.194001in}}%
\pgfpathcurveto{\pgfqpoint{1.747502in}{2.184792in}}{\pgfqpoint{1.759993in}{2.179618in}}{\pgfqpoint{1.773016in}{2.179618in}}%
\pgfpathlineto{\pgfqpoint{1.773016in}{2.179618in}}%
\pgfpathclose%
\pgfusepath{stroke,fill}%
\end{pgfscope}%
\begin{pgfscope}%
\pgfpathrectangle{\pgfqpoint{0.786164in}{0.768110in}}{\pgfqpoint{8.851069in}{7.081890in}}%
\pgfusepath{clip}%
\pgfsetbuttcap%
\pgfsetroundjoin%
\definecolor{currentfill}{rgb}{0.180629,0.429975,0.557282}%
\pgfsetfillcolor{currentfill}%
\pgfsetfillopacity{0.700000}%
\pgfsetlinewidth{0.501875pt}%
\definecolor{currentstroke}{rgb}{1.000000,1.000000,1.000000}%
\pgfsetstrokecolor{currentstroke}%
\pgfsetstrokeopacity{0.700000}%
\pgfsetdash{}{0pt}%
\pgfpathmoveto{\pgfqpoint{1.809549in}{2.201516in}}%
\pgfpathcurveto{\pgfqpoint{1.822571in}{2.201516in}}{\pgfqpoint{1.835063in}{2.206690in}}{\pgfqpoint{1.844271in}{2.215899in}}%
\pgfpathcurveto{\pgfqpoint{1.853479in}{2.225107in}}{\pgfqpoint{1.858653in}{2.237598in}}{\pgfqpoint{1.858653in}{2.250621in}}%
\pgfpathcurveto{\pgfqpoint{1.858653in}{2.263644in}}{\pgfqpoint{1.853479in}{2.276135in}}{\pgfqpoint{1.844271in}{2.285343in}}%
\pgfpathcurveto{\pgfqpoint{1.835063in}{2.294552in}}{\pgfqpoint{1.822571in}{2.299726in}}{\pgfqpoint{1.809549in}{2.299726in}}%
\pgfpathcurveto{\pgfqpoint{1.796526in}{2.299726in}}{\pgfqpoint{1.784035in}{2.294552in}}{\pgfqpoint{1.774827in}{2.285343in}}%
\pgfpathcurveto{\pgfqpoint{1.765618in}{2.276135in}}{\pgfqpoint{1.760444in}{2.263644in}}{\pgfqpoint{1.760444in}{2.250621in}}%
\pgfpathcurveto{\pgfqpoint{1.760444in}{2.237598in}}{\pgfqpoint{1.765618in}{2.225107in}}{\pgfqpoint{1.774827in}{2.215899in}}%
\pgfpathcurveto{\pgfqpoint{1.784035in}{2.206690in}}{\pgfqpoint{1.796526in}{2.201516in}}{\pgfqpoint{1.809549in}{2.201516in}}%
\pgfpathlineto{\pgfqpoint{1.809549in}{2.201516in}}%
\pgfpathclose%
\pgfusepath{stroke,fill}%
\end{pgfscope}%
\begin{pgfscope}%
\pgfpathrectangle{\pgfqpoint{0.786164in}{0.768110in}}{\pgfqpoint{8.851069in}{7.081890in}}%
\pgfusepath{clip}%
\pgfsetbuttcap%
\pgfsetroundjoin%
\definecolor{currentfill}{rgb}{0.187231,0.414746,0.556547}%
\pgfsetfillcolor{currentfill}%
\pgfsetfillopacity{0.700000}%
\pgfsetlinewidth{0.501875pt}%
\definecolor{currentstroke}{rgb}{1.000000,1.000000,1.000000}%
\pgfsetstrokecolor{currentstroke}%
\pgfsetstrokeopacity{0.700000}%
\pgfsetdash{}{0pt}%
\pgfpathmoveto{\pgfqpoint{1.800415in}{2.223415in}}%
\pgfpathcurveto{\pgfqpoint{1.813438in}{2.223415in}}{\pgfqpoint{1.825929in}{2.228589in}}{\pgfqpoint{1.835138in}{2.237797in}}%
\pgfpathcurveto{\pgfqpoint{1.844346in}{2.247005in}}{\pgfqpoint{1.849520in}{2.259497in}}{\pgfqpoint{1.849520in}{2.272519in}}%
\pgfpathcurveto{\pgfqpoint{1.849520in}{2.285542in}}{\pgfqpoint{1.844346in}{2.298033in}}{\pgfqpoint{1.835138in}{2.307241in}}%
\pgfpathcurveto{\pgfqpoint{1.825929in}{2.316450in}}{\pgfqpoint{1.813438in}{2.321624in}}{\pgfqpoint{1.800415in}{2.321624in}}%
\pgfpathcurveto{\pgfqpoint{1.787393in}{2.321624in}}{\pgfqpoint{1.774902in}{2.316450in}}{\pgfqpoint{1.765693in}{2.307241in}}%
\pgfpathcurveto{\pgfqpoint{1.756485in}{2.298033in}}{\pgfqpoint{1.751311in}{2.285542in}}{\pgfqpoint{1.751311in}{2.272519in}}%
\pgfpathcurveto{\pgfqpoint{1.751311in}{2.259497in}}{\pgfqpoint{1.756485in}{2.247005in}}{\pgfqpoint{1.765693in}{2.237797in}}%
\pgfpathcurveto{\pgfqpoint{1.774902in}{2.228589in}}{\pgfqpoint{1.787393in}{2.223415in}}{\pgfqpoint{1.800415in}{2.223415in}}%
\pgfpathlineto{\pgfqpoint{1.800415in}{2.223415in}}%
\pgfpathclose%
\pgfusepath{stroke,fill}%
\end{pgfscope}%
\begin{pgfscope}%
\pgfpathrectangle{\pgfqpoint{0.786164in}{0.768110in}}{\pgfqpoint{8.851069in}{7.081890in}}%
\pgfusepath{clip}%
\pgfsetbuttcap%
\pgfsetroundjoin%
\definecolor{currentfill}{rgb}{0.190631,0.407061,0.556089}%
\pgfsetfillcolor{currentfill}%
\pgfsetfillopacity{0.700000}%
\pgfsetlinewidth{0.501875pt}%
\definecolor{currentstroke}{rgb}{1.000000,1.000000,1.000000}%
\pgfsetstrokecolor{currentstroke}%
\pgfsetstrokeopacity{0.700000}%
\pgfsetdash{}{0pt}%
\pgfpathmoveto{\pgfqpoint{1.873482in}{2.354804in}}%
\pgfpathcurveto{\pgfqpoint{1.886504in}{2.354804in}}{\pgfqpoint{1.898996in}{2.359978in}}{\pgfqpoint{1.908204in}{2.369186in}}%
\pgfpathcurveto{\pgfqpoint{1.917412in}{2.378395in}}{\pgfqpoint{1.922586in}{2.390886in}}{\pgfqpoint{1.922586in}{2.403909in}}%
\pgfpathcurveto{\pgfqpoint{1.922586in}{2.416931in}}{\pgfqpoint{1.917412in}{2.429422in}}{\pgfqpoint{1.908204in}{2.438631in}}%
\pgfpathcurveto{\pgfqpoint{1.898996in}{2.447839in}}{\pgfqpoint{1.886504in}{2.453013in}}{\pgfqpoint{1.873482in}{2.453013in}}%
\pgfpathcurveto{\pgfqpoint{1.860459in}{2.453013in}}{\pgfqpoint{1.847968in}{2.447839in}}{\pgfqpoint{1.838760in}{2.438631in}}%
\pgfpathcurveto{\pgfqpoint{1.829551in}{2.429422in}}{\pgfqpoint{1.824377in}{2.416931in}}{\pgfqpoint{1.824377in}{2.403909in}}%
\pgfpathcurveto{\pgfqpoint{1.824377in}{2.390886in}}{\pgfqpoint{1.829551in}{2.378395in}}{\pgfqpoint{1.838760in}{2.369186in}}%
\pgfpathcurveto{\pgfqpoint{1.847968in}{2.359978in}}{\pgfqpoint{1.860459in}{2.354804in}}{\pgfqpoint{1.873482in}{2.354804in}}%
\pgfpathlineto{\pgfqpoint{1.873482in}{2.354804in}}%
\pgfpathclose%
\pgfusepath{stroke,fill}%
\end{pgfscope}%
\begin{pgfscope}%
\pgfpathrectangle{\pgfqpoint{0.786164in}{0.768110in}}{\pgfqpoint{8.851069in}{7.081890in}}%
\pgfusepath{clip}%
\pgfsetbuttcap%
\pgfsetroundjoin%
\definecolor{currentfill}{rgb}{0.190631,0.407061,0.556089}%
\pgfsetfillcolor{currentfill}%
\pgfsetfillopacity{0.700000}%
\pgfsetlinewidth{0.501875pt}%
\definecolor{currentstroke}{rgb}{1.000000,1.000000,1.000000}%
\pgfsetstrokecolor{currentstroke}%
\pgfsetstrokeopacity{0.700000}%
\pgfsetdash{}{0pt}%
\pgfpathmoveto{\pgfqpoint{1.818682in}{2.311008in}}%
\pgfpathcurveto{\pgfqpoint{1.831705in}{2.311008in}}{\pgfqpoint{1.844196in}{2.316182in}}{\pgfqpoint{1.853404in}{2.325390in}}%
\pgfpathcurveto{\pgfqpoint{1.862613in}{2.334598in}}{\pgfqpoint{1.867787in}{2.347089in}}{\pgfqpoint{1.867787in}{2.360112in}}%
\pgfpathcurveto{\pgfqpoint{1.867787in}{2.373135in}}{\pgfqpoint{1.862613in}{2.385626in}}{\pgfqpoint{1.853404in}{2.394834in}}%
\pgfpathcurveto{\pgfqpoint{1.844196in}{2.404043in}}{\pgfqpoint{1.831705in}{2.409217in}}{\pgfqpoint{1.818682in}{2.409217in}}%
\pgfpathcurveto{\pgfqpoint{1.805659in}{2.409217in}}{\pgfqpoint{1.793168in}{2.404043in}}{\pgfqpoint{1.783960in}{2.394834in}}%
\pgfpathcurveto{\pgfqpoint{1.774751in}{2.385626in}}{\pgfqpoint{1.769577in}{2.373135in}}{\pgfqpoint{1.769577in}{2.360112in}}%
\pgfpathcurveto{\pgfqpoint{1.769577in}{2.347089in}}{\pgfqpoint{1.774751in}{2.334598in}}{\pgfqpoint{1.783960in}{2.325390in}}%
\pgfpathcurveto{\pgfqpoint{1.793168in}{2.316182in}}{\pgfqpoint{1.805659in}{2.311008in}}{\pgfqpoint{1.818682in}{2.311008in}}%
\pgfpathlineto{\pgfqpoint{1.818682in}{2.311008in}}%
\pgfpathclose%
\pgfusepath{stroke,fill}%
\end{pgfscope}%
\begin{pgfscope}%
\pgfpathrectangle{\pgfqpoint{0.786164in}{0.768110in}}{\pgfqpoint{8.851069in}{7.081890in}}%
\pgfusepath{clip}%
\pgfsetbuttcap%
\pgfsetroundjoin%
\definecolor{currentfill}{rgb}{0.180629,0.429975,0.557282}%
\pgfsetfillcolor{currentfill}%
\pgfsetfillopacity{0.700000}%
\pgfsetlinewidth{0.501875pt}%
\definecolor{currentstroke}{rgb}{1.000000,1.000000,1.000000}%
\pgfsetstrokecolor{currentstroke}%
\pgfsetstrokeopacity{0.700000}%
\pgfsetdash{}{0pt}%
\pgfpathmoveto{\pgfqpoint{1.773016in}{2.201516in}}%
\pgfpathcurveto{\pgfqpoint{1.786038in}{2.201516in}}{\pgfqpoint{1.798529in}{2.206690in}}{\pgfqpoint{1.807738in}{2.215899in}}%
\pgfpathcurveto{\pgfqpoint{1.816946in}{2.225107in}}{\pgfqpoint{1.822120in}{2.237598in}}{\pgfqpoint{1.822120in}{2.250621in}}%
\pgfpathcurveto{\pgfqpoint{1.822120in}{2.263644in}}{\pgfqpoint{1.816946in}{2.276135in}}{\pgfqpoint{1.807738in}{2.285343in}}%
\pgfpathcurveto{\pgfqpoint{1.798529in}{2.294552in}}{\pgfqpoint{1.786038in}{2.299726in}}{\pgfqpoint{1.773016in}{2.299726in}}%
\pgfpathcurveto{\pgfqpoint{1.759993in}{2.299726in}}{\pgfqpoint{1.747502in}{2.294552in}}{\pgfqpoint{1.738293in}{2.285343in}}%
\pgfpathcurveto{\pgfqpoint{1.729085in}{2.276135in}}{\pgfqpoint{1.723911in}{2.263644in}}{\pgfqpoint{1.723911in}{2.250621in}}%
\pgfpathcurveto{\pgfqpoint{1.723911in}{2.237598in}}{\pgfqpoint{1.729085in}{2.225107in}}{\pgfqpoint{1.738293in}{2.215899in}}%
\pgfpathcurveto{\pgfqpoint{1.747502in}{2.206690in}}{\pgfqpoint{1.759993in}{2.201516in}}{\pgfqpoint{1.773016in}{2.201516in}}%
\pgfpathlineto{\pgfqpoint{1.773016in}{2.201516in}}%
\pgfpathclose%
\pgfusepath{stroke,fill}%
\end{pgfscope}%
\begin{pgfscope}%
\pgfpathrectangle{\pgfqpoint{0.786164in}{0.768110in}}{\pgfqpoint{8.851069in}{7.081890in}}%
\pgfusepath{clip}%
\pgfsetbuttcap%
\pgfsetroundjoin%
\definecolor{currentfill}{rgb}{0.171176,0.452530,0.557965}%
\pgfsetfillcolor{currentfill}%
\pgfsetfillopacity{0.700000}%
\pgfsetlinewidth{0.501875pt}%
\definecolor{currentstroke}{rgb}{1.000000,1.000000,1.000000}%
\pgfsetstrokecolor{currentstroke}%
\pgfsetstrokeopacity{0.700000}%
\pgfsetdash{}{0pt}%
\pgfpathmoveto{\pgfqpoint{1.754749in}{2.179618in}}%
\pgfpathcurveto{\pgfqpoint{1.767772in}{2.179618in}}{\pgfqpoint{1.780263in}{2.184792in}}{\pgfqpoint{1.789471in}{2.194001in}}%
\pgfpathcurveto{\pgfqpoint{1.798680in}{2.203209in}}{\pgfqpoint{1.803854in}{2.215700in}}{\pgfqpoint{1.803854in}{2.228723in}}%
\pgfpathcurveto{\pgfqpoint{1.803854in}{2.241745in}}{\pgfqpoint{1.798680in}{2.254237in}}{\pgfqpoint{1.789471in}{2.263445in}}%
\pgfpathcurveto{\pgfqpoint{1.780263in}{2.272653in}}{\pgfqpoint{1.767772in}{2.277827in}}{\pgfqpoint{1.754749in}{2.277827in}}%
\pgfpathcurveto{\pgfqpoint{1.741726in}{2.277827in}}{\pgfqpoint{1.729235in}{2.272653in}}{\pgfqpoint{1.720027in}{2.263445in}}%
\pgfpathcurveto{\pgfqpoint{1.710818in}{2.254237in}}{\pgfqpoint{1.705644in}{2.241745in}}{\pgfqpoint{1.705644in}{2.228723in}}%
\pgfpathcurveto{\pgfqpoint{1.705644in}{2.215700in}}{\pgfqpoint{1.710818in}{2.203209in}}{\pgfqpoint{1.720027in}{2.194001in}}%
\pgfpathcurveto{\pgfqpoint{1.729235in}{2.184792in}}{\pgfqpoint{1.741726in}{2.179618in}}{\pgfqpoint{1.754749in}{2.179618in}}%
\pgfpathlineto{\pgfqpoint{1.754749in}{2.179618in}}%
\pgfpathclose%
\pgfusepath{stroke,fill}%
\end{pgfscope}%
\begin{pgfscope}%
\pgfpathrectangle{\pgfqpoint{0.786164in}{0.768110in}}{\pgfqpoint{8.851069in}{7.081890in}}%
\pgfusepath{clip}%
\pgfsetbuttcap%
\pgfsetroundjoin%
\definecolor{currentfill}{rgb}{0.169646,0.456262,0.558030}%
\pgfsetfillcolor{currentfill}%
\pgfsetfillopacity{0.700000}%
\pgfsetlinewidth{0.501875pt}%
\definecolor{currentstroke}{rgb}{1.000000,1.000000,1.000000}%
\pgfsetstrokecolor{currentstroke}%
\pgfsetstrokeopacity{0.700000}%
\pgfsetdash{}{0pt}%
\pgfpathmoveto{\pgfqpoint{1.718216in}{2.245313in}}%
\pgfpathcurveto{\pgfqpoint{1.731239in}{2.245313in}}{\pgfqpoint{1.743730in}{2.250487in}}{\pgfqpoint{1.752938in}{2.259695in}}%
\pgfpathcurveto{\pgfqpoint{1.762147in}{2.268904in}}{\pgfqpoint{1.767321in}{2.281395in}}{\pgfqpoint{1.767321in}{2.294417in}}%
\pgfpathcurveto{\pgfqpoint{1.767321in}{2.307440in}}{\pgfqpoint{1.762147in}{2.319931in}}{\pgfqpoint{1.752938in}{2.329140in}}%
\pgfpathcurveto{\pgfqpoint{1.743730in}{2.338348in}}{\pgfqpoint{1.731239in}{2.343522in}}{\pgfqpoint{1.718216in}{2.343522in}}%
\pgfpathcurveto{\pgfqpoint{1.705193in}{2.343522in}}{\pgfqpoint{1.692702in}{2.338348in}}{\pgfqpoint{1.683494in}{2.329140in}}%
\pgfpathcurveto{\pgfqpoint{1.674285in}{2.319931in}}{\pgfqpoint{1.669111in}{2.307440in}}{\pgfqpoint{1.669111in}{2.294417in}}%
\pgfpathcurveto{\pgfqpoint{1.669111in}{2.281395in}}{\pgfqpoint{1.674285in}{2.268904in}}{\pgfqpoint{1.683494in}{2.259695in}}%
\pgfpathcurveto{\pgfqpoint{1.692702in}{2.250487in}}{\pgfqpoint{1.705193in}{2.245313in}}{\pgfqpoint{1.718216in}{2.245313in}}%
\pgfpathlineto{\pgfqpoint{1.718216in}{2.245313in}}%
\pgfpathclose%
\pgfusepath{stroke,fill}%
\end{pgfscope}%
\begin{pgfscope}%
\pgfpathrectangle{\pgfqpoint{0.786164in}{0.768110in}}{\pgfqpoint{8.851069in}{7.081890in}}%
\pgfusepath{clip}%
\pgfsetbuttcap%
\pgfsetroundjoin%
\definecolor{currentfill}{rgb}{0.162142,0.474838,0.558140}%
\pgfsetfillcolor{currentfill}%
\pgfsetfillopacity{0.700000}%
\pgfsetlinewidth{0.501875pt}%
\definecolor{currentstroke}{rgb}{1.000000,1.000000,1.000000}%
\pgfsetstrokecolor{currentstroke}%
\pgfsetstrokeopacity{0.700000}%
\pgfsetdash{}{0pt}%
\pgfpathmoveto{\pgfqpoint{1.626883in}{2.179618in}}%
\pgfpathcurveto{\pgfqpoint{1.639906in}{2.179618in}}{\pgfqpoint{1.652397in}{2.184792in}}{\pgfqpoint{1.661605in}{2.194001in}}%
\pgfpathcurveto{\pgfqpoint{1.670814in}{2.203209in}}{\pgfqpoint{1.675988in}{2.215700in}}{\pgfqpoint{1.675988in}{2.228723in}}%
\pgfpathcurveto{\pgfqpoint{1.675988in}{2.241745in}}{\pgfqpoint{1.670814in}{2.254237in}}{\pgfqpoint{1.661605in}{2.263445in}}%
\pgfpathcurveto{\pgfqpoint{1.652397in}{2.272653in}}{\pgfqpoint{1.639906in}{2.277827in}}{\pgfqpoint{1.626883in}{2.277827in}}%
\pgfpathcurveto{\pgfqpoint{1.613860in}{2.277827in}}{\pgfqpoint{1.601369in}{2.272653in}}{\pgfqpoint{1.592161in}{2.263445in}}%
\pgfpathcurveto{\pgfqpoint{1.582952in}{2.254237in}}{\pgfqpoint{1.577778in}{2.241745in}}{\pgfqpoint{1.577778in}{2.228723in}}%
\pgfpathcurveto{\pgfqpoint{1.577778in}{2.215700in}}{\pgfqpoint{1.582952in}{2.203209in}}{\pgfqpoint{1.592161in}{2.194001in}}%
\pgfpathcurveto{\pgfqpoint{1.601369in}{2.184792in}}{\pgfqpoint{1.613860in}{2.179618in}}{\pgfqpoint{1.626883in}{2.179618in}}%
\pgfpathlineto{\pgfqpoint{1.626883in}{2.179618in}}%
\pgfpathclose%
\pgfusepath{stroke,fill}%
\end{pgfscope}%
\begin{pgfscope}%
\pgfpathrectangle{\pgfqpoint{0.786164in}{0.768110in}}{\pgfqpoint{8.851069in}{7.081890in}}%
\pgfusepath{clip}%
\pgfsetbuttcap%
\pgfsetroundjoin%
\definecolor{currentfill}{rgb}{0.154815,0.493313,0.557840}%
\pgfsetfillcolor{currentfill}%
\pgfsetfillopacity{0.700000}%
\pgfsetlinewidth{0.501875pt}%
\definecolor{currentstroke}{rgb}{1.000000,1.000000,1.000000}%
\pgfsetstrokecolor{currentstroke}%
\pgfsetstrokeopacity{0.700000}%
\pgfsetdash{}{0pt}%
\pgfpathmoveto{\pgfqpoint{1.590350in}{2.092025in}}%
\pgfpathcurveto{\pgfqpoint{1.603373in}{2.092025in}}{\pgfqpoint{1.615864in}{2.097199in}}{\pgfqpoint{1.625072in}{2.106408in}}%
\pgfpathcurveto{\pgfqpoint{1.634281in}{2.115616in}}{\pgfqpoint{1.639454in}{2.128107in}}{\pgfqpoint{1.639454in}{2.141130in}}%
\pgfpathcurveto{\pgfqpoint{1.639454in}{2.154153in}}{\pgfqpoint{1.634281in}{2.166644in}}{\pgfqpoint{1.625072in}{2.175852in}}%
\pgfpathcurveto{\pgfqpoint{1.615864in}{2.185060in}}{\pgfqpoint{1.603373in}{2.190234in}}{\pgfqpoint{1.590350in}{2.190234in}}%
\pgfpathcurveto{\pgfqpoint{1.577327in}{2.190234in}}{\pgfqpoint{1.564836in}{2.185060in}}{\pgfqpoint{1.555628in}{2.175852in}}%
\pgfpathcurveto{\pgfqpoint{1.546419in}{2.166644in}}{\pgfqpoint{1.541245in}{2.154153in}}{\pgfqpoint{1.541245in}{2.141130in}}%
\pgfpathcurveto{\pgfqpoint{1.541245in}{2.128107in}}{\pgfqpoint{1.546419in}{2.115616in}}{\pgfqpoint{1.555628in}{2.106408in}}%
\pgfpathcurveto{\pgfqpoint{1.564836in}{2.097199in}}{\pgfqpoint{1.577327in}{2.092025in}}{\pgfqpoint{1.590350in}{2.092025in}}%
\pgfpathlineto{\pgfqpoint{1.590350in}{2.092025in}}%
\pgfpathclose%
\pgfusepath{stroke,fill}%
\end{pgfscope}%
\begin{pgfscope}%
\pgfpathrectangle{\pgfqpoint{0.786164in}{0.768110in}}{\pgfqpoint{8.851069in}{7.081890in}}%
\pgfusepath{clip}%
\pgfsetbuttcap%
\pgfsetroundjoin%
\definecolor{currentfill}{rgb}{0.156270,0.489624,0.557936}%
\pgfsetfillcolor{currentfill}%
\pgfsetfillopacity{0.700000}%
\pgfsetlinewidth{0.501875pt}%
\definecolor{currentstroke}{rgb}{1.000000,1.000000,1.000000}%
\pgfsetstrokecolor{currentstroke}%
\pgfsetstrokeopacity{0.700000}%
\pgfsetdash{}{0pt}%
\pgfpathmoveto{\pgfqpoint{1.508150in}{2.070127in}}%
\pgfpathcurveto{\pgfqpoint{1.521173in}{2.070127in}}{\pgfqpoint{1.533664in}{2.075301in}}{\pgfqpoint{1.542872in}{2.084509in}}%
\pgfpathcurveto{\pgfqpoint{1.552081in}{2.093718in}}{\pgfqpoint{1.557255in}{2.106209in}}{\pgfqpoint{1.557255in}{2.119232in}}%
\pgfpathcurveto{\pgfqpoint{1.557255in}{2.132254in}}{\pgfqpoint{1.552081in}{2.144745in}}{\pgfqpoint{1.542872in}{2.153954in}}%
\pgfpathcurveto{\pgfqpoint{1.533664in}{2.163162in}}{\pgfqpoint{1.521173in}{2.168336in}}{\pgfqpoint{1.508150in}{2.168336in}}%
\pgfpathcurveto{\pgfqpoint{1.495128in}{2.168336in}}{\pgfqpoint{1.482636in}{2.163162in}}{\pgfqpoint{1.473428in}{2.153954in}}%
\pgfpathcurveto{\pgfqpoint{1.464220in}{2.144745in}}{\pgfqpoint{1.459046in}{2.132254in}}{\pgfqpoint{1.459046in}{2.119232in}}%
\pgfpathcurveto{\pgfqpoint{1.459046in}{2.106209in}}{\pgfqpoint{1.464220in}{2.093718in}}{\pgfqpoint{1.473428in}{2.084509in}}%
\pgfpathcurveto{\pgfqpoint{1.482636in}{2.075301in}}{\pgfqpoint{1.495128in}{2.070127in}}{\pgfqpoint{1.508150in}{2.070127in}}%
\pgfpathlineto{\pgfqpoint{1.508150in}{2.070127in}}%
\pgfpathclose%
\pgfusepath{stroke,fill}%
\end{pgfscope}%
\begin{pgfscope}%
\pgfpathrectangle{\pgfqpoint{0.786164in}{0.768110in}}{\pgfqpoint{8.851069in}{7.081890in}}%
\pgfusepath{clip}%
\pgfsetbuttcap%
\pgfsetroundjoin%
\definecolor{currentfill}{rgb}{0.149039,0.508051,0.557250}%
\pgfsetfillcolor{currentfill}%
\pgfsetfillopacity{0.700000}%
\pgfsetlinewidth{0.501875pt}%
\definecolor{currentstroke}{rgb}{1.000000,1.000000,1.000000}%
\pgfsetstrokecolor{currentstroke}%
\pgfsetstrokeopacity{0.700000}%
\pgfsetdash{}{0pt}%
\pgfpathmoveto{\pgfqpoint{1.599483in}{2.113923in}}%
\pgfpathcurveto{\pgfqpoint{1.612506in}{2.113923in}}{\pgfqpoint{1.624997in}{2.119097in}}{\pgfqpoint{1.634205in}{2.128306in}}%
\pgfpathcurveto{\pgfqpoint{1.643414in}{2.137514in}}{\pgfqpoint{1.648588in}{2.150005in}}{\pgfqpoint{1.648588in}{2.163028in}}%
\pgfpathcurveto{\pgfqpoint{1.648588in}{2.176051in}}{\pgfqpoint{1.643414in}{2.188542in}}{\pgfqpoint{1.634205in}{2.197750in}}%
\pgfpathcurveto{\pgfqpoint{1.624997in}{2.206959in}}{\pgfqpoint{1.612506in}{2.212133in}}{\pgfqpoint{1.599483in}{2.212133in}}%
\pgfpathcurveto{\pgfqpoint{1.586460in}{2.212133in}}{\pgfqpoint{1.573969in}{2.206959in}}{\pgfqpoint{1.564761in}{2.197750in}}%
\pgfpathcurveto{\pgfqpoint{1.555552in}{2.188542in}}{\pgfqpoint{1.550379in}{2.176051in}}{\pgfqpoint{1.550379in}{2.163028in}}%
\pgfpathcurveto{\pgfqpoint{1.550379in}{2.150005in}}{\pgfqpoint{1.555552in}{2.137514in}}{\pgfqpoint{1.564761in}{2.128306in}}%
\pgfpathcurveto{\pgfqpoint{1.573969in}{2.119097in}}{\pgfqpoint{1.586460in}{2.113923in}}{\pgfqpoint{1.599483in}{2.113923in}}%
\pgfpathlineto{\pgfqpoint{1.599483in}{2.113923in}}%
\pgfpathclose%
\pgfusepath{stroke,fill}%
\end{pgfscope}%
\begin{pgfscope}%
\pgfpathrectangle{\pgfqpoint{0.786164in}{0.768110in}}{\pgfqpoint{8.851069in}{7.081890in}}%
\pgfusepath{clip}%
\pgfsetbuttcap%
\pgfsetroundjoin%
\definecolor{currentfill}{rgb}{0.153364,0.497000,0.557724}%
\pgfsetfillcolor{currentfill}%
\pgfsetfillopacity{0.700000}%
\pgfsetlinewidth{0.501875pt}%
\definecolor{currentstroke}{rgb}{1.000000,1.000000,1.000000}%
\pgfsetstrokecolor{currentstroke}%
\pgfsetstrokeopacity{0.700000}%
\pgfsetdash{}{0pt}%
\pgfpathmoveto{\pgfqpoint{1.636016in}{2.135822in}}%
\pgfpathcurveto{\pgfqpoint{1.649039in}{2.135822in}}{\pgfqpoint{1.661530in}{2.140996in}}{\pgfqpoint{1.670739in}{2.150204in}}%
\pgfpathcurveto{\pgfqpoint{1.679947in}{2.159413in}}{\pgfqpoint{1.685121in}{2.171904in}}{\pgfqpoint{1.685121in}{2.184926in}}%
\pgfpathcurveto{\pgfqpoint{1.685121in}{2.197949in}}{\pgfqpoint{1.679947in}{2.210440in}}{\pgfqpoint{1.670739in}{2.219649in}}%
\pgfpathcurveto{\pgfqpoint{1.661530in}{2.228857in}}{\pgfqpoint{1.649039in}{2.234031in}}{\pgfqpoint{1.636016in}{2.234031in}}%
\pgfpathcurveto{\pgfqpoint{1.622994in}{2.234031in}}{\pgfqpoint{1.610503in}{2.228857in}}{\pgfqpoint{1.601294in}{2.219649in}}%
\pgfpathcurveto{\pgfqpoint{1.592086in}{2.210440in}}{\pgfqpoint{1.586912in}{2.197949in}}{\pgfqpoint{1.586912in}{2.184926in}}%
\pgfpathcurveto{\pgfqpoint{1.586912in}{2.171904in}}{\pgfqpoint{1.592086in}{2.159413in}}{\pgfqpoint{1.601294in}{2.150204in}}%
\pgfpathcurveto{\pgfqpoint{1.610503in}{2.140996in}}{\pgfqpoint{1.622994in}{2.135822in}}{\pgfqpoint{1.636016in}{2.135822in}}%
\pgfpathlineto{\pgfqpoint{1.636016in}{2.135822in}}%
\pgfpathclose%
\pgfusepath{stroke,fill}%
\end{pgfscope}%
\begin{pgfscope}%
\pgfpathrectangle{\pgfqpoint{0.786164in}{0.768110in}}{\pgfqpoint{8.851069in}{7.081890in}}%
\pgfusepath{clip}%
\pgfsetbuttcap%
\pgfsetroundjoin%
\definecolor{currentfill}{rgb}{0.159194,0.482237,0.558073}%
\pgfsetfillcolor{currentfill}%
\pgfsetfillopacity{0.700000}%
\pgfsetlinewidth{0.501875pt}%
\definecolor{currentstroke}{rgb}{1.000000,1.000000,1.000000}%
\pgfsetstrokecolor{currentstroke}%
\pgfsetstrokeopacity{0.700000}%
\pgfsetdash{}{0pt}%
\pgfpathmoveto{\pgfqpoint{1.718216in}{2.245313in}}%
\pgfpathcurveto{\pgfqpoint{1.731239in}{2.245313in}}{\pgfqpoint{1.743730in}{2.250487in}}{\pgfqpoint{1.752938in}{2.259695in}}%
\pgfpathcurveto{\pgfqpoint{1.762147in}{2.268904in}}{\pgfqpoint{1.767321in}{2.281395in}}{\pgfqpoint{1.767321in}{2.294417in}}%
\pgfpathcurveto{\pgfqpoint{1.767321in}{2.307440in}}{\pgfqpoint{1.762147in}{2.319931in}}{\pgfqpoint{1.752938in}{2.329140in}}%
\pgfpathcurveto{\pgfqpoint{1.743730in}{2.338348in}}{\pgfqpoint{1.731239in}{2.343522in}}{\pgfqpoint{1.718216in}{2.343522in}}%
\pgfpathcurveto{\pgfqpoint{1.705193in}{2.343522in}}{\pgfqpoint{1.692702in}{2.338348in}}{\pgfqpoint{1.683494in}{2.329140in}}%
\pgfpathcurveto{\pgfqpoint{1.674285in}{2.319931in}}{\pgfqpoint{1.669111in}{2.307440in}}{\pgfqpoint{1.669111in}{2.294417in}}%
\pgfpathcurveto{\pgfqpoint{1.669111in}{2.281395in}}{\pgfqpoint{1.674285in}{2.268904in}}{\pgfqpoint{1.683494in}{2.259695in}}%
\pgfpathcurveto{\pgfqpoint{1.692702in}{2.250487in}}{\pgfqpoint{1.705193in}{2.245313in}}{\pgfqpoint{1.718216in}{2.245313in}}%
\pgfpathlineto{\pgfqpoint{1.718216in}{2.245313in}}%
\pgfpathclose%
\pgfusepath{stroke,fill}%
\end{pgfscope}%
\begin{pgfscope}%
\pgfpathrectangle{\pgfqpoint{0.786164in}{0.768110in}}{\pgfqpoint{8.851069in}{7.081890in}}%
\pgfusepath{clip}%
\pgfsetbuttcap%
\pgfsetroundjoin%
\definecolor{currentfill}{rgb}{0.154815,0.493313,0.557840}%
\pgfsetfillcolor{currentfill}%
\pgfsetfillopacity{0.700000}%
\pgfsetlinewidth{0.501875pt}%
\definecolor{currentstroke}{rgb}{1.000000,1.000000,1.000000}%
\pgfsetstrokecolor{currentstroke}%
\pgfsetstrokeopacity{0.700000}%
\pgfsetdash{}{0pt}%
\pgfpathmoveto{\pgfqpoint{1.699949in}{2.157720in}}%
\pgfpathcurveto{\pgfqpoint{1.712972in}{2.157720in}}{\pgfqpoint{1.725463in}{2.162894in}}{\pgfqpoint{1.734672in}{2.172102in}}%
\pgfpathcurveto{\pgfqpoint{1.743880in}{2.181311in}}{\pgfqpoint{1.749054in}{2.193802in}}{\pgfqpoint{1.749054in}{2.206825in}}%
\pgfpathcurveto{\pgfqpoint{1.749054in}{2.219847in}}{\pgfqpoint{1.743880in}{2.232338in}}{\pgfqpoint{1.734672in}{2.241547in}}%
\pgfpathcurveto{\pgfqpoint{1.725463in}{2.250755in}}{\pgfqpoint{1.712972in}{2.255929in}}{\pgfqpoint{1.699949in}{2.255929in}}%
\pgfpathcurveto{\pgfqpoint{1.686927in}{2.255929in}}{\pgfqpoint{1.674436in}{2.250755in}}{\pgfqpoint{1.665227in}{2.241547in}}%
\pgfpathcurveto{\pgfqpoint{1.656019in}{2.232338in}}{\pgfqpoint{1.650845in}{2.219847in}}{\pgfqpoint{1.650845in}{2.206825in}}%
\pgfpathcurveto{\pgfqpoint{1.650845in}{2.193802in}}{\pgfqpoint{1.656019in}{2.181311in}}{\pgfqpoint{1.665227in}{2.172102in}}%
\pgfpathcurveto{\pgfqpoint{1.674436in}{2.162894in}}{\pgfqpoint{1.686927in}{2.157720in}}{\pgfqpoint{1.699949in}{2.157720in}}%
\pgfpathlineto{\pgfqpoint{1.699949in}{2.157720in}}%
\pgfpathclose%
\pgfusepath{stroke,fill}%
\end{pgfscope}%
\begin{pgfscope}%
\pgfpathrectangle{\pgfqpoint{0.786164in}{0.768110in}}{\pgfqpoint{8.851069in}{7.081890in}}%
\pgfusepath{clip}%
\pgfsetbuttcap%
\pgfsetroundjoin%
\definecolor{currentfill}{rgb}{0.151918,0.500685,0.557587}%
\pgfsetfillcolor{currentfill}%
\pgfsetfillopacity{0.700000}%
\pgfsetlinewidth{0.501875pt}%
\definecolor{currentstroke}{rgb}{1.000000,1.000000,1.000000}%
\pgfsetstrokecolor{currentstroke}%
\pgfsetstrokeopacity{0.700000}%
\pgfsetdash{}{0pt}%
\pgfpathmoveto{\pgfqpoint{1.718216in}{2.157720in}}%
\pgfpathcurveto{\pgfqpoint{1.731239in}{2.157720in}}{\pgfqpoint{1.743730in}{2.162894in}}{\pgfqpoint{1.752938in}{2.172102in}}%
\pgfpathcurveto{\pgfqpoint{1.762147in}{2.181311in}}{\pgfqpoint{1.767321in}{2.193802in}}{\pgfqpoint{1.767321in}{2.206825in}}%
\pgfpathcurveto{\pgfqpoint{1.767321in}{2.219847in}}{\pgfqpoint{1.762147in}{2.232338in}}{\pgfqpoint{1.752938in}{2.241547in}}%
\pgfpathcurveto{\pgfqpoint{1.743730in}{2.250755in}}{\pgfqpoint{1.731239in}{2.255929in}}{\pgfqpoint{1.718216in}{2.255929in}}%
\pgfpathcurveto{\pgfqpoint{1.705193in}{2.255929in}}{\pgfqpoint{1.692702in}{2.250755in}}{\pgfqpoint{1.683494in}{2.241547in}}%
\pgfpathcurveto{\pgfqpoint{1.674285in}{2.232338in}}{\pgfqpoint{1.669111in}{2.219847in}}{\pgfqpoint{1.669111in}{2.206825in}}%
\pgfpathcurveto{\pgfqpoint{1.669111in}{2.193802in}}{\pgfqpoint{1.674285in}{2.181311in}}{\pgfqpoint{1.683494in}{2.172102in}}%
\pgfpathcurveto{\pgfqpoint{1.692702in}{2.162894in}}{\pgfqpoint{1.705193in}{2.157720in}}{\pgfqpoint{1.718216in}{2.157720in}}%
\pgfpathlineto{\pgfqpoint{1.718216in}{2.157720in}}%
\pgfpathclose%
\pgfusepath{stroke,fill}%
\end{pgfscope}%
\begin{pgfscope}%
\pgfpathrectangle{\pgfqpoint{0.786164in}{0.768110in}}{\pgfqpoint{8.851069in}{7.081890in}}%
\pgfusepath{clip}%
\pgfsetbuttcap%
\pgfsetroundjoin%
\definecolor{currentfill}{rgb}{0.137770,0.537492,0.554906}%
\pgfsetfillcolor{currentfill}%
\pgfsetfillopacity{0.700000}%
\pgfsetlinewidth{0.501875pt}%
\definecolor{currentstroke}{rgb}{1.000000,1.000000,1.000000}%
\pgfsetstrokecolor{currentstroke}%
\pgfsetstrokeopacity{0.700000}%
\pgfsetdash{}{0pt}%
\pgfpathmoveto{\pgfqpoint{1.626883in}{2.092025in}}%
\pgfpathcurveto{\pgfqpoint{1.639906in}{2.092025in}}{\pgfqpoint{1.652397in}{2.097199in}}{\pgfqpoint{1.661605in}{2.106408in}}%
\pgfpathcurveto{\pgfqpoint{1.670814in}{2.115616in}}{\pgfqpoint{1.675988in}{2.128107in}}{\pgfqpoint{1.675988in}{2.141130in}}%
\pgfpathcurveto{\pgfqpoint{1.675988in}{2.154153in}}{\pgfqpoint{1.670814in}{2.166644in}}{\pgfqpoint{1.661605in}{2.175852in}}%
\pgfpathcurveto{\pgfqpoint{1.652397in}{2.185060in}}{\pgfqpoint{1.639906in}{2.190234in}}{\pgfqpoint{1.626883in}{2.190234in}}%
\pgfpathcurveto{\pgfqpoint{1.613860in}{2.190234in}}{\pgfqpoint{1.601369in}{2.185060in}}{\pgfqpoint{1.592161in}{2.175852in}}%
\pgfpathcurveto{\pgfqpoint{1.582952in}{2.166644in}}{\pgfqpoint{1.577778in}{2.154153in}}{\pgfqpoint{1.577778in}{2.141130in}}%
\pgfpathcurveto{\pgfqpoint{1.577778in}{2.128107in}}{\pgfqpoint{1.582952in}{2.115616in}}{\pgfqpoint{1.592161in}{2.106408in}}%
\pgfpathcurveto{\pgfqpoint{1.601369in}{2.097199in}}{\pgfqpoint{1.613860in}{2.092025in}}{\pgfqpoint{1.626883in}{2.092025in}}%
\pgfpathlineto{\pgfqpoint{1.626883in}{2.092025in}}%
\pgfpathclose%
\pgfusepath{stroke,fill}%
\end{pgfscope}%
\begin{pgfscope}%
\pgfpathrectangle{\pgfqpoint{0.786164in}{0.768110in}}{\pgfqpoint{8.851069in}{7.081890in}}%
\pgfusepath{clip}%
\pgfsetbuttcap%
\pgfsetroundjoin%
\definecolor{currentfill}{rgb}{0.136408,0.541173,0.554483}%
\pgfsetfillcolor{currentfill}%
\pgfsetfillopacity{0.700000}%
\pgfsetlinewidth{0.501875pt}%
\definecolor{currentstroke}{rgb}{1.000000,1.000000,1.000000}%
\pgfsetstrokecolor{currentstroke}%
\pgfsetstrokeopacity{0.700000}%
\pgfsetdash{}{0pt}%
\pgfpathmoveto{\pgfqpoint{1.782149in}{2.135822in}}%
\pgfpathcurveto{\pgfqpoint{1.795172in}{2.135822in}}{\pgfqpoint{1.807663in}{2.140996in}}{\pgfqpoint{1.816871in}{2.150204in}}%
\pgfpathcurveto{\pgfqpoint{1.826080in}{2.159413in}}{\pgfqpoint{1.831254in}{2.171904in}}{\pgfqpoint{1.831254in}{2.184926in}}%
\pgfpathcurveto{\pgfqpoint{1.831254in}{2.197949in}}{\pgfqpoint{1.826080in}{2.210440in}}{\pgfqpoint{1.816871in}{2.219649in}}%
\pgfpathcurveto{\pgfqpoint{1.807663in}{2.228857in}}{\pgfqpoint{1.795172in}{2.234031in}}{\pgfqpoint{1.782149in}{2.234031in}}%
\pgfpathcurveto{\pgfqpoint{1.769126in}{2.234031in}}{\pgfqpoint{1.756635in}{2.228857in}}{\pgfqpoint{1.747427in}{2.219649in}}%
\pgfpathcurveto{\pgfqpoint{1.738218in}{2.210440in}}{\pgfqpoint{1.733044in}{2.197949in}}{\pgfqpoint{1.733044in}{2.184926in}}%
\pgfpathcurveto{\pgfqpoint{1.733044in}{2.171904in}}{\pgfqpoint{1.738218in}{2.159413in}}{\pgfqpoint{1.747427in}{2.150204in}}%
\pgfpathcurveto{\pgfqpoint{1.756635in}{2.140996in}}{\pgfqpoint{1.769126in}{2.135822in}}{\pgfqpoint{1.782149in}{2.135822in}}%
\pgfpathlineto{\pgfqpoint{1.782149in}{2.135822in}}%
\pgfpathclose%
\pgfusepath{stroke,fill}%
\end{pgfscope}%
\begin{pgfscope}%
\pgfpathrectangle{\pgfqpoint{0.786164in}{0.768110in}}{\pgfqpoint{8.851069in}{7.081890in}}%
\pgfusepath{clip}%
\pgfsetbuttcap%
\pgfsetroundjoin%
\definecolor{currentfill}{rgb}{0.147607,0.511733,0.557049}%
\pgfsetfillcolor{currentfill}%
\pgfsetfillopacity{0.700000}%
\pgfsetlinewidth{0.501875pt}%
\definecolor{currentstroke}{rgb}{1.000000,1.000000,1.000000}%
\pgfsetstrokecolor{currentstroke}%
\pgfsetstrokeopacity{0.700000}%
\pgfsetdash{}{0pt}%
\pgfpathmoveto{\pgfqpoint{1.836949in}{2.223415in}}%
\pgfpathcurveto{\pgfqpoint{1.849971in}{2.223415in}}{\pgfqpoint{1.862462in}{2.228589in}}{\pgfqpoint{1.871671in}{2.237797in}}%
\pgfpathcurveto{\pgfqpoint{1.880879in}{2.247005in}}{\pgfqpoint{1.886053in}{2.259497in}}{\pgfqpoint{1.886053in}{2.272519in}}%
\pgfpathcurveto{\pgfqpoint{1.886053in}{2.285542in}}{\pgfqpoint{1.880879in}{2.298033in}}{\pgfqpoint{1.871671in}{2.307241in}}%
\pgfpathcurveto{\pgfqpoint{1.862462in}{2.316450in}}{\pgfqpoint{1.849971in}{2.321624in}}{\pgfqpoint{1.836949in}{2.321624in}}%
\pgfpathcurveto{\pgfqpoint{1.823926in}{2.321624in}}{\pgfqpoint{1.811435in}{2.316450in}}{\pgfqpoint{1.802226in}{2.307241in}}%
\pgfpathcurveto{\pgfqpoint{1.793018in}{2.298033in}}{\pgfqpoint{1.787844in}{2.285542in}}{\pgfqpoint{1.787844in}{2.272519in}}%
\pgfpathcurveto{\pgfqpoint{1.787844in}{2.259497in}}{\pgfqpoint{1.793018in}{2.247005in}}{\pgfqpoint{1.802226in}{2.237797in}}%
\pgfpathcurveto{\pgfqpoint{1.811435in}{2.228589in}}{\pgfqpoint{1.823926in}{2.223415in}}{\pgfqpoint{1.836949in}{2.223415in}}%
\pgfpathlineto{\pgfqpoint{1.836949in}{2.223415in}}%
\pgfpathclose%
\pgfusepath{stroke,fill}%
\end{pgfscope}%
\begin{pgfscope}%
\pgfpathrectangle{\pgfqpoint{0.786164in}{0.768110in}}{\pgfqpoint{8.851069in}{7.081890in}}%
\pgfusepath{clip}%
\pgfsetbuttcap%
\pgfsetroundjoin%
\definecolor{currentfill}{rgb}{0.282327,0.094955,0.417331}%
\pgfsetfillcolor{currentfill}%
\pgfsetfillopacity{0.700000}%
\pgfsetlinewidth{0.501875pt}%
\definecolor{currentstroke}{rgb}{1.000000,1.000000,1.000000}%
\pgfsetstrokecolor{currentstroke}%
\pgfsetstrokeopacity{0.700000}%
\pgfsetdash{}{0pt}%
\pgfpathmoveto{\pgfqpoint{2.120081in}{2.639481in}}%
\pgfpathcurveto{\pgfqpoint{2.133103in}{2.639481in}}{\pgfqpoint{2.145594in}{2.644655in}}{\pgfqpoint{2.154803in}{2.653864in}}%
\pgfpathcurveto{\pgfqpoint{2.164011in}{2.663072in}}{\pgfqpoint{2.169185in}{2.675563in}}{\pgfqpoint{2.169185in}{2.688586in}}%
\pgfpathcurveto{\pgfqpoint{2.169185in}{2.701608in}}{\pgfqpoint{2.164011in}{2.714100in}}{\pgfqpoint{2.154803in}{2.723308in}}%
\pgfpathcurveto{\pgfqpoint{2.145594in}{2.732516in}}{\pgfqpoint{2.133103in}{2.737690in}}{\pgfqpoint{2.120081in}{2.737690in}}%
\pgfpathcurveto{\pgfqpoint{2.107058in}{2.737690in}}{\pgfqpoint{2.094567in}{2.732516in}}{\pgfqpoint{2.085358in}{2.723308in}}%
\pgfpathcurveto{\pgfqpoint{2.076150in}{2.714100in}}{\pgfqpoint{2.070976in}{2.701608in}}{\pgfqpoint{2.070976in}{2.688586in}}%
\pgfpathcurveto{\pgfqpoint{2.070976in}{2.675563in}}{\pgfqpoint{2.076150in}{2.663072in}}{\pgfqpoint{2.085358in}{2.653864in}}%
\pgfpathcurveto{\pgfqpoint{2.094567in}{2.644655in}}{\pgfqpoint{2.107058in}{2.639481in}}{\pgfqpoint{2.120081in}{2.639481in}}%
\pgfpathlineto{\pgfqpoint{2.120081in}{2.639481in}}%
\pgfpathclose%
\pgfusepath{stroke,fill}%
\end{pgfscope}%
\begin{pgfscope}%
\pgfpathrectangle{\pgfqpoint{0.786164in}{0.768110in}}{\pgfqpoint{8.851069in}{7.081890in}}%
\pgfusepath{clip}%
\pgfsetbuttcap%
\pgfsetroundjoin%
\definecolor{currentfill}{rgb}{0.282290,0.145912,0.461510}%
\pgfsetfillcolor{currentfill}%
\pgfsetfillopacity{0.700000}%
\pgfsetlinewidth{0.501875pt}%
\definecolor{currentstroke}{rgb}{1.000000,1.000000,1.000000}%
\pgfsetstrokecolor{currentstroke}%
\pgfsetstrokeopacity{0.700000}%
\pgfsetdash{}{0pt}%
\pgfpathmoveto{\pgfqpoint{2.257080in}{2.573786in}}%
\pgfpathcurveto{\pgfqpoint{2.270103in}{2.573786in}}{\pgfqpoint{2.282594in}{2.578960in}}{\pgfqpoint{2.291802in}{2.588169in}}%
\pgfpathcurveto{\pgfqpoint{2.301010in}{2.597377in}}{\pgfqpoint{2.306184in}{2.609868in}}{\pgfqpoint{2.306184in}{2.622891in}}%
\pgfpathcurveto{\pgfqpoint{2.306184in}{2.635914in}}{\pgfqpoint{2.301010in}{2.648405in}}{\pgfqpoint{2.291802in}{2.657613in}}%
\pgfpathcurveto{\pgfqpoint{2.282594in}{2.666822in}}{\pgfqpoint{2.270103in}{2.671996in}}{\pgfqpoint{2.257080in}{2.671996in}}%
\pgfpathcurveto{\pgfqpoint{2.244057in}{2.671996in}}{\pgfqpoint{2.231566in}{2.666822in}}{\pgfqpoint{2.222358in}{2.657613in}}%
\pgfpathcurveto{\pgfqpoint{2.213149in}{2.648405in}}{\pgfqpoint{2.207975in}{2.635914in}}{\pgfqpoint{2.207975in}{2.622891in}}%
\pgfpathcurveto{\pgfqpoint{2.207975in}{2.609868in}}{\pgfqpoint{2.213149in}{2.597377in}}{\pgfqpoint{2.222358in}{2.588169in}}%
\pgfpathcurveto{\pgfqpoint{2.231566in}{2.578960in}}{\pgfqpoint{2.244057in}{2.573786in}}{\pgfqpoint{2.257080in}{2.573786in}}%
\pgfpathlineto{\pgfqpoint{2.257080in}{2.573786in}}%
\pgfpathclose%
\pgfusepath{stroke,fill}%
\end{pgfscope}%
\begin{pgfscope}%
\pgfpathrectangle{\pgfqpoint{0.786164in}{0.768110in}}{\pgfqpoint{8.851069in}{7.081890in}}%
\pgfusepath{clip}%
\pgfsetbuttcap%
\pgfsetroundjoin%
\definecolor{currentfill}{rgb}{0.281412,0.155834,0.469201}%
\pgfsetfillcolor{currentfill}%
\pgfsetfillopacity{0.700000}%
\pgfsetlinewidth{0.501875pt}%
\definecolor{currentstroke}{rgb}{1.000000,1.000000,1.000000}%
\pgfsetstrokecolor{currentstroke}%
\pgfsetstrokeopacity{0.700000}%
\pgfsetdash{}{0pt}%
\pgfpathmoveto{\pgfqpoint{2.229680in}{2.595685in}}%
\pgfpathcurveto{\pgfqpoint{2.242703in}{2.595685in}}{\pgfqpoint{2.255194in}{2.600859in}}{\pgfqpoint{2.264402in}{2.610067in}}%
\pgfpathcurveto{\pgfqpoint{2.273611in}{2.619275in}}{\pgfqpoint{2.278785in}{2.631767in}}{\pgfqpoint{2.278785in}{2.644789in}}%
\pgfpathcurveto{\pgfqpoint{2.278785in}{2.657812in}}{\pgfqpoint{2.273611in}{2.670303in}}{\pgfqpoint{2.264402in}{2.679511in}}%
\pgfpathcurveto{\pgfqpoint{2.255194in}{2.688720in}}{\pgfqpoint{2.242703in}{2.693894in}}{\pgfqpoint{2.229680in}{2.693894in}}%
\pgfpathcurveto{\pgfqpoint{2.216657in}{2.693894in}}{\pgfqpoint{2.204166in}{2.688720in}}{\pgfqpoint{2.194958in}{2.679511in}}%
\pgfpathcurveto{\pgfqpoint{2.185749in}{2.670303in}}{\pgfqpoint{2.180575in}{2.657812in}}{\pgfqpoint{2.180575in}{2.644789in}}%
\pgfpathcurveto{\pgfqpoint{2.180575in}{2.631767in}}{\pgfqpoint{2.185749in}{2.619275in}}{\pgfqpoint{2.194958in}{2.610067in}}%
\pgfpathcurveto{\pgfqpoint{2.204166in}{2.600859in}}{\pgfqpoint{2.216657in}{2.595685in}}{\pgfqpoint{2.229680in}{2.595685in}}%
\pgfpathlineto{\pgfqpoint{2.229680in}{2.595685in}}%
\pgfpathclose%
\pgfusepath{stroke,fill}%
\end{pgfscope}%
\begin{pgfscope}%
\pgfpathrectangle{\pgfqpoint{0.786164in}{0.768110in}}{\pgfqpoint{8.851069in}{7.081890in}}%
\pgfusepath{clip}%
\pgfsetbuttcap%
\pgfsetroundjoin%
\definecolor{currentfill}{rgb}{0.278826,0.175490,0.483397}%
\pgfsetfillcolor{currentfill}%
\pgfsetfillopacity{0.700000}%
\pgfsetlinewidth{0.501875pt}%
\definecolor{currentstroke}{rgb}{1.000000,1.000000,1.000000}%
\pgfsetstrokecolor{currentstroke}%
\pgfsetstrokeopacity{0.700000}%
\pgfsetdash{}{0pt}%
\pgfpathmoveto{\pgfqpoint{2.174880in}{2.551888in}}%
\pgfpathcurveto{\pgfqpoint{2.187903in}{2.551888in}}{\pgfqpoint{2.200394in}{2.557062in}}{\pgfqpoint{2.209602in}{2.566271in}}%
\pgfpathcurveto{\pgfqpoint{2.218811in}{2.575479in}}{\pgfqpoint{2.223985in}{2.587970in}}{\pgfqpoint{2.223985in}{2.600993in}}%
\pgfpathcurveto{\pgfqpoint{2.223985in}{2.614015in}}{\pgfqpoint{2.218811in}{2.626507in}}{\pgfqpoint{2.209602in}{2.635715in}}%
\pgfpathcurveto{\pgfqpoint{2.200394in}{2.644923in}}{\pgfqpoint{2.187903in}{2.650097in}}{\pgfqpoint{2.174880in}{2.650097in}}%
\pgfpathcurveto{\pgfqpoint{2.161858in}{2.650097in}}{\pgfqpoint{2.149366in}{2.644923in}}{\pgfqpoint{2.140158in}{2.635715in}}%
\pgfpathcurveto{\pgfqpoint{2.130950in}{2.626507in}}{\pgfqpoint{2.125776in}{2.614015in}}{\pgfqpoint{2.125776in}{2.600993in}}%
\pgfpathcurveto{\pgfqpoint{2.125776in}{2.587970in}}{\pgfqpoint{2.130950in}{2.575479in}}{\pgfqpoint{2.140158in}{2.566271in}}%
\pgfpathcurveto{\pgfqpoint{2.149366in}{2.557062in}}{\pgfqpoint{2.161858in}{2.551888in}}{\pgfqpoint{2.174880in}{2.551888in}}%
\pgfpathlineto{\pgfqpoint{2.174880in}{2.551888in}}%
\pgfpathclose%
\pgfusepath{stroke,fill}%
\end{pgfscope}%
\begin{pgfscope}%
\pgfpathrectangle{\pgfqpoint{0.786164in}{0.768110in}}{\pgfqpoint{8.851069in}{7.081890in}}%
\pgfusepath{clip}%
\pgfsetbuttcap%
\pgfsetroundjoin%
\definecolor{currentfill}{rgb}{0.278826,0.175490,0.483397}%
\pgfsetfillcolor{currentfill}%
\pgfsetfillopacity{0.700000}%
\pgfsetlinewidth{0.501875pt}%
\definecolor{currentstroke}{rgb}{1.000000,1.000000,1.000000}%
\pgfsetstrokecolor{currentstroke}%
\pgfsetstrokeopacity{0.700000}%
\pgfsetdash{}{0pt}%
\pgfpathmoveto{\pgfqpoint{2.165747in}{2.464295in}}%
\pgfpathcurveto{\pgfqpoint{2.178770in}{2.464295in}}{\pgfqpoint{2.191261in}{2.469469in}}{\pgfqpoint{2.200469in}{2.478678in}}%
\pgfpathcurveto{\pgfqpoint{2.209678in}{2.487886in}}{\pgfqpoint{2.214852in}{2.500377in}}{\pgfqpoint{2.214852in}{2.513400in}}%
\pgfpathcurveto{\pgfqpoint{2.214852in}{2.526423in}}{\pgfqpoint{2.209678in}{2.538914in}}{\pgfqpoint{2.200469in}{2.548122in}}%
\pgfpathcurveto{\pgfqpoint{2.191261in}{2.557331in}}{\pgfqpoint{2.178770in}{2.562504in}}{\pgfqpoint{2.165747in}{2.562504in}}%
\pgfpathcurveto{\pgfqpoint{2.152724in}{2.562504in}}{\pgfqpoint{2.140233in}{2.557331in}}{\pgfqpoint{2.131025in}{2.548122in}}%
\pgfpathcurveto{\pgfqpoint{2.121816in}{2.538914in}}{\pgfqpoint{2.116642in}{2.526423in}}{\pgfqpoint{2.116642in}{2.513400in}}%
\pgfpathcurveto{\pgfqpoint{2.116642in}{2.500377in}}{\pgfqpoint{2.121816in}{2.487886in}}{\pgfqpoint{2.131025in}{2.478678in}}%
\pgfpathcurveto{\pgfqpoint{2.140233in}{2.469469in}}{\pgfqpoint{2.152724in}{2.464295in}}{\pgfqpoint{2.165747in}{2.464295in}}%
\pgfpathlineto{\pgfqpoint{2.165747in}{2.464295in}}%
\pgfpathclose%
\pgfusepath{stroke,fill}%
\end{pgfscope}%
\begin{pgfscope}%
\pgfpathrectangle{\pgfqpoint{0.786164in}{0.768110in}}{\pgfqpoint{8.851069in}{7.081890in}}%
\pgfusepath{clip}%
\pgfsetbuttcap%
\pgfsetroundjoin%
\definecolor{currentfill}{rgb}{0.265145,0.232956,0.516599}%
\pgfsetfillcolor{currentfill}%
\pgfsetfillopacity{0.700000}%
\pgfsetlinewidth{0.501875pt}%
\definecolor{currentstroke}{rgb}{1.000000,1.000000,1.000000}%
\pgfsetstrokecolor{currentstroke}%
\pgfsetstrokeopacity{0.700000}%
\pgfsetdash{}{0pt}%
\pgfpathmoveto{\pgfqpoint{2.056148in}{2.376702in}}%
\pgfpathcurveto{\pgfqpoint{2.069170in}{2.376702in}}{\pgfqpoint{2.081661in}{2.381876in}}{\pgfqpoint{2.090870in}{2.391085in}}%
\pgfpathcurveto{\pgfqpoint{2.100078in}{2.400293in}}{\pgfqpoint{2.105252in}{2.412784in}}{\pgfqpoint{2.105252in}{2.425807in}}%
\pgfpathcurveto{\pgfqpoint{2.105252in}{2.438830in}}{\pgfqpoint{2.100078in}{2.451321in}}{\pgfqpoint{2.090870in}{2.460529in}}%
\pgfpathcurveto{\pgfqpoint{2.081661in}{2.469738in}}{\pgfqpoint{2.069170in}{2.474912in}}{\pgfqpoint{2.056148in}{2.474912in}}%
\pgfpathcurveto{\pgfqpoint{2.043125in}{2.474912in}}{\pgfqpoint{2.030634in}{2.469738in}}{\pgfqpoint{2.021425in}{2.460529in}}%
\pgfpathcurveto{\pgfqpoint{2.012217in}{2.451321in}}{\pgfqpoint{2.007043in}{2.438830in}}{\pgfqpoint{2.007043in}{2.425807in}}%
\pgfpathcurveto{\pgfqpoint{2.007043in}{2.412784in}}{\pgfqpoint{2.012217in}{2.400293in}}{\pgfqpoint{2.021425in}{2.391085in}}%
\pgfpathcurveto{\pgfqpoint{2.030634in}{2.381876in}}{\pgfqpoint{2.043125in}{2.376702in}}{\pgfqpoint{2.056148in}{2.376702in}}%
\pgfpathlineto{\pgfqpoint{2.056148in}{2.376702in}}%
\pgfpathclose%
\pgfusepath{stroke,fill}%
\end{pgfscope}%
\begin{pgfscope}%
\pgfpathrectangle{\pgfqpoint{0.786164in}{0.768110in}}{\pgfqpoint{8.851069in}{7.081890in}}%
\pgfusepath{clip}%
\pgfsetbuttcap%
\pgfsetroundjoin%
\definecolor{currentfill}{rgb}{0.257322,0.256130,0.526563}%
\pgfsetfillcolor{currentfill}%
\pgfsetfillopacity{0.700000}%
\pgfsetlinewidth{0.501875pt}%
\definecolor{currentstroke}{rgb}{1.000000,1.000000,1.000000}%
\pgfsetstrokecolor{currentstroke}%
\pgfsetstrokeopacity{0.700000}%
\pgfsetdash{}{0pt}%
\pgfpathmoveto{\pgfqpoint{2.120081in}{2.376702in}}%
\pgfpathcurveto{\pgfqpoint{2.133103in}{2.376702in}}{\pgfqpoint{2.145594in}{2.381876in}}{\pgfqpoint{2.154803in}{2.391085in}}%
\pgfpathcurveto{\pgfqpoint{2.164011in}{2.400293in}}{\pgfqpoint{2.169185in}{2.412784in}}{\pgfqpoint{2.169185in}{2.425807in}}%
\pgfpathcurveto{\pgfqpoint{2.169185in}{2.438830in}}{\pgfqpoint{2.164011in}{2.451321in}}{\pgfqpoint{2.154803in}{2.460529in}}%
\pgfpathcurveto{\pgfqpoint{2.145594in}{2.469738in}}{\pgfqpoint{2.133103in}{2.474912in}}{\pgfqpoint{2.120081in}{2.474912in}}%
\pgfpathcurveto{\pgfqpoint{2.107058in}{2.474912in}}{\pgfqpoint{2.094567in}{2.469738in}}{\pgfqpoint{2.085358in}{2.460529in}}%
\pgfpathcurveto{\pgfqpoint{2.076150in}{2.451321in}}{\pgfqpoint{2.070976in}{2.438830in}}{\pgfqpoint{2.070976in}{2.425807in}}%
\pgfpathcurveto{\pgfqpoint{2.070976in}{2.412784in}}{\pgfqpoint{2.076150in}{2.400293in}}{\pgfqpoint{2.085358in}{2.391085in}}%
\pgfpathcurveto{\pgfqpoint{2.094567in}{2.381876in}}{\pgfqpoint{2.107058in}{2.376702in}}{\pgfqpoint{2.120081in}{2.376702in}}%
\pgfpathlineto{\pgfqpoint{2.120081in}{2.376702in}}%
\pgfpathclose%
\pgfusepath{stroke,fill}%
\end{pgfscope}%
\begin{pgfscope}%
\pgfpathrectangle{\pgfqpoint{0.786164in}{0.768110in}}{\pgfqpoint{8.851069in}{7.081890in}}%
\pgfusepath{clip}%
\pgfsetbuttcap%
\pgfsetroundjoin%
\definecolor{currentfill}{rgb}{0.250425,0.274290,0.533103}%
\pgfsetfillcolor{currentfill}%
\pgfsetfillopacity{0.700000}%
\pgfsetlinewidth{0.501875pt}%
\definecolor{currentstroke}{rgb}{1.000000,1.000000,1.000000}%
\pgfsetstrokecolor{currentstroke}%
\pgfsetstrokeopacity{0.700000}%
\pgfsetdash{}{0pt}%
\pgfpathmoveto{\pgfqpoint{2.110947in}{2.376702in}}%
\pgfpathcurveto{\pgfqpoint{2.123970in}{2.376702in}}{\pgfqpoint{2.136461in}{2.381876in}}{\pgfqpoint{2.145669in}{2.391085in}}%
\pgfpathcurveto{\pgfqpoint{2.154878in}{2.400293in}}{\pgfqpoint{2.160052in}{2.412784in}}{\pgfqpoint{2.160052in}{2.425807in}}%
\pgfpathcurveto{\pgfqpoint{2.160052in}{2.438830in}}{\pgfqpoint{2.154878in}{2.451321in}}{\pgfqpoint{2.145669in}{2.460529in}}%
\pgfpathcurveto{\pgfqpoint{2.136461in}{2.469738in}}{\pgfqpoint{2.123970in}{2.474912in}}{\pgfqpoint{2.110947in}{2.474912in}}%
\pgfpathcurveto{\pgfqpoint{2.097925in}{2.474912in}}{\pgfqpoint{2.085433in}{2.469738in}}{\pgfqpoint{2.076225in}{2.460529in}}%
\pgfpathcurveto{\pgfqpoint{2.067017in}{2.451321in}}{\pgfqpoint{2.061843in}{2.438830in}}{\pgfqpoint{2.061843in}{2.425807in}}%
\pgfpathcurveto{\pgfqpoint{2.061843in}{2.412784in}}{\pgfqpoint{2.067017in}{2.400293in}}{\pgfqpoint{2.076225in}{2.391085in}}%
\pgfpathcurveto{\pgfqpoint{2.085433in}{2.381876in}}{\pgfqpoint{2.097925in}{2.376702in}}{\pgfqpoint{2.110947in}{2.376702in}}%
\pgfpathlineto{\pgfqpoint{2.110947in}{2.376702in}}%
\pgfpathclose%
\pgfusepath{stroke,fill}%
\end{pgfscope}%
\begin{pgfscope}%
\pgfpathrectangle{\pgfqpoint{0.786164in}{0.768110in}}{\pgfqpoint{8.851069in}{7.081890in}}%
\pgfusepath{clip}%
\pgfsetbuttcap%
\pgfsetroundjoin%
\definecolor{currentfill}{rgb}{0.239346,0.300855,0.540844}%
\pgfsetfillcolor{currentfill}%
\pgfsetfillopacity{0.700000}%
\pgfsetlinewidth{0.501875pt}%
\definecolor{currentstroke}{rgb}{1.000000,1.000000,1.000000}%
\pgfsetstrokecolor{currentstroke}%
\pgfsetstrokeopacity{0.700000}%
\pgfsetdash{}{0pt}%
\pgfpathmoveto{\pgfqpoint{2.001348in}{2.289109in}}%
\pgfpathcurveto{\pgfqpoint{2.014370in}{2.289109in}}{\pgfqpoint{2.026862in}{2.294283in}}{\pgfqpoint{2.036070in}{2.303492in}}%
\pgfpathcurveto{\pgfqpoint{2.045278in}{2.312700in}}{\pgfqpoint{2.050452in}{2.325191in}}{\pgfqpoint{2.050452in}{2.338214in}}%
\pgfpathcurveto{\pgfqpoint{2.050452in}{2.351237in}}{\pgfqpoint{2.045278in}{2.363728in}}{\pgfqpoint{2.036070in}{2.372936in}}%
\pgfpathcurveto{\pgfqpoint{2.026862in}{2.382145in}}{\pgfqpoint{2.014370in}{2.387319in}}{\pgfqpoint{2.001348in}{2.387319in}}%
\pgfpathcurveto{\pgfqpoint{1.988325in}{2.387319in}}{\pgfqpoint{1.975834in}{2.382145in}}{\pgfqpoint{1.966626in}{2.372936in}}%
\pgfpathcurveto{\pgfqpoint{1.957417in}{2.363728in}}{\pgfqpoint{1.952243in}{2.351237in}}{\pgfqpoint{1.952243in}{2.338214in}}%
\pgfpathcurveto{\pgfqpoint{1.952243in}{2.325191in}}{\pgfqpoint{1.957417in}{2.312700in}}{\pgfqpoint{1.966626in}{2.303492in}}%
\pgfpathcurveto{\pgfqpoint{1.975834in}{2.294283in}}{\pgfqpoint{1.988325in}{2.289109in}}{\pgfqpoint{2.001348in}{2.289109in}}%
\pgfpathlineto{\pgfqpoint{2.001348in}{2.289109in}}%
\pgfpathclose%
\pgfusepath{stroke,fill}%
\end{pgfscope}%
\begin{pgfscope}%
\pgfpathrectangle{\pgfqpoint{0.786164in}{0.768110in}}{\pgfqpoint{8.851069in}{7.081890in}}%
\pgfusepath{clip}%
\pgfsetbuttcap%
\pgfsetroundjoin%
\definecolor{currentfill}{rgb}{0.233603,0.313828,0.543914}%
\pgfsetfillcolor{currentfill}%
\pgfsetfillopacity{0.700000}%
\pgfsetlinewidth{0.501875pt}%
\definecolor{currentstroke}{rgb}{1.000000,1.000000,1.000000}%
\pgfsetstrokecolor{currentstroke}%
\pgfsetstrokeopacity{0.700000}%
\pgfsetdash{}{0pt}%
\pgfpathmoveto{\pgfqpoint{1.946548in}{2.245313in}}%
\pgfpathcurveto{\pgfqpoint{1.959571in}{2.245313in}}{\pgfqpoint{1.972062in}{2.250487in}}{\pgfqpoint{1.981270in}{2.259695in}}%
\pgfpathcurveto{\pgfqpoint{1.990479in}{2.268904in}}{\pgfqpoint{1.995653in}{2.281395in}}{\pgfqpoint{1.995653in}{2.294417in}}%
\pgfpathcurveto{\pgfqpoint{1.995653in}{2.307440in}}{\pgfqpoint{1.990479in}{2.319931in}}{\pgfqpoint{1.981270in}{2.329140in}}%
\pgfpathcurveto{\pgfqpoint{1.972062in}{2.338348in}}{\pgfqpoint{1.959571in}{2.343522in}}{\pgfqpoint{1.946548in}{2.343522in}}%
\pgfpathcurveto{\pgfqpoint{1.933525in}{2.343522in}}{\pgfqpoint{1.921034in}{2.338348in}}{\pgfqpoint{1.911826in}{2.329140in}}%
\pgfpathcurveto{\pgfqpoint{1.902617in}{2.319931in}}{\pgfqpoint{1.897443in}{2.307440in}}{\pgfqpoint{1.897443in}{2.294417in}}%
\pgfpathcurveto{\pgfqpoint{1.897443in}{2.281395in}}{\pgfqpoint{1.902617in}{2.268904in}}{\pgfqpoint{1.911826in}{2.259695in}}%
\pgfpathcurveto{\pgfqpoint{1.921034in}{2.250487in}}{\pgfqpoint{1.933525in}{2.245313in}}{\pgfqpoint{1.946548in}{2.245313in}}%
\pgfpathlineto{\pgfqpoint{1.946548in}{2.245313in}}%
\pgfpathclose%
\pgfusepath{stroke,fill}%
\end{pgfscope}%
\begin{pgfscope}%
\pgfpathrectangle{\pgfqpoint{0.786164in}{0.768110in}}{\pgfqpoint{8.851069in}{7.081890in}}%
\pgfusepath{clip}%
\pgfsetbuttcap%
\pgfsetroundjoin%
\definecolor{currentfill}{rgb}{0.244972,0.287675,0.537260}%
\pgfsetfillcolor{currentfill}%
\pgfsetfillopacity{0.700000}%
\pgfsetlinewidth{0.501875pt}%
\definecolor{currentstroke}{rgb}{1.000000,1.000000,1.000000}%
\pgfsetstrokecolor{currentstroke}%
\pgfsetstrokeopacity{0.700000}%
\pgfsetdash{}{0pt}%
\pgfpathmoveto{\pgfqpoint{1.910015in}{2.223415in}}%
\pgfpathcurveto{\pgfqpoint{1.923038in}{2.223415in}}{\pgfqpoint{1.935529in}{2.228589in}}{\pgfqpoint{1.944737in}{2.237797in}}%
\pgfpathcurveto{\pgfqpoint{1.953946in}{2.247005in}}{\pgfqpoint{1.959120in}{2.259497in}}{\pgfqpoint{1.959120in}{2.272519in}}%
\pgfpathcurveto{\pgfqpoint{1.959120in}{2.285542in}}{\pgfqpoint{1.953946in}{2.298033in}}{\pgfqpoint{1.944737in}{2.307241in}}%
\pgfpathcurveto{\pgfqpoint{1.935529in}{2.316450in}}{\pgfqpoint{1.923038in}{2.321624in}}{\pgfqpoint{1.910015in}{2.321624in}}%
\pgfpathcurveto{\pgfqpoint{1.896992in}{2.321624in}}{\pgfqpoint{1.884501in}{2.316450in}}{\pgfqpoint{1.875293in}{2.307241in}}%
\pgfpathcurveto{\pgfqpoint{1.866084in}{2.298033in}}{\pgfqpoint{1.860910in}{2.285542in}}{\pgfqpoint{1.860910in}{2.272519in}}%
\pgfpathcurveto{\pgfqpoint{1.860910in}{2.259497in}}{\pgfqpoint{1.866084in}{2.247005in}}{\pgfqpoint{1.875293in}{2.237797in}}%
\pgfpathcurveto{\pgfqpoint{1.884501in}{2.228589in}}{\pgfqpoint{1.896992in}{2.223415in}}{\pgfqpoint{1.910015in}{2.223415in}}%
\pgfpathlineto{\pgfqpoint{1.910015in}{2.223415in}}%
\pgfpathclose%
\pgfusepath{stroke,fill}%
\end{pgfscope}%
\begin{pgfscope}%
\pgfpathrectangle{\pgfqpoint{0.786164in}{0.768110in}}{\pgfqpoint{8.851069in}{7.081890in}}%
\pgfusepath{clip}%
\pgfsetbuttcap%
\pgfsetroundjoin%
\definecolor{currentfill}{rgb}{0.246811,0.283237,0.535941}%
\pgfsetfillcolor{currentfill}%
\pgfsetfillopacity{0.700000}%
\pgfsetlinewidth{0.501875pt}%
\definecolor{currentstroke}{rgb}{1.000000,1.000000,1.000000}%
\pgfsetstrokecolor{currentstroke}%
\pgfsetstrokeopacity{0.700000}%
\pgfsetdash{}{0pt}%
\pgfpathmoveto{\pgfqpoint{2.037881in}{2.289109in}}%
\pgfpathcurveto{\pgfqpoint{2.050904in}{2.289109in}}{\pgfqpoint{2.063395in}{2.294283in}}{\pgfqpoint{2.072603in}{2.303492in}}%
\pgfpathcurveto{\pgfqpoint{2.081812in}{2.312700in}}{\pgfqpoint{2.086986in}{2.325191in}}{\pgfqpoint{2.086986in}{2.338214in}}%
\pgfpathcurveto{\pgfqpoint{2.086986in}{2.351237in}}{\pgfqpoint{2.081812in}{2.363728in}}{\pgfqpoint{2.072603in}{2.372936in}}%
\pgfpathcurveto{\pgfqpoint{2.063395in}{2.382145in}}{\pgfqpoint{2.050904in}{2.387319in}}{\pgfqpoint{2.037881in}{2.387319in}}%
\pgfpathcurveto{\pgfqpoint{2.024858in}{2.387319in}}{\pgfqpoint{2.012367in}{2.382145in}}{\pgfqpoint{2.003159in}{2.372936in}}%
\pgfpathcurveto{\pgfqpoint{1.993950in}{2.363728in}}{\pgfqpoint{1.988776in}{2.351237in}}{\pgfqpoint{1.988776in}{2.338214in}}%
\pgfpathcurveto{\pgfqpoint{1.988776in}{2.325191in}}{\pgfqpoint{1.993950in}{2.312700in}}{\pgfqpoint{2.003159in}{2.303492in}}%
\pgfpathcurveto{\pgfqpoint{2.012367in}{2.294283in}}{\pgfqpoint{2.024858in}{2.289109in}}{\pgfqpoint{2.037881in}{2.289109in}}%
\pgfpathlineto{\pgfqpoint{2.037881in}{2.289109in}}%
\pgfpathclose%
\pgfusepath{stroke,fill}%
\end{pgfscope}%
\begin{pgfscope}%
\pgfpathrectangle{\pgfqpoint{0.786164in}{0.768110in}}{\pgfqpoint{8.851069in}{7.081890in}}%
\pgfusepath{clip}%
\pgfsetbuttcap%
\pgfsetroundjoin%
\definecolor{currentfill}{rgb}{0.246811,0.283237,0.535941}%
\pgfsetfillcolor{currentfill}%
\pgfsetfillopacity{0.700000}%
\pgfsetlinewidth{0.501875pt}%
\definecolor{currentstroke}{rgb}{1.000000,1.000000,1.000000}%
\pgfsetstrokecolor{currentstroke}%
\pgfsetstrokeopacity{0.700000}%
\pgfsetdash{}{0pt}%
\pgfpathmoveto{\pgfqpoint{2.074414in}{2.332906in}}%
\pgfpathcurveto{\pgfqpoint{2.087437in}{2.332906in}}{\pgfqpoint{2.099928in}{2.338080in}}{\pgfqpoint{2.109136in}{2.347288in}}%
\pgfpathcurveto{\pgfqpoint{2.118345in}{2.356497in}}{\pgfqpoint{2.123519in}{2.368988in}}{\pgfqpoint{2.123519in}{2.382010in}}%
\pgfpathcurveto{\pgfqpoint{2.123519in}{2.395033in}}{\pgfqpoint{2.118345in}{2.407524in}}{\pgfqpoint{2.109136in}{2.416733in}}%
\pgfpathcurveto{\pgfqpoint{2.099928in}{2.425941in}}{\pgfqpoint{2.087437in}{2.431115in}}{\pgfqpoint{2.074414in}{2.431115in}}%
\pgfpathcurveto{\pgfqpoint{2.061391in}{2.431115in}}{\pgfqpoint{2.048900in}{2.425941in}}{\pgfqpoint{2.039692in}{2.416733in}}%
\pgfpathcurveto{\pgfqpoint{2.030483in}{2.407524in}}{\pgfqpoint{2.025309in}{2.395033in}}{\pgfqpoint{2.025309in}{2.382010in}}%
\pgfpathcurveto{\pgfqpoint{2.025309in}{2.368988in}}{\pgfqpoint{2.030483in}{2.356497in}}{\pgfqpoint{2.039692in}{2.347288in}}%
\pgfpathcurveto{\pgfqpoint{2.048900in}{2.338080in}}{\pgfqpoint{2.061391in}{2.332906in}}{\pgfqpoint{2.074414in}{2.332906in}}%
\pgfpathlineto{\pgfqpoint{2.074414in}{2.332906in}}%
\pgfpathclose%
\pgfusepath{stroke,fill}%
\end{pgfscope}%
\begin{pgfscope}%
\pgfpathrectangle{\pgfqpoint{0.786164in}{0.768110in}}{\pgfqpoint{8.851069in}{7.081890in}}%
\pgfusepath{clip}%
\pgfsetbuttcap%
\pgfsetroundjoin%
\definecolor{currentfill}{rgb}{0.252194,0.269783,0.531579}%
\pgfsetfillcolor{currentfill}%
\pgfsetfillopacity{0.700000}%
\pgfsetlinewidth{0.501875pt}%
\definecolor{currentstroke}{rgb}{1.000000,1.000000,1.000000}%
\pgfsetstrokecolor{currentstroke}%
\pgfsetstrokeopacity{0.700000}%
\pgfsetdash{}{0pt}%
\pgfpathmoveto{\pgfqpoint{2.156614in}{2.376702in}}%
\pgfpathcurveto{\pgfqpoint{2.169636in}{2.376702in}}{\pgfqpoint{2.182127in}{2.381876in}}{\pgfqpoint{2.191336in}{2.391085in}}%
\pgfpathcurveto{\pgfqpoint{2.200544in}{2.400293in}}{\pgfqpoint{2.205718in}{2.412784in}}{\pgfqpoint{2.205718in}{2.425807in}}%
\pgfpathcurveto{\pgfqpoint{2.205718in}{2.438830in}}{\pgfqpoint{2.200544in}{2.451321in}}{\pgfqpoint{2.191336in}{2.460529in}}%
\pgfpathcurveto{\pgfqpoint{2.182127in}{2.469738in}}{\pgfqpoint{2.169636in}{2.474912in}}{\pgfqpoint{2.156614in}{2.474912in}}%
\pgfpathcurveto{\pgfqpoint{2.143591in}{2.474912in}}{\pgfqpoint{2.131100in}{2.469738in}}{\pgfqpoint{2.121891in}{2.460529in}}%
\pgfpathcurveto{\pgfqpoint{2.112683in}{2.451321in}}{\pgfqpoint{2.107509in}{2.438830in}}{\pgfqpoint{2.107509in}{2.425807in}}%
\pgfpathcurveto{\pgfqpoint{2.107509in}{2.412784in}}{\pgfqpoint{2.112683in}{2.400293in}}{\pgfqpoint{2.121891in}{2.391085in}}%
\pgfpathcurveto{\pgfqpoint{2.131100in}{2.381876in}}{\pgfqpoint{2.143591in}{2.376702in}}{\pgfqpoint{2.156614in}{2.376702in}}%
\pgfpathlineto{\pgfqpoint{2.156614in}{2.376702in}}%
\pgfpathclose%
\pgfusepath{stroke,fill}%
\end{pgfscope}%
\begin{pgfscope}%
\pgfpathrectangle{\pgfqpoint{0.786164in}{0.768110in}}{\pgfqpoint{8.851069in}{7.081890in}}%
\pgfusepath{clip}%
\pgfsetbuttcap%
\pgfsetroundjoin%
\definecolor{currentfill}{rgb}{0.258965,0.251537,0.524736}%
\pgfsetfillcolor{currentfill}%
\pgfsetfillopacity{0.700000}%
\pgfsetlinewidth{0.501875pt}%
\definecolor{currentstroke}{rgb}{1.000000,1.000000,1.000000}%
\pgfsetstrokecolor{currentstroke}%
\pgfsetstrokeopacity{0.700000}%
\pgfsetdash{}{0pt}%
\pgfpathmoveto{\pgfqpoint{2.156614in}{2.376702in}}%
\pgfpathcurveto{\pgfqpoint{2.169636in}{2.376702in}}{\pgfqpoint{2.182127in}{2.381876in}}{\pgfqpoint{2.191336in}{2.391085in}}%
\pgfpathcurveto{\pgfqpoint{2.200544in}{2.400293in}}{\pgfqpoint{2.205718in}{2.412784in}}{\pgfqpoint{2.205718in}{2.425807in}}%
\pgfpathcurveto{\pgfqpoint{2.205718in}{2.438830in}}{\pgfqpoint{2.200544in}{2.451321in}}{\pgfqpoint{2.191336in}{2.460529in}}%
\pgfpathcurveto{\pgfqpoint{2.182127in}{2.469738in}}{\pgfqpoint{2.169636in}{2.474912in}}{\pgfqpoint{2.156614in}{2.474912in}}%
\pgfpathcurveto{\pgfqpoint{2.143591in}{2.474912in}}{\pgfqpoint{2.131100in}{2.469738in}}{\pgfqpoint{2.121891in}{2.460529in}}%
\pgfpathcurveto{\pgfqpoint{2.112683in}{2.451321in}}{\pgfqpoint{2.107509in}{2.438830in}}{\pgfqpoint{2.107509in}{2.425807in}}%
\pgfpathcurveto{\pgfqpoint{2.107509in}{2.412784in}}{\pgfqpoint{2.112683in}{2.400293in}}{\pgfqpoint{2.121891in}{2.391085in}}%
\pgfpathcurveto{\pgfqpoint{2.131100in}{2.381876in}}{\pgfqpoint{2.143591in}{2.376702in}}{\pgfqpoint{2.156614in}{2.376702in}}%
\pgfpathlineto{\pgfqpoint{2.156614in}{2.376702in}}%
\pgfpathclose%
\pgfusepath{stroke,fill}%
\end{pgfscope}%
\begin{pgfscope}%
\pgfpathrectangle{\pgfqpoint{0.786164in}{0.768110in}}{\pgfqpoint{8.851069in}{7.081890in}}%
\pgfusepath{clip}%
\pgfsetbuttcap%
\pgfsetroundjoin%
\definecolor{currentfill}{rgb}{0.258965,0.251537,0.524736}%
\pgfsetfillcolor{currentfill}%
\pgfsetfillopacity{0.700000}%
\pgfsetlinewidth{0.501875pt}%
\definecolor{currentstroke}{rgb}{1.000000,1.000000,1.000000}%
\pgfsetstrokecolor{currentstroke}%
\pgfsetstrokeopacity{0.700000}%
\pgfsetdash{}{0pt}%
\pgfpathmoveto{\pgfqpoint{2.165747in}{2.376702in}}%
\pgfpathcurveto{\pgfqpoint{2.178770in}{2.376702in}}{\pgfqpoint{2.191261in}{2.381876in}}{\pgfqpoint{2.200469in}{2.391085in}}%
\pgfpathcurveto{\pgfqpoint{2.209678in}{2.400293in}}{\pgfqpoint{2.214852in}{2.412784in}}{\pgfqpoint{2.214852in}{2.425807in}}%
\pgfpathcurveto{\pgfqpoint{2.214852in}{2.438830in}}{\pgfqpoint{2.209678in}{2.451321in}}{\pgfqpoint{2.200469in}{2.460529in}}%
\pgfpathcurveto{\pgfqpoint{2.191261in}{2.469738in}}{\pgfqpoint{2.178770in}{2.474912in}}{\pgfqpoint{2.165747in}{2.474912in}}%
\pgfpathcurveto{\pgfqpoint{2.152724in}{2.474912in}}{\pgfqpoint{2.140233in}{2.469738in}}{\pgfqpoint{2.131025in}{2.460529in}}%
\pgfpathcurveto{\pgfqpoint{2.121816in}{2.451321in}}{\pgfqpoint{2.116642in}{2.438830in}}{\pgfqpoint{2.116642in}{2.425807in}}%
\pgfpathcurveto{\pgfqpoint{2.116642in}{2.412784in}}{\pgfqpoint{2.121816in}{2.400293in}}{\pgfqpoint{2.131025in}{2.391085in}}%
\pgfpathcurveto{\pgfqpoint{2.140233in}{2.381876in}}{\pgfqpoint{2.152724in}{2.376702in}}{\pgfqpoint{2.165747in}{2.376702in}}%
\pgfpathlineto{\pgfqpoint{2.165747in}{2.376702in}}%
\pgfpathclose%
\pgfusepath{stroke,fill}%
\end{pgfscope}%
\begin{pgfscope}%
\pgfpathrectangle{\pgfqpoint{0.786164in}{0.768110in}}{\pgfqpoint{8.851069in}{7.081890in}}%
\pgfusepath{clip}%
\pgfsetbuttcap%
\pgfsetroundjoin%
\definecolor{currentfill}{rgb}{0.250425,0.274290,0.533103}%
\pgfsetfillcolor{currentfill}%
\pgfsetfillopacity{0.700000}%
\pgfsetlinewidth{0.501875pt}%
\definecolor{currentstroke}{rgb}{1.000000,1.000000,1.000000}%
\pgfsetstrokecolor{currentstroke}%
\pgfsetstrokeopacity{0.700000}%
\pgfsetdash{}{0pt}%
\pgfpathmoveto{\pgfqpoint{2.110947in}{2.267211in}}%
\pgfpathcurveto{\pgfqpoint{2.123970in}{2.267211in}}{\pgfqpoint{2.136461in}{2.272385in}}{\pgfqpoint{2.145669in}{2.281593in}}%
\pgfpathcurveto{\pgfqpoint{2.154878in}{2.290802in}}{\pgfqpoint{2.160052in}{2.303293in}}{\pgfqpoint{2.160052in}{2.316316in}}%
\pgfpathcurveto{\pgfqpoint{2.160052in}{2.329338in}}{\pgfqpoint{2.154878in}{2.341829in}}{\pgfqpoint{2.145669in}{2.351038in}}%
\pgfpathcurveto{\pgfqpoint{2.136461in}{2.360246in}}{\pgfqpoint{2.123970in}{2.365420in}}{\pgfqpoint{2.110947in}{2.365420in}}%
\pgfpathcurveto{\pgfqpoint{2.097925in}{2.365420in}}{\pgfqpoint{2.085433in}{2.360246in}}{\pgfqpoint{2.076225in}{2.351038in}}%
\pgfpathcurveto{\pgfqpoint{2.067017in}{2.341829in}}{\pgfqpoint{2.061843in}{2.329338in}}{\pgfqpoint{2.061843in}{2.316316in}}%
\pgfpathcurveto{\pgfqpoint{2.061843in}{2.303293in}}{\pgfqpoint{2.067017in}{2.290802in}}{\pgfqpoint{2.076225in}{2.281593in}}%
\pgfpathcurveto{\pgfqpoint{2.085433in}{2.272385in}}{\pgfqpoint{2.097925in}{2.267211in}}{\pgfqpoint{2.110947in}{2.267211in}}%
\pgfpathlineto{\pgfqpoint{2.110947in}{2.267211in}}%
\pgfpathclose%
\pgfusepath{stroke,fill}%
\end{pgfscope}%
\begin{pgfscope}%
\pgfpathrectangle{\pgfqpoint{0.786164in}{0.768110in}}{\pgfqpoint{8.851069in}{7.081890in}}%
\pgfusepath{clip}%
\pgfsetbuttcap%
\pgfsetroundjoin%
\definecolor{currentfill}{rgb}{0.248629,0.278775,0.534556}%
\pgfsetfillcolor{currentfill}%
\pgfsetfillopacity{0.700000}%
\pgfsetlinewidth{0.501875pt}%
\definecolor{currentstroke}{rgb}{1.000000,1.000000,1.000000}%
\pgfsetstrokecolor{currentstroke}%
\pgfsetstrokeopacity{0.700000}%
\pgfsetdash{}{0pt}%
\pgfpathmoveto{\pgfqpoint{2.220547in}{2.311008in}}%
\pgfpathcurveto{\pgfqpoint{2.233569in}{2.311008in}}{\pgfqpoint{2.246060in}{2.316182in}}{\pgfqpoint{2.255269in}{2.325390in}}%
\pgfpathcurveto{\pgfqpoint{2.264477in}{2.334598in}}{\pgfqpoint{2.269651in}{2.347089in}}{\pgfqpoint{2.269651in}{2.360112in}}%
\pgfpathcurveto{\pgfqpoint{2.269651in}{2.373135in}}{\pgfqpoint{2.264477in}{2.385626in}}{\pgfqpoint{2.255269in}{2.394834in}}%
\pgfpathcurveto{\pgfqpoint{2.246060in}{2.404043in}}{\pgfqpoint{2.233569in}{2.409217in}}{\pgfqpoint{2.220547in}{2.409217in}}%
\pgfpathcurveto{\pgfqpoint{2.207524in}{2.409217in}}{\pgfqpoint{2.195033in}{2.404043in}}{\pgfqpoint{2.185824in}{2.394834in}}%
\pgfpathcurveto{\pgfqpoint{2.176616in}{2.385626in}}{\pgfqpoint{2.171442in}{2.373135in}}{\pgfqpoint{2.171442in}{2.360112in}}%
\pgfpathcurveto{\pgfqpoint{2.171442in}{2.347089in}}{\pgfqpoint{2.176616in}{2.334598in}}{\pgfqpoint{2.185824in}{2.325390in}}%
\pgfpathcurveto{\pgfqpoint{2.195033in}{2.316182in}}{\pgfqpoint{2.207524in}{2.311008in}}{\pgfqpoint{2.220547in}{2.311008in}}%
\pgfpathlineto{\pgfqpoint{2.220547in}{2.311008in}}%
\pgfpathclose%
\pgfusepath{stroke,fill}%
\end{pgfscope}%
\begin{pgfscope}%
\pgfpathrectangle{\pgfqpoint{0.786164in}{0.768110in}}{\pgfqpoint{8.851069in}{7.081890in}}%
\pgfusepath{clip}%
\pgfsetbuttcap%
\pgfsetroundjoin%
\definecolor{currentfill}{rgb}{0.241237,0.296485,0.539709}%
\pgfsetfillcolor{currentfill}%
\pgfsetfillopacity{0.700000}%
\pgfsetlinewidth{0.501875pt}%
\definecolor{currentstroke}{rgb}{1.000000,1.000000,1.000000}%
\pgfsetstrokecolor{currentstroke}%
\pgfsetstrokeopacity{0.700000}%
\pgfsetdash{}{0pt}%
\pgfpathmoveto{\pgfqpoint{2.138347in}{2.245313in}}%
\pgfpathcurveto{\pgfqpoint{2.151370in}{2.245313in}}{\pgfqpoint{2.163861in}{2.250487in}}{\pgfqpoint{2.173069in}{2.259695in}}%
\pgfpathcurveto{\pgfqpoint{2.182278in}{2.268904in}}{\pgfqpoint{2.187452in}{2.281395in}}{\pgfqpoint{2.187452in}{2.294417in}}%
\pgfpathcurveto{\pgfqpoint{2.187452in}{2.307440in}}{\pgfqpoint{2.182278in}{2.319931in}}{\pgfqpoint{2.173069in}{2.329140in}}%
\pgfpathcurveto{\pgfqpoint{2.163861in}{2.338348in}}{\pgfqpoint{2.151370in}{2.343522in}}{\pgfqpoint{2.138347in}{2.343522in}}%
\pgfpathcurveto{\pgfqpoint{2.125324in}{2.343522in}}{\pgfqpoint{2.112833in}{2.338348in}}{\pgfqpoint{2.103625in}{2.329140in}}%
\pgfpathcurveto{\pgfqpoint{2.094416in}{2.319931in}}{\pgfqpoint{2.089242in}{2.307440in}}{\pgfqpoint{2.089242in}{2.294417in}}%
\pgfpathcurveto{\pgfqpoint{2.089242in}{2.281395in}}{\pgfqpoint{2.094416in}{2.268904in}}{\pgfqpoint{2.103625in}{2.259695in}}%
\pgfpathcurveto{\pgfqpoint{2.112833in}{2.250487in}}{\pgfqpoint{2.125324in}{2.245313in}}{\pgfqpoint{2.138347in}{2.245313in}}%
\pgfpathlineto{\pgfqpoint{2.138347in}{2.245313in}}%
\pgfpathclose%
\pgfusepath{stroke,fill}%
\end{pgfscope}%
\begin{pgfscope}%
\pgfpathrectangle{\pgfqpoint{0.786164in}{0.768110in}}{\pgfqpoint{8.851069in}{7.081890in}}%
\pgfusepath{clip}%
\pgfsetbuttcap%
\pgfsetroundjoin%
\definecolor{currentfill}{rgb}{0.277018,0.050344,0.375715}%
\pgfsetfillcolor{currentfill}%
\pgfsetfillopacity{0.700000}%
\pgfsetlinewidth{0.501875pt}%
\definecolor{currentstroke}{rgb}{1.000000,1.000000,1.000000}%
\pgfsetstrokecolor{currentstroke}%
\pgfsetstrokeopacity{0.700000}%
\pgfsetdash{}{0pt}%
\pgfpathmoveto{\pgfqpoint{3.188675in}{5.201575in}}%
\pgfpathcurveto{\pgfqpoint{3.201698in}{5.201575in}}{\pgfqpoint{3.214189in}{5.206749in}}{\pgfqpoint{3.223397in}{5.215957in}}%
\pgfpathcurveto{\pgfqpoint{3.232606in}{5.225166in}}{\pgfqpoint{3.237780in}{5.237657in}}{\pgfqpoint{3.237780in}{5.250679in}}%
\pgfpathcurveto{\pgfqpoint{3.237780in}{5.263702in}}{\pgfqpoint{3.232606in}{5.276193in}}{\pgfqpoint{3.223397in}{5.285402in}}%
\pgfpathcurveto{\pgfqpoint{3.214189in}{5.294610in}}{\pgfqpoint{3.201698in}{5.299784in}}{\pgfqpoint{3.188675in}{5.299784in}}%
\pgfpathcurveto{\pgfqpoint{3.175652in}{5.299784in}}{\pgfqpoint{3.163161in}{5.294610in}}{\pgfqpoint{3.153953in}{5.285402in}}%
\pgfpathcurveto{\pgfqpoint{3.144744in}{5.276193in}}{\pgfqpoint{3.139571in}{5.263702in}}{\pgfqpoint{3.139571in}{5.250679in}}%
\pgfpathcurveto{\pgfqpoint{3.139571in}{5.237657in}}{\pgfqpoint{3.144744in}{5.225166in}}{\pgfqpoint{3.153953in}{5.215957in}}%
\pgfpathcurveto{\pgfqpoint{3.163161in}{5.206749in}}{\pgfqpoint{3.175652in}{5.201575in}}{\pgfqpoint{3.188675in}{5.201575in}}%
\pgfpathlineto{\pgfqpoint{3.188675in}{5.201575in}}%
\pgfpathclose%
\pgfusepath{stroke,fill}%
\end{pgfscope}%
\begin{pgfscope}%
\pgfpathrectangle{\pgfqpoint{0.786164in}{0.768110in}}{\pgfqpoint{8.851069in}{7.081890in}}%
\pgfusepath{clip}%
\pgfsetbuttcap%
\pgfsetroundjoin%
\definecolor{currentfill}{rgb}{0.278791,0.062145,0.386592}%
\pgfsetfillcolor{currentfill}%
\pgfsetfillopacity{0.700000}%
\pgfsetlinewidth{0.501875pt}%
\definecolor{currentstroke}{rgb}{1.000000,1.000000,1.000000}%
\pgfsetstrokecolor{currentstroke}%
\pgfsetstrokeopacity{0.700000}%
\pgfsetdash{}{0pt}%
\pgfpathmoveto{\pgfqpoint{3.152142in}{5.223473in}}%
\pgfpathcurveto{\pgfqpoint{3.165165in}{5.223473in}}{\pgfqpoint{3.177656in}{5.228647in}}{\pgfqpoint{3.186864in}{5.237855in}}%
\pgfpathcurveto{\pgfqpoint{3.196073in}{5.247064in}}{\pgfqpoint{3.201247in}{5.259555in}}{\pgfqpoint{3.201247in}{5.272578in}}%
\pgfpathcurveto{\pgfqpoint{3.201247in}{5.285600in}}{\pgfqpoint{3.196073in}{5.298091in}}{\pgfqpoint{3.186864in}{5.307300in}}%
\pgfpathcurveto{\pgfqpoint{3.177656in}{5.316508in}}{\pgfqpoint{3.165165in}{5.321682in}}{\pgfqpoint{3.152142in}{5.321682in}}%
\pgfpathcurveto{\pgfqpoint{3.139119in}{5.321682in}}{\pgfqpoint{3.126628in}{5.316508in}}{\pgfqpoint{3.117420in}{5.307300in}}%
\pgfpathcurveto{\pgfqpoint{3.108211in}{5.298091in}}{\pgfqpoint{3.103037in}{5.285600in}}{\pgfqpoint{3.103037in}{5.272578in}}%
\pgfpathcurveto{\pgfqpoint{3.103037in}{5.259555in}}{\pgfqpoint{3.108211in}{5.247064in}}{\pgfqpoint{3.117420in}{5.237855in}}%
\pgfpathcurveto{\pgfqpoint{3.126628in}{5.228647in}}{\pgfqpoint{3.139119in}{5.223473in}}{\pgfqpoint{3.152142in}{5.223473in}}%
\pgfpathlineto{\pgfqpoint{3.152142in}{5.223473in}}%
\pgfpathclose%
\pgfusepath{stroke,fill}%
\end{pgfscope}%
\begin{pgfscope}%
\pgfpathrectangle{\pgfqpoint{0.786164in}{0.768110in}}{\pgfqpoint{8.851069in}{7.081890in}}%
\pgfusepath{clip}%
\pgfsetbuttcap%
\pgfsetroundjoin%
\definecolor{currentfill}{rgb}{0.279566,0.067836,0.391917}%
\pgfsetfillcolor{currentfill}%
\pgfsetfillopacity{0.700000}%
\pgfsetlinewidth{0.501875pt}%
\definecolor{currentstroke}{rgb}{1.000000,1.000000,1.000000}%
\pgfsetstrokecolor{currentstroke}%
\pgfsetstrokeopacity{0.700000}%
\pgfsetdash{}{0pt}%
\pgfpathmoveto{\pgfqpoint{3.106476in}{5.113982in}}%
\pgfpathcurveto{\pgfqpoint{3.119498in}{5.113982in}}{\pgfqpoint{3.131989in}{5.119156in}}{\pgfqpoint{3.141198in}{5.128364in}}%
\pgfpathcurveto{\pgfqpoint{3.150406in}{5.137573in}}{\pgfqpoint{3.155580in}{5.150064in}}{\pgfqpoint{3.155580in}{5.163086in}}%
\pgfpathcurveto{\pgfqpoint{3.155580in}{5.176109in}}{\pgfqpoint{3.150406in}{5.188600in}}{\pgfqpoint{3.141198in}{5.197809in}}%
\pgfpathcurveto{\pgfqpoint{3.131989in}{5.207017in}}{\pgfqpoint{3.119498in}{5.212191in}}{\pgfqpoint{3.106476in}{5.212191in}}%
\pgfpathcurveto{\pgfqpoint{3.093453in}{5.212191in}}{\pgfqpoint{3.080962in}{5.207017in}}{\pgfqpoint{3.071753in}{5.197809in}}%
\pgfpathcurveto{\pgfqpoint{3.062545in}{5.188600in}}{\pgfqpoint{3.057371in}{5.176109in}}{\pgfqpoint{3.057371in}{5.163086in}}%
\pgfpathcurveto{\pgfqpoint{3.057371in}{5.150064in}}{\pgfqpoint{3.062545in}{5.137573in}}{\pgfqpoint{3.071753in}{5.128364in}}%
\pgfpathcurveto{\pgfqpoint{3.080962in}{5.119156in}}{\pgfqpoint{3.093453in}{5.113982in}}{\pgfqpoint{3.106476in}{5.113982in}}%
\pgfpathlineto{\pgfqpoint{3.106476in}{5.113982in}}%
\pgfpathclose%
\pgfusepath{stroke,fill}%
\end{pgfscope}%
\begin{pgfscope}%
\pgfpathrectangle{\pgfqpoint{0.786164in}{0.768110in}}{\pgfqpoint{8.851069in}{7.081890in}}%
\pgfusepath{clip}%
\pgfsetbuttcap%
\pgfsetroundjoin%
\definecolor{currentfill}{rgb}{0.280894,0.078907,0.402329}%
\pgfsetfillcolor{currentfill}%
\pgfsetfillopacity{0.700000}%
\pgfsetlinewidth{0.501875pt}%
\definecolor{currentstroke}{rgb}{1.000000,1.000000,1.000000}%
\pgfsetstrokecolor{currentstroke}%
\pgfsetstrokeopacity{0.700000}%
\pgfsetdash{}{0pt}%
\pgfpathmoveto{\pgfqpoint{3.188675in}{4.894999in}}%
\pgfpathcurveto{\pgfqpoint{3.201698in}{4.894999in}}{\pgfqpoint{3.214189in}{4.900173in}}{\pgfqpoint{3.223397in}{4.909382in}}%
\pgfpathcurveto{\pgfqpoint{3.232606in}{4.918590in}}{\pgfqpoint{3.237780in}{4.931081in}}{\pgfqpoint{3.237780in}{4.944104in}}%
\pgfpathcurveto{\pgfqpoint{3.237780in}{4.957127in}}{\pgfqpoint{3.232606in}{4.969618in}}{\pgfqpoint{3.223397in}{4.978826in}}%
\pgfpathcurveto{\pgfqpoint{3.214189in}{4.988035in}}{\pgfqpoint{3.201698in}{4.993209in}}{\pgfqpoint{3.188675in}{4.993209in}}%
\pgfpathcurveto{\pgfqpoint{3.175652in}{4.993209in}}{\pgfqpoint{3.163161in}{4.988035in}}{\pgfqpoint{3.153953in}{4.978826in}}%
\pgfpathcurveto{\pgfqpoint{3.144744in}{4.969618in}}{\pgfqpoint{3.139571in}{4.957127in}}{\pgfqpoint{3.139571in}{4.944104in}}%
\pgfpathcurveto{\pgfqpoint{3.139571in}{4.931081in}}{\pgfqpoint{3.144744in}{4.918590in}}{\pgfqpoint{3.153953in}{4.909382in}}%
\pgfpathcurveto{\pgfqpoint{3.163161in}{4.900173in}}{\pgfqpoint{3.175652in}{4.894999in}}{\pgfqpoint{3.188675in}{4.894999in}}%
\pgfpathlineto{\pgfqpoint{3.188675in}{4.894999in}}%
\pgfpathclose%
\pgfusepath{stroke,fill}%
\end{pgfscope}%
\begin{pgfscope}%
\pgfpathrectangle{\pgfqpoint{0.786164in}{0.768110in}}{\pgfqpoint{8.851069in}{7.081890in}}%
\pgfusepath{clip}%
\pgfsetbuttcap%
\pgfsetroundjoin%
\definecolor{currentfill}{rgb}{0.281924,0.089666,0.412415}%
\pgfsetfillcolor{currentfill}%
\pgfsetfillopacity{0.700000}%
\pgfsetlinewidth{0.501875pt}%
\definecolor{currentstroke}{rgb}{1.000000,1.000000,1.000000}%
\pgfsetstrokecolor{currentstroke}%
\pgfsetstrokeopacity{0.700000}%
\pgfsetdash{}{0pt}%
\pgfpathmoveto{\pgfqpoint{3.060809in}{4.741712in}}%
\pgfpathcurveto{\pgfqpoint{3.073832in}{4.741712in}}{\pgfqpoint{3.086323in}{4.746886in}}{\pgfqpoint{3.095531in}{4.756094in}}%
\pgfpathcurveto{\pgfqpoint{3.104740in}{4.765303in}}{\pgfqpoint{3.109914in}{4.777794in}}{\pgfqpoint{3.109914in}{4.790816in}}%
\pgfpathcurveto{\pgfqpoint{3.109914in}{4.803839in}}{\pgfqpoint{3.104740in}{4.816330in}}{\pgfqpoint{3.095531in}{4.825539in}}%
\pgfpathcurveto{\pgfqpoint{3.086323in}{4.834747in}}{\pgfqpoint{3.073832in}{4.839921in}}{\pgfqpoint{3.060809in}{4.839921in}}%
\pgfpathcurveto{\pgfqpoint{3.047786in}{4.839921in}}{\pgfqpoint{3.035295in}{4.834747in}}{\pgfqpoint{3.026087in}{4.825539in}}%
\pgfpathcurveto{\pgfqpoint{3.016878in}{4.816330in}}{\pgfqpoint{3.011704in}{4.803839in}}{\pgfqpoint{3.011704in}{4.790816in}}%
\pgfpathcurveto{\pgfqpoint{3.011704in}{4.777794in}}{\pgfqpoint{3.016878in}{4.765303in}}{\pgfqpoint{3.026087in}{4.756094in}}%
\pgfpathcurveto{\pgfqpoint{3.035295in}{4.746886in}}{\pgfqpoint{3.047786in}{4.741712in}}{\pgfqpoint{3.060809in}{4.741712in}}%
\pgfpathlineto{\pgfqpoint{3.060809in}{4.741712in}}%
\pgfpathclose%
\pgfusepath{stroke,fill}%
\end{pgfscope}%
\begin{pgfscope}%
\pgfpathrectangle{\pgfqpoint{0.786164in}{0.768110in}}{\pgfqpoint{8.851069in}{7.081890in}}%
\pgfusepath{clip}%
\pgfsetbuttcap%
\pgfsetroundjoin%
\definecolor{currentfill}{rgb}{0.283091,0.110553,0.431554}%
\pgfsetfillcolor{currentfill}%
\pgfsetfillopacity{0.700000}%
\pgfsetlinewidth{0.501875pt}%
\definecolor{currentstroke}{rgb}{1.000000,1.000000,1.000000}%
\pgfsetstrokecolor{currentstroke}%
\pgfsetstrokeopacity{0.700000}%
\pgfsetdash{}{0pt}%
\pgfpathmoveto{\pgfqpoint{2.878143in}{4.369442in}}%
\pgfpathcurveto{\pgfqpoint{2.891166in}{4.369442in}}{\pgfqpoint{2.903657in}{4.374616in}}{\pgfqpoint{2.912866in}{4.383824in}}%
\pgfpathcurveto{\pgfqpoint{2.922074in}{4.393033in}}{\pgfqpoint{2.927248in}{4.405524in}}{\pgfqpoint{2.927248in}{4.418546in}}%
\pgfpathcurveto{\pgfqpoint{2.927248in}{4.431569in}}{\pgfqpoint{2.922074in}{4.444060in}}{\pgfqpoint{2.912866in}{4.453269in}}%
\pgfpathcurveto{\pgfqpoint{2.903657in}{4.462477in}}{\pgfqpoint{2.891166in}{4.467651in}}{\pgfqpoint{2.878143in}{4.467651in}}%
\pgfpathcurveto{\pgfqpoint{2.865121in}{4.467651in}}{\pgfqpoint{2.852630in}{4.462477in}}{\pgfqpoint{2.843421in}{4.453269in}}%
\pgfpathcurveto{\pgfqpoint{2.834213in}{4.444060in}}{\pgfqpoint{2.829039in}{4.431569in}}{\pgfqpoint{2.829039in}{4.418546in}}%
\pgfpathcurveto{\pgfqpoint{2.829039in}{4.405524in}}{\pgfqpoint{2.834213in}{4.393033in}}{\pgfqpoint{2.843421in}{4.383824in}}%
\pgfpathcurveto{\pgfqpoint{2.852630in}{4.374616in}}{\pgfqpoint{2.865121in}{4.369442in}}{\pgfqpoint{2.878143in}{4.369442in}}%
\pgfpathlineto{\pgfqpoint{2.878143in}{4.369442in}}%
\pgfpathclose%
\pgfusepath{stroke,fill}%
\end{pgfscope}%
\begin{pgfscope}%
\pgfpathrectangle{\pgfqpoint{0.786164in}{0.768110in}}{\pgfqpoint{8.851069in}{7.081890in}}%
\pgfusepath{clip}%
\pgfsetbuttcap%
\pgfsetroundjoin%
\definecolor{currentfill}{rgb}{0.283229,0.120777,0.440584}%
\pgfsetfillcolor{currentfill}%
\pgfsetfillopacity{0.700000}%
\pgfsetlinewidth{0.501875pt}%
\definecolor{currentstroke}{rgb}{1.000000,1.000000,1.000000}%
\pgfsetstrokecolor{currentstroke}%
\pgfsetstrokeopacity{0.700000}%
\pgfsetdash{}{0pt}%
\pgfpathmoveto{\pgfqpoint{2.823344in}{4.391340in}}%
\pgfpathcurveto{\pgfqpoint{2.836366in}{4.391340in}}{\pgfqpoint{2.848857in}{4.396514in}}{\pgfqpoint{2.858066in}{4.405722in}}%
\pgfpathcurveto{\pgfqpoint{2.867274in}{4.414931in}}{\pgfqpoint{2.872448in}{4.427422in}}{\pgfqpoint{2.872448in}{4.440445in}}%
\pgfpathcurveto{\pgfqpoint{2.872448in}{4.453467in}}{\pgfqpoint{2.867274in}{4.465958in}}{\pgfqpoint{2.858066in}{4.475167in}}%
\pgfpathcurveto{\pgfqpoint{2.848857in}{4.484375in}}{\pgfqpoint{2.836366in}{4.489549in}}{\pgfqpoint{2.823344in}{4.489549in}}%
\pgfpathcurveto{\pgfqpoint{2.810321in}{4.489549in}}{\pgfqpoint{2.797830in}{4.484375in}}{\pgfqpoint{2.788621in}{4.475167in}}%
\pgfpathcurveto{\pgfqpoint{2.779413in}{4.465958in}}{\pgfqpoint{2.774239in}{4.453467in}}{\pgfqpoint{2.774239in}{4.440445in}}%
\pgfpathcurveto{\pgfqpoint{2.774239in}{4.427422in}}{\pgfqpoint{2.779413in}{4.414931in}}{\pgfqpoint{2.788621in}{4.405722in}}%
\pgfpathcurveto{\pgfqpoint{2.797830in}{4.396514in}}{\pgfqpoint{2.810321in}{4.391340in}}{\pgfqpoint{2.823344in}{4.391340in}}%
\pgfpathlineto{\pgfqpoint{2.823344in}{4.391340in}}%
\pgfpathclose%
\pgfusepath{stroke,fill}%
\end{pgfscope}%
\begin{pgfscope}%
\pgfpathrectangle{\pgfqpoint{0.786164in}{0.768110in}}{\pgfqpoint{8.851069in}{7.081890in}}%
\pgfusepath{clip}%
\pgfsetbuttcap%
\pgfsetroundjoin%
\definecolor{currentfill}{rgb}{0.282623,0.140926,0.457517}%
\pgfsetfillcolor{currentfill}%
\pgfsetfillopacity{0.700000}%
\pgfsetlinewidth{0.501875pt}%
\definecolor{currentstroke}{rgb}{1.000000,1.000000,1.000000}%
\pgfsetstrokecolor{currentstroke}%
\pgfsetstrokeopacity{0.700000}%
\pgfsetdash{}{0pt}%
\pgfpathmoveto{\pgfqpoint{2.576745in}{4.128561in}}%
\pgfpathcurveto{\pgfqpoint{2.589768in}{4.128561in}}{\pgfqpoint{2.602259in}{4.133735in}}{\pgfqpoint{2.611467in}{4.142944in}}%
\pgfpathcurveto{\pgfqpoint{2.620676in}{4.152152in}}{\pgfqpoint{2.625850in}{4.164643in}}{\pgfqpoint{2.625850in}{4.177666in}}%
\pgfpathcurveto{\pgfqpoint{2.625850in}{4.190688in}}{\pgfqpoint{2.620676in}{4.203180in}}{\pgfqpoint{2.611467in}{4.212388in}}%
\pgfpathcurveto{\pgfqpoint{2.602259in}{4.221596in}}{\pgfqpoint{2.589768in}{4.226770in}}{\pgfqpoint{2.576745in}{4.226770in}}%
\pgfpathcurveto{\pgfqpoint{2.563722in}{4.226770in}}{\pgfqpoint{2.551231in}{4.221596in}}{\pgfqpoint{2.542023in}{4.212388in}}%
\pgfpathcurveto{\pgfqpoint{2.532814in}{4.203180in}}{\pgfqpoint{2.527640in}{4.190688in}}{\pgfqpoint{2.527640in}{4.177666in}}%
\pgfpathcurveto{\pgfqpoint{2.527640in}{4.164643in}}{\pgfqpoint{2.532814in}{4.152152in}}{\pgfqpoint{2.542023in}{4.142944in}}%
\pgfpathcurveto{\pgfqpoint{2.551231in}{4.133735in}}{\pgfqpoint{2.563722in}{4.128561in}}{\pgfqpoint{2.576745in}{4.128561in}}%
\pgfpathlineto{\pgfqpoint{2.576745in}{4.128561in}}%
\pgfpathclose%
\pgfusepath{stroke,fill}%
\end{pgfscope}%
\begin{pgfscope}%
\pgfpathrectangle{\pgfqpoint{0.786164in}{0.768110in}}{\pgfqpoint{8.851069in}{7.081890in}}%
\pgfusepath{clip}%
\pgfsetbuttcap%
\pgfsetroundjoin%
\definecolor{currentfill}{rgb}{0.282290,0.145912,0.461510}%
\pgfsetfillcolor{currentfill}%
\pgfsetfillopacity{0.700000}%
\pgfsetlinewidth{0.501875pt}%
\definecolor{currentstroke}{rgb}{1.000000,1.000000,1.000000}%
\pgfsetstrokecolor{currentstroke}%
\pgfsetstrokeopacity{0.700000}%
\pgfsetdash{}{0pt}%
\pgfpathmoveto{\pgfqpoint{2.595011in}{4.106663in}}%
\pgfpathcurveto{\pgfqpoint{2.608034in}{4.106663in}}{\pgfqpoint{2.620525in}{4.111837in}}{\pgfqpoint{2.629734in}{4.121045in}}%
\pgfpathcurveto{\pgfqpoint{2.638942in}{4.130254in}}{\pgfqpoint{2.644116in}{4.142745in}}{\pgfqpoint{2.644116in}{4.155768in}}%
\pgfpathcurveto{\pgfqpoint{2.644116in}{4.168790in}}{\pgfqpoint{2.638942in}{4.181281in}}{\pgfqpoint{2.629734in}{4.190490in}}%
\pgfpathcurveto{\pgfqpoint{2.620525in}{4.199698in}}{\pgfqpoint{2.608034in}{4.204872in}}{\pgfqpoint{2.595011in}{4.204872in}}%
\pgfpathcurveto{\pgfqpoint{2.581989in}{4.204872in}}{\pgfqpoint{2.569498in}{4.199698in}}{\pgfqpoint{2.560289in}{4.190490in}}%
\pgfpathcurveto{\pgfqpoint{2.551081in}{4.181281in}}{\pgfqpoint{2.545907in}{4.168790in}}{\pgfqpoint{2.545907in}{4.155768in}}%
\pgfpathcurveto{\pgfqpoint{2.545907in}{4.142745in}}{\pgfqpoint{2.551081in}{4.130254in}}{\pgfqpoint{2.560289in}{4.121045in}}%
\pgfpathcurveto{\pgfqpoint{2.569498in}{4.111837in}}{\pgfqpoint{2.581989in}{4.106663in}}{\pgfqpoint{2.595011in}{4.106663in}}%
\pgfpathlineto{\pgfqpoint{2.595011in}{4.106663in}}%
\pgfpathclose%
\pgfusepath{stroke,fill}%
\end{pgfscope}%
\begin{pgfscope}%
\pgfpathrectangle{\pgfqpoint{0.786164in}{0.768110in}}{\pgfqpoint{8.851069in}{7.081890in}}%
\pgfusepath{clip}%
\pgfsetbuttcap%
\pgfsetroundjoin%
\definecolor{currentfill}{rgb}{0.281412,0.155834,0.469201}%
\pgfsetfillcolor{currentfill}%
\pgfsetfillopacity{0.700000}%
\pgfsetlinewidth{0.501875pt}%
\definecolor{currentstroke}{rgb}{1.000000,1.000000,1.000000}%
\pgfsetstrokecolor{currentstroke}%
\pgfsetstrokeopacity{0.700000}%
\pgfsetdash{}{0pt}%
\pgfpathmoveto{\pgfqpoint{2.439746in}{4.128561in}}%
\pgfpathcurveto{\pgfqpoint{2.452768in}{4.128561in}}{\pgfqpoint{2.465259in}{4.133735in}}{\pgfqpoint{2.474468in}{4.142944in}}%
\pgfpathcurveto{\pgfqpoint{2.483676in}{4.152152in}}{\pgfqpoint{2.488850in}{4.164643in}}{\pgfqpoint{2.488850in}{4.177666in}}%
\pgfpathcurveto{\pgfqpoint{2.488850in}{4.190688in}}{\pgfqpoint{2.483676in}{4.203180in}}{\pgfqpoint{2.474468in}{4.212388in}}%
\pgfpathcurveto{\pgfqpoint{2.465259in}{4.221596in}}{\pgfqpoint{2.452768in}{4.226770in}}{\pgfqpoint{2.439746in}{4.226770in}}%
\pgfpathcurveto{\pgfqpoint{2.426723in}{4.226770in}}{\pgfqpoint{2.414232in}{4.221596in}}{\pgfqpoint{2.405023in}{4.212388in}}%
\pgfpathcurveto{\pgfqpoint{2.395815in}{4.203180in}}{\pgfqpoint{2.390641in}{4.190688in}}{\pgfqpoint{2.390641in}{4.177666in}}%
\pgfpathcurveto{\pgfqpoint{2.390641in}{4.164643in}}{\pgfqpoint{2.395815in}{4.152152in}}{\pgfqpoint{2.405023in}{4.142944in}}%
\pgfpathcurveto{\pgfqpoint{2.414232in}{4.133735in}}{\pgfqpoint{2.426723in}{4.128561in}}{\pgfqpoint{2.439746in}{4.128561in}}%
\pgfpathlineto{\pgfqpoint{2.439746in}{4.128561in}}%
\pgfpathclose%
\pgfusepath{stroke,fill}%
\end{pgfscope}%
\begin{pgfscope}%
\pgfpathrectangle{\pgfqpoint{0.786164in}{0.768110in}}{\pgfqpoint{8.851069in}{7.081890in}}%
\pgfusepath{clip}%
\pgfsetbuttcap%
\pgfsetroundjoin%
\definecolor{currentfill}{rgb}{0.278826,0.175490,0.483397}%
\pgfsetfillcolor{currentfill}%
\pgfsetfillopacity{0.700000}%
\pgfsetlinewidth{0.501875pt}%
\definecolor{currentstroke}{rgb}{1.000000,1.000000,1.000000}%
\pgfsetstrokecolor{currentstroke}%
\pgfsetstrokeopacity{0.700000}%
\pgfsetdash{}{0pt}%
\pgfpathmoveto{\pgfqpoint{2.458012in}{4.062866in}}%
\pgfpathcurveto{\pgfqpoint{2.471035in}{4.062866in}}{\pgfqpoint{2.483526in}{4.068040in}}{\pgfqpoint{2.492734in}{4.077249in}}%
\pgfpathcurveto{\pgfqpoint{2.501943in}{4.086457in}}{\pgfqpoint{2.507117in}{4.098948in}}{\pgfqpoint{2.507117in}{4.111971in}}%
\pgfpathcurveto{\pgfqpoint{2.507117in}{4.124994in}}{\pgfqpoint{2.501943in}{4.137485in}}{\pgfqpoint{2.492734in}{4.146693in}}%
\pgfpathcurveto{\pgfqpoint{2.483526in}{4.155902in}}{\pgfqpoint{2.471035in}{4.161076in}}{\pgfqpoint{2.458012in}{4.161076in}}%
\pgfpathcurveto{\pgfqpoint{2.444989in}{4.161076in}}{\pgfqpoint{2.432498in}{4.155902in}}{\pgfqpoint{2.423290in}{4.146693in}}%
\pgfpathcurveto{\pgfqpoint{2.414081in}{4.137485in}}{\pgfqpoint{2.408908in}{4.124994in}}{\pgfqpoint{2.408908in}{4.111971in}}%
\pgfpathcurveto{\pgfqpoint{2.408908in}{4.098948in}}{\pgfqpoint{2.414081in}{4.086457in}}{\pgfqpoint{2.423290in}{4.077249in}}%
\pgfpathcurveto{\pgfqpoint{2.432498in}{4.068040in}}{\pgfqpoint{2.444989in}{4.062866in}}{\pgfqpoint{2.458012in}{4.062866in}}%
\pgfpathlineto{\pgfqpoint{2.458012in}{4.062866in}}%
\pgfpathclose%
\pgfusepath{stroke,fill}%
\end{pgfscope}%
\begin{pgfscope}%
\pgfpathrectangle{\pgfqpoint{0.786164in}{0.768110in}}{\pgfqpoint{8.851069in}{7.081890in}}%
\pgfusepath{clip}%
\pgfsetbuttcap%
\pgfsetroundjoin%
\definecolor{currentfill}{rgb}{0.277134,0.185228,0.489898}%
\pgfsetfillcolor{currentfill}%
\pgfsetfillopacity{0.700000}%
\pgfsetlinewidth{0.501875pt}%
\definecolor{currentstroke}{rgb}{1.000000,1.000000,1.000000}%
\pgfsetstrokecolor{currentstroke}%
\pgfsetstrokeopacity{0.700000}%
\pgfsetdash{}{0pt}%
\pgfpathmoveto{\pgfqpoint{2.576745in}{4.194256in}}%
\pgfpathcurveto{\pgfqpoint{2.589768in}{4.194256in}}{\pgfqpoint{2.602259in}{4.199430in}}{\pgfqpoint{2.611467in}{4.208638in}}%
\pgfpathcurveto{\pgfqpoint{2.620676in}{4.217847in}}{\pgfqpoint{2.625850in}{4.230338in}}{\pgfqpoint{2.625850in}{4.243360in}}%
\pgfpathcurveto{\pgfqpoint{2.625850in}{4.256383in}}{\pgfqpoint{2.620676in}{4.268874in}}{\pgfqpoint{2.611467in}{4.278083in}}%
\pgfpathcurveto{\pgfqpoint{2.602259in}{4.287291in}}{\pgfqpoint{2.589768in}{4.292465in}}{\pgfqpoint{2.576745in}{4.292465in}}%
\pgfpathcurveto{\pgfqpoint{2.563722in}{4.292465in}}{\pgfqpoint{2.551231in}{4.287291in}}{\pgfqpoint{2.542023in}{4.278083in}}%
\pgfpathcurveto{\pgfqpoint{2.532814in}{4.268874in}}{\pgfqpoint{2.527640in}{4.256383in}}{\pgfqpoint{2.527640in}{4.243360in}}%
\pgfpathcurveto{\pgfqpoint{2.527640in}{4.230338in}}{\pgfqpoint{2.532814in}{4.217847in}}{\pgfqpoint{2.542023in}{4.208638in}}%
\pgfpathcurveto{\pgfqpoint{2.551231in}{4.199430in}}{\pgfqpoint{2.563722in}{4.194256in}}{\pgfqpoint{2.576745in}{4.194256in}}%
\pgfpathlineto{\pgfqpoint{2.576745in}{4.194256in}}%
\pgfpathclose%
\pgfusepath{stroke,fill}%
\end{pgfscope}%
\begin{pgfscope}%
\pgfpathrectangle{\pgfqpoint{0.786164in}{0.768110in}}{\pgfqpoint{8.851069in}{7.081890in}}%
\pgfusepath{clip}%
\pgfsetbuttcap%
\pgfsetroundjoin%
\definecolor{currentfill}{rgb}{0.276194,0.190074,0.493001}%
\pgfsetfillcolor{currentfill}%
\pgfsetfillopacity{0.700000}%
\pgfsetlinewidth{0.501875pt}%
\definecolor{currentstroke}{rgb}{1.000000,1.000000,1.000000}%
\pgfsetstrokecolor{currentstroke}%
\pgfsetstrokeopacity{0.700000}%
\pgfsetdash{}{0pt}%
\pgfpathmoveto{\pgfqpoint{2.686344in}{4.216154in}}%
\pgfpathcurveto{\pgfqpoint{2.699367in}{4.216154in}}{\pgfqpoint{2.711858in}{4.221328in}}{\pgfqpoint{2.721067in}{4.230537in}}%
\pgfpathcurveto{\pgfqpoint{2.730275in}{4.239745in}}{\pgfqpoint{2.735449in}{4.252236in}}{\pgfqpoint{2.735449in}{4.265259in}}%
\pgfpathcurveto{\pgfqpoint{2.735449in}{4.278281in}}{\pgfqpoint{2.730275in}{4.290773in}}{\pgfqpoint{2.721067in}{4.299981in}}%
\pgfpathcurveto{\pgfqpoint{2.711858in}{4.309189in}}{\pgfqpoint{2.699367in}{4.314363in}}{\pgfqpoint{2.686344in}{4.314363in}}%
\pgfpathcurveto{\pgfqpoint{2.673322in}{4.314363in}}{\pgfqpoint{2.660831in}{4.309189in}}{\pgfqpoint{2.651622in}{4.299981in}}%
\pgfpathcurveto{\pgfqpoint{2.642414in}{4.290773in}}{\pgfqpoint{2.637240in}{4.278281in}}{\pgfqpoint{2.637240in}{4.265259in}}%
\pgfpathcurveto{\pgfqpoint{2.637240in}{4.252236in}}{\pgfqpoint{2.642414in}{4.239745in}}{\pgfqpoint{2.651622in}{4.230537in}}%
\pgfpathcurveto{\pgfqpoint{2.660831in}{4.221328in}}{\pgfqpoint{2.673322in}{4.216154in}}{\pgfqpoint{2.686344in}{4.216154in}}%
\pgfpathlineto{\pgfqpoint{2.686344in}{4.216154in}}%
\pgfpathclose%
\pgfusepath{stroke,fill}%
\end{pgfscope}%
\begin{pgfscope}%
\pgfpathrectangle{\pgfqpoint{0.786164in}{0.768110in}}{\pgfqpoint{8.851069in}{7.081890in}}%
\pgfusepath{clip}%
\pgfsetbuttcap%
\pgfsetroundjoin%
\definecolor{currentfill}{rgb}{0.270595,0.214069,0.507052}%
\pgfsetfillcolor{currentfill}%
\pgfsetfillopacity{0.700000}%
\pgfsetlinewidth{0.501875pt}%
\definecolor{currentstroke}{rgb}{1.000000,1.000000,1.000000}%
\pgfsetstrokecolor{currentstroke}%
\pgfsetstrokeopacity{0.700000}%
\pgfsetdash{}{0pt}%
\pgfpathmoveto{\pgfqpoint{2.640678in}{4.259951in}}%
\pgfpathcurveto{\pgfqpoint{2.653701in}{4.259951in}}{\pgfqpoint{2.666192in}{4.265125in}}{\pgfqpoint{2.675400in}{4.274333in}}%
\pgfpathcurveto{\pgfqpoint{2.684609in}{4.283541in}}{\pgfqpoint{2.689783in}{4.296033in}}{\pgfqpoint{2.689783in}{4.309055in}}%
\pgfpathcurveto{\pgfqpoint{2.689783in}{4.322078in}}{\pgfqpoint{2.684609in}{4.334569in}}{\pgfqpoint{2.675400in}{4.343777in}}%
\pgfpathcurveto{\pgfqpoint{2.666192in}{4.352986in}}{\pgfqpoint{2.653701in}{4.358160in}}{\pgfqpoint{2.640678in}{4.358160in}}%
\pgfpathcurveto{\pgfqpoint{2.627655in}{4.358160in}}{\pgfqpoint{2.615164in}{4.352986in}}{\pgfqpoint{2.605956in}{4.343777in}}%
\pgfpathcurveto{\pgfqpoint{2.596747in}{4.334569in}}{\pgfqpoint{2.591573in}{4.322078in}}{\pgfqpoint{2.591573in}{4.309055in}}%
\pgfpathcurveto{\pgfqpoint{2.591573in}{4.296033in}}{\pgfqpoint{2.596747in}{4.283541in}}{\pgfqpoint{2.605956in}{4.274333in}}%
\pgfpathcurveto{\pgfqpoint{2.615164in}{4.265125in}}{\pgfqpoint{2.627655in}{4.259951in}}{\pgfqpoint{2.640678in}{4.259951in}}%
\pgfpathlineto{\pgfqpoint{2.640678in}{4.259951in}}%
\pgfpathclose%
\pgfusepath{stroke,fill}%
\end{pgfscope}%
\begin{pgfscope}%
\pgfpathrectangle{\pgfqpoint{0.786164in}{0.768110in}}{\pgfqpoint{8.851069in}{7.081890in}}%
\pgfusepath{clip}%
\pgfsetbuttcap%
\pgfsetroundjoin%
\definecolor{currentfill}{rgb}{0.267968,0.223549,0.512008}%
\pgfsetfillcolor{currentfill}%
\pgfsetfillopacity{0.700000}%
\pgfsetlinewidth{0.501875pt}%
\definecolor{currentstroke}{rgb}{1.000000,1.000000,1.000000}%
\pgfsetstrokecolor{currentstroke}%
\pgfsetstrokeopacity{0.700000}%
\pgfsetdash{}{0pt}%
\pgfpathmoveto{\pgfqpoint{2.613278in}{4.216154in}}%
\pgfpathcurveto{\pgfqpoint{2.626301in}{4.216154in}}{\pgfqpoint{2.638792in}{4.221328in}}{\pgfqpoint{2.648000in}{4.230537in}}%
\pgfpathcurveto{\pgfqpoint{2.657209in}{4.239745in}}{\pgfqpoint{2.662383in}{4.252236in}}{\pgfqpoint{2.662383in}{4.265259in}}%
\pgfpathcurveto{\pgfqpoint{2.662383in}{4.278281in}}{\pgfqpoint{2.657209in}{4.290773in}}{\pgfqpoint{2.648000in}{4.299981in}}%
\pgfpathcurveto{\pgfqpoint{2.638792in}{4.309189in}}{\pgfqpoint{2.626301in}{4.314363in}}{\pgfqpoint{2.613278in}{4.314363in}}%
\pgfpathcurveto{\pgfqpoint{2.600255in}{4.314363in}}{\pgfqpoint{2.587764in}{4.309189in}}{\pgfqpoint{2.578556in}{4.299981in}}%
\pgfpathcurveto{\pgfqpoint{2.569347in}{4.290773in}}{\pgfqpoint{2.564173in}{4.278281in}}{\pgfqpoint{2.564173in}{4.265259in}}%
\pgfpathcurveto{\pgfqpoint{2.564173in}{4.252236in}}{\pgfqpoint{2.569347in}{4.239745in}}{\pgfqpoint{2.578556in}{4.230537in}}%
\pgfpathcurveto{\pgfqpoint{2.587764in}{4.221328in}}{\pgfqpoint{2.600255in}{4.216154in}}{\pgfqpoint{2.613278in}{4.216154in}}%
\pgfpathlineto{\pgfqpoint{2.613278in}{4.216154in}}%
\pgfpathclose%
\pgfusepath{stroke,fill}%
\end{pgfscope}%
\begin{pgfscope}%
\pgfpathrectangle{\pgfqpoint{0.786164in}{0.768110in}}{\pgfqpoint{8.851069in}{7.081890in}}%
\pgfusepath{clip}%
\pgfsetbuttcap%
\pgfsetroundjoin%
\definecolor{currentfill}{rgb}{0.262138,0.242286,0.520837}%
\pgfsetfillcolor{currentfill}%
\pgfsetfillopacity{0.700000}%
\pgfsetlinewidth{0.501875pt}%
\definecolor{currentstroke}{rgb}{1.000000,1.000000,1.000000}%
\pgfsetstrokecolor{currentstroke}%
\pgfsetstrokeopacity{0.700000}%
\pgfsetdash{}{0pt}%
\pgfpathmoveto{\pgfqpoint{2.595011in}{4.062866in}}%
\pgfpathcurveto{\pgfqpoint{2.608034in}{4.062866in}}{\pgfqpoint{2.620525in}{4.068040in}}{\pgfqpoint{2.629734in}{4.077249in}}%
\pgfpathcurveto{\pgfqpoint{2.638942in}{4.086457in}}{\pgfqpoint{2.644116in}{4.098948in}}{\pgfqpoint{2.644116in}{4.111971in}}%
\pgfpathcurveto{\pgfqpoint{2.644116in}{4.124994in}}{\pgfqpoint{2.638942in}{4.137485in}}{\pgfqpoint{2.629734in}{4.146693in}}%
\pgfpathcurveto{\pgfqpoint{2.620525in}{4.155902in}}{\pgfqpoint{2.608034in}{4.161076in}}{\pgfqpoint{2.595011in}{4.161076in}}%
\pgfpathcurveto{\pgfqpoint{2.581989in}{4.161076in}}{\pgfqpoint{2.569498in}{4.155902in}}{\pgfqpoint{2.560289in}{4.146693in}}%
\pgfpathcurveto{\pgfqpoint{2.551081in}{4.137485in}}{\pgfqpoint{2.545907in}{4.124994in}}{\pgfqpoint{2.545907in}{4.111971in}}%
\pgfpathcurveto{\pgfqpoint{2.545907in}{4.098948in}}{\pgfqpoint{2.551081in}{4.086457in}}{\pgfqpoint{2.560289in}{4.077249in}}%
\pgfpathcurveto{\pgfqpoint{2.569498in}{4.068040in}}{\pgfqpoint{2.581989in}{4.062866in}}{\pgfqpoint{2.595011in}{4.062866in}}%
\pgfpathlineto{\pgfqpoint{2.595011in}{4.062866in}}%
\pgfpathclose%
\pgfusepath{stroke,fill}%
\end{pgfscope}%
\begin{pgfscope}%
\pgfpathrectangle{\pgfqpoint{0.786164in}{0.768110in}}{\pgfqpoint{8.851069in}{7.081890in}}%
\pgfusepath{clip}%
\pgfsetbuttcap%
\pgfsetroundjoin%
\definecolor{currentfill}{rgb}{0.233603,0.313828,0.543914}%
\pgfsetfillcolor{currentfill}%
\pgfsetfillopacity{0.700000}%
\pgfsetlinewidth{0.501875pt}%
\definecolor{currentstroke}{rgb}{1.000000,1.000000,1.000000}%
\pgfsetstrokecolor{currentstroke}%
\pgfsetstrokeopacity{0.700000}%
\pgfsetdash{}{0pt}%
\pgfpathmoveto{\pgfqpoint{2.339279in}{3.887681in}}%
\pgfpathcurveto{\pgfqpoint{2.352302in}{3.887681in}}{\pgfqpoint{2.364793in}{3.892855in}}{\pgfqpoint{2.374002in}{3.902063in}}%
\pgfpathcurveto{\pgfqpoint{2.383210in}{3.911271in}}{\pgfqpoint{2.388384in}{3.923762in}}{\pgfqpoint{2.388384in}{3.936785in}}%
\pgfpathcurveto{\pgfqpoint{2.388384in}{3.949808in}}{\pgfqpoint{2.383210in}{3.962299in}}{\pgfqpoint{2.374002in}{3.971507in}}%
\pgfpathcurveto{\pgfqpoint{2.364793in}{3.980716in}}{\pgfqpoint{2.352302in}{3.985890in}}{\pgfqpoint{2.339279in}{3.985890in}}%
\pgfpathcurveto{\pgfqpoint{2.326257in}{3.985890in}}{\pgfqpoint{2.313766in}{3.980716in}}{\pgfqpoint{2.304557in}{3.971507in}}%
\pgfpathcurveto{\pgfqpoint{2.295349in}{3.962299in}}{\pgfqpoint{2.290175in}{3.949808in}}{\pgfqpoint{2.290175in}{3.936785in}}%
\pgfpathcurveto{\pgfqpoint{2.290175in}{3.923762in}}{\pgfqpoint{2.295349in}{3.911271in}}{\pgfqpoint{2.304557in}{3.902063in}}%
\pgfpathcurveto{\pgfqpoint{2.313766in}{3.892855in}}{\pgfqpoint{2.326257in}{3.887681in}}{\pgfqpoint{2.339279in}{3.887681in}}%
\pgfpathlineto{\pgfqpoint{2.339279in}{3.887681in}}%
\pgfpathclose%
\pgfusepath{stroke,fill}%
\end{pgfscope}%
\begin{pgfscope}%
\pgfpathrectangle{\pgfqpoint{0.786164in}{0.768110in}}{\pgfqpoint{8.851069in}{7.081890in}}%
\pgfusepath{clip}%
\pgfsetbuttcap%
\pgfsetroundjoin%
\definecolor{currentfill}{rgb}{0.260571,0.246922,0.522828}%
\pgfsetfillcolor{currentfill}%
\pgfsetfillopacity{0.700000}%
\pgfsetlinewidth{0.501875pt}%
\definecolor{currentstroke}{rgb}{1.000000,1.000000,1.000000}%
\pgfsetstrokecolor{currentstroke}%
\pgfsetstrokeopacity{0.700000}%
\pgfsetdash{}{0pt}%
\pgfpathmoveto{\pgfqpoint{2.384946in}{3.975274in}}%
\pgfpathcurveto{\pgfqpoint{2.397969in}{3.975274in}}{\pgfqpoint{2.410460in}{3.980447in}}{\pgfqpoint{2.419668in}{3.989656in}}%
\pgfpathcurveto{\pgfqpoint{2.428877in}{3.998864in}}{\pgfqpoint{2.434050in}{4.011355in}}{\pgfqpoint{2.434050in}{4.024378in}}%
\pgfpathcurveto{\pgfqpoint{2.434050in}{4.037401in}}{\pgfqpoint{2.428877in}{4.049892in}}{\pgfqpoint{2.419668in}{4.059100in}}%
\pgfpathcurveto{\pgfqpoint{2.410460in}{4.068309in}}{\pgfqpoint{2.397969in}{4.073483in}}{\pgfqpoint{2.384946in}{4.073483in}}%
\pgfpathcurveto{\pgfqpoint{2.371923in}{4.073483in}}{\pgfqpoint{2.359432in}{4.068309in}}{\pgfqpoint{2.350224in}{4.059100in}}%
\pgfpathcurveto{\pgfqpoint{2.341015in}{4.049892in}}{\pgfqpoint{2.335841in}{4.037401in}}{\pgfqpoint{2.335841in}{4.024378in}}%
\pgfpathcurveto{\pgfqpoint{2.335841in}{4.011355in}}{\pgfqpoint{2.341015in}{3.998864in}}{\pgfqpoint{2.350224in}{3.989656in}}%
\pgfpathcurveto{\pgfqpoint{2.359432in}{3.980447in}}{\pgfqpoint{2.371923in}{3.975274in}}{\pgfqpoint{2.384946in}{3.975274in}}%
\pgfpathlineto{\pgfqpoint{2.384946in}{3.975274in}}%
\pgfpathclose%
\pgfusepath{stroke,fill}%
\end{pgfscope}%
\begin{pgfscope}%
\pgfpathrectangle{\pgfqpoint{0.786164in}{0.768110in}}{\pgfqpoint{8.851069in}{7.081890in}}%
\pgfusepath{clip}%
\pgfsetbuttcap%
\pgfsetroundjoin%
\definecolor{currentfill}{rgb}{0.255645,0.260703,0.528312}%
\pgfsetfillcolor{currentfill}%
\pgfsetfillopacity{0.700000}%
\pgfsetlinewidth{0.501875pt}%
\definecolor{currentstroke}{rgb}{1.000000,1.000000,1.000000}%
\pgfsetstrokecolor{currentstroke}%
\pgfsetstrokeopacity{0.700000}%
\pgfsetdash{}{0pt}%
\pgfpathmoveto{\pgfqpoint{2.412346in}{3.887681in}}%
\pgfpathcurveto{\pgfqpoint{2.425368in}{3.887681in}}{\pgfqpoint{2.437860in}{3.892855in}}{\pgfqpoint{2.447068in}{3.902063in}}%
\pgfpathcurveto{\pgfqpoint{2.456276in}{3.911271in}}{\pgfqpoint{2.461450in}{3.923762in}}{\pgfqpoint{2.461450in}{3.936785in}}%
\pgfpathcurveto{\pgfqpoint{2.461450in}{3.949808in}}{\pgfqpoint{2.456276in}{3.962299in}}{\pgfqpoint{2.447068in}{3.971507in}}%
\pgfpathcurveto{\pgfqpoint{2.437860in}{3.980716in}}{\pgfqpoint{2.425368in}{3.985890in}}{\pgfqpoint{2.412346in}{3.985890in}}%
\pgfpathcurveto{\pgfqpoint{2.399323in}{3.985890in}}{\pgfqpoint{2.386832in}{3.980716in}}{\pgfqpoint{2.377624in}{3.971507in}}%
\pgfpathcurveto{\pgfqpoint{2.368415in}{3.962299in}}{\pgfqpoint{2.363241in}{3.949808in}}{\pgfqpoint{2.363241in}{3.936785in}}%
\pgfpathcurveto{\pgfqpoint{2.363241in}{3.923762in}}{\pgfqpoint{2.368415in}{3.911271in}}{\pgfqpoint{2.377624in}{3.902063in}}%
\pgfpathcurveto{\pgfqpoint{2.386832in}{3.892855in}}{\pgfqpoint{2.399323in}{3.887681in}}{\pgfqpoint{2.412346in}{3.887681in}}%
\pgfpathlineto{\pgfqpoint{2.412346in}{3.887681in}}%
\pgfpathclose%
\pgfusepath{stroke,fill}%
\end{pgfscope}%
\begin{pgfscope}%
\pgfpathrectangle{\pgfqpoint{0.786164in}{0.768110in}}{\pgfqpoint{8.851069in}{7.081890in}}%
\pgfusepath{clip}%
\pgfsetbuttcap%
\pgfsetroundjoin%
\definecolor{currentfill}{rgb}{0.282884,0.135920,0.453427}%
\pgfsetfillcolor{currentfill}%
\pgfsetfillopacity{0.700000}%
\pgfsetlinewidth{0.501875pt}%
\definecolor{currentstroke}{rgb}{1.000000,1.000000,1.000000}%
\pgfsetstrokecolor{currentstroke}%
\pgfsetstrokeopacity{0.700000}%
\pgfsetdash{}{0pt}%
\pgfpathmoveto{\pgfqpoint{2.686344in}{3.186937in}}%
\pgfpathcurveto{\pgfqpoint{2.699367in}{3.186937in}}{\pgfqpoint{2.711858in}{3.192111in}}{\pgfqpoint{2.721067in}{3.201319in}}%
\pgfpathcurveto{\pgfqpoint{2.730275in}{3.210528in}}{\pgfqpoint{2.735449in}{3.223019in}}{\pgfqpoint{2.735449in}{3.236042in}}%
\pgfpathcurveto{\pgfqpoint{2.735449in}{3.249064in}}{\pgfqpoint{2.730275in}{3.261555in}}{\pgfqpoint{2.721067in}{3.270764in}}%
\pgfpathcurveto{\pgfqpoint{2.711858in}{3.279972in}}{\pgfqpoint{2.699367in}{3.285146in}}{\pgfqpoint{2.686344in}{3.285146in}}%
\pgfpathcurveto{\pgfqpoint{2.673322in}{3.285146in}}{\pgfqpoint{2.660831in}{3.279972in}}{\pgfqpoint{2.651622in}{3.270764in}}%
\pgfpathcurveto{\pgfqpoint{2.642414in}{3.261555in}}{\pgfqpoint{2.637240in}{3.249064in}}{\pgfqpoint{2.637240in}{3.236042in}}%
\pgfpathcurveto{\pgfqpoint{2.637240in}{3.223019in}}{\pgfqpoint{2.642414in}{3.210528in}}{\pgfqpoint{2.651622in}{3.201319in}}%
\pgfpathcurveto{\pgfqpoint{2.660831in}{3.192111in}}{\pgfqpoint{2.673322in}{3.186937in}}{\pgfqpoint{2.686344in}{3.186937in}}%
\pgfpathlineto{\pgfqpoint{2.686344in}{3.186937in}}%
\pgfpathclose%
\pgfusepath{stroke,fill}%
\end{pgfscope}%
\begin{pgfscope}%
\pgfpathrectangle{\pgfqpoint{0.786164in}{0.768110in}}{\pgfqpoint{8.851069in}{7.081890in}}%
\pgfusepath{clip}%
\pgfsetbuttcap%
\pgfsetroundjoin%
\definecolor{currentfill}{rgb}{0.281412,0.155834,0.469201}%
\pgfsetfillcolor{currentfill}%
\pgfsetfillopacity{0.700000}%
\pgfsetlinewidth{0.501875pt}%
\definecolor{currentstroke}{rgb}{1.000000,1.000000,1.000000}%
\pgfsetstrokecolor{currentstroke}%
\pgfsetstrokeopacity{0.700000}%
\pgfsetdash{}{0pt}%
\pgfpathmoveto{\pgfqpoint{2.722877in}{3.165039in}}%
\pgfpathcurveto{\pgfqpoint{2.735900in}{3.165039in}}{\pgfqpoint{2.748391in}{3.170213in}}{\pgfqpoint{2.757600in}{3.179421in}}%
\pgfpathcurveto{\pgfqpoint{2.766808in}{3.188630in}}{\pgfqpoint{2.771982in}{3.201121in}}{\pgfqpoint{2.771982in}{3.214143in}}%
\pgfpathcurveto{\pgfqpoint{2.771982in}{3.227166in}}{\pgfqpoint{2.766808in}{3.239657in}}{\pgfqpoint{2.757600in}{3.248866in}}%
\pgfpathcurveto{\pgfqpoint{2.748391in}{3.258074in}}{\pgfqpoint{2.735900in}{3.263248in}}{\pgfqpoint{2.722877in}{3.263248in}}%
\pgfpathcurveto{\pgfqpoint{2.709855in}{3.263248in}}{\pgfqpoint{2.697364in}{3.258074in}}{\pgfqpoint{2.688155in}{3.248866in}}%
\pgfpathcurveto{\pgfqpoint{2.678947in}{3.239657in}}{\pgfqpoint{2.673773in}{3.227166in}}{\pgfqpoint{2.673773in}{3.214143in}}%
\pgfpathcurveto{\pgfqpoint{2.673773in}{3.201121in}}{\pgfqpoint{2.678947in}{3.188630in}}{\pgfqpoint{2.688155in}{3.179421in}}%
\pgfpathcurveto{\pgfqpoint{2.697364in}{3.170213in}}{\pgfqpoint{2.709855in}{3.165039in}}{\pgfqpoint{2.722877in}{3.165039in}}%
\pgfpathlineto{\pgfqpoint{2.722877in}{3.165039in}}%
\pgfpathclose%
\pgfusepath{stroke,fill}%
\end{pgfscope}%
\begin{pgfscope}%
\pgfpathrectangle{\pgfqpoint{0.786164in}{0.768110in}}{\pgfqpoint{8.851069in}{7.081890in}}%
\pgfusepath{clip}%
\pgfsetbuttcap%
\pgfsetroundjoin%
\definecolor{currentfill}{rgb}{0.279574,0.170599,0.479997}%
\pgfsetfillcolor{currentfill}%
\pgfsetfillopacity{0.700000}%
\pgfsetlinewidth{0.501875pt}%
\definecolor{currentstroke}{rgb}{1.000000,1.000000,1.000000}%
\pgfsetstrokecolor{currentstroke}%
\pgfsetstrokeopacity{0.700000}%
\pgfsetdash{}{0pt}%
\pgfpathmoveto{\pgfqpoint{2.686344in}{3.121242in}}%
\pgfpathcurveto{\pgfqpoint{2.699367in}{3.121242in}}{\pgfqpoint{2.711858in}{3.126416in}}{\pgfqpoint{2.721067in}{3.135625in}}%
\pgfpathcurveto{\pgfqpoint{2.730275in}{3.144833in}}{\pgfqpoint{2.735449in}{3.157324in}}{\pgfqpoint{2.735449in}{3.170347in}}%
\pgfpathcurveto{\pgfqpoint{2.735449in}{3.183370in}}{\pgfqpoint{2.730275in}{3.195861in}}{\pgfqpoint{2.721067in}{3.205069in}}%
\pgfpathcurveto{\pgfqpoint{2.711858in}{3.214278in}}{\pgfqpoint{2.699367in}{3.219452in}}{\pgfqpoint{2.686344in}{3.219452in}}%
\pgfpathcurveto{\pgfqpoint{2.673322in}{3.219452in}}{\pgfqpoint{2.660831in}{3.214278in}}{\pgfqpoint{2.651622in}{3.205069in}}%
\pgfpathcurveto{\pgfqpoint{2.642414in}{3.195861in}}{\pgfqpoint{2.637240in}{3.183370in}}{\pgfqpoint{2.637240in}{3.170347in}}%
\pgfpathcurveto{\pgfqpoint{2.637240in}{3.157324in}}{\pgfqpoint{2.642414in}{3.144833in}}{\pgfqpoint{2.651622in}{3.135625in}}%
\pgfpathcurveto{\pgfqpoint{2.660831in}{3.126416in}}{\pgfqpoint{2.673322in}{3.121242in}}{\pgfqpoint{2.686344in}{3.121242in}}%
\pgfpathlineto{\pgfqpoint{2.686344in}{3.121242in}}%
\pgfpathclose%
\pgfusepath{stroke,fill}%
\end{pgfscope}%
\begin{pgfscope}%
\pgfpathrectangle{\pgfqpoint{0.786164in}{0.768110in}}{\pgfqpoint{8.851069in}{7.081890in}}%
\pgfusepath{clip}%
\pgfsetbuttcap%
\pgfsetroundjoin%
\definecolor{currentfill}{rgb}{0.274128,0.199721,0.498911}%
\pgfsetfillcolor{currentfill}%
\pgfsetfillopacity{0.700000}%
\pgfsetlinewidth{0.501875pt}%
\definecolor{currentstroke}{rgb}{1.000000,1.000000,1.000000}%
\pgfsetstrokecolor{currentstroke}%
\pgfsetstrokeopacity{0.700000}%
\pgfsetdash{}{0pt}%
\pgfpathmoveto{\pgfqpoint{2.668078in}{3.186937in}}%
\pgfpathcurveto{\pgfqpoint{2.681100in}{3.186937in}}{\pgfqpoint{2.693592in}{3.192111in}}{\pgfqpoint{2.702800in}{3.201319in}}%
\pgfpathcurveto{\pgfqpoint{2.712008in}{3.210528in}}{\pgfqpoint{2.717182in}{3.223019in}}{\pgfqpoint{2.717182in}{3.236042in}}%
\pgfpathcurveto{\pgfqpoint{2.717182in}{3.249064in}}{\pgfqpoint{2.712008in}{3.261555in}}{\pgfqpoint{2.702800in}{3.270764in}}%
\pgfpathcurveto{\pgfqpoint{2.693592in}{3.279972in}}{\pgfqpoint{2.681100in}{3.285146in}}{\pgfqpoint{2.668078in}{3.285146in}}%
\pgfpathcurveto{\pgfqpoint{2.655055in}{3.285146in}}{\pgfqpoint{2.642564in}{3.279972in}}{\pgfqpoint{2.633356in}{3.270764in}}%
\pgfpathcurveto{\pgfqpoint{2.624147in}{3.261555in}}{\pgfqpoint{2.618973in}{3.249064in}}{\pgfqpoint{2.618973in}{3.236042in}}%
\pgfpathcurveto{\pgfqpoint{2.618973in}{3.223019in}}{\pgfqpoint{2.624147in}{3.210528in}}{\pgfqpoint{2.633356in}{3.201319in}}%
\pgfpathcurveto{\pgfqpoint{2.642564in}{3.192111in}}{\pgfqpoint{2.655055in}{3.186937in}}{\pgfqpoint{2.668078in}{3.186937in}}%
\pgfpathlineto{\pgfqpoint{2.668078in}{3.186937in}}%
\pgfpathclose%
\pgfusepath{stroke,fill}%
\end{pgfscope}%
\begin{pgfscope}%
\pgfpathrectangle{\pgfqpoint{0.786164in}{0.768110in}}{\pgfqpoint{8.851069in}{7.081890in}}%
\pgfusepath{clip}%
\pgfsetbuttcap%
\pgfsetroundjoin%
\definecolor{currentfill}{rgb}{0.265145,0.232956,0.516599}%
\pgfsetfillcolor{currentfill}%
\pgfsetfillopacity{0.700000}%
\pgfsetlinewidth{0.501875pt}%
\definecolor{currentstroke}{rgb}{1.000000,1.000000,1.000000}%
\pgfsetstrokecolor{currentstroke}%
\pgfsetstrokeopacity{0.700000}%
\pgfsetdash{}{0pt}%
\pgfpathmoveto{\pgfqpoint{2.604145in}{3.055548in}}%
\pgfpathcurveto{\pgfqpoint{2.617167in}{3.055548in}}{\pgfqpoint{2.629659in}{3.060722in}}{\pgfqpoint{2.638867in}{3.069930in}}%
\pgfpathcurveto{\pgfqpoint{2.648075in}{3.079138in}}{\pgfqpoint{2.653249in}{3.091630in}}{\pgfqpoint{2.653249in}{3.104652in}}%
\pgfpathcurveto{\pgfqpoint{2.653249in}{3.117675in}}{\pgfqpoint{2.648075in}{3.130166in}}{\pgfqpoint{2.638867in}{3.139374in}}%
\pgfpathcurveto{\pgfqpoint{2.629659in}{3.148583in}}{\pgfqpoint{2.617167in}{3.153757in}}{\pgfqpoint{2.604145in}{3.153757in}}%
\pgfpathcurveto{\pgfqpoint{2.591122in}{3.153757in}}{\pgfqpoint{2.578631in}{3.148583in}}{\pgfqpoint{2.569423in}{3.139374in}}%
\pgfpathcurveto{\pgfqpoint{2.560214in}{3.130166in}}{\pgfqpoint{2.555040in}{3.117675in}}{\pgfqpoint{2.555040in}{3.104652in}}%
\pgfpathcurveto{\pgfqpoint{2.555040in}{3.091630in}}{\pgfqpoint{2.560214in}{3.079138in}}{\pgfqpoint{2.569423in}{3.069930in}}%
\pgfpathcurveto{\pgfqpoint{2.578631in}{3.060722in}}{\pgfqpoint{2.591122in}{3.055548in}}{\pgfqpoint{2.604145in}{3.055548in}}%
\pgfpathlineto{\pgfqpoint{2.604145in}{3.055548in}}%
\pgfpathclose%
\pgfusepath{stroke,fill}%
\end{pgfscope}%
\begin{pgfscope}%
\pgfpathrectangle{\pgfqpoint{0.786164in}{0.768110in}}{\pgfqpoint{8.851069in}{7.081890in}}%
\pgfusepath{clip}%
\pgfsetbuttcap%
\pgfsetroundjoin%
\definecolor{currentfill}{rgb}{0.257322,0.256130,0.526563}%
\pgfsetfillcolor{currentfill}%
\pgfsetfillopacity{0.700000}%
\pgfsetlinewidth{0.501875pt}%
\definecolor{currentstroke}{rgb}{1.000000,1.000000,1.000000}%
\pgfsetstrokecolor{currentstroke}%
\pgfsetstrokeopacity{0.700000}%
\pgfsetdash{}{0pt}%
\pgfpathmoveto{\pgfqpoint{2.412346in}{2.814667in}}%
\pgfpathcurveto{\pgfqpoint{2.425368in}{2.814667in}}{\pgfqpoint{2.437860in}{2.819841in}}{\pgfqpoint{2.447068in}{2.829049in}}%
\pgfpathcurveto{\pgfqpoint{2.456276in}{2.838258in}}{\pgfqpoint{2.461450in}{2.850749in}}{\pgfqpoint{2.461450in}{2.863772in}}%
\pgfpathcurveto{\pgfqpoint{2.461450in}{2.876794in}}{\pgfqpoint{2.456276in}{2.889285in}}{\pgfqpoint{2.447068in}{2.898494in}}%
\pgfpathcurveto{\pgfqpoint{2.437860in}{2.907702in}}{\pgfqpoint{2.425368in}{2.912876in}}{\pgfqpoint{2.412346in}{2.912876in}}%
\pgfpathcurveto{\pgfqpoint{2.399323in}{2.912876in}}{\pgfqpoint{2.386832in}{2.907702in}}{\pgfqpoint{2.377624in}{2.898494in}}%
\pgfpathcurveto{\pgfqpoint{2.368415in}{2.889285in}}{\pgfqpoint{2.363241in}{2.876794in}}{\pgfqpoint{2.363241in}{2.863772in}}%
\pgfpathcurveto{\pgfqpoint{2.363241in}{2.850749in}}{\pgfqpoint{2.368415in}{2.838258in}}{\pgfqpoint{2.377624in}{2.829049in}}%
\pgfpathcurveto{\pgfqpoint{2.386832in}{2.819841in}}{\pgfqpoint{2.399323in}{2.814667in}}{\pgfqpoint{2.412346in}{2.814667in}}%
\pgfpathlineto{\pgfqpoint{2.412346in}{2.814667in}}%
\pgfpathclose%
\pgfusepath{stroke,fill}%
\end{pgfscope}%
\begin{pgfscope}%
\pgfpathrectangle{\pgfqpoint{0.786164in}{0.768110in}}{\pgfqpoint{8.851069in}{7.081890in}}%
\pgfusepath{clip}%
\pgfsetbuttcap%
\pgfsetroundjoin%
\definecolor{currentfill}{rgb}{0.255645,0.260703,0.528312}%
\pgfsetfillcolor{currentfill}%
\pgfsetfillopacity{0.700000}%
\pgfsetlinewidth{0.501875pt}%
\definecolor{currentstroke}{rgb}{1.000000,1.000000,1.000000}%
\pgfsetstrokecolor{currentstroke}%
\pgfsetstrokeopacity{0.700000}%
\pgfsetdash{}{0pt}%
\pgfpathmoveto{\pgfqpoint{2.448879in}{2.836565in}}%
\pgfpathcurveto{\pgfqpoint{2.461902in}{2.836565in}}{\pgfqpoint{2.474393in}{2.841739in}}{\pgfqpoint{2.483601in}{2.850948in}}%
\pgfpathcurveto{\pgfqpoint{2.492810in}{2.860156in}}{\pgfqpoint{2.497984in}{2.872647in}}{\pgfqpoint{2.497984in}{2.885670in}}%
\pgfpathcurveto{\pgfqpoint{2.497984in}{2.898693in}}{\pgfqpoint{2.492810in}{2.911184in}}{\pgfqpoint{2.483601in}{2.920392in}}%
\pgfpathcurveto{\pgfqpoint{2.474393in}{2.929601in}}{\pgfqpoint{2.461902in}{2.934774in}}{\pgfqpoint{2.448879in}{2.934774in}}%
\pgfpathcurveto{\pgfqpoint{2.435856in}{2.934774in}}{\pgfqpoint{2.423365in}{2.929601in}}{\pgfqpoint{2.414157in}{2.920392in}}%
\pgfpathcurveto{\pgfqpoint{2.404948in}{2.911184in}}{\pgfqpoint{2.399774in}{2.898693in}}{\pgfqpoint{2.399774in}{2.885670in}}%
\pgfpathcurveto{\pgfqpoint{2.399774in}{2.872647in}}{\pgfqpoint{2.404948in}{2.860156in}}{\pgfqpoint{2.414157in}{2.850948in}}%
\pgfpathcurveto{\pgfqpoint{2.423365in}{2.841739in}}{\pgfqpoint{2.435856in}{2.836565in}}{\pgfqpoint{2.448879in}{2.836565in}}%
\pgfpathlineto{\pgfqpoint{2.448879in}{2.836565in}}%
\pgfpathclose%
\pgfusepath{stroke,fill}%
\end{pgfscope}%
\begin{pgfscope}%
\pgfpathrectangle{\pgfqpoint{0.786164in}{0.768110in}}{\pgfqpoint{8.851069in}{7.081890in}}%
\pgfusepath{clip}%
\pgfsetbuttcap%
\pgfsetroundjoin%
\definecolor{currentfill}{rgb}{0.257322,0.256130,0.526563}%
\pgfsetfillcolor{currentfill}%
\pgfsetfillopacity{0.700000}%
\pgfsetlinewidth{0.501875pt}%
\definecolor{currentstroke}{rgb}{1.000000,1.000000,1.000000}%
\pgfsetstrokecolor{currentstroke}%
\pgfsetstrokeopacity{0.700000}%
\pgfsetdash{}{0pt}%
\pgfpathmoveto{\pgfqpoint{2.366679in}{2.814667in}}%
\pgfpathcurveto{\pgfqpoint{2.379702in}{2.814667in}}{\pgfqpoint{2.392193in}{2.819841in}}{\pgfqpoint{2.401402in}{2.829049in}}%
\pgfpathcurveto{\pgfqpoint{2.410610in}{2.838258in}}{\pgfqpoint{2.415784in}{2.850749in}}{\pgfqpoint{2.415784in}{2.863772in}}%
\pgfpathcurveto{\pgfqpoint{2.415784in}{2.876794in}}{\pgfqpoint{2.410610in}{2.889285in}}{\pgfqpoint{2.401402in}{2.898494in}}%
\pgfpathcurveto{\pgfqpoint{2.392193in}{2.907702in}}{\pgfqpoint{2.379702in}{2.912876in}}{\pgfqpoint{2.366679in}{2.912876in}}%
\pgfpathcurveto{\pgfqpoint{2.353657in}{2.912876in}}{\pgfqpoint{2.341166in}{2.907702in}}{\pgfqpoint{2.331957in}{2.898494in}}%
\pgfpathcurveto{\pgfqpoint{2.322749in}{2.889285in}}{\pgfqpoint{2.317575in}{2.876794in}}{\pgfqpoint{2.317575in}{2.863772in}}%
\pgfpathcurveto{\pgfqpoint{2.317575in}{2.850749in}}{\pgfqpoint{2.322749in}{2.838258in}}{\pgfqpoint{2.331957in}{2.829049in}}%
\pgfpathcurveto{\pgfqpoint{2.341166in}{2.819841in}}{\pgfqpoint{2.353657in}{2.814667in}}{\pgfqpoint{2.366679in}{2.814667in}}%
\pgfpathlineto{\pgfqpoint{2.366679in}{2.814667in}}%
\pgfpathclose%
\pgfusepath{stroke,fill}%
\end{pgfscope}%
\begin{pgfscope}%
\pgfpathrectangle{\pgfqpoint{0.786164in}{0.768110in}}{\pgfqpoint{8.851069in}{7.081890in}}%
\pgfusepath{clip}%
\pgfsetbuttcap%
\pgfsetroundjoin%
\definecolor{currentfill}{rgb}{0.239346,0.300855,0.540844}%
\pgfsetfillcolor{currentfill}%
\pgfsetfillopacity{0.700000}%
\pgfsetlinewidth{0.501875pt}%
\definecolor{currentstroke}{rgb}{1.000000,1.000000,1.000000}%
\pgfsetstrokecolor{currentstroke}%
\pgfsetstrokeopacity{0.700000}%
\pgfsetdash{}{0pt}%
\pgfpathmoveto{\pgfqpoint{2.275346in}{2.770871in}}%
\pgfpathcurveto{\pgfqpoint{2.288369in}{2.770871in}}{\pgfqpoint{2.300860in}{2.776044in}}{\pgfqpoint{2.310069in}{2.785253in}}%
\pgfpathcurveto{\pgfqpoint{2.319277in}{2.794461in}}{\pgfqpoint{2.324451in}{2.806952in}}{\pgfqpoint{2.324451in}{2.819975in}}%
\pgfpathcurveto{\pgfqpoint{2.324451in}{2.832998in}}{\pgfqpoint{2.319277in}{2.845489in}}{\pgfqpoint{2.310069in}{2.854697in}}%
\pgfpathcurveto{\pgfqpoint{2.300860in}{2.863906in}}{\pgfqpoint{2.288369in}{2.869080in}}{\pgfqpoint{2.275346in}{2.869080in}}%
\pgfpathcurveto{\pgfqpoint{2.262324in}{2.869080in}}{\pgfqpoint{2.249833in}{2.863906in}}{\pgfqpoint{2.240624in}{2.854697in}}%
\pgfpathcurveto{\pgfqpoint{2.231416in}{2.845489in}}{\pgfqpoint{2.226242in}{2.832998in}}{\pgfqpoint{2.226242in}{2.819975in}}%
\pgfpathcurveto{\pgfqpoint{2.226242in}{2.806952in}}{\pgfqpoint{2.231416in}{2.794461in}}{\pgfqpoint{2.240624in}{2.785253in}}%
\pgfpathcurveto{\pgfqpoint{2.249833in}{2.776044in}}{\pgfqpoint{2.262324in}{2.770871in}}{\pgfqpoint{2.275346in}{2.770871in}}%
\pgfpathlineto{\pgfqpoint{2.275346in}{2.770871in}}%
\pgfpathclose%
\pgfusepath{stroke,fill}%
\end{pgfscope}%
\begin{pgfscope}%
\pgfpathrectangle{\pgfqpoint{0.786164in}{0.768110in}}{\pgfqpoint{8.851069in}{7.081890in}}%
\pgfusepath{clip}%
\pgfsetbuttcap%
\pgfsetroundjoin%
\definecolor{currentfill}{rgb}{0.229739,0.322361,0.545706}%
\pgfsetfillcolor{currentfill}%
\pgfsetfillopacity{0.700000}%
\pgfsetlinewidth{0.501875pt}%
\definecolor{currentstroke}{rgb}{1.000000,1.000000,1.000000}%
\pgfsetstrokecolor{currentstroke}%
\pgfsetstrokeopacity{0.700000}%
\pgfsetdash{}{0pt}%
\pgfpathmoveto{\pgfqpoint{2.174880in}{2.551888in}}%
\pgfpathcurveto{\pgfqpoint{2.187903in}{2.551888in}}{\pgfqpoint{2.200394in}{2.557062in}}{\pgfqpoint{2.209602in}{2.566271in}}%
\pgfpathcurveto{\pgfqpoint{2.218811in}{2.575479in}}{\pgfqpoint{2.223985in}{2.587970in}}{\pgfqpoint{2.223985in}{2.600993in}}%
\pgfpathcurveto{\pgfqpoint{2.223985in}{2.614015in}}{\pgfqpoint{2.218811in}{2.626507in}}{\pgfqpoint{2.209602in}{2.635715in}}%
\pgfpathcurveto{\pgfqpoint{2.200394in}{2.644923in}}{\pgfqpoint{2.187903in}{2.650097in}}{\pgfqpoint{2.174880in}{2.650097in}}%
\pgfpathcurveto{\pgfqpoint{2.161858in}{2.650097in}}{\pgfqpoint{2.149366in}{2.644923in}}{\pgfqpoint{2.140158in}{2.635715in}}%
\pgfpathcurveto{\pgfqpoint{2.130950in}{2.626507in}}{\pgfqpoint{2.125776in}{2.614015in}}{\pgfqpoint{2.125776in}{2.600993in}}%
\pgfpathcurveto{\pgfqpoint{2.125776in}{2.587970in}}{\pgfqpoint{2.130950in}{2.575479in}}{\pgfqpoint{2.140158in}{2.566271in}}%
\pgfpathcurveto{\pgfqpoint{2.149366in}{2.557062in}}{\pgfqpoint{2.161858in}{2.551888in}}{\pgfqpoint{2.174880in}{2.551888in}}%
\pgfpathlineto{\pgfqpoint{2.174880in}{2.551888in}}%
\pgfpathclose%
\pgfusepath{stroke,fill}%
\end{pgfscope}%
\begin{pgfscope}%
\pgfpathrectangle{\pgfqpoint{0.786164in}{0.768110in}}{\pgfqpoint{8.851069in}{7.081890in}}%
\pgfusepath{clip}%
\pgfsetbuttcap%
\pgfsetroundjoin%
\definecolor{currentfill}{rgb}{0.227802,0.326594,0.546532}%
\pgfsetfillcolor{currentfill}%
\pgfsetfillopacity{0.700000}%
\pgfsetlinewidth{0.501875pt}%
\definecolor{currentstroke}{rgb}{1.000000,1.000000,1.000000}%
\pgfsetstrokecolor{currentstroke}%
\pgfsetstrokeopacity{0.700000}%
\pgfsetdash{}{0pt}%
\pgfpathmoveto{\pgfqpoint{2.019614in}{2.464295in}}%
\pgfpathcurveto{\pgfqpoint{2.032637in}{2.464295in}}{\pgfqpoint{2.045128in}{2.469469in}}{\pgfqpoint{2.054337in}{2.478678in}}%
\pgfpathcurveto{\pgfqpoint{2.063545in}{2.487886in}}{\pgfqpoint{2.068719in}{2.500377in}}{\pgfqpoint{2.068719in}{2.513400in}}%
\pgfpathcurveto{\pgfqpoint{2.068719in}{2.526423in}}{\pgfqpoint{2.063545in}{2.538914in}}{\pgfqpoint{2.054337in}{2.548122in}}%
\pgfpathcurveto{\pgfqpoint{2.045128in}{2.557331in}}{\pgfqpoint{2.032637in}{2.562504in}}{\pgfqpoint{2.019614in}{2.562504in}}%
\pgfpathcurveto{\pgfqpoint{2.006592in}{2.562504in}}{\pgfqpoint{1.994101in}{2.557331in}}{\pgfqpoint{1.984892in}{2.548122in}}%
\pgfpathcurveto{\pgfqpoint{1.975684in}{2.538914in}}{\pgfqpoint{1.970510in}{2.526423in}}{\pgfqpoint{1.970510in}{2.513400in}}%
\pgfpathcurveto{\pgfqpoint{1.970510in}{2.500377in}}{\pgfqpoint{1.975684in}{2.487886in}}{\pgfqpoint{1.984892in}{2.478678in}}%
\pgfpathcurveto{\pgfqpoint{1.994101in}{2.469469in}}{\pgfqpoint{2.006592in}{2.464295in}}{\pgfqpoint{2.019614in}{2.464295in}}%
\pgfpathlineto{\pgfqpoint{2.019614in}{2.464295in}}%
\pgfpathclose%
\pgfusepath{stroke,fill}%
\end{pgfscope}%
\begin{pgfscope}%
\pgfpathrectangle{\pgfqpoint{0.786164in}{0.768110in}}{\pgfqpoint{8.851069in}{7.081890in}}%
\pgfusepath{clip}%
\pgfsetbuttcap%
\pgfsetroundjoin%
\definecolor{currentfill}{rgb}{0.223925,0.334994,0.548053}%
\pgfsetfillcolor{currentfill}%
\pgfsetfillopacity{0.700000}%
\pgfsetlinewidth{0.501875pt}%
\definecolor{currentstroke}{rgb}{1.000000,1.000000,1.000000}%
\pgfsetstrokecolor{currentstroke}%
\pgfsetstrokeopacity{0.700000}%
\pgfsetdash{}{0pt}%
\pgfpathmoveto{\pgfqpoint{2.120081in}{2.508092in}}%
\pgfpathcurveto{\pgfqpoint{2.133103in}{2.508092in}}{\pgfqpoint{2.145594in}{2.513266in}}{\pgfqpoint{2.154803in}{2.522474in}}%
\pgfpathcurveto{\pgfqpoint{2.164011in}{2.531683in}}{\pgfqpoint{2.169185in}{2.544174in}}{\pgfqpoint{2.169185in}{2.557196in}}%
\pgfpathcurveto{\pgfqpoint{2.169185in}{2.570219in}}{\pgfqpoint{2.164011in}{2.582710in}}{\pgfqpoint{2.154803in}{2.591919in}}%
\pgfpathcurveto{\pgfqpoint{2.145594in}{2.601127in}}{\pgfqpoint{2.133103in}{2.606301in}}{\pgfqpoint{2.120081in}{2.606301in}}%
\pgfpathcurveto{\pgfqpoint{2.107058in}{2.606301in}}{\pgfqpoint{2.094567in}{2.601127in}}{\pgfqpoint{2.085358in}{2.591919in}}%
\pgfpathcurveto{\pgfqpoint{2.076150in}{2.582710in}}{\pgfqpoint{2.070976in}{2.570219in}}{\pgfqpoint{2.070976in}{2.557196in}}%
\pgfpathcurveto{\pgfqpoint{2.070976in}{2.544174in}}{\pgfqpoint{2.076150in}{2.531683in}}{\pgfqpoint{2.085358in}{2.522474in}}%
\pgfpathcurveto{\pgfqpoint{2.094567in}{2.513266in}}{\pgfqpoint{2.107058in}{2.508092in}}{\pgfqpoint{2.120081in}{2.508092in}}%
\pgfpathlineto{\pgfqpoint{2.120081in}{2.508092in}}%
\pgfpathclose%
\pgfusepath{stroke,fill}%
\end{pgfscope}%
\begin{pgfscope}%
\pgfpathrectangle{\pgfqpoint{0.786164in}{0.768110in}}{\pgfqpoint{8.851069in}{7.081890in}}%
\pgfusepath{clip}%
\pgfsetbuttcap%
\pgfsetroundjoin%
\definecolor{currentfill}{rgb}{0.223925,0.334994,0.548053}%
\pgfsetfillcolor{currentfill}%
\pgfsetfillopacity{0.700000}%
\pgfsetlinewidth{0.501875pt}%
\definecolor{currentstroke}{rgb}{1.000000,1.000000,1.000000}%
\pgfsetstrokecolor{currentstroke}%
\pgfsetstrokeopacity{0.700000}%
\pgfsetdash{}{0pt}%
\pgfpathmoveto{\pgfqpoint{2.101814in}{2.486193in}}%
\pgfpathcurveto{\pgfqpoint{2.114837in}{2.486193in}}{\pgfqpoint{2.127328in}{2.491367in}}{\pgfqpoint{2.136536in}{2.500576in}}%
\pgfpathcurveto{\pgfqpoint{2.145745in}{2.509784in}}{\pgfqpoint{2.150919in}{2.522275in}}{\pgfqpoint{2.150919in}{2.535298in}}%
\pgfpathcurveto{\pgfqpoint{2.150919in}{2.548321in}}{\pgfqpoint{2.145745in}{2.560812in}}{\pgfqpoint{2.136536in}{2.570020in}}%
\pgfpathcurveto{\pgfqpoint{2.127328in}{2.579229in}}{\pgfqpoint{2.114837in}{2.584403in}}{\pgfqpoint{2.101814in}{2.584403in}}%
\pgfpathcurveto{\pgfqpoint{2.088791in}{2.584403in}}{\pgfqpoint{2.076300in}{2.579229in}}{\pgfqpoint{2.067092in}{2.570020in}}%
\pgfpathcurveto{\pgfqpoint{2.057883in}{2.560812in}}{\pgfqpoint{2.052709in}{2.548321in}}{\pgfqpoint{2.052709in}{2.535298in}}%
\pgfpathcurveto{\pgfqpoint{2.052709in}{2.522275in}}{\pgfqpoint{2.057883in}{2.509784in}}{\pgfqpoint{2.067092in}{2.500576in}}%
\pgfpathcurveto{\pgfqpoint{2.076300in}{2.491367in}}{\pgfqpoint{2.088791in}{2.486193in}}{\pgfqpoint{2.101814in}{2.486193in}}%
\pgfpathlineto{\pgfqpoint{2.101814in}{2.486193in}}%
\pgfpathclose%
\pgfusepath{stroke,fill}%
\end{pgfscope}%
\begin{pgfscope}%
\pgfpathrectangle{\pgfqpoint{0.786164in}{0.768110in}}{\pgfqpoint{8.851069in}{7.081890in}}%
\pgfusepath{clip}%
\pgfsetbuttcap%
\pgfsetroundjoin%
\definecolor{currentfill}{rgb}{0.218130,0.347432,0.550038}%
\pgfsetfillcolor{currentfill}%
\pgfsetfillopacity{0.700000}%
\pgfsetlinewidth{0.501875pt}%
\definecolor{currentstroke}{rgb}{1.000000,1.000000,1.000000}%
\pgfsetstrokecolor{currentstroke}%
\pgfsetstrokeopacity{0.700000}%
\pgfsetdash{}{0pt}%
\pgfpathmoveto{\pgfqpoint{2.120081in}{2.551888in}}%
\pgfpathcurveto{\pgfqpoint{2.133103in}{2.551888in}}{\pgfqpoint{2.145594in}{2.557062in}}{\pgfqpoint{2.154803in}{2.566271in}}%
\pgfpathcurveto{\pgfqpoint{2.164011in}{2.575479in}}{\pgfqpoint{2.169185in}{2.587970in}}{\pgfqpoint{2.169185in}{2.600993in}}%
\pgfpathcurveto{\pgfqpoint{2.169185in}{2.614015in}}{\pgfqpoint{2.164011in}{2.626507in}}{\pgfqpoint{2.154803in}{2.635715in}}%
\pgfpathcurveto{\pgfqpoint{2.145594in}{2.644923in}}{\pgfqpoint{2.133103in}{2.650097in}}{\pgfqpoint{2.120081in}{2.650097in}}%
\pgfpathcurveto{\pgfqpoint{2.107058in}{2.650097in}}{\pgfqpoint{2.094567in}{2.644923in}}{\pgfqpoint{2.085358in}{2.635715in}}%
\pgfpathcurveto{\pgfqpoint{2.076150in}{2.626507in}}{\pgfqpoint{2.070976in}{2.614015in}}{\pgfqpoint{2.070976in}{2.600993in}}%
\pgfpathcurveto{\pgfqpoint{2.070976in}{2.587970in}}{\pgfqpoint{2.076150in}{2.575479in}}{\pgfqpoint{2.085358in}{2.566271in}}%
\pgfpathcurveto{\pgfqpoint{2.094567in}{2.557062in}}{\pgfqpoint{2.107058in}{2.551888in}}{\pgfqpoint{2.120081in}{2.551888in}}%
\pgfpathlineto{\pgfqpoint{2.120081in}{2.551888in}}%
\pgfpathclose%
\pgfusepath{stroke,fill}%
\end{pgfscope}%
\begin{pgfscope}%
\pgfpathrectangle{\pgfqpoint{0.786164in}{0.768110in}}{\pgfqpoint{8.851069in}{7.081890in}}%
\pgfusepath{clip}%
\pgfsetbuttcap%
\pgfsetroundjoin%
\definecolor{currentfill}{rgb}{0.221989,0.339161,0.548752}%
\pgfsetfillcolor{currentfill}%
\pgfsetfillopacity{0.700000}%
\pgfsetlinewidth{0.501875pt}%
\definecolor{currentstroke}{rgb}{1.000000,1.000000,1.000000}%
\pgfsetstrokecolor{currentstroke}%
\pgfsetstrokeopacity{0.700000}%
\pgfsetdash{}{0pt}%
\pgfpathmoveto{\pgfqpoint{2.101814in}{2.464295in}}%
\pgfpathcurveto{\pgfqpoint{2.114837in}{2.464295in}}{\pgfqpoint{2.127328in}{2.469469in}}{\pgfqpoint{2.136536in}{2.478678in}}%
\pgfpathcurveto{\pgfqpoint{2.145745in}{2.487886in}}{\pgfqpoint{2.150919in}{2.500377in}}{\pgfqpoint{2.150919in}{2.513400in}}%
\pgfpathcurveto{\pgfqpoint{2.150919in}{2.526423in}}{\pgfqpoint{2.145745in}{2.538914in}}{\pgfqpoint{2.136536in}{2.548122in}}%
\pgfpathcurveto{\pgfqpoint{2.127328in}{2.557331in}}{\pgfqpoint{2.114837in}{2.562504in}}{\pgfqpoint{2.101814in}{2.562504in}}%
\pgfpathcurveto{\pgfqpoint{2.088791in}{2.562504in}}{\pgfqpoint{2.076300in}{2.557331in}}{\pgfqpoint{2.067092in}{2.548122in}}%
\pgfpathcurveto{\pgfqpoint{2.057883in}{2.538914in}}{\pgfqpoint{2.052709in}{2.526423in}}{\pgfqpoint{2.052709in}{2.513400in}}%
\pgfpathcurveto{\pgfqpoint{2.052709in}{2.500377in}}{\pgfqpoint{2.057883in}{2.487886in}}{\pgfqpoint{2.067092in}{2.478678in}}%
\pgfpathcurveto{\pgfqpoint{2.076300in}{2.469469in}}{\pgfqpoint{2.088791in}{2.464295in}}{\pgfqpoint{2.101814in}{2.464295in}}%
\pgfpathlineto{\pgfqpoint{2.101814in}{2.464295in}}%
\pgfpathclose%
\pgfusepath{stroke,fill}%
\end{pgfscope}%
\begin{pgfscope}%
\pgfpathrectangle{\pgfqpoint{0.786164in}{0.768110in}}{\pgfqpoint{8.851069in}{7.081890in}}%
\pgfusepath{clip}%
\pgfsetbuttcap%
\pgfsetroundjoin%
\definecolor{currentfill}{rgb}{0.218130,0.347432,0.550038}%
\pgfsetfillcolor{currentfill}%
\pgfsetfillopacity{0.700000}%
\pgfsetlinewidth{0.501875pt}%
\definecolor{currentstroke}{rgb}{1.000000,1.000000,1.000000}%
\pgfsetstrokecolor{currentstroke}%
\pgfsetstrokeopacity{0.700000}%
\pgfsetdash{}{0pt}%
\pgfpathmoveto{\pgfqpoint{2.101814in}{2.464295in}}%
\pgfpathcurveto{\pgfqpoint{2.114837in}{2.464295in}}{\pgfqpoint{2.127328in}{2.469469in}}{\pgfqpoint{2.136536in}{2.478678in}}%
\pgfpathcurveto{\pgfqpoint{2.145745in}{2.487886in}}{\pgfqpoint{2.150919in}{2.500377in}}{\pgfqpoint{2.150919in}{2.513400in}}%
\pgfpathcurveto{\pgfqpoint{2.150919in}{2.526423in}}{\pgfqpoint{2.145745in}{2.538914in}}{\pgfqpoint{2.136536in}{2.548122in}}%
\pgfpathcurveto{\pgfqpoint{2.127328in}{2.557331in}}{\pgfqpoint{2.114837in}{2.562504in}}{\pgfqpoint{2.101814in}{2.562504in}}%
\pgfpathcurveto{\pgfqpoint{2.088791in}{2.562504in}}{\pgfqpoint{2.076300in}{2.557331in}}{\pgfqpoint{2.067092in}{2.548122in}}%
\pgfpathcurveto{\pgfqpoint{2.057883in}{2.538914in}}{\pgfqpoint{2.052709in}{2.526423in}}{\pgfqpoint{2.052709in}{2.513400in}}%
\pgfpathcurveto{\pgfqpoint{2.052709in}{2.500377in}}{\pgfqpoint{2.057883in}{2.487886in}}{\pgfqpoint{2.067092in}{2.478678in}}%
\pgfpathcurveto{\pgfqpoint{2.076300in}{2.469469in}}{\pgfqpoint{2.088791in}{2.464295in}}{\pgfqpoint{2.101814in}{2.464295in}}%
\pgfpathlineto{\pgfqpoint{2.101814in}{2.464295in}}%
\pgfpathclose%
\pgfusepath{stroke,fill}%
\end{pgfscope}%
\begin{pgfscope}%
\pgfpathrectangle{\pgfqpoint{0.786164in}{0.768110in}}{\pgfqpoint{8.851069in}{7.081890in}}%
\pgfusepath{clip}%
\pgfsetbuttcap%
\pgfsetroundjoin%
\definecolor{currentfill}{rgb}{0.203063,0.379716,0.553925}%
\pgfsetfillcolor{currentfill}%
\pgfsetfillopacity{0.700000}%
\pgfsetlinewidth{0.501875pt}%
\definecolor{currentstroke}{rgb}{1.000000,1.000000,1.000000}%
\pgfsetstrokecolor{currentstroke}%
\pgfsetstrokeopacity{0.700000}%
\pgfsetdash{}{0pt}%
\pgfpathmoveto{\pgfqpoint{1.910015in}{2.332906in}}%
\pgfpathcurveto{\pgfqpoint{1.923038in}{2.332906in}}{\pgfqpoint{1.935529in}{2.338080in}}{\pgfqpoint{1.944737in}{2.347288in}}%
\pgfpathcurveto{\pgfqpoint{1.953946in}{2.356497in}}{\pgfqpoint{1.959120in}{2.368988in}}{\pgfqpoint{1.959120in}{2.382010in}}%
\pgfpathcurveto{\pgfqpoint{1.959120in}{2.395033in}}{\pgfqpoint{1.953946in}{2.407524in}}{\pgfqpoint{1.944737in}{2.416733in}}%
\pgfpathcurveto{\pgfqpoint{1.935529in}{2.425941in}}{\pgfqpoint{1.923038in}{2.431115in}}{\pgfqpoint{1.910015in}{2.431115in}}%
\pgfpathcurveto{\pgfqpoint{1.896992in}{2.431115in}}{\pgfqpoint{1.884501in}{2.425941in}}{\pgfqpoint{1.875293in}{2.416733in}}%
\pgfpathcurveto{\pgfqpoint{1.866084in}{2.407524in}}{\pgfqpoint{1.860910in}{2.395033in}}{\pgfqpoint{1.860910in}{2.382010in}}%
\pgfpathcurveto{\pgfqpoint{1.860910in}{2.368988in}}{\pgfqpoint{1.866084in}{2.356497in}}{\pgfqpoint{1.875293in}{2.347288in}}%
\pgfpathcurveto{\pgfqpoint{1.884501in}{2.338080in}}{\pgfqpoint{1.896992in}{2.332906in}}{\pgfqpoint{1.910015in}{2.332906in}}%
\pgfpathlineto{\pgfqpoint{1.910015in}{2.332906in}}%
\pgfpathclose%
\pgfusepath{stroke,fill}%
\end{pgfscope}%
\begin{pgfscope}%
\pgfpathrectangle{\pgfqpoint{0.786164in}{0.768110in}}{\pgfqpoint{8.851069in}{7.081890in}}%
\pgfusepath{clip}%
\pgfsetbuttcap%
\pgfsetroundjoin%
\definecolor{currentfill}{rgb}{0.214298,0.355619,0.551184}%
\pgfsetfillcolor{currentfill}%
\pgfsetfillopacity{0.700000}%
\pgfsetlinewidth{0.501875pt}%
\definecolor{currentstroke}{rgb}{1.000000,1.000000,1.000000}%
\pgfsetstrokecolor{currentstroke}%
\pgfsetstrokeopacity{0.700000}%
\pgfsetdash{}{0pt}%
\pgfpathmoveto{\pgfqpoint{2.120081in}{2.486193in}}%
\pgfpathcurveto{\pgfqpoint{2.133103in}{2.486193in}}{\pgfqpoint{2.145594in}{2.491367in}}{\pgfqpoint{2.154803in}{2.500576in}}%
\pgfpathcurveto{\pgfqpoint{2.164011in}{2.509784in}}{\pgfqpoint{2.169185in}{2.522275in}}{\pgfqpoint{2.169185in}{2.535298in}}%
\pgfpathcurveto{\pgfqpoint{2.169185in}{2.548321in}}{\pgfqpoint{2.164011in}{2.560812in}}{\pgfqpoint{2.154803in}{2.570020in}}%
\pgfpathcurveto{\pgfqpoint{2.145594in}{2.579229in}}{\pgfqpoint{2.133103in}{2.584403in}}{\pgfqpoint{2.120081in}{2.584403in}}%
\pgfpathcurveto{\pgfqpoint{2.107058in}{2.584403in}}{\pgfqpoint{2.094567in}{2.579229in}}{\pgfqpoint{2.085358in}{2.570020in}}%
\pgfpathcurveto{\pgfqpoint{2.076150in}{2.560812in}}{\pgfqpoint{2.070976in}{2.548321in}}{\pgfqpoint{2.070976in}{2.535298in}}%
\pgfpathcurveto{\pgfqpoint{2.070976in}{2.522275in}}{\pgfqpoint{2.076150in}{2.509784in}}{\pgfqpoint{2.085358in}{2.500576in}}%
\pgfpathcurveto{\pgfqpoint{2.094567in}{2.491367in}}{\pgfqpoint{2.107058in}{2.486193in}}{\pgfqpoint{2.120081in}{2.486193in}}%
\pgfpathlineto{\pgfqpoint{2.120081in}{2.486193in}}%
\pgfpathclose%
\pgfusepath{stroke,fill}%
\end{pgfscope}%
\begin{pgfscope}%
\pgfpathrectangle{\pgfqpoint{0.786164in}{0.768110in}}{\pgfqpoint{8.851069in}{7.081890in}}%
\pgfusepath{clip}%
\pgfsetbuttcap%
\pgfsetroundjoin%
\definecolor{currentfill}{rgb}{0.212395,0.359683,0.551710}%
\pgfsetfillcolor{currentfill}%
\pgfsetfillopacity{0.700000}%
\pgfsetlinewidth{0.501875pt}%
\definecolor{currentstroke}{rgb}{1.000000,1.000000,1.000000}%
\pgfsetstrokecolor{currentstroke}%
\pgfsetstrokeopacity{0.700000}%
\pgfsetdash{}{0pt}%
\pgfpathmoveto{\pgfqpoint{2.028748in}{2.486193in}}%
\pgfpathcurveto{\pgfqpoint{2.041770in}{2.486193in}}{\pgfqpoint{2.054261in}{2.491367in}}{\pgfqpoint{2.063470in}{2.500576in}}%
\pgfpathcurveto{\pgfqpoint{2.072678in}{2.509784in}}{\pgfqpoint{2.077852in}{2.522275in}}{\pgfqpoint{2.077852in}{2.535298in}}%
\pgfpathcurveto{\pgfqpoint{2.077852in}{2.548321in}}{\pgfqpoint{2.072678in}{2.560812in}}{\pgfqpoint{2.063470in}{2.570020in}}%
\pgfpathcurveto{\pgfqpoint{2.054261in}{2.579229in}}{\pgfqpoint{2.041770in}{2.584403in}}{\pgfqpoint{2.028748in}{2.584403in}}%
\pgfpathcurveto{\pgfqpoint{2.015725in}{2.584403in}}{\pgfqpoint{2.003234in}{2.579229in}}{\pgfqpoint{1.994025in}{2.570020in}}%
\pgfpathcurveto{\pgfqpoint{1.984817in}{2.560812in}}{\pgfqpoint{1.979643in}{2.548321in}}{\pgfqpoint{1.979643in}{2.535298in}}%
\pgfpathcurveto{\pgfqpoint{1.979643in}{2.522275in}}{\pgfqpoint{1.984817in}{2.509784in}}{\pgfqpoint{1.994025in}{2.500576in}}%
\pgfpathcurveto{\pgfqpoint{2.003234in}{2.491367in}}{\pgfqpoint{2.015725in}{2.486193in}}{\pgfqpoint{2.028748in}{2.486193in}}%
\pgfpathlineto{\pgfqpoint{2.028748in}{2.486193in}}%
\pgfpathclose%
\pgfusepath{stroke,fill}%
\end{pgfscope}%
\begin{pgfscope}%
\pgfpathrectangle{\pgfqpoint{0.786164in}{0.768110in}}{\pgfqpoint{8.851069in}{7.081890in}}%
\pgfusepath{clip}%
\pgfsetbuttcap%
\pgfsetroundjoin%
\definecolor{currentfill}{rgb}{0.225863,0.330805,0.547314}%
\pgfsetfillcolor{currentfill}%
\pgfsetfillopacity{0.700000}%
\pgfsetlinewidth{0.501875pt}%
\definecolor{currentstroke}{rgb}{1.000000,1.000000,1.000000}%
\pgfsetstrokecolor{currentstroke}%
\pgfsetstrokeopacity{0.700000}%
\pgfsetdash{}{0pt}%
\pgfpathmoveto{\pgfqpoint{2.193147in}{2.092025in}}%
\pgfpathcurveto{\pgfqpoint{2.206170in}{2.092025in}}{\pgfqpoint{2.218661in}{2.097199in}}{\pgfqpoint{2.227869in}{2.106408in}}%
\pgfpathcurveto{\pgfqpoint{2.237077in}{2.115616in}}{\pgfqpoint{2.242251in}{2.128107in}}{\pgfqpoint{2.242251in}{2.141130in}}%
\pgfpathcurveto{\pgfqpoint{2.242251in}{2.154153in}}{\pgfqpoint{2.237077in}{2.166644in}}{\pgfqpoint{2.227869in}{2.175852in}}%
\pgfpathcurveto{\pgfqpoint{2.218661in}{2.185060in}}{\pgfqpoint{2.206170in}{2.190234in}}{\pgfqpoint{2.193147in}{2.190234in}}%
\pgfpathcurveto{\pgfqpoint{2.180124in}{2.190234in}}{\pgfqpoint{2.167633in}{2.185060in}}{\pgfqpoint{2.158425in}{2.175852in}}%
\pgfpathcurveto{\pgfqpoint{2.149216in}{2.166644in}}{\pgfqpoint{2.144042in}{2.154153in}}{\pgfqpoint{2.144042in}{2.141130in}}%
\pgfpathcurveto{\pgfqpoint{2.144042in}{2.128107in}}{\pgfqpoint{2.149216in}{2.115616in}}{\pgfqpoint{2.158425in}{2.106408in}}%
\pgfpathcurveto{\pgfqpoint{2.167633in}{2.097199in}}{\pgfqpoint{2.180124in}{2.092025in}}{\pgfqpoint{2.193147in}{2.092025in}}%
\pgfpathlineto{\pgfqpoint{2.193147in}{2.092025in}}%
\pgfpathclose%
\pgfusepath{stroke,fill}%
\end{pgfscope}%
\begin{pgfscope}%
\pgfpathrectangle{\pgfqpoint{0.786164in}{0.768110in}}{\pgfqpoint{8.851069in}{7.081890in}}%
\pgfusepath{clip}%
\pgfsetbuttcap%
\pgfsetroundjoin%
\definecolor{currentfill}{rgb}{0.229739,0.322361,0.545706}%
\pgfsetfillcolor{currentfill}%
\pgfsetfillopacity{0.700000}%
\pgfsetlinewidth{0.501875pt}%
\definecolor{currentstroke}{rgb}{1.000000,1.000000,1.000000}%
\pgfsetstrokecolor{currentstroke}%
\pgfsetstrokeopacity{0.700000}%
\pgfsetdash{}{0pt}%
\pgfpathmoveto{\pgfqpoint{2.083547in}{2.223415in}}%
\pgfpathcurveto{\pgfqpoint{2.096570in}{2.223415in}}{\pgfqpoint{2.109061in}{2.228589in}}{\pgfqpoint{2.118270in}{2.237797in}}%
\pgfpathcurveto{\pgfqpoint{2.127478in}{2.247005in}}{\pgfqpoint{2.132652in}{2.259497in}}{\pgfqpoint{2.132652in}{2.272519in}}%
\pgfpathcurveto{\pgfqpoint{2.132652in}{2.285542in}}{\pgfqpoint{2.127478in}{2.298033in}}{\pgfqpoint{2.118270in}{2.307241in}}%
\pgfpathcurveto{\pgfqpoint{2.109061in}{2.316450in}}{\pgfqpoint{2.096570in}{2.321624in}}{\pgfqpoint{2.083547in}{2.321624in}}%
\pgfpathcurveto{\pgfqpoint{2.070525in}{2.321624in}}{\pgfqpoint{2.058034in}{2.316450in}}{\pgfqpoint{2.048825in}{2.307241in}}%
\pgfpathcurveto{\pgfqpoint{2.039617in}{2.298033in}}{\pgfqpoint{2.034443in}{2.285542in}}{\pgfqpoint{2.034443in}{2.272519in}}%
\pgfpathcurveto{\pgfqpoint{2.034443in}{2.259497in}}{\pgfqpoint{2.039617in}{2.247005in}}{\pgfqpoint{2.048825in}{2.237797in}}%
\pgfpathcurveto{\pgfqpoint{2.058034in}{2.228589in}}{\pgfqpoint{2.070525in}{2.223415in}}{\pgfqpoint{2.083547in}{2.223415in}}%
\pgfpathlineto{\pgfqpoint{2.083547in}{2.223415in}}%
\pgfpathclose%
\pgfusepath{stroke,fill}%
\end{pgfscope}%
\begin{pgfscope}%
\pgfpathrectangle{\pgfqpoint{0.786164in}{0.768110in}}{\pgfqpoint{8.851069in}{7.081890in}}%
\pgfusepath{clip}%
\pgfsetbuttcap%
\pgfsetroundjoin%
\definecolor{currentfill}{rgb}{0.227802,0.326594,0.546532}%
\pgfsetfillcolor{currentfill}%
\pgfsetfillopacity{0.700000}%
\pgfsetlinewidth{0.501875pt}%
\definecolor{currentstroke}{rgb}{1.000000,1.000000,1.000000}%
\pgfsetstrokecolor{currentstroke}%
\pgfsetstrokeopacity{0.700000}%
\pgfsetdash{}{0pt}%
\pgfpathmoveto{\pgfqpoint{2.074414in}{2.289109in}}%
\pgfpathcurveto{\pgfqpoint{2.087437in}{2.289109in}}{\pgfqpoint{2.099928in}{2.294283in}}{\pgfqpoint{2.109136in}{2.303492in}}%
\pgfpathcurveto{\pgfqpoint{2.118345in}{2.312700in}}{\pgfqpoint{2.123519in}{2.325191in}}{\pgfqpoint{2.123519in}{2.338214in}}%
\pgfpathcurveto{\pgfqpoint{2.123519in}{2.351237in}}{\pgfqpoint{2.118345in}{2.363728in}}{\pgfqpoint{2.109136in}{2.372936in}}%
\pgfpathcurveto{\pgfqpoint{2.099928in}{2.382145in}}{\pgfqpoint{2.087437in}{2.387319in}}{\pgfqpoint{2.074414in}{2.387319in}}%
\pgfpathcurveto{\pgfqpoint{2.061391in}{2.387319in}}{\pgfqpoint{2.048900in}{2.382145in}}{\pgfqpoint{2.039692in}{2.372936in}}%
\pgfpathcurveto{\pgfqpoint{2.030483in}{2.363728in}}{\pgfqpoint{2.025309in}{2.351237in}}{\pgfqpoint{2.025309in}{2.338214in}}%
\pgfpathcurveto{\pgfqpoint{2.025309in}{2.325191in}}{\pgfqpoint{2.030483in}{2.312700in}}{\pgfqpoint{2.039692in}{2.303492in}}%
\pgfpathcurveto{\pgfqpoint{2.048900in}{2.294283in}}{\pgfqpoint{2.061391in}{2.289109in}}{\pgfqpoint{2.074414in}{2.289109in}}%
\pgfpathlineto{\pgfqpoint{2.074414in}{2.289109in}}%
\pgfpathclose%
\pgfusepath{stroke,fill}%
\end{pgfscope}%
\begin{pgfscope}%
\pgfpathrectangle{\pgfqpoint{0.786164in}{0.768110in}}{\pgfqpoint{8.851069in}{7.081890in}}%
\pgfusepath{clip}%
\pgfsetbuttcap%
\pgfsetroundjoin%
\definecolor{currentfill}{rgb}{0.231674,0.318106,0.544834}%
\pgfsetfillcolor{currentfill}%
\pgfsetfillopacity{0.700000}%
\pgfsetlinewidth{0.501875pt}%
\definecolor{currentstroke}{rgb}{1.000000,1.000000,1.000000}%
\pgfsetstrokecolor{currentstroke}%
\pgfsetstrokeopacity{0.700000}%
\pgfsetdash{}{0pt}%
\pgfpathmoveto{\pgfqpoint{2.165747in}{2.332906in}}%
\pgfpathcurveto{\pgfqpoint{2.178770in}{2.332906in}}{\pgfqpoint{2.191261in}{2.338080in}}{\pgfqpoint{2.200469in}{2.347288in}}%
\pgfpathcurveto{\pgfqpoint{2.209678in}{2.356497in}}{\pgfqpoint{2.214852in}{2.368988in}}{\pgfqpoint{2.214852in}{2.382010in}}%
\pgfpathcurveto{\pgfqpoint{2.214852in}{2.395033in}}{\pgfqpoint{2.209678in}{2.407524in}}{\pgfqpoint{2.200469in}{2.416733in}}%
\pgfpathcurveto{\pgfqpoint{2.191261in}{2.425941in}}{\pgfqpoint{2.178770in}{2.431115in}}{\pgfqpoint{2.165747in}{2.431115in}}%
\pgfpathcurveto{\pgfqpoint{2.152724in}{2.431115in}}{\pgfqpoint{2.140233in}{2.425941in}}{\pgfqpoint{2.131025in}{2.416733in}}%
\pgfpathcurveto{\pgfqpoint{2.121816in}{2.407524in}}{\pgfqpoint{2.116642in}{2.395033in}}{\pgfqpoint{2.116642in}{2.382010in}}%
\pgfpathcurveto{\pgfqpoint{2.116642in}{2.368988in}}{\pgfqpoint{2.121816in}{2.356497in}}{\pgfqpoint{2.131025in}{2.347288in}}%
\pgfpathcurveto{\pgfqpoint{2.140233in}{2.338080in}}{\pgfqpoint{2.152724in}{2.332906in}}{\pgfqpoint{2.165747in}{2.332906in}}%
\pgfpathlineto{\pgfqpoint{2.165747in}{2.332906in}}%
\pgfpathclose%
\pgfusepath{stroke,fill}%
\end{pgfscope}%
\begin{pgfscope}%
\pgfpathrectangle{\pgfqpoint{0.786164in}{0.768110in}}{\pgfqpoint{8.851069in}{7.081890in}}%
\pgfusepath{clip}%
\pgfsetbuttcap%
\pgfsetroundjoin%
\definecolor{currentfill}{rgb}{0.221989,0.339161,0.548752}%
\pgfsetfillcolor{currentfill}%
\pgfsetfillopacity{0.700000}%
\pgfsetlinewidth{0.501875pt}%
\definecolor{currentstroke}{rgb}{1.000000,1.000000,1.000000}%
\pgfsetstrokecolor{currentstroke}%
\pgfsetstrokeopacity{0.700000}%
\pgfsetdash{}{0pt}%
\pgfpathmoveto{\pgfqpoint{2.229680in}{2.223415in}}%
\pgfpathcurveto{\pgfqpoint{2.242703in}{2.223415in}}{\pgfqpoint{2.255194in}{2.228589in}}{\pgfqpoint{2.264402in}{2.237797in}}%
\pgfpathcurveto{\pgfqpoint{2.273611in}{2.247005in}}{\pgfqpoint{2.278785in}{2.259497in}}{\pgfqpoint{2.278785in}{2.272519in}}%
\pgfpathcurveto{\pgfqpoint{2.278785in}{2.285542in}}{\pgfqpoint{2.273611in}{2.298033in}}{\pgfqpoint{2.264402in}{2.307241in}}%
\pgfpathcurveto{\pgfqpoint{2.255194in}{2.316450in}}{\pgfqpoint{2.242703in}{2.321624in}}{\pgfqpoint{2.229680in}{2.321624in}}%
\pgfpathcurveto{\pgfqpoint{2.216657in}{2.321624in}}{\pgfqpoint{2.204166in}{2.316450in}}{\pgfqpoint{2.194958in}{2.307241in}}%
\pgfpathcurveto{\pgfqpoint{2.185749in}{2.298033in}}{\pgfqpoint{2.180575in}{2.285542in}}{\pgfqpoint{2.180575in}{2.272519in}}%
\pgfpathcurveto{\pgfqpoint{2.180575in}{2.259497in}}{\pgfqpoint{2.185749in}{2.247005in}}{\pgfqpoint{2.194958in}{2.237797in}}%
\pgfpathcurveto{\pgfqpoint{2.204166in}{2.228589in}}{\pgfqpoint{2.216657in}{2.223415in}}{\pgfqpoint{2.229680in}{2.223415in}}%
\pgfpathlineto{\pgfqpoint{2.229680in}{2.223415in}}%
\pgfpathclose%
\pgfusepath{stroke,fill}%
\end{pgfscope}%
\begin{pgfscope}%
\pgfpathrectangle{\pgfqpoint{0.786164in}{0.768110in}}{\pgfqpoint{8.851069in}{7.081890in}}%
\pgfusepath{clip}%
\pgfsetbuttcap%
\pgfsetroundjoin%
\definecolor{currentfill}{rgb}{0.206756,0.371758,0.553117}%
\pgfsetfillcolor{currentfill}%
\pgfsetfillopacity{0.700000}%
\pgfsetlinewidth{0.501875pt}%
\definecolor{currentstroke}{rgb}{1.000000,1.000000,1.000000}%
\pgfsetstrokecolor{currentstroke}%
\pgfsetstrokeopacity{0.700000}%
\pgfsetdash{}{0pt}%
\pgfpathmoveto{\pgfqpoint{2.156614in}{1.894941in}}%
\pgfpathcurveto{\pgfqpoint{2.169636in}{1.894941in}}{\pgfqpoint{2.182127in}{1.900115in}}{\pgfqpoint{2.191336in}{1.909323in}}%
\pgfpathcurveto{\pgfqpoint{2.200544in}{1.918532in}}{\pgfqpoint{2.205718in}{1.931023in}}{\pgfqpoint{2.205718in}{1.944046in}}%
\pgfpathcurveto{\pgfqpoint{2.205718in}{1.957068in}}{\pgfqpoint{2.200544in}{1.969559in}}{\pgfqpoint{2.191336in}{1.978768in}}%
\pgfpathcurveto{\pgfqpoint{2.182127in}{1.987976in}}{\pgfqpoint{2.169636in}{1.993150in}}{\pgfqpoint{2.156614in}{1.993150in}}%
\pgfpathcurveto{\pgfqpoint{2.143591in}{1.993150in}}{\pgfqpoint{2.131100in}{1.987976in}}{\pgfqpoint{2.121891in}{1.978768in}}%
\pgfpathcurveto{\pgfqpoint{2.112683in}{1.969559in}}{\pgfqpoint{2.107509in}{1.957068in}}{\pgfqpoint{2.107509in}{1.944046in}}%
\pgfpathcurveto{\pgfqpoint{2.107509in}{1.931023in}}{\pgfqpoint{2.112683in}{1.918532in}}{\pgfqpoint{2.121891in}{1.909323in}}%
\pgfpathcurveto{\pgfqpoint{2.131100in}{1.900115in}}{\pgfqpoint{2.143591in}{1.894941in}}{\pgfqpoint{2.156614in}{1.894941in}}%
\pgfpathlineto{\pgfqpoint{2.156614in}{1.894941in}}%
\pgfpathclose%
\pgfusepath{stroke,fill}%
\end{pgfscope}%
\begin{pgfscope}%
\pgfpathrectangle{\pgfqpoint{0.786164in}{0.768110in}}{\pgfqpoint{8.851069in}{7.081890in}}%
\pgfusepath{clip}%
\pgfsetbuttcap%
\pgfsetroundjoin%
\definecolor{currentfill}{rgb}{0.208623,0.367752,0.552675}%
\pgfsetfillcolor{currentfill}%
\pgfsetfillopacity{0.700000}%
\pgfsetlinewidth{0.501875pt}%
\definecolor{currentstroke}{rgb}{1.000000,1.000000,1.000000}%
\pgfsetstrokecolor{currentstroke}%
\pgfsetstrokeopacity{0.700000}%
\pgfsetdash{}{0pt}%
\pgfpathmoveto{\pgfqpoint{1.690816in}{1.763552in}}%
\pgfpathcurveto{\pgfqpoint{1.703839in}{1.763552in}}{\pgfqpoint{1.716330in}{1.768726in}}{\pgfqpoint{1.725538in}{1.777934in}}%
\pgfpathcurveto{\pgfqpoint{1.734747in}{1.787143in}}{\pgfqpoint{1.739921in}{1.799634in}}{\pgfqpoint{1.739921in}{1.812656in}}%
\pgfpathcurveto{\pgfqpoint{1.739921in}{1.825679in}}{\pgfqpoint{1.734747in}{1.838170in}}{\pgfqpoint{1.725538in}{1.847379in}}%
\pgfpathcurveto{\pgfqpoint{1.716330in}{1.856587in}}{\pgfqpoint{1.703839in}{1.861761in}}{\pgfqpoint{1.690816in}{1.861761in}}%
\pgfpathcurveto{\pgfqpoint{1.677793in}{1.861761in}}{\pgfqpoint{1.665302in}{1.856587in}}{\pgfqpoint{1.656094in}{1.847379in}}%
\pgfpathcurveto{\pgfqpoint{1.646885in}{1.838170in}}{\pgfqpoint{1.641711in}{1.825679in}}{\pgfqpoint{1.641711in}{1.812656in}}%
\pgfpathcurveto{\pgfqpoint{1.641711in}{1.799634in}}{\pgfqpoint{1.646885in}{1.787143in}}{\pgfqpoint{1.656094in}{1.777934in}}%
\pgfpathcurveto{\pgfqpoint{1.665302in}{1.768726in}}{\pgfqpoint{1.677793in}{1.763552in}}{\pgfqpoint{1.690816in}{1.763552in}}%
\pgfpathlineto{\pgfqpoint{1.690816in}{1.763552in}}%
\pgfpathclose%
\pgfusepath{stroke,fill}%
\end{pgfscope}%
\begin{pgfscope}%
\pgfpathrectangle{\pgfqpoint{0.786164in}{0.768110in}}{\pgfqpoint{8.851069in}{7.081890in}}%
\pgfusepath{clip}%
\pgfsetbuttcap%
\pgfsetroundjoin%
\definecolor{currentfill}{rgb}{0.204903,0.375746,0.553533}%
\pgfsetfillcolor{currentfill}%
\pgfsetfillopacity{0.700000}%
\pgfsetlinewidth{0.501875pt}%
\definecolor{currentstroke}{rgb}{1.000000,1.000000,1.000000}%
\pgfsetstrokecolor{currentstroke}%
\pgfsetstrokeopacity{0.700000}%
\pgfsetdash{}{0pt}%
\pgfpathmoveto{\pgfqpoint{1.654283in}{1.785450in}}%
\pgfpathcurveto{\pgfqpoint{1.667306in}{1.785450in}}{\pgfqpoint{1.679797in}{1.790624in}}{\pgfqpoint{1.689005in}{1.799832in}}%
\pgfpathcurveto{\pgfqpoint{1.698214in}{1.809041in}}{\pgfqpoint{1.703388in}{1.821532in}}{\pgfqpoint{1.703388in}{1.834555in}}%
\pgfpathcurveto{\pgfqpoint{1.703388in}{1.847577in}}{\pgfqpoint{1.698214in}{1.860068in}}{\pgfqpoint{1.689005in}{1.869277in}}%
\pgfpathcurveto{\pgfqpoint{1.679797in}{1.878485in}}{\pgfqpoint{1.667306in}{1.883659in}}{\pgfqpoint{1.654283in}{1.883659in}}%
\pgfpathcurveto{\pgfqpoint{1.641260in}{1.883659in}}{\pgfqpoint{1.628769in}{1.878485in}}{\pgfqpoint{1.619561in}{1.869277in}}%
\pgfpathcurveto{\pgfqpoint{1.610352in}{1.860068in}}{\pgfqpoint{1.605178in}{1.847577in}}{\pgfqpoint{1.605178in}{1.834555in}}%
\pgfpathcurveto{\pgfqpoint{1.605178in}{1.821532in}}{\pgfqpoint{1.610352in}{1.809041in}}{\pgfqpoint{1.619561in}{1.799832in}}%
\pgfpathcurveto{\pgfqpoint{1.628769in}{1.790624in}}{\pgfqpoint{1.641260in}{1.785450in}}{\pgfqpoint{1.654283in}{1.785450in}}%
\pgfpathlineto{\pgfqpoint{1.654283in}{1.785450in}}%
\pgfpathclose%
\pgfusepath{stroke,fill}%
\end{pgfscope}%
\begin{pgfscope}%
\pgfpathrectangle{\pgfqpoint{0.786164in}{0.768110in}}{\pgfqpoint{8.851069in}{7.081890in}}%
\pgfusepath{clip}%
\pgfsetbuttcap%
\pgfsetroundjoin%
\definecolor{currentfill}{rgb}{0.195860,0.395433,0.555276}%
\pgfsetfillcolor{currentfill}%
\pgfsetfillopacity{0.700000}%
\pgfsetlinewidth{0.501875pt}%
\definecolor{currentstroke}{rgb}{1.000000,1.000000,1.000000}%
\pgfsetstrokecolor{currentstroke}%
\pgfsetstrokeopacity{0.700000}%
\pgfsetdash{}{0pt}%
\pgfpathmoveto{\pgfqpoint{1.699949in}{1.829246in}}%
\pgfpathcurveto{\pgfqpoint{1.712972in}{1.829246in}}{\pgfqpoint{1.725463in}{1.834420in}}{\pgfqpoint{1.734672in}{1.843629in}}%
\pgfpathcurveto{\pgfqpoint{1.743880in}{1.852837in}}{\pgfqpoint{1.749054in}{1.865328in}}{\pgfqpoint{1.749054in}{1.878351in}}%
\pgfpathcurveto{\pgfqpoint{1.749054in}{1.891374in}}{\pgfqpoint{1.743880in}{1.903865in}}{\pgfqpoint{1.734672in}{1.913073in}}%
\pgfpathcurveto{\pgfqpoint{1.725463in}{1.922282in}}{\pgfqpoint{1.712972in}{1.927456in}}{\pgfqpoint{1.699949in}{1.927456in}}%
\pgfpathcurveto{\pgfqpoint{1.686927in}{1.927456in}}{\pgfqpoint{1.674436in}{1.922282in}}{\pgfqpoint{1.665227in}{1.913073in}}%
\pgfpathcurveto{\pgfqpoint{1.656019in}{1.903865in}}{\pgfqpoint{1.650845in}{1.891374in}}{\pgfqpoint{1.650845in}{1.878351in}}%
\pgfpathcurveto{\pgfqpoint{1.650845in}{1.865328in}}{\pgfqpoint{1.656019in}{1.852837in}}{\pgfqpoint{1.665227in}{1.843629in}}%
\pgfpathcurveto{\pgfqpoint{1.674436in}{1.834420in}}{\pgfqpoint{1.686927in}{1.829246in}}{\pgfqpoint{1.699949in}{1.829246in}}%
\pgfpathlineto{\pgfqpoint{1.699949in}{1.829246in}}%
\pgfpathclose%
\pgfusepath{stroke,fill}%
\end{pgfscope}%
\begin{pgfscope}%
\pgfpathrectangle{\pgfqpoint{0.786164in}{0.768110in}}{\pgfqpoint{8.851069in}{7.081890in}}%
\pgfusepath{clip}%
\pgfsetbuttcap%
\pgfsetroundjoin%
\definecolor{currentfill}{rgb}{0.187231,0.414746,0.556547}%
\pgfsetfillcolor{currentfill}%
\pgfsetfillopacity{0.700000}%
\pgfsetlinewidth{0.501875pt}%
\definecolor{currentstroke}{rgb}{1.000000,1.000000,1.000000}%
\pgfsetstrokecolor{currentstroke}%
\pgfsetstrokeopacity{0.700000}%
\pgfsetdash{}{0pt}%
\pgfpathmoveto{\pgfqpoint{1.663416in}{1.807348in}}%
\pgfpathcurveto{\pgfqpoint{1.676439in}{1.807348in}}{\pgfqpoint{1.688930in}{1.812522in}}{\pgfqpoint{1.698138in}{1.821731in}}%
\pgfpathcurveto{\pgfqpoint{1.707347in}{1.830939in}}{\pgfqpoint{1.712521in}{1.843430in}}{\pgfqpoint{1.712521in}{1.856453in}}%
\pgfpathcurveto{\pgfqpoint{1.712521in}{1.869475in}}{\pgfqpoint{1.707347in}{1.881967in}}{\pgfqpoint{1.698138in}{1.891175in}}%
\pgfpathcurveto{\pgfqpoint{1.688930in}{1.900383in}}{\pgfqpoint{1.676439in}{1.905557in}}{\pgfqpoint{1.663416in}{1.905557in}}%
\pgfpathcurveto{\pgfqpoint{1.650393in}{1.905557in}}{\pgfqpoint{1.637902in}{1.900383in}}{\pgfqpoint{1.628694in}{1.891175in}}%
\pgfpathcurveto{\pgfqpoint{1.619485in}{1.881967in}}{\pgfqpoint{1.614312in}{1.869475in}}{\pgfqpoint{1.614312in}{1.856453in}}%
\pgfpathcurveto{\pgfqpoint{1.614312in}{1.843430in}}{\pgfqpoint{1.619485in}{1.830939in}}{\pgfqpoint{1.628694in}{1.821731in}}%
\pgfpathcurveto{\pgfqpoint{1.637902in}{1.812522in}}{\pgfqpoint{1.650393in}{1.807348in}}{\pgfqpoint{1.663416in}{1.807348in}}%
\pgfpathlineto{\pgfqpoint{1.663416in}{1.807348in}}%
\pgfpathclose%
\pgfusepath{stroke,fill}%
\end{pgfscope}%
\begin{pgfscope}%
\pgfpathrectangle{\pgfqpoint{0.786164in}{0.768110in}}{\pgfqpoint{8.851069in}{7.081890in}}%
\pgfusepath{clip}%
\pgfsetbuttcap%
\pgfsetroundjoin%
\definecolor{currentfill}{rgb}{0.180629,0.429975,0.557282}%
\pgfsetfillcolor{currentfill}%
\pgfsetfillopacity{0.700000}%
\pgfsetlinewidth{0.501875pt}%
\definecolor{currentstroke}{rgb}{1.000000,1.000000,1.000000}%
\pgfsetstrokecolor{currentstroke}%
\pgfsetstrokeopacity{0.700000}%
\pgfsetdash{}{0pt}%
\pgfpathmoveto{\pgfqpoint{1.663416in}{1.851145in}}%
\pgfpathcurveto{\pgfqpoint{1.676439in}{1.851145in}}{\pgfqpoint{1.688930in}{1.856319in}}{\pgfqpoint{1.698138in}{1.865527in}}%
\pgfpathcurveto{\pgfqpoint{1.707347in}{1.874735in}}{\pgfqpoint{1.712521in}{1.887227in}}{\pgfqpoint{1.712521in}{1.900249in}}%
\pgfpathcurveto{\pgfqpoint{1.712521in}{1.913272in}}{\pgfqpoint{1.707347in}{1.925763in}}{\pgfqpoint{1.698138in}{1.934971in}}%
\pgfpathcurveto{\pgfqpoint{1.688930in}{1.944180in}}{\pgfqpoint{1.676439in}{1.949354in}}{\pgfqpoint{1.663416in}{1.949354in}}%
\pgfpathcurveto{\pgfqpoint{1.650393in}{1.949354in}}{\pgfqpoint{1.637902in}{1.944180in}}{\pgfqpoint{1.628694in}{1.934971in}}%
\pgfpathcurveto{\pgfqpoint{1.619485in}{1.925763in}}{\pgfqpoint{1.614312in}{1.913272in}}{\pgfqpoint{1.614312in}{1.900249in}}%
\pgfpathcurveto{\pgfqpoint{1.614312in}{1.887227in}}{\pgfqpoint{1.619485in}{1.874735in}}{\pgfqpoint{1.628694in}{1.865527in}}%
\pgfpathcurveto{\pgfqpoint{1.637902in}{1.856319in}}{\pgfqpoint{1.650393in}{1.851145in}}{\pgfqpoint{1.663416in}{1.851145in}}%
\pgfpathlineto{\pgfqpoint{1.663416in}{1.851145in}}%
\pgfpathclose%
\pgfusepath{stroke,fill}%
\end{pgfscope}%
\begin{pgfscope}%
\pgfpathrectangle{\pgfqpoint{0.786164in}{0.768110in}}{\pgfqpoint{8.851069in}{7.081890in}}%
\pgfusepath{clip}%
\pgfsetbuttcap%
\pgfsetroundjoin%
\definecolor{currentfill}{rgb}{0.168126,0.459988,0.558082}%
\pgfsetfillcolor{currentfill}%
\pgfsetfillopacity{0.700000}%
\pgfsetlinewidth{0.501875pt}%
\definecolor{currentstroke}{rgb}{1.000000,1.000000,1.000000}%
\pgfsetstrokecolor{currentstroke}%
\pgfsetstrokeopacity{0.700000}%
\pgfsetdash{}{0pt}%
\pgfpathmoveto{\pgfqpoint{1.690816in}{1.960636in}}%
\pgfpathcurveto{\pgfqpoint{1.703839in}{1.960636in}}{\pgfqpoint{1.716330in}{1.965810in}}{\pgfqpoint{1.725538in}{1.975018in}}%
\pgfpathcurveto{\pgfqpoint{1.734747in}{1.984227in}}{\pgfqpoint{1.739921in}{1.996718in}}{\pgfqpoint{1.739921in}{2.009740in}}%
\pgfpathcurveto{\pgfqpoint{1.739921in}{2.022763in}}{\pgfqpoint{1.734747in}{2.035254in}}{\pgfqpoint{1.725538in}{2.044463in}}%
\pgfpathcurveto{\pgfqpoint{1.716330in}{2.053671in}}{\pgfqpoint{1.703839in}{2.058845in}}{\pgfqpoint{1.690816in}{2.058845in}}%
\pgfpathcurveto{\pgfqpoint{1.677793in}{2.058845in}}{\pgfqpoint{1.665302in}{2.053671in}}{\pgfqpoint{1.656094in}{2.044463in}}%
\pgfpathcurveto{\pgfqpoint{1.646885in}{2.035254in}}{\pgfqpoint{1.641711in}{2.022763in}}{\pgfqpoint{1.641711in}{2.009740in}}%
\pgfpathcurveto{\pgfqpoint{1.641711in}{1.996718in}}{\pgfqpoint{1.646885in}{1.984227in}}{\pgfqpoint{1.656094in}{1.975018in}}%
\pgfpathcurveto{\pgfqpoint{1.665302in}{1.965810in}}{\pgfqpoint{1.677793in}{1.960636in}}{\pgfqpoint{1.690816in}{1.960636in}}%
\pgfpathlineto{\pgfqpoint{1.690816in}{1.960636in}}%
\pgfpathclose%
\pgfusepath{stroke,fill}%
\end{pgfscope}%
\begin{pgfscope}%
\pgfpathrectangle{\pgfqpoint{0.786164in}{0.768110in}}{\pgfqpoint{8.851069in}{7.081890in}}%
\pgfusepath{clip}%
\pgfsetbuttcap%
\pgfsetroundjoin%
\definecolor{currentfill}{rgb}{0.168126,0.459988,0.558082}%
\pgfsetfillcolor{currentfill}%
\pgfsetfillopacity{0.700000}%
\pgfsetlinewidth{0.501875pt}%
\definecolor{currentstroke}{rgb}{1.000000,1.000000,1.000000}%
\pgfsetstrokecolor{currentstroke}%
\pgfsetstrokeopacity{0.700000}%
\pgfsetdash{}{0pt}%
\pgfpathmoveto{\pgfqpoint{1.800415in}{2.026330in}}%
\pgfpathcurveto{\pgfqpoint{1.813438in}{2.026330in}}{\pgfqpoint{1.825929in}{2.031504in}}{\pgfqpoint{1.835138in}{2.040713in}}%
\pgfpathcurveto{\pgfqpoint{1.844346in}{2.049921in}}{\pgfqpoint{1.849520in}{2.062412in}}{\pgfqpoint{1.849520in}{2.075435in}}%
\pgfpathcurveto{\pgfqpoint{1.849520in}{2.088458in}}{\pgfqpoint{1.844346in}{2.100949in}}{\pgfqpoint{1.835138in}{2.110157in}}%
\pgfpathcurveto{\pgfqpoint{1.825929in}{2.119366in}}{\pgfqpoint{1.813438in}{2.124540in}}{\pgfqpoint{1.800415in}{2.124540in}}%
\pgfpathcurveto{\pgfqpoint{1.787393in}{2.124540in}}{\pgfqpoint{1.774902in}{2.119366in}}{\pgfqpoint{1.765693in}{2.110157in}}%
\pgfpathcurveto{\pgfqpoint{1.756485in}{2.100949in}}{\pgfqpoint{1.751311in}{2.088458in}}{\pgfqpoint{1.751311in}{2.075435in}}%
\pgfpathcurveto{\pgfqpoint{1.751311in}{2.062412in}}{\pgfqpoint{1.756485in}{2.049921in}}{\pgfqpoint{1.765693in}{2.040713in}}%
\pgfpathcurveto{\pgfqpoint{1.774902in}{2.031504in}}{\pgfqpoint{1.787393in}{2.026330in}}{\pgfqpoint{1.800415in}{2.026330in}}%
\pgfpathlineto{\pgfqpoint{1.800415in}{2.026330in}}%
\pgfpathclose%
\pgfusepath{stroke,fill}%
\end{pgfscope}%
\begin{pgfscope}%
\pgfpathrectangle{\pgfqpoint{0.786164in}{0.768110in}}{\pgfqpoint{8.851069in}{7.081890in}}%
\pgfusepath{clip}%
\pgfsetbuttcap%
\pgfsetroundjoin%
\definecolor{currentfill}{rgb}{0.166617,0.463708,0.558119}%
\pgfsetfillcolor{currentfill}%
\pgfsetfillopacity{0.700000}%
\pgfsetlinewidth{0.501875pt}%
\definecolor{currentstroke}{rgb}{1.000000,1.000000,1.000000}%
\pgfsetstrokecolor{currentstroke}%
\pgfsetstrokeopacity{0.700000}%
\pgfsetdash{}{0pt}%
\pgfpathmoveto{\pgfqpoint{1.864348in}{2.113923in}}%
\pgfpathcurveto{\pgfqpoint{1.877371in}{2.113923in}}{\pgfqpoint{1.889862in}{2.119097in}}{\pgfqpoint{1.899071in}{2.128306in}}%
\pgfpathcurveto{\pgfqpoint{1.908279in}{2.137514in}}{\pgfqpoint{1.913453in}{2.150005in}}{\pgfqpoint{1.913453in}{2.163028in}}%
\pgfpathcurveto{\pgfqpoint{1.913453in}{2.176051in}}{\pgfqpoint{1.908279in}{2.188542in}}{\pgfqpoint{1.899071in}{2.197750in}}%
\pgfpathcurveto{\pgfqpoint{1.889862in}{2.206959in}}{\pgfqpoint{1.877371in}{2.212133in}}{\pgfqpoint{1.864348in}{2.212133in}}%
\pgfpathcurveto{\pgfqpoint{1.851326in}{2.212133in}}{\pgfqpoint{1.838835in}{2.206959in}}{\pgfqpoint{1.829626in}{2.197750in}}%
\pgfpathcurveto{\pgfqpoint{1.820418in}{2.188542in}}{\pgfqpoint{1.815244in}{2.176051in}}{\pgfqpoint{1.815244in}{2.163028in}}%
\pgfpathcurveto{\pgfqpoint{1.815244in}{2.150005in}}{\pgfqpoint{1.820418in}{2.137514in}}{\pgfqpoint{1.829626in}{2.128306in}}%
\pgfpathcurveto{\pgfqpoint{1.838835in}{2.119097in}}{\pgfqpoint{1.851326in}{2.113923in}}{\pgfqpoint{1.864348in}{2.113923in}}%
\pgfpathlineto{\pgfqpoint{1.864348in}{2.113923in}}%
\pgfpathclose%
\pgfusepath{stroke,fill}%
\end{pgfscope}%
\begin{pgfscope}%
\pgfpathrectangle{\pgfqpoint{0.786164in}{0.768110in}}{\pgfqpoint{8.851069in}{7.081890in}}%
\pgfusepath{clip}%
\pgfsetbuttcap%
\pgfsetroundjoin%
\definecolor{currentfill}{rgb}{0.174274,0.445044,0.557792}%
\pgfsetfillcolor{currentfill}%
\pgfsetfillopacity{0.700000}%
\pgfsetlinewidth{0.501875pt}%
\definecolor{currentstroke}{rgb}{1.000000,1.000000,1.000000}%
\pgfsetstrokecolor{currentstroke}%
\pgfsetstrokeopacity{0.700000}%
\pgfsetdash{}{0pt}%
\pgfpathmoveto{\pgfqpoint{1.946548in}{2.157720in}}%
\pgfpathcurveto{\pgfqpoint{1.959571in}{2.157720in}}{\pgfqpoint{1.972062in}{2.162894in}}{\pgfqpoint{1.981270in}{2.172102in}}%
\pgfpathcurveto{\pgfqpoint{1.990479in}{2.181311in}}{\pgfqpoint{1.995653in}{2.193802in}}{\pgfqpoint{1.995653in}{2.206825in}}%
\pgfpathcurveto{\pgfqpoint{1.995653in}{2.219847in}}{\pgfqpoint{1.990479in}{2.232338in}}{\pgfqpoint{1.981270in}{2.241547in}}%
\pgfpathcurveto{\pgfqpoint{1.972062in}{2.250755in}}{\pgfqpoint{1.959571in}{2.255929in}}{\pgfqpoint{1.946548in}{2.255929in}}%
\pgfpathcurveto{\pgfqpoint{1.933525in}{2.255929in}}{\pgfqpoint{1.921034in}{2.250755in}}{\pgfqpoint{1.911826in}{2.241547in}}%
\pgfpathcurveto{\pgfqpoint{1.902617in}{2.232338in}}{\pgfqpoint{1.897443in}{2.219847in}}{\pgfqpoint{1.897443in}{2.206825in}}%
\pgfpathcurveto{\pgfqpoint{1.897443in}{2.193802in}}{\pgfqpoint{1.902617in}{2.181311in}}{\pgfqpoint{1.911826in}{2.172102in}}%
\pgfpathcurveto{\pgfqpoint{1.921034in}{2.162894in}}{\pgfqpoint{1.933525in}{2.157720in}}{\pgfqpoint{1.946548in}{2.157720in}}%
\pgfpathlineto{\pgfqpoint{1.946548in}{2.157720in}}%
\pgfpathclose%
\pgfusepath{stroke,fill}%
\end{pgfscope}%
\begin{pgfscope}%
\pgfpathrectangle{\pgfqpoint{0.786164in}{0.768110in}}{\pgfqpoint{8.851069in}{7.081890in}}%
\pgfusepath{clip}%
\pgfsetbuttcap%
\pgfsetroundjoin%
\definecolor{currentfill}{rgb}{0.169646,0.456262,0.558030}%
\pgfsetfillcolor{currentfill}%
\pgfsetfillopacity{0.700000}%
\pgfsetlinewidth{0.501875pt}%
\definecolor{currentstroke}{rgb}{1.000000,1.000000,1.000000}%
\pgfsetstrokecolor{currentstroke}%
\pgfsetstrokeopacity{0.700000}%
\pgfsetdash{}{0pt}%
\pgfpathmoveto{\pgfqpoint{1.928281in}{2.157720in}}%
\pgfpathcurveto{\pgfqpoint{1.941304in}{2.157720in}}{\pgfqpoint{1.953795in}{2.162894in}}{\pgfqpoint{1.963004in}{2.172102in}}%
\pgfpathcurveto{\pgfqpoint{1.972212in}{2.181311in}}{\pgfqpoint{1.977386in}{2.193802in}}{\pgfqpoint{1.977386in}{2.206825in}}%
\pgfpathcurveto{\pgfqpoint{1.977386in}{2.219847in}}{\pgfqpoint{1.972212in}{2.232338in}}{\pgfqpoint{1.963004in}{2.241547in}}%
\pgfpathcurveto{\pgfqpoint{1.953795in}{2.250755in}}{\pgfqpoint{1.941304in}{2.255929in}}{\pgfqpoint{1.928281in}{2.255929in}}%
\pgfpathcurveto{\pgfqpoint{1.915259in}{2.255929in}}{\pgfqpoint{1.902768in}{2.250755in}}{\pgfqpoint{1.893559in}{2.241547in}}%
\pgfpathcurveto{\pgfqpoint{1.884351in}{2.232338in}}{\pgfqpoint{1.879177in}{2.219847in}}{\pgfqpoint{1.879177in}{2.206825in}}%
\pgfpathcurveto{\pgfqpoint{1.879177in}{2.193802in}}{\pgfqpoint{1.884351in}{2.181311in}}{\pgfqpoint{1.893559in}{2.172102in}}%
\pgfpathcurveto{\pgfqpoint{1.902768in}{2.162894in}}{\pgfqpoint{1.915259in}{2.157720in}}{\pgfqpoint{1.928281in}{2.157720in}}%
\pgfpathlineto{\pgfqpoint{1.928281in}{2.157720in}}%
\pgfpathclose%
\pgfusepath{stroke,fill}%
\end{pgfscope}%
\begin{pgfscope}%
\pgfpathrectangle{\pgfqpoint{0.786164in}{0.768110in}}{\pgfqpoint{8.851069in}{7.081890in}}%
\pgfusepath{clip}%
\pgfsetbuttcap%
\pgfsetroundjoin%
\definecolor{currentfill}{rgb}{0.162142,0.474838,0.558140}%
\pgfsetfillcolor{currentfill}%
\pgfsetfillopacity{0.700000}%
\pgfsetlinewidth{0.501875pt}%
\definecolor{currentstroke}{rgb}{1.000000,1.000000,1.000000}%
\pgfsetstrokecolor{currentstroke}%
\pgfsetstrokeopacity{0.700000}%
\pgfsetdash{}{0pt}%
\pgfpathmoveto{\pgfqpoint{1.910015in}{2.135822in}}%
\pgfpathcurveto{\pgfqpoint{1.923038in}{2.135822in}}{\pgfqpoint{1.935529in}{2.140996in}}{\pgfqpoint{1.944737in}{2.150204in}}%
\pgfpathcurveto{\pgfqpoint{1.953946in}{2.159413in}}{\pgfqpoint{1.959120in}{2.171904in}}{\pgfqpoint{1.959120in}{2.184926in}}%
\pgfpathcurveto{\pgfqpoint{1.959120in}{2.197949in}}{\pgfqpoint{1.953946in}{2.210440in}}{\pgfqpoint{1.944737in}{2.219649in}}%
\pgfpathcurveto{\pgfqpoint{1.935529in}{2.228857in}}{\pgfqpoint{1.923038in}{2.234031in}}{\pgfqpoint{1.910015in}{2.234031in}}%
\pgfpathcurveto{\pgfqpoint{1.896992in}{2.234031in}}{\pgfqpoint{1.884501in}{2.228857in}}{\pgfqpoint{1.875293in}{2.219649in}}%
\pgfpathcurveto{\pgfqpoint{1.866084in}{2.210440in}}{\pgfqpoint{1.860910in}{2.197949in}}{\pgfqpoint{1.860910in}{2.184926in}}%
\pgfpathcurveto{\pgfqpoint{1.860910in}{2.171904in}}{\pgfqpoint{1.866084in}{2.159413in}}{\pgfqpoint{1.875293in}{2.150204in}}%
\pgfpathcurveto{\pgfqpoint{1.884501in}{2.140996in}}{\pgfqpoint{1.896992in}{2.135822in}}{\pgfqpoint{1.910015in}{2.135822in}}%
\pgfpathlineto{\pgfqpoint{1.910015in}{2.135822in}}%
\pgfpathclose%
\pgfusepath{stroke,fill}%
\end{pgfscope}%
\begin{pgfscope}%
\pgfpathrectangle{\pgfqpoint{0.786164in}{0.768110in}}{\pgfqpoint{8.851069in}{7.081890in}}%
\pgfusepath{clip}%
\pgfsetbuttcap%
\pgfsetroundjoin%
\definecolor{currentfill}{rgb}{0.153364,0.497000,0.557724}%
\pgfsetfillcolor{currentfill}%
\pgfsetfillopacity{0.700000}%
\pgfsetlinewidth{0.501875pt}%
\definecolor{currentstroke}{rgb}{1.000000,1.000000,1.000000}%
\pgfsetstrokecolor{currentstroke}%
\pgfsetstrokeopacity{0.700000}%
\pgfsetdash{}{0pt}%
\pgfpathmoveto{\pgfqpoint{2.037881in}{2.179618in}}%
\pgfpathcurveto{\pgfqpoint{2.050904in}{2.179618in}}{\pgfqpoint{2.063395in}{2.184792in}}{\pgfqpoint{2.072603in}{2.194001in}}%
\pgfpathcurveto{\pgfqpoint{2.081812in}{2.203209in}}{\pgfqpoint{2.086986in}{2.215700in}}{\pgfqpoint{2.086986in}{2.228723in}}%
\pgfpathcurveto{\pgfqpoint{2.086986in}{2.241745in}}{\pgfqpoint{2.081812in}{2.254237in}}{\pgfqpoint{2.072603in}{2.263445in}}%
\pgfpathcurveto{\pgfqpoint{2.063395in}{2.272653in}}{\pgfqpoint{2.050904in}{2.277827in}}{\pgfqpoint{2.037881in}{2.277827in}}%
\pgfpathcurveto{\pgfqpoint{2.024858in}{2.277827in}}{\pgfqpoint{2.012367in}{2.272653in}}{\pgfqpoint{2.003159in}{2.263445in}}%
\pgfpathcurveto{\pgfqpoint{1.993950in}{2.254237in}}{\pgfqpoint{1.988776in}{2.241745in}}{\pgfqpoint{1.988776in}{2.228723in}}%
\pgfpathcurveto{\pgfqpoint{1.988776in}{2.215700in}}{\pgfqpoint{1.993950in}{2.203209in}}{\pgfqpoint{2.003159in}{2.194001in}}%
\pgfpathcurveto{\pgfqpoint{2.012367in}{2.184792in}}{\pgfqpoint{2.024858in}{2.179618in}}{\pgfqpoint{2.037881in}{2.179618in}}%
\pgfpathlineto{\pgfqpoint{2.037881in}{2.179618in}}%
\pgfpathclose%
\pgfusepath{stroke,fill}%
\end{pgfscope}%
\begin{pgfscope}%
\pgfpathrectangle{\pgfqpoint{0.786164in}{0.768110in}}{\pgfqpoint{8.851069in}{7.081890in}}%
\pgfusepath{clip}%
\pgfsetbuttcap%
\pgfsetroundjoin%
\definecolor{currentfill}{rgb}{0.146180,0.515413,0.556823}%
\pgfsetfillcolor{currentfill}%
\pgfsetfillopacity{0.700000}%
\pgfsetlinewidth{0.501875pt}%
\definecolor{currentstroke}{rgb}{1.000000,1.000000,1.000000}%
\pgfsetstrokecolor{currentstroke}%
\pgfsetstrokeopacity{0.700000}%
\pgfsetdash{}{0pt}%
\pgfpathmoveto{\pgfqpoint{1.910015in}{2.026330in}}%
\pgfpathcurveto{\pgfqpoint{1.923038in}{2.026330in}}{\pgfqpoint{1.935529in}{2.031504in}}{\pgfqpoint{1.944737in}{2.040713in}}%
\pgfpathcurveto{\pgfqpoint{1.953946in}{2.049921in}}{\pgfqpoint{1.959120in}{2.062412in}}{\pgfqpoint{1.959120in}{2.075435in}}%
\pgfpathcurveto{\pgfqpoint{1.959120in}{2.088458in}}{\pgfqpoint{1.953946in}{2.100949in}}{\pgfqpoint{1.944737in}{2.110157in}}%
\pgfpathcurveto{\pgfqpoint{1.935529in}{2.119366in}}{\pgfqpoint{1.923038in}{2.124540in}}{\pgfqpoint{1.910015in}{2.124540in}}%
\pgfpathcurveto{\pgfqpoint{1.896992in}{2.124540in}}{\pgfqpoint{1.884501in}{2.119366in}}{\pgfqpoint{1.875293in}{2.110157in}}%
\pgfpathcurveto{\pgfqpoint{1.866084in}{2.100949in}}{\pgfqpoint{1.860910in}{2.088458in}}{\pgfqpoint{1.860910in}{2.075435in}}%
\pgfpathcurveto{\pgfqpoint{1.860910in}{2.062412in}}{\pgfqpoint{1.866084in}{2.049921in}}{\pgfqpoint{1.875293in}{2.040713in}}%
\pgfpathcurveto{\pgfqpoint{1.884501in}{2.031504in}}{\pgfqpoint{1.896992in}{2.026330in}}{\pgfqpoint{1.910015in}{2.026330in}}%
\pgfpathlineto{\pgfqpoint{1.910015in}{2.026330in}}%
\pgfpathclose%
\pgfusepath{stroke,fill}%
\end{pgfscope}%
\begin{pgfscope}%
\pgfpathrectangle{\pgfqpoint{0.786164in}{0.768110in}}{\pgfqpoint{8.851069in}{7.081890in}}%
\pgfusepath{clip}%
\pgfsetbuttcap%
\pgfsetroundjoin%
\definecolor{currentfill}{rgb}{0.271305,0.019942,0.347269}%
\pgfsetfillcolor{currentfill}%
\pgfsetfillopacity{0.700000}%
\pgfsetlinewidth{0.501875pt}%
\definecolor{currentstroke}{rgb}{1.000000,1.000000,1.000000}%
\pgfsetstrokecolor{currentstroke}%
\pgfsetstrokeopacity{0.700000}%
\pgfsetdash{}{0pt}%
\pgfpathmoveto{\pgfqpoint{9.161845in}{7.435195in}}%
\pgfpathcurveto{\pgfqpoint{9.174868in}{7.435195in}}{\pgfqpoint{9.187359in}{7.440369in}}{\pgfqpoint{9.196567in}{7.449577in}}%
\pgfpathcurveto{\pgfqpoint{9.205776in}{7.458786in}}{\pgfqpoint{9.210950in}{7.471277in}}{\pgfqpoint{9.210950in}{7.484299in}}%
\pgfpathcurveto{\pgfqpoint{9.210950in}{7.497322in}}{\pgfqpoint{9.205776in}{7.509813in}}{\pgfqpoint{9.196567in}{7.519022in}}%
\pgfpathcurveto{\pgfqpoint{9.187359in}{7.528230in}}{\pgfqpoint{9.174868in}{7.533404in}}{\pgfqpoint{9.161845in}{7.533404in}}%
\pgfpathcurveto{\pgfqpoint{9.148822in}{7.533404in}}{\pgfqpoint{9.136331in}{7.528230in}}{\pgfqpoint{9.127123in}{7.519022in}}%
\pgfpathcurveto{\pgfqpoint{9.117915in}{7.509813in}}{\pgfqpoint{9.112741in}{7.497322in}}{\pgfqpoint{9.112741in}{7.484299in}}%
\pgfpathcurveto{\pgfqpoint{9.112741in}{7.471277in}}{\pgfqpoint{9.117915in}{7.458786in}}{\pgfqpoint{9.127123in}{7.449577in}}%
\pgfpathcurveto{\pgfqpoint{9.136331in}{7.440369in}}{\pgfqpoint{9.148822in}{7.435195in}}{\pgfqpoint{9.161845in}{7.435195in}}%
\pgfpathlineto{\pgfqpoint{9.161845in}{7.435195in}}%
\pgfpathclose%
\pgfusepath{stroke,fill}%
\end{pgfscope}%
\begin{pgfscope}%
\pgfpathrectangle{\pgfqpoint{0.786164in}{0.768110in}}{\pgfqpoint{8.851069in}{7.081890in}}%
\pgfusepath{clip}%
\pgfsetbuttcap%
\pgfsetroundjoin%
\definecolor{currentfill}{rgb}{0.273809,0.031497,0.358853}%
\pgfsetfillcolor{currentfill}%
\pgfsetfillopacity{0.700000}%
\pgfsetlinewidth{0.501875pt}%
\definecolor{currentstroke}{rgb}{1.000000,1.000000,1.000000}%
\pgfsetstrokecolor{currentstroke}%
\pgfsetstrokeopacity{0.700000}%
\pgfsetdash{}{0pt}%
\pgfpathmoveto{\pgfqpoint{9.234911in}{7.478991in}}%
\pgfpathcurveto{\pgfqpoint{9.247934in}{7.478991in}}{\pgfqpoint{9.260425in}{7.484165in}}{\pgfqpoint{9.269634in}{7.493374in}}%
\pgfpathcurveto{\pgfqpoint{9.278842in}{7.502582in}}{\pgfqpoint{9.284016in}{7.515073in}}{\pgfqpoint{9.284016in}{7.528096in}}%
\pgfpathcurveto{\pgfqpoint{9.284016in}{7.541119in}}{\pgfqpoint{9.278842in}{7.553610in}}{\pgfqpoint{9.269634in}{7.562818in}}%
\pgfpathcurveto{\pgfqpoint{9.260425in}{7.572027in}}{\pgfqpoint{9.247934in}{7.577201in}}{\pgfqpoint{9.234911in}{7.577201in}}%
\pgfpathcurveto{\pgfqpoint{9.221889in}{7.577201in}}{\pgfqpoint{9.209398in}{7.572027in}}{\pgfqpoint{9.200189in}{7.562818in}}%
\pgfpathcurveto{\pgfqpoint{9.190981in}{7.553610in}}{\pgfqpoint{9.185807in}{7.541119in}}{\pgfqpoint{9.185807in}{7.528096in}}%
\pgfpathcurveto{\pgfqpoint{9.185807in}{7.515073in}}{\pgfqpoint{9.190981in}{7.502582in}}{\pgfqpoint{9.200189in}{7.493374in}}%
\pgfpathcurveto{\pgfqpoint{9.209398in}{7.484165in}}{\pgfqpoint{9.221889in}{7.478991in}}{\pgfqpoint{9.234911in}{7.478991in}}%
\pgfpathlineto{\pgfqpoint{9.234911in}{7.478991in}}%
\pgfpathclose%
\pgfusepath{stroke,fill}%
\end{pgfscope}%
\begin{pgfscope}%
\pgfpathrectangle{\pgfqpoint{0.786164in}{0.768110in}}{\pgfqpoint{8.851069in}{7.081890in}}%
\pgfusepath{clip}%
\pgfsetbuttcap%
\pgfsetroundjoin%
\definecolor{currentfill}{rgb}{0.273809,0.031497,0.358853}%
\pgfsetfillcolor{currentfill}%
\pgfsetfillopacity{0.700000}%
\pgfsetlinewidth{0.501875pt}%
\definecolor{currentstroke}{rgb}{1.000000,1.000000,1.000000}%
\pgfsetstrokecolor{currentstroke}%
\pgfsetstrokeopacity{0.700000}%
\pgfsetdash{}{0pt}%
\pgfpathmoveto{\pgfqpoint{8.933513in}{7.303805in}}%
\pgfpathcurveto{\pgfqpoint{8.946536in}{7.303805in}}{\pgfqpoint{8.959027in}{7.308979in}}{\pgfqpoint{8.968235in}{7.318188in}}%
\pgfpathcurveto{\pgfqpoint{8.977444in}{7.327396in}}{\pgfqpoint{8.982618in}{7.339887in}}{\pgfqpoint{8.982618in}{7.352910in}}%
\pgfpathcurveto{\pgfqpoint{8.982618in}{7.365933in}}{\pgfqpoint{8.977444in}{7.378424in}}{\pgfqpoint{8.968235in}{7.387632in}}%
\pgfpathcurveto{\pgfqpoint{8.959027in}{7.396841in}}{\pgfqpoint{8.946536in}{7.402015in}}{\pgfqpoint{8.933513in}{7.402015in}}%
\pgfpathcurveto{\pgfqpoint{8.920490in}{7.402015in}}{\pgfqpoint{8.907999in}{7.396841in}}{\pgfqpoint{8.898791in}{7.387632in}}%
\pgfpathcurveto{\pgfqpoint{8.889582in}{7.378424in}}{\pgfqpoint{8.884408in}{7.365933in}}{\pgfqpoint{8.884408in}{7.352910in}}%
\pgfpathcurveto{\pgfqpoint{8.884408in}{7.339887in}}{\pgfqpoint{8.889582in}{7.327396in}}{\pgfqpoint{8.898791in}{7.318188in}}%
\pgfpathcurveto{\pgfqpoint{8.907999in}{7.308979in}}{\pgfqpoint{8.920490in}{7.303805in}}{\pgfqpoint{8.933513in}{7.303805in}}%
\pgfpathlineto{\pgfqpoint{8.933513in}{7.303805in}}%
\pgfpathclose%
\pgfusepath{stroke,fill}%
\end{pgfscope}%
\begin{pgfscope}%
\pgfpathrectangle{\pgfqpoint{0.786164in}{0.768110in}}{\pgfqpoint{8.851069in}{7.081890in}}%
\pgfusepath{clip}%
\pgfsetbuttcap%
\pgfsetroundjoin%
\definecolor{currentfill}{rgb}{0.278791,0.062145,0.386592}%
\pgfsetfillcolor{currentfill}%
\pgfsetfillopacity{0.700000}%
\pgfsetlinewidth{0.501875pt}%
\definecolor{currentstroke}{rgb}{1.000000,1.000000,1.000000}%
\pgfsetstrokecolor{currentstroke}%
\pgfsetstrokeopacity{0.700000}%
\pgfsetdash{}{0pt}%
\pgfpathmoveto{\pgfqpoint{8.641248in}{7.019128in}}%
\pgfpathcurveto{\pgfqpoint{8.654270in}{7.019128in}}{\pgfqpoint{8.666762in}{7.024302in}}{\pgfqpoint{8.675970in}{7.033511in}}%
\pgfpathcurveto{\pgfqpoint{8.685178in}{7.042719in}}{\pgfqpoint{8.690352in}{7.055210in}}{\pgfqpoint{8.690352in}{7.068233in}}%
\pgfpathcurveto{\pgfqpoint{8.690352in}{7.081256in}}{\pgfqpoint{8.685178in}{7.093747in}}{\pgfqpoint{8.675970in}{7.102955in}}%
\pgfpathcurveto{\pgfqpoint{8.666762in}{7.112164in}}{\pgfqpoint{8.654270in}{7.117338in}}{\pgfqpoint{8.641248in}{7.117338in}}%
\pgfpathcurveto{\pgfqpoint{8.628225in}{7.117338in}}{\pgfqpoint{8.615734in}{7.112164in}}{\pgfqpoint{8.606526in}{7.102955in}}%
\pgfpathcurveto{\pgfqpoint{8.597317in}{7.093747in}}{\pgfqpoint{8.592143in}{7.081256in}}{\pgfqpoint{8.592143in}{7.068233in}}%
\pgfpathcurveto{\pgfqpoint{8.592143in}{7.055210in}}{\pgfqpoint{8.597317in}{7.042719in}}{\pgfqpoint{8.606526in}{7.033511in}}%
\pgfpathcurveto{\pgfqpoint{8.615734in}{7.024302in}}{\pgfqpoint{8.628225in}{7.019128in}}{\pgfqpoint{8.641248in}{7.019128in}}%
\pgfpathlineto{\pgfqpoint{8.641248in}{7.019128in}}%
\pgfpathclose%
\pgfusepath{stroke,fill}%
\end{pgfscope}%
\begin{pgfscope}%
\pgfpathrectangle{\pgfqpoint{0.786164in}{0.768110in}}{\pgfqpoint{8.851069in}{7.081890in}}%
\pgfusepath{clip}%
\pgfsetbuttcap%
\pgfsetroundjoin%
\definecolor{currentfill}{rgb}{0.278791,0.062145,0.386592}%
\pgfsetfillcolor{currentfill}%
\pgfsetfillopacity{0.700000}%
\pgfsetlinewidth{0.501875pt}%
\definecolor{currentstroke}{rgb}{1.000000,1.000000,1.000000}%
\pgfsetstrokecolor{currentstroke}%
\pgfsetstrokeopacity{0.700000}%
\pgfsetdash{}{0pt}%
\pgfpathmoveto{\pgfqpoint{8.485982in}{6.865841in}}%
\pgfpathcurveto{\pgfqpoint{8.499005in}{6.865841in}}{\pgfqpoint{8.511496in}{6.871015in}}{\pgfqpoint{8.520704in}{6.880223in}}%
\pgfpathcurveto{\pgfqpoint{8.529913in}{6.889432in}}{\pgfqpoint{8.535087in}{6.901923in}}{\pgfqpoint{8.535087in}{6.914945in}}%
\pgfpathcurveto{\pgfqpoint{8.535087in}{6.927968in}}{\pgfqpoint{8.529913in}{6.940459in}}{\pgfqpoint{8.520704in}{6.949668in}}%
\pgfpathcurveto{\pgfqpoint{8.511496in}{6.958876in}}{\pgfqpoint{8.499005in}{6.964050in}}{\pgfqpoint{8.485982in}{6.964050in}}%
\pgfpathcurveto{\pgfqpoint{8.472959in}{6.964050in}}{\pgfqpoint{8.460468in}{6.958876in}}{\pgfqpoint{8.451260in}{6.949668in}}%
\pgfpathcurveto{\pgfqpoint{8.442051in}{6.940459in}}{\pgfqpoint{8.436877in}{6.927968in}}{\pgfqpoint{8.436877in}{6.914945in}}%
\pgfpathcurveto{\pgfqpoint{8.436877in}{6.901923in}}{\pgfqpoint{8.442051in}{6.889432in}}{\pgfqpoint{8.451260in}{6.880223in}}%
\pgfpathcurveto{\pgfqpoint{8.460468in}{6.871015in}}{\pgfqpoint{8.472959in}{6.865841in}}{\pgfqpoint{8.485982in}{6.865841in}}%
\pgfpathlineto{\pgfqpoint{8.485982in}{6.865841in}}%
\pgfpathclose%
\pgfusepath{stroke,fill}%
\end{pgfscope}%
\begin{pgfscope}%
\pgfpathrectangle{\pgfqpoint{0.786164in}{0.768110in}}{\pgfqpoint{8.851069in}{7.081890in}}%
\pgfusepath{clip}%
\pgfsetbuttcap%
\pgfsetroundjoin%
\definecolor{currentfill}{rgb}{0.278791,0.062145,0.386592}%
\pgfsetfillcolor{currentfill}%
\pgfsetfillopacity{0.700000}%
\pgfsetlinewidth{0.501875pt}%
\definecolor{currentstroke}{rgb}{1.000000,1.000000,1.000000}%
\pgfsetstrokecolor{currentstroke}%
\pgfsetstrokeopacity{0.700000}%
\pgfsetdash{}{0pt}%
\pgfpathmoveto{\pgfqpoint{7.837518in}{6.493571in}}%
\pgfpathcurveto{\pgfqpoint{7.850541in}{6.493571in}}{\pgfqpoint{7.863032in}{6.498745in}}{\pgfqpoint{7.872241in}{6.507953in}}%
\pgfpathcurveto{\pgfqpoint{7.881449in}{6.517162in}}{\pgfqpoint{7.886623in}{6.529653in}}{\pgfqpoint{7.886623in}{6.542675in}}%
\pgfpathcurveto{\pgfqpoint{7.886623in}{6.555698in}}{\pgfqpoint{7.881449in}{6.568189in}}{\pgfqpoint{7.872241in}{6.577398in}}%
\pgfpathcurveto{\pgfqpoint{7.863032in}{6.586606in}}{\pgfqpoint{7.850541in}{6.591780in}}{\pgfqpoint{7.837518in}{6.591780in}}%
\pgfpathcurveto{\pgfqpoint{7.824496in}{6.591780in}}{\pgfqpoint{7.812005in}{6.586606in}}{\pgfqpoint{7.802796in}{6.577398in}}%
\pgfpathcurveto{\pgfqpoint{7.793588in}{6.568189in}}{\pgfqpoint{7.788414in}{6.555698in}}{\pgfqpoint{7.788414in}{6.542675in}}%
\pgfpathcurveto{\pgfqpoint{7.788414in}{6.529653in}}{\pgfqpoint{7.793588in}{6.517162in}}{\pgfqpoint{7.802796in}{6.507953in}}%
\pgfpathcurveto{\pgfqpoint{7.812005in}{6.498745in}}{\pgfqpoint{7.824496in}{6.493571in}}{\pgfqpoint{7.837518in}{6.493571in}}%
\pgfpathlineto{\pgfqpoint{7.837518in}{6.493571in}}%
\pgfpathclose%
\pgfusepath{stroke,fill}%
\end{pgfscope}%
\begin{pgfscope}%
\pgfpathrectangle{\pgfqpoint{0.786164in}{0.768110in}}{\pgfqpoint{8.851069in}{7.081890in}}%
\pgfusepath{clip}%
\pgfsetbuttcap%
\pgfsetroundjoin%
\definecolor{currentfill}{rgb}{0.278791,0.062145,0.386592}%
\pgfsetfillcolor{currentfill}%
\pgfsetfillopacity{0.700000}%
\pgfsetlinewidth{0.501875pt}%
\definecolor{currentstroke}{rgb}{1.000000,1.000000,1.000000}%
\pgfsetstrokecolor{currentstroke}%
\pgfsetstrokeopacity{0.700000}%
\pgfsetdash{}{0pt}%
\pgfpathmoveto{\pgfqpoint{8.175450in}{6.734451in}}%
\pgfpathcurveto{\pgfqpoint{8.188473in}{6.734451in}}{\pgfqpoint{8.200964in}{6.739625in}}{\pgfqpoint{8.210172in}{6.748834in}}%
\pgfpathcurveto{\pgfqpoint{8.219381in}{6.758042in}}{\pgfqpoint{8.224555in}{6.770533in}}{\pgfqpoint{8.224555in}{6.783556in}}%
\pgfpathcurveto{\pgfqpoint{8.224555in}{6.796579in}}{\pgfqpoint{8.219381in}{6.809070in}}{\pgfqpoint{8.210172in}{6.818278in}}%
\pgfpathcurveto{\pgfqpoint{8.200964in}{6.827487in}}{\pgfqpoint{8.188473in}{6.832661in}}{\pgfqpoint{8.175450in}{6.832661in}}%
\pgfpathcurveto{\pgfqpoint{8.162427in}{6.832661in}}{\pgfqpoint{8.149936in}{6.827487in}}{\pgfqpoint{8.140728in}{6.818278in}}%
\pgfpathcurveto{\pgfqpoint{8.131519in}{6.809070in}}{\pgfqpoint{8.126345in}{6.796579in}}{\pgfqpoint{8.126345in}{6.783556in}}%
\pgfpathcurveto{\pgfqpoint{8.126345in}{6.770533in}}{\pgfqpoint{8.131519in}{6.758042in}}{\pgfqpoint{8.140728in}{6.748834in}}%
\pgfpathcurveto{\pgfqpoint{8.149936in}{6.739625in}}{\pgfqpoint{8.162427in}{6.734451in}}{\pgfqpoint{8.175450in}{6.734451in}}%
\pgfpathlineto{\pgfqpoint{8.175450in}{6.734451in}}%
\pgfpathclose%
\pgfusepath{stroke,fill}%
\end{pgfscope}%
\begin{pgfscope}%
\pgfpathrectangle{\pgfqpoint{0.786164in}{0.768110in}}{\pgfqpoint{8.851069in}{7.081890in}}%
\pgfusepath{clip}%
\pgfsetbuttcap%
\pgfsetroundjoin%
\definecolor{currentfill}{rgb}{0.278791,0.062145,0.386592}%
\pgfsetfillcolor{currentfill}%
\pgfsetfillopacity{0.700000}%
\pgfsetlinewidth{0.501875pt}%
\definecolor{currentstroke}{rgb}{1.000000,1.000000,1.000000}%
\pgfsetstrokecolor{currentstroke}%
\pgfsetstrokeopacity{0.700000}%
\pgfsetdash{}{0pt}%
\pgfpathmoveto{\pgfqpoint{7.855785in}{6.493571in}}%
\pgfpathcurveto{\pgfqpoint{7.868808in}{6.493571in}}{\pgfqpoint{7.881299in}{6.498745in}}{\pgfqpoint{7.890507in}{6.507953in}}%
\pgfpathcurveto{\pgfqpoint{7.899716in}{6.517162in}}{\pgfqpoint{7.904890in}{6.529653in}}{\pgfqpoint{7.904890in}{6.542675in}}%
\pgfpathcurveto{\pgfqpoint{7.904890in}{6.555698in}}{\pgfqpoint{7.899716in}{6.568189in}}{\pgfqpoint{7.890507in}{6.577398in}}%
\pgfpathcurveto{\pgfqpoint{7.881299in}{6.586606in}}{\pgfqpoint{7.868808in}{6.591780in}}{\pgfqpoint{7.855785in}{6.591780in}}%
\pgfpathcurveto{\pgfqpoint{7.842762in}{6.591780in}}{\pgfqpoint{7.830271in}{6.586606in}}{\pgfqpoint{7.821063in}{6.577398in}}%
\pgfpathcurveto{\pgfqpoint{7.811854in}{6.568189in}}{\pgfqpoint{7.806680in}{6.555698in}}{\pgfqpoint{7.806680in}{6.542675in}}%
\pgfpathcurveto{\pgfqpoint{7.806680in}{6.529653in}}{\pgfqpoint{7.811854in}{6.517162in}}{\pgfqpoint{7.821063in}{6.507953in}}%
\pgfpathcurveto{\pgfqpoint{7.830271in}{6.498745in}}{\pgfqpoint{7.842762in}{6.493571in}}{\pgfqpoint{7.855785in}{6.493571in}}%
\pgfpathlineto{\pgfqpoint{7.855785in}{6.493571in}}%
\pgfpathclose%
\pgfusepath{stroke,fill}%
\end{pgfscope}%
\begin{pgfscope}%
\pgfpathrectangle{\pgfqpoint{0.786164in}{0.768110in}}{\pgfqpoint{8.851069in}{7.081890in}}%
\pgfusepath{clip}%
\pgfsetbuttcap%
\pgfsetroundjoin%
\definecolor{currentfill}{rgb}{0.279566,0.067836,0.391917}%
\pgfsetfillcolor{currentfill}%
\pgfsetfillopacity{0.700000}%
\pgfsetlinewidth{0.501875pt}%
\definecolor{currentstroke}{rgb}{1.000000,1.000000,1.000000}%
\pgfsetstrokecolor{currentstroke}%
\pgfsetstrokeopacity{0.700000}%
\pgfsetdash{}{0pt}%
\pgfpathmoveto{\pgfqpoint{7.472187in}{6.208894in}}%
\pgfpathcurveto{\pgfqpoint{7.485210in}{6.208894in}}{\pgfqpoint{7.497701in}{6.214068in}}{\pgfqpoint{7.506909in}{6.223276in}}%
\pgfpathcurveto{\pgfqpoint{7.516118in}{6.232484in}}{\pgfqpoint{7.521292in}{6.244976in}}{\pgfqpoint{7.521292in}{6.257998in}}%
\pgfpathcurveto{\pgfqpoint{7.521292in}{6.271021in}}{\pgfqpoint{7.516118in}{6.283512in}}{\pgfqpoint{7.506909in}{6.292720in}}%
\pgfpathcurveto{\pgfqpoint{7.497701in}{6.301929in}}{\pgfqpoint{7.485210in}{6.307103in}}{\pgfqpoint{7.472187in}{6.307103in}}%
\pgfpathcurveto{\pgfqpoint{7.459164in}{6.307103in}}{\pgfqpoint{7.446673in}{6.301929in}}{\pgfqpoint{7.437465in}{6.292720in}}%
\pgfpathcurveto{\pgfqpoint{7.428256in}{6.283512in}}{\pgfqpoint{7.423082in}{6.271021in}}{\pgfqpoint{7.423082in}{6.257998in}}%
\pgfpathcurveto{\pgfqpoint{7.423082in}{6.244976in}}{\pgfqpoint{7.428256in}{6.232484in}}{\pgfqpoint{7.437465in}{6.223276in}}%
\pgfpathcurveto{\pgfqpoint{7.446673in}{6.214068in}}{\pgfqpoint{7.459164in}{6.208894in}}{\pgfqpoint{7.472187in}{6.208894in}}%
\pgfpathlineto{\pgfqpoint{7.472187in}{6.208894in}}%
\pgfpathclose%
\pgfusepath{stroke,fill}%
\end{pgfscope}%
\begin{pgfscope}%
\pgfpathrectangle{\pgfqpoint{0.786164in}{0.768110in}}{\pgfqpoint{8.851069in}{7.081890in}}%
\pgfusepath{clip}%
\pgfsetbuttcap%
\pgfsetroundjoin%
\definecolor{currentfill}{rgb}{0.280894,0.078907,0.402329}%
\pgfsetfillcolor{currentfill}%
\pgfsetfillopacity{0.700000}%
\pgfsetlinewidth{0.501875pt}%
\definecolor{currentstroke}{rgb}{1.000000,1.000000,1.000000}%
\pgfsetstrokecolor{currentstroke}%
\pgfsetstrokeopacity{0.700000}%
\pgfsetdash{}{0pt}%
\pgfpathmoveto{\pgfqpoint{7.097722in}{5.814725in}}%
\pgfpathcurveto{\pgfqpoint{7.110745in}{5.814725in}}{\pgfqpoint{7.123236in}{5.819899in}}{\pgfqpoint{7.132444in}{5.829108in}}%
\pgfpathcurveto{\pgfqpoint{7.141653in}{5.838316in}}{\pgfqpoint{7.146827in}{5.850807in}}{\pgfqpoint{7.146827in}{5.863830in}}%
\pgfpathcurveto{\pgfqpoint{7.146827in}{5.876853in}}{\pgfqpoint{7.141653in}{5.889344in}}{\pgfqpoint{7.132444in}{5.898552in}}%
\pgfpathcurveto{\pgfqpoint{7.123236in}{5.907761in}}{\pgfqpoint{7.110745in}{5.912935in}}{\pgfqpoint{7.097722in}{5.912935in}}%
\pgfpathcurveto{\pgfqpoint{7.084699in}{5.912935in}}{\pgfqpoint{7.072208in}{5.907761in}}{\pgfqpoint{7.063000in}{5.898552in}}%
\pgfpathcurveto{\pgfqpoint{7.053792in}{5.889344in}}{\pgfqpoint{7.048618in}{5.876853in}}{\pgfqpoint{7.048618in}{5.863830in}}%
\pgfpathcurveto{\pgfqpoint{7.048618in}{5.850807in}}{\pgfqpoint{7.053792in}{5.838316in}}{\pgfqpoint{7.063000in}{5.829108in}}%
\pgfpathcurveto{\pgfqpoint{7.072208in}{5.819899in}}{\pgfqpoint{7.084699in}{5.814725in}}{\pgfqpoint{7.097722in}{5.814725in}}%
\pgfpathlineto{\pgfqpoint{7.097722in}{5.814725in}}%
\pgfpathclose%
\pgfusepath{stroke,fill}%
\end{pgfscope}%
\begin{pgfscope}%
\pgfpathrectangle{\pgfqpoint{0.786164in}{0.768110in}}{\pgfqpoint{8.851069in}{7.081890in}}%
\pgfusepath{clip}%
\pgfsetbuttcap%
\pgfsetroundjoin%
\definecolor{currentfill}{rgb}{0.282327,0.094955,0.417331}%
\pgfsetfillcolor{currentfill}%
\pgfsetfillopacity{0.700000}%
\pgfsetlinewidth{0.501875pt}%
\definecolor{currentstroke}{rgb}{1.000000,1.000000,1.000000}%
\pgfsetstrokecolor{currentstroke}%
\pgfsetstrokeopacity{0.700000}%
\pgfsetdash{}{0pt}%
\pgfpathmoveto{\pgfqpoint{6.750657in}{5.573845in}}%
\pgfpathcurveto{\pgfqpoint{6.763680in}{5.573845in}}{\pgfqpoint{6.776171in}{5.579019in}}{\pgfqpoint{6.785379in}{5.588227in}}%
\pgfpathcurveto{\pgfqpoint{6.794588in}{5.597436in}}{\pgfqpoint{6.799762in}{5.609927in}}{\pgfqpoint{6.799762in}{5.622949in}}%
\pgfpathcurveto{\pgfqpoint{6.799762in}{5.635972in}}{\pgfqpoint{6.794588in}{5.648463in}}{\pgfqpoint{6.785379in}{5.657672in}}%
\pgfpathcurveto{\pgfqpoint{6.776171in}{5.666880in}}{\pgfqpoint{6.763680in}{5.672054in}}{\pgfqpoint{6.750657in}{5.672054in}}%
\pgfpathcurveto{\pgfqpoint{6.737635in}{5.672054in}}{\pgfqpoint{6.725143in}{5.666880in}}{\pgfqpoint{6.715935in}{5.657672in}}%
\pgfpathcurveto{\pgfqpoint{6.706727in}{5.648463in}}{\pgfqpoint{6.701553in}{5.635972in}}{\pgfqpoint{6.701553in}{5.622949in}}%
\pgfpathcurveto{\pgfqpoint{6.701553in}{5.609927in}}{\pgfqpoint{6.706727in}{5.597436in}}{\pgfqpoint{6.715935in}{5.588227in}}%
\pgfpathcurveto{\pgfqpoint{6.725143in}{5.579019in}}{\pgfqpoint{6.737635in}{5.573845in}}{\pgfqpoint{6.750657in}{5.573845in}}%
\pgfpathlineto{\pgfqpoint{6.750657in}{5.573845in}}%
\pgfpathclose%
\pgfusepath{stroke,fill}%
\end{pgfscope}%
\begin{pgfscope}%
\pgfpathrectangle{\pgfqpoint{0.786164in}{0.768110in}}{\pgfqpoint{8.851069in}{7.081890in}}%
\pgfusepath{clip}%
\pgfsetbuttcap%
\pgfsetroundjoin%
\definecolor{currentfill}{rgb}{0.282910,0.105393,0.426902}%
\pgfsetfillcolor{currentfill}%
\pgfsetfillopacity{0.700000}%
\pgfsetlinewidth{0.501875pt}%
\definecolor{currentstroke}{rgb}{1.000000,1.000000,1.000000}%
\pgfsetstrokecolor{currentstroke}%
\pgfsetstrokeopacity{0.700000}%
\pgfsetdash{}{0pt}%
\pgfpathmoveto{\pgfqpoint{6.513192in}{5.376761in}}%
\pgfpathcurveto{\pgfqpoint{6.526215in}{5.376761in}}{\pgfqpoint{6.538706in}{5.381935in}}{\pgfqpoint{6.547914in}{5.391143in}}%
\pgfpathcurveto{\pgfqpoint{6.557122in}{5.400351in}}{\pgfqpoint{6.562296in}{5.412843in}}{\pgfqpoint{6.562296in}{5.425865in}}%
\pgfpathcurveto{\pgfqpoint{6.562296in}{5.438888in}}{\pgfqpoint{6.557122in}{5.451379in}}{\pgfqpoint{6.547914in}{5.460587in}}%
\pgfpathcurveto{\pgfqpoint{6.538706in}{5.469796in}}{\pgfqpoint{6.526215in}{5.474970in}}{\pgfqpoint{6.513192in}{5.474970in}}%
\pgfpathcurveto{\pgfqpoint{6.500169in}{5.474970in}}{\pgfqpoint{6.487678in}{5.469796in}}{\pgfqpoint{6.478470in}{5.460587in}}%
\pgfpathcurveto{\pgfqpoint{6.469261in}{5.451379in}}{\pgfqpoint{6.464087in}{5.438888in}}{\pgfqpoint{6.464087in}{5.425865in}}%
\pgfpathcurveto{\pgfqpoint{6.464087in}{5.412843in}}{\pgfqpoint{6.469261in}{5.400351in}}{\pgfqpoint{6.478470in}{5.391143in}}%
\pgfpathcurveto{\pgfqpoint{6.487678in}{5.381935in}}{\pgfqpoint{6.500169in}{5.376761in}}{\pgfqpoint{6.513192in}{5.376761in}}%
\pgfpathlineto{\pgfqpoint{6.513192in}{5.376761in}}%
\pgfpathclose%
\pgfusepath{stroke,fill}%
\end{pgfscope}%
\begin{pgfscope}%
\pgfpathrectangle{\pgfqpoint{0.786164in}{0.768110in}}{\pgfqpoint{8.851069in}{7.081890in}}%
\pgfusepath{clip}%
\pgfsetbuttcap%
\pgfsetroundjoin%
\definecolor{currentfill}{rgb}{0.283197,0.115680,0.436115}%
\pgfsetfillcolor{currentfill}%
\pgfsetfillopacity{0.700000}%
\pgfsetlinewidth{0.501875pt}%
\definecolor{currentstroke}{rgb}{1.000000,1.000000,1.000000}%
\pgfsetstrokecolor{currentstroke}%
\pgfsetstrokeopacity{0.700000}%
\pgfsetdash{}{0pt}%
\pgfpathmoveto{\pgfqpoint{6.385326in}{5.201575in}}%
\pgfpathcurveto{\pgfqpoint{6.398348in}{5.201575in}}{\pgfqpoint{6.410840in}{5.206749in}}{\pgfqpoint{6.420048in}{5.215957in}}%
\pgfpathcurveto{\pgfqpoint{6.429256in}{5.225166in}}{\pgfqpoint{6.434430in}{5.237657in}}{\pgfqpoint{6.434430in}{5.250679in}}%
\pgfpathcurveto{\pgfqpoint{6.434430in}{5.263702in}}{\pgfqpoint{6.429256in}{5.276193in}}{\pgfqpoint{6.420048in}{5.285402in}}%
\pgfpathcurveto{\pgfqpoint{6.410840in}{5.294610in}}{\pgfqpoint{6.398348in}{5.299784in}}{\pgfqpoint{6.385326in}{5.299784in}}%
\pgfpathcurveto{\pgfqpoint{6.372303in}{5.299784in}}{\pgfqpoint{6.359812in}{5.294610in}}{\pgfqpoint{6.350604in}{5.285402in}}%
\pgfpathcurveto{\pgfqpoint{6.341395in}{5.276193in}}{\pgfqpoint{6.336221in}{5.263702in}}{\pgfqpoint{6.336221in}{5.250679in}}%
\pgfpathcurveto{\pgfqpoint{6.336221in}{5.237657in}}{\pgfqpoint{6.341395in}{5.225166in}}{\pgfqpoint{6.350604in}{5.215957in}}%
\pgfpathcurveto{\pgfqpoint{6.359812in}{5.206749in}}{\pgfqpoint{6.372303in}{5.201575in}}{\pgfqpoint{6.385326in}{5.201575in}}%
\pgfpathlineto{\pgfqpoint{6.385326in}{5.201575in}}%
\pgfpathclose%
\pgfusepath{stroke,fill}%
\end{pgfscope}%
\begin{pgfscope}%
\pgfpathrectangle{\pgfqpoint{0.786164in}{0.768110in}}{\pgfqpoint{8.851069in}{7.081890in}}%
\pgfusepath{clip}%
\pgfsetbuttcap%
\pgfsetroundjoin%
\definecolor{currentfill}{rgb}{0.283072,0.130895,0.449241}%
\pgfsetfillcolor{currentfill}%
\pgfsetfillopacity{0.700000}%
\pgfsetlinewidth{0.501875pt}%
\definecolor{currentstroke}{rgb}{1.000000,1.000000,1.000000}%
\pgfsetstrokecolor{currentstroke}%
\pgfsetstrokeopacity{0.700000}%
\pgfsetdash{}{0pt}%
\pgfpathmoveto{\pgfqpoint{6.440125in}{5.289168in}}%
\pgfpathcurveto{\pgfqpoint{6.453148in}{5.289168in}}{\pgfqpoint{6.465639in}{5.294342in}}{\pgfqpoint{6.474848in}{5.303550in}}%
\pgfpathcurveto{\pgfqpoint{6.484056in}{5.312759in}}{\pgfqpoint{6.489230in}{5.325250in}}{\pgfqpoint{6.489230in}{5.338272in}}%
\pgfpathcurveto{\pgfqpoint{6.489230in}{5.351295in}}{\pgfqpoint{6.484056in}{5.363786in}}{\pgfqpoint{6.474848in}{5.372995in}}%
\pgfpathcurveto{\pgfqpoint{6.465639in}{5.382203in}}{\pgfqpoint{6.453148in}{5.387377in}}{\pgfqpoint{6.440125in}{5.387377in}}%
\pgfpathcurveto{\pgfqpoint{6.427103in}{5.387377in}}{\pgfqpoint{6.414612in}{5.382203in}}{\pgfqpoint{6.405403in}{5.372995in}}%
\pgfpathcurveto{\pgfqpoint{6.396195in}{5.363786in}}{\pgfqpoint{6.391021in}{5.351295in}}{\pgfqpoint{6.391021in}{5.338272in}}%
\pgfpathcurveto{\pgfqpoint{6.391021in}{5.325250in}}{\pgfqpoint{6.396195in}{5.312759in}}{\pgfqpoint{6.405403in}{5.303550in}}%
\pgfpathcurveto{\pgfqpoint{6.414612in}{5.294342in}}{\pgfqpoint{6.427103in}{5.289168in}}{\pgfqpoint{6.440125in}{5.289168in}}%
\pgfpathlineto{\pgfqpoint{6.440125in}{5.289168in}}%
\pgfpathclose%
\pgfusepath{stroke,fill}%
\end{pgfscope}%
\begin{pgfscope}%
\pgfpathrectangle{\pgfqpoint{0.786164in}{0.768110in}}{\pgfqpoint{8.851069in}{7.081890in}}%
\pgfusepath{clip}%
\pgfsetbuttcap%
\pgfsetroundjoin%
\definecolor{currentfill}{rgb}{0.277134,0.185228,0.489898}%
\pgfsetfillcolor{currentfill}%
\pgfsetfillopacity{0.700000}%
\pgfsetlinewidth{0.501875pt}%
\definecolor{currentstroke}{rgb}{1.000000,1.000000,1.000000}%
\pgfsetstrokecolor{currentstroke}%
\pgfsetstrokeopacity{0.700000}%
\pgfsetdash{}{0pt}%
\pgfpathmoveto{\pgfqpoint{6.577125in}{5.398659in}}%
\pgfpathcurveto{\pgfqpoint{6.590148in}{5.398659in}}{\pgfqpoint{6.602639in}{5.403833in}}{\pgfqpoint{6.611847in}{5.413041in}}%
\pgfpathcurveto{\pgfqpoint{6.621055in}{5.422250in}}{\pgfqpoint{6.626229in}{5.434741in}}{\pgfqpoint{6.626229in}{5.447763in}}%
\pgfpathcurveto{\pgfqpoint{6.626229in}{5.460786in}}{\pgfqpoint{6.621055in}{5.473277in}}{\pgfqpoint{6.611847in}{5.482486in}}%
\pgfpathcurveto{\pgfqpoint{6.602639in}{5.491694in}}{\pgfqpoint{6.590148in}{5.496868in}}{\pgfqpoint{6.577125in}{5.496868in}}%
\pgfpathcurveto{\pgfqpoint{6.564102in}{5.496868in}}{\pgfqpoint{6.551611in}{5.491694in}}{\pgfqpoint{6.542403in}{5.482486in}}%
\pgfpathcurveto{\pgfqpoint{6.533194in}{5.473277in}}{\pgfqpoint{6.528020in}{5.460786in}}{\pgfqpoint{6.528020in}{5.447763in}}%
\pgfpathcurveto{\pgfqpoint{6.528020in}{5.434741in}}{\pgfqpoint{6.533194in}{5.422250in}}{\pgfqpoint{6.542403in}{5.413041in}}%
\pgfpathcurveto{\pgfqpoint{6.551611in}{5.403833in}}{\pgfqpoint{6.564102in}{5.398659in}}{\pgfqpoint{6.577125in}{5.398659in}}%
\pgfpathlineto{\pgfqpoint{6.577125in}{5.398659in}}%
\pgfpathclose%
\pgfusepath{stroke,fill}%
\end{pgfscope}%
\begin{pgfscope}%
\pgfpathrectangle{\pgfqpoint{0.786164in}{0.768110in}}{\pgfqpoint{8.851069in}{7.081890in}}%
\pgfusepath{clip}%
\pgfsetbuttcap%
\pgfsetroundjoin%
\definecolor{currentfill}{rgb}{0.282290,0.145912,0.461510}%
\pgfsetfillcolor{currentfill}%
\pgfsetfillopacity{0.700000}%
\pgfsetlinewidth{0.501875pt}%
\definecolor{currentstroke}{rgb}{1.000000,1.000000,1.000000}%
\pgfsetstrokecolor{currentstroke}%
\pgfsetstrokeopacity{0.700000}%
\pgfsetdash{}{0pt}%
\pgfpathmoveto{\pgfqpoint{6.504059in}{5.311066in}}%
\pgfpathcurveto{\pgfqpoint{6.517081in}{5.311066in}}{\pgfqpoint{6.529572in}{5.316240in}}{\pgfqpoint{6.538781in}{5.325448in}}%
\pgfpathcurveto{\pgfqpoint{6.547989in}{5.334657in}}{\pgfqpoint{6.553163in}{5.347148in}}{\pgfqpoint{6.553163in}{5.360171in}}%
\pgfpathcurveto{\pgfqpoint{6.553163in}{5.373193in}}{\pgfqpoint{6.547989in}{5.385684in}}{\pgfqpoint{6.538781in}{5.394893in}}%
\pgfpathcurveto{\pgfqpoint{6.529572in}{5.404101in}}{\pgfqpoint{6.517081in}{5.409275in}}{\pgfqpoint{6.504059in}{5.409275in}}%
\pgfpathcurveto{\pgfqpoint{6.491036in}{5.409275in}}{\pgfqpoint{6.478545in}{5.404101in}}{\pgfqpoint{6.469336in}{5.394893in}}%
\pgfpathcurveto{\pgfqpoint{6.460128in}{5.385684in}}{\pgfqpoint{6.454954in}{5.373193in}}{\pgfqpoint{6.454954in}{5.360171in}}%
\pgfpathcurveto{\pgfqpoint{6.454954in}{5.347148in}}{\pgfqpoint{6.460128in}{5.334657in}}{\pgfqpoint{6.469336in}{5.325448in}}%
\pgfpathcurveto{\pgfqpoint{6.478545in}{5.316240in}}{\pgfqpoint{6.491036in}{5.311066in}}{\pgfqpoint{6.504059in}{5.311066in}}%
\pgfpathlineto{\pgfqpoint{6.504059in}{5.311066in}}%
\pgfpathclose%
\pgfusepath{stroke,fill}%
\end{pgfscope}%
\begin{pgfscope}%
\pgfpathrectangle{\pgfqpoint{0.786164in}{0.768110in}}{\pgfqpoint{8.851069in}{7.081890in}}%
\pgfusepath{clip}%
\pgfsetbuttcap%
\pgfsetroundjoin%
\definecolor{currentfill}{rgb}{0.263663,0.237631,0.518762}%
\pgfsetfillcolor{currentfill}%
\pgfsetfillopacity{0.700000}%
\pgfsetlinewidth{0.501875pt}%
\definecolor{currentstroke}{rgb}{1.000000,1.000000,1.000000}%
\pgfsetstrokecolor{currentstroke}%
\pgfsetstrokeopacity{0.700000}%
\pgfsetdash{}{0pt}%
\pgfpathmoveto{\pgfqpoint{5.581596in}{4.566526in}}%
\pgfpathcurveto{\pgfqpoint{5.594619in}{4.566526in}}{\pgfqpoint{5.607110in}{4.571700in}}{\pgfqpoint{5.616319in}{4.580908in}}%
\pgfpathcurveto{\pgfqpoint{5.625527in}{4.590117in}}{\pgfqpoint{5.630701in}{4.602608in}}{\pgfqpoint{5.630701in}{4.615631in}}%
\pgfpathcurveto{\pgfqpoint{5.630701in}{4.628653in}}{\pgfqpoint{5.625527in}{4.641144in}}{\pgfqpoint{5.616319in}{4.650353in}}%
\pgfpathcurveto{\pgfqpoint{5.607110in}{4.659561in}}{\pgfqpoint{5.594619in}{4.664735in}}{\pgfqpoint{5.581596in}{4.664735in}}%
\pgfpathcurveto{\pgfqpoint{5.568574in}{4.664735in}}{\pgfqpoint{5.556083in}{4.659561in}}{\pgfqpoint{5.546874in}{4.650353in}}%
\pgfpathcurveto{\pgfqpoint{5.537666in}{4.641144in}}{\pgfqpoint{5.532492in}{4.628653in}}{\pgfqpoint{5.532492in}{4.615631in}}%
\pgfpathcurveto{\pgfqpoint{5.532492in}{4.602608in}}{\pgfqpoint{5.537666in}{4.590117in}}{\pgfqpoint{5.546874in}{4.580908in}}%
\pgfpathcurveto{\pgfqpoint{5.556083in}{4.571700in}}{\pgfqpoint{5.568574in}{4.566526in}}{\pgfqpoint{5.581596in}{4.566526in}}%
\pgfpathlineto{\pgfqpoint{5.581596in}{4.566526in}}%
\pgfpathclose%
\pgfusepath{stroke,fill}%
\end{pgfscope}%
\begin{pgfscope}%
\pgfpathrectangle{\pgfqpoint{0.786164in}{0.768110in}}{\pgfqpoint{8.851069in}{7.081890in}}%
\pgfusepath{clip}%
\pgfsetbuttcap%
\pgfsetroundjoin%
\definecolor{currentfill}{rgb}{0.263663,0.237631,0.518762}%
\pgfsetfillcolor{currentfill}%
\pgfsetfillopacity{0.700000}%
\pgfsetlinewidth{0.501875pt}%
\definecolor{currentstroke}{rgb}{1.000000,1.000000,1.000000}%
\pgfsetstrokecolor{currentstroke}%
\pgfsetstrokeopacity{0.700000}%
\pgfsetdash{}{0pt}%
\pgfpathmoveto{\pgfqpoint{5.855595in}{4.676017in}}%
\pgfpathcurveto{\pgfqpoint{5.868618in}{4.676017in}}{\pgfqpoint{5.881109in}{4.681191in}}{\pgfqpoint{5.890317in}{4.690399in}}%
\pgfpathcurveto{\pgfqpoint{5.899526in}{4.699608in}}{\pgfqpoint{5.904700in}{4.712099in}}{\pgfqpoint{5.904700in}{4.725122in}}%
\pgfpathcurveto{\pgfqpoint{5.904700in}{4.738144in}}{\pgfqpoint{5.899526in}{4.750635in}}{\pgfqpoint{5.890317in}{4.759844in}}%
\pgfpathcurveto{\pgfqpoint{5.881109in}{4.769052in}}{\pgfqpoint{5.868618in}{4.774226in}}{\pgfqpoint{5.855595in}{4.774226in}}%
\pgfpathcurveto{\pgfqpoint{5.842572in}{4.774226in}}{\pgfqpoint{5.830081in}{4.769052in}}{\pgfqpoint{5.820873in}{4.759844in}}%
\pgfpathcurveto{\pgfqpoint{5.811664in}{4.750635in}}{\pgfqpoint{5.806490in}{4.738144in}}{\pgfqpoint{5.806490in}{4.725122in}}%
\pgfpathcurveto{\pgfqpoint{5.806490in}{4.712099in}}{\pgfqpoint{5.811664in}{4.699608in}}{\pgfqpoint{5.820873in}{4.690399in}}%
\pgfpathcurveto{\pgfqpoint{5.830081in}{4.681191in}}{\pgfqpoint{5.842572in}{4.676017in}}{\pgfqpoint{5.855595in}{4.676017in}}%
\pgfpathlineto{\pgfqpoint{5.855595in}{4.676017in}}%
\pgfpathclose%
\pgfusepath{stroke,fill}%
\end{pgfscope}%
\begin{pgfscope}%
\pgfpathrectangle{\pgfqpoint{0.786164in}{0.768110in}}{\pgfqpoint{8.851069in}{7.081890in}}%
\pgfusepath{clip}%
\pgfsetbuttcap%
\pgfsetroundjoin%
\definecolor{currentfill}{rgb}{0.246811,0.283237,0.535941}%
\pgfsetfillcolor{currentfill}%
\pgfsetfillopacity{0.700000}%
\pgfsetlinewidth{0.501875pt}%
\definecolor{currentstroke}{rgb}{1.000000,1.000000,1.000000}%
\pgfsetstrokecolor{currentstroke}%
\pgfsetstrokeopacity{0.700000}%
\pgfsetdash{}{0pt}%
\pgfpathmoveto{\pgfqpoint{5.188865in}{4.194256in}}%
\pgfpathcurveto{\pgfqpoint{5.201888in}{4.194256in}}{\pgfqpoint{5.214379in}{4.199430in}}{\pgfqpoint{5.223587in}{4.208638in}}%
\pgfpathcurveto{\pgfqpoint{5.232796in}{4.217847in}}{\pgfqpoint{5.237970in}{4.230338in}}{\pgfqpoint{5.237970in}{4.243360in}}%
\pgfpathcurveto{\pgfqpoint{5.237970in}{4.256383in}}{\pgfqpoint{5.232796in}{4.268874in}}{\pgfqpoint{5.223587in}{4.278083in}}%
\pgfpathcurveto{\pgfqpoint{5.214379in}{4.287291in}}{\pgfqpoint{5.201888in}{4.292465in}}{\pgfqpoint{5.188865in}{4.292465in}}%
\pgfpathcurveto{\pgfqpoint{5.175842in}{4.292465in}}{\pgfqpoint{5.163351in}{4.287291in}}{\pgfqpoint{5.154143in}{4.278083in}}%
\pgfpathcurveto{\pgfqpoint{5.144934in}{4.268874in}}{\pgfqpoint{5.139760in}{4.256383in}}{\pgfqpoint{5.139760in}{4.243360in}}%
\pgfpathcurveto{\pgfqpoint{5.139760in}{4.230338in}}{\pgfqpoint{5.144934in}{4.217847in}}{\pgfqpoint{5.154143in}{4.208638in}}%
\pgfpathcurveto{\pgfqpoint{5.163351in}{4.199430in}}{\pgfqpoint{5.175842in}{4.194256in}}{\pgfqpoint{5.188865in}{4.194256in}}%
\pgfpathlineto{\pgfqpoint{5.188865in}{4.194256in}}%
\pgfpathclose%
\pgfusepath{stroke,fill}%
\end{pgfscope}%
\begin{pgfscope}%
\pgfpathrectangle{\pgfqpoint{0.786164in}{0.768110in}}{\pgfqpoint{8.851069in}{7.081890in}}%
\pgfusepath{clip}%
\pgfsetbuttcap%
\pgfsetroundjoin%
\definecolor{currentfill}{rgb}{0.129933,0.559582,0.551864}%
\pgfsetfillcolor{currentfill}%
\pgfsetfillopacity{0.700000}%
\pgfsetlinewidth{0.501875pt}%
\definecolor{currentstroke}{rgb}{1.000000,1.000000,1.000000}%
\pgfsetstrokecolor{currentstroke}%
\pgfsetstrokeopacity{0.700000}%
\pgfsetdash{}{0pt}%
\pgfpathmoveto{\pgfqpoint{1.654283in}{1.456976in}}%
\pgfpathcurveto{\pgfqpoint{1.667306in}{1.456976in}}{\pgfqpoint{1.679797in}{1.462150in}}{\pgfqpoint{1.689005in}{1.471359in}}%
\pgfpathcurveto{\pgfqpoint{1.698214in}{1.480567in}}{\pgfqpoint{1.703388in}{1.493058in}}{\pgfqpoint{1.703388in}{1.506081in}}%
\pgfpathcurveto{\pgfqpoint{1.703388in}{1.519104in}}{\pgfqpoint{1.698214in}{1.531595in}}{\pgfqpoint{1.689005in}{1.540803in}}%
\pgfpathcurveto{\pgfqpoint{1.679797in}{1.550012in}}{\pgfqpoint{1.667306in}{1.555186in}}{\pgfqpoint{1.654283in}{1.555186in}}%
\pgfpathcurveto{\pgfqpoint{1.641260in}{1.555186in}}{\pgfqpoint{1.628769in}{1.550012in}}{\pgfqpoint{1.619561in}{1.540803in}}%
\pgfpathcurveto{\pgfqpoint{1.610352in}{1.531595in}}{\pgfqpoint{1.605178in}{1.519104in}}{\pgfqpoint{1.605178in}{1.506081in}}%
\pgfpathcurveto{\pgfqpoint{1.605178in}{1.493058in}}{\pgfqpoint{1.610352in}{1.480567in}}{\pgfqpoint{1.619561in}{1.471359in}}%
\pgfpathcurveto{\pgfqpoint{1.628769in}{1.462150in}}{\pgfqpoint{1.641260in}{1.456976in}}{\pgfqpoint{1.654283in}{1.456976in}}%
\pgfpathlineto{\pgfqpoint{1.654283in}{1.456976in}}%
\pgfpathclose%
\pgfusepath{stroke,fill}%
\end{pgfscope}%
\begin{pgfscope}%
\pgfpathrectangle{\pgfqpoint{0.786164in}{0.768110in}}{\pgfqpoint{8.851069in}{7.081890in}}%
\pgfusepath{clip}%
\pgfsetbuttcap%
\pgfsetroundjoin%
\definecolor{currentfill}{rgb}{0.132444,0.552216,0.553018}%
\pgfsetfillcolor{currentfill}%
\pgfsetfillopacity{0.700000}%
\pgfsetlinewidth{0.501875pt}%
\definecolor{currentstroke}{rgb}{1.000000,1.000000,1.000000}%
\pgfsetstrokecolor{currentstroke}%
\pgfsetstrokeopacity{0.700000}%
\pgfsetdash{}{0pt}%
\pgfpathmoveto{\pgfqpoint{1.709083in}{1.544569in}}%
\pgfpathcurveto{\pgfqpoint{1.722105in}{1.544569in}}{\pgfqpoint{1.734596in}{1.549743in}}{\pgfqpoint{1.743805in}{1.558952in}}%
\pgfpathcurveto{\pgfqpoint{1.753013in}{1.568160in}}{\pgfqpoint{1.758187in}{1.580651in}}{\pgfqpoint{1.758187in}{1.593674in}}%
\pgfpathcurveto{\pgfqpoint{1.758187in}{1.606697in}}{\pgfqpoint{1.753013in}{1.619188in}}{\pgfqpoint{1.743805in}{1.628396in}}%
\pgfpathcurveto{\pgfqpoint{1.734596in}{1.637605in}}{\pgfqpoint{1.722105in}{1.642779in}}{\pgfqpoint{1.709083in}{1.642779in}}%
\pgfpathcurveto{\pgfqpoint{1.696060in}{1.642779in}}{\pgfqpoint{1.683569in}{1.637605in}}{\pgfqpoint{1.674360in}{1.628396in}}%
\pgfpathcurveto{\pgfqpoint{1.665152in}{1.619188in}}{\pgfqpoint{1.659978in}{1.606697in}}{\pgfqpoint{1.659978in}{1.593674in}}%
\pgfpathcurveto{\pgfqpoint{1.659978in}{1.580651in}}{\pgfqpoint{1.665152in}{1.568160in}}{\pgfqpoint{1.674360in}{1.558952in}}%
\pgfpathcurveto{\pgfqpoint{1.683569in}{1.549743in}}{\pgfqpoint{1.696060in}{1.544569in}}{\pgfqpoint{1.709083in}{1.544569in}}%
\pgfpathlineto{\pgfqpoint{1.709083in}{1.544569in}}%
\pgfpathclose%
\pgfusepath{stroke,fill}%
\end{pgfscope}%
\begin{pgfscope}%
\pgfpathrectangle{\pgfqpoint{0.786164in}{0.768110in}}{\pgfqpoint{8.851069in}{7.081890in}}%
\pgfusepath{clip}%
\pgfsetbuttcap%
\pgfsetroundjoin%
\definecolor{currentfill}{rgb}{0.137770,0.537492,0.554906}%
\pgfsetfillcolor{currentfill}%
\pgfsetfillopacity{0.700000}%
\pgfsetlinewidth{0.501875pt}%
\definecolor{currentstroke}{rgb}{1.000000,1.000000,1.000000}%
\pgfsetstrokecolor{currentstroke}%
\pgfsetstrokeopacity{0.700000}%
\pgfsetdash{}{0pt}%
\pgfpathmoveto{\pgfqpoint{1.791282in}{1.522671in}}%
\pgfpathcurveto{\pgfqpoint{1.804305in}{1.522671in}}{\pgfqpoint{1.816796in}{1.527845in}}{\pgfqpoint{1.826004in}{1.537053in}}%
\pgfpathcurveto{\pgfqpoint{1.835213in}{1.546262in}}{\pgfqpoint{1.840387in}{1.558753in}}{\pgfqpoint{1.840387in}{1.571776in}}%
\pgfpathcurveto{\pgfqpoint{1.840387in}{1.584798in}}{\pgfqpoint{1.835213in}{1.597289in}}{\pgfqpoint{1.826004in}{1.606498in}}%
\pgfpathcurveto{\pgfqpoint{1.816796in}{1.615706in}}{\pgfqpoint{1.804305in}{1.620880in}}{\pgfqpoint{1.791282in}{1.620880in}}%
\pgfpathcurveto{\pgfqpoint{1.778259in}{1.620880in}}{\pgfqpoint{1.765768in}{1.615706in}}{\pgfqpoint{1.756560in}{1.606498in}}%
\pgfpathcurveto{\pgfqpoint{1.747352in}{1.597289in}}{\pgfqpoint{1.742178in}{1.584798in}}{\pgfqpoint{1.742178in}{1.571776in}}%
\pgfpathcurveto{\pgfqpoint{1.742178in}{1.558753in}}{\pgfqpoint{1.747352in}{1.546262in}}{\pgfqpoint{1.756560in}{1.537053in}}%
\pgfpathcurveto{\pgfqpoint{1.765768in}{1.527845in}}{\pgfqpoint{1.778259in}{1.522671in}}{\pgfqpoint{1.791282in}{1.522671in}}%
\pgfpathlineto{\pgfqpoint{1.791282in}{1.522671in}}%
\pgfpathclose%
\pgfusepath{stroke,fill}%
\end{pgfscope}%
\begin{pgfscope}%
\pgfpathrectangle{\pgfqpoint{0.786164in}{0.768110in}}{\pgfqpoint{8.851069in}{7.081890in}}%
\pgfusepath{clip}%
\pgfsetbuttcap%
\pgfsetroundjoin%
\definecolor{currentfill}{rgb}{0.146180,0.515413,0.556823}%
\pgfsetfillcolor{currentfill}%
\pgfsetfillopacity{0.700000}%
\pgfsetlinewidth{0.501875pt}%
\definecolor{currentstroke}{rgb}{1.000000,1.000000,1.000000}%
\pgfsetstrokecolor{currentstroke}%
\pgfsetstrokeopacity{0.700000}%
\pgfsetdash{}{0pt}%
\pgfpathmoveto{\pgfqpoint{1.855215in}{1.610264in}}%
\pgfpathcurveto{\pgfqpoint{1.868238in}{1.610264in}}{\pgfqpoint{1.880729in}{1.615438in}}{\pgfqpoint{1.889937in}{1.624646in}}%
\pgfpathcurveto{\pgfqpoint{1.899146in}{1.633855in}}{\pgfqpoint{1.904320in}{1.646346in}}{\pgfqpoint{1.904320in}{1.659369in}}%
\pgfpathcurveto{\pgfqpoint{1.904320in}{1.672391in}}{\pgfqpoint{1.899146in}{1.684882in}}{\pgfqpoint{1.889937in}{1.694091in}}%
\pgfpathcurveto{\pgfqpoint{1.880729in}{1.703299in}}{\pgfqpoint{1.868238in}{1.708473in}}{\pgfqpoint{1.855215in}{1.708473in}}%
\pgfpathcurveto{\pgfqpoint{1.842192in}{1.708473in}}{\pgfqpoint{1.829701in}{1.703299in}}{\pgfqpoint{1.820493in}{1.694091in}}%
\pgfpathcurveto{\pgfqpoint{1.811285in}{1.684882in}}{\pgfqpoint{1.806111in}{1.672391in}}{\pgfqpoint{1.806111in}{1.659369in}}%
\pgfpathcurveto{\pgfqpoint{1.806111in}{1.646346in}}{\pgfqpoint{1.811285in}{1.633855in}}{\pgfqpoint{1.820493in}{1.624646in}}%
\pgfpathcurveto{\pgfqpoint{1.829701in}{1.615438in}}{\pgfqpoint{1.842192in}{1.610264in}}{\pgfqpoint{1.855215in}{1.610264in}}%
\pgfpathlineto{\pgfqpoint{1.855215in}{1.610264in}}%
\pgfpathclose%
\pgfusepath{stroke,fill}%
\end{pgfscope}%
\begin{pgfscope}%
\pgfpathrectangle{\pgfqpoint{0.786164in}{0.768110in}}{\pgfqpoint{8.851069in}{7.081890in}}%
\pgfusepath{clip}%
\pgfsetbuttcap%
\pgfsetroundjoin%
\definecolor{currentfill}{rgb}{0.144759,0.519093,0.556572}%
\pgfsetfillcolor{currentfill}%
\pgfsetfillopacity{0.700000}%
\pgfsetlinewidth{0.501875pt}%
\definecolor{currentstroke}{rgb}{1.000000,1.000000,1.000000}%
\pgfsetstrokecolor{currentstroke}%
\pgfsetstrokeopacity{0.700000}%
\pgfsetdash{}{0pt}%
\pgfpathmoveto{\pgfqpoint{1.791282in}{1.544569in}}%
\pgfpathcurveto{\pgfqpoint{1.804305in}{1.544569in}}{\pgfqpoint{1.816796in}{1.549743in}}{\pgfqpoint{1.826004in}{1.558952in}}%
\pgfpathcurveto{\pgfqpoint{1.835213in}{1.568160in}}{\pgfqpoint{1.840387in}{1.580651in}}{\pgfqpoint{1.840387in}{1.593674in}}%
\pgfpathcurveto{\pgfqpoint{1.840387in}{1.606697in}}{\pgfqpoint{1.835213in}{1.619188in}}{\pgfqpoint{1.826004in}{1.628396in}}%
\pgfpathcurveto{\pgfqpoint{1.816796in}{1.637605in}}{\pgfqpoint{1.804305in}{1.642779in}}{\pgfqpoint{1.791282in}{1.642779in}}%
\pgfpathcurveto{\pgfqpoint{1.778259in}{1.642779in}}{\pgfqpoint{1.765768in}{1.637605in}}{\pgfqpoint{1.756560in}{1.628396in}}%
\pgfpathcurveto{\pgfqpoint{1.747352in}{1.619188in}}{\pgfqpoint{1.742178in}{1.606697in}}{\pgfqpoint{1.742178in}{1.593674in}}%
\pgfpathcurveto{\pgfqpoint{1.742178in}{1.580651in}}{\pgfqpoint{1.747352in}{1.568160in}}{\pgfqpoint{1.756560in}{1.558952in}}%
\pgfpathcurveto{\pgfqpoint{1.765768in}{1.549743in}}{\pgfqpoint{1.778259in}{1.544569in}}{\pgfqpoint{1.791282in}{1.544569in}}%
\pgfpathlineto{\pgfqpoint{1.791282in}{1.544569in}}%
\pgfpathclose%
\pgfusepath{stroke,fill}%
\end{pgfscope}%
\begin{pgfscope}%
\pgfpathrectangle{\pgfqpoint{0.786164in}{0.768110in}}{\pgfqpoint{8.851069in}{7.081890in}}%
\pgfusepath{clip}%
\pgfsetbuttcap%
\pgfsetroundjoin%
\definecolor{currentfill}{rgb}{0.123463,0.581687,0.547445}%
\pgfsetfillcolor{currentfill}%
\pgfsetfillopacity{0.700000}%
\pgfsetlinewidth{0.501875pt}%
\definecolor{currentstroke}{rgb}{1.000000,1.000000,1.000000}%
\pgfsetstrokecolor{currentstroke}%
\pgfsetstrokeopacity{0.700000}%
\pgfsetdash{}{0pt}%
\pgfpathmoveto{\pgfqpoint{1.763882in}{1.785450in}}%
\pgfpathcurveto{\pgfqpoint{1.776905in}{1.785450in}}{\pgfqpoint{1.789396in}{1.790624in}}{\pgfqpoint{1.798605in}{1.799832in}}%
\pgfpathcurveto{\pgfqpoint{1.807813in}{1.809041in}}{\pgfqpoint{1.812987in}{1.821532in}}{\pgfqpoint{1.812987in}{1.834555in}}%
\pgfpathcurveto{\pgfqpoint{1.812987in}{1.847577in}}{\pgfqpoint{1.807813in}{1.860068in}}{\pgfqpoint{1.798605in}{1.869277in}}%
\pgfpathcurveto{\pgfqpoint{1.789396in}{1.878485in}}{\pgfqpoint{1.776905in}{1.883659in}}{\pgfqpoint{1.763882in}{1.883659in}}%
\pgfpathcurveto{\pgfqpoint{1.750860in}{1.883659in}}{\pgfqpoint{1.738369in}{1.878485in}}{\pgfqpoint{1.729160in}{1.869277in}}%
\pgfpathcurveto{\pgfqpoint{1.719952in}{1.860068in}}{\pgfqpoint{1.714778in}{1.847577in}}{\pgfqpoint{1.714778in}{1.834555in}}%
\pgfpathcurveto{\pgfqpoint{1.714778in}{1.821532in}}{\pgfqpoint{1.719952in}{1.809041in}}{\pgfqpoint{1.729160in}{1.799832in}}%
\pgfpathcurveto{\pgfqpoint{1.738369in}{1.790624in}}{\pgfqpoint{1.750860in}{1.785450in}}{\pgfqpoint{1.763882in}{1.785450in}}%
\pgfpathlineto{\pgfqpoint{1.763882in}{1.785450in}}%
\pgfpathclose%
\pgfusepath{stroke,fill}%
\end{pgfscope}%
\begin{pgfscope}%
\pgfpathrectangle{\pgfqpoint{0.786164in}{0.768110in}}{\pgfqpoint{8.851069in}{7.081890in}}%
\pgfusepath{clip}%
\pgfsetbuttcap%
\pgfsetroundjoin%
\definecolor{currentfill}{rgb}{0.141935,0.526453,0.555991}%
\pgfsetfillcolor{currentfill}%
\pgfsetfillopacity{0.700000}%
\pgfsetlinewidth{0.501875pt}%
\definecolor{currentstroke}{rgb}{1.000000,1.000000,1.000000}%
\pgfsetstrokecolor{currentstroke}%
\pgfsetstrokeopacity{0.700000}%
\pgfsetdash{}{0pt}%
\pgfpathmoveto{\pgfqpoint{1.855215in}{2.092025in}}%
\pgfpathcurveto{\pgfqpoint{1.868238in}{2.092025in}}{\pgfqpoint{1.880729in}{2.097199in}}{\pgfqpoint{1.889937in}{2.106408in}}%
\pgfpathcurveto{\pgfqpoint{1.899146in}{2.115616in}}{\pgfqpoint{1.904320in}{2.128107in}}{\pgfqpoint{1.904320in}{2.141130in}}%
\pgfpathcurveto{\pgfqpoint{1.904320in}{2.154153in}}{\pgfqpoint{1.899146in}{2.166644in}}{\pgfqpoint{1.889937in}{2.175852in}}%
\pgfpathcurveto{\pgfqpoint{1.880729in}{2.185060in}}{\pgfqpoint{1.868238in}{2.190234in}}{\pgfqpoint{1.855215in}{2.190234in}}%
\pgfpathcurveto{\pgfqpoint{1.842192in}{2.190234in}}{\pgfqpoint{1.829701in}{2.185060in}}{\pgfqpoint{1.820493in}{2.175852in}}%
\pgfpathcurveto{\pgfqpoint{1.811285in}{2.166644in}}{\pgfqpoint{1.806111in}{2.154153in}}{\pgfqpoint{1.806111in}{2.141130in}}%
\pgfpathcurveto{\pgfqpoint{1.806111in}{2.128107in}}{\pgfqpoint{1.811285in}{2.115616in}}{\pgfqpoint{1.820493in}{2.106408in}}%
\pgfpathcurveto{\pgfqpoint{1.829701in}{2.097199in}}{\pgfqpoint{1.842192in}{2.092025in}}{\pgfqpoint{1.855215in}{2.092025in}}%
\pgfpathlineto{\pgfqpoint{1.855215in}{2.092025in}}%
\pgfpathclose%
\pgfusepath{stroke,fill}%
\end{pgfscope}%
\begin{pgfscope}%
\pgfpathrectangle{\pgfqpoint{0.786164in}{0.768110in}}{\pgfqpoint{8.851069in}{7.081890in}}%
\pgfusepath{clip}%
\pgfsetbuttcap%
\pgfsetroundjoin%
\definecolor{currentfill}{rgb}{0.126453,0.570633,0.549841}%
\pgfsetfillcolor{currentfill}%
\pgfsetfillopacity{0.700000}%
\pgfsetlinewidth{0.501875pt}%
\definecolor{currentstroke}{rgb}{1.000000,1.000000,1.000000}%
\pgfsetstrokecolor{currentstroke}%
\pgfsetstrokeopacity{0.700000}%
\pgfsetdash{}{0pt}%
\pgfpathmoveto{\pgfqpoint{1.773016in}{2.004432in}}%
\pgfpathcurveto{\pgfqpoint{1.786038in}{2.004432in}}{\pgfqpoint{1.798529in}{2.009606in}}{\pgfqpoint{1.807738in}{2.018815in}}%
\pgfpathcurveto{\pgfqpoint{1.816946in}{2.028023in}}{\pgfqpoint{1.822120in}{2.040514in}}{\pgfqpoint{1.822120in}{2.053537in}}%
\pgfpathcurveto{\pgfqpoint{1.822120in}{2.066560in}}{\pgfqpoint{1.816946in}{2.079051in}}{\pgfqpoint{1.807738in}{2.088259in}}%
\pgfpathcurveto{\pgfqpoint{1.798529in}{2.097468in}}{\pgfqpoint{1.786038in}{2.102642in}}{\pgfqpoint{1.773016in}{2.102642in}}%
\pgfpathcurveto{\pgfqpoint{1.759993in}{2.102642in}}{\pgfqpoint{1.747502in}{2.097468in}}{\pgfqpoint{1.738293in}{2.088259in}}%
\pgfpathcurveto{\pgfqpoint{1.729085in}{2.079051in}}{\pgfqpoint{1.723911in}{2.066560in}}{\pgfqpoint{1.723911in}{2.053537in}}%
\pgfpathcurveto{\pgfqpoint{1.723911in}{2.040514in}}{\pgfqpoint{1.729085in}{2.028023in}}{\pgfqpoint{1.738293in}{2.018815in}}%
\pgfpathcurveto{\pgfqpoint{1.747502in}{2.009606in}}{\pgfqpoint{1.759993in}{2.004432in}}{\pgfqpoint{1.773016in}{2.004432in}}%
\pgfpathlineto{\pgfqpoint{1.773016in}{2.004432in}}%
\pgfpathclose%
\pgfusepath{stroke,fill}%
\end{pgfscope}%
\begin{pgfscope}%
\pgfpathrectangle{\pgfqpoint{0.786164in}{0.768110in}}{\pgfqpoint{8.851069in}{7.081890in}}%
\pgfusepath{clip}%
\pgfsetbuttcap%
\pgfsetroundjoin%
\definecolor{currentfill}{rgb}{0.119738,0.603785,0.541400}%
\pgfsetfillcolor{currentfill}%
\pgfsetfillopacity{0.700000}%
\pgfsetlinewidth{0.501875pt}%
\definecolor{currentstroke}{rgb}{1.000000,1.000000,1.000000}%
\pgfsetstrokecolor{currentstroke}%
\pgfsetstrokeopacity{0.700000}%
\pgfsetdash{}{0pt}%
\pgfpathmoveto{\pgfqpoint{1.864348in}{1.851145in}}%
\pgfpathcurveto{\pgfqpoint{1.877371in}{1.851145in}}{\pgfqpoint{1.889862in}{1.856319in}}{\pgfqpoint{1.899071in}{1.865527in}}%
\pgfpathcurveto{\pgfqpoint{1.908279in}{1.874735in}}{\pgfqpoint{1.913453in}{1.887227in}}{\pgfqpoint{1.913453in}{1.900249in}}%
\pgfpathcurveto{\pgfqpoint{1.913453in}{1.913272in}}{\pgfqpoint{1.908279in}{1.925763in}}{\pgfqpoint{1.899071in}{1.934971in}}%
\pgfpathcurveto{\pgfqpoint{1.889862in}{1.944180in}}{\pgfqpoint{1.877371in}{1.949354in}}{\pgfqpoint{1.864348in}{1.949354in}}%
\pgfpathcurveto{\pgfqpoint{1.851326in}{1.949354in}}{\pgfqpoint{1.838835in}{1.944180in}}{\pgfqpoint{1.829626in}{1.934971in}}%
\pgfpathcurveto{\pgfqpoint{1.820418in}{1.925763in}}{\pgfqpoint{1.815244in}{1.913272in}}{\pgfqpoint{1.815244in}{1.900249in}}%
\pgfpathcurveto{\pgfqpoint{1.815244in}{1.887227in}}{\pgfqpoint{1.820418in}{1.874735in}}{\pgfqpoint{1.829626in}{1.865527in}}%
\pgfpathcurveto{\pgfqpoint{1.838835in}{1.856319in}}{\pgfqpoint{1.851326in}{1.851145in}}{\pgfqpoint{1.864348in}{1.851145in}}%
\pgfpathlineto{\pgfqpoint{1.864348in}{1.851145in}}%
\pgfpathclose%
\pgfusepath{stroke,fill}%
\end{pgfscope}%
\begin{pgfscope}%
\pgfpathrectangle{\pgfqpoint{0.786164in}{0.768110in}}{\pgfqpoint{8.851069in}{7.081890in}}%
\pgfusepath{clip}%
\pgfsetbuttcap%
\pgfsetroundjoin%
\definecolor{currentfill}{rgb}{0.120081,0.622161,0.534946}%
\pgfsetfillcolor{currentfill}%
\pgfsetfillopacity{0.700000}%
\pgfsetlinewidth{0.501875pt}%
\definecolor{currentstroke}{rgb}{1.000000,1.000000,1.000000}%
\pgfsetstrokecolor{currentstroke}%
\pgfsetstrokeopacity{0.700000}%
\pgfsetdash{}{0pt}%
\pgfpathmoveto{\pgfqpoint{1.855215in}{1.982534in}}%
\pgfpathcurveto{\pgfqpoint{1.868238in}{1.982534in}}{\pgfqpoint{1.880729in}{1.987708in}}{\pgfqpoint{1.889937in}{1.996916in}}%
\pgfpathcurveto{\pgfqpoint{1.899146in}{2.006125in}}{\pgfqpoint{1.904320in}{2.018616in}}{\pgfqpoint{1.904320in}{2.031639in}}%
\pgfpathcurveto{\pgfqpoint{1.904320in}{2.044661in}}{\pgfqpoint{1.899146in}{2.057152in}}{\pgfqpoint{1.889937in}{2.066361in}}%
\pgfpathcurveto{\pgfqpoint{1.880729in}{2.075569in}}{\pgfqpoint{1.868238in}{2.080743in}}{\pgfqpoint{1.855215in}{2.080743in}}%
\pgfpathcurveto{\pgfqpoint{1.842192in}{2.080743in}}{\pgfqpoint{1.829701in}{2.075569in}}{\pgfqpoint{1.820493in}{2.066361in}}%
\pgfpathcurveto{\pgfqpoint{1.811285in}{2.057152in}}{\pgfqpoint{1.806111in}{2.044661in}}{\pgfqpoint{1.806111in}{2.031639in}}%
\pgfpathcurveto{\pgfqpoint{1.806111in}{2.018616in}}{\pgfqpoint{1.811285in}{2.006125in}}{\pgfqpoint{1.820493in}{1.996916in}}%
\pgfpathcurveto{\pgfqpoint{1.829701in}{1.987708in}}{\pgfqpoint{1.842192in}{1.982534in}}{\pgfqpoint{1.855215in}{1.982534in}}%
\pgfpathlineto{\pgfqpoint{1.855215in}{1.982534in}}%
\pgfpathclose%
\pgfusepath{stroke,fill}%
\end{pgfscope}%
\begin{pgfscope}%
\pgfpathrectangle{\pgfqpoint{0.786164in}{0.768110in}}{\pgfqpoint{8.851069in}{7.081890in}}%
\pgfusepath{clip}%
\pgfsetbuttcap%
\pgfsetroundjoin%
\definecolor{currentfill}{rgb}{0.126326,0.644107,0.525311}%
\pgfsetfillcolor{currentfill}%
\pgfsetfillopacity{0.700000}%
\pgfsetlinewidth{0.501875pt}%
\definecolor{currentstroke}{rgb}{1.000000,1.000000,1.000000}%
\pgfsetstrokecolor{currentstroke}%
\pgfsetstrokeopacity{0.700000}%
\pgfsetdash{}{0pt}%
\pgfpathmoveto{\pgfqpoint{1.873482in}{2.420499in}}%
\pgfpathcurveto{\pgfqpoint{1.886504in}{2.420499in}}{\pgfqpoint{1.898996in}{2.425673in}}{\pgfqpoint{1.908204in}{2.434881in}}%
\pgfpathcurveto{\pgfqpoint{1.917412in}{2.444090in}}{\pgfqpoint{1.922586in}{2.456581in}}{\pgfqpoint{1.922586in}{2.469603in}}%
\pgfpathcurveto{\pgfqpoint{1.922586in}{2.482626in}}{\pgfqpoint{1.917412in}{2.495117in}}{\pgfqpoint{1.908204in}{2.504326in}}%
\pgfpathcurveto{\pgfqpoint{1.898996in}{2.513534in}}{\pgfqpoint{1.886504in}{2.518708in}}{\pgfqpoint{1.873482in}{2.518708in}}%
\pgfpathcurveto{\pgfqpoint{1.860459in}{2.518708in}}{\pgfqpoint{1.847968in}{2.513534in}}{\pgfqpoint{1.838760in}{2.504326in}}%
\pgfpathcurveto{\pgfqpoint{1.829551in}{2.495117in}}{\pgfqpoint{1.824377in}{2.482626in}}{\pgfqpoint{1.824377in}{2.469603in}}%
\pgfpathcurveto{\pgfqpoint{1.824377in}{2.456581in}}{\pgfqpoint{1.829551in}{2.444090in}}{\pgfqpoint{1.838760in}{2.434881in}}%
\pgfpathcurveto{\pgfqpoint{1.847968in}{2.425673in}}{\pgfqpoint{1.860459in}{2.420499in}}{\pgfqpoint{1.873482in}{2.420499in}}%
\pgfpathlineto{\pgfqpoint{1.873482in}{2.420499in}}%
\pgfpathclose%
\pgfusepath{stroke,fill}%
\end{pgfscope}%
\begin{pgfscope}%
\pgfpathrectangle{\pgfqpoint{0.786164in}{0.768110in}}{\pgfqpoint{8.851069in}{7.081890in}}%
\pgfusepath{clip}%
\pgfsetbuttcap%
\pgfsetroundjoin%
\definecolor{currentfill}{rgb}{0.121380,0.629492,0.531973}%
\pgfsetfillcolor{currentfill}%
\pgfsetfillopacity{0.700000}%
\pgfsetlinewidth{0.501875pt}%
\definecolor{currentstroke}{rgb}{1.000000,1.000000,1.000000}%
\pgfsetstrokecolor{currentstroke}%
\pgfsetstrokeopacity{0.700000}%
\pgfsetdash{}{0pt}%
\pgfpathmoveto{\pgfqpoint{1.846082in}{2.289109in}}%
\pgfpathcurveto{\pgfqpoint{1.859105in}{2.289109in}}{\pgfqpoint{1.871596in}{2.294283in}}{\pgfqpoint{1.880804in}{2.303492in}}%
\pgfpathcurveto{\pgfqpoint{1.890013in}{2.312700in}}{\pgfqpoint{1.895187in}{2.325191in}}{\pgfqpoint{1.895187in}{2.338214in}}%
\pgfpathcurveto{\pgfqpoint{1.895187in}{2.351237in}}{\pgfqpoint{1.890013in}{2.363728in}}{\pgfqpoint{1.880804in}{2.372936in}}%
\pgfpathcurveto{\pgfqpoint{1.871596in}{2.382145in}}{\pgfqpoint{1.859105in}{2.387319in}}{\pgfqpoint{1.846082in}{2.387319in}}%
\pgfpathcurveto{\pgfqpoint{1.833059in}{2.387319in}}{\pgfqpoint{1.820568in}{2.382145in}}{\pgfqpoint{1.811360in}{2.372936in}}%
\pgfpathcurveto{\pgfqpoint{1.802151in}{2.363728in}}{\pgfqpoint{1.796977in}{2.351237in}}{\pgfqpoint{1.796977in}{2.338214in}}%
\pgfpathcurveto{\pgfqpoint{1.796977in}{2.325191in}}{\pgfqpoint{1.802151in}{2.312700in}}{\pgfqpoint{1.811360in}{2.303492in}}%
\pgfpathcurveto{\pgfqpoint{1.820568in}{2.294283in}}{\pgfqpoint{1.833059in}{2.289109in}}{\pgfqpoint{1.846082in}{2.289109in}}%
\pgfpathlineto{\pgfqpoint{1.846082in}{2.289109in}}%
\pgfpathclose%
\pgfusepath{stroke,fill}%
\end{pgfscope}%
\begin{pgfscope}%
\pgfpathrectangle{\pgfqpoint{0.786164in}{0.768110in}}{\pgfqpoint{8.851069in}{7.081890in}}%
\pgfusepath{clip}%
\pgfsetbuttcap%
\pgfsetroundjoin%
\definecolor{currentfill}{rgb}{0.120081,0.622161,0.534946}%
\pgfsetfillcolor{currentfill}%
\pgfsetfillopacity{0.700000}%
\pgfsetlinewidth{0.501875pt}%
\definecolor{currentstroke}{rgb}{1.000000,1.000000,1.000000}%
\pgfsetstrokecolor{currentstroke}%
\pgfsetstrokeopacity{0.700000}%
\pgfsetdash{}{0pt}%
\pgfpathmoveto{\pgfqpoint{1.873482in}{2.092025in}}%
\pgfpathcurveto{\pgfqpoint{1.886504in}{2.092025in}}{\pgfqpoint{1.898996in}{2.097199in}}{\pgfqpoint{1.908204in}{2.106408in}}%
\pgfpathcurveto{\pgfqpoint{1.917412in}{2.115616in}}{\pgfqpoint{1.922586in}{2.128107in}}{\pgfqpoint{1.922586in}{2.141130in}}%
\pgfpathcurveto{\pgfqpoint{1.922586in}{2.154153in}}{\pgfqpoint{1.917412in}{2.166644in}}{\pgfqpoint{1.908204in}{2.175852in}}%
\pgfpathcurveto{\pgfqpoint{1.898996in}{2.185060in}}{\pgfqpoint{1.886504in}{2.190234in}}{\pgfqpoint{1.873482in}{2.190234in}}%
\pgfpathcurveto{\pgfqpoint{1.860459in}{2.190234in}}{\pgfqpoint{1.847968in}{2.185060in}}{\pgfqpoint{1.838760in}{2.175852in}}%
\pgfpathcurveto{\pgfqpoint{1.829551in}{2.166644in}}{\pgfqpoint{1.824377in}{2.154153in}}{\pgfqpoint{1.824377in}{2.141130in}}%
\pgfpathcurveto{\pgfqpoint{1.824377in}{2.128107in}}{\pgfqpoint{1.829551in}{2.115616in}}{\pgfqpoint{1.838760in}{2.106408in}}%
\pgfpathcurveto{\pgfqpoint{1.847968in}{2.097199in}}{\pgfqpoint{1.860459in}{2.092025in}}{\pgfqpoint{1.873482in}{2.092025in}}%
\pgfpathlineto{\pgfqpoint{1.873482in}{2.092025in}}%
\pgfpathclose%
\pgfusepath{stroke,fill}%
\end{pgfscope}%
\begin{pgfscope}%
\pgfpathrectangle{\pgfqpoint{0.786164in}{0.768110in}}{\pgfqpoint{8.851069in}{7.081890in}}%
\pgfusepath{clip}%
\pgfsetbuttcap%
\pgfsetroundjoin%
\definecolor{currentfill}{rgb}{0.130067,0.651384,0.521608}%
\pgfsetfillcolor{currentfill}%
\pgfsetfillopacity{0.700000}%
\pgfsetlinewidth{0.501875pt}%
\definecolor{currentstroke}{rgb}{1.000000,1.000000,1.000000}%
\pgfsetstrokecolor{currentstroke}%
\pgfsetstrokeopacity{0.700000}%
\pgfsetdash{}{0pt}%
\pgfpathmoveto{\pgfqpoint{1.973948in}{1.960636in}}%
\pgfpathcurveto{\pgfqpoint{1.986971in}{1.960636in}}{\pgfqpoint{1.999462in}{1.965810in}}{\pgfqpoint{2.008670in}{1.975018in}}%
\pgfpathcurveto{\pgfqpoint{2.017879in}{1.984227in}}{\pgfqpoint{2.023053in}{1.996718in}}{\pgfqpoint{2.023053in}{2.009740in}}%
\pgfpathcurveto{\pgfqpoint{2.023053in}{2.022763in}}{\pgfqpoint{2.017879in}{2.035254in}}{\pgfqpoint{2.008670in}{2.044463in}}%
\pgfpathcurveto{\pgfqpoint{1.999462in}{2.053671in}}{\pgfqpoint{1.986971in}{2.058845in}}{\pgfqpoint{1.973948in}{2.058845in}}%
\pgfpathcurveto{\pgfqpoint{1.960925in}{2.058845in}}{\pgfqpoint{1.948434in}{2.053671in}}{\pgfqpoint{1.939226in}{2.044463in}}%
\pgfpathcurveto{\pgfqpoint{1.930017in}{2.035254in}}{\pgfqpoint{1.924843in}{2.022763in}}{\pgfqpoint{1.924843in}{2.009740in}}%
\pgfpathcurveto{\pgfqpoint{1.924843in}{1.996718in}}{\pgfqpoint{1.930017in}{1.984227in}}{\pgfqpoint{1.939226in}{1.975018in}}%
\pgfpathcurveto{\pgfqpoint{1.948434in}{1.965810in}}{\pgfqpoint{1.960925in}{1.960636in}}{\pgfqpoint{1.973948in}{1.960636in}}%
\pgfpathlineto{\pgfqpoint{1.973948in}{1.960636in}}%
\pgfpathclose%
\pgfusepath{stroke,fill}%
\end{pgfscope}%
\begin{pgfscope}%
\pgfpathrectangle{\pgfqpoint{0.786164in}{0.768110in}}{\pgfqpoint{8.851069in}{7.081890in}}%
\pgfusepath{clip}%
\pgfsetbuttcap%
\pgfsetroundjoin%
\definecolor{currentfill}{rgb}{0.140210,0.665859,0.513427}%
\pgfsetfillcolor{currentfill}%
\pgfsetfillopacity{0.700000}%
\pgfsetlinewidth{0.501875pt}%
\definecolor{currentstroke}{rgb}{1.000000,1.000000,1.000000}%
\pgfsetstrokecolor{currentstroke}%
\pgfsetstrokeopacity{0.700000}%
\pgfsetdash{}{0pt}%
\pgfpathmoveto{\pgfqpoint{2.092681in}{2.311008in}}%
\pgfpathcurveto{\pgfqpoint{2.105703in}{2.311008in}}{\pgfqpoint{2.118194in}{2.316182in}}{\pgfqpoint{2.127403in}{2.325390in}}%
\pgfpathcurveto{\pgfqpoint{2.136611in}{2.334598in}}{\pgfqpoint{2.141785in}{2.347089in}}{\pgfqpoint{2.141785in}{2.360112in}}%
\pgfpathcurveto{\pgfqpoint{2.141785in}{2.373135in}}{\pgfqpoint{2.136611in}{2.385626in}}{\pgfqpoint{2.127403in}{2.394834in}}%
\pgfpathcurveto{\pgfqpoint{2.118194in}{2.404043in}}{\pgfqpoint{2.105703in}{2.409217in}}{\pgfqpoint{2.092681in}{2.409217in}}%
\pgfpathcurveto{\pgfqpoint{2.079658in}{2.409217in}}{\pgfqpoint{2.067167in}{2.404043in}}{\pgfqpoint{2.057958in}{2.394834in}}%
\pgfpathcurveto{\pgfqpoint{2.048750in}{2.385626in}}{\pgfqpoint{2.043576in}{2.373135in}}{\pgfqpoint{2.043576in}{2.360112in}}%
\pgfpathcurveto{\pgfqpoint{2.043576in}{2.347089in}}{\pgfqpoint{2.048750in}{2.334598in}}{\pgfqpoint{2.057958in}{2.325390in}}%
\pgfpathcurveto{\pgfqpoint{2.067167in}{2.316182in}}{\pgfqpoint{2.079658in}{2.311008in}}{\pgfqpoint{2.092681in}{2.311008in}}%
\pgfpathlineto{\pgfqpoint{2.092681in}{2.311008in}}%
\pgfpathclose%
\pgfusepath{stroke,fill}%
\end{pgfscope}%
\begin{pgfscope}%
\pgfpathrectangle{\pgfqpoint{0.786164in}{0.768110in}}{\pgfqpoint{8.851069in}{7.081890in}}%
\pgfusepath{clip}%
\pgfsetbuttcap%
\pgfsetroundjoin%
\definecolor{currentfill}{rgb}{0.150148,0.676631,0.506589}%
\pgfsetfillcolor{currentfill}%
\pgfsetfillopacity{0.700000}%
\pgfsetlinewidth{0.501875pt}%
\definecolor{currentstroke}{rgb}{1.000000,1.000000,1.000000}%
\pgfsetstrokecolor{currentstroke}%
\pgfsetstrokeopacity{0.700000}%
\pgfsetdash{}{0pt}%
\pgfpathmoveto{\pgfqpoint{2.047014in}{2.135822in}}%
\pgfpathcurveto{\pgfqpoint{2.060037in}{2.135822in}}{\pgfqpoint{2.072528in}{2.140996in}}{\pgfqpoint{2.081736in}{2.150204in}}%
\pgfpathcurveto{\pgfqpoint{2.090945in}{2.159413in}}{\pgfqpoint{2.096119in}{2.171904in}}{\pgfqpoint{2.096119in}{2.184926in}}%
\pgfpathcurveto{\pgfqpoint{2.096119in}{2.197949in}}{\pgfqpoint{2.090945in}{2.210440in}}{\pgfqpoint{2.081736in}{2.219649in}}%
\pgfpathcurveto{\pgfqpoint{2.072528in}{2.228857in}}{\pgfqpoint{2.060037in}{2.234031in}}{\pgfqpoint{2.047014in}{2.234031in}}%
\pgfpathcurveto{\pgfqpoint{2.033992in}{2.234031in}}{\pgfqpoint{2.021500in}{2.228857in}}{\pgfqpoint{2.012292in}{2.219649in}}%
\pgfpathcurveto{\pgfqpoint{2.003084in}{2.210440in}}{\pgfqpoint{1.997910in}{2.197949in}}{\pgfqpoint{1.997910in}{2.184926in}}%
\pgfpathcurveto{\pgfqpoint{1.997910in}{2.171904in}}{\pgfqpoint{2.003084in}{2.159413in}}{\pgfqpoint{2.012292in}{2.150204in}}%
\pgfpathcurveto{\pgfqpoint{2.021500in}{2.140996in}}{\pgfqpoint{2.033992in}{2.135822in}}{\pgfqpoint{2.047014in}{2.135822in}}%
\pgfpathlineto{\pgfqpoint{2.047014in}{2.135822in}}%
\pgfpathclose%
\pgfusepath{stroke,fill}%
\end{pgfscope}%
\begin{pgfscope}%
\pgfpathrectangle{\pgfqpoint{0.786164in}{0.768110in}}{\pgfqpoint{8.851069in}{7.081890in}}%
\pgfusepath{clip}%
\pgfsetbuttcap%
\pgfsetroundjoin%
\definecolor{currentfill}{rgb}{0.170948,0.694384,0.493803}%
\pgfsetfillcolor{currentfill}%
\pgfsetfillopacity{0.700000}%
\pgfsetlinewidth{0.501875pt}%
\definecolor{currentstroke}{rgb}{1.000000,1.000000,1.000000}%
\pgfsetstrokecolor{currentstroke}%
\pgfsetstrokeopacity{0.700000}%
\pgfsetdash{}{0pt}%
\pgfpathmoveto{\pgfqpoint{1.919148in}{2.332906in}}%
\pgfpathcurveto{\pgfqpoint{1.932171in}{2.332906in}}{\pgfqpoint{1.944662in}{2.338080in}}{\pgfqpoint{1.953870in}{2.347288in}}%
\pgfpathcurveto{\pgfqpoint{1.963079in}{2.356497in}}{\pgfqpoint{1.968253in}{2.368988in}}{\pgfqpoint{1.968253in}{2.382010in}}%
\pgfpathcurveto{\pgfqpoint{1.968253in}{2.395033in}}{\pgfqpoint{1.963079in}{2.407524in}}{\pgfqpoint{1.953870in}{2.416733in}}%
\pgfpathcurveto{\pgfqpoint{1.944662in}{2.425941in}}{\pgfqpoint{1.932171in}{2.431115in}}{\pgfqpoint{1.919148in}{2.431115in}}%
\pgfpathcurveto{\pgfqpoint{1.906125in}{2.431115in}}{\pgfqpoint{1.893634in}{2.425941in}}{\pgfqpoint{1.884426in}{2.416733in}}%
\pgfpathcurveto{\pgfqpoint{1.875218in}{2.407524in}}{\pgfqpoint{1.870044in}{2.395033in}}{\pgfqpoint{1.870044in}{2.382010in}}%
\pgfpathcurveto{\pgfqpoint{1.870044in}{2.368988in}}{\pgfqpoint{1.875218in}{2.356497in}}{\pgfqpoint{1.884426in}{2.347288in}}%
\pgfpathcurveto{\pgfqpoint{1.893634in}{2.338080in}}{\pgfqpoint{1.906125in}{2.332906in}}{\pgfqpoint{1.919148in}{2.332906in}}%
\pgfpathlineto{\pgfqpoint{1.919148in}{2.332906in}}%
\pgfpathclose%
\pgfusepath{stroke,fill}%
\end{pgfscope}%
\begin{pgfscope}%
\pgfpathrectangle{\pgfqpoint{0.786164in}{0.768110in}}{\pgfqpoint{8.851069in}{7.081890in}}%
\pgfusepath{clip}%
\pgfsetbuttcap%
\pgfsetroundjoin%
\definecolor{currentfill}{rgb}{0.170948,0.694384,0.493803}%
\pgfsetfillcolor{currentfill}%
\pgfsetfillopacity{0.700000}%
\pgfsetlinewidth{0.501875pt}%
\definecolor{currentstroke}{rgb}{1.000000,1.000000,1.000000}%
\pgfsetstrokecolor{currentstroke}%
\pgfsetstrokeopacity{0.700000}%
\pgfsetdash{}{0pt}%
\pgfpathmoveto{\pgfqpoint{2.037881in}{2.551888in}}%
\pgfpathcurveto{\pgfqpoint{2.050904in}{2.551888in}}{\pgfqpoint{2.063395in}{2.557062in}}{\pgfqpoint{2.072603in}{2.566271in}}%
\pgfpathcurveto{\pgfqpoint{2.081812in}{2.575479in}}{\pgfqpoint{2.086986in}{2.587970in}}{\pgfqpoint{2.086986in}{2.600993in}}%
\pgfpathcurveto{\pgfqpoint{2.086986in}{2.614015in}}{\pgfqpoint{2.081812in}{2.626507in}}{\pgfqpoint{2.072603in}{2.635715in}}%
\pgfpathcurveto{\pgfqpoint{2.063395in}{2.644923in}}{\pgfqpoint{2.050904in}{2.650097in}}{\pgfqpoint{2.037881in}{2.650097in}}%
\pgfpathcurveto{\pgfqpoint{2.024858in}{2.650097in}}{\pgfqpoint{2.012367in}{2.644923in}}{\pgfqpoint{2.003159in}{2.635715in}}%
\pgfpathcurveto{\pgfqpoint{1.993950in}{2.626507in}}{\pgfqpoint{1.988776in}{2.614015in}}{\pgfqpoint{1.988776in}{2.600993in}}%
\pgfpathcurveto{\pgfqpoint{1.988776in}{2.587970in}}{\pgfqpoint{1.993950in}{2.575479in}}{\pgfqpoint{2.003159in}{2.566271in}}%
\pgfpathcurveto{\pgfqpoint{2.012367in}{2.557062in}}{\pgfqpoint{2.024858in}{2.551888in}}{\pgfqpoint{2.037881in}{2.551888in}}%
\pgfpathlineto{\pgfqpoint{2.037881in}{2.551888in}}%
\pgfpathclose%
\pgfusepath{stroke,fill}%
\end{pgfscope}%
\begin{pgfscope}%
\pgfpathrectangle{\pgfqpoint{0.786164in}{0.768110in}}{\pgfqpoint{8.851069in}{7.081890in}}%
\pgfusepath{clip}%
\pgfsetbuttcap%
\pgfsetroundjoin%
\definecolor{currentfill}{rgb}{0.191090,0.708366,0.482284}%
\pgfsetfillcolor{currentfill}%
\pgfsetfillopacity{0.700000}%
\pgfsetlinewidth{0.501875pt}%
\definecolor{currentstroke}{rgb}{1.000000,1.000000,1.000000}%
\pgfsetstrokecolor{currentstroke}%
\pgfsetstrokeopacity{0.700000}%
\pgfsetdash{}{0pt}%
\pgfpathmoveto{\pgfqpoint{1.964815in}{2.836565in}}%
\pgfpathcurveto{\pgfqpoint{1.977837in}{2.836565in}}{\pgfqpoint{1.990328in}{2.841739in}}{\pgfqpoint{1.999537in}{2.850948in}}%
\pgfpathcurveto{\pgfqpoint{2.008745in}{2.860156in}}{\pgfqpoint{2.013919in}{2.872647in}}{\pgfqpoint{2.013919in}{2.885670in}}%
\pgfpathcurveto{\pgfqpoint{2.013919in}{2.898693in}}{\pgfqpoint{2.008745in}{2.911184in}}{\pgfqpoint{1.999537in}{2.920392in}}%
\pgfpathcurveto{\pgfqpoint{1.990328in}{2.929601in}}{\pgfqpoint{1.977837in}{2.934774in}}{\pgfqpoint{1.964815in}{2.934774in}}%
\pgfpathcurveto{\pgfqpoint{1.951792in}{2.934774in}}{\pgfqpoint{1.939301in}{2.929601in}}{\pgfqpoint{1.930092in}{2.920392in}}%
\pgfpathcurveto{\pgfqpoint{1.920884in}{2.911184in}}{\pgfqpoint{1.915710in}{2.898693in}}{\pgfqpoint{1.915710in}{2.885670in}}%
\pgfpathcurveto{\pgfqpoint{1.915710in}{2.872647in}}{\pgfqpoint{1.920884in}{2.860156in}}{\pgfqpoint{1.930092in}{2.850948in}}%
\pgfpathcurveto{\pgfqpoint{1.939301in}{2.841739in}}{\pgfqpoint{1.951792in}{2.836565in}}{\pgfqpoint{1.964815in}{2.836565in}}%
\pgfpathlineto{\pgfqpoint{1.964815in}{2.836565in}}%
\pgfpathclose%
\pgfusepath{stroke,fill}%
\end{pgfscope}%
\begin{pgfscope}%
\pgfpathrectangle{\pgfqpoint{0.786164in}{0.768110in}}{\pgfqpoint{8.851069in}{7.081890in}}%
\pgfusepath{clip}%
\pgfsetbuttcap%
\pgfsetroundjoin%
\definecolor{currentfill}{rgb}{0.267004,0.004874,0.329415}%
\pgfsetfillcolor{currentfill}%
\pgfsetfillopacity{0.700000}%
\pgfsetlinewidth{0.501875pt}%
\definecolor{currentstroke}{rgb}{1.000000,1.000000,1.000000}%
\pgfsetstrokecolor{currentstroke}%
\pgfsetstrokeopacity{0.700000}%
\pgfsetdash{}{0pt}%
\pgfpathmoveto{\pgfqpoint{1.992214in}{2.902260in}}%
\pgfpathcurveto{\pgfqpoint{2.005237in}{2.902260in}}{\pgfqpoint{2.017728in}{2.907434in}}{\pgfqpoint{2.026937in}{2.916642in}}%
\pgfpathcurveto{\pgfqpoint{2.036145in}{2.925851in}}{\pgfqpoint{2.041319in}{2.938342in}}{\pgfqpoint{2.041319in}{2.951365in}}%
\pgfpathcurveto{\pgfqpoint{2.041319in}{2.964387in}}{\pgfqpoint{2.036145in}{2.976878in}}{\pgfqpoint{2.026937in}{2.986087in}}%
\pgfpathcurveto{\pgfqpoint{2.017728in}{2.995295in}}{\pgfqpoint{2.005237in}{3.000469in}}{\pgfqpoint{1.992214in}{3.000469in}}%
\pgfpathcurveto{\pgfqpoint{1.979192in}{3.000469in}}{\pgfqpoint{1.966701in}{2.995295in}}{\pgfqpoint{1.957492in}{2.986087in}}%
\pgfpathcurveto{\pgfqpoint{1.948284in}{2.976878in}}{\pgfqpoint{1.943110in}{2.964387in}}{\pgfqpoint{1.943110in}{2.951365in}}%
\pgfpathcurveto{\pgfqpoint{1.943110in}{2.938342in}}{\pgfqpoint{1.948284in}{2.925851in}}{\pgfqpoint{1.957492in}{2.916642in}}%
\pgfpathcurveto{\pgfqpoint{1.966701in}{2.907434in}}{\pgfqpoint{1.979192in}{2.902260in}}{\pgfqpoint{1.992214in}{2.902260in}}%
\pgfpathlineto{\pgfqpoint{1.992214in}{2.902260in}}%
\pgfpathclose%
\pgfusepath{stroke,fill}%
\end{pgfscope}%
\begin{pgfscope}%
\pgfpathrectangle{\pgfqpoint{0.786164in}{0.768110in}}{\pgfqpoint{8.851069in}{7.081890in}}%
\pgfusepath{clip}%
\pgfsetbuttcap%
\pgfsetroundjoin%
\definecolor{currentfill}{rgb}{0.267004,0.004874,0.329415}%
\pgfsetfillcolor{currentfill}%
\pgfsetfillopacity{0.700000}%
\pgfsetlinewidth{0.501875pt}%
\definecolor{currentstroke}{rgb}{1.000000,1.000000,1.000000}%
\pgfsetstrokecolor{currentstroke}%
\pgfsetstrokeopacity{0.700000}%
\pgfsetdash{}{0pt}%
\pgfpathmoveto{\pgfqpoint{1.955681in}{2.792769in}}%
\pgfpathcurveto{\pgfqpoint{1.968704in}{2.792769in}}{\pgfqpoint{1.981195in}{2.797943in}}{\pgfqpoint{1.990404in}{2.807151in}}%
\pgfpathcurveto{\pgfqpoint{1.999612in}{2.816360in}}{\pgfqpoint{2.004786in}{2.828851in}}{\pgfqpoint{2.004786in}{2.841873in}}%
\pgfpathcurveto{\pgfqpoint{2.004786in}{2.854896in}}{\pgfqpoint{1.999612in}{2.867387in}}{\pgfqpoint{1.990404in}{2.876596in}}%
\pgfpathcurveto{\pgfqpoint{1.981195in}{2.885804in}}{\pgfqpoint{1.968704in}{2.890978in}}{\pgfqpoint{1.955681in}{2.890978in}}%
\pgfpathcurveto{\pgfqpoint{1.942659in}{2.890978in}}{\pgfqpoint{1.930168in}{2.885804in}}{\pgfqpoint{1.920959in}{2.876596in}}%
\pgfpathcurveto{\pgfqpoint{1.911751in}{2.867387in}}{\pgfqpoint{1.906577in}{2.854896in}}{\pgfqpoint{1.906577in}{2.841873in}}%
\pgfpathcurveto{\pgfqpoint{1.906577in}{2.828851in}}{\pgfqpoint{1.911751in}{2.816360in}}{\pgfqpoint{1.920959in}{2.807151in}}%
\pgfpathcurveto{\pgfqpoint{1.930168in}{2.797943in}}{\pgfqpoint{1.942659in}{2.792769in}}{\pgfqpoint{1.955681in}{2.792769in}}%
\pgfpathlineto{\pgfqpoint{1.955681in}{2.792769in}}%
\pgfpathclose%
\pgfusepath{stroke,fill}%
\end{pgfscope}%
\begin{pgfscope}%
\pgfpathrectangle{\pgfqpoint{0.786164in}{0.768110in}}{\pgfqpoint{8.851069in}{7.081890in}}%
\pgfusepath{clip}%
\pgfsetbuttcap%
\pgfsetroundjoin%
\definecolor{currentfill}{rgb}{0.267004,0.004874,0.329415}%
\pgfsetfillcolor{currentfill}%
\pgfsetfillopacity{0.700000}%
\pgfsetlinewidth{0.501875pt}%
\definecolor{currentstroke}{rgb}{1.000000,1.000000,1.000000}%
\pgfsetstrokecolor{currentstroke}%
\pgfsetstrokeopacity{0.700000}%
\pgfsetdash{}{0pt}%
\pgfpathmoveto{\pgfqpoint{1.946548in}{2.792769in}}%
\pgfpathcurveto{\pgfqpoint{1.959571in}{2.792769in}}{\pgfqpoint{1.972062in}{2.797943in}}{\pgfqpoint{1.981270in}{2.807151in}}%
\pgfpathcurveto{\pgfqpoint{1.990479in}{2.816360in}}{\pgfqpoint{1.995653in}{2.828851in}}{\pgfqpoint{1.995653in}{2.841873in}}%
\pgfpathcurveto{\pgfqpoint{1.995653in}{2.854896in}}{\pgfqpoint{1.990479in}{2.867387in}}{\pgfqpoint{1.981270in}{2.876596in}}%
\pgfpathcurveto{\pgfqpoint{1.972062in}{2.885804in}}{\pgfqpoint{1.959571in}{2.890978in}}{\pgfqpoint{1.946548in}{2.890978in}}%
\pgfpathcurveto{\pgfqpoint{1.933525in}{2.890978in}}{\pgfqpoint{1.921034in}{2.885804in}}{\pgfqpoint{1.911826in}{2.876596in}}%
\pgfpathcurveto{\pgfqpoint{1.902617in}{2.867387in}}{\pgfqpoint{1.897443in}{2.854896in}}{\pgfqpoint{1.897443in}{2.841873in}}%
\pgfpathcurveto{\pgfqpoint{1.897443in}{2.828851in}}{\pgfqpoint{1.902617in}{2.816360in}}{\pgfqpoint{1.911826in}{2.807151in}}%
\pgfpathcurveto{\pgfqpoint{1.921034in}{2.797943in}}{\pgfqpoint{1.933525in}{2.792769in}}{\pgfqpoint{1.946548in}{2.792769in}}%
\pgfpathlineto{\pgfqpoint{1.946548in}{2.792769in}}%
\pgfpathclose%
\pgfusepath{stroke,fill}%
\end{pgfscope}%
\begin{pgfscope}%
\pgfpathrectangle{\pgfqpoint{0.786164in}{0.768110in}}{\pgfqpoint{8.851069in}{7.081890in}}%
\pgfusepath{clip}%
\pgfsetbuttcap%
\pgfsetroundjoin%
\definecolor{currentfill}{rgb}{0.267004,0.004874,0.329415}%
\pgfsetfillcolor{currentfill}%
\pgfsetfillopacity{0.700000}%
\pgfsetlinewidth{0.501875pt}%
\definecolor{currentstroke}{rgb}{1.000000,1.000000,1.000000}%
\pgfsetstrokecolor{currentstroke}%
\pgfsetstrokeopacity{0.700000}%
\pgfsetdash{}{0pt}%
\pgfpathmoveto{\pgfqpoint{2.010481in}{2.858463in}}%
\pgfpathcurveto{\pgfqpoint{2.023504in}{2.858463in}}{\pgfqpoint{2.035995in}{2.863637in}}{\pgfqpoint{2.045203in}{2.872846in}}%
\pgfpathcurveto{\pgfqpoint{2.054412in}{2.882054in}}{\pgfqpoint{2.059586in}{2.894545in}}{\pgfqpoint{2.059586in}{2.907568in}}%
\pgfpathcurveto{\pgfqpoint{2.059586in}{2.920591in}}{\pgfqpoint{2.054412in}{2.933082in}}{\pgfqpoint{2.045203in}{2.942290in}}%
\pgfpathcurveto{\pgfqpoint{2.035995in}{2.951499in}}{\pgfqpoint{2.023504in}{2.956673in}}{\pgfqpoint{2.010481in}{2.956673in}}%
\pgfpathcurveto{\pgfqpoint{1.997458in}{2.956673in}}{\pgfqpoint{1.984967in}{2.951499in}}{\pgfqpoint{1.975759in}{2.942290in}}%
\pgfpathcurveto{\pgfqpoint{1.966550in}{2.933082in}}{\pgfqpoint{1.961376in}{2.920591in}}{\pgfqpoint{1.961376in}{2.907568in}}%
\pgfpathcurveto{\pgfqpoint{1.961376in}{2.894545in}}{\pgfqpoint{1.966550in}{2.882054in}}{\pgfqpoint{1.975759in}{2.872846in}}%
\pgfpathcurveto{\pgfqpoint{1.984967in}{2.863637in}}{\pgfqpoint{1.997458in}{2.858463in}}{\pgfqpoint{2.010481in}{2.858463in}}%
\pgfpathlineto{\pgfqpoint{2.010481in}{2.858463in}}%
\pgfpathclose%
\pgfusepath{stroke,fill}%
\end{pgfscope}%
\begin{pgfscope}%
\pgfpathrectangle{\pgfqpoint{0.786164in}{0.768110in}}{\pgfqpoint{8.851069in}{7.081890in}}%
\pgfusepath{clip}%
\pgfsetbuttcap%
\pgfsetroundjoin%
\definecolor{currentfill}{rgb}{0.267004,0.004874,0.329415}%
\pgfsetfillcolor{currentfill}%
\pgfsetfillopacity{0.700000}%
\pgfsetlinewidth{0.501875pt}%
\definecolor{currentstroke}{rgb}{1.000000,1.000000,1.000000}%
\pgfsetstrokecolor{currentstroke}%
\pgfsetstrokeopacity{0.700000}%
\pgfsetdash{}{0pt}%
\pgfpathmoveto{\pgfqpoint{2.019614in}{2.814667in}}%
\pgfpathcurveto{\pgfqpoint{2.032637in}{2.814667in}}{\pgfqpoint{2.045128in}{2.819841in}}{\pgfqpoint{2.054337in}{2.829049in}}%
\pgfpathcurveto{\pgfqpoint{2.063545in}{2.838258in}}{\pgfqpoint{2.068719in}{2.850749in}}{\pgfqpoint{2.068719in}{2.863772in}}%
\pgfpathcurveto{\pgfqpoint{2.068719in}{2.876794in}}{\pgfqpoint{2.063545in}{2.889285in}}{\pgfqpoint{2.054337in}{2.898494in}}%
\pgfpathcurveto{\pgfqpoint{2.045128in}{2.907702in}}{\pgfqpoint{2.032637in}{2.912876in}}{\pgfqpoint{2.019614in}{2.912876in}}%
\pgfpathcurveto{\pgfqpoint{2.006592in}{2.912876in}}{\pgfqpoint{1.994101in}{2.907702in}}{\pgfqpoint{1.984892in}{2.898494in}}%
\pgfpathcurveto{\pgfqpoint{1.975684in}{2.889285in}}{\pgfqpoint{1.970510in}{2.876794in}}{\pgfqpoint{1.970510in}{2.863772in}}%
\pgfpathcurveto{\pgfqpoint{1.970510in}{2.850749in}}{\pgfqpoint{1.975684in}{2.838258in}}{\pgfqpoint{1.984892in}{2.829049in}}%
\pgfpathcurveto{\pgfqpoint{1.994101in}{2.819841in}}{\pgfqpoint{2.006592in}{2.814667in}}{\pgfqpoint{2.019614in}{2.814667in}}%
\pgfpathlineto{\pgfqpoint{2.019614in}{2.814667in}}%
\pgfpathclose%
\pgfusepath{stroke,fill}%
\end{pgfscope}%
\begin{pgfscope}%
\pgfpathrectangle{\pgfqpoint{0.786164in}{0.768110in}}{\pgfqpoint{8.851069in}{7.081890in}}%
\pgfusepath{clip}%
\pgfsetbuttcap%
\pgfsetroundjoin%
\definecolor{currentfill}{rgb}{0.267004,0.004874,0.329415}%
\pgfsetfillcolor{currentfill}%
\pgfsetfillopacity{0.700000}%
\pgfsetlinewidth{0.501875pt}%
\definecolor{currentstroke}{rgb}{1.000000,1.000000,1.000000}%
\pgfsetstrokecolor{currentstroke}%
\pgfsetstrokeopacity{0.700000}%
\pgfsetdash{}{0pt}%
\pgfpathmoveto{\pgfqpoint{1.818682in}{2.705176in}}%
\pgfpathcurveto{\pgfqpoint{1.831705in}{2.705176in}}{\pgfqpoint{1.844196in}{2.710350in}}{\pgfqpoint{1.853404in}{2.719558in}}%
\pgfpathcurveto{\pgfqpoint{1.862613in}{2.728767in}}{\pgfqpoint{1.867787in}{2.741258in}}{\pgfqpoint{1.867787in}{2.754280in}}%
\pgfpathcurveto{\pgfqpoint{1.867787in}{2.767303in}}{\pgfqpoint{1.862613in}{2.779794in}}{\pgfqpoint{1.853404in}{2.789003in}}%
\pgfpathcurveto{\pgfqpoint{1.844196in}{2.798211in}}{\pgfqpoint{1.831705in}{2.803385in}}{\pgfqpoint{1.818682in}{2.803385in}}%
\pgfpathcurveto{\pgfqpoint{1.805659in}{2.803385in}}{\pgfqpoint{1.793168in}{2.798211in}}{\pgfqpoint{1.783960in}{2.789003in}}%
\pgfpathcurveto{\pgfqpoint{1.774751in}{2.779794in}}{\pgfqpoint{1.769577in}{2.767303in}}{\pgfqpoint{1.769577in}{2.754280in}}%
\pgfpathcurveto{\pgfqpoint{1.769577in}{2.741258in}}{\pgfqpoint{1.774751in}{2.728767in}}{\pgfqpoint{1.783960in}{2.719558in}}%
\pgfpathcurveto{\pgfqpoint{1.793168in}{2.710350in}}{\pgfqpoint{1.805659in}{2.705176in}}{\pgfqpoint{1.818682in}{2.705176in}}%
\pgfpathlineto{\pgfqpoint{1.818682in}{2.705176in}}%
\pgfpathclose%
\pgfusepath{stroke,fill}%
\end{pgfscope}%
\begin{pgfscope}%
\pgfpathrectangle{\pgfqpoint{0.786164in}{0.768110in}}{\pgfqpoint{8.851069in}{7.081890in}}%
\pgfusepath{clip}%
\pgfsetbuttcap%
\pgfsetroundjoin%
\definecolor{currentfill}{rgb}{0.271305,0.019942,0.347269}%
\pgfsetfillcolor{currentfill}%
\pgfsetfillopacity{0.700000}%
\pgfsetlinewidth{0.501875pt}%
\definecolor{currentstroke}{rgb}{1.000000,1.000000,1.000000}%
\pgfsetstrokecolor{currentstroke}%
\pgfsetstrokeopacity{0.700000}%
\pgfsetdash{}{0pt}%
\pgfpathmoveto{\pgfqpoint{1.919148in}{2.748972in}}%
\pgfpathcurveto{\pgfqpoint{1.932171in}{2.748972in}}{\pgfqpoint{1.944662in}{2.754146in}}{\pgfqpoint{1.953870in}{2.763355in}}%
\pgfpathcurveto{\pgfqpoint{1.963079in}{2.772563in}}{\pgfqpoint{1.968253in}{2.785054in}}{\pgfqpoint{1.968253in}{2.798077in}}%
\pgfpathcurveto{\pgfqpoint{1.968253in}{2.811100in}}{\pgfqpoint{1.963079in}{2.823591in}}{\pgfqpoint{1.953870in}{2.832799in}}%
\pgfpathcurveto{\pgfqpoint{1.944662in}{2.842008in}}{\pgfqpoint{1.932171in}{2.847182in}}{\pgfqpoint{1.919148in}{2.847182in}}%
\pgfpathcurveto{\pgfqpoint{1.906125in}{2.847182in}}{\pgfqpoint{1.893634in}{2.842008in}}{\pgfqpoint{1.884426in}{2.832799in}}%
\pgfpathcurveto{\pgfqpoint{1.875218in}{2.823591in}}{\pgfqpoint{1.870044in}{2.811100in}}{\pgfqpoint{1.870044in}{2.798077in}}%
\pgfpathcurveto{\pgfqpoint{1.870044in}{2.785054in}}{\pgfqpoint{1.875218in}{2.772563in}}{\pgfqpoint{1.884426in}{2.763355in}}%
\pgfpathcurveto{\pgfqpoint{1.893634in}{2.754146in}}{\pgfqpoint{1.906125in}{2.748972in}}{\pgfqpoint{1.919148in}{2.748972in}}%
\pgfpathlineto{\pgfqpoint{1.919148in}{2.748972in}}%
\pgfpathclose%
\pgfusepath{stroke,fill}%
\end{pgfscope}%
\begin{pgfscope}%
\pgfpathrectangle{\pgfqpoint{0.786164in}{0.768110in}}{\pgfqpoint{8.851069in}{7.081890in}}%
\pgfusepath{clip}%
\pgfsetbuttcap%
\pgfsetroundjoin%
\definecolor{currentfill}{rgb}{0.274952,0.037752,0.364543}%
\pgfsetfillcolor{currentfill}%
\pgfsetfillopacity{0.700000}%
\pgfsetlinewidth{0.501875pt}%
\definecolor{currentstroke}{rgb}{1.000000,1.000000,1.000000}%
\pgfsetstrokecolor{currentstroke}%
\pgfsetstrokeopacity{0.700000}%
\pgfsetdash{}{0pt}%
\pgfpathmoveto{\pgfqpoint{1.910015in}{2.748972in}}%
\pgfpathcurveto{\pgfqpoint{1.923038in}{2.748972in}}{\pgfqpoint{1.935529in}{2.754146in}}{\pgfqpoint{1.944737in}{2.763355in}}%
\pgfpathcurveto{\pgfqpoint{1.953946in}{2.772563in}}{\pgfqpoint{1.959120in}{2.785054in}}{\pgfqpoint{1.959120in}{2.798077in}}%
\pgfpathcurveto{\pgfqpoint{1.959120in}{2.811100in}}{\pgfqpoint{1.953946in}{2.823591in}}{\pgfqpoint{1.944737in}{2.832799in}}%
\pgfpathcurveto{\pgfqpoint{1.935529in}{2.842008in}}{\pgfqpoint{1.923038in}{2.847182in}}{\pgfqpoint{1.910015in}{2.847182in}}%
\pgfpathcurveto{\pgfqpoint{1.896992in}{2.847182in}}{\pgfqpoint{1.884501in}{2.842008in}}{\pgfqpoint{1.875293in}{2.832799in}}%
\pgfpathcurveto{\pgfqpoint{1.866084in}{2.823591in}}{\pgfqpoint{1.860910in}{2.811100in}}{\pgfqpoint{1.860910in}{2.798077in}}%
\pgfpathcurveto{\pgfqpoint{1.860910in}{2.785054in}}{\pgfqpoint{1.866084in}{2.772563in}}{\pgfqpoint{1.875293in}{2.763355in}}%
\pgfpathcurveto{\pgfqpoint{1.884501in}{2.754146in}}{\pgfqpoint{1.896992in}{2.748972in}}{\pgfqpoint{1.910015in}{2.748972in}}%
\pgfpathlineto{\pgfqpoint{1.910015in}{2.748972in}}%
\pgfpathclose%
\pgfusepath{stroke,fill}%
\end{pgfscope}%
\begin{pgfscope}%
\pgfpathrectangle{\pgfqpoint{0.786164in}{0.768110in}}{\pgfqpoint{8.851069in}{7.081890in}}%
\pgfusepath{clip}%
\pgfsetbuttcap%
\pgfsetroundjoin%
\definecolor{currentfill}{rgb}{0.278791,0.062145,0.386592}%
\pgfsetfillcolor{currentfill}%
\pgfsetfillopacity{0.700000}%
\pgfsetlinewidth{0.501875pt}%
\definecolor{currentstroke}{rgb}{1.000000,1.000000,1.000000}%
\pgfsetstrokecolor{currentstroke}%
\pgfsetstrokeopacity{0.700000}%
\pgfsetdash{}{0pt}%
\pgfpathmoveto{\pgfqpoint{1.983081in}{2.814667in}}%
\pgfpathcurveto{\pgfqpoint{1.996104in}{2.814667in}}{\pgfqpoint{2.008595in}{2.819841in}}{\pgfqpoint{2.017803in}{2.829049in}}%
\pgfpathcurveto{\pgfqpoint{2.027012in}{2.838258in}}{\pgfqpoint{2.032186in}{2.850749in}}{\pgfqpoint{2.032186in}{2.863772in}}%
\pgfpathcurveto{\pgfqpoint{2.032186in}{2.876794in}}{\pgfqpoint{2.027012in}{2.889285in}}{\pgfqpoint{2.017803in}{2.898494in}}%
\pgfpathcurveto{\pgfqpoint{2.008595in}{2.907702in}}{\pgfqpoint{1.996104in}{2.912876in}}{\pgfqpoint{1.983081in}{2.912876in}}%
\pgfpathcurveto{\pgfqpoint{1.970059in}{2.912876in}}{\pgfqpoint{1.957567in}{2.907702in}}{\pgfqpoint{1.948359in}{2.898494in}}%
\pgfpathcurveto{\pgfqpoint{1.939151in}{2.889285in}}{\pgfqpoint{1.933977in}{2.876794in}}{\pgfqpoint{1.933977in}{2.863772in}}%
\pgfpathcurveto{\pgfqpoint{1.933977in}{2.850749in}}{\pgfqpoint{1.939151in}{2.838258in}}{\pgfqpoint{1.948359in}{2.829049in}}%
\pgfpathcurveto{\pgfqpoint{1.957567in}{2.819841in}}{\pgfqpoint{1.970059in}{2.814667in}}{\pgfqpoint{1.983081in}{2.814667in}}%
\pgfpathlineto{\pgfqpoint{1.983081in}{2.814667in}}%
\pgfpathclose%
\pgfusepath{stroke,fill}%
\end{pgfscope}%
\begin{pgfscope}%
\pgfpathrectangle{\pgfqpoint{0.786164in}{0.768110in}}{\pgfqpoint{8.851069in}{7.081890in}}%
\pgfusepath{clip}%
\pgfsetbuttcap%
\pgfsetroundjoin%
\definecolor{currentfill}{rgb}{0.281446,0.084320,0.407414}%
\pgfsetfillcolor{currentfill}%
\pgfsetfillopacity{0.700000}%
\pgfsetlinewidth{0.501875pt}%
\definecolor{currentstroke}{rgb}{1.000000,1.000000,1.000000}%
\pgfsetstrokecolor{currentstroke}%
\pgfsetstrokeopacity{0.700000}%
\pgfsetdash{}{0pt}%
\pgfpathmoveto{\pgfqpoint{1.736482in}{2.639481in}}%
\pgfpathcurveto{\pgfqpoint{1.749505in}{2.639481in}}{\pgfqpoint{1.761996in}{2.644655in}}{\pgfqpoint{1.771205in}{2.653864in}}%
\pgfpathcurveto{\pgfqpoint{1.780413in}{2.663072in}}{\pgfqpoint{1.785587in}{2.675563in}}{\pgfqpoint{1.785587in}{2.688586in}}%
\pgfpathcurveto{\pgfqpoint{1.785587in}{2.701608in}}{\pgfqpoint{1.780413in}{2.714100in}}{\pgfqpoint{1.771205in}{2.723308in}}%
\pgfpathcurveto{\pgfqpoint{1.761996in}{2.732516in}}{\pgfqpoint{1.749505in}{2.737690in}}{\pgfqpoint{1.736482in}{2.737690in}}%
\pgfpathcurveto{\pgfqpoint{1.723460in}{2.737690in}}{\pgfqpoint{1.710969in}{2.732516in}}{\pgfqpoint{1.701760in}{2.723308in}}%
\pgfpathcurveto{\pgfqpoint{1.692552in}{2.714100in}}{\pgfqpoint{1.687378in}{2.701608in}}{\pgfqpoint{1.687378in}{2.688586in}}%
\pgfpathcurveto{\pgfqpoint{1.687378in}{2.675563in}}{\pgfqpoint{1.692552in}{2.663072in}}{\pgfqpoint{1.701760in}{2.653864in}}%
\pgfpathcurveto{\pgfqpoint{1.710969in}{2.644655in}}{\pgfqpoint{1.723460in}{2.639481in}}{\pgfqpoint{1.736482in}{2.639481in}}%
\pgfpathlineto{\pgfqpoint{1.736482in}{2.639481in}}%
\pgfpathclose%
\pgfusepath{stroke,fill}%
\end{pgfscope}%
\begin{pgfscope}%
\pgfpathrectangle{\pgfqpoint{0.786164in}{0.768110in}}{\pgfqpoint{8.851069in}{7.081890in}}%
\pgfusepath{clip}%
\pgfsetbuttcap%
\pgfsetroundjoin%
\definecolor{currentfill}{rgb}{0.282910,0.105393,0.426902}%
\pgfsetfillcolor{currentfill}%
\pgfsetfillopacity{0.700000}%
\pgfsetlinewidth{0.501875pt}%
\definecolor{currentstroke}{rgb}{1.000000,1.000000,1.000000}%
\pgfsetstrokecolor{currentstroke}%
\pgfsetstrokeopacity{0.700000}%
\pgfsetdash{}{0pt}%
\pgfpathmoveto{\pgfqpoint{1.745616in}{2.617583in}}%
\pgfpathcurveto{\pgfqpoint{1.758638in}{2.617583in}}{\pgfqpoint{1.771130in}{2.622757in}}{\pgfqpoint{1.780338in}{2.631965in}}%
\pgfpathcurveto{\pgfqpoint{1.789546in}{2.641174in}}{\pgfqpoint{1.794720in}{2.653665in}}{\pgfqpoint{1.794720in}{2.666687in}}%
\pgfpathcurveto{\pgfqpoint{1.794720in}{2.679710in}}{\pgfqpoint{1.789546in}{2.692201in}}{\pgfqpoint{1.780338in}{2.701410in}}%
\pgfpathcurveto{\pgfqpoint{1.771130in}{2.710618in}}{\pgfqpoint{1.758638in}{2.715792in}}{\pgfqpoint{1.745616in}{2.715792in}}%
\pgfpathcurveto{\pgfqpoint{1.732593in}{2.715792in}}{\pgfqpoint{1.720102in}{2.710618in}}{\pgfqpoint{1.710894in}{2.701410in}}%
\pgfpathcurveto{\pgfqpoint{1.701685in}{2.692201in}}{\pgfqpoint{1.696511in}{2.679710in}}{\pgfqpoint{1.696511in}{2.666687in}}%
\pgfpathcurveto{\pgfqpoint{1.696511in}{2.653665in}}{\pgfqpoint{1.701685in}{2.641174in}}{\pgfqpoint{1.710894in}{2.631965in}}%
\pgfpathcurveto{\pgfqpoint{1.720102in}{2.622757in}}{\pgfqpoint{1.732593in}{2.617583in}}{\pgfqpoint{1.745616in}{2.617583in}}%
\pgfpathlineto{\pgfqpoint{1.745616in}{2.617583in}}%
\pgfpathclose%
\pgfusepath{stroke,fill}%
\end{pgfscope}%
\begin{pgfscope}%
\pgfpathrectangle{\pgfqpoint{0.786164in}{0.768110in}}{\pgfqpoint{8.851069in}{7.081890in}}%
\pgfusepath{clip}%
\pgfsetbuttcap%
\pgfsetroundjoin%
\definecolor{currentfill}{rgb}{0.283091,0.110553,0.431554}%
\pgfsetfillcolor{currentfill}%
\pgfsetfillopacity{0.700000}%
\pgfsetlinewidth{0.501875pt}%
\definecolor{currentstroke}{rgb}{1.000000,1.000000,1.000000}%
\pgfsetstrokecolor{currentstroke}%
\pgfsetstrokeopacity{0.700000}%
\pgfsetdash{}{0pt}%
\pgfpathmoveto{\pgfqpoint{1.416817in}{2.245313in}}%
\pgfpathcurveto{\pgfqpoint{1.429840in}{2.245313in}}{\pgfqpoint{1.442331in}{2.250487in}}{\pgfqpoint{1.451540in}{2.259695in}}%
\pgfpathcurveto{\pgfqpoint{1.460748in}{2.268904in}}{\pgfqpoint{1.465922in}{2.281395in}}{\pgfqpoint{1.465922in}{2.294417in}}%
\pgfpathcurveto{\pgfqpoint{1.465922in}{2.307440in}}{\pgfqpoint{1.460748in}{2.319931in}}{\pgfqpoint{1.451540in}{2.329140in}}%
\pgfpathcurveto{\pgfqpoint{1.442331in}{2.338348in}}{\pgfqpoint{1.429840in}{2.343522in}}{\pgfqpoint{1.416817in}{2.343522in}}%
\pgfpathcurveto{\pgfqpoint{1.403795in}{2.343522in}}{\pgfqpoint{1.391304in}{2.338348in}}{\pgfqpoint{1.382095in}{2.329140in}}%
\pgfpathcurveto{\pgfqpoint{1.372887in}{2.319931in}}{\pgfqpoint{1.367713in}{2.307440in}}{\pgfqpoint{1.367713in}{2.294417in}}%
\pgfpathcurveto{\pgfqpoint{1.367713in}{2.281395in}}{\pgfqpoint{1.372887in}{2.268904in}}{\pgfqpoint{1.382095in}{2.259695in}}%
\pgfpathcurveto{\pgfqpoint{1.391304in}{2.250487in}}{\pgfqpoint{1.403795in}{2.245313in}}{\pgfqpoint{1.416817in}{2.245313in}}%
\pgfpathlineto{\pgfqpoint{1.416817in}{2.245313in}}%
\pgfpathclose%
\pgfusepath{stroke,fill}%
\end{pgfscope}%
\begin{pgfscope}%
\pgfpathrectangle{\pgfqpoint{0.786164in}{0.768110in}}{\pgfqpoint{8.851069in}{7.081890in}}%
\pgfusepath{clip}%
\pgfsetbuttcap%
\pgfsetroundjoin%
\definecolor{currentfill}{rgb}{0.283072,0.130895,0.449241}%
\pgfsetfillcolor{currentfill}%
\pgfsetfillopacity{0.700000}%
\pgfsetlinewidth{0.501875pt}%
\definecolor{currentstroke}{rgb}{1.000000,1.000000,1.000000}%
\pgfsetstrokecolor{currentstroke}%
\pgfsetstrokeopacity{0.700000}%
\pgfsetdash{}{0pt}%
\pgfpathmoveto{\pgfqpoint{1.298085in}{2.092025in}}%
\pgfpathcurveto{\pgfqpoint{1.311107in}{2.092025in}}{\pgfqpoint{1.323598in}{2.097199in}}{\pgfqpoint{1.332807in}{2.106408in}}%
\pgfpathcurveto{\pgfqpoint{1.342015in}{2.115616in}}{\pgfqpoint{1.347189in}{2.128107in}}{\pgfqpoint{1.347189in}{2.141130in}}%
\pgfpathcurveto{\pgfqpoint{1.347189in}{2.154153in}}{\pgfqpoint{1.342015in}{2.166644in}}{\pgfqpoint{1.332807in}{2.175852in}}%
\pgfpathcurveto{\pgfqpoint{1.323598in}{2.185060in}}{\pgfqpoint{1.311107in}{2.190234in}}{\pgfqpoint{1.298085in}{2.190234in}}%
\pgfpathcurveto{\pgfqpoint{1.285062in}{2.190234in}}{\pgfqpoint{1.272571in}{2.185060in}}{\pgfqpoint{1.263362in}{2.175852in}}%
\pgfpathcurveto{\pgfqpoint{1.254154in}{2.166644in}}{\pgfqpoint{1.248980in}{2.154153in}}{\pgfqpoint{1.248980in}{2.141130in}}%
\pgfpathcurveto{\pgfqpoint{1.248980in}{2.128107in}}{\pgfqpoint{1.254154in}{2.115616in}}{\pgfqpoint{1.263362in}{2.106408in}}%
\pgfpathcurveto{\pgfqpoint{1.272571in}{2.097199in}}{\pgfqpoint{1.285062in}{2.092025in}}{\pgfqpoint{1.298085in}{2.092025in}}%
\pgfpathlineto{\pgfqpoint{1.298085in}{2.092025in}}%
\pgfpathclose%
\pgfusepath{stroke,fill}%
\end{pgfscope}%
\begin{pgfscope}%
\pgfpathrectangle{\pgfqpoint{0.786164in}{0.768110in}}{\pgfqpoint{8.851069in}{7.081890in}}%
\pgfusepath{clip}%
\pgfsetbuttcap%
\pgfsetroundjoin%
\definecolor{currentfill}{rgb}{0.281887,0.150881,0.465405}%
\pgfsetfillcolor{currentfill}%
\pgfsetfillopacity{0.700000}%
\pgfsetlinewidth{0.501875pt}%
\definecolor{currentstroke}{rgb}{1.000000,1.000000,1.000000}%
\pgfsetstrokecolor{currentstroke}%
\pgfsetstrokeopacity{0.700000}%
\pgfsetdash{}{0pt}%
\pgfpathmoveto{\pgfqpoint{1.453351in}{2.157720in}}%
\pgfpathcurveto{\pgfqpoint{1.466373in}{2.157720in}}{\pgfqpoint{1.478864in}{2.162894in}}{\pgfqpoint{1.488073in}{2.172102in}}%
\pgfpathcurveto{\pgfqpoint{1.497281in}{2.181311in}}{\pgfqpoint{1.502455in}{2.193802in}}{\pgfqpoint{1.502455in}{2.206825in}}%
\pgfpathcurveto{\pgfqpoint{1.502455in}{2.219847in}}{\pgfqpoint{1.497281in}{2.232338in}}{\pgfqpoint{1.488073in}{2.241547in}}%
\pgfpathcurveto{\pgfqpoint{1.478864in}{2.250755in}}{\pgfqpoint{1.466373in}{2.255929in}}{\pgfqpoint{1.453351in}{2.255929in}}%
\pgfpathcurveto{\pgfqpoint{1.440328in}{2.255929in}}{\pgfqpoint{1.427837in}{2.250755in}}{\pgfqpoint{1.418628in}{2.241547in}}%
\pgfpathcurveto{\pgfqpoint{1.409420in}{2.232338in}}{\pgfqpoint{1.404246in}{2.219847in}}{\pgfqpoint{1.404246in}{2.206825in}}%
\pgfpathcurveto{\pgfqpoint{1.404246in}{2.193802in}}{\pgfqpoint{1.409420in}{2.181311in}}{\pgfqpoint{1.418628in}{2.172102in}}%
\pgfpathcurveto{\pgfqpoint{1.427837in}{2.162894in}}{\pgfqpoint{1.440328in}{2.157720in}}{\pgfqpoint{1.453351in}{2.157720in}}%
\pgfpathlineto{\pgfqpoint{1.453351in}{2.157720in}}%
\pgfpathclose%
\pgfusepath{stroke,fill}%
\end{pgfscope}%
\begin{pgfscope}%
\pgfpathrectangle{\pgfqpoint{0.786164in}{0.768110in}}{\pgfqpoint{8.851069in}{7.081890in}}%
\pgfusepath{clip}%
\pgfsetbuttcap%
\pgfsetroundjoin%
\definecolor{currentfill}{rgb}{0.280255,0.165693,0.476498}%
\pgfsetfillcolor{currentfill}%
\pgfsetfillopacity{0.700000}%
\pgfsetlinewidth{0.501875pt}%
\definecolor{currentstroke}{rgb}{1.000000,1.000000,1.000000}%
\pgfsetstrokecolor{currentstroke}%
\pgfsetstrokeopacity{0.700000}%
\pgfsetdash{}{0pt}%
\pgfpathmoveto{\pgfqpoint{1.416817in}{2.135822in}}%
\pgfpathcurveto{\pgfqpoint{1.429840in}{2.135822in}}{\pgfqpoint{1.442331in}{2.140996in}}{\pgfqpoint{1.451540in}{2.150204in}}%
\pgfpathcurveto{\pgfqpoint{1.460748in}{2.159413in}}{\pgfqpoint{1.465922in}{2.171904in}}{\pgfqpoint{1.465922in}{2.184926in}}%
\pgfpathcurveto{\pgfqpoint{1.465922in}{2.197949in}}{\pgfqpoint{1.460748in}{2.210440in}}{\pgfqpoint{1.451540in}{2.219649in}}%
\pgfpathcurveto{\pgfqpoint{1.442331in}{2.228857in}}{\pgfqpoint{1.429840in}{2.234031in}}{\pgfqpoint{1.416817in}{2.234031in}}%
\pgfpathcurveto{\pgfqpoint{1.403795in}{2.234031in}}{\pgfqpoint{1.391304in}{2.228857in}}{\pgfqpoint{1.382095in}{2.219649in}}%
\pgfpathcurveto{\pgfqpoint{1.372887in}{2.210440in}}{\pgfqpoint{1.367713in}{2.197949in}}{\pgfqpoint{1.367713in}{2.184926in}}%
\pgfpathcurveto{\pgfqpoint{1.367713in}{2.171904in}}{\pgfqpoint{1.372887in}{2.159413in}}{\pgfqpoint{1.382095in}{2.150204in}}%
\pgfpathcurveto{\pgfqpoint{1.391304in}{2.140996in}}{\pgfqpoint{1.403795in}{2.135822in}}{\pgfqpoint{1.416817in}{2.135822in}}%
\pgfpathlineto{\pgfqpoint{1.416817in}{2.135822in}}%
\pgfpathclose%
\pgfusepath{stroke,fill}%
\end{pgfscope}%
\begin{pgfscope}%
\pgfpathrectangle{\pgfqpoint{0.786164in}{0.768110in}}{\pgfqpoint{8.851069in}{7.081890in}}%
\pgfusepath{clip}%
\pgfsetbuttcap%
\pgfsetroundjoin%
\definecolor{currentfill}{rgb}{0.279574,0.170599,0.479997}%
\pgfsetfillcolor{currentfill}%
\pgfsetfillopacity{0.700000}%
\pgfsetlinewidth{0.501875pt}%
\definecolor{currentstroke}{rgb}{1.000000,1.000000,1.000000}%
\pgfsetstrokecolor{currentstroke}%
\pgfsetstrokeopacity{0.700000}%
\pgfsetdash{}{0pt}%
\pgfpathmoveto{\pgfqpoint{1.453351in}{2.157720in}}%
\pgfpathcurveto{\pgfqpoint{1.466373in}{2.157720in}}{\pgfqpoint{1.478864in}{2.162894in}}{\pgfqpoint{1.488073in}{2.172102in}}%
\pgfpathcurveto{\pgfqpoint{1.497281in}{2.181311in}}{\pgfqpoint{1.502455in}{2.193802in}}{\pgfqpoint{1.502455in}{2.206825in}}%
\pgfpathcurveto{\pgfqpoint{1.502455in}{2.219847in}}{\pgfqpoint{1.497281in}{2.232338in}}{\pgfqpoint{1.488073in}{2.241547in}}%
\pgfpathcurveto{\pgfqpoint{1.478864in}{2.250755in}}{\pgfqpoint{1.466373in}{2.255929in}}{\pgfqpoint{1.453351in}{2.255929in}}%
\pgfpathcurveto{\pgfqpoint{1.440328in}{2.255929in}}{\pgfqpoint{1.427837in}{2.250755in}}{\pgfqpoint{1.418628in}{2.241547in}}%
\pgfpathcurveto{\pgfqpoint{1.409420in}{2.232338in}}{\pgfqpoint{1.404246in}{2.219847in}}{\pgfqpoint{1.404246in}{2.206825in}}%
\pgfpathcurveto{\pgfqpoint{1.404246in}{2.193802in}}{\pgfqpoint{1.409420in}{2.181311in}}{\pgfqpoint{1.418628in}{2.172102in}}%
\pgfpathcurveto{\pgfqpoint{1.427837in}{2.162894in}}{\pgfqpoint{1.440328in}{2.157720in}}{\pgfqpoint{1.453351in}{2.157720in}}%
\pgfpathlineto{\pgfqpoint{1.453351in}{2.157720in}}%
\pgfpathclose%
\pgfusepath{stroke,fill}%
\end{pgfscope}%
\begin{pgfscope}%
\pgfpathrectangle{\pgfqpoint{0.786164in}{0.768110in}}{\pgfqpoint{8.851069in}{7.081890in}}%
\pgfusepath{clip}%
\pgfsetbuttcap%
\pgfsetroundjoin%
\definecolor{currentfill}{rgb}{0.269308,0.218818,0.509577}%
\pgfsetfillcolor{currentfill}%
\pgfsetfillopacity{0.700000}%
\pgfsetlinewidth{0.501875pt}%
\definecolor{currentstroke}{rgb}{1.000000,1.000000,1.000000}%
\pgfsetstrokecolor{currentstroke}%
\pgfsetstrokeopacity{0.700000}%
\pgfsetdash{}{0pt}%
\pgfpathmoveto{\pgfqpoint{1.188485in}{1.982534in}}%
\pgfpathcurveto{\pgfqpoint{1.201508in}{1.982534in}}{\pgfqpoint{1.213999in}{1.987708in}}{\pgfqpoint{1.223207in}{1.996916in}}%
\pgfpathcurveto{\pgfqpoint{1.232416in}{2.006125in}}{\pgfqpoint{1.237590in}{2.018616in}}{\pgfqpoint{1.237590in}{2.031639in}}%
\pgfpathcurveto{\pgfqpoint{1.237590in}{2.044661in}}{\pgfqpoint{1.232416in}{2.057152in}}{\pgfqpoint{1.223207in}{2.066361in}}%
\pgfpathcurveto{\pgfqpoint{1.213999in}{2.075569in}}{\pgfqpoint{1.201508in}{2.080743in}}{\pgfqpoint{1.188485in}{2.080743in}}%
\pgfpathcurveto{\pgfqpoint{1.175462in}{2.080743in}}{\pgfqpoint{1.162971in}{2.075569in}}{\pgfqpoint{1.153763in}{2.066361in}}%
\pgfpathcurveto{\pgfqpoint{1.144555in}{2.057152in}}{\pgfqpoint{1.139381in}{2.044661in}}{\pgfqpoint{1.139381in}{2.031639in}}%
\pgfpathcurveto{\pgfqpoint{1.139381in}{2.018616in}}{\pgfqpoint{1.144555in}{2.006125in}}{\pgfqpoint{1.153763in}{1.996916in}}%
\pgfpathcurveto{\pgfqpoint{1.162971in}{1.987708in}}{\pgfqpoint{1.175462in}{1.982534in}}{\pgfqpoint{1.188485in}{1.982534in}}%
\pgfpathlineto{\pgfqpoint{1.188485in}{1.982534in}}%
\pgfpathclose%
\pgfusepath{stroke,fill}%
\end{pgfscope}%
\begin{pgfscope}%
\pgfpathrectangle{\pgfqpoint{0.786164in}{0.768110in}}{\pgfqpoint{8.851069in}{7.081890in}}%
\pgfusepath{clip}%
\pgfsetbuttcap%
\pgfsetroundjoin%
\definecolor{currentfill}{rgb}{0.258965,0.251537,0.524736}%
\pgfsetfillcolor{currentfill}%
\pgfsetfillopacity{0.700000}%
\pgfsetlinewidth{0.501875pt}%
\definecolor{currentstroke}{rgb}{1.000000,1.000000,1.000000}%
\pgfsetstrokecolor{currentstroke}%
\pgfsetstrokeopacity{0.700000}%
\pgfsetdash{}{0pt}%
\pgfpathmoveto{\pgfqpoint{1.225018in}{2.004432in}}%
\pgfpathcurveto{\pgfqpoint{1.238041in}{2.004432in}}{\pgfqpoint{1.250532in}{2.009606in}}{\pgfqpoint{1.259741in}{2.018815in}}%
\pgfpathcurveto{\pgfqpoint{1.268949in}{2.028023in}}{\pgfqpoint{1.274123in}{2.040514in}}{\pgfqpoint{1.274123in}{2.053537in}}%
\pgfpathcurveto{\pgfqpoint{1.274123in}{2.066560in}}{\pgfqpoint{1.268949in}{2.079051in}}{\pgfqpoint{1.259741in}{2.088259in}}%
\pgfpathcurveto{\pgfqpoint{1.250532in}{2.097468in}}{\pgfqpoint{1.238041in}{2.102642in}}{\pgfqpoint{1.225018in}{2.102642in}}%
\pgfpathcurveto{\pgfqpoint{1.211996in}{2.102642in}}{\pgfqpoint{1.199505in}{2.097468in}}{\pgfqpoint{1.190296in}{2.088259in}}%
\pgfpathcurveto{\pgfqpoint{1.181088in}{2.079051in}}{\pgfqpoint{1.175914in}{2.066560in}}{\pgfqpoint{1.175914in}{2.053537in}}%
\pgfpathcurveto{\pgfqpoint{1.175914in}{2.040514in}}{\pgfqpoint{1.181088in}{2.028023in}}{\pgfqpoint{1.190296in}{2.018815in}}%
\pgfpathcurveto{\pgfqpoint{1.199505in}{2.009606in}}{\pgfqpoint{1.211996in}{2.004432in}}{\pgfqpoint{1.225018in}{2.004432in}}%
\pgfpathlineto{\pgfqpoint{1.225018in}{2.004432in}}%
\pgfpathclose%
\pgfusepath{stroke,fill}%
\end{pgfscope}%
\begin{pgfscope}%
\pgfpathrectangle{\pgfqpoint{0.786164in}{0.768110in}}{\pgfqpoint{8.851069in}{7.081890in}}%
\pgfusepath{clip}%
\pgfsetbuttcap%
\pgfsetroundjoin%
\definecolor{currentfill}{rgb}{0.253935,0.265254,0.529983}%
\pgfsetfillcolor{currentfill}%
\pgfsetfillopacity{0.700000}%
\pgfsetlinewidth{0.501875pt}%
\definecolor{currentstroke}{rgb}{1.000000,1.000000,1.000000}%
\pgfsetstrokecolor{currentstroke}%
\pgfsetstrokeopacity{0.700000}%
\pgfsetdash{}{0pt}%
\pgfpathmoveto{\pgfqpoint{1.398551in}{2.113923in}}%
\pgfpathcurveto{\pgfqpoint{1.411574in}{2.113923in}}{\pgfqpoint{1.424065in}{2.119097in}}{\pgfqpoint{1.433273in}{2.128306in}}%
\pgfpathcurveto{\pgfqpoint{1.442481in}{2.137514in}}{\pgfqpoint{1.447655in}{2.150005in}}{\pgfqpoint{1.447655in}{2.163028in}}%
\pgfpathcurveto{\pgfqpoint{1.447655in}{2.176051in}}{\pgfqpoint{1.442481in}{2.188542in}}{\pgfqpoint{1.433273in}{2.197750in}}%
\pgfpathcurveto{\pgfqpoint{1.424065in}{2.206959in}}{\pgfqpoint{1.411574in}{2.212133in}}{\pgfqpoint{1.398551in}{2.212133in}}%
\pgfpathcurveto{\pgfqpoint{1.385528in}{2.212133in}}{\pgfqpoint{1.373037in}{2.206959in}}{\pgfqpoint{1.363829in}{2.197750in}}%
\pgfpathcurveto{\pgfqpoint{1.354620in}{2.188542in}}{\pgfqpoint{1.349446in}{2.176051in}}{\pgfqpoint{1.349446in}{2.163028in}}%
\pgfpathcurveto{\pgfqpoint{1.349446in}{2.150005in}}{\pgfqpoint{1.354620in}{2.137514in}}{\pgfqpoint{1.363829in}{2.128306in}}%
\pgfpathcurveto{\pgfqpoint{1.373037in}{2.119097in}}{\pgfqpoint{1.385528in}{2.113923in}}{\pgfqpoint{1.398551in}{2.113923in}}%
\pgfpathlineto{\pgfqpoint{1.398551in}{2.113923in}}%
\pgfpathclose%
\pgfusepath{stroke,fill}%
\end{pgfscope}%
\begin{pgfscope}%
\pgfpathrectangle{\pgfqpoint{0.786164in}{0.768110in}}{\pgfqpoint{8.851069in}{7.081890in}}%
\pgfusepath{clip}%
\pgfsetbuttcap%
\pgfsetroundjoin%
\definecolor{currentfill}{rgb}{0.276022,0.044167,0.370164}%
\pgfsetfillcolor{currentfill}%
\pgfsetfillopacity{0.700000}%
\pgfsetlinewidth{0.501875pt}%
\definecolor{currentstroke}{rgb}{1.000000,1.000000,1.000000}%
\pgfsetstrokecolor{currentstroke}%
\pgfsetstrokeopacity{0.700000}%
\pgfsetdash{}{0pt}%
\pgfpathmoveto{\pgfqpoint{3.882805in}{4.216154in}}%
\pgfpathcurveto{\pgfqpoint{3.895828in}{4.216154in}}{\pgfqpoint{3.908319in}{4.221328in}}{\pgfqpoint{3.917527in}{4.230537in}}%
\pgfpathcurveto{\pgfqpoint{3.926736in}{4.239745in}}{\pgfqpoint{3.931910in}{4.252236in}}{\pgfqpoint{3.931910in}{4.265259in}}%
\pgfpathcurveto{\pgfqpoint{3.931910in}{4.278281in}}{\pgfqpoint{3.926736in}{4.290773in}}{\pgfqpoint{3.917527in}{4.299981in}}%
\pgfpathcurveto{\pgfqpoint{3.908319in}{4.309189in}}{\pgfqpoint{3.895828in}{4.314363in}}{\pgfqpoint{3.882805in}{4.314363in}}%
\pgfpathcurveto{\pgfqpoint{3.869782in}{4.314363in}}{\pgfqpoint{3.857291in}{4.309189in}}{\pgfqpoint{3.848083in}{4.299981in}}%
\pgfpathcurveto{\pgfqpoint{3.838874in}{4.290773in}}{\pgfqpoint{3.833700in}{4.278281in}}{\pgfqpoint{3.833700in}{4.265259in}}%
\pgfpathcurveto{\pgfqpoint{3.833700in}{4.252236in}}{\pgfqpoint{3.838874in}{4.239745in}}{\pgfqpoint{3.848083in}{4.230537in}}%
\pgfpathcurveto{\pgfqpoint{3.857291in}{4.221328in}}{\pgfqpoint{3.869782in}{4.216154in}}{\pgfqpoint{3.882805in}{4.216154in}}%
\pgfpathlineto{\pgfqpoint{3.882805in}{4.216154in}}%
\pgfpathclose%
\pgfusepath{stroke,fill}%
\end{pgfscope}%
\begin{pgfscope}%
\pgfpathrectangle{\pgfqpoint{0.786164in}{0.768110in}}{\pgfqpoint{8.851069in}{7.081890in}}%
\pgfusepath{clip}%
\pgfsetbuttcap%
\pgfsetroundjoin%
\definecolor{currentfill}{rgb}{0.277941,0.056324,0.381191}%
\pgfsetfillcolor{currentfill}%
\pgfsetfillopacity{0.700000}%
\pgfsetlinewidth{0.501875pt}%
\definecolor{currentstroke}{rgb}{1.000000,1.000000,1.000000}%
\pgfsetstrokecolor{currentstroke}%
\pgfsetstrokeopacity{0.700000}%
\pgfsetdash{}{0pt}%
\pgfpathmoveto{\pgfqpoint{3.800605in}{4.128561in}}%
\pgfpathcurveto{\pgfqpoint{3.813628in}{4.128561in}}{\pgfqpoint{3.826119in}{4.133735in}}{\pgfqpoint{3.835328in}{4.142944in}}%
\pgfpathcurveto{\pgfqpoint{3.844536in}{4.152152in}}{\pgfqpoint{3.849710in}{4.164643in}}{\pgfqpoint{3.849710in}{4.177666in}}%
\pgfpathcurveto{\pgfqpoint{3.849710in}{4.190688in}}{\pgfqpoint{3.844536in}{4.203180in}}{\pgfqpoint{3.835328in}{4.212388in}}%
\pgfpathcurveto{\pgfqpoint{3.826119in}{4.221596in}}{\pgfqpoint{3.813628in}{4.226770in}}{\pgfqpoint{3.800605in}{4.226770in}}%
\pgfpathcurveto{\pgfqpoint{3.787583in}{4.226770in}}{\pgfqpoint{3.775092in}{4.221596in}}{\pgfqpoint{3.765883in}{4.212388in}}%
\pgfpathcurveto{\pgfqpoint{3.756675in}{4.203180in}}{\pgfqpoint{3.751501in}{4.190688in}}{\pgfqpoint{3.751501in}{4.177666in}}%
\pgfpathcurveto{\pgfqpoint{3.751501in}{4.164643in}}{\pgfqpoint{3.756675in}{4.152152in}}{\pgfqpoint{3.765883in}{4.142944in}}%
\pgfpathcurveto{\pgfqpoint{3.775092in}{4.133735in}}{\pgfqpoint{3.787583in}{4.128561in}}{\pgfqpoint{3.800605in}{4.128561in}}%
\pgfpathlineto{\pgfqpoint{3.800605in}{4.128561in}}%
\pgfpathclose%
\pgfusepath{stroke,fill}%
\end{pgfscope}%
\begin{pgfscope}%
\pgfpathrectangle{\pgfqpoint{0.786164in}{0.768110in}}{\pgfqpoint{8.851069in}{7.081890in}}%
\pgfusepath{clip}%
\pgfsetbuttcap%
\pgfsetroundjoin%
\definecolor{currentfill}{rgb}{0.278791,0.062145,0.386592}%
\pgfsetfillcolor{currentfill}%
\pgfsetfillopacity{0.700000}%
\pgfsetlinewidth{0.501875pt}%
\definecolor{currentstroke}{rgb}{1.000000,1.000000,1.000000}%
\pgfsetstrokecolor{currentstroke}%
\pgfsetstrokeopacity{0.700000}%
\pgfsetdash{}{0pt}%
\pgfpathmoveto{\pgfqpoint{3.754939in}{4.062866in}}%
\pgfpathcurveto{\pgfqpoint{3.767962in}{4.062866in}}{\pgfqpoint{3.780453in}{4.068040in}}{\pgfqpoint{3.789661in}{4.077249in}}%
\pgfpathcurveto{\pgfqpoint{3.798870in}{4.086457in}}{\pgfqpoint{3.804044in}{4.098948in}}{\pgfqpoint{3.804044in}{4.111971in}}%
\pgfpathcurveto{\pgfqpoint{3.804044in}{4.124994in}}{\pgfqpoint{3.798870in}{4.137485in}}{\pgfqpoint{3.789661in}{4.146693in}}%
\pgfpathcurveto{\pgfqpoint{3.780453in}{4.155902in}}{\pgfqpoint{3.767962in}{4.161076in}}{\pgfqpoint{3.754939in}{4.161076in}}%
\pgfpathcurveto{\pgfqpoint{3.741916in}{4.161076in}}{\pgfqpoint{3.729425in}{4.155902in}}{\pgfqpoint{3.720217in}{4.146693in}}%
\pgfpathcurveto{\pgfqpoint{3.711008in}{4.137485in}}{\pgfqpoint{3.705834in}{4.124994in}}{\pgfqpoint{3.705834in}{4.111971in}}%
\pgfpathcurveto{\pgfqpoint{3.705834in}{4.098948in}}{\pgfqpoint{3.711008in}{4.086457in}}{\pgfqpoint{3.720217in}{4.077249in}}%
\pgfpathcurveto{\pgfqpoint{3.729425in}{4.068040in}}{\pgfqpoint{3.741916in}{4.062866in}}{\pgfqpoint{3.754939in}{4.062866in}}%
\pgfpathlineto{\pgfqpoint{3.754939in}{4.062866in}}%
\pgfpathclose%
\pgfusepath{stroke,fill}%
\end{pgfscope}%
\begin{pgfscope}%
\pgfpathrectangle{\pgfqpoint{0.786164in}{0.768110in}}{\pgfqpoint{8.851069in}{7.081890in}}%
\pgfusepath{clip}%
\pgfsetbuttcap%
\pgfsetroundjoin%
\definecolor{currentfill}{rgb}{0.280267,0.073417,0.397163}%
\pgfsetfillcolor{currentfill}%
\pgfsetfillopacity{0.700000}%
\pgfsetlinewidth{0.501875pt}%
\definecolor{currentstroke}{rgb}{1.000000,1.000000,1.000000}%
\pgfsetstrokecolor{currentstroke}%
\pgfsetstrokeopacity{0.700000}%
\pgfsetdash{}{0pt}%
\pgfpathmoveto{\pgfqpoint{3.736672in}{4.040968in}}%
\pgfpathcurveto{\pgfqpoint{3.749695in}{4.040968in}}{\pgfqpoint{3.762186in}{4.046142in}}{\pgfqpoint{3.771395in}{4.055351in}}%
\pgfpathcurveto{\pgfqpoint{3.780603in}{4.064559in}}{\pgfqpoint{3.785777in}{4.077050in}}{\pgfqpoint{3.785777in}{4.090073in}}%
\pgfpathcurveto{\pgfqpoint{3.785777in}{4.103096in}}{\pgfqpoint{3.780603in}{4.115587in}}{\pgfqpoint{3.771395in}{4.124795in}}%
\pgfpathcurveto{\pgfqpoint{3.762186in}{4.134004in}}{\pgfqpoint{3.749695in}{4.139177in}}{\pgfqpoint{3.736672in}{4.139177in}}%
\pgfpathcurveto{\pgfqpoint{3.723650in}{4.139177in}}{\pgfqpoint{3.711159in}{4.134004in}}{\pgfqpoint{3.701950in}{4.124795in}}%
\pgfpathcurveto{\pgfqpoint{3.692742in}{4.115587in}}{\pgfqpoint{3.687568in}{4.103096in}}{\pgfqpoint{3.687568in}{4.090073in}}%
\pgfpathcurveto{\pgfqpoint{3.687568in}{4.077050in}}{\pgfqpoint{3.692742in}{4.064559in}}{\pgfqpoint{3.701950in}{4.055351in}}%
\pgfpathcurveto{\pgfqpoint{3.711159in}{4.046142in}}{\pgfqpoint{3.723650in}{4.040968in}}{\pgfqpoint{3.736672in}{4.040968in}}%
\pgfpathlineto{\pgfqpoint{3.736672in}{4.040968in}}%
\pgfpathclose%
\pgfusepath{stroke,fill}%
\end{pgfscope}%
\begin{pgfscope}%
\pgfpathrectangle{\pgfqpoint{0.786164in}{0.768110in}}{\pgfqpoint{8.851069in}{7.081890in}}%
\pgfusepath{clip}%
\pgfsetbuttcap%
\pgfsetroundjoin%
\definecolor{currentfill}{rgb}{0.280894,0.078907,0.402329}%
\pgfsetfillcolor{currentfill}%
\pgfsetfillopacity{0.700000}%
\pgfsetlinewidth{0.501875pt}%
\definecolor{currentstroke}{rgb}{1.000000,1.000000,1.000000}%
\pgfsetstrokecolor{currentstroke}%
\pgfsetstrokeopacity{0.700000}%
\pgfsetdash{}{0pt}%
\pgfpathmoveto{\pgfqpoint{3.754939in}{4.019070in}}%
\pgfpathcurveto{\pgfqpoint{3.767962in}{4.019070in}}{\pgfqpoint{3.780453in}{4.024244in}}{\pgfqpoint{3.789661in}{4.033452in}}%
\pgfpathcurveto{\pgfqpoint{3.798870in}{4.042661in}}{\pgfqpoint{3.804044in}{4.055152in}}{\pgfqpoint{3.804044in}{4.068175in}}%
\pgfpathcurveto{\pgfqpoint{3.804044in}{4.081197in}}{\pgfqpoint{3.798870in}{4.093688in}}{\pgfqpoint{3.789661in}{4.102897in}}%
\pgfpathcurveto{\pgfqpoint{3.780453in}{4.112105in}}{\pgfqpoint{3.767962in}{4.117279in}}{\pgfqpoint{3.754939in}{4.117279in}}%
\pgfpathcurveto{\pgfqpoint{3.741916in}{4.117279in}}{\pgfqpoint{3.729425in}{4.112105in}}{\pgfqpoint{3.720217in}{4.102897in}}%
\pgfpathcurveto{\pgfqpoint{3.711008in}{4.093688in}}{\pgfqpoint{3.705834in}{4.081197in}}{\pgfqpoint{3.705834in}{4.068175in}}%
\pgfpathcurveto{\pgfqpoint{3.705834in}{4.055152in}}{\pgfqpoint{3.711008in}{4.042661in}}{\pgfqpoint{3.720217in}{4.033452in}}%
\pgfpathcurveto{\pgfqpoint{3.729425in}{4.024244in}}{\pgfqpoint{3.741916in}{4.019070in}}{\pgfqpoint{3.754939in}{4.019070in}}%
\pgfpathlineto{\pgfqpoint{3.754939in}{4.019070in}}%
\pgfpathclose%
\pgfusepath{stroke,fill}%
\end{pgfscope}%
\begin{pgfscope}%
\pgfpathrectangle{\pgfqpoint{0.786164in}{0.768110in}}{\pgfqpoint{8.851069in}{7.081890in}}%
\pgfusepath{clip}%
\pgfsetbuttcap%
\pgfsetroundjoin%
\definecolor{currentfill}{rgb}{0.282327,0.094955,0.417331}%
\pgfsetfillcolor{currentfill}%
\pgfsetfillopacity{0.700000}%
\pgfsetlinewidth{0.501875pt}%
\definecolor{currentstroke}{rgb}{1.000000,1.000000,1.000000}%
\pgfsetstrokecolor{currentstroke}%
\pgfsetstrokeopacity{0.700000}%
\pgfsetdash{}{0pt}%
\pgfpathmoveto{\pgfqpoint{3.608806in}{3.909579in}}%
\pgfpathcurveto{\pgfqpoint{3.621829in}{3.909579in}}{\pgfqpoint{3.634320in}{3.914753in}}{\pgfqpoint{3.643529in}{3.923961in}}%
\pgfpathcurveto{\pgfqpoint{3.652737in}{3.933170in}}{\pgfqpoint{3.657911in}{3.945661in}}{\pgfqpoint{3.657911in}{3.958683in}}%
\pgfpathcurveto{\pgfqpoint{3.657911in}{3.971706in}}{\pgfqpoint{3.652737in}{3.984197in}}{\pgfqpoint{3.643529in}{3.993406in}}%
\pgfpathcurveto{\pgfqpoint{3.634320in}{4.002614in}}{\pgfqpoint{3.621829in}{4.007788in}}{\pgfqpoint{3.608806in}{4.007788in}}%
\pgfpathcurveto{\pgfqpoint{3.595784in}{4.007788in}}{\pgfqpoint{3.583293in}{4.002614in}}{\pgfqpoint{3.574084in}{3.993406in}}%
\pgfpathcurveto{\pgfqpoint{3.564876in}{3.984197in}}{\pgfqpoint{3.559702in}{3.971706in}}{\pgfqpoint{3.559702in}{3.958683in}}%
\pgfpathcurveto{\pgfqpoint{3.559702in}{3.945661in}}{\pgfqpoint{3.564876in}{3.933170in}}{\pgfqpoint{3.574084in}{3.923961in}}%
\pgfpathcurveto{\pgfqpoint{3.583293in}{3.914753in}}{\pgfqpoint{3.595784in}{3.909579in}}{\pgfqpoint{3.608806in}{3.909579in}}%
\pgfpathlineto{\pgfqpoint{3.608806in}{3.909579in}}%
\pgfpathclose%
\pgfusepath{stroke,fill}%
\end{pgfscope}%
\begin{pgfscope}%
\pgfpathrectangle{\pgfqpoint{0.786164in}{0.768110in}}{\pgfqpoint{8.851069in}{7.081890in}}%
\pgfusepath{clip}%
\pgfsetbuttcap%
\pgfsetroundjoin%
\definecolor{currentfill}{rgb}{0.281446,0.084320,0.407414}%
\pgfsetfillcolor{currentfill}%
\pgfsetfillopacity{0.700000}%
\pgfsetlinewidth{0.501875pt}%
\definecolor{currentstroke}{rgb}{1.000000,1.000000,1.000000}%
\pgfsetstrokecolor{currentstroke}%
\pgfsetstrokeopacity{0.700000}%
\pgfsetdash{}{0pt}%
\pgfpathmoveto{\pgfqpoint{3.818872in}{4.062866in}}%
\pgfpathcurveto{\pgfqpoint{3.831895in}{4.062866in}}{\pgfqpoint{3.844386in}{4.068040in}}{\pgfqpoint{3.853594in}{4.077249in}}%
\pgfpathcurveto{\pgfqpoint{3.862803in}{4.086457in}}{\pgfqpoint{3.867977in}{4.098948in}}{\pgfqpoint{3.867977in}{4.111971in}}%
\pgfpathcurveto{\pgfqpoint{3.867977in}{4.124994in}}{\pgfqpoint{3.862803in}{4.137485in}}{\pgfqpoint{3.853594in}{4.146693in}}%
\pgfpathcurveto{\pgfqpoint{3.844386in}{4.155902in}}{\pgfqpoint{3.831895in}{4.161076in}}{\pgfqpoint{3.818872in}{4.161076in}}%
\pgfpathcurveto{\pgfqpoint{3.805849in}{4.161076in}}{\pgfqpoint{3.793358in}{4.155902in}}{\pgfqpoint{3.784150in}{4.146693in}}%
\pgfpathcurveto{\pgfqpoint{3.774941in}{4.137485in}}{\pgfqpoint{3.769767in}{4.124994in}}{\pgfqpoint{3.769767in}{4.111971in}}%
\pgfpathcurveto{\pgfqpoint{3.769767in}{4.098948in}}{\pgfqpoint{3.774941in}{4.086457in}}{\pgfqpoint{3.784150in}{4.077249in}}%
\pgfpathcurveto{\pgfqpoint{3.793358in}{4.068040in}}{\pgfqpoint{3.805849in}{4.062866in}}{\pgfqpoint{3.818872in}{4.062866in}}%
\pgfpathlineto{\pgfqpoint{3.818872in}{4.062866in}}%
\pgfpathclose%
\pgfusepath{stroke,fill}%
\end{pgfscope}%
\begin{pgfscope}%
\pgfpathrectangle{\pgfqpoint{0.786164in}{0.768110in}}{\pgfqpoint{8.851069in}{7.081890in}}%
\pgfusepath{clip}%
\pgfsetbuttcap%
\pgfsetroundjoin%
\definecolor{currentfill}{rgb}{0.282656,0.100196,0.422160}%
\pgfsetfillcolor{currentfill}%
\pgfsetfillopacity{0.700000}%
\pgfsetlinewidth{0.501875pt}%
\definecolor{currentstroke}{rgb}{1.000000,1.000000,1.000000}%
\pgfsetstrokecolor{currentstroke}%
\pgfsetstrokeopacity{0.700000}%
\pgfsetdash{}{0pt}%
\pgfpathmoveto{\pgfqpoint{3.544873in}{3.865782in}}%
\pgfpathcurveto{\pgfqpoint{3.557896in}{3.865782in}}{\pgfqpoint{3.570387in}{3.870956in}}{\pgfqpoint{3.579596in}{3.880165in}}%
\pgfpathcurveto{\pgfqpoint{3.588804in}{3.889373in}}{\pgfqpoint{3.593978in}{3.901864in}}{\pgfqpoint{3.593978in}{3.914887in}}%
\pgfpathcurveto{\pgfqpoint{3.593978in}{3.927910in}}{\pgfqpoint{3.588804in}{3.940401in}}{\pgfqpoint{3.579596in}{3.949609in}}%
\pgfpathcurveto{\pgfqpoint{3.570387in}{3.958818in}}{\pgfqpoint{3.557896in}{3.963992in}}{\pgfqpoint{3.544873in}{3.963992in}}%
\pgfpathcurveto{\pgfqpoint{3.531851in}{3.963992in}}{\pgfqpoint{3.519360in}{3.958818in}}{\pgfqpoint{3.510151in}{3.949609in}}%
\pgfpathcurveto{\pgfqpoint{3.500943in}{3.940401in}}{\pgfqpoint{3.495769in}{3.927910in}}{\pgfqpoint{3.495769in}{3.914887in}}%
\pgfpathcurveto{\pgfqpoint{3.495769in}{3.901864in}}{\pgfqpoint{3.500943in}{3.889373in}}{\pgfqpoint{3.510151in}{3.880165in}}%
\pgfpathcurveto{\pgfqpoint{3.519360in}{3.870956in}}{\pgfqpoint{3.531851in}{3.865782in}}{\pgfqpoint{3.544873in}{3.865782in}}%
\pgfpathlineto{\pgfqpoint{3.544873in}{3.865782in}}%
\pgfpathclose%
\pgfusepath{stroke,fill}%
\end{pgfscope}%
\begin{pgfscope}%
\pgfpathrectangle{\pgfqpoint{0.786164in}{0.768110in}}{\pgfqpoint{8.851069in}{7.081890in}}%
\pgfusepath{clip}%
\pgfsetbuttcap%
\pgfsetroundjoin%
\definecolor{currentfill}{rgb}{0.282656,0.100196,0.422160}%
\pgfsetfillcolor{currentfill}%
\pgfsetfillopacity{0.700000}%
\pgfsetlinewidth{0.501875pt}%
\definecolor{currentstroke}{rgb}{1.000000,1.000000,1.000000}%
\pgfsetstrokecolor{currentstroke}%
\pgfsetstrokeopacity{0.700000}%
\pgfsetdash{}{0pt}%
\pgfpathmoveto{\pgfqpoint{3.508340in}{3.778189in}}%
\pgfpathcurveto{\pgfqpoint{3.521363in}{3.778189in}}{\pgfqpoint{3.533854in}{3.783363in}}{\pgfqpoint{3.543062in}{3.792572in}}%
\pgfpathcurveto{\pgfqpoint{3.552271in}{3.801780in}}{\pgfqpoint{3.557445in}{3.814271in}}{\pgfqpoint{3.557445in}{3.827294in}}%
\pgfpathcurveto{\pgfqpoint{3.557445in}{3.840317in}}{\pgfqpoint{3.552271in}{3.852808in}}{\pgfqpoint{3.543062in}{3.862016in}}%
\pgfpathcurveto{\pgfqpoint{3.533854in}{3.871225in}}{\pgfqpoint{3.521363in}{3.876399in}}{\pgfqpoint{3.508340in}{3.876399in}}%
\pgfpathcurveto{\pgfqpoint{3.495318in}{3.876399in}}{\pgfqpoint{3.482826in}{3.871225in}}{\pgfqpoint{3.473618in}{3.862016in}}%
\pgfpathcurveto{\pgfqpoint{3.464410in}{3.852808in}}{\pgfqpoint{3.459236in}{3.840317in}}{\pgfqpoint{3.459236in}{3.827294in}}%
\pgfpathcurveto{\pgfqpoint{3.459236in}{3.814271in}}{\pgfqpoint{3.464410in}{3.801780in}}{\pgfqpoint{3.473618in}{3.792572in}}%
\pgfpathcurveto{\pgfqpoint{3.482826in}{3.783363in}}{\pgfqpoint{3.495318in}{3.778189in}}{\pgfqpoint{3.508340in}{3.778189in}}%
\pgfpathlineto{\pgfqpoint{3.508340in}{3.778189in}}%
\pgfpathclose%
\pgfusepath{stroke,fill}%
\end{pgfscope}%
\begin{pgfscope}%
\pgfpathrectangle{\pgfqpoint{0.786164in}{0.768110in}}{\pgfqpoint{8.851069in}{7.081890in}}%
\pgfusepath{clip}%
\pgfsetbuttcap%
\pgfsetroundjoin%
\definecolor{currentfill}{rgb}{0.282656,0.100196,0.422160}%
\pgfsetfillcolor{currentfill}%
\pgfsetfillopacity{0.700000}%
\pgfsetlinewidth{0.501875pt}%
\definecolor{currentstroke}{rgb}{1.000000,1.000000,1.000000}%
\pgfsetstrokecolor{currentstroke}%
\pgfsetstrokeopacity{0.700000}%
\pgfsetdash{}{0pt}%
\pgfpathmoveto{\pgfqpoint{3.471807in}{3.756291in}}%
\pgfpathcurveto{\pgfqpoint{3.484830in}{3.756291in}}{\pgfqpoint{3.497321in}{3.761465in}}{\pgfqpoint{3.506529in}{3.770674in}}%
\pgfpathcurveto{\pgfqpoint{3.515738in}{3.779882in}}{\pgfqpoint{3.520912in}{3.792373in}}{\pgfqpoint{3.520912in}{3.805396in}}%
\pgfpathcurveto{\pgfqpoint{3.520912in}{3.818418in}}{\pgfqpoint{3.515738in}{3.830910in}}{\pgfqpoint{3.506529in}{3.840118in}}%
\pgfpathcurveto{\pgfqpoint{3.497321in}{3.849326in}}{\pgfqpoint{3.484830in}{3.854500in}}{\pgfqpoint{3.471807in}{3.854500in}}%
\pgfpathcurveto{\pgfqpoint{3.458784in}{3.854500in}}{\pgfqpoint{3.446293in}{3.849326in}}{\pgfqpoint{3.437085in}{3.840118in}}%
\pgfpathcurveto{\pgfqpoint{3.427876in}{3.830910in}}{\pgfqpoint{3.422702in}{3.818418in}}{\pgfqpoint{3.422702in}{3.805396in}}%
\pgfpathcurveto{\pgfqpoint{3.422702in}{3.792373in}}{\pgfqpoint{3.427876in}{3.779882in}}{\pgfqpoint{3.437085in}{3.770674in}}%
\pgfpathcurveto{\pgfqpoint{3.446293in}{3.761465in}}{\pgfqpoint{3.458784in}{3.756291in}}{\pgfqpoint{3.471807in}{3.756291in}}%
\pgfpathlineto{\pgfqpoint{3.471807in}{3.756291in}}%
\pgfpathclose%
\pgfusepath{stroke,fill}%
\end{pgfscope}%
\begin{pgfscope}%
\pgfpathrectangle{\pgfqpoint{0.786164in}{0.768110in}}{\pgfqpoint{8.851069in}{7.081890in}}%
\pgfusepath{clip}%
\pgfsetbuttcap%
\pgfsetroundjoin%
\definecolor{currentfill}{rgb}{0.283197,0.115680,0.436115}%
\pgfsetfillcolor{currentfill}%
\pgfsetfillopacity{0.700000}%
\pgfsetlinewidth{0.501875pt}%
\definecolor{currentstroke}{rgb}{1.000000,1.000000,1.000000}%
\pgfsetstrokecolor{currentstroke}%
\pgfsetstrokeopacity{0.700000}%
\pgfsetdash{}{0pt}%
\pgfpathmoveto{\pgfqpoint{3.252608in}{3.668698in}}%
\pgfpathcurveto{\pgfqpoint{3.265631in}{3.668698in}}{\pgfqpoint{3.278122in}{3.673872in}}{\pgfqpoint{3.287330in}{3.683081in}}%
\pgfpathcurveto{\pgfqpoint{3.296539in}{3.692289in}}{\pgfqpoint{3.301713in}{3.704780in}}{\pgfqpoint{3.301713in}{3.717803in}}%
\pgfpathcurveto{\pgfqpoint{3.301713in}{3.730826in}}{\pgfqpoint{3.296539in}{3.743317in}}{\pgfqpoint{3.287330in}{3.752525in}}%
\pgfpathcurveto{\pgfqpoint{3.278122in}{3.761733in}}{\pgfqpoint{3.265631in}{3.766907in}}{\pgfqpoint{3.252608in}{3.766907in}}%
\pgfpathcurveto{\pgfqpoint{3.239585in}{3.766907in}}{\pgfqpoint{3.227094in}{3.761733in}}{\pgfqpoint{3.217886in}{3.752525in}}%
\pgfpathcurveto{\pgfqpoint{3.208678in}{3.743317in}}{\pgfqpoint{3.203504in}{3.730826in}}{\pgfqpoint{3.203504in}{3.717803in}}%
\pgfpathcurveto{\pgfqpoint{3.203504in}{3.704780in}}{\pgfqpoint{3.208678in}{3.692289in}}{\pgfqpoint{3.217886in}{3.683081in}}%
\pgfpathcurveto{\pgfqpoint{3.227094in}{3.673872in}}{\pgfqpoint{3.239585in}{3.668698in}}{\pgfqpoint{3.252608in}{3.668698in}}%
\pgfpathlineto{\pgfqpoint{3.252608in}{3.668698in}}%
\pgfpathclose%
\pgfusepath{stroke,fill}%
\end{pgfscope}%
\begin{pgfscope}%
\pgfpathrectangle{\pgfqpoint{0.786164in}{0.768110in}}{\pgfqpoint{8.851069in}{7.081890in}}%
\pgfusepath{clip}%
\pgfsetbuttcap%
\pgfsetroundjoin%
\definecolor{currentfill}{rgb}{0.283229,0.120777,0.440584}%
\pgfsetfillcolor{currentfill}%
\pgfsetfillopacity{0.700000}%
\pgfsetlinewidth{0.501875pt}%
\definecolor{currentstroke}{rgb}{1.000000,1.000000,1.000000}%
\pgfsetstrokecolor{currentstroke}%
\pgfsetstrokeopacity{0.700000}%
\pgfsetdash{}{0pt}%
\pgfpathmoveto{\pgfqpoint{3.051676in}{3.756291in}}%
\pgfpathcurveto{\pgfqpoint{3.064699in}{3.756291in}}{\pgfqpoint{3.077190in}{3.761465in}}{\pgfqpoint{3.086398in}{3.770674in}}%
\pgfpathcurveto{\pgfqpoint{3.095607in}{3.779882in}}{\pgfqpoint{3.100780in}{3.792373in}}{\pgfqpoint{3.100780in}{3.805396in}}%
\pgfpathcurveto{\pgfqpoint{3.100780in}{3.818418in}}{\pgfqpoint{3.095607in}{3.830910in}}{\pgfqpoint{3.086398in}{3.840118in}}%
\pgfpathcurveto{\pgfqpoint{3.077190in}{3.849326in}}{\pgfqpoint{3.064699in}{3.854500in}}{\pgfqpoint{3.051676in}{3.854500in}}%
\pgfpathcurveto{\pgfqpoint{3.038653in}{3.854500in}}{\pgfqpoint{3.026162in}{3.849326in}}{\pgfqpoint{3.016954in}{3.840118in}}%
\pgfpathcurveto{\pgfqpoint{3.007745in}{3.830910in}}{\pgfqpoint{3.002571in}{3.818418in}}{\pgfqpoint{3.002571in}{3.805396in}}%
\pgfpathcurveto{\pgfqpoint{3.002571in}{3.792373in}}{\pgfqpoint{3.007745in}{3.779882in}}{\pgfqpoint{3.016954in}{3.770674in}}%
\pgfpathcurveto{\pgfqpoint{3.026162in}{3.761465in}}{\pgfqpoint{3.038653in}{3.756291in}}{\pgfqpoint{3.051676in}{3.756291in}}%
\pgfpathlineto{\pgfqpoint{3.051676in}{3.756291in}}%
\pgfpathclose%
\pgfusepath{stroke,fill}%
\end{pgfscope}%
\begin{pgfscope}%
\pgfpathrectangle{\pgfqpoint{0.786164in}{0.768110in}}{\pgfqpoint{8.851069in}{7.081890in}}%
\pgfusepath{clip}%
\pgfsetbuttcap%
\pgfsetroundjoin%
\definecolor{currentfill}{rgb}{0.283187,0.125848,0.444960}%
\pgfsetfillcolor{currentfill}%
\pgfsetfillopacity{0.700000}%
\pgfsetlinewidth{0.501875pt}%
\definecolor{currentstroke}{rgb}{1.000000,1.000000,1.000000}%
\pgfsetstrokecolor{currentstroke}%
\pgfsetstrokeopacity{0.700000}%
\pgfsetdash{}{0pt}%
\pgfpathmoveto{\pgfqpoint{3.234342in}{3.756291in}}%
\pgfpathcurveto{\pgfqpoint{3.247364in}{3.756291in}}{\pgfqpoint{3.259855in}{3.761465in}}{\pgfqpoint{3.269064in}{3.770674in}}%
\pgfpathcurveto{\pgfqpoint{3.278272in}{3.779882in}}{\pgfqpoint{3.283446in}{3.792373in}}{\pgfqpoint{3.283446in}{3.805396in}}%
\pgfpathcurveto{\pgfqpoint{3.283446in}{3.818418in}}{\pgfqpoint{3.278272in}{3.830910in}}{\pgfqpoint{3.269064in}{3.840118in}}%
\pgfpathcurveto{\pgfqpoint{3.259855in}{3.849326in}}{\pgfqpoint{3.247364in}{3.854500in}}{\pgfqpoint{3.234342in}{3.854500in}}%
\pgfpathcurveto{\pgfqpoint{3.221319in}{3.854500in}}{\pgfqpoint{3.208828in}{3.849326in}}{\pgfqpoint{3.199619in}{3.840118in}}%
\pgfpathcurveto{\pgfqpoint{3.190411in}{3.830910in}}{\pgfqpoint{3.185237in}{3.818418in}}{\pgfqpoint{3.185237in}{3.805396in}}%
\pgfpathcurveto{\pgfqpoint{3.185237in}{3.792373in}}{\pgfqpoint{3.190411in}{3.779882in}}{\pgfqpoint{3.199619in}{3.770674in}}%
\pgfpathcurveto{\pgfqpoint{3.208828in}{3.761465in}}{\pgfqpoint{3.221319in}{3.756291in}}{\pgfqpoint{3.234342in}{3.756291in}}%
\pgfpathlineto{\pgfqpoint{3.234342in}{3.756291in}}%
\pgfpathclose%
\pgfusepath{stroke,fill}%
\end{pgfscope}%
\begin{pgfscope}%
\pgfpathrectangle{\pgfqpoint{0.786164in}{0.768110in}}{\pgfqpoint{8.851069in}{7.081890in}}%
\pgfusepath{clip}%
\pgfsetbuttcap%
\pgfsetroundjoin%
\definecolor{currentfill}{rgb}{0.282884,0.135920,0.453427}%
\pgfsetfillcolor{currentfill}%
\pgfsetfillopacity{0.700000}%
\pgfsetlinewidth{0.501875pt}%
\definecolor{currentstroke}{rgb}{1.000000,1.000000,1.000000}%
\pgfsetstrokecolor{currentstroke}%
\pgfsetstrokeopacity{0.700000}%
\pgfsetdash{}{0pt}%
\pgfpathmoveto{\pgfqpoint{3.143009in}{3.690596in}}%
\pgfpathcurveto{\pgfqpoint{3.156031in}{3.690596in}}{\pgfqpoint{3.168523in}{3.695770in}}{\pgfqpoint{3.177731in}{3.704979in}}%
\pgfpathcurveto{\pgfqpoint{3.186939in}{3.714187in}}{\pgfqpoint{3.192113in}{3.726678in}}{\pgfqpoint{3.192113in}{3.739701in}}%
\pgfpathcurveto{\pgfqpoint{3.192113in}{3.752724in}}{\pgfqpoint{3.186939in}{3.765215in}}{\pgfqpoint{3.177731in}{3.774423in}}%
\pgfpathcurveto{\pgfqpoint{3.168523in}{3.783632in}}{\pgfqpoint{3.156031in}{3.788806in}}{\pgfqpoint{3.143009in}{3.788806in}}%
\pgfpathcurveto{\pgfqpoint{3.129986in}{3.788806in}}{\pgfqpoint{3.117495in}{3.783632in}}{\pgfqpoint{3.108286in}{3.774423in}}%
\pgfpathcurveto{\pgfqpoint{3.099078in}{3.765215in}}{\pgfqpoint{3.093904in}{3.752724in}}{\pgfqpoint{3.093904in}{3.739701in}}%
\pgfpathcurveto{\pgfqpoint{3.093904in}{3.726678in}}{\pgfqpoint{3.099078in}{3.714187in}}{\pgfqpoint{3.108286in}{3.704979in}}%
\pgfpathcurveto{\pgfqpoint{3.117495in}{3.695770in}}{\pgfqpoint{3.129986in}{3.690596in}}{\pgfqpoint{3.143009in}{3.690596in}}%
\pgfpathlineto{\pgfqpoint{3.143009in}{3.690596in}}%
\pgfpathclose%
\pgfusepath{stroke,fill}%
\end{pgfscope}%
\begin{pgfscope}%
\pgfpathrectangle{\pgfqpoint{0.786164in}{0.768110in}}{\pgfqpoint{8.851069in}{7.081890in}}%
\pgfusepath{clip}%
\pgfsetbuttcap%
\pgfsetroundjoin%
\definecolor{currentfill}{rgb}{0.281412,0.155834,0.469201}%
\pgfsetfillcolor{currentfill}%
\pgfsetfillopacity{0.700000}%
\pgfsetlinewidth{0.501875pt}%
\definecolor{currentstroke}{rgb}{1.000000,1.000000,1.000000}%
\pgfsetstrokecolor{currentstroke}%
\pgfsetstrokeopacity{0.700000}%
\pgfsetdash{}{0pt}%
\pgfpathmoveto{\pgfqpoint{2.969476in}{3.603003in}}%
\pgfpathcurveto{\pgfqpoint{2.982499in}{3.603003in}}{\pgfqpoint{2.994990in}{3.608177in}}{\pgfqpoint{3.004198in}{3.617386in}}%
\pgfpathcurveto{\pgfqpoint{3.013407in}{3.626594in}}{\pgfqpoint{3.018581in}{3.639085in}}{\pgfqpoint{3.018581in}{3.652108in}}%
\pgfpathcurveto{\pgfqpoint{3.018581in}{3.665131in}}{\pgfqpoint{3.013407in}{3.677622in}}{\pgfqpoint{3.004198in}{3.686830in}}%
\pgfpathcurveto{\pgfqpoint{2.994990in}{3.696039in}}{\pgfqpoint{2.982499in}{3.701213in}}{\pgfqpoint{2.969476in}{3.701213in}}%
\pgfpathcurveto{\pgfqpoint{2.956454in}{3.701213in}}{\pgfqpoint{2.943962in}{3.696039in}}{\pgfqpoint{2.934754in}{3.686830in}}%
\pgfpathcurveto{\pgfqpoint{2.925546in}{3.677622in}}{\pgfqpoint{2.920372in}{3.665131in}}{\pgfqpoint{2.920372in}{3.652108in}}%
\pgfpathcurveto{\pgfqpoint{2.920372in}{3.639085in}}{\pgfqpoint{2.925546in}{3.626594in}}{\pgfqpoint{2.934754in}{3.617386in}}%
\pgfpathcurveto{\pgfqpoint{2.943962in}{3.608177in}}{\pgfqpoint{2.956454in}{3.603003in}}{\pgfqpoint{2.969476in}{3.603003in}}%
\pgfpathlineto{\pgfqpoint{2.969476in}{3.603003in}}%
\pgfpathclose%
\pgfusepath{stroke,fill}%
\end{pgfscope}%
\begin{pgfscope}%
\pgfpathrectangle{\pgfqpoint{0.786164in}{0.768110in}}{\pgfqpoint{8.851069in}{7.081890in}}%
\pgfusepath{clip}%
\pgfsetbuttcap%
\pgfsetroundjoin%
\definecolor{currentfill}{rgb}{0.277134,0.185228,0.489898}%
\pgfsetfillcolor{currentfill}%
\pgfsetfillopacity{0.700000}%
\pgfsetlinewidth{0.501875pt}%
\definecolor{currentstroke}{rgb}{1.000000,1.000000,1.000000}%
\pgfsetstrokecolor{currentstroke}%
\pgfsetstrokeopacity{0.700000}%
\pgfsetdash{}{0pt}%
\pgfpathmoveto{\pgfqpoint{3.234342in}{3.515411in}}%
\pgfpathcurveto{\pgfqpoint{3.247364in}{3.515411in}}{\pgfqpoint{3.259855in}{3.520585in}}{\pgfqpoint{3.269064in}{3.529793in}}%
\pgfpathcurveto{\pgfqpoint{3.278272in}{3.539001in}}{\pgfqpoint{3.283446in}{3.551492in}}{\pgfqpoint{3.283446in}{3.564515in}}%
\pgfpathcurveto{\pgfqpoint{3.283446in}{3.577538in}}{\pgfqpoint{3.278272in}{3.590029in}}{\pgfqpoint{3.269064in}{3.599237in}}%
\pgfpathcurveto{\pgfqpoint{3.259855in}{3.608446in}}{\pgfqpoint{3.247364in}{3.613620in}}{\pgfqpoint{3.234342in}{3.613620in}}%
\pgfpathcurveto{\pgfqpoint{3.221319in}{3.613620in}}{\pgfqpoint{3.208828in}{3.608446in}}{\pgfqpoint{3.199619in}{3.599237in}}%
\pgfpathcurveto{\pgfqpoint{3.190411in}{3.590029in}}{\pgfqpoint{3.185237in}{3.577538in}}{\pgfqpoint{3.185237in}{3.564515in}}%
\pgfpathcurveto{\pgfqpoint{3.185237in}{3.551492in}}{\pgfqpoint{3.190411in}{3.539001in}}{\pgfqpoint{3.199619in}{3.529793in}}%
\pgfpathcurveto{\pgfqpoint{3.208828in}{3.520585in}}{\pgfqpoint{3.221319in}{3.515411in}}{\pgfqpoint{3.234342in}{3.515411in}}%
\pgfpathlineto{\pgfqpoint{3.234342in}{3.515411in}}%
\pgfpathclose%
\pgfusepath{stroke,fill}%
\end{pgfscope}%
\begin{pgfscope}%
\pgfpathrectangle{\pgfqpoint{0.786164in}{0.768110in}}{\pgfqpoint{8.851069in}{7.081890in}}%
\pgfusepath{clip}%
\pgfsetbuttcap%
\pgfsetroundjoin%
\definecolor{currentfill}{rgb}{0.250425,0.274290,0.533103}%
\pgfsetfillcolor{currentfill}%
\pgfsetfillopacity{0.700000}%
\pgfsetlinewidth{0.501875pt}%
\definecolor{currentstroke}{rgb}{1.000000,1.000000,1.000000}%
\pgfsetstrokecolor{currentstroke}%
\pgfsetstrokeopacity{0.700000}%
\pgfsetdash{}{0pt}%
\pgfpathmoveto{\pgfqpoint{2.932943in}{3.230733in}}%
\pgfpathcurveto{\pgfqpoint{2.945966in}{3.230733in}}{\pgfqpoint{2.958457in}{3.235907in}}{\pgfqpoint{2.967665in}{3.245116in}}%
\pgfpathcurveto{\pgfqpoint{2.976874in}{3.254324in}}{\pgfqpoint{2.982048in}{3.266815in}}{\pgfqpoint{2.982048in}{3.279838in}}%
\pgfpathcurveto{\pgfqpoint{2.982048in}{3.292861in}}{\pgfqpoint{2.976874in}{3.305352in}}{\pgfqpoint{2.967665in}{3.314560in}}%
\pgfpathcurveto{\pgfqpoint{2.958457in}{3.323769in}}{\pgfqpoint{2.945966in}{3.328943in}}{\pgfqpoint{2.932943in}{3.328943in}}%
\pgfpathcurveto{\pgfqpoint{2.919920in}{3.328943in}}{\pgfqpoint{2.907429in}{3.323769in}}{\pgfqpoint{2.898221in}{3.314560in}}%
\pgfpathcurveto{\pgfqpoint{2.889012in}{3.305352in}}{\pgfqpoint{2.883838in}{3.292861in}}{\pgfqpoint{2.883838in}{3.279838in}}%
\pgfpathcurveto{\pgfqpoint{2.883838in}{3.266815in}}{\pgfqpoint{2.889012in}{3.254324in}}{\pgfqpoint{2.898221in}{3.245116in}}%
\pgfpathcurveto{\pgfqpoint{2.907429in}{3.235907in}}{\pgfqpoint{2.919920in}{3.230733in}}{\pgfqpoint{2.932943in}{3.230733in}}%
\pgfpathlineto{\pgfqpoint{2.932943in}{3.230733in}}%
\pgfpathclose%
\pgfusepath{stroke,fill}%
\end{pgfscope}%
\begin{pgfscope}%
\pgfpathrectangle{\pgfqpoint{0.786164in}{0.768110in}}{\pgfqpoint{8.851069in}{7.081890in}}%
\pgfusepath{clip}%
\pgfsetbuttcap%
\pgfsetroundjoin%
\definecolor{currentfill}{rgb}{0.255645,0.260703,0.528312}%
\pgfsetfillcolor{currentfill}%
\pgfsetfillopacity{0.700000}%
\pgfsetlinewidth{0.501875pt}%
\definecolor{currentstroke}{rgb}{1.000000,1.000000,1.000000}%
\pgfsetstrokecolor{currentstroke}%
\pgfsetstrokeopacity{0.700000}%
\pgfsetdash{}{0pt}%
\pgfpathmoveto{\pgfqpoint{3.033409in}{3.252632in}}%
\pgfpathcurveto{\pgfqpoint{3.046432in}{3.252632in}}{\pgfqpoint{3.058923in}{3.257806in}}{\pgfqpoint{3.068131in}{3.267014in}}%
\pgfpathcurveto{\pgfqpoint{3.077340in}{3.276223in}}{\pgfqpoint{3.082514in}{3.288714in}}{\pgfqpoint{3.082514in}{3.301736in}}%
\pgfpathcurveto{\pgfqpoint{3.082514in}{3.314759in}}{\pgfqpoint{3.077340in}{3.327250in}}{\pgfqpoint{3.068131in}{3.336459in}}%
\pgfpathcurveto{\pgfqpoint{3.058923in}{3.345667in}}{\pgfqpoint{3.046432in}{3.350841in}}{\pgfqpoint{3.033409in}{3.350841in}}%
\pgfpathcurveto{\pgfqpoint{3.020387in}{3.350841in}}{\pgfqpoint{3.007895in}{3.345667in}}{\pgfqpoint{2.998687in}{3.336459in}}%
\pgfpathcurveto{\pgfqpoint{2.989479in}{3.327250in}}{\pgfqpoint{2.984305in}{3.314759in}}{\pgfqpoint{2.984305in}{3.301736in}}%
\pgfpathcurveto{\pgfqpoint{2.984305in}{3.288714in}}{\pgfqpoint{2.989479in}{3.276223in}}{\pgfqpoint{2.998687in}{3.267014in}}%
\pgfpathcurveto{\pgfqpoint{3.007895in}{3.257806in}}{\pgfqpoint{3.020387in}{3.252632in}}{\pgfqpoint{3.033409in}{3.252632in}}%
\pgfpathlineto{\pgfqpoint{3.033409in}{3.252632in}}%
\pgfpathclose%
\pgfusepath{stroke,fill}%
\end{pgfscope}%
\begin{pgfscope}%
\pgfpathrectangle{\pgfqpoint{0.786164in}{0.768110in}}{\pgfqpoint{8.851069in}{7.081890in}}%
\pgfusepath{clip}%
\pgfsetbuttcap%
\pgfsetroundjoin%
\definecolor{currentfill}{rgb}{0.243113,0.292092,0.538516}%
\pgfsetfillcolor{currentfill}%
\pgfsetfillopacity{0.700000}%
\pgfsetlinewidth{0.501875pt}%
\definecolor{currentstroke}{rgb}{1.000000,1.000000,1.000000}%
\pgfsetstrokecolor{currentstroke}%
\pgfsetstrokeopacity{0.700000}%
\pgfsetdash{}{0pt}%
\pgfpathmoveto{\pgfqpoint{2.768544in}{3.099344in}}%
\pgfpathcurveto{\pgfqpoint{2.781567in}{3.099344in}}{\pgfqpoint{2.794058in}{3.104518in}}{\pgfqpoint{2.803266in}{3.113726in}}%
\pgfpathcurveto{\pgfqpoint{2.812475in}{3.122935in}}{\pgfqpoint{2.817649in}{3.135426in}}{\pgfqpoint{2.817649in}{3.148449in}}%
\pgfpathcurveto{\pgfqpoint{2.817649in}{3.161471in}}{\pgfqpoint{2.812475in}{3.173962in}}{\pgfqpoint{2.803266in}{3.183171in}}%
\pgfpathcurveto{\pgfqpoint{2.794058in}{3.192379in}}{\pgfqpoint{2.781567in}{3.197553in}}{\pgfqpoint{2.768544in}{3.197553in}}%
\pgfpathcurveto{\pgfqpoint{2.755521in}{3.197553in}}{\pgfqpoint{2.743030in}{3.192379in}}{\pgfqpoint{2.733822in}{3.183171in}}%
\pgfpathcurveto{\pgfqpoint{2.724613in}{3.173962in}}{\pgfqpoint{2.719439in}{3.161471in}}{\pgfqpoint{2.719439in}{3.148449in}}%
\pgfpathcurveto{\pgfqpoint{2.719439in}{3.135426in}}{\pgfqpoint{2.724613in}{3.122935in}}{\pgfqpoint{2.733822in}{3.113726in}}%
\pgfpathcurveto{\pgfqpoint{2.743030in}{3.104518in}}{\pgfqpoint{2.755521in}{3.099344in}}{\pgfqpoint{2.768544in}{3.099344in}}%
\pgfpathlineto{\pgfqpoint{2.768544in}{3.099344in}}%
\pgfpathclose%
\pgfusepath{stroke,fill}%
\end{pgfscope}%
\begin{pgfscope}%
\pgfpathrectangle{\pgfqpoint{0.786164in}{0.768110in}}{\pgfqpoint{8.851069in}{7.081890in}}%
\pgfusepath{clip}%
\pgfsetbuttcap%
\pgfsetroundjoin%
\definecolor{currentfill}{rgb}{0.282290,0.145912,0.461510}%
\pgfsetfillcolor{currentfill}%
\pgfsetfillopacity{0.700000}%
\pgfsetlinewidth{0.501875pt}%
\definecolor{currentstroke}{rgb}{1.000000,1.000000,1.000000}%
\pgfsetstrokecolor{currentstroke}%
\pgfsetstrokeopacity{0.700000}%
\pgfsetdash{}{0pt}%
\pgfpathmoveto{\pgfqpoint{1.955681in}{3.033649in}}%
\pgfpathcurveto{\pgfqpoint{1.968704in}{3.033649in}}{\pgfqpoint{1.981195in}{3.038823in}}{\pgfqpoint{1.990404in}{3.048032in}}%
\pgfpathcurveto{\pgfqpoint{1.999612in}{3.057240in}}{\pgfqpoint{2.004786in}{3.069731in}}{\pgfqpoint{2.004786in}{3.082754in}}%
\pgfpathcurveto{\pgfqpoint{2.004786in}{3.095777in}}{\pgfqpoint{1.999612in}{3.108268in}}{\pgfqpoint{1.990404in}{3.117476in}}%
\pgfpathcurveto{\pgfqpoint{1.981195in}{3.126685in}}{\pgfqpoint{1.968704in}{3.131859in}}{\pgfqpoint{1.955681in}{3.131859in}}%
\pgfpathcurveto{\pgfqpoint{1.942659in}{3.131859in}}{\pgfqpoint{1.930168in}{3.126685in}}{\pgfqpoint{1.920959in}{3.117476in}}%
\pgfpathcurveto{\pgfqpoint{1.911751in}{3.108268in}}{\pgfqpoint{1.906577in}{3.095777in}}{\pgfqpoint{1.906577in}{3.082754in}}%
\pgfpathcurveto{\pgfqpoint{1.906577in}{3.069731in}}{\pgfqpoint{1.911751in}{3.057240in}}{\pgfqpoint{1.920959in}{3.048032in}}%
\pgfpathcurveto{\pgfqpoint{1.930168in}{3.038823in}}{\pgfqpoint{1.942659in}{3.033649in}}{\pgfqpoint{1.955681in}{3.033649in}}%
\pgfpathlineto{\pgfqpoint{1.955681in}{3.033649in}}%
\pgfpathclose%
\pgfusepath{stroke,fill}%
\end{pgfscope}%
\begin{pgfscope}%
\pgfpathrectangle{\pgfqpoint{0.786164in}{0.768110in}}{\pgfqpoint{8.851069in}{7.081890in}}%
\pgfusepath{clip}%
\pgfsetbuttcap%
\pgfsetroundjoin%
\definecolor{currentfill}{rgb}{0.282290,0.145912,0.461510}%
\pgfsetfillcolor{currentfill}%
\pgfsetfillopacity{0.700000}%
\pgfsetlinewidth{0.501875pt}%
\definecolor{currentstroke}{rgb}{1.000000,1.000000,1.000000}%
\pgfsetstrokecolor{currentstroke}%
\pgfsetstrokeopacity{0.700000}%
\pgfsetdash{}{0pt}%
\pgfpathmoveto{\pgfqpoint{1.973948in}{3.033649in}}%
\pgfpathcurveto{\pgfqpoint{1.986971in}{3.033649in}}{\pgfqpoint{1.999462in}{3.038823in}}{\pgfqpoint{2.008670in}{3.048032in}}%
\pgfpathcurveto{\pgfqpoint{2.017879in}{3.057240in}}{\pgfqpoint{2.023053in}{3.069731in}}{\pgfqpoint{2.023053in}{3.082754in}}%
\pgfpathcurveto{\pgfqpoint{2.023053in}{3.095777in}}{\pgfqpoint{2.017879in}{3.108268in}}{\pgfqpoint{2.008670in}{3.117476in}}%
\pgfpathcurveto{\pgfqpoint{1.999462in}{3.126685in}}{\pgfqpoint{1.986971in}{3.131859in}}{\pgfqpoint{1.973948in}{3.131859in}}%
\pgfpathcurveto{\pgfqpoint{1.960925in}{3.131859in}}{\pgfqpoint{1.948434in}{3.126685in}}{\pgfqpoint{1.939226in}{3.117476in}}%
\pgfpathcurveto{\pgfqpoint{1.930017in}{3.108268in}}{\pgfqpoint{1.924843in}{3.095777in}}{\pgfqpoint{1.924843in}{3.082754in}}%
\pgfpathcurveto{\pgfqpoint{1.924843in}{3.069731in}}{\pgfqpoint{1.930017in}{3.057240in}}{\pgfqpoint{1.939226in}{3.048032in}}%
\pgfpathcurveto{\pgfqpoint{1.948434in}{3.038823in}}{\pgfqpoint{1.960925in}{3.033649in}}{\pgfqpoint{1.973948in}{3.033649in}}%
\pgfpathlineto{\pgfqpoint{1.973948in}{3.033649in}}%
\pgfpathclose%
\pgfusepath{stroke,fill}%
\end{pgfscope}%
\begin{pgfscope}%
\pgfpathrectangle{\pgfqpoint{0.786164in}{0.768110in}}{\pgfqpoint{8.851069in}{7.081890in}}%
\pgfusepath{clip}%
\pgfsetbuttcap%
\pgfsetroundjoin%
\definecolor{currentfill}{rgb}{0.282290,0.145912,0.461510}%
\pgfsetfillcolor{currentfill}%
\pgfsetfillopacity{0.700000}%
\pgfsetlinewidth{0.501875pt}%
\definecolor{currentstroke}{rgb}{1.000000,1.000000,1.000000}%
\pgfsetstrokecolor{currentstroke}%
\pgfsetstrokeopacity{0.700000}%
\pgfsetdash{}{0pt}%
\pgfpathmoveto{\pgfqpoint{2.083547in}{3.165039in}}%
\pgfpathcurveto{\pgfqpoint{2.096570in}{3.165039in}}{\pgfqpoint{2.109061in}{3.170213in}}{\pgfqpoint{2.118270in}{3.179421in}}%
\pgfpathcurveto{\pgfqpoint{2.127478in}{3.188630in}}{\pgfqpoint{2.132652in}{3.201121in}}{\pgfqpoint{2.132652in}{3.214143in}}%
\pgfpathcurveto{\pgfqpoint{2.132652in}{3.227166in}}{\pgfqpoint{2.127478in}{3.239657in}}{\pgfqpoint{2.118270in}{3.248866in}}%
\pgfpathcurveto{\pgfqpoint{2.109061in}{3.258074in}}{\pgfqpoint{2.096570in}{3.263248in}}{\pgfqpoint{2.083547in}{3.263248in}}%
\pgfpathcurveto{\pgfqpoint{2.070525in}{3.263248in}}{\pgfqpoint{2.058034in}{3.258074in}}{\pgfqpoint{2.048825in}{3.248866in}}%
\pgfpathcurveto{\pgfqpoint{2.039617in}{3.239657in}}{\pgfqpoint{2.034443in}{3.227166in}}{\pgfqpoint{2.034443in}{3.214143in}}%
\pgfpathcurveto{\pgfqpoint{2.034443in}{3.201121in}}{\pgfqpoint{2.039617in}{3.188630in}}{\pgfqpoint{2.048825in}{3.179421in}}%
\pgfpathcurveto{\pgfqpoint{2.058034in}{3.170213in}}{\pgfqpoint{2.070525in}{3.165039in}}{\pgfqpoint{2.083547in}{3.165039in}}%
\pgfpathlineto{\pgfqpoint{2.083547in}{3.165039in}}%
\pgfpathclose%
\pgfusepath{stroke,fill}%
\end{pgfscope}%
\begin{pgfscope}%
\pgfpathrectangle{\pgfqpoint{0.786164in}{0.768110in}}{\pgfqpoint{8.851069in}{7.081890in}}%
\pgfusepath{clip}%
\pgfsetbuttcap%
\pgfsetroundjoin%
\definecolor{currentfill}{rgb}{0.282290,0.145912,0.461510}%
\pgfsetfillcolor{currentfill}%
\pgfsetfillopacity{0.700000}%
\pgfsetlinewidth{0.501875pt}%
\definecolor{currentstroke}{rgb}{1.000000,1.000000,1.000000}%
\pgfsetstrokecolor{currentstroke}%
\pgfsetstrokeopacity{0.700000}%
\pgfsetdash{}{0pt}%
\pgfpathmoveto{\pgfqpoint{2.074414in}{3.186937in}}%
\pgfpathcurveto{\pgfqpoint{2.087437in}{3.186937in}}{\pgfqpoint{2.099928in}{3.192111in}}{\pgfqpoint{2.109136in}{3.201319in}}%
\pgfpathcurveto{\pgfqpoint{2.118345in}{3.210528in}}{\pgfqpoint{2.123519in}{3.223019in}}{\pgfqpoint{2.123519in}{3.236042in}}%
\pgfpathcurveto{\pgfqpoint{2.123519in}{3.249064in}}{\pgfqpoint{2.118345in}{3.261555in}}{\pgfqpoint{2.109136in}{3.270764in}}%
\pgfpathcurveto{\pgfqpoint{2.099928in}{3.279972in}}{\pgfqpoint{2.087437in}{3.285146in}}{\pgfqpoint{2.074414in}{3.285146in}}%
\pgfpathcurveto{\pgfqpoint{2.061391in}{3.285146in}}{\pgfqpoint{2.048900in}{3.279972in}}{\pgfqpoint{2.039692in}{3.270764in}}%
\pgfpathcurveto{\pgfqpoint{2.030483in}{3.261555in}}{\pgfqpoint{2.025309in}{3.249064in}}{\pgfqpoint{2.025309in}{3.236042in}}%
\pgfpathcurveto{\pgfqpoint{2.025309in}{3.223019in}}{\pgfqpoint{2.030483in}{3.210528in}}{\pgfqpoint{2.039692in}{3.201319in}}%
\pgfpathcurveto{\pgfqpoint{2.048900in}{3.192111in}}{\pgfqpoint{2.061391in}{3.186937in}}{\pgfqpoint{2.074414in}{3.186937in}}%
\pgfpathlineto{\pgfqpoint{2.074414in}{3.186937in}}%
\pgfpathclose%
\pgfusepath{stroke,fill}%
\end{pgfscope}%
\begin{pgfscope}%
\pgfpathrectangle{\pgfqpoint{0.786164in}{0.768110in}}{\pgfqpoint{8.851069in}{7.081890in}}%
\pgfusepath{clip}%
\pgfsetbuttcap%
\pgfsetroundjoin%
\definecolor{currentfill}{rgb}{0.280868,0.160771,0.472899}%
\pgfsetfillcolor{currentfill}%
\pgfsetfillopacity{0.700000}%
\pgfsetlinewidth{0.501875pt}%
\definecolor{currentstroke}{rgb}{1.000000,1.000000,1.000000}%
\pgfsetstrokecolor{currentstroke}%
\pgfsetstrokeopacity{0.700000}%
\pgfsetdash{}{0pt}%
\pgfpathmoveto{\pgfqpoint{2.101814in}{3.165039in}}%
\pgfpathcurveto{\pgfqpoint{2.114837in}{3.165039in}}{\pgfqpoint{2.127328in}{3.170213in}}{\pgfqpoint{2.136536in}{3.179421in}}%
\pgfpathcurveto{\pgfqpoint{2.145745in}{3.188630in}}{\pgfqpoint{2.150919in}{3.201121in}}{\pgfqpoint{2.150919in}{3.214143in}}%
\pgfpathcurveto{\pgfqpoint{2.150919in}{3.227166in}}{\pgfqpoint{2.145745in}{3.239657in}}{\pgfqpoint{2.136536in}{3.248866in}}%
\pgfpathcurveto{\pgfqpoint{2.127328in}{3.258074in}}{\pgfqpoint{2.114837in}{3.263248in}}{\pgfqpoint{2.101814in}{3.263248in}}%
\pgfpathcurveto{\pgfqpoint{2.088791in}{3.263248in}}{\pgfqpoint{2.076300in}{3.258074in}}{\pgfqpoint{2.067092in}{3.248866in}}%
\pgfpathcurveto{\pgfqpoint{2.057883in}{3.239657in}}{\pgfqpoint{2.052709in}{3.227166in}}{\pgfqpoint{2.052709in}{3.214143in}}%
\pgfpathcurveto{\pgfqpoint{2.052709in}{3.201121in}}{\pgfqpoint{2.057883in}{3.188630in}}{\pgfqpoint{2.067092in}{3.179421in}}%
\pgfpathcurveto{\pgfqpoint{2.076300in}{3.170213in}}{\pgfqpoint{2.088791in}{3.165039in}}{\pgfqpoint{2.101814in}{3.165039in}}%
\pgfpathlineto{\pgfqpoint{2.101814in}{3.165039in}}%
\pgfpathclose%
\pgfusepath{stroke,fill}%
\end{pgfscope}%
\begin{pgfscope}%
\pgfpathrectangle{\pgfqpoint{0.786164in}{0.768110in}}{\pgfqpoint{8.851069in}{7.081890in}}%
\pgfusepath{clip}%
\pgfsetbuttcap%
\pgfsetroundjoin%
\definecolor{currentfill}{rgb}{0.278012,0.180367,0.486697}%
\pgfsetfillcolor{currentfill}%
\pgfsetfillopacity{0.700000}%
\pgfsetlinewidth{0.501875pt}%
\definecolor{currentstroke}{rgb}{1.000000,1.000000,1.000000}%
\pgfsetstrokecolor{currentstroke}%
\pgfsetstrokeopacity{0.700000}%
\pgfsetdash{}{0pt}%
\pgfpathmoveto{\pgfqpoint{2.019614in}{3.055548in}}%
\pgfpathcurveto{\pgfqpoint{2.032637in}{3.055548in}}{\pgfqpoint{2.045128in}{3.060722in}}{\pgfqpoint{2.054337in}{3.069930in}}%
\pgfpathcurveto{\pgfqpoint{2.063545in}{3.079138in}}{\pgfqpoint{2.068719in}{3.091630in}}{\pgfqpoint{2.068719in}{3.104652in}}%
\pgfpathcurveto{\pgfqpoint{2.068719in}{3.117675in}}{\pgfqpoint{2.063545in}{3.130166in}}{\pgfqpoint{2.054337in}{3.139374in}}%
\pgfpathcurveto{\pgfqpoint{2.045128in}{3.148583in}}{\pgfqpoint{2.032637in}{3.153757in}}{\pgfqpoint{2.019614in}{3.153757in}}%
\pgfpathcurveto{\pgfqpoint{2.006592in}{3.153757in}}{\pgfqpoint{1.994101in}{3.148583in}}{\pgfqpoint{1.984892in}{3.139374in}}%
\pgfpathcurveto{\pgfqpoint{1.975684in}{3.130166in}}{\pgfqpoint{1.970510in}{3.117675in}}{\pgfqpoint{1.970510in}{3.104652in}}%
\pgfpathcurveto{\pgfqpoint{1.970510in}{3.091630in}}{\pgfqpoint{1.975684in}{3.079138in}}{\pgfqpoint{1.984892in}{3.069930in}}%
\pgfpathcurveto{\pgfqpoint{1.994101in}{3.060722in}}{\pgfqpoint{2.006592in}{3.055548in}}{\pgfqpoint{2.019614in}{3.055548in}}%
\pgfpathlineto{\pgfqpoint{2.019614in}{3.055548in}}%
\pgfpathclose%
\pgfusepath{stroke,fill}%
\end{pgfscope}%
\begin{pgfscope}%
\pgfpathrectangle{\pgfqpoint{0.786164in}{0.768110in}}{\pgfqpoint{8.851069in}{7.081890in}}%
\pgfusepath{clip}%
\pgfsetbuttcap%
\pgfsetroundjoin%
\definecolor{currentfill}{rgb}{0.276194,0.190074,0.493001}%
\pgfsetfillcolor{currentfill}%
\pgfsetfillopacity{0.700000}%
\pgfsetlinewidth{0.501875pt}%
\definecolor{currentstroke}{rgb}{1.000000,1.000000,1.000000}%
\pgfsetstrokecolor{currentstroke}%
\pgfsetstrokeopacity{0.700000}%
\pgfsetdash{}{0pt}%
\pgfpathmoveto{\pgfqpoint{2.193147in}{3.165039in}}%
\pgfpathcurveto{\pgfqpoint{2.206170in}{3.165039in}}{\pgfqpoint{2.218661in}{3.170213in}}{\pgfqpoint{2.227869in}{3.179421in}}%
\pgfpathcurveto{\pgfqpoint{2.237077in}{3.188630in}}{\pgfqpoint{2.242251in}{3.201121in}}{\pgfqpoint{2.242251in}{3.214143in}}%
\pgfpathcurveto{\pgfqpoint{2.242251in}{3.227166in}}{\pgfqpoint{2.237077in}{3.239657in}}{\pgfqpoint{2.227869in}{3.248866in}}%
\pgfpathcurveto{\pgfqpoint{2.218661in}{3.258074in}}{\pgfqpoint{2.206170in}{3.263248in}}{\pgfqpoint{2.193147in}{3.263248in}}%
\pgfpathcurveto{\pgfqpoint{2.180124in}{3.263248in}}{\pgfqpoint{2.167633in}{3.258074in}}{\pgfqpoint{2.158425in}{3.248866in}}%
\pgfpathcurveto{\pgfqpoint{2.149216in}{3.239657in}}{\pgfqpoint{2.144042in}{3.227166in}}{\pgfqpoint{2.144042in}{3.214143in}}%
\pgfpathcurveto{\pgfqpoint{2.144042in}{3.201121in}}{\pgfqpoint{2.149216in}{3.188630in}}{\pgfqpoint{2.158425in}{3.179421in}}%
\pgfpathcurveto{\pgfqpoint{2.167633in}{3.170213in}}{\pgfqpoint{2.180124in}{3.165039in}}{\pgfqpoint{2.193147in}{3.165039in}}%
\pgfpathlineto{\pgfqpoint{2.193147in}{3.165039in}}%
\pgfpathclose%
\pgfusepath{stroke,fill}%
\end{pgfscope}%
\begin{pgfscope}%
\pgfpathrectangle{\pgfqpoint{0.786164in}{0.768110in}}{\pgfqpoint{8.851069in}{7.081890in}}%
\pgfusepath{clip}%
\pgfsetbuttcap%
\pgfsetroundjoin%
\definecolor{currentfill}{rgb}{0.271828,0.209303,0.504434}%
\pgfsetfillcolor{currentfill}%
\pgfsetfillopacity{0.700000}%
\pgfsetlinewidth{0.501875pt}%
\definecolor{currentstroke}{rgb}{1.000000,1.000000,1.000000}%
\pgfsetstrokecolor{currentstroke}%
\pgfsetstrokeopacity{0.700000}%
\pgfsetdash{}{0pt}%
\pgfpathmoveto{\pgfqpoint{2.193147in}{3.121242in}}%
\pgfpathcurveto{\pgfqpoint{2.206170in}{3.121242in}}{\pgfqpoint{2.218661in}{3.126416in}}{\pgfqpoint{2.227869in}{3.135625in}}%
\pgfpathcurveto{\pgfqpoint{2.237077in}{3.144833in}}{\pgfqpoint{2.242251in}{3.157324in}}{\pgfqpoint{2.242251in}{3.170347in}}%
\pgfpathcurveto{\pgfqpoint{2.242251in}{3.183370in}}{\pgfqpoint{2.237077in}{3.195861in}}{\pgfqpoint{2.227869in}{3.205069in}}%
\pgfpathcurveto{\pgfqpoint{2.218661in}{3.214278in}}{\pgfqpoint{2.206170in}{3.219452in}}{\pgfqpoint{2.193147in}{3.219452in}}%
\pgfpathcurveto{\pgfqpoint{2.180124in}{3.219452in}}{\pgfqpoint{2.167633in}{3.214278in}}{\pgfqpoint{2.158425in}{3.205069in}}%
\pgfpathcurveto{\pgfqpoint{2.149216in}{3.195861in}}{\pgfqpoint{2.144042in}{3.183370in}}{\pgfqpoint{2.144042in}{3.170347in}}%
\pgfpathcurveto{\pgfqpoint{2.144042in}{3.157324in}}{\pgfqpoint{2.149216in}{3.144833in}}{\pgfqpoint{2.158425in}{3.135625in}}%
\pgfpathcurveto{\pgfqpoint{2.167633in}{3.126416in}}{\pgfqpoint{2.180124in}{3.121242in}}{\pgfqpoint{2.193147in}{3.121242in}}%
\pgfpathlineto{\pgfqpoint{2.193147in}{3.121242in}}%
\pgfpathclose%
\pgfusepath{stroke,fill}%
\end{pgfscope}%
\begin{pgfscope}%
\pgfpathrectangle{\pgfqpoint{0.786164in}{0.768110in}}{\pgfqpoint{8.851069in}{7.081890in}}%
\pgfusepath{clip}%
\pgfsetbuttcap%
\pgfsetroundjoin%
\definecolor{currentfill}{rgb}{0.267968,0.223549,0.512008}%
\pgfsetfillcolor{currentfill}%
\pgfsetfillopacity{0.700000}%
\pgfsetlinewidth{0.501875pt}%
\definecolor{currentstroke}{rgb}{1.000000,1.000000,1.000000}%
\pgfsetstrokecolor{currentstroke}%
\pgfsetstrokeopacity{0.700000}%
\pgfsetdash{}{0pt}%
\pgfpathmoveto{\pgfqpoint{2.101814in}{3.077446in}}%
\pgfpathcurveto{\pgfqpoint{2.114837in}{3.077446in}}{\pgfqpoint{2.127328in}{3.082620in}}{\pgfqpoint{2.136536in}{3.091828in}}%
\pgfpathcurveto{\pgfqpoint{2.145745in}{3.101037in}}{\pgfqpoint{2.150919in}{3.113528in}}{\pgfqpoint{2.150919in}{3.126550in}}%
\pgfpathcurveto{\pgfqpoint{2.150919in}{3.139573in}}{\pgfqpoint{2.145745in}{3.152064in}}{\pgfqpoint{2.136536in}{3.161273in}}%
\pgfpathcurveto{\pgfqpoint{2.127328in}{3.170481in}}{\pgfqpoint{2.114837in}{3.175655in}}{\pgfqpoint{2.101814in}{3.175655in}}%
\pgfpathcurveto{\pgfqpoint{2.088791in}{3.175655in}}{\pgfqpoint{2.076300in}{3.170481in}}{\pgfqpoint{2.067092in}{3.161273in}}%
\pgfpathcurveto{\pgfqpoint{2.057883in}{3.152064in}}{\pgfqpoint{2.052709in}{3.139573in}}{\pgfqpoint{2.052709in}{3.126550in}}%
\pgfpathcurveto{\pgfqpoint{2.052709in}{3.113528in}}{\pgfqpoint{2.057883in}{3.101037in}}{\pgfqpoint{2.067092in}{3.091828in}}%
\pgfpathcurveto{\pgfqpoint{2.076300in}{3.082620in}}{\pgfqpoint{2.088791in}{3.077446in}}{\pgfqpoint{2.101814in}{3.077446in}}%
\pgfpathlineto{\pgfqpoint{2.101814in}{3.077446in}}%
\pgfpathclose%
\pgfusepath{stroke,fill}%
\end{pgfscope}%
\begin{pgfscope}%
\pgfpathrectangle{\pgfqpoint{0.786164in}{0.768110in}}{\pgfqpoint{8.851069in}{7.081890in}}%
\pgfusepath{clip}%
\pgfsetbuttcap%
\pgfsetroundjoin%
\definecolor{currentfill}{rgb}{0.265145,0.232956,0.516599}%
\pgfsetfillcolor{currentfill}%
\pgfsetfillopacity{0.700000}%
\pgfsetlinewidth{0.501875pt}%
\definecolor{currentstroke}{rgb}{1.000000,1.000000,1.000000}%
\pgfsetstrokecolor{currentstroke}%
\pgfsetstrokeopacity{0.700000}%
\pgfsetdash{}{0pt}%
\pgfpathmoveto{\pgfqpoint{2.120081in}{3.033649in}}%
\pgfpathcurveto{\pgfqpoint{2.133103in}{3.033649in}}{\pgfqpoint{2.145594in}{3.038823in}}{\pgfqpoint{2.154803in}{3.048032in}}%
\pgfpathcurveto{\pgfqpoint{2.164011in}{3.057240in}}{\pgfqpoint{2.169185in}{3.069731in}}{\pgfqpoint{2.169185in}{3.082754in}}%
\pgfpathcurveto{\pgfqpoint{2.169185in}{3.095777in}}{\pgfqpoint{2.164011in}{3.108268in}}{\pgfqpoint{2.154803in}{3.117476in}}%
\pgfpathcurveto{\pgfqpoint{2.145594in}{3.126685in}}{\pgfqpoint{2.133103in}{3.131859in}}{\pgfqpoint{2.120081in}{3.131859in}}%
\pgfpathcurveto{\pgfqpoint{2.107058in}{3.131859in}}{\pgfqpoint{2.094567in}{3.126685in}}{\pgfqpoint{2.085358in}{3.117476in}}%
\pgfpathcurveto{\pgfqpoint{2.076150in}{3.108268in}}{\pgfqpoint{2.070976in}{3.095777in}}{\pgfqpoint{2.070976in}{3.082754in}}%
\pgfpathcurveto{\pgfqpoint{2.070976in}{3.069731in}}{\pgfqpoint{2.076150in}{3.057240in}}{\pgfqpoint{2.085358in}{3.048032in}}%
\pgfpathcurveto{\pgfqpoint{2.094567in}{3.038823in}}{\pgfqpoint{2.107058in}{3.033649in}}{\pgfqpoint{2.120081in}{3.033649in}}%
\pgfpathlineto{\pgfqpoint{2.120081in}{3.033649in}}%
\pgfpathclose%
\pgfusepath{stroke,fill}%
\end{pgfscope}%
\begin{pgfscope}%
\pgfpathrectangle{\pgfqpoint{0.786164in}{0.768110in}}{\pgfqpoint{8.851069in}{7.081890in}}%
\pgfusepath{clip}%
\pgfsetbuttcap%
\pgfsetroundjoin%
\definecolor{currentfill}{rgb}{0.265145,0.232956,0.516599}%
\pgfsetfillcolor{currentfill}%
\pgfsetfillopacity{0.700000}%
\pgfsetlinewidth{0.501875pt}%
\definecolor{currentstroke}{rgb}{1.000000,1.000000,1.000000}%
\pgfsetstrokecolor{currentstroke}%
\pgfsetstrokeopacity{0.700000}%
\pgfsetdash{}{0pt}%
\pgfpathmoveto{\pgfqpoint{2.019614in}{3.011751in}}%
\pgfpathcurveto{\pgfqpoint{2.032637in}{3.011751in}}{\pgfqpoint{2.045128in}{3.016925in}}{\pgfqpoint{2.054337in}{3.026134in}}%
\pgfpathcurveto{\pgfqpoint{2.063545in}{3.035342in}}{\pgfqpoint{2.068719in}{3.047833in}}{\pgfqpoint{2.068719in}{3.060856in}}%
\pgfpathcurveto{\pgfqpoint{2.068719in}{3.073878in}}{\pgfqpoint{2.063545in}{3.086370in}}{\pgfqpoint{2.054337in}{3.095578in}}%
\pgfpathcurveto{\pgfqpoint{2.045128in}{3.104786in}}{\pgfqpoint{2.032637in}{3.109960in}}{\pgfqpoint{2.019614in}{3.109960in}}%
\pgfpathcurveto{\pgfqpoint{2.006592in}{3.109960in}}{\pgfqpoint{1.994101in}{3.104786in}}{\pgfqpoint{1.984892in}{3.095578in}}%
\pgfpathcurveto{\pgfqpoint{1.975684in}{3.086370in}}{\pgfqpoint{1.970510in}{3.073878in}}{\pgfqpoint{1.970510in}{3.060856in}}%
\pgfpathcurveto{\pgfqpoint{1.970510in}{3.047833in}}{\pgfqpoint{1.975684in}{3.035342in}}{\pgfqpoint{1.984892in}{3.026134in}}%
\pgfpathcurveto{\pgfqpoint{1.994101in}{3.016925in}}{\pgfqpoint{2.006592in}{3.011751in}}{\pgfqpoint{2.019614in}{3.011751in}}%
\pgfpathlineto{\pgfqpoint{2.019614in}{3.011751in}}%
\pgfpathclose%
\pgfusepath{stroke,fill}%
\end{pgfscope}%
\begin{pgfscope}%
\pgfpathrectangle{\pgfqpoint{0.786164in}{0.768110in}}{\pgfqpoint{8.851069in}{7.081890in}}%
\pgfusepath{clip}%
\pgfsetbuttcap%
\pgfsetroundjoin%
\definecolor{currentfill}{rgb}{0.263663,0.237631,0.518762}%
\pgfsetfillcolor{currentfill}%
\pgfsetfillopacity{0.700000}%
\pgfsetlinewidth{0.501875pt}%
\definecolor{currentstroke}{rgb}{1.000000,1.000000,1.000000}%
\pgfsetstrokecolor{currentstroke}%
\pgfsetstrokeopacity{0.700000}%
\pgfsetdash{}{0pt}%
\pgfpathmoveto{\pgfqpoint{2.037881in}{3.033649in}}%
\pgfpathcurveto{\pgfqpoint{2.050904in}{3.033649in}}{\pgfqpoint{2.063395in}{3.038823in}}{\pgfqpoint{2.072603in}{3.048032in}}%
\pgfpathcurveto{\pgfqpoint{2.081812in}{3.057240in}}{\pgfqpoint{2.086986in}{3.069731in}}{\pgfqpoint{2.086986in}{3.082754in}}%
\pgfpathcurveto{\pgfqpoint{2.086986in}{3.095777in}}{\pgfqpoint{2.081812in}{3.108268in}}{\pgfqpoint{2.072603in}{3.117476in}}%
\pgfpathcurveto{\pgfqpoint{2.063395in}{3.126685in}}{\pgfqpoint{2.050904in}{3.131859in}}{\pgfqpoint{2.037881in}{3.131859in}}%
\pgfpathcurveto{\pgfqpoint{2.024858in}{3.131859in}}{\pgfqpoint{2.012367in}{3.126685in}}{\pgfqpoint{2.003159in}{3.117476in}}%
\pgfpathcurveto{\pgfqpoint{1.993950in}{3.108268in}}{\pgfqpoint{1.988776in}{3.095777in}}{\pgfqpoint{1.988776in}{3.082754in}}%
\pgfpathcurveto{\pgfqpoint{1.988776in}{3.069731in}}{\pgfqpoint{1.993950in}{3.057240in}}{\pgfqpoint{2.003159in}{3.048032in}}%
\pgfpathcurveto{\pgfqpoint{2.012367in}{3.038823in}}{\pgfqpoint{2.024858in}{3.033649in}}{\pgfqpoint{2.037881in}{3.033649in}}%
\pgfpathlineto{\pgfqpoint{2.037881in}{3.033649in}}%
\pgfpathclose%
\pgfusepath{stroke,fill}%
\end{pgfscope}%
\begin{pgfscope}%
\pgfpathrectangle{\pgfqpoint{0.786164in}{0.768110in}}{\pgfqpoint{8.851069in}{7.081890in}}%
\pgfusepath{clip}%
\pgfsetbuttcap%
\pgfsetroundjoin%
\definecolor{currentfill}{rgb}{0.266580,0.228262,0.514349}%
\pgfsetfillcolor{currentfill}%
\pgfsetfillopacity{0.700000}%
\pgfsetlinewidth{0.501875pt}%
\definecolor{currentstroke}{rgb}{1.000000,1.000000,1.000000}%
\pgfsetstrokecolor{currentstroke}%
\pgfsetstrokeopacity{0.700000}%
\pgfsetdash{}{0pt}%
\pgfpathmoveto{\pgfqpoint{2.156614in}{3.077446in}}%
\pgfpathcurveto{\pgfqpoint{2.169636in}{3.077446in}}{\pgfqpoint{2.182127in}{3.082620in}}{\pgfqpoint{2.191336in}{3.091828in}}%
\pgfpathcurveto{\pgfqpoint{2.200544in}{3.101037in}}{\pgfqpoint{2.205718in}{3.113528in}}{\pgfqpoint{2.205718in}{3.126550in}}%
\pgfpathcurveto{\pgfqpoint{2.205718in}{3.139573in}}{\pgfqpoint{2.200544in}{3.152064in}}{\pgfqpoint{2.191336in}{3.161273in}}%
\pgfpathcurveto{\pgfqpoint{2.182127in}{3.170481in}}{\pgfqpoint{2.169636in}{3.175655in}}{\pgfqpoint{2.156614in}{3.175655in}}%
\pgfpathcurveto{\pgfqpoint{2.143591in}{3.175655in}}{\pgfqpoint{2.131100in}{3.170481in}}{\pgfqpoint{2.121891in}{3.161273in}}%
\pgfpathcurveto{\pgfqpoint{2.112683in}{3.152064in}}{\pgfqpoint{2.107509in}{3.139573in}}{\pgfqpoint{2.107509in}{3.126550in}}%
\pgfpathcurveto{\pgfqpoint{2.107509in}{3.113528in}}{\pgfqpoint{2.112683in}{3.101037in}}{\pgfqpoint{2.121891in}{3.091828in}}%
\pgfpathcurveto{\pgfqpoint{2.131100in}{3.082620in}}{\pgfqpoint{2.143591in}{3.077446in}}{\pgfqpoint{2.156614in}{3.077446in}}%
\pgfpathlineto{\pgfqpoint{2.156614in}{3.077446in}}%
\pgfpathclose%
\pgfusepath{stroke,fill}%
\end{pgfscope}%
\begin{pgfscope}%
\pgfpathrectangle{\pgfqpoint{0.786164in}{0.768110in}}{\pgfqpoint{8.851069in}{7.081890in}}%
\pgfusepath{clip}%
\pgfsetbuttcap%
\pgfsetroundjoin%
\definecolor{currentfill}{rgb}{0.267968,0.223549,0.512008}%
\pgfsetfillcolor{currentfill}%
\pgfsetfillopacity{0.700000}%
\pgfsetlinewidth{0.501875pt}%
\definecolor{currentstroke}{rgb}{1.000000,1.000000,1.000000}%
\pgfsetstrokecolor{currentstroke}%
\pgfsetstrokeopacity{0.700000}%
\pgfsetdash{}{0pt}%
\pgfpathmoveto{\pgfqpoint{2.257080in}{3.143141in}}%
\pgfpathcurveto{\pgfqpoint{2.270103in}{3.143141in}}{\pgfqpoint{2.282594in}{3.148314in}}{\pgfqpoint{2.291802in}{3.157523in}}%
\pgfpathcurveto{\pgfqpoint{2.301010in}{3.166731in}}{\pgfqpoint{2.306184in}{3.179222in}}{\pgfqpoint{2.306184in}{3.192245in}}%
\pgfpathcurveto{\pgfqpoint{2.306184in}{3.205268in}}{\pgfqpoint{2.301010in}{3.217759in}}{\pgfqpoint{2.291802in}{3.226967in}}%
\pgfpathcurveto{\pgfqpoint{2.282594in}{3.236176in}}{\pgfqpoint{2.270103in}{3.241350in}}{\pgfqpoint{2.257080in}{3.241350in}}%
\pgfpathcurveto{\pgfqpoint{2.244057in}{3.241350in}}{\pgfqpoint{2.231566in}{3.236176in}}{\pgfqpoint{2.222358in}{3.226967in}}%
\pgfpathcurveto{\pgfqpoint{2.213149in}{3.217759in}}{\pgfqpoint{2.207975in}{3.205268in}}{\pgfqpoint{2.207975in}{3.192245in}}%
\pgfpathcurveto{\pgfqpoint{2.207975in}{3.179222in}}{\pgfqpoint{2.213149in}{3.166731in}}{\pgfqpoint{2.222358in}{3.157523in}}%
\pgfpathcurveto{\pgfqpoint{2.231566in}{3.148314in}}{\pgfqpoint{2.244057in}{3.143141in}}{\pgfqpoint{2.257080in}{3.143141in}}%
\pgfpathlineto{\pgfqpoint{2.257080in}{3.143141in}}%
\pgfpathclose%
\pgfusepath{stroke,fill}%
\end{pgfscope}%
\begin{pgfscope}%
\pgfpathrectangle{\pgfqpoint{0.786164in}{0.768110in}}{\pgfqpoint{8.851069in}{7.081890in}}%
\pgfusepath{clip}%
\pgfsetbuttcap%
\pgfsetroundjoin%
\definecolor{currentfill}{rgb}{0.243113,0.292092,0.538516}%
\pgfsetfillcolor{currentfill}%
\pgfsetfillopacity{0.700000}%
\pgfsetlinewidth{0.501875pt}%
\definecolor{currentstroke}{rgb}{1.000000,1.000000,1.000000}%
\pgfsetstrokecolor{currentstroke}%
\pgfsetstrokeopacity{0.700000}%
\pgfsetdash{}{0pt}%
\pgfpathmoveto{\pgfqpoint{2.375813in}{3.165039in}}%
\pgfpathcurveto{\pgfqpoint{2.388835in}{3.165039in}}{\pgfqpoint{2.401326in}{3.170213in}}{\pgfqpoint{2.410535in}{3.179421in}}%
\pgfpathcurveto{\pgfqpoint{2.419743in}{3.188630in}}{\pgfqpoint{2.424917in}{3.201121in}}{\pgfqpoint{2.424917in}{3.214143in}}%
\pgfpathcurveto{\pgfqpoint{2.424917in}{3.227166in}}{\pgfqpoint{2.419743in}{3.239657in}}{\pgfqpoint{2.410535in}{3.248866in}}%
\pgfpathcurveto{\pgfqpoint{2.401326in}{3.258074in}}{\pgfqpoint{2.388835in}{3.263248in}}{\pgfqpoint{2.375813in}{3.263248in}}%
\pgfpathcurveto{\pgfqpoint{2.362790in}{3.263248in}}{\pgfqpoint{2.350299in}{3.258074in}}{\pgfqpoint{2.341090in}{3.248866in}}%
\pgfpathcurveto{\pgfqpoint{2.331882in}{3.239657in}}{\pgfqpoint{2.326708in}{3.227166in}}{\pgfqpoint{2.326708in}{3.214143in}}%
\pgfpathcurveto{\pgfqpoint{2.326708in}{3.201121in}}{\pgfqpoint{2.331882in}{3.188630in}}{\pgfqpoint{2.341090in}{3.179421in}}%
\pgfpathcurveto{\pgfqpoint{2.350299in}{3.170213in}}{\pgfqpoint{2.362790in}{3.165039in}}{\pgfqpoint{2.375813in}{3.165039in}}%
\pgfpathlineto{\pgfqpoint{2.375813in}{3.165039in}}%
\pgfpathclose%
\pgfusepath{stroke,fill}%
\end{pgfscope}%
\begin{pgfscope}%
\pgfpathrectangle{\pgfqpoint{0.786164in}{0.768110in}}{\pgfqpoint{8.851069in}{7.081890in}}%
\pgfusepath{clip}%
\pgfsetbuttcap%
\pgfsetroundjoin%
\definecolor{currentfill}{rgb}{0.239346,0.300855,0.540844}%
\pgfsetfillcolor{currentfill}%
\pgfsetfillopacity{0.700000}%
\pgfsetlinewidth{0.501875pt}%
\definecolor{currentstroke}{rgb}{1.000000,1.000000,1.000000}%
\pgfsetstrokecolor{currentstroke}%
\pgfsetstrokeopacity{0.700000}%
\pgfsetdash{}{0pt}%
\pgfpathmoveto{\pgfqpoint{2.284480in}{3.143141in}}%
\pgfpathcurveto{\pgfqpoint{2.297502in}{3.143141in}}{\pgfqpoint{2.309993in}{3.148314in}}{\pgfqpoint{2.319202in}{3.157523in}}%
\pgfpathcurveto{\pgfqpoint{2.328410in}{3.166731in}}{\pgfqpoint{2.333584in}{3.179222in}}{\pgfqpoint{2.333584in}{3.192245in}}%
\pgfpathcurveto{\pgfqpoint{2.333584in}{3.205268in}}{\pgfqpoint{2.328410in}{3.217759in}}{\pgfqpoint{2.319202in}{3.226967in}}%
\pgfpathcurveto{\pgfqpoint{2.309993in}{3.236176in}}{\pgfqpoint{2.297502in}{3.241350in}}{\pgfqpoint{2.284480in}{3.241350in}}%
\pgfpathcurveto{\pgfqpoint{2.271457in}{3.241350in}}{\pgfqpoint{2.258966in}{3.236176in}}{\pgfqpoint{2.249757in}{3.226967in}}%
\pgfpathcurveto{\pgfqpoint{2.240549in}{3.217759in}}{\pgfqpoint{2.235375in}{3.205268in}}{\pgfqpoint{2.235375in}{3.192245in}}%
\pgfpathcurveto{\pgfqpoint{2.235375in}{3.179222in}}{\pgfqpoint{2.240549in}{3.166731in}}{\pgfqpoint{2.249757in}{3.157523in}}%
\pgfpathcurveto{\pgfqpoint{2.258966in}{3.148314in}}{\pgfqpoint{2.271457in}{3.143141in}}{\pgfqpoint{2.284480in}{3.143141in}}%
\pgfpathlineto{\pgfqpoint{2.284480in}{3.143141in}}%
\pgfpathclose%
\pgfusepath{stroke,fill}%
\end{pgfscope}%
\begin{pgfscope}%
\pgfpathrectangle{\pgfqpoint{0.786164in}{0.768110in}}{\pgfqpoint{8.851069in}{7.081890in}}%
\pgfusepath{clip}%
\pgfsetbuttcap%
\pgfsetroundjoin%
\definecolor{currentfill}{rgb}{0.233603,0.313828,0.543914}%
\pgfsetfillcolor{currentfill}%
\pgfsetfillopacity{0.700000}%
\pgfsetlinewidth{0.501875pt}%
\definecolor{currentstroke}{rgb}{1.000000,1.000000,1.000000}%
\pgfsetstrokecolor{currentstroke}%
\pgfsetstrokeopacity{0.700000}%
\pgfsetdash{}{0pt}%
\pgfpathmoveto{\pgfqpoint{2.211413in}{3.033649in}}%
\pgfpathcurveto{\pgfqpoint{2.224436in}{3.033649in}}{\pgfqpoint{2.236927in}{3.038823in}}{\pgfqpoint{2.246136in}{3.048032in}}%
\pgfpathcurveto{\pgfqpoint{2.255344in}{3.057240in}}{\pgfqpoint{2.260518in}{3.069731in}}{\pgfqpoint{2.260518in}{3.082754in}}%
\pgfpathcurveto{\pgfqpoint{2.260518in}{3.095777in}}{\pgfqpoint{2.255344in}{3.108268in}}{\pgfqpoint{2.246136in}{3.117476in}}%
\pgfpathcurveto{\pgfqpoint{2.236927in}{3.126685in}}{\pgfqpoint{2.224436in}{3.131859in}}{\pgfqpoint{2.211413in}{3.131859in}}%
\pgfpathcurveto{\pgfqpoint{2.198391in}{3.131859in}}{\pgfqpoint{2.185900in}{3.126685in}}{\pgfqpoint{2.176691in}{3.117476in}}%
\pgfpathcurveto{\pgfqpoint{2.167483in}{3.108268in}}{\pgfqpoint{2.162309in}{3.095777in}}{\pgfqpoint{2.162309in}{3.082754in}}%
\pgfpathcurveto{\pgfqpoint{2.162309in}{3.069731in}}{\pgfqpoint{2.167483in}{3.057240in}}{\pgfqpoint{2.176691in}{3.048032in}}%
\pgfpathcurveto{\pgfqpoint{2.185900in}{3.038823in}}{\pgfqpoint{2.198391in}{3.033649in}}{\pgfqpoint{2.211413in}{3.033649in}}%
\pgfpathlineto{\pgfqpoint{2.211413in}{3.033649in}}%
\pgfpathclose%
\pgfusepath{stroke,fill}%
\end{pgfscope}%
\begin{pgfscope}%
\pgfpathrectangle{\pgfqpoint{0.786164in}{0.768110in}}{\pgfqpoint{8.851069in}{7.081890in}}%
\pgfusepath{clip}%
\pgfsetbuttcap%
\pgfsetroundjoin%
\definecolor{currentfill}{rgb}{0.237441,0.305202,0.541921}%
\pgfsetfillcolor{currentfill}%
\pgfsetfillopacity{0.700000}%
\pgfsetlinewidth{0.501875pt}%
\definecolor{currentstroke}{rgb}{1.000000,1.000000,1.000000}%
\pgfsetstrokecolor{currentstroke}%
\pgfsetstrokeopacity{0.700000}%
\pgfsetdash{}{0pt}%
\pgfpathmoveto{\pgfqpoint{2.439746in}{3.252632in}}%
\pgfpathcurveto{\pgfqpoint{2.452768in}{3.252632in}}{\pgfqpoint{2.465259in}{3.257806in}}{\pgfqpoint{2.474468in}{3.267014in}}%
\pgfpathcurveto{\pgfqpoint{2.483676in}{3.276223in}}{\pgfqpoint{2.488850in}{3.288714in}}{\pgfqpoint{2.488850in}{3.301736in}}%
\pgfpathcurveto{\pgfqpoint{2.488850in}{3.314759in}}{\pgfqpoint{2.483676in}{3.327250in}}{\pgfqpoint{2.474468in}{3.336459in}}%
\pgfpathcurveto{\pgfqpoint{2.465259in}{3.345667in}}{\pgfqpoint{2.452768in}{3.350841in}}{\pgfqpoint{2.439746in}{3.350841in}}%
\pgfpathcurveto{\pgfqpoint{2.426723in}{3.350841in}}{\pgfqpoint{2.414232in}{3.345667in}}{\pgfqpoint{2.405023in}{3.336459in}}%
\pgfpathcurveto{\pgfqpoint{2.395815in}{3.327250in}}{\pgfqpoint{2.390641in}{3.314759in}}{\pgfqpoint{2.390641in}{3.301736in}}%
\pgfpathcurveto{\pgfqpoint{2.390641in}{3.288714in}}{\pgfqpoint{2.395815in}{3.276223in}}{\pgfqpoint{2.405023in}{3.267014in}}%
\pgfpathcurveto{\pgfqpoint{2.414232in}{3.257806in}}{\pgfqpoint{2.426723in}{3.252632in}}{\pgfqpoint{2.439746in}{3.252632in}}%
\pgfpathlineto{\pgfqpoint{2.439746in}{3.252632in}}%
\pgfpathclose%
\pgfusepath{stroke,fill}%
\end{pgfscope}%
\begin{pgfscope}%
\pgfpathrectangle{\pgfqpoint{0.786164in}{0.768110in}}{\pgfqpoint{8.851069in}{7.081890in}}%
\pgfusepath{clip}%
\pgfsetbuttcap%
\pgfsetroundjoin%
\definecolor{currentfill}{rgb}{0.227802,0.326594,0.546532}%
\pgfsetfillcolor{currentfill}%
\pgfsetfillopacity{0.700000}%
\pgfsetlinewidth{0.501875pt}%
\definecolor{currentstroke}{rgb}{1.000000,1.000000,1.000000}%
\pgfsetstrokecolor{currentstroke}%
\pgfsetstrokeopacity{0.700000}%
\pgfsetdash{}{0pt}%
\pgfpathmoveto{\pgfqpoint{2.311880in}{3.077446in}}%
\pgfpathcurveto{\pgfqpoint{2.324902in}{3.077446in}}{\pgfqpoint{2.337393in}{3.082620in}}{\pgfqpoint{2.346602in}{3.091828in}}%
\pgfpathcurveto{\pgfqpoint{2.355810in}{3.101037in}}{\pgfqpoint{2.360984in}{3.113528in}}{\pgfqpoint{2.360984in}{3.126550in}}%
\pgfpathcurveto{\pgfqpoint{2.360984in}{3.139573in}}{\pgfqpoint{2.355810in}{3.152064in}}{\pgfqpoint{2.346602in}{3.161273in}}%
\pgfpathcurveto{\pgfqpoint{2.337393in}{3.170481in}}{\pgfqpoint{2.324902in}{3.175655in}}{\pgfqpoint{2.311880in}{3.175655in}}%
\pgfpathcurveto{\pgfqpoint{2.298857in}{3.175655in}}{\pgfqpoint{2.286366in}{3.170481in}}{\pgfqpoint{2.277157in}{3.161273in}}%
\pgfpathcurveto{\pgfqpoint{2.267949in}{3.152064in}}{\pgfqpoint{2.262775in}{3.139573in}}{\pgfqpoint{2.262775in}{3.126550in}}%
\pgfpathcurveto{\pgfqpoint{2.262775in}{3.113528in}}{\pgfqpoint{2.267949in}{3.101037in}}{\pgfqpoint{2.277157in}{3.091828in}}%
\pgfpathcurveto{\pgfqpoint{2.286366in}{3.082620in}}{\pgfqpoint{2.298857in}{3.077446in}}{\pgfqpoint{2.311880in}{3.077446in}}%
\pgfpathlineto{\pgfqpoint{2.311880in}{3.077446in}}%
\pgfpathclose%
\pgfusepath{stroke,fill}%
\end{pgfscope}%
\begin{pgfscope}%
\pgfpathrectangle{\pgfqpoint{0.786164in}{0.768110in}}{\pgfqpoint{8.851069in}{7.081890in}}%
\pgfusepath{clip}%
\pgfsetbuttcap%
\pgfsetroundjoin%
\definecolor{currentfill}{rgb}{0.210503,0.363727,0.552206}%
\pgfsetfillcolor{currentfill}%
\pgfsetfillopacity{0.700000}%
\pgfsetlinewidth{0.501875pt}%
\definecolor{currentstroke}{rgb}{1.000000,1.000000,1.000000}%
\pgfsetstrokecolor{currentstroke}%
\pgfsetstrokeopacity{0.700000}%
\pgfsetdash{}{0pt}%
\pgfpathmoveto{\pgfqpoint{1.983081in}{2.727074in}}%
\pgfpathcurveto{\pgfqpoint{1.996104in}{2.727074in}}{\pgfqpoint{2.008595in}{2.732248in}}{\pgfqpoint{2.017803in}{2.741456in}}%
\pgfpathcurveto{\pgfqpoint{2.027012in}{2.750665in}}{\pgfqpoint{2.032186in}{2.763156in}}{\pgfqpoint{2.032186in}{2.776179in}}%
\pgfpathcurveto{\pgfqpoint{2.032186in}{2.789201in}}{\pgfqpoint{2.027012in}{2.801692in}}{\pgfqpoint{2.017803in}{2.810901in}}%
\pgfpathcurveto{\pgfqpoint{2.008595in}{2.820109in}}{\pgfqpoint{1.996104in}{2.825283in}}{\pgfqpoint{1.983081in}{2.825283in}}%
\pgfpathcurveto{\pgfqpoint{1.970059in}{2.825283in}}{\pgfqpoint{1.957567in}{2.820109in}}{\pgfqpoint{1.948359in}{2.810901in}}%
\pgfpathcurveto{\pgfqpoint{1.939151in}{2.801692in}}{\pgfqpoint{1.933977in}{2.789201in}}{\pgfqpoint{1.933977in}{2.776179in}}%
\pgfpathcurveto{\pgfqpoint{1.933977in}{2.763156in}}{\pgfqpoint{1.939151in}{2.750665in}}{\pgfqpoint{1.948359in}{2.741456in}}%
\pgfpathcurveto{\pgfqpoint{1.957567in}{2.732248in}}{\pgfqpoint{1.970059in}{2.727074in}}{\pgfqpoint{1.983081in}{2.727074in}}%
\pgfpathlineto{\pgfqpoint{1.983081in}{2.727074in}}%
\pgfpathclose%
\pgfusepath{stroke,fill}%
\end{pgfscope}%
\begin{pgfscope}%
\pgfpathrectangle{\pgfqpoint{0.786164in}{0.768110in}}{\pgfqpoint{8.851069in}{7.081890in}}%
\pgfusepath{clip}%
\pgfsetbuttcap%
\pgfsetroundjoin%
\definecolor{currentfill}{rgb}{0.208623,0.367752,0.552675}%
\pgfsetfillcolor{currentfill}%
\pgfsetfillopacity{0.700000}%
\pgfsetlinewidth{0.501875pt}%
\definecolor{currentstroke}{rgb}{1.000000,1.000000,1.000000}%
\pgfsetstrokecolor{currentstroke}%
\pgfsetstrokeopacity{0.700000}%
\pgfsetdash{}{0pt}%
\pgfpathmoveto{\pgfqpoint{2.037881in}{2.924158in}}%
\pgfpathcurveto{\pgfqpoint{2.050904in}{2.924158in}}{\pgfqpoint{2.063395in}{2.929332in}}{\pgfqpoint{2.072603in}{2.938541in}}%
\pgfpathcurveto{\pgfqpoint{2.081812in}{2.947749in}}{\pgfqpoint{2.086986in}{2.960240in}}{\pgfqpoint{2.086986in}{2.973263in}}%
\pgfpathcurveto{\pgfqpoint{2.086986in}{2.986286in}}{\pgfqpoint{2.081812in}{2.998777in}}{\pgfqpoint{2.072603in}{3.007985in}}%
\pgfpathcurveto{\pgfqpoint{2.063395in}{3.017193in}}{\pgfqpoint{2.050904in}{3.022367in}}{\pgfqpoint{2.037881in}{3.022367in}}%
\pgfpathcurveto{\pgfqpoint{2.024858in}{3.022367in}}{\pgfqpoint{2.012367in}{3.017193in}}{\pgfqpoint{2.003159in}{3.007985in}}%
\pgfpathcurveto{\pgfqpoint{1.993950in}{2.998777in}}{\pgfqpoint{1.988776in}{2.986286in}}{\pgfqpoint{1.988776in}{2.973263in}}%
\pgfpathcurveto{\pgfqpoint{1.988776in}{2.960240in}}{\pgfqpoint{1.993950in}{2.947749in}}{\pgfqpoint{2.003159in}{2.938541in}}%
\pgfpathcurveto{\pgfqpoint{2.012367in}{2.929332in}}{\pgfqpoint{2.024858in}{2.924158in}}{\pgfqpoint{2.037881in}{2.924158in}}%
\pgfpathlineto{\pgfqpoint{2.037881in}{2.924158in}}%
\pgfpathclose%
\pgfusepath{stroke,fill}%
\end{pgfscope}%
\begin{pgfscope}%
\pgfpathrectangle{\pgfqpoint{0.786164in}{0.768110in}}{\pgfqpoint{8.851069in}{7.081890in}}%
\pgfusepath{clip}%
\pgfsetbuttcap%
\pgfsetroundjoin%
\definecolor{currentfill}{rgb}{0.199430,0.387607,0.554642}%
\pgfsetfillcolor{currentfill}%
\pgfsetfillopacity{0.700000}%
\pgfsetlinewidth{0.501875pt}%
\definecolor{currentstroke}{rgb}{1.000000,1.000000,1.000000}%
\pgfsetstrokecolor{currentstroke}%
\pgfsetstrokeopacity{0.700000}%
\pgfsetdash{}{0pt}%
\pgfpathmoveto{\pgfqpoint{1.955681in}{2.727074in}}%
\pgfpathcurveto{\pgfqpoint{1.968704in}{2.727074in}}{\pgfqpoint{1.981195in}{2.732248in}}{\pgfqpoint{1.990404in}{2.741456in}}%
\pgfpathcurveto{\pgfqpoint{1.999612in}{2.750665in}}{\pgfqpoint{2.004786in}{2.763156in}}{\pgfqpoint{2.004786in}{2.776179in}}%
\pgfpathcurveto{\pgfqpoint{2.004786in}{2.789201in}}{\pgfqpoint{1.999612in}{2.801692in}}{\pgfqpoint{1.990404in}{2.810901in}}%
\pgfpathcurveto{\pgfqpoint{1.981195in}{2.820109in}}{\pgfqpoint{1.968704in}{2.825283in}}{\pgfqpoint{1.955681in}{2.825283in}}%
\pgfpathcurveto{\pgfqpoint{1.942659in}{2.825283in}}{\pgfqpoint{1.930168in}{2.820109in}}{\pgfqpoint{1.920959in}{2.810901in}}%
\pgfpathcurveto{\pgfqpoint{1.911751in}{2.801692in}}{\pgfqpoint{1.906577in}{2.789201in}}{\pgfqpoint{1.906577in}{2.776179in}}%
\pgfpathcurveto{\pgfqpoint{1.906577in}{2.763156in}}{\pgfqpoint{1.911751in}{2.750665in}}{\pgfqpoint{1.920959in}{2.741456in}}%
\pgfpathcurveto{\pgfqpoint{1.930168in}{2.732248in}}{\pgfqpoint{1.942659in}{2.727074in}}{\pgfqpoint{1.955681in}{2.727074in}}%
\pgfpathlineto{\pgfqpoint{1.955681in}{2.727074in}}%
\pgfpathclose%
\pgfusepath{stroke,fill}%
\end{pgfscope}%
\begin{pgfscope}%
\pgfpathrectangle{\pgfqpoint{0.786164in}{0.768110in}}{\pgfqpoint{8.851069in}{7.081890in}}%
\pgfusepath{clip}%
\pgfsetbuttcap%
\pgfsetroundjoin%
\definecolor{currentfill}{rgb}{0.192357,0.403199,0.555836}%
\pgfsetfillcolor{currentfill}%
\pgfsetfillopacity{0.700000}%
\pgfsetlinewidth{0.501875pt}%
\definecolor{currentstroke}{rgb}{1.000000,1.000000,1.000000}%
\pgfsetstrokecolor{currentstroke}%
\pgfsetstrokeopacity{0.700000}%
\pgfsetdash{}{0pt}%
\pgfpathmoveto{\pgfqpoint{1.937415in}{2.639481in}}%
\pgfpathcurveto{\pgfqpoint{1.950437in}{2.639481in}}{\pgfqpoint{1.962929in}{2.644655in}}{\pgfqpoint{1.972137in}{2.653864in}}%
\pgfpathcurveto{\pgfqpoint{1.981345in}{2.663072in}}{\pgfqpoint{1.986519in}{2.675563in}}{\pgfqpoint{1.986519in}{2.688586in}}%
\pgfpathcurveto{\pgfqpoint{1.986519in}{2.701608in}}{\pgfqpoint{1.981345in}{2.714100in}}{\pgfqpoint{1.972137in}{2.723308in}}%
\pgfpathcurveto{\pgfqpoint{1.962929in}{2.732516in}}{\pgfqpoint{1.950437in}{2.737690in}}{\pgfqpoint{1.937415in}{2.737690in}}%
\pgfpathcurveto{\pgfqpoint{1.924392in}{2.737690in}}{\pgfqpoint{1.911901in}{2.732516in}}{\pgfqpoint{1.902693in}{2.723308in}}%
\pgfpathcurveto{\pgfqpoint{1.893484in}{2.714100in}}{\pgfqpoint{1.888310in}{2.701608in}}{\pgfqpoint{1.888310in}{2.688586in}}%
\pgfpathcurveto{\pgfqpoint{1.888310in}{2.675563in}}{\pgfqpoint{1.893484in}{2.663072in}}{\pgfqpoint{1.902693in}{2.653864in}}%
\pgfpathcurveto{\pgfqpoint{1.911901in}{2.644655in}}{\pgfqpoint{1.924392in}{2.639481in}}{\pgfqpoint{1.937415in}{2.639481in}}%
\pgfpathlineto{\pgfqpoint{1.937415in}{2.639481in}}%
\pgfpathclose%
\pgfusepath{stroke,fill}%
\end{pgfscope}%
\begin{pgfscope}%
\pgfpathrectangle{\pgfqpoint{0.786164in}{0.768110in}}{\pgfqpoint{8.851069in}{7.081890in}}%
\pgfusepath{clip}%
\pgfsetbuttcap%
\pgfsetroundjoin%
\definecolor{currentfill}{rgb}{0.185556,0.418570,0.556753}%
\pgfsetfillcolor{currentfill}%
\pgfsetfillopacity{0.700000}%
\pgfsetlinewidth{0.501875pt}%
\definecolor{currentstroke}{rgb}{1.000000,1.000000,1.000000}%
\pgfsetstrokecolor{currentstroke}%
\pgfsetstrokeopacity{0.700000}%
\pgfsetdash{}{0pt}%
\pgfpathmoveto{\pgfqpoint{1.919148in}{2.464295in}}%
\pgfpathcurveto{\pgfqpoint{1.932171in}{2.464295in}}{\pgfqpoint{1.944662in}{2.469469in}}{\pgfqpoint{1.953870in}{2.478678in}}%
\pgfpathcurveto{\pgfqpoint{1.963079in}{2.487886in}}{\pgfqpoint{1.968253in}{2.500377in}}{\pgfqpoint{1.968253in}{2.513400in}}%
\pgfpathcurveto{\pgfqpoint{1.968253in}{2.526423in}}{\pgfqpoint{1.963079in}{2.538914in}}{\pgfqpoint{1.953870in}{2.548122in}}%
\pgfpathcurveto{\pgfqpoint{1.944662in}{2.557331in}}{\pgfqpoint{1.932171in}{2.562504in}}{\pgfqpoint{1.919148in}{2.562504in}}%
\pgfpathcurveto{\pgfqpoint{1.906125in}{2.562504in}}{\pgfqpoint{1.893634in}{2.557331in}}{\pgfqpoint{1.884426in}{2.548122in}}%
\pgfpathcurveto{\pgfqpoint{1.875218in}{2.538914in}}{\pgfqpoint{1.870044in}{2.526423in}}{\pgfqpoint{1.870044in}{2.513400in}}%
\pgfpathcurveto{\pgfqpoint{1.870044in}{2.500377in}}{\pgfqpoint{1.875218in}{2.487886in}}{\pgfqpoint{1.884426in}{2.478678in}}%
\pgfpathcurveto{\pgfqpoint{1.893634in}{2.469469in}}{\pgfqpoint{1.906125in}{2.464295in}}{\pgfqpoint{1.919148in}{2.464295in}}%
\pgfpathlineto{\pgfqpoint{1.919148in}{2.464295in}}%
\pgfpathclose%
\pgfusepath{stroke,fill}%
\end{pgfscope}%
\begin{pgfscope}%
\pgfpathrectangle{\pgfqpoint{0.786164in}{0.768110in}}{\pgfqpoint{8.851069in}{7.081890in}}%
\pgfusepath{clip}%
\pgfsetbuttcap%
\pgfsetroundjoin%
\definecolor{currentfill}{rgb}{0.175841,0.441290,0.557685}%
\pgfsetfillcolor{currentfill}%
\pgfsetfillopacity{0.700000}%
\pgfsetlinewidth{0.501875pt}%
\definecolor{currentstroke}{rgb}{1.000000,1.000000,1.000000}%
\pgfsetstrokecolor{currentstroke}%
\pgfsetstrokeopacity{0.700000}%
\pgfsetdash{}{0pt}%
\pgfpathmoveto{\pgfqpoint{1.919148in}{2.376702in}}%
\pgfpathcurveto{\pgfqpoint{1.932171in}{2.376702in}}{\pgfqpoint{1.944662in}{2.381876in}}{\pgfqpoint{1.953870in}{2.391085in}}%
\pgfpathcurveto{\pgfqpoint{1.963079in}{2.400293in}}{\pgfqpoint{1.968253in}{2.412784in}}{\pgfqpoint{1.968253in}{2.425807in}}%
\pgfpathcurveto{\pgfqpoint{1.968253in}{2.438830in}}{\pgfqpoint{1.963079in}{2.451321in}}{\pgfqpoint{1.953870in}{2.460529in}}%
\pgfpathcurveto{\pgfqpoint{1.944662in}{2.469738in}}{\pgfqpoint{1.932171in}{2.474912in}}{\pgfqpoint{1.919148in}{2.474912in}}%
\pgfpathcurveto{\pgfqpoint{1.906125in}{2.474912in}}{\pgfqpoint{1.893634in}{2.469738in}}{\pgfqpoint{1.884426in}{2.460529in}}%
\pgfpathcurveto{\pgfqpoint{1.875218in}{2.451321in}}{\pgfqpoint{1.870044in}{2.438830in}}{\pgfqpoint{1.870044in}{2.425807in}}%
\pgfpathcurveto{\pgfqpoint{1.870044in}{2.412784in}}{\pgfqpoint{1.875218in}{2.400293in}}{\pgfqpoint{1.884426in}{2.391085in}}%
\pgfpathcurveto{\pgfqpoint{1.893634in}{2.381876in}}{\pgfqpoint{1.906125in}{2.376702in}}{\pgfqpoint{1.919148in}{2.376702in}}%
\pgfpathlineto{\pgfqpoint{1.919148in}{2.376702in}}%
\pgfpathclose%
\pgfusepath{stroke,fill}%
\end{pgfscope}%
\begin{pgfscope}%
\pgfpathrectangle{\pgfqpoint{0.786164in}{0.768110in}}{\pgfqpoint{8.851069in}{7.081890in}}%
\pgfusepath{clip}%
\pgfsetbuttcap%
\pgfsetroundjoin%
\definecolor{currentfill}{rgb}{0.177423,0.437527,0.557565}%
\pgfsetfillcolor{currentfill}%
\pgfsetfillopacity{0.700000}%
\pgfsetlinewidth{0.501875pt}%
\definecolor{currentstroke}{rgb}{1.000000,1.000000,1.000000}%
\pgfsetstrokecolor{currentstroke}%
\pgfsetstrokeopacity{0.700000}%
\pgfsetdash{}{0pt}%
\pgfpathmoveto{\pgfqpoint{1.827815in}{2.376702in}}%
\pgfpathcurveto{\pgfqpoint{1.840838in}{2.376702in}}{\pgfqpoint{1.853329in}{2.381876in}}{\pgfqpoint{1.862538in}{2.391085in}}%
\pgfpathcurveto{\pgfqpoint{1.871746in}{2.400293in}}{\pgfqpoint{1.876920in}{2.412784in}}{\pgfqpoint{1.876920in}{2.425807in}}%
\pgfpathcurveto{\pgfqpoint{1.876920in}{2.438830in}}{\pgfqpoint{1.871746in}{2.451321in}}{\pgfqpoint{1.862538in}{2.460529in}}%
\pgfpathcurveto{\pgfqpoint{1.853329in}{2.469738in}}{\pgfqpoint{1.840838in}{2.474912in}}{\pgfqpoint{1.827815in}{2.474912in}}%
\pgfpathcurveto{\pgfqpoint{1.814793in}{2.474912in}}{\pgfqpoint{1.802302in}{2.469738in}}{\pgfqpoint{1.793093in}{2.460529in}}%
\pgfpathcurveto{\pgfqpoint{1.783885in}{2.451321in}}{\pgfqpoint{1.778711in}{2.438830in}}{\pgfqpoint{1.778711in}{2.425807in}}%
\pgfpathcurveto{\pgfqpoint{1.778711in}{2.412784in}}{\pgfqpoint{1.783885in}{2.400293in}}{\pgfqpoint{1.793093in}{2.391085in}}%
\pgfpathcurveto{\pgfqpoint{1.802302in}{2.381876in}}{\pgfqpoint{1.814793in}{2.376702in}}{\pgfqpoint{1.827815in}{2.376702in}}%
\pgfpathlineto{\pgfqpoint{1.827815in}{2.376702in}}%
\pgfpathclose%
\pgfusepath{stroke,fill}%
\end{pgfscope}%
\begin{pgfscope}%
\pgfpathrectangle{\pgfqpoint{0.786164in}{0.768110in}}{\pgfqpoint{8.851069in}{7.081890in}}%
\pgfusepath{clip}%
\pgfsetbuttcap%
\pgfsetroundjoin%
\definecolor{currentfill}{rgb}{0.174274,0.445044,0.557792}%
\pgfsetfillcolor{currentfill}%
\pgfsetfillopacity{0.700000}%
\pgfsetlinewidth{0.501875pt}%
\definecolor{currentstroke}{rgb}{1.000000,1.000000,1.000000}%
\pgfsetstrokecolor{currentstroke}%
\pgfsetstrokeopacity{0.700000}%
\pgfsetdash{}{0pt}%
\pgfpathmoveto{\pgfqpoint{1.773016in}{2.420499in}}%
\pgfpathcurveto{\pgfqpoint{1.786038in}{2.420499in}}{\pgfqpoint{1.798529in}{2.425673in}}{\pgfqpoint{1.807738in}{2.434881in}}%
\pgfpathcurveto{\pgfqpoint{1.816946in}{2.444090in}}{\pgfqpoint{1.822120in}{2.456581in}}{\pgfqpoint{1.822120in}{2.469603in}}%
\pgfpathcurveto{\pgfqpoint{1.822120in}{2.482626in}}{\pgfqpoint{1.816946in}{2.495117in}}{\pgfqpoint{1.807738in}{2.504326in}}%
\pgfpathcurveto{\pgfqpoint{1.798529in}{2.513534in}}{\pgfqpoint{1.786038in}{2.518708in}}{\pgfqpoint{1.773016in}{2.518708in}}%
\pgfpathcurveto{\pgfqpoint{1.759993in}{2.518708in}}{\pgfqpoint{1.747502in}{2.513534in}}{\pgfqpoint{1.738293in}{2.504326in}}%
\pgfpathcurveto{\pgfqpoint{1.729085in}{2.495117in}}{\pgfqpoint{1.723911in}{2.482626in}}{\pgfqpoint{1.723911in}{2.469603in}}%
\pgfpathcurveto{\pgfqpoint{1.723911in}{2.456581in}}{\pgfqpoint{1.729085in}{2.444090in}}{\pgfqpoint{1.738293in}{2.434881in}}%
\pgfpathcurveto{\pgfqpoint{1.747502in}{2.425673in}}{\pgfqpoint{1.759993in}{2.420499in}}{\pgfqpoint{1.773016in}{2.420499in}}%
\pgfpathlineto{\pgfqpoint{1.773016in}{2.420499in}}%
\pgfpathclose%
\pgfusepath{stroke,fill}%
\end{pgfscope}%
\begin{pgfscope}%
\pgfpathrectangle{\pgfqpoint{0.786164in}{0.768110in}}{\pgfqpoint{8.851069in}{7.081890in}}%
\pgfusepath{clip}%
\pgfsetbuttcap%
\pgfsetroundjoin%
\definecolor{currentfill}{rgb}{0.174274,0.445044,0.557792}%
\pgfsetfillcolor{currentfill}%
\pgfsetfillopacity{0.700000}%
\pgfsetlinewidth{0.501875pt}%
\definecolor{currentstroke}{rgb}{1.000000,1.000000,1.000000}%
\pgfsetstrokecolor{currentstroke}%
\pgfsetstrokeopacity{0.700000}%
\pgfsetdash{}{0pt}%
\pgfpathmoveto{\pgfqpoint{1.699949in}{2.420499in}}%
\pgfpathcurveto{\pgfqpoint{1.712972in}{2.420499in}}{\pgfqpoint{1.725463in}{2.425673in}}{\pgfqpoint{1.734672in}{2.434881in}}%
\pgfpathcurveto{\pgfqpoint{1.743880in}{2.444090in}}{\pgfqpoint{1.749054in}{2.456581in}}{\pgfqpoint{1.749054in}{2.469603in}}%
\pgfpathcurveto{\pgfqpoint{1.749054in}{2.482626in}}{\pgfqpoint{1.743880in}{2.495117in}}{\pgfqpoint{1.734672in}{2.504326in}}%
\pgfpathcurveto{\pgfqpoint{1.725463in}{2.513534in}}{\pgfqpoint{1.712972in}{2.518708in}}{\pgfqpoint{1.699949in}{2.518708in}}%
\pgfpathcurveto{\pgfqpoint{1.686927in}{2.518708in}}{\pgfqpoint{1.674436in}{2.513534in}}{\pgfqpoint{1.665227in}{2.504326in}}%
\pgfpathcurveto{\pgfqpoint{1.656019in}{2.495117in}}{\pgfqpoint{1.650845in}{2.482626in}}{\pgfqpoint{1.650845in}{2.469603in}}%
\pgfpathcurveto{\pgfqpoint{1.650845in}{2.456581in}}{\pgfqpoint{1.656019in}{2.444090in}}{\pgfqpoint{1.665227in}{2.434881in}}%
\pgfpathcurveto{\pgfqpoint{1.674436in}{2.425673in}}{\pgfqpoint{1.686927in}{2.420499in}}{\pgfqpoint{1.699949in}{2.420499in}}%
\pgfpathlineto{\pgfqpoint{1.699949in}{2.420499in}}%
\pgfpathclose%
\pgfusepath{stroke,fill}%
\end{pgfscope}%
\begin{pgfscope}%
\pgfpathrectangle{\pgfqpoint{0.786164in}{0.768110in}}{\pgfqpoint{8.851069in}{7.081890in}}%
\pgfusepath{clip}%
\pgfsetbuttcap%
\pgfsetroundjoin%
\definecolor{currentfill}{rgb}{0.168126,0.459988,0.558082}%
\pgfsetfillcolor{currentfill}%
\pgfsetfillopacity{0.700000}%
\pgfsetlinewidth{0.501875pt}%
\definecolor{currentstroke}{rgb}{1.000000,1.000000,1.000000}%
\pgfsetstrokecolor{currentstroke}%
\pgfsetstrokeopacity{0.700000}%
\pgfsetdash{}{0pt}%
\pgfpathmoveto{\pgfqpoint{1.709083in}{2.420499in}}%
\pgfpathcurveto{\pgfqpoint{1.722105in}{2.420499in}}{\pgfqpoint{1.734596in}{2.425673in}}{\pgfqpoint{1.743805in}{2.434881in}}%
\pgfpathcurveto{\pgfqpoint{1.753013in}{2.444090in}}{\pgfqpoint{1.758187in}{2.456581in}}{\pgfqpoint{1.758187in}{2.469603in}}%
\pgfpathcurveto{\pgfqpoint{1.758187in}{2.482626in}}{\pgfqpoint{1.753013in}{2.495117in}}{\pgfqpoint{1.743805in}{2.504326in}}%
\pgfpathcurveto{\pgfqpoint{1.734596in}{2.513534in}}{\pgfqpoint{1.722105in}{2.518708in}}{\pgfqpoint{1.709083in}{2.518708in}}%
\pgfpathcurveto{\pgfqpoint{1.696060in}{2.518708in}}{\pgfqpoint{1.683569in}{2.513534in}}{\pgfqpoint{1.674360in}{2.504326in}}%
\pgfpathcurveto{\pgfqpoint{1.665152in}{2.495117in}}{\pgfqpoint{1.659978in}{2.482626in}}{\pgfqpoint{1.659978in}{2.469603in}}%
\pgfpathcurveto{\pgfqpoint{1.659978in}{2.456581in}}{\pgfqpoint{1.665152in}{2.444090in}}{\pgfqpoint{1.674360in}{2.434881in}}%
\pgfpathcurveto{\pgfqpoint{1.683569in}{2.425673in}}{\pgfqpoint{1.696060in}{2.420499in}}{\pgfqpoint{1.709083in}{2.420499in}}%
\pgfpathlineto{\pgfqpoint{1.709083in}{2.420499in}}%
\pgfpathclose%
\pgfusepath{stroke,fill}%
\end{pgfscope}%
\begin{pgfscope}%
\pgfpathrectangle{\pgfqpoint{0.786164in}{0.768110in}}{\pgfqpoint{8.851069in}{7.081890in}}%
\pgfusepath{clip}%
\pgfsetbuttcap%
\pgfsetroundjoin%
\definecolor{currentfill}{rgb}{0.146180,0.515413,0.556823}%
\pgfsetfillcolor{currentfill}%
\pgfsetfillopacity{0.700000}%
\pgfsetlinewidth{0.501875pt}%
\definecolor{currentstroke}{rgb}{1.000000,1.000000,1.000000}%
\pgfsetstrokecolor{currentstroke}%
\pgfsetstrokeopacity{0.700000}%
\pgfsetdash{}{0pt}%
\pgfpathmoveto{\pgfqpoint{1.690816in}{2.311008in}}%
\pgfpathcurveto{\pgfqpoint{1.703839in}{2.311008in}}{\pgfqpoint{1.716330in}{2.316182in}}{\pgfqpoint{1.725538in}{2.325390in}}%
\pgfpathcurveto{\pgfqpoint{1.734747in}{2.334598in}}{\pgfqpoint{1.739921in}{2.347089in}}{\pgfqpoint{1.739921in}{2.360112in}}%
\pgfpathcurveto{\pgfqpoint{1.739921in}{2.373135in}}{\pgfqpoint{1.734747in}{2.385626in}}{\pgfqpoint{1.725538in}{2.394834in}}%
\pgfpathcurveto{\pgfqpoint{1.716330in}{2.404043in}}{\pgfqpoint{1.703839in}{2.409217in}}{\pgfqpoint{1.690816in}{2.409217in}}%
\pgfpathcurveto{\pgfqpoint{1.677793in}{2.409217in}}{\pgfqpoint{1.665302in}{2.404043in}}{\pgfqpoint{1.656094in}{2.394834in}}%
\pgfpathcurveto{\pgfqpoint{1.646885in}{2.385626in}}{\pgfqpoint{1.641711in}{2.373135in}}{\pgfqpoint{1.641711in}{2.360112in}}%
\pgfpathcurveto{\pgfqpoint{1.641711in}{2.347089in}}{\pgfqpoint{1.646885in}{2.334598in}}{\pgfqpoint{1.656094in}{2.325390in}}%
\pgfpathcurveto{\pgfqpoint{1.665302in}{2.316182in}}{\pgfqpoint{1.677793in}{2.311008in}}{\pgfqpoint{1.690816in}{2.311008in}}%
\pgfpathlineto{\pgfqpoint{1.690816in}{2.311008in}}%
\pgfpathclose%
\pgfusepath{stroke,fill}%
\end{pgfscope}%
\begin{pgfscope}%
\pgfpathrectangle{\pgfqpoint{0.786164in}{0.768110in}}{\pgfqpoint{8.851069in}{7.081890in}}%
\pgfusepath{clip}%
\pgfsetbuttcap%
\pgfsetroundjoin%
\definecolor{currentfill}{rgb}{0.140536,0.530132,0.555659}%
\pgfsetfillcolor{currentfill}%
\pgfsetfillopacity{0.700000}%
\pgfsetlinewidth{0.501875pt}%
\definecolor{currentstroke}{rgb}{1.000000,1.000000,1.000000}%
\pgfsetstrokecolor{currentstroke}%
\pgfsetstrokeopacity{0.700000}%
\pgfsetdash{}{0pt}%
\pgfpathmoveto{\pgfqpoint{1.782149in}{2.442397in}}%
\pgfpathcurveto{\pgfqpoint{1.795172in}{2.442397in}}{\pgfqpoint{1.807663in}{2.447571in}}{\pgfqpoint{1.816871in}{2.456779in}}%
\pgfpathcurveto{\pgfqpoint{1.826080in}{2.465988in}}{\pgfqpoint{1.831254in}{2.478479in}}{\pgfqpoint{1.831254in}{2.491502in}}%
\pgfpathcurveto{\pgfqpoint{1.831254in}{2.504524in}}{\pgfqpoint{1.826080in}{2.517015in}}{\pgfqpoint{1.816871in}{2.526224in}}%
\pgfpathcurveto{\pgfqpoint{1.807663in}{2.535432in}}{\pgfqpoint{1.795172in}{2.540606in}}{\pgfqpoint{1.782149in}{2.540606in}}%
\pgfpathcurveto{\pgfqpoint{1.769126in}{2.540606in}}{\pgfqpoint{1.756635in}{2.535432in}}{\pgfqpoint{1.747427in}{2.526224in}}%
\pgfpathcurveto{\pgfqpoint{1.738218in}{2.517015in}}{\pgfqpoint{1.733044in}{2.504524in}}{\pgfqpoint{1.733044in}{2.491502in}}%
\pgfpathcurveto{\pgfqpoint{1.733044in}{2.478479in}}{\pgfqpoint{1.738218in}{2.465988in}}{\pgfqpoint{1.747427in}{2.456779in}}%
\pgfpathcurveto{\pgfqpoint{1.756635in}{2.447571in}}{\pgfqpoint{1.769126in}{2.442397in}}{\pgfqpoint{1.782149in}{2.442397in}}%
\pgfpathlineto{\pgfqpoint{1.782149in}{2.442397in}}%
\pgfpathclose%
\pgfusepath{stroke,fill}%
\end{pgfscope}%
\begin{pgfscope}%
\pgfpathrectangle{\pgfqpoint{0.786164in}{0.768110in}}{\pgfqpoint{8.851069in}{7.081890in}}%
\pgfusepath{clip}%
\pgfsetbuttcap%
\pgfsetroundjoin%
\definecolor{currentfill}{rgb}{0.139147,0.533812,0.555298}%
\pgfsetfillcolor{currentfill}%
\pgfsetfillopacity{0.700000}%
\pgfsetlinewidth{0.501875pt}%
\definecolor{currentstroke}{rgb}{1.000000,1.000000,1.000000}%
\pgfsetstrokecolor{currentstroke}%
\pgfsetstrokeopacity{0.700000}%
\pgfsetdash{}{0pt}%
\pgfpathmoveto{\pgfqpoint{1.800415in}{2.508092in}}%
\pgfpathcurveto{\pgfqpoint{1.813438in}{2.508092in}}{\pgfqpoint{1.825929in}{2.513266in}}{\pgfqpoint{1.835138in}{2.522474in}}%
\pgfpathcurveto{\pgfqpoint{1.844346in}{2.531683in}}{\pgfqpoint{1.849520in}{2.544174in}}{\pgfqpoint{1.849520in}{2.557196in}}%
\pgfpathcurveto{\pgfqpoint{1.849520in}{2.570219in}}{\pgfqpoint{1.844346in}{2.582710in}}{\pgfqpoint{1.835138in}{2.591919in}}%
\pgfpathcurveto{\pgfqpoint{1.825929in}{2.601127in}}{\pgfqpoint{1.813438in}{2.606301in}}{\pgfqpoint{1.800415in}{2.606301in}}%
\pgfpathcurveto{\pgfqpoint{1.787393in}{2.606301in}}{\pgfqpoint{1.774902in}{2.601127in}}{\pgfqpoint{1.765693in}{2.591919in}}%
\pgfpathcurveto{\pgfqpoint{1.756485in}{2.582710in}}{\pgfqpoint{1.751311in}{2.570219in}}{\pgfqpoint{1.751311in}{2.557196in}}%
\pgfpathcurveto{\pgfqpoint{1.751311in}{2.544174in}}{\pgfqpoint{1.756485in}{2.531683in}}{\pgfqpoint{1.765693in}{2.522474in}}%
\pgfpathcurveto{\pgfqpoint{1.774902in}{2.513266in}}{\pgfqpoint{1.787393in}{2.508092in}}{\pgfqpoint{1.800415in}{2.508092in}}%
\pgfpathlineto{\pgfqpoint{1.800415in}{2.508092in}}%
\pgfpathclose%
\pgfusepath{stroke,fill}%
\end{pgfscope}%
\begin{pgfscope}%
\pgfpathrectangle{\pgfqpoint{0.786164in}{0.768110in}}{\pgfqpoint{8.851069in}{7.081890in}}%
\pgfusepath{clip}%
\pgfsetbuttcap%
\pgfsetroundjoin%
\definecolor{currentfill}{rgb}{0.140536,0.530132,0.555659}%
\pgfsetfillcolor{currentfill}%
\pgfsetfillopacity{0.700000}%
\pgfsetlinewidth{0.501875pt}%
\definecolor{currentstroke}{rgb}{1.000000,1.000000,1.000000}%
\pgfsetstrokecolor{currentstroke}%
\pgfsetstrokeopacity{0.700000}%
\pgfsetdash{}{0pt}%
\pgfpathmoveto{\pgfqpoint{1.891748in}{3.077446in}}%
\pgfpathcurveto{\pgfqpoint{1.904771in}{3.077446in}}{\pgfqpoint{1.917262in}{3.082620in}}{\pgfqpoint{1.926471in}{3.091828in}}%
\pgfpathcurveto{\pgfqpoint{1.935679in}{3.101037in}}{\pgfqpoint{1.940853in}{3.113528in}}{\pgfqpoint{1.940853in}{3.126550in}}%
\pgfpathcurveto{\pgfqpoint{1.940853in}{3.139573in}}{\pgfqpoint{1.935679in}{3.152064in}}{\pgfqpoint{1.926471in}{3.161273in}}%
\pgfpathcurveto{\pgfqpoint{1.917262in}{3.170481in}}{\pgfqpoint{1.904771in}{3.175655in}}{\pgfqpoint{1.891748in}{3.175655in}}%
\pgfpathcurveto{\pgfqpoint{1.878726in}{3.175655in}}{\pgfqpoint{1.866235in}{3.170481in}}{\pgfqpoint{1.857026in}{3.161273in}}%
\pgfpathcurveto{\pgfqpoint{1.847818in}{3.152064in}}{\pgfqpoint{1.842644in}{3.139573in}}{\pgfqpoint{1.842644in}{3.126550in}}%
\pgfpathcurveto{\pgfqpoint{1.842644in}{3.113528in}}{\pgfqpoint{1.847818in}{3.101037in}}{\pgfqpoint{1.857026in}{3.091828in}}%
\pgfpathcurveto{\pgfqpoint{1.866235in}{3.082620in}}{\pgfqpoint{1.878726in}{3.077446in}}{\pgfqpoint{1.891748in}{3.077446in}}%
\pgfpathlineto{\pgfqpoint{1.891748in}{3.077446in}}%
\pgfpathclose%
\pgfusepath{stroke,fill}%
\end{pgfscope}%
\begin{pgfscope}%
\pgfpathrectangle{\pgfqpoint{0.786164in}{0.768110in}}{\pgfqpoint{8.851069in}{7.081890in}}%
\pgfusepath{clip}%
\pgfsetbuttcap%
\pgfsetroundjoin%
\definecolor{currentfill}{rgb}{0.143343,0.522773,0.556295}%
\pgfsetfillcolor{currentfill}%
\pgfsetfillopacity{0.700000}%
\pgfsetlinewidth{0.501875pt}%
\definecolor{currentstroke}{rgb}{1.000000,1.000000,1.000000}%
\pgfsetstrokecolor{currentstroke}%
\pgfsetstrokeopacity{0.700000}%
\pgfsetdash{}{0pt}%
\pgfpathmoveto{\pgfqpoint{1.882615in}{2.464295in}}%
\pgfpathcurveto{\pgfqpoint{1.895638in}{2.464295in}}{\pgfqpoint{1.908129in}{2.469469in}}{\pgfqpoint{1.917337in}{2.478678in}}%
\pgfpathcurveto{\pgfqpoint{1.926546in}{2.487886in}}{\pgfqpoint{1.931720in}{2.500377in}}{\pgfqpoint{1.931720in}{2.513400in}}%
\pgfpathcurveto{\pgfqpoint{1.931720in}{2.526423in}}{\pgfqpoint{1.926546in}{2.538914in}}{\pgfqpoint{1.917337in}{2.548122in}}%
\pgfpathcurveto{\pgfqpoint{1.908129in}{2.557331in}}{\pgfqpoint{1.895638in}{2.562504in}}{\pgfqpoint{1.882615in}{2.562504in}}%
\pgfpathcurveto{\pgfqpoint{1.869592in}{2.562504in}}{\pgfqpoint{1.857101in}{2.557331in}}{\pgfqpoint{1.847893in}{2.548122in}}%
\pgfpathcurveto{\pgfqpoint{1.838684in}{2.538914in}}{\pgfqpoint{1.833510in}{2.526423in}}{\pgfqpoint{1.833510in}{2.513400in}}%
\pgfpathcurveto{\pgfqpoint{1.833510in}{2.500377in}}{\pgfqpoint{1.838684in}{2.487886in}}{\pgfqpoint{1.847893in}{2.478678in}}%
\pgfpathcurveto{\pgfqpoint{1.857101in}{2.469469in}}{\pgfqpoint{1.869592in}{2.464295in}}{\pgfqpoint{1.882615in}{2.464295in}}%
\pgfpathlineto{\pgfqpoint{1.882615in}{2.464295in}}%
\pgfpathclose%
\pgfusepath{stroke,fill}%
\end{pgfscope}%
\begin{pgfscope}%
\pgfpathrectangle{\pgfqpoint{0.786164in}{0.768110in}}{\pgfqpoint{8.851069in}{7.081890in}}%
\pgfusepath{clip}%
\pgfsetbuttcap%
\pgfsetroundjoin%
\definecolor{currentfill}{rgb}{0.140536,0.530132,0.555659}%
\pgfsetfillcolor{currentfill}%
\pgfsetfillopacity{0.700000}%
\pgfsetlinewidth{0.501875pt}%
\definecolor{currentstroke}{rgb}{1.000000,1.000000,1.000000}%
\pgfsetstrokecolor{currentstroke}%
\pgfsetstrokeopacity{0.700000}%
\pgfsetdash{}{0pt}%
\pgfpathmoveto{\pgfqpoint{1.836949in}{2.376702in}}%
\pgfpathcurveto{\pgfqpoint{1.849971in}{2.376702in}}{\pgfqpoint{1.862462in}{2.381876in}}{\pgfqpoint{1.871671in}{2.391085in}}%
\pgfpathcurveto{\pgfqpoint{1.880879in}{2.400293in}}{\pgfqpoint{1.886053in}{2.412784in}}{\pgfqpoint{1.886053in}{2.425807in}}%
\pgfpathcurveto{\pgfqpoint{1.886053in}{2.438830in}}{\pgfqpoint{1.880879in}{2.451321in}}{\pgfqpoint{1.871671in}{2.460529in}}%
\pgfpathcurveto{\pgfqpoint{1.862462in}{2.469738in}}{\pgfqpoint{1.849971in}{2.474912in}}{\pgfqpoint{1.836949in}{2.474912in}}%
\pgfpathcurveto{\pgfqpoint{1.823926in}{2.474912in}}{\pgfqpoint{1.811435in}{2.469738in}}{\pgfqpoint{1.802226in}{2.460529in}}%
\pgfpathcurveto{\pgfqpoint{1.793018in}{2.451321in}}{\pgfqpoint{1.787844in}{2.438830in}}{\pgfqpoint{1.787844in}{2.425807in}}%
\pgfpathcurveto{\pgfqpoint{1.787844in}{2.412784in}}{\pgfqpoint{1.793018in}{2.400293in}}{\pgfqpoint{1.802226in}{2.391085in}}%
\pgfpathcurveto{\pgfqpoint{1.811435in}{2.381876in}}{\pgfqpoint{1.823926in}{2.376702in}}{\pgfqpoint{1.836949in}{2.376702in}}%
\pgfpathlineto{\pgfqpoint{1.836949in}{2.376702in}}%
\pgfpathclose%
\pgfusepath{stroke,fill}%
\end{pgfscope}%
\begin{pgfscope}%
\pgfpathrectangle{\pgfqpoint{0.786164in}{0.768110in}}{\pgfqpoint{8.851069in}{7.081890in}}%
\pgfusepath{clip}%
\pgfsetbuttcap%
\pgfsetroundjoin%
\definecolor{currentfill}{rgb}{0.124395,0.578002,0.548287}%
\pgfsetfillcolor{currentfill}%
\pgfsetfillopacity{0.700000}%
\pgfsetlinewidth{0.501875pt}%
\definecolor{currentstroke}{rgb}{1.000000,1.000000,1.000000}%
\pgfsetstrokecolor{currentstroke}%
\pgfsetstrokeopacity{0.700000}%
\pgfsetdash{}{0pt}%
\pgfpathmoveto{\pgfqpoint{1.599483in}{2.179618in}}%
\pgfpathcurveto{\pgfqpoint{1.612506in}{2.179618in}}{\pgfqpoint{1.624997in}{2.184792in}}{\pgfqpoint{1.634205in}{2.194001in}}%
\pgfpathcurveto{\pgfqpoint{1.643414in}{2.203209in}}{\pgfqpoint{1.648588in}{2.215700in}}{\pgfqpoint{1.648588in}{2.228723in}}%
\pgfpathcurveto{\pgfqpoint{1.648588in}{2.241745in}}{\pgfqpoint{1.643414in}{2.254237in}}{\pgfqpoint{1.634205in}{2.263445in}}%
\pgfpathcurveto{\pgfqpoint{1.624997in}{2.272653in}}{\pgfqpoint{1.612506in}{2.277827in}}{\pgfqpoint{1.599483in}{2.277827in}}%
\pgfpathcurveto{\pgfqpoint{1.586460in}{2.277827in}}{\pgfqpoint{1.573969in}{2.272653in}}{\pgfqpoint{1.564761in}{2.263445in}}%
\pgfpathcurveto{\pgfqpoint{1.555552in}{2.254237in}}{\pgfqpoint{1.550379in}{2.241745in}}{\pgfqpoint{1.550379in}{2.228723in}}%
\pgfpathcurveto{\pgfqpoint{1.550379in}{2.215700in}}{\pgfqpoint{1.555552in}{2.203209in}}{\pgfqpoint{1.564761in}{2.194001in}}%
\pgfpathcurveto{\pgfqpoint{1.573969in}{2.184792in}}{\pgfqpoint{1.586460in}{2.179618in}}{\pgfqpoint{1.599483in}{2.179618in}}%
\pgfpathlineto{\pgfqpoint{1.599483in}{2.179618in}}%
\pgfpathclose%
\pgfusepath{stroke,fill}%
\end{pgfscope}%
\begin{pgfscope}%
\pgfpathrectangle{\pgfqpoint{0.786164in}{0.768110in}}{\pgfqpoint{8.851069in}{7.081890in}}%
\pgfusepath{clip}%
\pgfsetbuttcap%
\pgfsetroundjoin%
\definecolor{currentfill}{rgb}{0.124395,0.578002,0.548287}%
\pgfsetfillcolor{currentfill}%
\pgfsetfillopacity{0.700000}%
\pgfsetlinewidth{0.501875pt}%
\definecolor{currentstroke}{rgb}{1.000000,1.000000,1.000000}%
\pgfsetstrokecolor{currentstroke}%
\pgfsetstrokeopacity{0.700000}%
\pgfsetdash{}{0pt}%
\pgfpathmoveto{\pgfqpoint{1.590350in}{2.113923in}}%
\pgfpathcurveto{\pgfqpoint{1.603373in}{2.113923in}}{\pgfqpoint{1.615864in}{2.119097in}}{\pgfqpoint{1.625072in}{2.128306in}}%
\pgfpathcurveto{\pgfqpoint{1.634281in}{2.137514in}}{\pgfqpoint{1.639454in}{2.150005in}}{\pgfqpoint{1.639454in}{2.163028in}}%
\pgfpathcurveto{\pgfqpoint{1.639454in}{2.176051in}}{\pgfqpoint{1.634281in}{2.188542in}}{\pgfqpoint{1.625072in}{2.197750in}}%
\pgfpathcurveto{\pgfqpoint{1.615864in}{2.206959in}}{\pgfqpoint{1.603373in}{2.212133in}}{\pgfqpoint{1.590350in}{2.212133in}}%
\pgfpathcurveto{\pgfqpoint{1.577327in}{2.212133in}}{\pgfqpoint{1.564836in}{2.206959in}}{\pgfqpoint{1.555628in}{2.197750in}}%
\pgfpathcurveto{\pgfqpoint{1.546419in}{2.188542in}}{\pgfqpoint{1.541245in}{2.176051in}}{\pgfqpoint{1.541245in}{2.163028in}}%
\pgfpathcurveto{\pgfqpoint{1.541245in}{2.150005in}}{\pgfqpoint{1.546419in}{2.137514in}}{\pgfqpoint{1.555628in}{2.128306in}}%
\pgfpathcurveto{\pgfqpoint{1.564836in}{2.119097in}}{\pgfqpoint{1.577327in}{2.113923in}}{\pgfqpoint{1.590350in}{2.113923in}}%
\pgfpathlineto{\pgfqpoint{1.590350in}{2.113923in}}%
\pgfpathclose%
\pgfusepath{stroke,fill}%
\end{pgfscope}%
\begin{pgfscope}%
\pgfpathrectangle{\pgfqpoint{0.786164in}{0.768110in}}{\pgfqpoint{8.851069in}{7.081890in}}%
\pgfusepath{clip}%
\pgfsetbuttcap%
\pgfsetroundjoin%
\definecolor{currentfill}{rgb}{0.121831,0.589055,0.545623}%
\pgfsetfillcolor{currentfill}%
\pgfsetfillopacity{0.700000}%
\pgfsetlinewidth{0.501875pt}%
\definecolor{currentstroke}{rgb}{1.000000,1.000000,1.000000}%
\pgfsetstrokecolor{currentstroke}%
\pgfsetstrokeopacity{0.700000}%
\pgfsetdash{}{0pt}%
\pgfpathmoveto{\pgfqpoint{1.699949in}{2.179618in}}%
\pgfpathcurveto{\pgfqpoint{1.712972in}{2.179618in}}{\pgfqpoint{1.725463in}{2.184792in}}{\pgfqpoint{1.734672in}{2.194001in}}%
\pgfpathcurveto{\pgfqpoint{1.743880in}{2.203209in}}{\pgfqpoint{1.749054in}{2.215700in}}{\pgfqpoint{1.749054in}{2.228723in}}%
\pgfpathcurveto{\pgfqpoint{1.749054in}{2.241745in}}{\pgfqpoint{1.743880in}{2.254237in}}{\pgfqpoint{1.734672in}{2.263445in}}%
\pgfpathcurveto{\pgfqpoint{1.725463in}{2.272653in}}{\pgfqpoint{1.712972in}{2.277827in}}{\pgfqpoint{1.699949in}{2.277827in}}%
\pgfpathcurveto{\pgfqpoint{1.686927in}{2.277827in}}{\pgfqpoint{1.674436in}{2.272653in}}{\pgfqpoint{1.665227in}{2.263445in}}%
\pgfpathcurveto{\pgfqpoint{1.656019in}{2.254237in}}{\pgfqpoint{1.650845in}{2.241745in}}{\pgfqpoint{1.650845in}{2.228723in}}%
\pgfpathcurveto{\pgfqpoint{1.650845in}{2.215700in}}{\pgfqpoint{1.656019in}{2.203209in}}{\pgfqpoint{1.665227in}{2.194001in}}%
\pgfpathcurveto{\pgfqpoint{1.674436in}{2.184792in}}{\pgfqpoint{1.686927in}{2.179618in}}{\pgfqpoint{1.699949in}{2.179618in}}%
\pgfpathlineto{\pgfqpoint{1.699949in}{2.179618in}}%
\pgfpathclose%
\pgfusepath{stroke,fill}%
\end{pgfscope}%
\begin{pgfscope}%
\pgfpathrectangle{\pgfqpoint{0.786164in}{0.768110in}}{\pgfqpoint{8.851069in}{7.081890in}}%
\pgfusepath{clip}%
\pgfsetbuttcap%
\pgfsetroundjoin%
\definecolor{currentfill}{rgb}{0.229739,0.322361,0.545706}%
\pgfsetfillcolor{currentfill}%
\pgfsetfillopacity{0.700000}%
\pgfsetlinewidth{0.501875pt}%
\definecolor{currentstroke}{rgb}{1.000000,1.000000,1.000000}%
\pgfsetstrokecolor{currentstroke}%
\pgfsetstrokeopacity{0.700000}%
\pgfsetdash{}{0pt}%
\pgfpathmoveto{\pgfqpoint{1.462484in}{2.267211in}}%
\pgfpathcurveto{\pgfqpoint{1.475507in}{2.267211in}}{\pgfqpoint{1.487998in}{2.272385in}}{\pgfqpoint{1.497206in}{2.281593in}}%
\pgfpathcurveto{\pgfqpoint{1.506414in}{2.290802in}}{\pgfqpoint{1.511588in}{2.303293in}}{\pgfqpoint{1.511588in}{2.316316in}}%
\pgfpathcurveto{\pgfqpoint{1.511588in}{2.329338in}}{\pgfqpoint{1.506414in}{2.341829in}}{\pgfqpoint{1.497206in}{2.351038in}}%
\pgfpathcurveto{\pgfqpoint{1.487998in}{2.360246in}}{\pgfqpoint{1.475507in}{2.365420in}}{\pgfqpoint{1.462484in}{2.365420in}}%
\pgfpathcurveto{\pgfqpoint{1.449461in}{2.365420in}}{\pgfqpoint{1.436970in}{2.360246in}}{\pgfqpoint{1.427762in}{2.351038in}}%
\pgfpathcurveto{\pgfqpoint{1.418553in}{2.341829in}}{\pgfqpoint{1.413379in}{2.329338in}}{\pgfqpoint{1.413379in}{2.316316in}}%
\pgfpathcurveto{\pgfqpoint{1.413379in}{2.303293in}}{\pgfqpoint{1.418553in}{2.290802in}}{\pgfqpoint{1.427762in}{2.281593in}}%
\pgfpathcurveto{\pgfqpoint{1.436970in}{2.272385in}}{\pgfqpoint{1.449461in}{2.267211in}}{\pgfqpoint{1.462484in}{2.267211in}}%
\pgfpathlineto{\pgfqpoint{1.462484in}{2.267211in}}%
\pgfpathclose%
\pgfusepath{stroke,fill}%
\end{pgfscope}%
\begin{pgfscope}%
\pgfpathrectangle{\pgfqpoint{0.786164in}{0.768110in}}{\pgfqpoint{8.851069in}{7.081890in}}%
\pgfusepath{clip}%
\pgfsetbuttcap%
\pgfsetroundjoin%
\definecolor{currentfill}{rgb}{0.223925,0.334994,0.548053}%
\pgfsetfillcolor{currentfill}%
\pgfsetfillopacity{0.700000}%
\pgfsetlinewidth{0.501875pt}%
\definecolor{currentstroke}{rgb}{1.000000,1.000000,1.000000}%
\pgfsetstrokecolor{currentstroke}%
\pgfsetstrokeopacity{0.700000}%
\pgfsetdash{}{0pt}%
\pgfpathmoveto{\pgfqpoint{1.416817in}{2.245313in}}%
\pgfpathcurveto{\pgfqpoint{1.429840in}{2.245313in}}{\pgfqpoint{1.442331in}{2.250487in}}{\pgfqpoint{1.451540in}{2.259695in}}%
\pgfpathcurveto{\pgfqpoint{1.460748in}{2.268904in}}{\pgfqpoint{1.465922in}{2.281395in}}{\pgfqpoint{1.465922in}{2.294417in}}%
\pgfpathcurveto{\pgfqpoint{1.465922in}{2.307440in}}{\pgfqpoint{1.460748in}{2.319931in}}{\pgfqpoint{1.451540in}{2.329140in}}%
\pgfpathcurveto{\pgfqpoint{1.442331in}{2.338348in}}{\pgfqpoint{1.429840in}{2.343522in}}{\pgfqpoint{1.416817in}{2.343522in}}%
\pgfpathcurveto{\pgfqpoint{1.403795in}{2.343522in}}{\pgfqpoint{1.391304in}{2.338348in}}{\pgfqpoint{1.382095in}{2.329140in}}%
\pgfpathcurveto{\pgfqpoint{1.372887in}{2.319931in}}{\pgfqpoint{1.367713in}{2.307440in}}{\pgfqpoint{1.367713in}{2.294417in}}%
\pgfpathcurveto{\pgfqpoint{1.367713in}{2.281395in}}{\pgfqpoint{1.372887in}{2.268904in}}{\pgfqpoint{1.382095in}{2.259695in}}%
\pgfpathcurveto{\pgfqpoint{1.391304in}{2.250487in}}{\pgfqpoint{1.403795in}{2.245313in}}{\pgfqpoint{1.416817in}{2.245313in}}%
\pgfpathlineto{\pgfqpoint{1.416817in}{2.245313in}}%
\pgfpathclose%
\pgfusepath{stroke,fill}%
\end{pgfscope}%
\begin{pgfscope}%
\pgfpathrectangle{\pgfqpoint{0.786164in}{0.768110in}}{\pgfqpoint{8.851069in}{7.081890in}}%
\pgfusepath{clip}%
\pgfsetbuttcap%
\pgfsetroundjoin%
\definecolor{currentfill}{rgb}{0.225863,0.330805,0.547314}%
\pgfsetfillcolor{currentfill}%
\pgfsetfillopacity{0.700000}%
\pgfsetlinewidth{0.501875pt}%
\definecolor{currentstroke}{rgb}{1.000000,1.000000,1.000000}%
\pgfsetstrokecolor{currentstroke}%
\pgfsetstrokeopacity{0.700000}%
\pgfsetdash{}{0pt}%
\pgfpathmoveto{\pgfqpoint{1.489884in}{2.289109in}}%
\pgfpathcurveto{\pgfqpoint{1.502906in}{2.289109in}}{\pgfqpoint{1.515397in}{2.294283in}}{\pgfqpoint{1.524606in}{2.303492in}}%
\pgfpathcurveto{\pgfqpoint{1.533814in}{2.312700in}}{\pgfqpoint{1.538988in}{2.325191in}}{\pgfqpoint{1.538988in}{2.338214in}}%
\pgfpathcurveto{\pgfqpoint{1.538988in}{2.351237in}}{\pgfqpoint{1.533814in}{2.363728in}}{\pgfqpoint{1.524606in}{2.372936in}}%
\pgfpathcurveto{\pgfqpoint{1.515397in}{2.382145in}}{\pgfqpoint{1.502906in}{2.387319in}}{\pgfqpoint{1.489884in}{2.387319in}}%
\pgfpathcurveto{\pgfqpoint{1.476861in}{2.387319in}}{\pgfqpoint{1.464370in}{2.382145in}}{\pgfqpoint{1.455161in}{2.372936in}}%
\pgfpathcurveto{\pgfqpoint{1.445953in}{2.363728in}}{\pgfqpoint{1.440779in}{2.351237in}}{\pgfqpoint{1.440779in}{2.338214in}}%
\pgfpathcurveto{\pgfqpoint{1.440779in}{2.325191in}}{\pgfqpoint{1.445953in}{2.312700in}}{\pgfqpoint{1.455161in}{2.303492in}}%
\pgfpathcurveto{\pgfqpoint{1.464370in}{2.294283in}}{\pgfqpoint{1.476861in}{2.289109in}}{\pgfqpoint{1.489884in}{2.289109in}}%
\pgfpathlineto{\pgfqpoint{1.489884in}{2.289109in}}%
\pgfpathclose%
\pgfusepath{stroke,fill}%
\end{pgfscope}%
\begin{pgfscope}%
\pgfpathrectangle{\pgfqpoint{0.786164in}{0.768110in}}{\pgfqpoint{8.851069in}{7.081890in}}%
\pgfusepath{clip}%
\pgfsetbuttcap%
\pgfsetroundjoin%
\definecolor{currentfill}{rgb}{0.218130,0.347432,0.550038}%
\pgfsetfillcolor{currentfill}%
\pgfsetfillopacity{0.700000}%
\pgfsetlinewidth{0.501875pt}%
\definecolor{currentstroke}{rgb}{1.000000,1.000000,1.000000}%
\pgfsetstrokecolor{currentstroke}%
\pgfsetstrokeopacity{0.700000}%
\pgfsetdash{}{0pt}%
\pgfpathmoveto{\pgfqpoint{1.508150in}{2.311008in}}%
\pgfpathcurveto{\pgfqpoint{1.521173in}{2.311008in}}{\pgfqpoint{1.533664in}{2.316182in}}{\pgfqpoint{1.542872in}{2.325390in}}%
\pgfpathcurveto{\pgfqpoint{1.552081in}{2.334598in}}{\pgfqpoint{1.557255in}{2.347089in}}{\pgfqpoint{1.557255in}{2.360112in}}%
\pgfpathcurveto{\pgfqpoint{1.557255in}{2.373135in}}{\pgfqpoint{1.552081in}{2.385626in}}{\pgfqpoint{1.542872in}{2.394834in}}%
\pgfpathcurveto{\pgfqpoint{1.533664in}{2.404043in}}{\pgfqpoint{1.521173in}{2.409217in}}{\pgfqpoint{1.508150in}{2.409217in}}%
\pgfpathcurveto{\pgfqpoint{1.495128in}{2.409217in}}{\pgfqpoint{1.482636in}{2.404043in}}{\pgfqpoint{1.473428in}{2.394834in}}%
\pgfpathcurveto{\pgfqpoint{1.464220in}{2.385626in}}{\pgfqpoint{1.459046in}{2.373135in}}{\pgfqpoint{1.459046in}{2.360112in}}%
\pgfpathcurveto{\pgfqpoint{1.459046in}{2.347089in}}{\pgfqpoint{1.464220in}{2.334598in}}{\pgfqpoint{1.473428in}{2.325390in}}%
\pgfpathcurveto{\pgfqpoint{1.482636in}{2.316182in}}{\pgfqpoint{1.495128in}{2.311008in}}{\pgfqpoint{1.508150in}{2.311008in}}%
\pgfpathlineto{\pgfqpoint{1.508150in}{2.311008in}}%
\pgfpathclose%
\pgfusepath{stroke,fill}%
\end{pgfscope}%
\begin{pgfscope}%
\pgfpathrectangle{\pgfqpoint{0.786164in}{0.768110in}}{\pgfqpoint{8.851069in}{7.081890in}}%
\pgfusepath{clip}%
\pgfsetbuttcap%
\pgfsetroundjoin%
\definecolor{currentfill}{rgb}{0.203063,0.379716,0.553925}%
\pgfsetfillcolor{currentfill}%
\pgfsetfillopacity{0.700000}%
\pgfsetlinewidth{0.501875pt}%
\definecolor{currentstroke}{rgb}{1.000000,1.000000,1.000000}%
\pgfsetstrokecolor{currentstroke}%
\pgfsetstrokeopacity{0.700000}%
\pgfsetdash{}{0pt}%
\pgfpathmoveto{\pgfqpoint{1.535550in}{2.289109in}}%
\pgfpathcurveto{\pgfqpoint{1.548573in}{2.289109in}}{\pgfqpoint{1.561064in}{2.294283in}}{\pgfqpoint{1.570272in}{2.303492in}}%
\pgfpathcurveto{\pgfqpoint{1.579481in}{2.312700in}}{\pgfqpoint{1.584655in}{2.325191in}}{\pgfqpoint{1.584655in}{2.338214in}}%
\pgfpathcurveto{\pgfqpoint{1.584655in}{2.351237in}}{\pgfqpoint{1.579481in}{2.363728in}}{\pgfqpoint{1.570272in}{2.372936in}}%
\pgfpathcurveto{\pgfqpoint{1.561064in}{2.382145in}}{\pgfqpoint{1.548573in}{2.387319in}}{\pgfqpoint{1.535550in}{2.387319in}}%
\pgfpathcurveto{\pgfqpoint{1.522527in}{2.387319in}}{\pgfqpoint{1.510036in}{2.382145in}}{\pgfqpoint{1.500828in}{2.372936in}}%
\pgfpathcurveto{\pgfqpoint{1.491619in}{2.363728in}}{\pgfqpoint{1.486445in}{2.351237in}}{\pgfqpoint{1.486445in}{2.338214in}}%
\pgfpathcurveto{\pgfqpoint{1.486445in}{2.325191in}}{\pgfqpoint{1.491619in}{2.312700in}}{\pgfqpoint{1.500828in}{2.303492in}}%
\pgfpathcurveto{\pgfqpoint{1.510036in}{2.294283in}}{\pgfqpoint{1.522527in}{2.289109in}}{\pgfqpoint{1.535550in}{2.289109in}}%
\pgfpathlineto{\pgfqpoint{1.535550in}{2.289109in}}%
\pgfpathclose%
\pgfusepath{stroke,fill}%
\end{pgfscope}%
\begin{pgfscope}%
\pgfpathrectangle{\pgfqpoint{0.786164in}{0.768110in}}{\pgfqpoint{8.851069in}{7.081890in}}%
\pgfusepath{clip}%
\pgfsetbuttcap%
\pgfsetroundjoin%
\definecolor{currentfill}{rgb}{0.190631,0.407061,0.556089}%
\pgfsetfillcolor{currentfill}%
\pgfsetfillopacity{0.700000}%
\pgfsetlinewidth{0.501875pt}%
\definecolor{currentstroke}{rgb}{1.000000,1.000000,1.000000}%
\pgfsetstrokecolor{currentstroke}%
\pgfsetstrokeopacity{0.700000}%
\pgfsetdash{}{0pt}%
\pgfpathmoveto{\pgfqpoint{1.334618in}{2.113923in}}%
\pgfpathcurveto{\pgfqpoint{1.347641in}{2.113923in}}{\pgfqpoint{1.360132in}{2.119097in}}{\pgfqpoint{1.369340in}{2.128306in}}%
\pgfpathcurveto{\pgfqpoint{1.378548in}{2.137514in}}{\pgfqpoint{1.383722in}{2.150005in}}{\pgfqpoint{1.383722in}{2.163028in}}%
\pgfpathcurveto{\pgfqpoint{1.383722in}{2.176051in}}{\pgfqpoint{1.378548in}{2.188542in}}{\pgfqpoint{1.369340in}{2.197750in}}%
\pgfpathcurveto{\pgfqpoint{1.360132in}{2.206959in}}{\pgfqpoint{1.347641in}{2.212133in}}{\pgfqpoint{1.334618in}{2.212133in}}%
\pgfpathcurveto{\pgfqpoint{1.321595in}{2.212133in}}{\pgfqpoint{1.309104in}{2.206959in}}{\pgfqpoint{1.299896in}{2.197750in}}%
\pgfpathcurveto{\pgfqpoint{1.290687in}{2.188542in}}{\pgfqpoint{1.285513in}{2.176051in}}{\pgfqpoint{1.285513in}{2.163028in}}%
\pgfpathcurveto{\pgfqpoint{1.285513in}{2.150005in}}{\pgfqpoint{1.290687in}{2.137514in}}{\pgfqpoint{1.299896in}{2.128306in}}%
\pgfpathcurveto{\pgfqpoint{1.309104in}{2.119097in}}{\pgfqpoint{1.321595in}{2.113923in}}{\pgfqpoint{1.334618in}{2.113923in}}%
\pgfpathlineto{\pgfqpoint{1.334618in}{2.113923in}}%
\pgfpathclose%
\pgfusepath{stroke,fill}%
\end{pgfscope}%
\begin{pgfscope}%
\pgfpathrectangle{\pgfqpoint{0.786164in}{0.768110in}}{\pgfqpoint{8.851069in}{7.081890in}}%
\pgfusepath{clip}%
\pgfsetbuttcap%
\pgfsetroundjoin%
\definecolor{currentfill}{rgb}{0.185556,0.418570,0.556753}%
\pgfsetfillcolor{currentfill}%
\pgfsetfillopacity{0.700000}%
\pgfsetlinewidth{0.501875pt}%
\definecolor{currentstroke}{rgb}{1.000000,1.000000,1.000000}%
\pgfsetstrokecolor{currentstroke}%
\pgfsetstrokeopacity{0.700000}%
\pgfsetdash{}{0pt}%
\pgfpathmoveto{\pgfqpoint{1.362018in}{2.004432in}}%
\pgfpathcurveto{\pgfqpoint{1.375040in}{2.004432in}}{\pgfqpoint{1.387531in}{2.009606in}}{\pgfqpoint{1.396740in}{2.018815in}}%
\pgfpathcurveto{\pgfqpoint{1.405948in}{2.028023in}}{\pgfqpoint{1.411122in}{2.040514in}}{\pgfqpoint{1.411122in}{2.053537in}}%
\pgfpathcurveto{\pgfqpoint{1.411122in}{2.066560in}}{\pgfqpoint{1.405948in}{2.079051in}}{\pgfqpoint{1.396740in}{2.088259in}}%
\pgfpathcurveto{\pgfqpoint{1.387531in}{2.097468in}}{\pgfqpoint{1.375040in}{2.102642in}}{\pgfqpoint{1.362018in}{2.102642in}}%
\pgfpathcurveto{\pgfqpoint{1.348995in}{2.102642in}}{\pgfqpoint{1.336504in}{2.097468in}}{\pgfqpoint{1.327295in}{2.088259in}}%
\pgfpathcurveto{\pgfqpoint{1.318087in}{2.079051in}}{\pgfqpoint{1.312913in}{2.066560in}}{\pgfqpoint{1.312913in}{2.053537in}}%
\pgfpathcurveto{\pgfqpoint{1.312913in}{2.040514in}}{\pgfqpoint{1.318087in}{2.028023in}}{\pgfqpoint{1.327295in}{2.018815in}}%
\pgfpathcurveto{\pgfqpoint{1.336504in}{2.009606in}}{\pgfqpoint{1.348995in}{2.004432in}}{\pgfqpoint{1.362018in}{2.004432in}}%
\pgfpathlineto{\pgfqpoint{1.362018in}{2.004432in}}%
\pgfpathclose%
\pgfusepath{stroke,fill}%
\end{pgfscope}%
\begin{pgfscope}%
\pgfpathrectangle{\pgfqpoint{0.786164in}{0.768110in}}{\pgfqpoint{8.851069in}{7.081890in}}%
\pgfusepath{clip}%
\pgfsetbuttcap%
\pgfsetroundjoin%
\definecolor{currentfill}{rgb}{0.192357,0.403199,0.555836}%
\pgfsetfillcolor{currentfill}%
\pgfsetfillopacity{0.700000}%
\pgfsetlinewidth{0.501875pt}%
\definecolor{currentstroke}{rgb}{1.000000,1.000000,1.000000}%
\pgfsetstrokecolor{currentstroke}%
\pgfsetstrokeopacity{0.700000}%
\pgfsetdash{}{0pt}%
\pgfpathmoveto{\pgfqpoint{1.389418in}{2.070127in}}%
\pgfpathcurveto{\pgfqpoint{1.402440in}{2.070127in}}{\pgfqpoint{1.414931in}{2.075301in}}{\pgfqpoint{1.424140in}{2.084509in}}%
\pgfpathcurveto{\pgfqpoint{1.433348in}{2.093718in}}{\pgfqpoint{1.438522in}{2.106209in}}{\pgfqpoint{1.438522in}{2.119232in}}%
\pgfpathcurveto{\pgfqpoint{1.438522in}{2.132254in}}{\pgfqpoint{1.433348in}{2.144745in}}{\pgfqpoint{1.424140in}{2.153954in}}%
\pgfpathcurveto{\pgfqpoint{1.414931in}{2.163162in}}{\pgfqpoint{1.402440in}{2.168336in}}{\pgfqpoint{1.389418in}{2.168336in}}%
\pgfpathcurveto{\pgfqpoint{1.376395in}{2.168336in}}{\pgfqpoint{1.363904in}{2.163162in}}{\pgfqpoint{1.354695in}{2.153954in}}%
\pgfpathcurveto{\pgfqpoint{1.345487in}{2.144745in}}{\pgfqpoint{1.340313in}{2.132254in}}{\pgfqpoint{1.340313in}{2.119232in}}%
\pgfpathcurveto{\pgfqpoint{1.340313in}{2.106209in}}{\pgfqpoint{1.345487in}{2.093718in}}{\pgfqpoint{1.354695in}{2.084509in}}%
\pgfpathcurveto{\pgfqpoint{1.363904in}{2.075301in}}{\pgfqpoint{1.376395in}{2.070127in}}{\pgfqpoint{1.389418in}{2.070127in}}%
\pgfpathlineto{\pgfqpoint{1.389418in}{2.070127in}}%
\pgfpathclose%
\pgfusepath{stroke,fill}%
\end{pgfscope}%
\begin{pgfscope}%
\pgfpathrectangle{\pgfqpoint{0.786164in}{0.768110in}}{\pgfqpoint{8.851069in}{7.081890in}}%
\pgfusepath{clip}%
\pgfsetbuttcap%
\pgfsetroundjoin%
\definecolor{currentfill}{rgb}{0.185556,0.418570,0.556753}%
\pgfsetfillcolor{currentfill}%
\pgfsetfillopacity{0.700000}%
\pgfsetlinewidth{0.501875pt}%
\definecolor{currentstroke}{rgb}{1.000000,1.000000,1.000000}%
\pgfsetstrokecolor{currentstroke}%
\pgfsetstrokeopacity{0.700000}%
\pgfsetdash{}{0pt}%
\pgfpathmoveto{\pgfqpoint{1.389418in}{2.026330in}}%
\pgfpathcurveto{\pgfqpoint{1.402440in}{2.026330in}}{\pgfqpoint{1.414931in}{2.031504in}}{\pgfqpoint{1.424140in}{2.040713in}}%
\pgfpathcurveto{\pgfqpoint{1.433348in}{2.049921in}}{\pgfqpoint{1.438522in}{2.062412in}}{\pgfqpoint{1.438522in}{2.075435in}}%
\pgfpathcurveto{\pgfqpoint{1.438522in}{2.088458in}}{\pgfqpoint{1.433348in}{2.100949in}}{\pgfqpoint{1.424140in}{2.110157in}}%
\pgfpathcurveto{\pgfqpoint{1.414931in}{2.119366in}}{\pgfqpoint{1.402440in}{2.124540in}}{\pgfqpoint{1.389418in}{2.124540in}}%
\pgfpathcurveto{\pgfqpoint{1.376395in}{2.124540in}}{\pgfqpoint{1.363904in}{2.119366in}}{\pgfqpoint{1.354695in}{2.110157in}}%
\pgfpathcurveto{\pgfqpoint{1.345487in}{2.100949in}}{\pgfqpoint{1.340313in}{2.088458in}}{\pgfqpoint{1.340313in}{2.075435in}}%
\pgfpathcurveto{\pgfqpoint{1.340313in}{2.062412in}}{\pgfqpoint{1.345487in}{2.049921in}}{\pgfqpoint{1.354695in}{2.040713in}}%
\pgfpathcurveto{\pgfqpoint{1.363904in}{2.031504in}}{\pgfqpoint{1.376395in}{2.026330in}}{\pgfqpoint{1.389418in}{2.026330in}}%
\pgfpathlineto{\pgfqpoint{1.389418in}{2.026330in}}%
\pgfpathclose%
\pgfusepath{stroke,fill}%
\end{pgfscope}%
\begin{pgfscope}%
\pgfpathrectangle{\pgfqpoint{0.786164in}{0.768110in}}{\pgfqpoint{8.851069in}{7.081890in}}%
\pgfusepath{clip}%
\pgfsetbuttcap%
\pgfsetroundjoin%
\definecolor{currentfill}{rgb}{0.179019,0.433756,0.557430}%
\pgfsetfillcolor{currentfill}%
\pgfsetfillopacity{0.700000}%
\pgfsetlinewidth{0.501875pt}%
\definecolor{currentstroke}{rgb}{1.000000,1.000000,1.000000}%
\pgfsetstrokecolor{currentstroke}%
\pgfsetstrokeopacity{0.700000}%
\pgfsetdash{}{0pt}%
\pgfpathmoveto{\pgfqpoint{1.261552in}{1.894941in}}%
\pgfpathcurveto{\pgfqpoint{1.274574in}{1.894941in}}{\pgfqpoint{1.287065in}{1.900115in}}{\pgfqpoint{1.296274in}{1.909323in}}%
\pgfpathcurveto{\pgfqpoint{1.305482in}{1.918532in}}{\pgfqpoint{1.310656in}{1.931023in}}{\pgfqpoint{1.310656in}{1.944046in}}%
\pgfpathcurveto{\pgfqpoint{1.310656in}{1.957068in}}{\pgfqpoint{1.305482in}{1.969559in}}{\pgfqpoint{1.296274in}{1.978768in}}%
\pgfpathcurveto{\pgfqpoint{1.287065in}{1.987976in}}{\pgfqpoint{1.274574in}{1.993150in}}{\pgfqpoint{1.261552in}{1.993150in}}%
\pgfpathcurveto{\pgfqpoint{1.248529in}{1.993150in}}{\pgfqpoint{1.236038in}{1.987976in}}{\pgfqpoint{1.226829in}{1.978768in}}%
\pgfpathcurveto{\pgfqpoint{1.217621in}{1.969559in}}{\pgfqpoint{1.212447in}{1.957068in}}{\pgfqpoint{1.212447in}{1.944046in}}%
\pgfpathcurveto{\pgfqpoint{1.212447in}{1.931023in}}{\pgfqpoint{1.217621in}{1.918532in}}{\pgfqpoint{1.226829in}{1.909323in}}%
\pgfpathcurveto{\pgfqpoint{1.236038in}{1.900115in}}{\pgfqpoint{1.248529in}{1.894941in}}{\pgfqpoint{1.261552in}{1.894941in}}%
\pgfpathlineto{\pgfqpoint{1.261552in}{1.894941in}}%
\pgfpathclose%
\pgfusepath{stroke,fill}%
\end{pgfscope}%
\begin{pgfscope}%
\pgfpathrectangle{\pgfqpoint{0.786164in}{0.768110in}}{\pgfqpoint{8.851069in}{7.081890in}}%
\pgfusepath{clip}%
\pgfsetbuttcap%
\pgfsetroundjoin%
\definecolor{currentfill}{rgb}{0.172719,0.448791,0.557885}%
\pgfsetfillcolor{currentfill}%
\pgfsetfillopacity{0.700000}%
\pgfsetlinewidth{0.501875pt}%
\definecolor{currentstroke}{rgb}{1.000000,1.000000,1.000000}%
\pgfsetstrokecolor{currentstroke}%
\pgfsetstrokeopacity{0.700000}%
\pgfsetdash{}{0pt}%
\pgfpathmoveto{\pgfqpoint{1.252418in}{1.785450in}}%
\pgfpathcurveto{\pgfqpoint{1.265441in}{1.785450in}}{\pgfqpoint{1.277932in}{1.790624in}}{\pgfqpoint{1.287140in}{1.799832in}}%
\pgfpathcurveto{\pgfqpoint{1.296349in}{1.809041in}}{\pgfqpoint{1.301523in}{1.821532in}}{\pgfqpoint{1.301523in}{1.834555in}}%
\pgfpathcurveto{\pgfqpoint{1.301523in}{1.847577in}}{\pgfqpoint{1.296349in}{1.860068in}}{\pgfqpoint{1.287140in}{1.869277in}}%
\pgfpathcurveto{\pgfqpoint{1.277932in}{1.878485in}}{\pgfqpoint{1.265441in}{1.883659in}}{\pgfqpoint{1.252418in}{1.883659in}}%
\pgfpathcurveto{\pgfqpoint{1.239396in}{1.883659in}}{\pgfqpoint{1.226904in}{1.878485in}}{\pgfqpoint{1.217696in}{1.869277in}}%
\pgfpathcurveto{\pgfqpoint{1.208488in}{1.860068in}}{\pgfqpoint{1.203314in}{1.847577in}}{\pgfqpoint{1.203314in}{1.834555in}}%
\pgfpathcurveto{\pgfqpoint{1.203314in}{1.821532in}}{\pgfqpoint{1.208488in}{1.809041in}}{\pgfqpoint{1.217696in}{1.799832in}}%
\pgfpathcurveto{\pgfqpoint{1.226904in}{1.790624in}}{\pgfqpoint{1.239396in}{1.785450in}}{\pgfqpoint{1.252418in}{1.785450in}}%
\pgfpathlineto{\pgfqpoint{1.252418in}{1.785450in}}%
\pgfpathclose%
\pgfusepath{stroke,fill}%
\end{pgfscope}%
\begin{pgfscope}%
\pgfpathrectangle{\pgfqpoint{0.786164in}{0.768110in}}{\pgfqpoint{8.851069in}{7.081890in}}%
\pgfusepath{clip}%
\pgfsetbuttcap%
\pgfsetroundjoin%
\definecolor{currentfill}{rgb}{0.174274,0.445044,0.557792}%
\pgfsetfillcolor{currentfill}%
\pgfsetfillopacity{0.700000}%
\pgfsetlinewidth{0.501875pt}%
\definecolor{currentstroke}{rgb}{1.000000,1.000000,1.000000}%
\pgfsetstrokecolor{currentstroke}%
\pgfsetstrokeopacity{0.700000}%
\pgfsetdash{}{0pt}%
\pgfpathmoveto{\pgfqpoint{1.288951in}{1.785450in}}%
\pgfpathcurveto{\pgfqpoint{1.301974in}{1.785450in}}{\pgfqpoint{1.314465in}{1.790624in}}{\pgfqpoint{1.323674in}{1.799832in}}%
\pgfpathcurveto{\pgfqpoint{1.332882in}{1.809041in}}{\pgfqpoint{1.338056in}{1.821532in}}{\pgfqpoint{1.338056in}{1.834555in}}%
\pgfpathcurveto{\pgfqpoint{1.338056in}{1.847577in}}{\pgfqpoint{1.332882in}{1.860068in}}{\pgfqpoint{1.323674in}{1.869277in}}%
\pgfpathcurveto{\pgfqpoint{1.314465in}{1.878485in}}{\pgfqpoint{1.301974in}{1.883659in}}{\pgfqpoint{1.288951in}{1.883659in}}%
\pgfpathcurveto{\pgfqpoint{1.275929in}{1.883659in}}{\pgfqpoint{1.263438in}{1.878485in}}{\pgfqpoint{1.254229in}{1.869277in}}%
\pgfpathcurveto{\pgfqpoint{1.245021in}{1.860068in}}{\pgfqpoint{1.239847in}{1.847577in}}{\pgfqpoint{1.239847in}{1.834555in}}%
\pgfpathcurveto{\pgfqpoint{1.239847in}{1.821532in}}{\pgfqpoint{1.245021in}{1.809041in}}{\pgfqpoint{1.254229in}{1.799832in}}%
\pgfpathcurveto{\pgfqpoint{1.263438in}{1.790624in}}{\pgfqpoint{1.275929in}{1.785450in}}{\pgfqpoint{1.288951in}{1.785450in}}%
\pgfpathlineto{\pgfqpoint{1.288951in}{1.785450in}}%
\pgfpathclose%
\pgfusepath{stroke,fill}%
\end{pgfscope}%
\begin{pgfscope}%
\pgfpathrectangle{\pgfqpoint{0.786164in}{0.768110in}}{\pgfqpoint{8.851069in}{7.081890in}}%
\pgfusepath{clip}%
\pgfsetbuttcap%
\pgfsetroundjoin%
\definecolor{currentfill}{rgb}{0.172719,0.448791,0.557885}%
\pgfsetfillcolor{currentfill}%
\pgfsetfillopacity{0.700000}%
\pgfsetlinewidth{0.501875pt}%
\definecolor{currentstroke}{rgb}{1.000000,1.000000,1.000000}%
\pgfsetstrokecolor{currentstroke}%
\pgfsetstrokeopacity{0.700000}%
\pgfsetdash{}{0pt}%
\pgfpathmoveto{\pgfqpoint{1.298085in}{1.741653in}}%
\pgfpathcurveto{\pgfqpoint{1.311107in}{1.741653in}}{\pgfqpoint{1.323598in}{1.746827in}}{\pgfqpoint{1.332807in}{1.756036in}}%
\pgfpathcurveto{\pgfqpoint{1.342015in}{1.765244in}}{\pgfqpoint{1.347189in}{1.777735in}}{\pgfqpoint{1.347189in}{1.790758in}}%
\pgfpathcurveto{\pgfqpoint{1.347189in}{1.803781in}}{\pgfqpoint{1.342015in}{1.816272in}}{\pgfqpoint{1.332807in}{1.825480in}}%
\pgfpathcurveto{\pgfqpoint{1.323598in}{1.834689in}}{\pgfqpoint{1.311107in}{1.839863in}}{\pgfqpoint{1.298085in}{1.839863in}}%
\pgfpathcurveto{\pgfqpoint{1.285062in}{1.839863in}}{\pgfqpoint{1.272571in}{1.834689in}}{\pgfqpoint{1.263362in}{1.825480in}}%
\pgfpathcurveto{\pgfqpoint{1.254154in}{1.816272in}}{\pgfqpoint{1.248980in}{1.803781in}}{\pgfqpoint{1.248980in}{1.790758in}}%
\pgfpathcurveto{\pgfqpoint{1.248980in}{1.777735in}}{\pgfqpoint{1.254154in}{1.765244in}}{\pgfqpoint{1.263362in}{1.756036in}}%
\pgfpathcurveto{\pgfqpoint{1.272571in}{1.746827in}}{\pgfqpoint{1.285062in}{1.741653in}}{\pgfqpoint{1.298085in}{1.741653in}}%
\pgfpathlineto{\pgfqpoint{1.298085in}{1.741653in}}%
\pgfpathclose%
\pgfusepath{stroke,fill}%
\end{pgfscope}%
\begin{pgfscope}%
\pgfpathrectangle{\pgfqpoint{0.786164in}{0.768110in}}{\pgfqpoint{8.851069in}{7.081890in}}%
\pgfusepath{clip}%
\pgfsetbuttcap%
\pgfsetroundjoin%
\definecolor{currentfill}{rgb}{0.175841,0.441290,0.557685}%
\pgfsetfillcolor{currentfill}%
\pgfsetfillopacity{0.700000}%
\pgfsetlinewidth{0.501875pt}%
\definecolor{currentstroke}{rgb}{1.000000,1.000000,1.000000}%
\pgfsetstrokecolor{currentstroke}%
\pgfsetstrokeopacity{0.700000}%
\pgfsetdash{}{0pt}%
\pgfpathmoveto{\pgfqpoint{1.389418in}{1.807348in}}%
\pgfpathcurveto{\pgfqpoint{1.402440in}{1.807348in}}{\pgfqpoint{1.414931in}{1.812522in}}{\pgfqpoint{1.424140in}{1.821731in}}%
\pgfpathcurveto{\pgfqpoint{1.433348in}{1.830939in}}{\pgfqpoint{1.438522in}{1.843430in}}{\pgfqpoint{1.438522in}{1.856453in}}%
\pgfpathcurveto{\pgfqpoint{1.438522in}{1.869475in}}{\pgfqpoint{1.433348in}{1.881967in}}{\pgfqpoint{1.424140in}{1.891175in}}%
\pgfpathcurveto{\pgfqpoint{1.414931in}{1.900383in}}{\pgfqpoint{1.402440in}{1.905557in}}{\pgfqpoint{1.389418in}{1.905557in}}%
\pgfpathcurveto{\pgfqpoint{1.376395in}{1.905557in}}{\pgfqpoint{1.363904in}{1.900383in}}{\pgfqpoint{1.354695in}{1.891175in}}%
\pgfpathcurveto{\pgfqpoint{1.345487in}{1.881967in}}{\pgfqpoint{1.340313in}{1.869475in}}{\pgfqpoint{1.340313in}{1.856453in}}%
\pgfpathcurveto{\pgfqpoint{1.340313in}{1.843430in}}{\pgfqpoint{1.345487in}{1.830939in}}{\pgfqpoint{1.354695in}{1.821731in}}%
\pgfpathcurveto{\pgfqpoint{1.363904in}{1.812522in}}{\pgfqpoint{1.376395in}{1.807348in}}{\pgfqpoint{1.389418in}{1.807348in}}%
\pgfpathlineto{\pgfqpoint{1.389418in}{1.807348in}}%
\pgfpathclose%
\pgfusepath{stroke,fill}%
\end{pgfscope}%
\begin{pgfscope}%
\pgfpathrectangle{\pgfqpoint{0.786164in}{0.768110in}}{\pgfqpoint{8.851069in}{7.081890in}}%
\pgfusepath{clip}%
\pgfsetbuttcap%
\pgfsetroundjoin%
\definecolor{currentfill}{rgb}{0.179019,0.433756,0.557430}%
\pgfsetfillcolor{currentfill}%
\pgfsetfillopacity{0.700000}%
\pgfsetlinewidth{0.501875pt}%
\definecolor{currentstroke}{rgb}{1.000000,1.000000,1.000000}%
\pgfsetstrokecolor{currentstroke}%
\pgfsetstrokeopacity{0.700000}%
\pgfsetdash{}{0pt}%
\pgfpathmoveto{\pgfqpoint{1.398551in}{1.829246in}}%
\pgfpathcurveto{\pgfqpoint{1.411574in}{1.829246in}}{\pgfqpoint{1.424065in}{1.834420in}}{\pgfqpoint{1.433273in}{1.843629in}}%
\pgfpathcurveto{\pgfqpoint{1.442481in}{1.852837in}}{\pgfqpoint{1.447655in}{1.865328in}}{\pgfqpoint{1.447655in}{1.878351in}}%
\pgfpathcurveto{\pgfqpoint{1.447655in}{1.891374in}}{\pgfqpoint{1.442481in}{1.903865in}}{\pgfqpoint{1.433273in}{1.913073in}}%
\pgfpathcurveto{\pgfqpoint{1.424065in}{1.922282in}}{\pgfqpoint{1.411574in}{1.927456in}}{\pgfqpoint{1.398551in}{1.927456in}}%
\pgfpathcurveto{\pgfqpoint{1.385528in}{1.927456in}}{\pgfqpoint{1.373037in}{1.922282in}}{\pgfqpoint{1.363829in}{1.913073in}}%
\pgfpathcurveto{\pgfqpoint{1.354620in}{1.903865in}}{\pgfqpoint{1.349446in}{1.891374in}}{\pgfqpoint{1.349446in}{1.878351in}}%
\pgfpathcurveto{\pgfqpoint{1.349446in}{1.865328in}}{\pgfqpoint{1.354620in}{1.852837in}}{\pgfqpoint{1.363829in}{1.843629in}}%
\pgfpathcurveto{\pgfqpoint{1.373037in}{1.834420in}}{\pgfqpoint{1.385528in}{1.829246in}}{\pgfqpoint{1.398551in}{1.829246in}}%
\pgfpathlineto{\pgfqpoint{1.398551in}{1.829246in}}%
\pgfpathclose%
\pgfusepath{stroke,fill}%
\end{pgfscope}%
\begin{pgfscope}%
\pgfpathrectangle{\pgfqpoint{0.786164in}{0.768110in}}{\pgfqpoint{8.851069in}{7.081890in}}%
\pgfusepath{clip}%
\pgfsetbuttcap%
\pgfsetroundjoin%
\definecolor{currentfill}{rgb}{0.177423,0.437527,0.557565}%
\pgfsetfillcolor{currentfill}%
\pgfsetfillopacity{0.700000}%
\pgfsetlinewidth{0.501875pt}%
\definecolor{currentstroke}{rgb}{1.000000,1.000000,1.000000}%
\pgfsetstrokecolor{currentstroke}%
\pgfsetstrokeopacity{0.700000}%
\pgfsetdash{}{0pt}%
\pgfpathmoveto{\pgfqpoint{1.389418in}{1.807348in}}%
\pgfpathcurveto{\pgfqpoint{1.402440in}{1.807348in}}{\pgfqpoint{1.414931in}{1.812522in}}{\pgfqpoint{1.424140in}{1.821731in}}%
\pgfpathcurveto{\pgfqpoint{1.433348in}{1.830939in}}{\pgfqpoint{1.438522in}{1.843430in}}{\pgfqpoint{1.438522in}{1.856453in}}%
\pgfpathcurveto{\pgfqpoint{1.438522in}{1.869475in}}{\pgfqpoint{1.433348in}{1.881967in}}{\pgfqpoint{1.424140in}{1.891175in}}%
\pgfpathcurveto{\pgfqpoint{1.414931in}{1.900383in}}{\pgfqpoint{1.402440in}{1.905557in}}{\pgfqpoint{1.389418in}{1.905557in}}%
\pgfpathcurveto{\pgfqpoint{1.376395in}{1.905557in}}{\pgfqpoint{1.363904in}{1.900383in}}{\pgfqpoint{1.354695in}{1.891175in}}%
\pgfpathcurveto{\pgfqpoint{1.345487in}{1.881967in}}{\pgfqpoint{1.340313in}{1.869475in}}{\pgfqpoint{1.340313in}{1.856453in}}%
\pgfpathcurveto{\pgfqpoint{1.340313in}{1.843430in}}{\pgfqpoint{1.345487in}{1.830939in}}{\pgfqpoint{1.354695in}{1.821731in}}%
\pgfpathcurveto{\pgfqpoint{1.363904in}{1.812522in}}{\pgfqpoint{1.376395in}{1.807348in}}{\pgfqpoint{1.389418in}{1.807348in}}%
\pgfpathlineto{\pgfqpoint{1.389418in}{1.807348in}}%
\pgfpathclose%
\pgfusepath{stroke,fill}%
\end{pgfscope}%
\begin{pgfscope}%
\pgfpathrectangle{\pgfqpoint{0.786164in}{0.768110in}}{\pgfqpoint{8.851069in}{7.081890in}}%
\pgfusepath{clip}%
\pgfsetbuttcap%
\pgfsetroundjoin%
\definecolor{currentfill}{rgb}{0.175841,0.441290,0.557685}%
\pgfsetfillcolor{currentfill}%
\pgfsetfillopacity{0.700000}%
\pgfsetlinewidth{0.501875pt}%
\definecolor{currentstroke}{rgb}{1.000000,1.000000,1.000000}%
\pgfsetstrokecolor{currentstroke}%
\pgfsetstrokeopacity{0.700000}%
\pgfsetdash{}{0pt}%
\pgfpathmoveto{\pgfqpoint{1.343751in}{1.719755in}}%
\pgfpathcurveto{\pgfqpoint{1.356774in}{1.719755in}}{\pgfqpoint{1.369265in}{1.724929in}}{\pgfqpoint{1.378473in}{1.734138in}}%
\pgfpathcurveto{\pgfqpoint{1.387682in}{1.743346in}}{\pgfqpoint{1.392856in}{1.755837in}}{\pgfqpoint{1.392856in}{1.768860in}}%
\pgfpathcurveto{\pgfqpoint{1.392856in}{1.781883in}}{\pgfqpoint{1.387682in}{1.794374in}}{\pgfqpoint{1.378473in}{1.803582in}}%
\pgfpathcurveto{\pgfqpoint{1.369265in}{1.812790in}}{\pgfqpoint{1.356774in}{1.817964in}}{\pgfqpoint{1.343751in}{1.817964in}}%
\pgfpathcurveto{\pgfqpoint{1.330728in}{1.817964in}}{\pgfqpoint{1.318237in}{1.812790in}}{\pgfqpoint{1.309029in}{1.803582in}}%
\pgfpathcurveto{\pgfqpoint{1.299820in}{1.794374in}}{\pgfqpoint{1.294646in}{1.781883in}}{\pgfqpoint{1.294646in}{1.768860in}}%
\pgfpathcurveto{\pgfqpoint{1.294646in}{1.755837in}}{\pgfqpoint{1.299820in}{1.743346in}}{\pgfqpoint{1.309029in}{1.734138in}}%
\pgfpathcurveto{\pgfqpoint{1.318237in}{1.724929in}}{\pgfqpoint{1.330728in}{1.719755in}}{\pgfqpoint{1.343751in}{1.719755in}}%
\pgfpathlineto{\pgfqpoint{1.343751in}{1.719755in}}%
\pgfpathclose%
\pgfusepath{stroke,fill}%
\end{pgfscope}%
\begin{pgfscope}%
\pgfpathrectangle{\pgfqpoint{0.786164in}{0.768110in}}{\pgfqpoint{8.851069in}{7.081890in}}%
\pgfusepath{clip}%
\pgfsetbuttcap%
\pgfsetroundjoin%
\definecolor{currentfill}{rgb}{0.179019,0.433756,0.557430}%
\pgfsetfillcolor{currentfill}%
\pgfsetfillopacity{0.700000}%
\pgfsetlinewidth{0.501875pt}%
\definecolor{currentstroke}{rgb}{1.000000,1.000000,1.000000}%
\pgfsetstrokecolor{currentstroke}%
\pgfsetstrokeopacity{0.700000}%
\pgfsetdash{}{0pt}%
\pgfpathmoveto{\pgfqpoint{1.453351in}{1.785450in}}%
\pgfpathcurveto{\pgfqpoint{1.466373in}{1.785450in}}{\pgfqpoint{1.478864in}{1.790624in}}{\pgfqpoint{1.488073in}{1.799832in}}%
\pgfpathcurveto{\pgfqpoint{1.497281in}{1.809041in}}{\pgfqpoint{1.502455in}{1.821532in}}{\pgfqpoint{1.502455in}{1.834555in}}%
\pgfpathcurveto{\pgfqpoint{1.502455in}{1.847577in}}{\pgfqpoint{1.497281in}{1.860068in}}{\pgfqpoint{1.488073in}{1.869277in}}%
\pgfpathcurveto{\pgfqpoint{1.478864in}{1.878485in}}{\pgfqpoint{1.466373in}{1.883659in}}{\pgfqpoint{1.453351in}{1.883659in}}%
\pgfpathcurveto{\pgfqpoint{1.440328in}{1.883659in}}{\pgfqpoint{1.427837in}{1.878485in}}{\pgfqpoint{1.418628in}{1.869277in}}%
\pgfpathcurveto{\pgfqpoint{1.409420in}{1.860068in}}{\pgfqpoint{1.404246in}{1.847577in}}{\pgfqpoint{1.404246in}{1.834555in}}%
\pgfpathcurveto{\pgfqpoint{1.404246in}{1.821532in}}{\pgfqpoint{1.409420in}{1.809041in}}{\pgfqpoint{1.418628in}{1.799832in}}%
\pgfpathcurveto{\pgfqpoint{1.427837in}{1.790624in}}{\pgfqpoint{1.440328in}{1.785450in}}{\pgfqpoint{1.453351in}{1.785450in}}%
\pgfpathlineto{\pgfqpoint{1.453351in}{1.785450in}}%
\pgfpathclose%
\pgfusepath{stroke,fill}%
\end{pgfscope}%
\begin{pgfscope}%
\pgfpathrectangle{\pgfqpoint{0.786164in}{0.768110in}}{\pgfqpoint{8.851069in}{7.081890in}}%
\pgfusepath{clip}%
\pgfsetbuttcap%
\pgfsetroundjoin%
\definecolor{currentfill}{rgb}{0.177423,0.437527,0.557565}%
\pgfsetfillcolor{currentfill}%
\pgfsetfillopacity{0.700000}%
\pgfsetlinewidth{0.501875pt}%
\definecolor{currentstroke}{rgb}{1.000000,1.000000,1.000000}%
\pgfsetstrokecolor{currentstroke}%
\pgfsetstrokeopacity{0.700000}%
\pgfsetdash{}{0pt}%
\pgfpathmoveto{\pgfqpoint{1.362018in}{1.741653in}}%
\pgfpathcurveto{\pgfqpoint{1.375040in}{1.741653in}}{\pgfqpoint{1.387531in}{1.746827in}}{\pgfqpoint{1.396740in}{1.756036in}}%
\pgfpathcurveto{\pgfqpoint{1.405948in}{1.765244in}}{\pgfqpoint{1.411122in}{1.777735in}}{\pgfqpoint{1.411122in}{1.790758in}}%
\pgfpathcurveto{\pgfqpoint{1.411122in}{1.803781in}}{\pgfqpoint{1.405948in}{1.816272in}}{\pgfqpoint{1.396740in}{1.825480in}}%
\pgfpathcurveto{\pgfqpoint{1.387531in}{1.834689in}}{\pgfqpoint{1.375040in}{1.839863in}}{\pgfqpoint{1.362018in}{1.839863in}}%
\pgfpathcurveto{\pgfqpoint{1.348995in}{1.839863in}}{\pgfqpoint{1.336504in}{1.834689in}}{\pgfqpoint{1.327295in}{1.825480in}}%
\pgfpathcurveto{\pgfqpoint{1.318087in}{1.816272in}}{\pgfqpoint{1.312913in}{1.803781in}}{\pgfqpoint{1.312913in}{1.790758in}}%
\pgfpathcurveto{\pgfqpoint{1.312913in}{1.777735in}}{\pgfqpoint{1.318087in}{1.765244in}}{\pgfqpoint{1.327295in}{1.756036in}}%
\pgfpathcurveto{\pgfqpoint{1.336504in}{1.746827in}}{\pgfqpoint{1.348995in}{1.741653in}}{\pgfqpoint{1.362018in}{1.741653in}}%
\pgfpathlineto{\pgfqpoint{1.362018in}{1.741653in}}%
\pgfpathclose%
\pgfusepath{stroke,fill}%
\end{pgfscope}%
\begin{pgfscope}%
\pgfpathrectangle{\pgfqpoint{0.786164in}{0.768110in}}{\pgfqpoint{8.851069in}{7.081890in}}%
\pgfusepath{clip}%
\pgfsetbuttcap%
\pgfsetroundjoin%
\definecolor{currentfill}{rgb}{0.124780,0.640461,0.527068}%
\pgfsetfillcolor{currentfill}%
\pgfsetfillopacity{0.700000}%
\pgfsetlinewidth{0.501875pt}%
\definecolor{currentstroke}{rgb}{1.000000,1.000000,1.000000}%
\pgfsetstrokecolor{currentstroke}%
\pgfsetstrokeopacity{0.700000}%
\pgfsetdash{}{0pt}%
\pgfpathmoveto{\pgfqpoint{4.997066in}{1.544569in}}%
\pgfpathcurveto{\pgfqpoint{5.010089in}{1.544569in}}{\pgfqpoint{5.022580in}{1.549743in}}{\pgfqpoint{5.031788in}{1.558952in}}%
\pgfpathcurveto{\pgfqpoint{5.040997in}{1.568160in}}{\pgfqpoint{5.046171in}{1.580651in}}{\pgfqpoint{5.046171in}{1.593674in}}%
\pgfpathcurveto{\pgfqpoint{5.046171in}{1.606697in}}{\pgfqpoint{5.040997in}{1.619188in}}{\pgfqpoint{5.031788in}{1.628396in}}%
\pgfpathcurveto{\pgfqpoint{5.022580in}{1.637605in}}{\pgfqpoint{5.010089in}{1.642779in}}{\pgfqpoint{4.997066in}{1.642779in}}%
\pgfpathcurveto{\pgfqpoint{4.984043in}{1.642779in}}{\pgfqpoint{4.971552in}{1.637605in}}{\pgfqpoint{4.962344in}{1.628396in}}%
\pgfpathcurveto{\pgfqpoint{4.953135in}{1.619188in}}{\pgfqpoint{4.947961in}{1.606697in}}{\pgfqpoint{4.947961in}{1.593674in}}%
\pgfpathcurveto{\pgfqpoint{4.947961in}{1.580651in}}{\pgfqpoint{4.953135in}{1.568160in}}{\pgfqpoint{4.962344in}{1.558952in}}%
\pgfpathcurveto{\pgfqpoint{4.971552in}{1.549743in}}{\pgfqpoint{4.984043in}{1.544569in}}{\pgfqpoint{4.997066in}{1.544569in}}%
\pgfpathlineto{\pgfqpoint{4.997066in}{1.544569in}}%
\pgfpathclose%
\pgfusepath{stroke,fill}%
\end{pgfscope}%
\begin{pgfscope}%
\pgfpathrectangle{\pgfqpoint{0.786164in}{0.768110in}}{\pgfqpoint{8.851069in}{7.081890in}}%
\pgfusepath{clip}%
\pgfsetbuttcap%
\pgfsetroundjoin%
\definecolor{currentfill}{rgb}{0.137339,0.662252,0.515571}%
\pgfsetfillcolor{currentfill}%
\pgfsetfillopacity{0.700000}%
\pgfsetlinewidth{0.501875pt}%
\definecolor{currentstroke}{rgb}{1.000000,1.000000,1.000000}%
\pgfsetstrokecolor{currentstroke}%
\pgfsetstrokeopacity{0.700000}%
\pgfsetdash{}{0pt}%
\pgfpathmoveto{\pgfqpoint{4.832667in}{1.435078in}}%
\pgfpathcurveto{\pgfqpoint{4.845690in}{1.435078in}}{\pgfqpoint{4.858181in}{1.440252in}}{\pgfqpoint{4.867389in}{1.449461in}}%
\pgfpathcurveto{\pgfqpoint{4.876598in}{1.458669in}}{\pgfqpoint{4.881772in}{1.471160in}}{\pgfqpoint{4.881772in}{1.484183in}}%
\pgfpathcurveto{\pgfqpoint{4.881772in}{1.497205in}}{\pgfqpoint{4.876598in}{1.509697in}}{\pgfqpoint{4.867389in}{1.518905in}}%
\pgfpathcurveto{\pgfqpoint{4.858181in}{1.528113in}}{\pgfqpoint{4.845690in}{1.533287in}}{\pgfqpoint{4.832667in}{1.533287in}}%
\pgfpathcurveto{\pgfqpoint{4.819644in}{1.533287in}}{\pgfqpoint{4.807153in}{1.528113in}}{\pgfqpoint{4.797945in}{1.518905in}}%
\pgfpathcurveto{\pgfqpoint{4.788736in}{1.509697in}}{\pgfqpoint{4.783562in}{1.497205in}}{\pgfqpoint{4.783562in}{1.484183in}}%
\pgfpathcurveto{\pgfqpoint{4.783562in}{1.471160in}}{\pgfqpoint{4.788736in}{1.458669in}}{\pgfqpoint{4.797945in}{1.449461in}}%
\pgfpathcurveto{\pgfqpoint{4.807153in}{1.440252in}}{\pgfqpoint{4.819644in}{1.435078in}}{\pgfqpoint{4.832667in}{1.435078in}}%
\pgfpathlineto{\pgfqpoint{4.832667in}{1.435078in}}%
\pgfpathclose%
\pgfusepath{stroke,fill}%
\end{pgfscope}%
\begin{pgfscope}%
\pgfpathrectangle{\pgfqpoint{0.786164in}{0.768110in}}{\pgfqpoint{8.851069in}{7.081890in}}%
\pgfusepath{clip}%
\pgfsetbuttcap%
\pgfsetroundjoin%
\definecolor{currentfill}{rgb}{0.162016,0.687316,0.499129}%
\pgfsetfillcolor{currentfill}%
\pgfsetfillopacity{0.700000}%
\pgfsetlinewidth{0.501875pt}%
\definecolor{currentstroke}{rgb}{1.000000,1.000000,1.000000}%
\pgfsetstrokecolor{currentstroke}%
\pgfsetstrokeopacity{0.700000}%
\pgfsetdash{}{0pt}%
\pgfpathmoveto{\pgfqpoint{4.668268in}{1.325587in}}%
\pgfpathcurveto{\pgfqpoint{4.681290in}{1.325587in}}{\pgfqpoint{4.693782in}{1.330761in}}{\pgfqpoint{4.702990in}{1.339969in}}%
\pgfpathcurveto{\pgfqpoint{4.712198in}{1.349178in}}{\pgfqpoint{4.717372in}{1.361669in}}{\pgfqpoint{4.717372in}{1.374692in}}%
\pgfpathcurveto{\pgfqpoint{4.717372in}{1.387714in}}{\pgfqpoint{4.712198in}{1.400205in}}{\pgfqpoint{4.702990in}{1.409414in}}%
\pgfpathcurveto{\pgfqpoint{4.693782in}{1.418622in}}{\pgfqpoint{4.681290in}{1.423796in}}{\pgfqpoint{4.668268in}{1.423796in}}%
\pgfpathcurveto{\pgfqpoint{4.655245in}{1.423796in}}{\pgfqpoint{4.642754in}{1.418622in}}{\pgfqpoint{4.633546in}{1.409414in}}%
\pgfpathcurveto{\pgfqpoint{4.624337in}{1.400205in}}{\pgfqpoint{4.619163in}{1.387714in}}{\pgfqpoint{4.619163in}{1.374692in}}%
\pgfpathcurveto{\pgfqpoint{4.619163in}{1.361669in}}{\pgfqpoint{4.624337in}{1.349178in}}{\pgfqpoint{4.633546in}{1.339969in}}%
\pgfpathcurveto{\pgfqpoint{4.642754in}{1.330761in}}{\pgfqpoint{4.655245in}{1.325587in}}{\pgfqpoint{4.668268in}{1.325587in}}%
\pgfpathlineto{\pgfqpoint{4.668268in}{1.325587in}}%
\pgfpathclose%
\pgfusepath{stroke,fill}%
\end{pgfscope}%
\begin{pgfscope}%
\pgfpathrectangle{\pgfqpoint{0.786164in}{0.768110in}}{\pgfqpoint{8.851069in}{7.081890in}}%
\pgfusepath{clip}%
\pgfsetbuttcap%
\pgfsetroundjoin%
\definecolor{currentfill}{rgb}{0.191090,0.708366,0.482284}%
\pgfsetfillcolor{currentfill}%
\pgfsetfillopacity{0.700000}%
\pgfsetlinewidth{0.501875pt}%
\definecolor{currentstroke}{rgb}{1.000000,1.000000,1.000000}%
\pgfsetstrokecolor{currentstroke}%
\pgfsetstrokeopacity{0.700000}%
\pgfsetdash{}{0pt}%
\pgfpathmoveto{\pgfqpoint{4.604335in}{1.303689in}}%
\pgfpathcurveto{\pgfqpoint{4.617357in}{1.303689in}}{\pgfqpoint{4.629848in}{1.308863in}}{\pgfqpoint{4.639057in}{1.318071in}}%
\pgfpathcurveto{\pgfqpoint{4.648265in}{1.327280in}}{\pgfqpoint{4.653439in}{1.339771in}}{\pgfqpoint{4.653439in}{1.352793in}}%
\pgfpathcurveto{\pgfqpoint{4.653439in}{1.365816in}}{\pgfqpoint{4.648265in}{1.378307in}}{\pgfqpoint{4.639057in}{1.387516in}}%
\pgfpathcurveto{\pgfqpoint{4.629848in}{1.396724in}}{\pgfqpoint{4.617357in}{1.401898in}}{\pgfqpoint{4.604335in}{1.401898in}}%
\pgfpathcurveto{\pgfqpoint{4.591312in}{1.401898in}}{\pgfqpoint{4.578821in}{1.396724in}}{\pgfqpoint{4.569612in}{1.387516in}}%
\pgfpathcurveto{\pgfqpoint{4.560404in}{1.378307in}}{\pgfqpoint{4.555230in}{1.365816in}}{\pgfqpoint{4.555230in}{1.352793in}}%
\pgfpathcurveto{\pgfqpoint{4.555230in}{1.339771in}}{\pgfqpoint{4.560404in}{1.327280in}}{\pgfqpoint{4.569612in}{1.318071in}}%
\pgfpathcurveto{\pgfqpoint{4.578821in}{1.308863in}}{\pgfqpoint{4.591312in}{1.303689in}}{\pgfqpoint{4.604335in}{1.303689in}}%
\pgfpathlineto{\pgfqpoint{4.604335in}{1.303689in}}%
\pgfpathclose%
\pgfusepath{stroke,fill}%
\end{pgfscope}%
\begin{pgfscope}%
\pgfpathrectangle{\pgfqpoint{0.786164in}{0.768110in}}{\pgfqpoint{8.851069in}{7.081890in}}%
\pgfusepath{clip}%
\pgfsetbuttcap%
\pgfsetroundjoin%
\definecolor{currentfill}{rgb}{0.208030,0.718701,0.472873}%
\pgfsetfillcolor{currentfill}%
\pgfsetfillopacity{0.700000}%
\pgfsetlinewidth{0.501875pt}%
\definecolor{currentstroke}{rgb}{1.000000,1.000000,1.000000}%
\pgfsetstrokecolor{currentstroke}%
\pgfsetstrokeopacity{0.700000}%
\pgfsetdash{}{0pt}%
\pgfpathmoveto{\pgfqpoint{4.549535in}{1.259892in}}%
\pgfpathcurveto{\pgfqpoint{4.562558in}{1.259892in}}{\pgfqpoint{4.575049in}{1.265066in}}{\pgfqpoint{4.584257in}{1.274275in}}%
\pgfpathcurveto{\pgfqpoint{4.593466in}{1.283483in}}{\pgfqpoint{4.598640in}{1.295974in}}{\pgfqpoint{4.598640in}{1.308997in}}%
\pgfpathcurveto{\pgfqpoint{4.598640in}{1.322020in}}{\pgfqpoint{4.593466in}{1.334511in}}{\pgfqpoint{4.584257in}{1.343719in}}%
\pgfpathcurveto{\pgfqpoint{4.575049in}{1.352928in}}{\pgfqpoint{4.562558in}{1.358101in}}{\pgfqpoint{4.549535in}{1.358101in}}%
\pgfpathcurveto{\pgfqpoint{4.536512in}{1.358101in}}{\pgfqpoint{4.524021in}{1.352928in}}{\pgfqpoint{4.514813in}{1.343719in}}%
\pgfpathcurveto{\pgfqpoint{4.505604in}{1.334511in}}{\pgfqpoint{4.500430in}{1.322020in}}{\pgfqpoint{4.500430in}{1.308997in}}%
\pgfpathcurveto{\pgfqpoint{4.500430in}{1.295974in}}{\pgfqpoint{4.505604in}{1.283483in}}{\pgfqpoint{4.514813in}{1.274275in}}%
\pgfpathcurveto{\pgfqpoint{4.524021in}{1.265066in}}{\pgfqpoint{4.536512in}{1.259892in}}{\pgfqpoint{4.549535in}{1.259892in}}%
\pgfpathlineto{\pgfqpoint{4.549535in}{1.259892in}}%
\pgfpathclose%
\pgfusepath{stroke,fill}%
\end{pgfscope}%
\begin{pgfscope}%
\pgfpathrectangle{\pgfqpoint{0.786164in}{0.768110in}}{\pgfqpoint{8.851069in}{7.081890in}}%
\pgfusepath{clip}%
\pgfsetbuttcap%
\pgfsetroundjoin%
\definecolor{currentfill}{rgb}{0.288921,0.758394,0.428426}%
\pgfsetfillcolor{currentfill}%
\pgfsetfillopacity{0.700000}%
\pgfsetlinewidth{0.501875pt}%
\definecolor{currentstroke}{rgb}{1.000000,1.000000,1.000000}%
\pgfsetstrokecolor{currentstroke}%
\pgfsetstrokeopacity{0.700000}%
\pgfsetdash{}{0pt}%
\pgfpathmoveto{\pgfqpoint{4.102004in}{1.172299in}}%
\pgfpathcurveto{\pgfqpoint{4.115027in}{1.172299in}}{\pgfqpoint{4.127518in}{1.177473in}}{\pgfqpoint{4.136726in}{1.186682in}}%
\pgfpathcurveto{\pgfqpoint{4.145935in}{1.195890in}}{\pgfqpoint{4.151109in}{1.208381in}}{\pgfqpoint{4.151109in}{1.221404in}}%
\pgfpathcurveto{\pgfqpoint{4.151109in}{1.234427in}}{\pgfqpoint{4.145935in}{1.246918in}}{\pgfqpoint{4.136726in}{1.256126in}}%
\pgfpathcurveto{\pgfqpoint{4.127518in}{1.265335in}}{\pgfqpoint{4.115027in}{1.270509in}}{\pgfqpoint{4.102004in}{1.270509in}}%
\pgfpathcurveto{\pgfqpoint{4.088981in}{1.270509in}}{\pgfqpoint{4.076490in}{1.265335in}}{\pgfqpoint{4.067282in}{1.256126in}}%
\pgfpathcurveto{\pgfqpoint{4.058073in}{1.246918in}}{\pgfqpoint{4.052899in}{1.234427in}}{\pgfqpoint{4.052899in}{1.221404in}}%
\pgfpathcurveto{\pgfqpoint{4.052899in}{1.208381in}}{\pgfqpoint{4.058073in}{1.195890in}}{\pgfqpoint{4.067282in}{1.186682in}}%
\pgfpathcurveto{\pgfqpoint{4.076490in}{1.177473in}}{\pgfqpoint{4.088981in}{1.172299in}}{\pgfqpoint{4.102004in}{1.172299in}}%
\pgfpathlineto{\pgfqpoint{4.102004in}{1.172299in}}%
\pgfpathclose%
\pgfusepath{stroke,fill}%
\end{pgfscope}%
\begin{pgfscope}%
\pgfpathrectangle{\pgfqpoint{0.786164in}{0.768110in}}{\pgfqpoint{8.851069in}{7.081890in}}%
\pgfusepath{clip}%
\pgfsetbuttcap%
\pgfsetroundjoin%
\definecolor{currentfill}{rgb}{0.259857,0.745492,0.444467}%
\pgfsetfillcolor{currentfill}%
\pgfsetfillopacity{0.700000}%
\pgfsetlinewidth{0.501875pt}%
\definecolor{currentstroke}{rgb}{1.000000,1.000000,1.000000}%
\pgfsetstrokecolor{currentstroke}%
\pgfsetstrokeopacity{0.700000}%
\pgfsetdash{}{0pt}%
\pgfpathmoveto{\pgfqpoint{4.576935in}{1.237994in}}%
\pgfpathcurveto{\pgfqpoint{4.589958in}{1.237994in}}{\pgfqpoint{4.602449in}{1.243168in}}{\pgfqpoint{4.611657in}{1.252376in}}%
\pgfpathcurveto{\pgfqpoint{4.620866in}{1.261585in}}{\pgfqpoint{4.626039in}{1.274076in}}{\pgfqpoint{4.626039in}{1.287099in}}%
\pgfpathcurveto{\pgfqpoint{4.626039in}{1.300121in}}{\pgfqpoint{4.620866in}{1.312612in}}{\pgfqpoint{4.611657in}{1.321821in}}%
\pgfpathcurveto{\pgfqpoint{4.602449in}{1.331029in}}{\pgfqpoint{4.589958in}{1.336203in}}{\pgfqpoint{4.576935in}{1.336203in}}%
\pgfpathcurveto{\pgfqpoint{4.563912in}{1.336203in}}{\pgfqpoint{4.551421in}{1.331029in}}{\pgfqpoint{4.542213in}{1.321821in}}%
\pgfpathcurveto{\pgfqpoint{4.533004in}{1.312612in}}{\pgfqpoint{4.527830in}{1.300121in}}{\pgfqpoint{4.527830in}{1.287099in}}%
\pgfpathcurveto{\pgfqpoint{4.527830in}{1.274076in}}{\pgfqpoint{4.533004in}{1.261585in}}{\pgfqpoint{4.542213in}{1.252376in}}%
\pgfpathcurveto{\pgfqpoint{4.551421in}{1.243168in}}{\pgfqpoint{4.563912in}{1.237994in}}{\pgfqpoint{4.576935in}{1.237994in}}%
\pgfpathlineto{\pgfqpoint{4.576935in}{1.237994in}}%
\pgfpathclose%
\pgfusepath{stroke,fill}%
\end{pgfscope}%
\begin{pgfscope}%
\pgfpathrectangle{\pgfqpoint{0.786164in}{0.768110in}}{\pgfqpoint{8.851069in}{7.081890in}}%
\pgfusepath{clip}%
\pgfsetbuttcap%
\pgfsetroundjoin%
\definecolor{currentfill}{rgb}{0.304148,0.764704,0.419943}%
\pgfsetfillcolor{currentfill}%
\pgfsetfillopacity{0.700000}%
\pgfsetlinewidth{0.501875pt}%
\definecolor{currentstroke}{rgb}{1.000000,1.000000,1.000000}%
\pgfsetstrokecolor{currentstroke}%
\pgfsetstrokeopacity{0.700000}%
\pgfsetdash{}{0pt}%
\pgfpathmoveto{\pgfqpoint{4.467335in}{1.172299in}}%
\pgfpathcurveto{\pgfqpoint{4.480358in}{1.172299in}}{\pgfqpoint{4.492849in}{1.177473in}}{\pgfqpoint{4.502058in}{1.186682in}}%
\pgfpathcurveto{\pgfqpoint{4.511266in}{1.195890in}}{\pgfqpoint{4.516440in}{1.208381in}}{\pgfqpoint{4.516440in}{1.221404in}}%
\pgfpathcurveto{\pgfqpoint{4.516440in}{1.234427in}}{\pgfqpoint{4.511266in}{1.246918in}}{\pgfqpoint{4.502058in}{1.256126in}}%
\pgfpathcurveto{\pgfqpoint{4.492849in}{1.265335in}}{\pgfqpoint{4.480358in}{1.270509in}}{\pgfqpoint{4.467335in}{1.270509in}}%
\pgfpathcurveto{\pgfqpoint{4.454313in}{1.270509in}}{\pgfqpoint{4.441822in}{1.265335in}}{\pgfqpoint{4.432613in}{1.256126in}}%
\pgfpathcurveto{\pgfqpoint{4.423405in}{1.246918in}}{\pgfqpoint{4.418231in}{1.234427in}}{\pgfqpoint{4.418231in}{1.221404in}}%
\pgfpathcurveto{\pgfqpoint{4.418231in}{1.208381in}}{\pgfqpoint{4.423405in}{1.195890in}}{\pgfqpoint{4.432613in}{1.186682in}}%
\pgfpathcurveto{\pgfqpoint{4.441822in}{1.177473in}}{\pgfqpoint{4.454313in}{1.172299in}}{\pgfqpoint{4.467335in}{1.172299in}}%
\pgfpathlineto{\pgfqpoint{4.467335in}{1.172299in}}%
\pgfpathclose%
\pgfusepath{stroke,fill}%
\end{pgfscope}%
\begin{pgfscope}%
\pgfpathrectangle{\pgfqpoint{0.786164in}{0.768110in}}{\pgfqpoint{8.851069in}{7.081890in}}%
\pgfusepath{clip}%
\pgfsetbuttcap%
\pgfsetroundjoin%
\definecolor{currentfill}{rgb}{0.360741,0.785964,0.387814}%
\pgfsetfillcolor{currentfill}%
\pgfsetfillopacity{0.700000}%
\pgfsetlinewidth{0.501875pt}%
\definecolor{currentstroke}{rgb}{1.000000,1.000000,1.000000}%
\pgfsetstrokecolor{currentstroke}%
\pgfsetstrokeopacity{0.700000}%
\pgfsetdash{}{0pt}%
\pgfpathmoveto{\pgfqpoint{4.421669in}{1.106605in}}%
\pgfpathcurveto{\pgfqpoint{4.434692in}{1.106605in}}{\pgfqpoint{4.447183in}{1.111779in}}{\pgfqpoint{4.456391in}{1.120987in}}%
\pgfpathcurveto{\pgfqpoint{4.465600in}{1.130195in}}{\pgfqpoint{4.470774in}{1.142686in}}{\pgfqpoint{4.470774in}{1.155709in}}%
\pgfpathcurveto{\pgfqpoint{4.470774in}{1.168732in}}{\pgfqpoint{4.465600in}{1.181223in}}{\pgfqpoint{4.456391in}{1.190431in}}%
\pgfpathcurveto{\pgfqpoint{4.447183in}{1.199640in}}{\pgfqpoint{4.434692in}{1.204814in}}{\pgfqpoint{4.421669in}{1.204814in}}%
\pgfpathcurveto{\pgfqpoint{4.408646in}{1.204814in}}{\pgfqpoint{4.396155in}{1.199640in}}{\pgfqpoint{4.386947in}{1.190431in}}%
\pgfpathcurveto{\pgfqpoint{4.377738in}{1.181223in}}{\pgfqpoint{4.372564in}{1.168732in}}{\pgfqpoint{4.372564in}{1.155709in}}%
\pgfpathcurveto{\pgfqpoint{4.372564in}{1.142686in}}{\pgfqpoint{4.377738in}{1.130195in}}{\pgfqpoint{4.386947in}{1.120987in}}%
\pgfpathcurveto{\pgfqpoint{4.396155in}{1.111779in}}{\pgfqpoint{4.408646in}{1.106605in}}{\pgfqpoint{4.421669in}{1.106605in}}%
\pgfpathlineto{\pgfqpoint{4.421669in}{1.106605in}}%
\pgfpathclose%
\pgfusepath{stroke,fill}%
\end{pgfscope}%
\begin{pgfscope}%
\pgfpathrectangle{\pgfqpoint{0.786164in}{0.768110in}}{\pgfqpoint{8.851069in}{7.081890in}}%
\pgfusepath{clip}%
\pgfsetbuttcap%
\pgfsetroundjoin%
\definecolor{currentfill}{rgb}{0.386433,0.794644,0.372886}%
\pgfsetfillcolor{currentfill}%
\pgfsetfillopacity{0.700000}%
\pgfsetlinewidth{0.501875pt}%
\definecolor{currentstroke}{rgb}{1.000000,1.000000,1.000000}%
\pgfsetstrokecolor{currentstroke}%
\pgfsetstrokeopacity{0.700000}%
\pgfsetdash{}{0pt}%
\pgfpathmoveto{\pgfqpoint{4.293803in}{1.040910in}}%
\pgfpathcurveto{\pgfqpoint{4.306826in}{1.040910in}}{\pgfqpoint{4.319317in}{1.046084in}}{\pgfqpoint{4.328525in}{1.055292in}}%
\pgfpathcurveto{\pgfqpoint{4.337734in}{1.064501in}}{\pgfqpoint{4.342908in}{1.076992in}}{\pgfqpoint{4.342908in}{1.090014in}}%
\pgfpathcurveto{\pgfqpoint{4.342908in}{1.103037in}}{\pgfqpoint{4.337734in}{1.115528in}}{\pgfqpoint{4.328525in}{1.124737in}}%
\pgfpathcurveto{\pgfqpoint{4.319317in}{1.133945in}}{\pgfqpoint{4.306826in}{1.139119in}}{\pgfqpoint{4.293803in}{1.139119in}}%
\pgfpathcurveto{\pgfqpoint{4.280780in}{1.139119in}}{\pgfqpoint{4.268289in}{1.133945in}}{\pgfqpoint{4.259081in}{1.124737in}}%
\pgfpathcurveto{\pgfqpoint{4.249872in}{1.115528in}}{\pgfqpoint{4.244698in}{1.103037in}}{\pgfqpoint{4.244698in}{1.090014in}}%
\pgfpathcurveto{\pgfqpoint{4.244698in}{1.076992in}}{\pgfqpoint{4.249872in}{1.064501in}}{\pgfqpoint{4.259081in}{1.055292in}}%
\pgfpathcurveto{\pgfqpoint{4.268289in}{1.046084in}}{\pgfqpoint{4.280780in}{1.040910in}}{\pgfqpoint{4.293803in}{1.040910in}}%
\pgfpathlineto{\pgfqpoint{4.293803in}{1.040910in}}%
\pgfpathclose%
\pgfusepath{stroke,fill}%
\end{pgfscope}%
\begin{pgfscope}%
\pgfpathrectangle{\pgfqpoint{0.786164in}{0.768110in}}{\pgfqpoint{8.851069in}{7.081890in}}%
\pgfusepath{clip}%
\pgfsetbuttcap%
\pgfsetroundjoin%
\definecolor{currentfill}{rgb}{0.421908,0.805774,0.351910}%
\pgfsetfillcolor{currentfill}%
\pgfsetfillopacity{0.700000}%
\pgfsetlinewidth{0.501875pt}%
\definecolor{currentstroke}{rgb}{1.000000,1.000000,1.000000}%
\pgfsetstrokecolor{currentstroke}%
\pgfsetstrokeopacity{0.700000}%
\pgfsetdash{}{0pt}%
\pgfpathmoveto{\pgfqpoint{4.202470in}{1.040910in}}%
\pgfpathcurveto{\pgfqpoint{4.215493in}{1.040910in}}{\pgfqpoint{4.227984in}{1.046084in}}{\pgfqpoint{4.237192in}{1.055292in}}%
\pgfpathcurveto{\pgfqpoint{4.246401in}{1.064501in}}{\pgfqpoint{4.251575in}{1.076992in}}{\pgfqpoint{4.251575in}{1.090014in}}%
\pgfpathcurveto{\pgfqpoint{4.251575in}{1.103037in}}{\pgfqpoint{4.246401in}{1.115528in}}{\pgfqpoint{4.237192in}{1.124737in}}%
\pgfpathcurveto{\pgfqpoint{4.227984in}{1.133945in}}{\pgfqpoint{4.215493in}{1.139119in}}{\pgfqpoint{4.202470in}{1.139119in}}%
\pgfpathcurveto{\pgfqpoint{4.189447in}{1.139119in}}{\pgfqpoint{4.176956in}{1.133945in}}{\pgfqpoint{4.167748in}{1.124737in}}%
\pgfpathcurveto{\pgfqpoint{4.158539in}{1.115528in}}{\pgfqpoint{4.153365in}{1.103037in}}{\pgfqpoint{4.153365in}{1.090014in}}%
\pgfpathcurveto{\pgfqpoint{4.153365in}{1.076992in}}{\pgfqpoint{4.158539in}{1.064501in}}{\pgfqpoint{4.167748in}{1.055292in}}%
\pgfpathcurveto{\pgfqpoint{4.176956in}{1.046084in}}{\pgfqpoint{4.189447in}{1.040910in}}{\pgfqpoint{4.202470in}{1.040910in}}%
\pgfpathlineto{\pgfqpoint{4.202470in}{1.040910in}}%
\pgfpathclose%
\pgfusepath{stroke,fill}%
\end{pgfscope}%
\begin{pgfscope}%
\pgfpathrectangle{\pgfqpoint{0.786164in}{0.768110in}}{\pgfqpoint{8.851069in}{7.081890in}}%
\pgfusepath{clip}%
\pgfsetbuttcap%
\pgfsetroundjoin%
\definecolor{currentfill}{rgb}{0.458674,0.816363,0.329727}%
\pgfsetfillcolor{currentfill}%
\pgfsetfillopacity{0.700000}%
\pgfsetlinewidth{0.501875pt}%
\definecolor{currentstroke}{rgb}{1.000000,1.000000,1.000000}%
\pgfsetstrokecolor{currentstroke}%
\pgfsetstrokeopacity{0.700000}%
\pgfsetdash{}{0pt}%
\pgfpathmoveto{\pgfqpoint{3.955871in}{1.084706in}}%
\pgfpathcurveto{\pgfqpoint{3.968894in}{1.084706in}}{\pgfqpoint{3.981385in}{1.089880in}}{\pgfqpoint{3.990594in}{1.099089in}}%
\pgfpathcurveto{\pgfqpoint{3.999802in}{1.108297in}}{\pgfqpoint{4.004976in}{1.120788in}}{\pgfqpoint{4.004976in}{1.133811in}}%
\pgfpathcurveto{\pgfqpoint{4.004976in}{1.146834in}}{\pgfqpoint{3.999802in}{1.159325in}}{\pgfqpoint{3.990594in}{1.168533in}}%
\pgfpathcurveto{\pgfqpoint{3.981385in}{1.177742in}}{\pgfqpoint{3.968894in}{1.182916in}}{\pgfqpoint{3.955871in}{1.182916in}}%
\pgfpathcurveto{\pgfqpoint{3.942849in}{1.182916in}}{\pgfqpoint{3.930358in}{1.177742in}}{\pgfqpoint{3.921149in}{1.168533in}}%
\pgfpathcurveto{\pgfqpoint{3.911941in}{1.159325in}}{\pgfqpoint{3.906767in}{1.146834in}}{\pgfqpoint{3.906767in}{1.133811in}}%
\pgfpathcurveto{\pgfqpoint{3.906767in}{1.120788in}}{\pgfqpoint{3.911941in}{1.108297in}}{\pgfqpoint{3.921149in}{1.099089in}}%
\pgfpathcurveto{\pgfqpoint{3.930358in}{1.089880in}}{\pgfqpoint{3.942849in}{1.084706in}}{\pgfqpoint{3.955871in}{1.084706in}}%
\pgfpathlineto{\pgfqpoint{3.955871in}{1.084706in}}%
\pgfpathclose%
\pgfusepath{stroke,fill}%
\end{pgfscope}%
\begin{pgfscope}%
\pgfpathrectangle{\pgfqpoint{0.786164in}{0.768110in}}{\pgfqpoint{8.851069in}{7.081890in}}%
\pgfusepath{clip}%
\pgfsetbuttcap%
\pgfsetroundjoin%
\definecolor{currentfill}{rgb}{0.468053,0.818921,0.323998}%
\pgfsetfillcolor{currentfill}%
\pgfsetfillopacity{0.700000}%
\pgfsetlinewidth{0.501875pt}%
\definecolor{currentstroke}{rgb}{1.000000,1.000000,1.000000}%
\pgfsetstrokecolor{currentstroke}%
\pgfsetstrokeopacity{0.700000}%
\pgfsetdash{}{0pt}%
\pgfpathmoveto{\pgfqpoint{4.047204in}{1.128503in}}%
\pgfpathcurveto{\pgfqpoint{4.060227in}{1.128503in}}{\pgfqpoint{4.072718in}{1.133677in}}{\pgfqpoint{4.081926in}{1.142885in}}%
\pgfpathcurveto{\pgfqpoint{4.091135in}{1.152094in}}{\pgfqpoint{4.096309in}{1.164585in}}{\pgfqpoint{4.096309in}{1.177607in}}%
\pgfpathcurveto{\pgfqpoint{4.096309in}{1.190630in}}{\pgfqpoint{4.091135in}{1.203121in}}{\pgfqpoint{4.081926in}{1.212330in}}%
\pgfpathcurveto{\pgfqpoint{4.072718in}{1.221538in}}{\pgfqpoint{4.060227in}{1.226712in}}{\pgfqpoint{4.047204in}{1.226712in}}%
\pgfpathcurveto{\pgfqpoint{4.034181in}{1.226712in}}{\pgfqpoint{4.021690in}{1.221538in}}{\pgfqpoint{4.012482in}{1.212330in}}%
\pgfpathcurveto{\pgfqpoint{4.003274in}{1.203121in}}{\pgfqpoint{3.998100in}{1.190630in}}{\pgfqpoint{3.998100in}{1.177607in}}%
\pgfpathcurveto{\pgfqpoint{3.998100in}{1.164585in}}{\pgfqpoint{4.003274in}{1.152094in}}{\pgfqpoint{4.012482in}{1.142885in}}%
\pgfpathcurveto{\pgfqpoint{4.021690in}{1.133677in}}{\pgfqpoint{4.034181in}{1.128503in}}{\pgfqpoint{4.047204in}{1.128503in}}%
\pgfpathlineto{\pgfqpoint{4.047204in}{1.128503in}}%
\pgfpathclose%
\pgfusepath{stroke,fill}%
\end{pgfscope}%
\begin{pgfscope}%
\pgfpathrectangle{\pgfqpoint{0.786164in}{0.768110in}}{\pgfqpoint{8.851069in}{7.081890in}}%
\pgfusepath{clip}%
\pgfsetbuttcap%
\pgfsetroundjoin%
\definecolor{currentfill}{rgb}{0.506271,0.828786,0.300362}%
\pgfsetfillcolor{currentfill}%
\pgfsetfillopacity{0.700000}%
\pgfsetlinewidth{0.501875pt}%
\definecolor{currentstroke}{rgb}{1.000000,1.000000,1.000000}%
\pgfsetstrokecolor{currentstroke}%
\pgfsetstrokeopacity{0.700000}%
\pgfsetdash{}{0pt}%
\pgfpathmoveto{\pgfqpoint{4.083737in}{1.237994in}}%
\pgfpathcurveto{\pgfqpoint{4.096760in}{1.237994in}}{\pgfqpoint{4.109251in}{1.243168in}}{\pgfqpoint{4.118460in}{1.252376in}}%
\pgfpathcurveto{\pgfqpoint{4.127668in}{1.261585in}}{\pgfqpoint{4.132842in}{1.274076in}}{\pgfqpoint{4.132842in}{1.287099in}}%
\pgfpathcurveto{\pgfqpoint{4.132842in}{1.300121in}}{\pgfqpoint{4.127668in}{1.312612in}}{\pgfqpoint{4.118460in}{1.321821in}}%
\pgfpathcurveto{\pgfqpoint{4.109251in}{1.331029in}}{\pgfqpoint{4.096760in}{1.336203in}}{\pgfqpoint{4.083737in}{1.336203in}}%
\pgfpathcurveto{\pgfqpoint{4.070715in}{1.336203in}}{\pgfqpoint{4.058224in}{1.331029in}}{\pgfqpoint{4.049015in}{1.321821in}}%
\pgfpathcurveto{\pgfqpoint{4.039807in}{1.312612in}}{\pgfqpoint{4.034633in}{1.300121in}}{\pgfqpoint{4.034633in}{1.287099in}}%
\pgfpathcurveto{\pgfqpoint{4.034633in}{1.274076in}}{\pgfqpoint{4.039807in}{1.261585in}}{\pgfqpoint{4.049015in}{1.252376in}}%
\pgfpathcurveto{\pgfqpoint{4.058224in}{1.243168in}}{\pgfqpoint{4.070715in}{1.237994in}}{\pgfqpoint{4.083737in}{1.237994in}}%
\pgfpathlineto{\pgfqpoint{4.083737in}{1.237994in}}%
\pgfpathclose%
\pgfusepath{stroke,fill}%
\end{pgfscope}%
\begin{pgfscope}%
\pgfpathrectangle{\pgfqpoint{0.786164in}{0.768110in}}{\pgfqpoint{8.851069in}{7.081890in}}%
\pgfusepath{clip}%
\pgfsetbuttcap%
\pgfsetroundjoin%
\definecolor{currentfill}{rgb}{0.525776,0.833491,0.288127}%
\pgfsetfillcolor{currentfill}%
\pgfsetfillopacity{0.700000}%
\pgfsetlinewidth{0.501875pt}%
\definecolor{currentstroke}{rgb}{1.000000,1.000000,1.000000}%
\pgfsetstrokecolor{currentstroke}%
\pgfsetstrokeopacity{0.700000}%
\pgfsetdash{}{0pt}%
\pgfpathmoveto{\pgfqpoint{4.111137in}{1.369383in}}%
\pgfpathcurveto{\pgfqpoint{4.124160in}{1.369383in}}{\pgfqpoint{4.136651in}{1.374557in}}{\pgfqpoint{4.145859in}{1.383766in}}%
\pgfpathcurveto{\pgfqpoint{4.155068in}{1.392974in}}{\pgfqpoint{4.160242in}{1.405465in}}{\pgfqpoint{4.160242in}{1.418488in}}%
\pgfpathcurveto{\pgfqpoint{4.160242in}{1.431511in}}{\pgfqpoint{4.155068in}{1.444002in}}{\pgfqpoint{4.145859in}{1.453210in}}%
\pgfpathcurveto{\pgfqpoint{4.136651in}{1.462419in}}{\pgfqpoint{4.124160in}{1.467593in}}{\pgfqpoint{4.111137in}{1.467593in}}%
\pgfpathcurveto{\pgfqpoint{4.098114in}{1.467593in}}{\pgfqpoint{4.085623in}{1.462419in}}{\pgfqpoint{4.076415in}{1.453210in}}%
\pgfpathcurveto{\pgfqpoint{4.067207in}{1.444002in}}{\pgfqpoint{4.062033in}{1.431511in}}{\pgfqpoint{4.062033in}{1.418488in}}%
\pgfpathcurveto{\pgfqpoint{4.062033in}{1.405465in}}{\pgfqpoint{4.067207in}{1.392974in}}{\pgfqpoint{4.076415in}{1.383766in}}%
\pgfpathcurveto{\pgfqpoint{4.085623in}{1.374557in}}{\pgfqpoint{4.098114in}{1.369383in}}{\pgfqpoint{4.111137in}{1.369383in}}%
\pgfpathlineto{\pgfqpoint{4.111137in}{1.369383in}}%
\pgfpathclose%
\pgfusepath{stroke,fill}%
\end{pgfscope}%
\begin{pgfscope}%
\pgfpathrectangle{\pgfqpoint{0.786164in}{0.768110in}}{\pgfqpoint{8.851069in}{7.081890in}}%
\pgfusepath{clip}%
\pgfsetbuttcap%
\pgfsetroundjoin%
\definecolor{currentfill}{rgb}{0.595839,0.848717,0.243329}%
\pgfsetfillcolor{currentfill}%
\pgfsetfillopacity{0.700000}%
\pgfsetlinewidth{0.501875pt}%
\definecolor{currentstroke}{rgb}{1.000000,1.000000,1.000000}%
\pgfsetstrokecolor{currentstroke}%
\pgfsetstrokeopacity{0.700000}%
\pgfsetdash{}{0pt}%
\pgfpathmoveto{\pgfqpoint{3.937605in}{1.281790in}}%
\pgfpathcurveto{\pgfqpoint{3.950627in}{1.281790in}}{\pgfqpoint{3.963119in}{1.286964in}}{\pgfqpoint{3.972327in}{1.296173in}}%
\pgfpathcurveto{\pgfqpoint{3.981535in}{1.305381in}}{\pgfqpoint{3.986709in}{1.317872in}}{\pgfqpoint{3.986709in}{1.330895in}}%
\pgfpathcurveto{\pgfqpoint{3.986709in}{1.343918in}}{\pgfqpoint{3.981535in}{1.356409in}}{\pgfqpoint{3.972327in}{1.365617in}}%
\pgfpathcurveto{\pgfqpoint{3.963119in}{1.374826in}}{\pgfqpoint{3.950627in}{1.380000in}}{\pgfqpoint{3.937605in}{1.380000in}}%
\pgfpathcurveto{\pgfqpoint{3.924582in}{1.380000in}}{\pgfqpoint{3.912091in}{1.374826in}}{\pgfqpoint{3.902883in}{1.365617in}}%
\pgfpathcurveto{\pgfqpoint{3.893674in}{1.356409in}}{\pgfqpoint{3.888500in}{1.343918in}}{\pgfqpoint{3.888500in}{1.330895in}}%
\pgfpathcurveto{\pgfqpoint{3.888500in}{1.317872in}}{\pgfqpoint{3.893674in}{1.305381in}}{\pgfqpoint{3.902883in}{1.296173in}}%
\pgfpathcurveto{\pgfqpoint{3.912091in}{1.286964in}}{\pgfqpoint{3.924582in}{1.281790in}}{\pgfqpoint{3.937605in}{1.281790in}}%
\pgfpathlineto{\pgfqpoint{3.937605in}{1.281790in}}%
\pgfpathclose%
\pgfusepath{stroke,fill}%
\end{pgfscope}%
\begin{pgfscope}%
\pgfpathrectangle{\pgfqpoint{0.786164in}{0.768110in}}{\pgfqpoint{8.851069in}{7.081890in}}%
\pgfusepath{clip}%
\pgfsetbuttcap%
\pgfsetroundjoin%
\definecolor{currentfill}{rgb}{0.772852,0.877868,0.131109}%
\pgfsetfillcolor{currentfill}%
\pgfsetfillopacity{0.700000}%
\pgfsetlinewidth{0.501875pt}%
\definecolor{currentstroke}{rgb}{1.000000,1.000000,1.000000}%
\pgfsetstrokecolor{currentstroke}%
\pgfsetstrokeopacity{0.700000}%
\pgfsetdash{}{0pt}%
\pgfpathmoveto{\pgfqpoint{3.517474in}{1.106605in}}%
\pgfpathcurveto{\pgfqpoint{3.530496in}{1.106605in}}{\pgfqpoint{3.542987in}{1.111779in}}{\pgfqpoint{3.552196in}{1.120987in}}%
\pgfpathcurveto{\pgfqpoint{3.561404in}{1.130195in}}{\pgfqpoint{3.566578in}{1.142686in}}{\pgfqpoint{3.566578in}{1.155709in}}%
\pgfpathcurveto{\pgfqpoint{3.566578in}{1.168732in}}{\pgfqpoint{3.561404in}{1.181223in}}{\pgfqpoint{3.552196in}{1.190431in}}%
\pgfpathcurveto{\pgfqpoint{3.542987in}{1.199640in}}{\pgfqpoint{3.530496in}{1.204814in}}{\pgfqpoint{3.517474in}{1.204814in}}%
\pgfpathcurveto{\pgfqpoint{3.504451in}{1.204814in}}{\pgfqpoint{3.491960in}{1.199640in}}{\pgfqpoint{3.482751in}{1.190431in}}%
\pgfpathcurveto{\pgfqpoint{3.473543in}{1.181223in}}{\pgfqpoint{3.468369in}{1.168732in}}{\pgfqpoint{3.468369in}{1.155709in}}%
\pgfpathcurveto{\pgfqpoint{3.468369in}{1.142686in}}{\pgfqpoint{3.473543in}{1.130195in}}{\pgfqpoint{3.482751in}{1.120987in}}%
\pgfpathcurveto{\pgfqpoint{3.491960in}{1.111779in}}{\pgfqpoint{3.504451in}{1.106605in}}{\pgfqpoint{3.517474in}{1.106605in}}%
\pgfpathlineto{\pgfqpoint{3.517474in}{1.106605in}}%
\pgfpathclose%
\pgfusepath{stroke,fill}%
\end{pgfscope}%
\begin{pgfscope}%
\pgfpathrectangle{\pgfqpoint{0.786164in}{0.768110in}}{\pgfqpoint{8.851069in}{7.081890in}}%
\pgfusepath{clip}%
\pgfsetbuttcap%
\pgfsetroundjoin%
\definecolor{currentfill}{rgb}{0.876168,0.891125,0.095250}%
\pgfsetfillcolor{currentfill}%
\pgfsetfillopacity{0.700000}%
\pgfsetlinewidth{0.501875pt}%
\definecolor{currentstroke}{rgb}{1.000000,1.000000,1.000000}%
\pgfsetstrokecolor{currentstroke}%
\pgfsetstrokeopacity{0.700000}%
\pgfsetdash{}{0pt}%
\pgfpathmoveto{\pgfqpoint{3.672739in}{1.128503in}}%
\pgfpathcurveto{\pgfqpoint{3.685762in}{1.128503in}}{\pgfqpoint{3.698253in}{1.133677in}}{\pgfqpoint{3.707462in}{1.142885in}}%
\pgfpathcurveto{\pgfqpoint{3.716670in}{1.152094in}}{\pgfqpoint{3.721844in}{1.164585in}}{\pgfqpoint{3.721844in}{1.177607in}}%
\pgfpathcurveto{\pgfqpoint{3.721844in}{1.190630in}}{\pgfqpoint{3.716670in}{1.203121in}}{\pgfqpoint{3.707462in}{1.212330in}}%
\pgfpathcurveto{\pgfqpoint{3.698253in}{1.221538in}}{\pgfqpoint{3.685762in}{1.226712in}}{\pgfqpoint{3.672739in}{1.226712in}}%
\pgfpathcurveto{\pgfqpoint{3.659717in}{1.226712in}}{\pgfqpoint{3.647226in}{1.221538in}}{\pgfqpoint{3.638017in}{1.212330in}}%
\pgfpathcurveto{\pgfqpoint{3.628809in}{1.203121in}}{\pgfqpoint{3.623635in}{1.190630in}}{\pgfqpoint{3.623635in}{1.177607in}}%
\pgfpathcurveto{\pgfqpoint{3.623635in}{1.164585in}}{\pgfqpoint{3.628809in}{1.152094in}}{\pgfqpoint{3.638017in}{1.142885in}}%
\pgfpathcurveto{\pgfqpoint{3.647226in}{1.133677in}}{\pgfqpoint{3.659717in}{1.128503in}}{\pgfqpoint{3.672739in}{1.128503in}}%
\pgfpathlineto{\pgfqpoint{3.672739in}{1.128503in}}%
\pgfpathclose%
\pgfusepath{stroke,fill}%
\end{pgfscope}%
\begin{pgfscope}%
\pgfpathrectangle{\pgfqpoint{0.786164in}{0.768110in}}{\pgfqpoint{8.851069in}{7.081890in}}%
\pgfusepath{clip}%
\pgfsetbuttcap%
\pgfsetroundjoin%
\definecolor{currentfill}{rgb}{0.993248,0.906157,0.143936}%
\pgfsetfillcolor{currentfill}%
\pgfsetfillopacity{0.700000}%
\pgfsetlinewidth{0.501875pt}%
\definecolor{currentstroke}{rgb}{1.000000,1.000000,1.000000}%
\pgfsetstrokecolor{currentstroke}%
\pgfsetstrokeopacity{0.700000}%
\pgfsetdash{}{0pt}%
\pgfpathmoveto{\pgfqpoint{3.572273in}{1.150401in}}%
\pgfpathcurveto{\pgfqpoint{3.585296in}{1.150401in}}{\pgfqpoint{3.597787in}{1.155575in}}{\pgfqpoint{3.606995in}{1.164783in}}%
\pgfpathcurveto{\pgfqpoint{3.616204in}{1.173992in}}{\pgfqpoint{3.621378in}{1.186483in}}{\pgfqpoint{3.621378in}{1.199506in}}%
\pgfpathcurveto{\pgfqpoint{3.621378in}{1.212528in}}{\pgfqpoint{3.616204in}{1.225019in}}{\pgfqpoint{3.606995in}{1.234228in}}%
\pgfpathcurveto{\pgfqpoint{3.597787in}{1.243436in}}{\pgfqpoint{3.585296in}{1.248610in}}{\pgfqpoint{3.572273in}{1.248610in}}%
\pgfpathcurveto{\pgfqpoint{3.559251in}{1.248610in}}{\pgfqpoint{3.546759in}{1.243436in}}{\pgfqpoint{3.537551in}{1.234228in}}%
\pgfpathcurveto{\pgfqpoint{3.528343in}{1.225019in}}{\pgfqpoint{3.523169in}{1.212528in}}{\pgfqpoint{3.523169in}{1.199506in}}%
\pgfpathcurveto{\pgfqpoint{3.523169in}{1.186483in}}{\pgfqpoint{3.528343in}{1.173992in}}{\pgfqpoint{3.537551in}{1.164783in}}%
\pgfpathcurveto{\pgfqpoint{3.546759in}{1.155575in}}{\pgfqpoint{3.559251in}{1.150401in}}{\pgfqpoint{3.572273in}{1.150401in}}%
\pgfpathlineto{\pgfqpoint{3.572273in}{1.150401in}}%
\pgfpathclose%
\pgfusepath{stroke,fill}%
\end{pgfscope}%
\begin{pgfscope}%
\pgfpathrectangle{\pgfqpoint{0.786164in}{0.768110in}}{\pgfqpoint{8.851069in}{7.081890in}}%
\pgfusepath{clip}%
\pgfsetbuttcap%
\pgfsetroundjoin%
\definecolor{currentfill}{rgb}{0.216210,0.351535,0.550627}%
\pgfsetfillcolor{currentfill}%
\pgfsetfillopacity{0.700000}%
\pgfsetlinewidth{0.501875pt}%
\definecolor{currentstroke}{rgb}{1.000000,1.000000,1.000000}%
\pgfsetstrokecolor{currentstroke}%
\pgfsetstrokeopacity{0.700000}%
\pgfsetdash{}{0pt}%
\pgfpathmoveto{\pgfqpoint{3.069942in}{2.464295in}}%
\pgfpathcurveto{\pgfqpoint{3.082965in}{2.464295in}}{\pgfqpoint{3.095456in}{2.469469in}}{\pgfqpoint{3.104665in}{2.478678in}}%
\pgfpathcurveto{\pgfqpoint{3.113873in}{2.487886in}}{\pgfqpoint{3.119047in}{2.500377in}}{\pgfqpoint{3.119047in}{2.513400in}}%
\pgfpathcurveto{\pgfqpoint{3.119047in}{2.526423in}}{\pgfqpoint{3.113873in}{2.538914in}}{\pgfqpoint{3.104665in}{2.548122in}}%
\pgfpathcurveto{\pgfqpoint{3.095456in}{2.557331in}}{\pgfqpoint{3.082965in}{2.562504in}}{\pgfqpoint{3.069942in}{2.562504in}}%
\pgfpathcurveto{\pgfqpoint{3.056920in}{2.562504in}}{\pgfqpoint{3.044429in}{2.557331in}}{\pgfqpoint{3.035220in}{2.548122in}}%
\pgfpathcurveto{\pgfqpoint{3.026012in}{2.538914in}}{\pgfqpoint{3.020838in}{2.526423in}}{\pgfqpoint{3.020838in}{2.513400in}}%
\pgfpathcurveto{\pgfqpoint{3.020838in}{2.500377in}}{\pgfqpoint{3.026012in}{2.487886in}}{\pgfqpoint{3.035220in}{2.478678in}}%
\pgfpathcurveto{\pgfqpoint{3.044429in}{2.469469in}}{\pgfqpoint{3.056920in}{2.464295in}}{\pgfqpoint{3.069942in}{2.464295in}}%
\pgfpathlineto{\pgfqpoint{3.069942in}{2.464295in}}%
\pgfpathclose%
\pgfusepath{stroke,fill}%
\end{pgfscope}%
\begin{pgfscope}%
\pgfpathrectangle{\pgfqpoint{0.786164in}{0.768110in}}{\pgfqpoint{8.851069in}{7.081890in}}%
\pgfusepath{clip}%
\pgfsetbuttcap%
\pgfsetroundjoin%
\definecolor{currentfill}{rgb}{0.206756,0.371758,0.553117}%
\pgfsetfillcolor{currentfill}%
\pgfsetfillopacity{0.700000}%
\pgfsetlinewidth{0.501875pt}%
\definecolor{currentstroke}{rgb}{1.000000,1.000000,1.000000}%
\pgfsetstrokecolor{currentstroke}%
\pgfsetstrokeopacity{0.700000}%
\pgfsetdash{}{0pt}%
\pgfpathmoveto{\pgfqpoint{3.179542in}{2.486193in}}%
\pgfpathcurveto{\pgfqpoint{3.192565in}{2.486193in}}{\pgfqpoint{3.205056in}{2.491367in}}{\pgfqpoint{3.214264in}{2.500576in}}%
\pgfpathcurveto{\pgfqpoint{3.223473in}{2.509784in}}{\pgfqpoint{3.228647in}{2.522275in}}{\pgfqpoint{3.228647in}{2.535298in}}%
\pgfpathcurveto{\pgfqpoint{3.228647in}{2.548321in}}{\pgfqpoint{3.223473in}{2.560812in}}{\pgfqpoint{3.214264in}{2.570020in}}%
\pgfpathcurveto{\pgfqpoint{3.205056in}{2.579229in}}{\pgfqpoint{3.192565in}{2.584403in}}{\pgfqpoint{3.179542in}{2.584403in}}%
\pgfpathcurveto{\pgfqpoint{3.166519in}{2.584403in}}{\pgfqpoint{3.154028in}{2.579229in}}{\pgfqpoint{3.144820in}{2.570020in}}%
\pgfpathcurveto{\pgfqpoint{3.135611in}{2.560812in}}{\pgfqpoint{3.130437in}{2.548321in}}{\pgfqpoint{3.130437in}{2.535298in}}%
\pgfpathcurveto{\pgfqpoint{3.130437in}{2.522275in}}{\pgfqpoint{3.135611in}{2.509784in}}{\pgfqpoint{3.144820in}{2.500576in}}%
\pgfpathcurveto{\pgfqpoint{3.154028in}{2.491367in}}{\pgfqpoint{3.166519in}{2.486193in}}{\pgfqpoint{3.179542in}{2.486193in}}%
\pgfpathlineto{\pgfqpoint{3.179542in}{2.486193in}}%
\pgfpathclose%
\pgfusepath{stroke,fill}%
\end{pgfscope}%
\begin{pgfscope}%
\pgfpathrectangle{\pgfqpoint{0.786164in}{0.768110in}}{\pgfqpoint{8.851069in}{7.081890in}}%
\pgfusepath{clip}%
\pgfsetbuttcap%
\pgfsetroundjoin%
\definecolor{currentfill}{rgb}{0.216210,0.351535,0.550627}%
\pgfsetfillcolor{currentfill}%
\pgfsetfillopacity{0.700000}%
\pgfsetlinewidth{0.501875pt}%
\definecolor{currentstroke}{rgb}{1.000000,1.000000,1.000000}%
\pgfsetstrokecolor{currentstroke}%
\pgfsetstrokeopacity{0.700000}%
\pgfsetdash{}{0pt}%
\pgfpathmoveto{\pgfqpoint{3.143009in}{2.508092in}}%
\pgfpathcurveto{\pgfqpoint{3.156031in}{2.508092in}}{\pgfqpoint{3.168523in}{2.513266in}}{\pgfqpoint{3.177731in}{2.522474in}}%
\pgfpathcurveto{\pgfqpoint{3.186939in}{2.531683in}}{\pgfqpoint{3.192113in}{2.544174in}}{\pgfqpoint{3.192113in}{2.557196in}}%
\pgfpathcurveto{\pgfqpoint{3.192113in}{2.570219in}}{\pgfqpoint{3.186939in}{2.582710in}}{\pgfqpoint{3.177731in}{2.591919in}}%
\pgfpathcurveto{\pgfqpoint{3.168523in}{2.601127in}}{\pgfqpoint{3.156031in}{2.606301in}}{\pgfqpoint{3.143009in}{2.606301in}}%
\pgfpathcurveto{\pgfqpoint{3.129986in}{2.606301in}}{\pgfqpoint{3.117495in}{2.601127in}}{\pgfqpoint{3.108286in}{2.591919in}}%
\pgfpathcurveto{\pgfqpoint{3.099078in}{2.582710in}}{\pgfqpoint{3.093904in}{2.570219in}}{\pgfqpoint{3.093904in}{2.557196in}}%
\pgfpathcurveto{\pgfqpoint{3.093904in}{2.544174in}}{\pgfqpoint{3.099078in}{2.531683in}}{\pgfqpoint{3.108286in}{2.522474in}}%
\pgfpathcurveto{\pgfqpoint{3.117495in}{2.513266in}}{\pgfqpoint{3.129986in}{2.508092in}}{\pgfqpoint{3.143009in}{2.508092in}}%
\pgfpathlineto{\pgfqpoint{3.143009in}{2.508092in}}%
\pgfpathclose%
\pgfusepath{stroke,fill}%
\end{pgfscope}%
\begin{pgfscope}%
\pgfpathrectangle{\pgfqpoint{0.786164in}{0.768110in}}{\pgfqpoint{8.851069in}{7.081890in}}%
\pgfusepath{clip}%
\pgfsetbuttcap%
\pgfsetroundjoin%
\definecolor{currentfill}{rgb}{0.206756,0.371758,0.553117}%
\pgfsetfillcolor{currentfill}%
\pgfsetfillopacity{0.700000}%
\pgfsetlinewidth{0.501875pt}%
\definecolor{currentstroke}{rgb}{1.000000,1.000000,1.000000}%
\pgfsetstrokecolor{currentstroke}%
\pgfsetstrokeopacity{0.700000}%
\pgfsetdash{}{0pt}%
\pgfpathmoveto{\pgfqpoint{3.170409in}{2.486193in}}%
\pgfpathcurveto{\pgfqpoint{3.183431in}{2.486193in}}{\pgfqpoint{3.195922in}{2.491367in}}{\pgfqpoint{3.205131in}{2.500576in}}%
\pgfpathcurveto{\pgfqpoint{3.214339in}{2.509784in}}{\pgfqpoint{3.219513in}{2.522275in}}{\pgfqpoint{3.219513in}{2.535298in}}%
\pgfpathcurveto{\pgfqpoint{3.219513in}{2.548321in}}{\pgfqpoint{3.214339in}{2.560812in}}{\pgfqpoint{3.205131in}{2.570020in}}%
\pgfpathcurveto{\pgfqpoint{3.195922in}{2.579229in}}{\pgfqpoint{3.183431in}{2.584403in}}{\pgfqpoint{3.170409in}{2.584403in}}%
\pgfpathcurveto{\pgfqpoint{3.157386in}{2.584403in}}{\pgfqpoint{3.144895in}{2.579229in}}{\pgfqpoint{3.135686in}{2.570020in}}%
\pgfpathcurveto{\pgfqpoint{3.126478in}{2.560812in}}{\pgfqpoint{3.121304in}{2.548321in}}{\pgfqpoint{3.121304in}{2.535298in}}%
\pgfpathcurveto{\pgfqpoint{3.121304in}{2.522275in}}{\pgfqpoint{3.126478in}{2.509784in}}{\pgfqpoint{3.135686in}{2.500576in}}%
\pgfpathcurveto{\pgfqpoint{3.144895in}{2.491367in}}{\pgfqpoint{3.157386in}{2.486193in}}{\pgfqpoint{3.170409in}{2.486193in}}%
\pgfpathlineto{\pgfqpoint{3.170409in}{2.486193in}}%
\pgfpathclose%
\pgfusepath{stroke,fill}%
\end{pgfscope}%
\begin{pgfscope}%
\pgfpathrectangle{\pgfqpoint{0.786164in}{0.768110in}}{\pgfqpoint{8.851069in}{7.081890in}}%
\pgfusepath{clip}%
\pgfsetbuttcap%
\pgfsetroundjoin%
\definecolor{currentfill}{rgb}{0.214298,0.355619,0.551184}%
\pgfsetfillcolor{currentfill}%
\pgfsetfillopacity{0.700000}%
\pgfsetlinewidth{0.501875pt}%
\definecolor{currentstroke}{rgb}{1.000000,1.000000,1.000000}%
\pgfsetstrokecolor{currentstroke}%
\pgfsetstrokeopacity{0.700000}%
\pgfsetdash{}{0pt}%
\pgfpathmoveto{\pgfqpoint{3.362208in}{2.595685in}}%
\pgfpathcurveto{\pgfqpoint{3.375230in}{2.595685in}}{\pgfqpoint{3.387721in}{2.600859in}}{\pgfqpoint{3.396930in}{2.610067in}}%
\pgfpathcurveto{\pgfqpoint{3.406138in}{2.619275in}}{\pgfqpoint{3.411312in}{2.631767in}}{\pgfqpoint{3.411312in}{2.644789in}}%
\pgfpathcurveto{\pgfqpoint{3.411312in}{2.657812in}}{\pgfqpoint{3.406138in}{2.670303in}}{\pgfqpoint{3.396930in}{2.679511in}}%
\pgfpathcurveto{\pgfqpoint{3.387721in}{2.688720in}}{\pgfqpoint{3.375230in}{2.693894in}}{\pgfqpoint{3.362208in}{2.693894in}}%
\pgfpathcurveto{\pgfqpoint{3.349185in}{2.693894in}}{\pgfqpoint{3.336694in}{2.688720in}}{\pgfqpoint{3.327485in}{2.679511in}}%
\pgfpathcurveto{\pgfqpoint{3.318277in}{2.670303in}}{\pgfqpoint{3.313103in}{2.657812in}}{\pgfqpoint{3.313103in}{2.644789in}}%
\pgfpathcurveto{\pgfqpoint{3.313103in}{2.631767in}}{\pgfqpoint{3.318277in}{2.619275in}}{\pgfqpoint{3.327485in}{2.610067in}}%
\pgfpathcurveto{\pgfqpoint{3.336694in}{2.600859in}}{\pgfqpoint{3.349185in}{2.595685in}}{\pgfqpoint{3.362208in}{2.595685in}}%
\pgfpathlineto{\pgfqpoint{3.362208in}{2.595685in}}%
\pgfpathclose%
\pgfusepath{stroke,fill}%
\end{pgfscope}%
\begin{pgfscope}%
\pgfpathrectangle{\pgfqpoint{0.786164in}{0.768110in}}{\pgfqpoint{8.851069in}{7.081890in}}%
\pgfusepath{clip}%
\pgfsetbuttcap%
\pgfsetroundjoin%
\definecolor{currentfill}{rgb}{0.199430,0.387607,0.554642}%
\pgfsetfillcolor{currentfill}%
\pgfsetfillopacity{0.700000}%
\pgfsetlinewidth{0.501875pt}%
\definecolor{currentstroke}{rgb}{1.000000,1.000000,1.000000}%
\pgfsetstrokecolor{currentstroke}%
\pgfsetstrokeopacity{0.700000}%
\pgfsetdash{}{0pt}%
\pgfpathmoveto{\pgfqpoint{2.923810in}{2.354804in}}%
\pgfpathcurveto{\pgfqpoint{2.936833in}{2.354804in}}{\pgfqpoint{2.949324in}{2.359978in}}{\pgfqpoint{2.958532in}{2.369186in}}%
\pgfpathcurveto{\pgfqpoint{2.967740in}{2.378395in}}{\pgfqpoint{2.972914in}{2.390886in}}{\pgfqpoint{2.972914in}{2.403909in}}%
\pgfpathcurveto{\pgfqpoint{2.972914in}{2.416931in}}{\pgfqpoint{2.967740in}{2.429422in}}{\pgfqpoint{2.958532in}{2.438631in}}%
\pgfpathcurveto{\pgfqpoint{2.949324in}{2.447839in}}{\pgfqpoint{2.936833in}{2.453013in}}{\pgfqpoint{2.923810in}{2.453013in}}%
\pgfpathcurveto{\pgfqpoint{2.910787in}{2.453013in}}{\pgfqpoint{2.898296in}{2.447839in}}{\pgfqpoint{2.889088in}{2.438631in}}%
\pgfpathcurveto{\pgfqpoint{2.879879in}{2.429422in}}{\pgfqpoint{2.874705in}{2.416931in}}{\pgfqpoint{2.874705in}{2.403909in}}%
\pgfpathcurveto{\pgfqpoint{2.874705in}{2.390886in}}{\pgfqpoint{2.879879in}{2.378395in}}{\pgfqpoint{2.889088in}{2.369186in}}%
\pgfpathcurveto{\pgfqpoint{2.898296in}{2.359978in}}{\pgfqpoint{2.910787in}{2.354804in}}{\pgfqpoint{2.923810in}{2.354804in}}%
\pgfpathlineto{\pgfqpoint{2.923810in}{2.354804in}}%
\pgfpathclose%
\pgfusepath{stroke,fill}%
\end{pgfscope}%
\begin{pgfscope}%
\pgfpathrectangle{\pgfqpoint{0.786164in}{0.768110in}}{\pgfqpoint{8.851069in}{7.081890in}}%
\pgfusepath{clip}%
\pgfsetbuttcap%
\pgfsetroundjoin%
\definecolor{currentfill}{rgb}{0.197636,0.391528,0.554969}%
\pgfsetfillcolor{currentfill}%
\pgfsetfillopacity{0.700000}%
\pgfsetlinewidth{0.501875pt}%
\definecolor{currentstroke}{rgb}{1.000000,1.000000,1.000000}%
\pgfsetstrokecolor{currentstroke}%
\pgfsetstrokeopacity{0.700000}%
\pgfsetdash{}{0pt}%
\pgfpathmoveto{\pgfqpoint{3.006009in}{2.376702in}}%
\pgfpathcurveto{\pgfqpoint{3.019032in}{2.376702in}}{\pgfqpoint{3.031523in}{2.381876in}}{\pgfqpoint{3.040732in}{2.391085in}}%
\pgfpathcurveto{\pgfqpoint{3.049940in}{2.400293in}}{\pgfqpoint{3.055114in}{2.412784in}}{\pgfqpoint{3.055114in}{2.425807in}}%
\pgfpathcurveto{\pgfqpoint{3.055114in}{2.438830in}}{\pgfqpoint{3.049940in}{2.451321in}}{\pgfqpoint{3.040732in}{2.460529in}}%
\pgfpathcurveto{\pgfqpoint{3.031523in}{2.469738in}}{\pgfqpoint{3.019032in}{2.474912in}}{\pgfqpoint{3.006009in}{2.474912in}}%
\pgfpathcurveto{\pgfqpoint{2.992987in}{2.474912in}}{\pgfqpoint{2.980496in}{2.469738in}}{\pgfqpoint{2.971287in}{2.460529in}}%
\pgfpathcurveto{\pgfqpoint{2.962079in}{2.451321in}}{\pgfqpoint{2.956905in}{2.438830in}}{\pgfqpoint{2.956905in}{2.425807in}}%
\pgfpathcurveto{\pgfqpoint{2.956905in}{2.412784in}}{\pgfqpoint{2.962079in}{2.400293in}}{\pgfqpoint{2.971287in}{2.391085in}}%
\pgfpathcurveto{\pgfqpoint{2.980496in}{2.381876in}}{\pgfqpoint{2.992987in}{2.376702in}}{\pgfqpoint{3.006009in}{2.376702in}}%
\pgfpathlineto{\pgfqpoint{3.006009in}{2.376702in}}%
\pgfpathclose%
\pgfusepath{stroke,fill}%
\end{pgfscope}%
\begin{pgfscope}%
\pgfpathrectangle{\pgfqpoint{0.786164in}{0.768110in}}{\pgfqpoint{8.851069in}{7.081890in}}%
\pgfusepath{clip}%
\pgfsetbuttcap%
\pgfsetroundjoin%
\definecolor{currentfill}{rgb}{0.199430,0.387607,0.554642}%
\pgfsetfillcolor{currentfill}%
\pgfsetfillopacity{0.700000}%
\pgfsetlinewidth{0.501875pt}%
\definecolor{currentstroke}{rgb}{1.000000,1.000000,1.000000}%
\pgfsetstrokecolor{currentstroke}%
\pgfsetstrokeopacity{0.700000}%
\pgfsetdash{}{0pt}%
\pgfpathmoveto{\pgfqpoint{3.033409in}{2.376702in}}%
\pgfpathcurveto{\pgfqpoint{3.046432in}{2.376702in}}{\pgfqpoint{3.058923in}{2.381876in}}{\pgfqpoint{3.068131in}{2.391085in}}%
\pgfpathcurveto{\pgfqpoint{3.077340in}{2.400293in}}{\pgfqpoint{3.082514in}{2.412784in}}{\pgfqpoint{3.082514in}{2.425807in}}%
\pgfpathcurveto{\pgfqpoint{3.082514in}{2.438830in}}{\pgfqpoint{3.077340in}{2.451321in}}{\pgfqpoint{3.068131in}{2.460529in}}%
\pgfpathcurveto{\pgfqpoint{3.058923in}{2.469738in}}{\pgfqpoint{3.046432in}{2.474912in}}{\pgfqpoint{3.033409in}{2.474912in}}%
\pgfpathcurveto{\pgfqpoint{3.020387in}{2.474912in}}{\pgfqpoint{3.007895in}{2.469738in}}{\pgfqpoint{2.998687in}{2.460529in}}%
\pgfpathcurveto{\pgfqpoint{2.989479in}{2.451321in}}{\pgfqpoint{2.984305in}{2.438830in}}{\pgfqpoint{2.984305in}{2.425807in}}%
\pgfpathcurveto{\pgfqpoint{2.984305in}{2.412784in}}{\pgfqpoint{2.989479in}{2.400293in}}{\pgfqpoint{2.998687in}{2.391085in}}%
\pgfpathcurveto{\pgfqpoint{3.007895in}{2.381876in}}{\pgfqpoint{3.020387in}{2.376702in}}{\pgfqpoint{3.033409in}{2.376702in}}%
\pgfpathlineto{\pgfqpoint{3.033409in}{2.376702in}}%
\pgfpathclose%
\pgfusepath{stroke,fill}%
\end{pgfscope}%
\begin{pgfscope}%
\pgfpathrectangle{\pgfqpoint{0.786164in}{0.768110in}}{\pgfqpoint{8.851069in}{7.081890in}}%
\pgfusepath{clip}%
\pgfsetbuttcap%
\pgfsetroundjoin%
\definecolor{currentfill}{rgb}{0.194100,0.399323,0.555565}%
\pgfsetfillcolor{currentfill}%
\pgfsetfillopacity{0.700000}%
\pgfsetlinewidth{0.501875pt}%
\definecolor{currentstroke}{rgb}{1.000000,1.000000,1.000000}%
\pgfsetstrokecolor{currentstroke}%
\pgfsetstrokeopacity{0.700000}%
\pgfsetdash{}{0pt}%
\pgfpathmoveto{\pgfqpoint{2.905543in}{2.311008in}}%
\pgfpathcurveto{\pgfqpoint{2.918566in}{2.311008in}}{\pgfqpoint{2.931057in}{2.316182in}}{\pgfqpoint{2.940265in}{2.325390in}}%
\pgfpathcurveto{\pgfqpoint{2.949474in}{2.334598in}}{\pgfqpoint{2.954648in}{2.347089in}}{\pgfqpoint{2.954648in}{2.360112in}}%
\pgfpathcurveto{\pgfqpoint{2.954648in}{2.373135in}}{\pgfqpoint{2.949474in}{2.385626in}}{\pgfqpoint{2.940265in}{2.394834in}}%
\pgfpathcurveto{\pgfqpoint{2.931057in}{2.404043in}}{\pgfqpoint{2.918566in}{2.409217in}}{\pgfqpoint{2.905543in}{2.409217in}}%
\pgfpathcurveto{\pgfqpoint{2.892521in}{2.409217in}}{\pgfqpoint{2.880029in}{2.404043in}}{\pgfqpoint{2.870821in}{2.394834in}}%
\pgfpathcurveto{\pgfqpoint{2.861613in}{2.385626in}}{\pgfqpoint{2.856439in}{2.373135in}}{\pgfqpoint{2.856439in}{2.360112in}}%
\pgfpathcurveto{\pgfqpoint{2.856439in}{2.347089in}}{\pgfqpoint{2.861613in}{2.334598in}}{\pgfqpoint{2.870821in}{2.325390in}}%
\pgfpathcurveto{\pgfqpoint{2.880029in}{2.316182in}}{\pgfqpoint{2.892521in}{2.311008in}}{\pgfqpoint{2.905543in}{2.311008in}}%
\pgfpathlineto{\pgfqpoint{2.905543in}{2.311008in}}%
\pgfpathclose%
\pgfusepath{stroke,fill}%
\end{pgfscope}%
\begin{pgfscope}%
\pgfpathrectangle{\pgfqpoint{0.786164in}{0.768110in}}{\pgfqpoint{8.851069in}{7.081890in}}%
\pgfusepath{clip}%
\pgfsetbuttcap%
\pgfsetroundjoin%
\definecolor{currentfill}{rgb}{0.183898,0.422383,0.556944}%
\pgfsetfillcolor{currentfill}%
\pgfsetfillopacity{0.700000}%
\pgfsetlinewidth{0.501875pt}%
\definecolor{currentstroke}{rgb}{1.000000,1.000000,1.000000}%
\pgfsetstrokecolor{currentstroke}%
\pgfsetstrokeopacity{0.700000}%
\pgfsetdash{}{0pt}%
\pgfpathmoveto{\pgfqpoint{2.823344in}{2.420499in}}%
\pgfpathcurveto{\pgfqpoint{2.836366in}{2.420499in}}{\pgfqpoint{2.848857in}{2.425673in}}{\pgfqpoint{2.858066in}{2.434881in}}%
\pgfpathcurveto{\pgfqpoint{2.867274in}{2.444090in}}{\pgfqpoint{2.872448in}{2.456581in}}{\pgfqpoint{2.872448in}{2.469603in}}%
\pgfpathcurveto{\pgfqpoint{2.872448in}{2.482626in}}{\pgfqpoint{2.867274in}{2.495117in}}{\pgfqpoint{2.858066in}{2.504326in}}%
\pgfpathcurveto{\pgfqpoint{2.848857in}{2.513534in}}{\pgfqpoint{2.836366in}{2.518708in}}{\pgfqpoint{2.823344in}{2.518708in}}%
\pgfpathcurveto{\pgfqpoint{2.810321in}{2.518708in}}{\pgfqpoint{2.797830in}{2.513534in}}{\pgfqpoint{2.788621in}{2.504326in}}%
\pgfpathcurveto{\pgfqpoint{2.779413in}{2.495117in}}{\pgfqpoint{2.774239in}{2.482626in}}{\pgfqpoint{2.774239in}{2.469603in}}%
\pgfpathcurveto{\pgfqpoint{2.774239in}{2.456581in}}{\pgfqpoint{2.779413in}{2.444090in}}{\pgfqpoint{2.788621in}{2.434881in}}%
\pgfpathcurveto{\pgfqpoint{2.797830in}{2.425673in}}{\pgfqpoint{2.810321in}{2.420499in}}{\pgfqpoint{2.823344in}{2.420499in}}%
\pgfpathlineto{\pgfqpoint{2.823344in}{2.420499in}}%
\pgfpathclose%
\pgfusepath{stroke,fill}%
\end{pgfscope}%
\begin{pgfscope}%
\pgfpathrectangle{\pgfqpoint{0.786164in}{0.768110in}}{\pgfqpoint{8.851069in}{7.081890in}}%
\pgfusepath{clip}%
\pgfsetbuttcap%
\pgfsetroundjoin%
\definecolor{currentfill}{rgb}{0.188923,0.410910,0.556326}%
\pgfsetfillcolor{currentfill}%
\pgfsetfillopacity{0.700000}%
\pgfsetlinewidth{0.501875pt}%
\definecolor{currentstroke}{rgb}{1.000000,1.000000,1.000000}%
\pgfsetstrokecolor{currentstroke}%
\pgfsetstrokeopacity{0.700000}%
\pgfsetdash{}{0pt}%
\pgfpathmoveto{\pgfqpoint{2.704611in}{2.858463in}}%
\pgfpathcurveto{\pgfqpoint{2.717634in}{2.858463in}}{\pgfqpoint{2.730125in}{2.863637in}}{\pgfqpoint{2.739333in}{2.872846in}}%
\pgfpathcurveto{\pgfqpoint{2.748542in}{2.882054in}}{\pgfqpoint{2.753716in}{2.894545in}}{\pgfqpoint{2.753716in}{2.907568in}}%
\pgfpathcurveto{\pgfqpoint{2.753716in}{2.920591in}}{\pgfqpoint{2.748542in}{2.933082in}}{\pgfqpoint{2.739333in}{2.942290in}}%
\pgfpathcurveto{\pgfqpoint{2.730125in}{2.951499in}}{\pgfqpoint{2.717634in}{2.956673in}}{\pgfqpoint{2.704611in}{2.956673in}}%
\pgfpathcurveto{\pgfqpoint{2.691588in}{2.956673in}}{\pgfqpoint{2.679097in}{2.951499in}}{\pgfqpoint{2.669889in}{2.942290in}}%
\pgfpathcurveto{\pgfqpoint{2.660680in}{2.933082in}}{\pgfqpoint{2.655506in}{2.920591in}}{\pgfqpoint{2.655506in}{2.907568in}}%
\pgfpathcurveto{\pgfqpoint{2.655506in}{2.894545in}}{\pgfqpoint{2.660680in}{2.882054in}}{\pgfqpoint{2.669889in}{2.872846in}}%
\pgfpathcurveto{\pgfqpoint{2.679097in}{2.863637in}}{\pgfqpoint{2.691588in}{2.858463in}}{\pgfqpoint{2.704611in}{2.858463in}}%
\pgfpathlineto{\pgfqpoint{2.704611in}{2.858463in}}%
\pgfpathclose%
\pgfusepath{stroke,fill}%
\end{pgfscope}%
\begin{pgfscope}%
\pgfpathrectangle{\pgfqpoint{0.786164in}{0.768110in}}{\pgfqpoint{8.851069in}{7.081890in}}%
\pgfusepath{clip}%
\pgfsetbuttcap%
\pgfsetroundjoin%
\definecolor{currentfill}{rgb}{0.185556,0.418570,0.556753}%
\pgfsetfillcolor{currentfill}%
\pgfsetfillopacity{0.700000}%
\pgfsetlinewidth{0.501875pt}%
\definecolor{currentstroke}{rgb}{1.000000,1.000000,1.000000}%
\pgfsetstrokecolor{currentstroke}%
\pgfsetstrokeopacity{0.700000}%
\pgfsetdash{}{0pt}%
\pgfpathmoveto{\pgfqpoint{2.677211in}{2.880362in}}%
\pgfpathcurveto{\pgfqpoint{2.690234in}{2.880362in}}{\pgfqpoint{2.702725in}{2.885536in}}{\pgfqpoint{2.711933in}{2.894744in}}%
\pgfpathcurveto{\pgfqpoint{2.721142in}{2.903953in}}{\pgfqpoint{2.726316in}{2.916444in}}{\pgfqpoint{2.726316in}{2.929466in}}%
\pgfpathcurveto{\pgfqpoint{2.726316in}{2.942489in}}{\pgfqpoint{2.721142in}{2.954980in}}{\pgfqpoint{2.711933in}{2.964189in}}%
\pgfpathcurveto{\pgfqpoint{2.702725in}{2.973397in}}{\pgfqpoint{2.690234in}{2.978571in}}{\pgfqpoint{2.677211in}{2.978571in}}%
\pgfpathcurveto{\pgfqpoint{2.664188in}{2.978571in}}{\pgfqpoint{2.651697in}{2.973397in}}{\pgfqpoint{2.642489in}{2.964189in}}%
\pgfpathcurveto{\pgfqpoint{2.633280in}{2.954980in}}{\pgfqpoint{2.628106in}{2.942489in}}{\pgfqpoint{2.628106in}{2.929466in}}%
\pgfpathcurveto{\pgfqpoint{2.628106in}{2.916444in}}{\pgfqpoint{2.633280in}{2.903953in}}{\pgfqpoint{2.642489in}{2.894744in}}%
\pgfpathcurveto{\pgfqpoint{2.651697in}{2.885536in}}{\pgfqpoint{2.664188in}{2.880362in}}{\pgfqpoint{2.677211in}{2.880362in}}%
\pgfpathlineto{\pgfqpoint{2.677211in}{2.880362in}}%
\pgfpathclose%
\pgfusepath{stroke,fill}%
\end{pgfscope}%
\begin{pgfscope}%
\pgfpathrectangle{\pgfqpoint{0.786164in}{0.768110in}}{\pgfqpoint{8.851069in}{7.081890in}}%
\pgfusepath{clip}%
\pgfsetbuttcap%
\pgfsetroundjoin%
\definecolor{currentfill}{rgb}{0.192357,0.403199,0.555836}%
\pgfsetfillcolor{currentfill}%
\pgfsetfillopacity{0.700000}%
\pgfsetlinewidth{0.501875pt}%
\definecolor{currentstroke}{rgb}{1.000000,1.000000,1.000000}%
\pgfsetstrokecolor{currentstroke}%
\pgfsetstrokeopacity{0.700000}%
\pgfsetdash{}{0pt}%
\pgfpathmoveto{\pgfqpoint{2.768544in}{2.989853in}}%
\pgfpathcurveto{\pgfqpoint{2.781567in}{2.989853in}}{\pgfqpoint{2.794058in}{2.995027in}}{\pgfqpoint{2.803266in}{3.004235in}}%
\pgfpathcurveto{\pgfqpoint{2.812475in}{3.013444in}}{\pgfqpoint{2.817649in}{3.025935in}}{\pgfqpoint{2.817649in}{3.038958in}}%
\pgfpathcurveto{\pgfqpoint{2.817649in}{3.051980in}}{\pgfqpoint{2.812475in}{3.064471in}}{\pgfqpoint{2.803266in}{3.073680in}}%
\pgfpathcurveto{\pgfqpoint{2.794058in}{3.082888in}}{\pgfqpoint{2.781567in}{3.088062in}}{\pgfqpoint{2.768544in}{3.088062in}}%
\pgfpathcurveto{\pgfqpoint{2.755521in}{3.088062in}}{\pgfqpoint{2.743030in}{3.082888in}}{\pgfqpoint{2.733822in}{3.073680in}}%
\pgfpathcurveto{\pgfqpoint{2.724613in}{3.064471in}}{\pgfqpoint{2.719439in}{3.051980in}}{\pgfqpoint{2.719439in}{3.038958in}}%
\pgfpathcurveto{\pgfqpoint{2.719439in}{3.025935in}}{\pgfqpoint{2.724613in}{3.013444in}}{\pgfqpoint{2.733822in}{3.004235in}}%
\pgfpathcurveto{\pgfqpoint{2.743030in}{2.995027in}}{\pgfqpoint{2.755521in}{2.989853in}}{\pgfqpoint{2.768544in}{2.989853in}}%
\pgfpathlineto{\pgfqpoint{2.768544in}{2.989853in}}%
\pgfpathclose%
\pgfusepath{stroke,fill}%
\end{pgfscope}%
\begin{pgfscope}%
\pgfpathrectangle{\pgfqpoint{0.786164in}{0.768110in}}{\pgfqpoint{8.851069in}{7.081890in}}%
\pgfusepath{clip}%
\pgfsetbuttcap%
\pgfsetroundjoin%
\definecolor{currentfill}{rgb}{0.194100,0.399323,0.555565}%
\pgfsetfillcolor{currentfill}%
\pgfsetfillopacity{0.700000}%
\pgfsetlinewidth{0.501875pt}%
\definecolor{currentstroke}{rgb}{1.000000,1.000000,1.000000}%
\pgfsetstrokecolor{currentstroke}%
\pgfsetstrokeopacity{0.700000}%
\pgfsetdash{}{0pt}%
\pgfpathmoveto{\pgfqpoint{2.850744in}{2.989853in}}%
\pgfpathcurveto{\pgfqpoint{2.863766in}{2.989853in}}{\pgfqpoint{2.876257in}{2.995027in}}{\pgfqpoint{2.885466in}{3.004235in}}%
\pgfpathcurveto{\pgfqpoint{2.894674in}{3.013444in}}{\pgfqpoint{2.899848in}{3.025935in}}{\pgfqpoint{2.899848in}{3.038958in}}%
\pgfpathcurveto{\pgfqpoint{2.899848in}{3.051980in}}{\pgfqpoint{2.894674in}{3.064471in}}{\pgfqpoint{2.885466in}{3.073680in}}%
\pgfpathcurveto{\pgfqpoint{2.876257in}{3.082888in}}{\pgfqpoint{2.863766in}{3.088062in}}{\pgfqpoint{2.850744in}{3.088062in}}%
\pgfpathcurveto{\pgfqpoint{2.837721in}{3.088062in}}{\pgfqpoint{2.825230in}{3.082888in}}{\pgfqpoint{2.816021in}{3.073680in}}%
\pgfpathcurveto{\pgfqpoint{2.806813in}{3.064471in}}{\pgfqpoint{2.801639in}{3.051980in}}{\pgfqpoint{2.801639in}{3.038958in}}%
\pgfpathcurveto{\pgfqpoint{2.801639in}{3.025935in}}{\pgfqpoint{2.806813in}{3.013444in}}{\pgfqpoint{2.816021in}{3.004235in}}%
\pgfpathcurveto{\pgfqpoint{2.825230in}{2.995027in}}{\pgfqpoint{2.837721in}{2.989853in}}{\pgfqpoint{2.850744in}{2.989853in}}%
\pgfpathlineto{\pgfqpoint{2.850744in}{2.989853in}}%
\pgfpathclose%
\pgfusepath{stroke,fill}%
\end{pgfscope}%
\begin{pgfscope}%
\pgfpathrectangle{\pgfqpoint{0.786164in}{0.768110in}}{\pgfqpoint{8.851069in}{7.081890in}}%
\pgfusepath{clip}%
\pgfsetbuttcap%
\pgfsetroundjoin%
\definecolor{currentfill}{rgb}{0.195860,0.395433,0.555276}%
\pgfsetfillcolor{currentfill}%
\pgfsetfillopacity{0.700000}%
\pgfsetlinewidth{0.501875pt}%
\definecolor{currentstroke}{rgb}{1.000000,1.000000,1.000000}%
\pgfsetstrokecolor{currentstroke}%
\pgfsetstrokeopacity{0.700000}%
\pgfsetdash{}{0pt}%
\pgfpathmoveto{\pgfqpoint{2.805077in}{2.967955in}}%
\pgfpathcurveto{\pgfqpoint{2.818100in}{2.967955in}}{\pgfqpoint{2.830591in}{2.973129in}}{\pgfqpoint{2.839799in}{2.982337in}}%
\pgfpathcurveto{\pgfqpoint{2.849008in}{2.991545in}}{\pgfqpoint{2.854182in}{3.004037in}}{\pgfqpoint{2.854182in}{3.017059in}}%
\pgfpathcurveto{\pgfqpoint{2.854182in}{3.030082in}}{\pgfqpoint{2.849008in}{3.042573in}}{\pgfqpoint{2.839799in}{3.051781in}}%
\pgfpathcurveto{\pgfqpoint{2.830591in}{3.060990in}}{\pgfqpoint{2.818100in}{3.066164in}}{\pgfqpoint{2.805077in}{3.066164in}}%
\pgfpathcurveto{\pgfqpoint{2.792054in}{3.066164in}}{\pgfqpoint{2.779563in}{3.060990in}}{\pgfqpoint{2.770355in}{3.051781in}}%
\pgfpathcurveto{\pgfqpoint{2.761146in}{3.042573in}}{\pgfqpoint{2.755972in}{3.030082in}}{\pgfqpoint{2.755972in}{3.017059in}}%
\pgfpathcurveto{\pgfqpoint{2.755972in}{3.004037in}}{\pgfqpoint{2.761146in}{2.991545in}}{\pgfqpoint{2.770355in}{2.982337in}}%
\pgfpathcurveto{\pgfqpoint{2.779563in}{2.973129in}}{\pgfqpoint{2.792054in}{2.967955in}}{\pgfqpoint{2.805077in}{2.967955in}}%
\pgfpathlineto{\pgfqpoint{2.805077in}{2.967955in}}%
\pgfpathclose%
\pgfusepath{stroke,fill}%
\end{pgfscope}%
\begin{pgfscope}%
\pgfpathrectangle{\pgfqpoint{0.786164in}{0.768110in}}{\pgfqpoint{8.851069in}{7.081890in}}%
\pgfusepath{clip}%
\pgfsetbuttcap%
\pgfsetroundjoin%
\definecolor{currentfill}{rgb}{0.192357,0.403199,0.555836}%
\pgfsetfillcolor{currentfill}%
\pgfsetfillopacity{0.700000}%
\pgfsetlinewidth{0.501875pt}%
\definecolor{currentstroke}{rgb}{1.000000,1.000000,1.000000}%
\pgfsetstrokecolor{currentstroke}%
\pgfsetstrokeopacity{0.700000}%
\pgfsetdash{}{0pt}%
\pgfpathmoveto{\pgfqpoint{2.722877in}{2.748972in}}%
\pgfpathcurveto{\pgfqpoint{2.735900in}{2.748972in}}{\pgfqpoint{2.748391in}{2.754146in}}{\pgfqpoint{2.757600in}{2.763355in}}%
\pgfpathcurveto{\pgfqpoint{2.766808in}{2.772563in}}{\pgfqpoint{2.771982in}{2.785054in}}{\pgfqpoint{2.771982in}{2.798077in}}%
\pgfpathcurveto{\pgfqpoint{2.771982in}{2.811100in}}{\pgfqpoint{2.766808in}{2.823591in}}{\pgfqpoint{2.757600in}{2.832799in}}%
\pgfpathcurveto{\pgfqpoint{2.748391in}{2.842008in}}{\pgfqpoint{2.735900in}{2.847182in}}{\pgfqpoint{2.722877in}{2.847182in}}%
\pgfpathcurveto{\pgfqpoint{2.709855in}{2.847182in}}{\pgfqpoint{2.697364in}{2.842008in}}{\pgfqpoint{2.688155in}{2.832799in}}%
\pgfpathcurveto{\pgfqpoint{2.678947in}{2.823591in}}{\pgfqpoint{2.673773in}{2.811100in}}{\pgfqpoint{2.673773in}{2.798077in}}%
\pgfpathcurveto{\pgfqpoint{2.673773in}{2.785054in}}{\pgfqpoint{2.678947in}{2.772563in}}{\pgfqpoint{2.688155in}{2.763355in}}%
\pgfpathcurveto{\pgfqpoint{2.697364in}{2.754146in}}{\pgfqpoint{2.709855in}{2.748972in}}{\pgfqpoint{2.722877in}{2.748972in}}%
\pgfpathlineto{\pgfqpoint{2.722877in}{2.748972in}}%
\pgfpathclose%
\pgfusepath{stroke,fill}%
\end{pgfscope}%
\begin{pgfscope}%
\pgfpathrectangle{\pgfqpoint{0.786164in}{0.768110in}}{\pgfqpoint{8.851069in}{7.081890in}}%
\pgfusepath{clip}%
\pgfsetbuttcap%
\pgfsetroundjoin%
\definecolor{currentfill}{rgb}{0.172719,0.448791,0.557885}%
\pgfsetfillcolor{currentfill}%
\pgfsetfillopacity{0.700000}%
\pgfsetlinewidth{0.501875pt}%
\definecolor{currentstroke}{rgb}{1.000000,1.000000,1.000000}%
\pgfsetstrokecolor{currentstroke}%
\pgfsetstrokeopacity{0.700000}%
\pgfsetdash{}{0pt}%
\pgfpathmoveto{\pgfqpoint{2.549345in}{2.639481in}}%
\pgfpathcurveto{\pgfqpoint{2.562368in}{2.639481in}}{\pgfqpoint{2.574859in}{2.644655in}}{\pgfqpoint{2.584067in}{2.653864in}}%
\pgfpathcurveto{\pgfqpoint{2.593276in}{2.663072in}}{\pgfqpoint{2.598450in}{2.675563in}}{\pgfqpoint{2.598450in}{2.688586in}}%
\pgfpathcurveto{\pgfqpoint{2.598450in}{2.701608in}}{\pgfqpoint{2.593276in}{2.714100in}}{\pgfqpoint{2.584067in}{2.723308in}}%
\pgfpathcurveto{\pgfqpoint{2.574859in}{2.732516in}}{\pgfqpoint{2.562368in}{2.737690in}}{\pgfqpoint{2.549345in}{2.737690in}}%
\pgfpathcurveto{\pgfqpoint{2.536322in}{2.737690in}}{\pgfqpoint{2.523831in}{2.732516in}}{\pgfqpoint{2.514623in}{2.723308in}}%
\pgfpathcurveto{\pgfqpoint{2.505414in}{2.714100in}}{\pgfqpoint{2.500240in}{2.701608in}}{\pgfqpoint{2.500240in}{2.688586in}}%
\pgfpathcurveto{\pgfqpoint{2.500240in}{2.675563in}}{\pgfqpoint{2.505414in}{2.663072in}}{\pgfqpoint{2.514623in}{2.653864in}}%
\pgfpathcurveto{\pgfqpoint{2.523831in}{2.644655in}}{\pgfqpoint{2.536322in}{2.639481in}}{\pgfqpoint{2.549345in}{2.639481in}}%
\pgfpathlineto{\pgfqpoint{2.549345in}{2.639481in}}%
\pgfpathclose%
\pgfusepath{stroke,fill}%
\end{pgfscope}%
\begin{pgfscope}%
\pgfpathrectangle{\pgfqpoint{0.786164in}{0.768110in}}{\pgfqpoint{8.851069in}{7.081890in}}%
\pgfusepath{clip}%
\pgfsetbuttcap%
\pgfsetroundjoin%
\definecolor{currentfill}{rgb}{0.172719,0.448791,0.557885}%
\pgfsetfillcolor{currentfill}%
\pgfsetfillopacity{0.700000}%
\pgfsetlinewidth{0.501875pt}%
\definecolor{currentstroke}{rgb}{1.000000,1.000000,1.000000}%
\pgfsetstrokecolor{currentstroke}%
\pgfsetstrokeopacity{0.700000}%
\pgfsetdash{}{0pt}%
\pgfpathmoveto{\pgfqpoint{2.613278in}{2.661379in}}%
\pgfpathcurveto{\pgfqpoint{2.626301in}{2.661379in}}{\pgfqpoint{2.638792in}{2.666553in}}{\pgfqpoint{2.648000in}{2.675762in}}%
\pgfpathcurveto{\pgfqpoint{2.657209in}{2.684970in}}{\pgfqpoint{2.662383in}{2.697461in}}{\pgfqpoint{2.662383in}{2.710484in}}%
\pgfpathcurveto{\pgfqpoint{2.662383in}{2.723507in}}{\pgfqpoint{2.657209in}{2.735998in}}{\pgfqpoint{2.648000in}{2.745206in}}%
\pgfpathcurveto{\pgfqpoint{2.638792in}{2.754415in}}{\pgfqpoint{2.626301in}{2.759589in}}{\pgfqpoint{2.613278in}{2.759589in}}%
\pgfpathcurveto{\pgfqpoint{2.600255in}{2.759589in}}{\pgfqpoint{2.587764in}{2.754415in}}{\pgfqpoint{2.578556in}{2.745206in}}%
\pgfpathcurveto{\pgfqpoint{2.569347in}{2.735998in}}{\pgfqpoint{2.564173in}{2.723507in}}{\pgfqpoint{2.564173in}{2.710484in}}%
\pgfpathcurveto{\pgfqpoint{2.564173in}{2.697461in}}{\pgfqpoint{2.569347in}{2.684970in}}{\pgfqpoint{2.578556in}{2.675762in}}%
\pgfpathcurveto{\pgfqpoint{2.587764in}{2.666553in}}{\pgfqpoint{2.600255in}{2.661379in}}{\pgfqpoint{2.613278in}{2.661379in}}%
\pgfpathlineto{\pgfqpoint{2.613278in}{2.661379in}}%
\pgfpathclose%
\pgfusepath{stroke,fill}%
\end{pgfscope}%
\begin{pgfscope}%
\pgfpathrectangle{\pgfqpoint{0.786164in}{0.768110in}}{\pgfqpoint{8.851069in}{7.081890in}}%
\pgfusepath{clip}%
\pgfsetbuttcap%
\pgfsetroundjoin%
\definecolor{currentfill}{rgb}{0.172719,0.448791,0.557885}%
\pgfsetfillcolor{currentfill}%
\pgfsetfillopacity{0.700000}%
\pgfsetlinewidth{0.501875pt}%
\definecolor{currentstroke}{rgb}{1.000000,1.000000,1.000000}%
\pgfsetstrokecolor{currentstroke}%
\pgfsetstrokeopacity{0.700000}%
\pgfsetdash{}{0pt}%
\pgfpathmoveto{\pgfqpoint{2.549345in}{2.617583in}}%
\pgfpathcurveto{\pgfqpoint{2.562368in}{2.617583in}}{\pgfqpoint{2.574859in}{2.622757in}}{\pgfqpoint{2.584067in}{2.631965in}}%
\pgfpathcurveto{\pgfqpoint{2.593276in}{2.641174in}}{\pgfqpoint{2.598450in}{2.653665in}}{\pgfqpoint{2.598450in}{2.666687in}}%
\pgfpathcurveto{\pgfqpoint{2.598450in}{2.679710in}}{\pgfqpoint{2.593276in}{2.692201in}}{\pgfqpoint{2.584067in}{2.701410in}}%
\pgfpathcurveto{\pgfqpoint{2.574859in}{2.710618in}}{\pgfqpoint{2.562368in}{2.715792in}}{\pgfqpoint{2.549345in}{2.715792in}}%
\pgfpathcurveto{\pgfqpoint{2.536322in}{2.715792in}}{\pgfqpoint{2.523831in}{2.710618in}}{\pgfqpoint{2.514623in}{2.701410in}}%
\pgfpathcurveto{\pgfqpoint{2.505414in}{2.692201in}}{\pgfqpoint{2.500240in}{2.679710in}}{\pgfqpoint{2.500240in}{2.666687in}}%
\pgfpathcurveto{\pgfqpoint{2.500240in}{2.653665in}}{\pgfqpoint{2.505414in}{2.641174in}}{\pgfqpoint{2.514623in}{2.631965in}}%
\pgfpathcurveto{\pgfqpoint{2.523831in}{2.622757in}}{\pgfqpoint{2.536322in}{2.617583in}}{\pgfqpoint{2.549345in}{2.617583in}}%
\pgfpathlineto{\pgfqpoint{2.549345in}{2.617583in}}%
\pgfpathclose%
\pgfusepath{stroke,fill}%
\end{pgfscope}%
\begin{pgfscope}%
\pgfpathrectangle{\pgfqpoint{0.786164in}{0.768110in}}{\pgfqpoint{8.851069in}{7.081890in}}%
\pgfusepath{clip}%
\pgfsetbuttcap%
\pgfsetroundjoin%
\definecolor{currentfill}{rgb}{0.282884,0.135920,0.453427}%
\pgfsetfillcolor{currentfill}%
\pgfsetfillopacity{0.700000}%
\pgfsetlinewidth{0.501875pt}%
\definecolor{currentstroke}{rgb}{1.000000,1.000000,1.000000}%
\pgfsetstrokecolor{currentstroke}%
\pgfsetstrokeopacity{0.700000}%
\pgfsetdash{}{0pt}%
\pgfpathmoveto{\pgfqpoint{2.750277in}{2.748972in}}%
\pgfpathcurveto{\pgfqpoint{2.763300in}{2.748972in}}{\pgfqpoint{2.775791in}{2.754146in}}{\pgfqpoint{2.785000in}{2.763355in}}%
\pgfpathcurveto{\pgfqpoint{2.794208in}{2.772563in}}{\pgfqpoint{2.799382in}{2.785054in}}{\pgfqpoint{2.799382in}{2.798077in}}%
\pgfpathcurveto{\pgfqpoint{2.799382in}{2.811100in}}{\pgfqpoint{2.794208in}{2.823591in}}{\pgfqpoint{2.785000in}{2.832799in}}%
\pgfpathcurveto{\pgfqpoint{2.775791in}{2.842008in}}{\pgfqpoint{2.763300in}{2.847182in}}{\pgfqpoint{2.750277in}{2.847182in}}%
\pgfpathcurveto{\pgfqpoint{2.737255in}{2.847182in}}{\pgfqpoint{2.724764in}{2.842008in}}{\pgfqpoint{2.715555in}{2.832799in}}%
\pgfpathcurveto{\pgfqpoint{2.706347in}{2.823591in}}{\pgfqpoint{2.701173in}{2.811100in}}{\pgfqpoint{2.701173in}{2.798077in}}%
\pgfpathcurveto{\pgfqpoint{2.701173in}{2.785054in}}{\pgfqpoint{2.706347in}{2.772563in}}{\pgfqpoint{2.715555in}{2.763355in}}%
\pgfpathcurveto{\pgfqpoint{2.724764in}{2.754146in}}{\pgfqpoint{2.737255in}{2.748972in}}{\pgfqpoint{2.750277in}{2.748972in}}%
\pgfpathlineto{\pgfqpoint{2.750277in}{2.748972in}}%
\pgfpathclose%
\pgfusepath{stroke,fill}%
\end{pgfscope}%
\begin{pgfscope}%
\pgfpathrectangle{\pgfqpoint{0.786164in}{0.768110in}}{\pgfqpoint{8.851069in}{7.081890in}}%
\pgfusepath{clip}%
\pgfsetbuttcap%
\pgfsetroundjoin%
\definecolor{currentfill}{rgb}{0.282884,0.135920,0.453427}%
\pgfsetfillcolor{currentfill}%
\pgfsetfillopacity{0.700000}%
\pgfsetlinewidth{0.501875pt}%
\definecolor{currentstroke}{rgb}{1.000000,1.000000,1.000000}%
\pgfsetstrokecolor{currentstroke}%
\pgfsetstrokeopacity{0.700000}%
\pgfsetdash{}{0pt}%
\pgfpathmoveto{\pgfqpoint{2.832477in}{2.924158in}}%
\pgfpathcurveto{\pgfqpoint{2.845500in}{2.924158in}}{\pgfqpoint{2.857991in}{2.929332in}}{\pgfqpoint{2.867199in}{2.938541in}}%
\pgfpathcurveto{\pgfqpoint{2.876408in}{2.947749in}}{\pgfqpoint{2.881582in}{2.960240in}}{\pgfqpoint{2.881582in}{2.973263in}}%
\pgfpathcurveto{\pgfqpoint{2.881582in}{2.986286in}}{\pgfqpoint{2.876408in}{2.998777in}}{\pgfqpoint{2.867199in}{3.007985in}}%
\pgfpathcurveto{\pgfqpoint{2.857991in}{3.017193in}}{\pgfqpoint{2.845500in}{3.022367in}}{\pgfqpoint{2.832477in}{3.022367in}}%
\pgfpathcurveto{\pgfqpoint{2.819454in}{3.022367in}}{\pgfqpoint{2.806963in}{3.017193in}}{\pgfqpoint{2.797755in}{3.007985in}}%
\pgfpathcurveto{\pgfqpoint{2.788546in}{2.998777in}}{\pgfqpoint{2.783372in}{2.986286in}}{\pgfqpoint{2.783372in}{2.973263in}}%
\pgfpathcurveto{\pgfqpoint{2.783372in}{2.960240in}}{\pgfqpoint{2.788546in}{2.947749in}}{\pgfqpoint{2.797755in}{2.938541in}}%
\pgfpathcurveto{\pgfqpoint{2.806963in}{2.929332in}}{\pgfqpoint{2.819454in}{2.924158in}}{\pgfqpoint{2.832477in}{2.924158in}}%
\pgfpathlineto{\pgfqpoint{2.832477in}{2.924158in}}%
\pgfpathclose%
\pgfusepath{stroke,fill}%
\end{pgfscope}%
\begin{pgfscope}%
\pgfpathrectangle{\pgfqpoint{0.786164in}{0.768110in}}{\pgfqpoint{8.851069in}{7.081890in}}%
\pgfusepath{clip}%
\pgfsetbuttcap%
\pgfsetroundjoin%
\definecolor{currentfill}{rgb}{0.282623,0.140926,0.457517}%
\pgfsetfillcolor{currentfill}%
\pgfsetfillopacity{0.700000}%
\pgfsetlinewidth{0.501875pt}%
\definecolor{currentstroke}{rgb}{1.000000,1.000000,1.000000}%
\pgfsetstrokecolor{currentstroke}%
\pgfsetstrokeopacity{0.700000}%
\pgfsetdash{}{0pt}%
\pgfpathmoveto{\pgfqpoint{2.805077in}{2.792769in}}%
\pgfpathcurveto{\pgfqpoint{2.818100in}{2.792769in}}{\pgfqpoint{2.830591in}{2.797943in}}{\pgfqpoint{2.839799in}{2.807151in}}%
\pgfpathcurveto{\pgfqpoint{2.849008in}{2.816360in}}{\pgfqpoint{2.854182in}{2.828851in}}{\pgfqpoint{2.854182in}{2.841873in}}%
\pgfpathcurveto{\pgfqpoint{2.854182in}{2.854896in}}{\pgfqpoint{2.849008in}{2.867387in}}{\pgfqpoint{2.839799in}{2.876596in}}%
\pgfpathcurveto{\pgfqpoint{2.830591in}{2.885804in}}{\pgfqpoint{2.818100in}{2.890978in}}{\pgfqpoint{2.805077in}{2.890978in}}%
\pgfpathcurveto{\pgfqpoint{2.792054in}{2.890978in}}{\pgfqpoint{2.779563in}{2.885804in}}{\pgfqpoint{2.770355in}{2.876596in}}%
\pgfpathcurveto{\pgfqpoint{2.761146in}{2.867387in}}{\pgfqpoint{2.755972in}{2.854896in}}{\pgfqpoint{2.755972in}{2.841873in}}%
\pgfpathcurveto{\pgfqpoint{2.755972in}{2.828851in}}{\pgfqpoint{2.761146in}{2.816360in}}{\pgfqpoint{2.770355in}{2.807151in}}%
\pgfpathcurveto{\pgfqpoint{2.779563in}{2.797943in}}{\pgfqpoint{2.792054in}{2.792769in}}{\pgfqpoint{2.805077in}{2.792769in}}%
\pgfpathlineto{\pgfqpoint{2.805077in}{2.792769in}}%
\pgfpathclose%
\pgfusepath{stroke,fill}%
\end{pgfscope}%
\begin{pgfscope}%
\pgfpathrectangle{\pgfqpoint{0.786164in}{0.768110in}}{\pgfqpoint{8.851069in}{7.081890in}}%
\pgfusepath{clip}%
\pgfsetbuttcap%
\pgfsetroundjoin%
\definecolor{currentfill}{rgb}{0.280868,0.160771,0.472899}%
\pgfsetfillcolor{currentfill}%
\pgfsetfillopacity{0.700000}%
\pgfsetlinewidth{0.501875pt}%
\definecolor{currentstroke}{rgb}{1.000000,1.000000,1.000000}%
\pgfsetstrokecolor{currentstroke}%
\pgfsetstrokeopacity{0.700000}%
\pgfsetdash{}{0pt}%
\pgfpathmoveto{\pgfqpoint{2.668078in}{2.727074in}}%
\pgfpathcurveto{\pgfqpoint{2.681100in}{2.727074in}}{\pgfqpoint{2.693592in}{2.732248in}}{\pgfqpoint{2.702800in}{2.741456in}}%
\pgfpathcurveto{\pgfqpoint{2.712008in}{2.750665in}}{\pgfqpoint{2.717182in}{2.763156in}}{\pgfqpoint{2.717182in}{2.776179in}}%
\pgfpathcurveto{\pgfqpoint{2.717182in}{2.789201in}}{\pgfqpoint{2.712008in}{2.801692in}}{\pgfqpoint{2.702800in}{2.810901in}}%
\pgfpathcurveto{\pgfqpoint{2.693592in}{2.820109in}}{\pgfqpoint{2.681100in}{2.825283in}}{\pgfqpoint{2.668078in}{2.825283in}}%
\pgfpathcurveto{\pgfqpoint{2.655055in}{2.825283in}}{\pgfqpoint{2.642564in}{2.820109in}}{\pgfqpoint{2.633356in}{2.810901in}}%
\pgfpathcurveto{\pgfqpoint{2.624147in}{2.801692in}}{\pgfqpoint{2.618973in}{2.789201in}}{\pgfqpoint{2.618973in}{2.776179in}}%
\pgfpathcurveto{\pgfqpoint{2.618973in}{2.763156in}}{\pgfqpoint{2.624147in}{2.750665in}}{\pgfqpoint{2.633356in}{2.741456in}}%
\pgfpathcurveto{\pgfqpoint{2.642564in}{2.732248in}}{\pgfqpoint{2.655055in}{2.727074in}}{\pgfqpoint{2.668078in}{2.727074in}}%
\pgfpathlineto{\pgfqpoint{2.668078in}{2.727074in}}%
\pgfpathclose%
\pgfusepath{stroke,fill}%
\end{pgfscope}%
\begin{pgfscope}%
\pgfpathrectangle{\pgfqpoint{0.786164in}{0.768110in}}{\pgfqpoint{8.851069in}{7.081890in}}%
\pgfusepath{clip}%
\pgfsetbuttcap%
\pgfsetroundjoin%
\definecolor{currentfill}{rgb}{0.280868,0.160771,0.472899}%
\pgfsetfillcolor{currentfill}%
\pgfsetfillopacity{0.700000}%
\pgfsetlinewidth{0.501875pt}%
\definecolor{currentstroke}{rgb}{1.000000,1.000000,1.000000}%
\pgfsetstrokecolor{currentstroke}%
\pgfsetstrokeopacity{0.700000}%
\pgfsetdash{}{0pt}%
\pgfpathmoveto{\pgfqpoint{2.759411in}{2.792769in}}%
\pgfpathcurveto{\pgfqpoint{2.772433in}{2.792769in}}{\pgfqpoint{2.784924in}{2.797943in}}{\pgfqpoint{2.794133in}{2.807151in}}%
\pgfpathcurveto{\pgfqpoint{2.803341in}{2.816360in}}{\pgfqpoint{2.808515in}{2.828851in}}{\pgfqpoint{2.808515in}{2.841873in}}%
\pgfpathcurveto{\pgfqpoint{2.808515in}{2.854896in}}{\pgfqpoint{2.803341in}{2.867387in}}{\pgfqpoint{2.794133in}{2.876596in}}%
\pgfpathcurveto{\pgfqpoint{2.784924in}{2.885804in}}{\pgfqpoint{2.772433in}{2.890978in}}{\pgfqpoint{2.759411in}{2.890978in}}%
\pgfpathcurveto{\pgfqpoint{2.746388in}{2.890978in}}{\pgfqpoint{2.733897in}{2.885804in}}{\pgfqpoint{2.724688in}{2.876596in}}%
\pgfpathcurveto{\pgfqpoint{2.715480in}{2.867387in}}{\pgfqpoint{2.710306in}{2.854896in}}{\pgfqpoint{2.710306in}{2.841873in}}%
\pgfpathcurveto{\pgfqpoint{2.710306in}{2.828851in}}{\pgfqpoint{2.715480in}{2.816360in}}{\pgfqpoint{2.724688in}{2.807151in}}%
\pgfpathcurveto{\pgfqpoint{2.733897in}{2.797943in}}{\pgfqpoint{2.746388in}{2.792769in}}{\pgfqpoint{2.759411in}{2.792769in}}%
\pgfpathlineto{\pgfqpoint{2.759411in}{2.792769in}}%
\pgfpathclose%
\pgfusepath{stroke,fill}%
\end{pgfscope}%
\begin{pgfscope}%
\pgfpathrectangle{\pgfqpoint{0.786164in}{0.768110in}}{\pgfqpoint{8.851069in}{7.081890in}}%
\pgfusepath{clip}%
\pgfsetbuttcap%
\pgfsetroundjoin%
\definecolor{currentfill}{rgb}{0.276194,0.190074,0.493001}%
\pgfsetfillcolor{currentfill}%
\pgfsetfillopacity{0.700000}%
\pgfsetlinewidth{0.501875pt}%
\definecolor{currentstroke}{rgb}{1.000000,1.000000,1.000000}%
\pgfsetstrokecolor{currentstroke}%
\pgfsetstrokeopacity{0.700000}%
\pgfsetdash{}{0pt}%
\pgfpathmoveto{\pgfqpoint{2.503679in}{2.639481in}}%
\pgfpathcurveto{\pgfqpoint{2.516701in}{2.639481in}}{\pgfqpoint{2.529192in}{2.644655in}}{\pgfqpoint{2.538401in}{2.653864in}}%
\pgfpathcurveto{\pgfqpoint{2.547609in}{2.663072in}}{\pgfqpoint{2.552783in}{2.675563in}}{\pgfqpoint{2.552783in}{2.688586in}}%
\pgfpathcurveto{\pgfqpoint{2.552783in}{2.701608in}}{\pgfqpoint{2.547609in}{2.714100in}}{\pgfqpoint{2.538401in}{2.723308in}}%
\pgfpathcurveto{\pgfqpoint{2.529192in}{2.732516in}}{\pgfqpoint{2.516701in}{2.737690in}}{\pgfqpoint{2.503679in}{2.737690in}}%
\pgfpathcurveto{\pgfqpoint{2.490656in}{2.737690in}}{\pgfqpoint{2.478165in}{2.732516in}}{\pgfqpoint{2.468956in}{2.723308in}}%
\pgfpathcurveto{\pgfqpoint{2.459748in}{2.714100in}}{\pgfqpoint{2.454574in}{2.701608in}}{\pgfqpoint{2.454574in}{2.688586in}}%
\pgfpathcurveto{\pgfqpoint{2.454574in}{2.675563in}}{\pgfqpoint{2.459748in}{2.663072in}}{\pgfqpoint{2.468956in}{2.653864in}}%
\pgfpathcurveto{\pgfqpoint{2.478165in}{2.644655in}}{\pgfqpoint{2.490656in}{2.639481in}}{\pgfqpoint{2.503679in}{2.639481in}}%
\pgfpathlineto{\pgfqpoint{2.503679in}{2.639481in}}%
\pgfpathclose%
\pgfusepath{stroke,fill}%
\end{pgfscope}%
\begin{pgfscope}%
\pgfpathrectangle{\pgfqpoint{0.786164in}{0.768110in}}{\pgfqpoint{8.851069in}{7.081890in}}%
\pgfusepath{clip}%
\pgfsetbuttcap%
\pgfsetroundjoin%
\definecolor{currentfill}{rgb}{0.277134,0.185228,0.489898}%
\pgfsetfillcolor{currentfill}%
\pgfsetfillopacity{0.700000}%
\pgfsetlinewidth{0.501875pt}%
\definecolor{currentstroke}{rgb}{1.000000,1.000000,1.000000}%
\pgfsetstrokecolor{currentstroke}%
\pgfsetstrokeopacity{0.700000}%
\pgfsetdash{}{0pt}%
\pgfpathmoveto{\pgfqpoint{2.814210in}{2.705176in}}%
\pgfpathcurveto{\pgfqpoint{2.827233in}{2.705176in}}{\pgfqpoint{2.839724in}{2.710350in}}{\pgfqpoint{2.848933in}{2.719558in}}%
\pgfpathcurveto{\pgfqpoint{2.858141in}{2.728767in}}{\pgfqpoint{2.863315in}{2.741258in}}{\pgfqpoint{2.863315in}{2.754280in}}%
\pgfpathcurveto{\pgfqpoint{2.863315in}{2.767303in}}{\pgfqpoint{2.858141in}{2.779794in}}{\pgfqpoint{2.848933in}{2.789003in}}%
\pgfpathcurveto{\pgfqpoint{2.839724in}{2.798211in}}{\pgfqpoint{2.827233in}{2.803385in}}{\pgfqpoint{2.814210in}{2.803385in}}%
\pgfpathcurveto{\pgfqpoint{2.801188in}{2.803385in}}{\pgfqpoint{2.788697in}{2.798211in}}{\pgfqpoint{2.779488in}{2.789003in}}%
\pgfpathcurveto{\pgfqpoint{2.770280in}{2.779794in}}{\pgfqpoint{2.765106in}{2.767303in}}{\pgfqpoint{2.765106in}{2.754280in}}%
\pgfpathcurveto{\pgfqpoint{2.765106in}{2.741258in}}{\pgfqpoint{2.770280in}{2.728767in}}{\pgfqpoint{2.779488in}{2.719558in}}%
\pgfpathcurveto{\pgfqpoint{2.788697in}{2.710350in}}{\pgfqpoint{2.801188in}{2.705176in}}{\pgfqpoint{2.814210in}{2.705176in}}%
\pgfpathlineto{\pgfqpoint{2.814210in}{2.705176in}}%
\pgfpathclose%
\pgfusepath{stroke,fill}%
\end{pgfscope}%
\begin{pgfscope}%
\pgfpathrectangle{\pgfqpoint{0.786164in}{0.768110in}}{\pgfqpoint{8.851069in}{7.081890in}}%
\pgfusepath{clip}%
\pgfsetbuttcap%
\pgfsetroundjoin%
\definecolor{currentfill}{rgb}{0.271828,0.209303,0.504434}%
\pgfsetfillcolor{currentfill}%
\pgfsetfillopacity{0.700000}%
\pgfsetlinewidth{0.501875pt}%
\definecolor{currentstroke}{rgb}{1.000000,1.000000,1.000000}%
\pgfsetstrokecolor{currentstroke}%
\pgfsetstrokeopacity{0.700000}%
\pgfsetdash{}{0pt}%
\pgfpathmoveto{\pgfqpoint{2.686344in}{2.639481in}}%
\pgfpathcurveto{\pgfqpoint{2.699367in}{2.639481in}}{\pgfqpoint{2.711858in}{2.644655in}}{\pgfqpoint{2.721067in}{2.653864in}}%
\pgfpathcurveto{\pgfqpoint{2.730275in}{2.663072in}}{\pgfqpoint{2.735449in}{2.675563in}}{\pgfqpoint{2.735449in}{2.688586in}}%
\pgfpathcurveto{\pgfqpoint{2.735449in}{2.701608in}}{\pgfqpoint{2.730275in}{2.714100in}}{\pgfqpoint{2.721067in}{2.723308in}}%
\pgfpathcurveto{\pgfqpoint{2.711858in}{2.732516in}}{\pgfqpoint{2.699367in}{2.737690in}}{\pgfqpoint{2.686344in}{2.737690in}}%
\pgfpathcurveto{\pgfqpoint{2.673322in}{2.737690in}}{\pgfqpoint{2.660831in}{2.732516in}}{\pgfqpoint{2.651622in}{2.723308in}}%
\pgfpathcurveto{\pgfqpoint{2.642414in}{2.714100in}}{\pgfqpoint{2.637240in}{2.701608in}}{\pgfqpoint{2.637240in}{2.688586in}}%
\pgfpathcurveto{\pgfqpoint{2.637240in}{2.675563in}}{\pgfqpoint{2.642414in}{2.663072in}}{\pgfqpoint{2.651622in}{2.653864in}}%
\pgfpathcurveto{\pgfqpoint{2.660831in}{2.644655in}}{\pgfqpoint{2.673322in}{2.639481in}}{\pgfqpoint{2.686344in}{2.639481in}}%
\pgfpathlineto{\pgfqpoint{2.686344in}{2.639481in}}%
\pgfpathclose%
\pgfusepath{stroke,fill}%
\end{pgfscope}%
\begin{pgfscope}%
\pgfpathrectangle{\pgfqpoint{0.786164in}{0.768110in}}{\pgfqpoint{8.851069in}{7.081890in}}%
\pgfusepath{clip}%
\pgfsetbuttcap%
\pgfsetroundjoin%
\definecolor{currentfill}{rgb}{0.270595,0.214069,0.507052}%
\pgfsetfillcolor{currentfill}%
\pgfsetfillopacity{0.700000}%
\pgfsetlinewidth{0.501875pt}%
\definecolor{currentstroke}{rgb}{1.000000,1.000000,1.000000}%
\pgfsetstrokecolor{currentstroke}%
\pgfsetstrokeopacity{0.700000}%
\pgfsetdash{}{0pt}%
\pgfpathmoveto{\pgfqpoint{2.631545in}{2.486193in}}%
\pgfpathcurveto{\pgfqpoint{2.644567in}{2.486193in}}{\pgfqpoint{2.657058in}{2.491367in}}{\pgfqpoint{2.666267in}{2.500576in}}%
\pgfpathcurveto{\pgfqpoint{2.675475in}{2.509784in}}{\pgfqpoint{2.680649in}{2.522275in}}{\pgfqpoint{2.680649in}{2.535298in}}%
\pgfpathcurveto{\pgfqpoint{2.680649in}{2.548321in}}{\pgfqpoint{2.675475in}{2.560812in}}{\pgfqpoint{2.666267in}{2.570020in}}%
\pgfpathcurveto{\pgfqpoint{2.657058in}{2.579229in}}{\pgfqpoint{2.644567in}{2.584403in}}{\pgfqpoint{2.631545in}{2.584403in}}%
\pgfpathcurveto{\pgfqpoint{2.618522in}{2.584403in}}{\pgfqpoint{2.606031in}{2.579229in}}{\pgfqpoint{2.596822in}{2.570020in}}%
\pgfpathcurveto{\pgfqpoint{2.587614in}{2.560812in}}{\pgfqpoint{2.582440in}{2.548321in}}{\pgfqpoint{2.582440in}{2.535298in}}%
\pgfpathcurveto{\pgfqpoint{2.582440in}{2.522275in}}{\pgfqpoint{2.587614in}{2.509784in}}{\pgfqpoint{2.596822in}{2.500576in}}%
\pgfpathcurveto{\pgfqpoint{2.606031in}{2.491367in}}{\pgfqpoint{2.618522in}{2.486193in}}{\pgfqpoint{2.631545in}{2.486193in}}%
\pgfpathlineto{\pgfqpoint{2.631545in}{2.486193in}}%
\pgfpathclose%
\pgfusepath{stroke,fill}%
\end{pgfscope}%
\begin{pgfscope}%
\pgfpathrectangle{\pgfqpoint{0.786164in}{0.768110in}}{\pgfqpoint{8.851069in}{7.081890in}}%
\pgfusepath{clip}%
\pgfsetbuttcap%
\pgfsetroundjoin%
\definecolor{currentfill}{rgb}{0.273006,0.204520,0.501721}%
\pgfsetfillcolor{currentfill}%
\pgfsetfillopacity{0.700000}%
\pgfsetlinewidth{0.501875pt}%
\definecolor{currentstroke}{rgb}{1.000000,1.000000,1.000000}%
\pgfsetstrokecolor{currentstroke}%
\pgfsetstrokeopacity{0.700000}%
\pgfsetdash{}{0pt}%
\pgfpathmoveto{\pgfqpoint{2.640678in}{2.442397in}}%
\pgfpathcurveto{\pgfqpoint{2.653701in}{2.442397in}}{\pgfqpoint{2.666192in}{2.447571in}}{\pgfqpoint{2.675400in}{2.456779in}}%
\pgfpathcurveto{\pgfqpoint{2.684609in}{2.465988in}}{\pgfqpoint{2.689783in}{2.478479in}}{\pgfqpoint{2.689783in}{2.491502in}}%
\pgfpathcurveto{\pgfqpoint{2.689783in}{2.504524in}}{\pgfqpoint{2.684609in}{2.517015in}}{\pgfqpoint{2.675400in}{2.526224in}}%
\pgfpathcurveto{\pgfqpoint{2.666192in}{2.535432in}}{\pgfqpoint{2.653701in}{2.540606in}}{\pgfqpoint{2.640678in}{2.540606in}}%
\pgfpathcurveto{\pgfqpoint{2.627655in}{2.540606in}}{\pgfqpoint{2.615164in}{2.535432in}}{\pgfqpoint{2.605956in}{2.526224in}}%
\pgfpathcurveto{\pgfqpoint{2.596747in}{2.517015in}}{\pgfqpoint{2.591573in}{2.504524in}}{\pgfqpoint{2.591573in}{2.491502in}}%
\pgfpathcurveto{\pgfqpoint{2.591573in}{2.478479in}}{\pgfqpoint{2.596747in}{2.465988in}}{\pgfqpoint{2.605956in}{2.456779in}}%
\pgfpathcurveto{\pgfqpoint{2.615164in}{2.447571in}}{\pgfqpoint{2.627655in}{2.442397in}}{\pgfqpoint{2.640678in}{2.442397in}}%
\pgfpathlineto{\pgfqpoint{2.640678in}{2.442397in}}%
\pgfpathclose%
\pgfusepath{stroke,fill}%
\end{pgfscope}%
\begin{pgfscope}%
\pgfpathrectangle{\pgfqpoint{0.786164in}{0.768110in}}{\pgfqpoint{8.851069in}{7.081890in}}%
\pgfusepath{clip}%
\pgfsetbuttcap%
\pgfsetroundjoin%
\definecolor{currentfill}{rgb}{0.263663,0.237631,0.518762}%
\pgfsetfillcolor{currentfill}%
\pgfsetfillopacity{0.700000}%
\pgfsetlinewidth{0.501875pt}%
\definecolor{currentstroke}{rgb}{1.000000,1.000000,1.000000}%
\pgfsetstrokecolor{currentstroke}%
\pgfsetstrokeopacity{0.700000}%
\pgfsetdash{}{0pt}%
\pgfpathmoveto{\pgfqpoint{2.439746in}{2.442397in}}%
\pgfpathcurveto{\pgfqpoint{2.452768in}{2.442397in}}{\pgfqpoint{2.465259in}{2.447571in}}{\pgfqpoint{2.474468in}{2.456779in}}%
\pgfpathcurveto{\pgfqpoint{2.483676in}{2.465988in}}{\pgfqpoint{2.488850in}{2.478479in}}{\pgfqpoint{2.488850in}{2.491502in}}%
\pgfpathcurveto{\pgfqpoint{2.488850in}{2.504524in}}{\pgfqpoint{2.483676in}{2.517015in}}{\pgfqpoint{2.474468in}{2.526224in}}%
\pgfpathcurveto{\pgfqpoint{2.465259in}{2.535432in}}{\pgfqpoint{2.452768in}{2.540606in}}{\pgfqpoint{2.439746in}{2.540606in}}%
\pgfpathcurveto{\pgfqpoint{2.426723in}{2.540606in}}{\pgfqpoint{2.414232in}{2.535432in}}{\pgfqpoint{2.405023in}{2.526224in}}%
\pgfpathcurveto{\pgfqpoint{2.395815in}{2.517015in}}{\pgfqpoint{2.390641in}{2.504524in}}{\pgfqpoint{2.390641in}{2.491502in}}%
\pgfpathcurveto{\pgfqpoint{2.390641in}{2.478479in}}{\pgfqpoint{2.395815in}{2.465988in}}{\pgfqpoint{2.405023in}{2.456779in}}%
\pgfpathcurveto{\pgfqpoint{2.414232in}{2.447571in}}{\pgfqpoint{2.426723in}{2.442397in}}{\pgfqpoint{2.439746in}{2.442397in}}%
\pgfpathlineto{\pgfqpoint{2.439746in}{2.442397in}}%
\pgfpathclose%
\pgfusepath{stroke,fill}%
\end{pgfscope}%
\begin{pgfscope}%
\pgfpathrectangle{\pgfqpoint{0.786164in}{0.768110in}}{\pgfqpoint{8.851069in}{7.081890in}}%
\pgfusepath{clip}%
\pgfsetbuttcap%
\pgfsetroundjoin%
\definecolor{currentfill}{rgb}{0.257322,0.256130,0.526563}%
\pgfsetfillcolor{currentfill}%
\pgfsetfillopacity{0.700000}%
\pgfsetlinewidth{0.501875pt}%
\definecolor{currentstroke}{rgb}{1.000000,1.000000,1.000000}%
\pgfsetstrokecolor{currentstroke}%
\pgfsetstrokeopacity{0.700000}%
\pgfsetdash{}{0pt}%
\pgfpathmoveto{\pgfqpoint{2.531078in}{2.464295in}}%
\pgfpathcurveto{\pgfqpoint{2.544101in}{2.464295in}}{\pgfqpoint{2.556592in}{2.469469in}}{\pgfqpoint{2.565801in}{2.478678in}}%
\pgfpathcurveto{\pgfqpoint{2.575009in}{2.487886in}}{\pgfqpoint{2.580183in}{2.500377in}}{\pgfqpoint{2.580183in}{2.513400in}}%
\pgfpathcurveto{\pgfqpoint{2.580183in}{2.526423in}}{\pgfqpoint{2.575009in}{2.538914in}}{\pgfqpoint{2.565801in}{2.548122in}}%
\pgfpathcurveto{\pgfqpoint{2.556592in}{2.557331in}}{\pgfqpoint{2.544101in}{2.562504in}}{\pgfqpoint{2.531078in}{2.562504in}}%
\pgfpathcurveto{\pgfqpoint{2.518056in}{2.562504in}}{\pgfqpoint{2.505565in}{2.557331in}}{\pgfqpoint{2.496356in}{2.548122in}}%
\pgfpathcurveto{\pgfqpoint{2.487148in}{2.538914in}}{\pgfqpoint{2.481974in}{2.526423in}}{\pgfqpoint{2.481974in}{2.513400in}}%
\pgfpathcurveto{\pgfqpoint{2.481974in}{2.500377in}}{\pgfqpoint{2.487148in}{2.487886in}}{\pgfqpoint{2.496356in}{2.478678in}}%
\pgfpathcurveto{\pgfqpoint{2.505565in}{2.469469in}}{\pgfqpoint{2.518056in}{2.464295in}}{\pgfqpoint{2.531078in}{2.464295in}}%
\pgfpathlineto{\pgfqpoint{2.531078in}{2.464295in}}%
\pgfpathclose%
\pgfusepath{stroke,fill}%
\end{pgfscope}%
\begin{pgfscope}%
\pgfpathrectangle{\pgfqpoint{0.786164in}{0.768110in}}{\pgfqpoint{8.851069in}{7.081890in}}%
\pgfusepath{clip}%
\pgfsetbuttcap%
\pgfsetroundjoin%
\definecolor{currentfill}{rgb}{0.262138,0.242286,0.520837}%
\pgfsetfillcolor{currentfill}%
\pgfsetfillopacity{0.700000}%
\pgfsetlinewidth{0.501875pt}%
\definecolor{currentstroke}{rgb}{1.000000,1.000000,1.000000}%
\pgfsetstrokecolor{currentstroke}%
\pgfsetstrokeopacity{0.700000}%
\pgfsetdash{}{0pt}%
\pgfpathmoveto{\pgfqpoint{2.567612in}{2.486193in}}%
\pgfpathcurveto{\pgfqpoint{2.580634in}{2.486193in}}{\pgfqpoint{2.593125in}{2.491367in}}{\pgfqpoint{2.602334in}{2.500576in}}%
\pgfpathcurveto{\pgfqpoint{2.611542in}{2.509784in}}{\pgfqpoint{2.616716in}{2.522275in}}{\pgfqpoint{2.616716in}{2.535298in}}%
\pgfpathcurveto{\pgfqpoint{2.616716in}{2.548321in}}{\pgfqpoint{2.611542in}{2.560812in}}{\pgfqpoint{2.602334in}{2.570020in}}%
\pgfpathcurveto{\pgfqpoint{2.593125in}{2.579229in}}{\pgfqpoint{2.580634in}{2.584403in}}{\pgfqpoint{2.567612in}{2.584403in}}%
\pgfpathcurveto{\pgfqpoint{2.554589in}{2.584403in}}{\pgfqpoint{2.542098in}{2.579229in}}{\pgfqpoint{2.532889in}{2.570020in}}%
\pgfpathcurveto{\pgfqpoint{2.523681in}{2.560812in}}{\pgfqpoint{2.518507in}{2.548321in}}{\pgfqpoint{2.518507in}{2.535298in}}%
\pgfpathcurveto{\pgfqpoint{2.518507in}{2.522275in}}{\pgfqpoint{2.523681in}{2.509784in}}{\pgfqpoint{2.532889in}{2.500576in}}%
\pgfpathcurveto{\pgfqpoint{2.542098in}{2.491367in}}{\pgfqpoint{2.554589in}{2.486193in}}{\pgfqpoint{2.567612in}{2.486193in}}%
\pgfpathlineto{\pgfqpoint{2.567612in}{2.486193in}}%
\pgfpathclose%
\pgfusepath{stroke,fill}%
\end{pgfscope}%
\begin{pgfscope}%
\pgfpathrectangle{\pgfqpoint{0.786164in}{0.768110in}}{\pgfqpoint{8.851069in}{7.081890in}}%
\pgfusepath{clip}%
\pgfsetbuttcap%
\pgfsetroundjoin%
\definecolor{currentfill}{rgb}{0.265145,0.232956,0.516599}%
\pgfsetfillcolor{currentfill}%
\pgfsetfillopacity{0.700000}%
\pgfsetlinewidth{0.501875pt}%
\definecolor{currentstroke}{rgb}{1.000000,1.000000,1.000000}%
\pgfsetstrokecolor{currentstroke}%
\pgfsetstrokeopacity{0.700000}%
\pgfsetdash{}{0pt}%
\pgfpathmoveto{\pgfqpoint{2.686344in}{2.529990in}}%
\pgfpathcurveto{\pgfqpoint{2.699367in}{2.529990in}}{\pgfqpoint{2.711858in}{2.535164in}}{\pgfqpoint{2.721067in}{2.544372in}}%
\pgfpathcurveto{\pgfqpoint{2.730275in}{2.553581in}}{\pgfqpoint{2.735449in}{2.566072in}}{\pgfqpoint{2.735449in}{2.579095in}}%
\pgfpathcurveto{\pgfqpoint{2.735449in}{2.592117in}}{\pgfqpoint{2.730275in}{2.604608in}}{\pgfqpoint{2.721067in}{2.613817in}}%
\pgfpathcurveto{\pgfqpoint{2.711858in}{2.623025in}}{\pgfqpoint{2.699367in}{2.628199in}}{\pgfqpoint{2.686344in}{2.628199in}}%
\pgfpathcurveto{\pgfqpoint{2.673322in}{2.628199in}}{\pgfqpoint{2.660831in}{2.623025in}}{\pgfqpoint{2.651622in}{2.613817in}}%
\pgfpathcurveto{\pgfqpoint{2.642414in}{2.604608in}}{\pgfqpoint{2.637240in}{2.592117in}}{\pgfqpoint{2.637240in}{2.579095in}}%
\pgfpathcurveto{\pgfqpoint{2.637240in}{2.566072in}}{\pgfqpoint{2.642414in}{2.553581in}}{\pgfqpoint{2.651622in}{2.544372in}}%
\pgfpathcurveto{\pgfqpoint{2.660831in}{2.535164in}}{\pgfqpoint{2.673322in}{2.529990in}}{\pgfqpoint{2.686344in}{2.529990in}}%
\pgfpathlineto{\pgfqpoint{2.686344in}{2.529990in}}%
\pgfpathclose%
\pgfusepath{stroke,fill}%
\end{pgfscope}%
\begin{pgfscope}%
\pgfpathrectangle{\pgfqpoint{0.786164in}{0.768110in}}{\pgfqpoint{8.851069in}{7.081890in}}%
\pgfusepath{clip}%
\pgfsetbuttcap%
\pgfsetroundjoin%
\definecolor{currentfill}{rgb}{0.263663,0.237631,0.518762}%
\pgfsetfillcolor{currentfill}%
\pgfsetfillopacity{0.700000}%
\pgfsetlinewidth{0.501875pt}%
\definecolor{currentstroke}{rgb}{1.000000,1.000000,1.000000}%
\pgfsetstrokecolor{currentstroke}%
\pgfsetstrokeopacity{0.700000}%
\pgfsetdash{}{0pt}%
\pgfpathmoveto{\pgfqpoint{2.604145in}{2.551888in}}%
\pgfpathcurveto{\pgfqpoint{2.617167in}{2.551888in}}{\pgfqpoint{2.629659in}{2.557062in}}{\pgfqpoint{2.638867in}{2.566271in}}%
\pgfpathcurveto{\pgfqpoint{2.648075in}{2.575479in}}{\pgfqpoint{2.653249in}{2.587970in}}{\pgfqpoint{2.653249in}{2.600993in}}%
\pgfpathcurveto{\pgfqpoint{2.653249in}{2.614015in}}{\pgfqpoint{2.648075in}{2.626507in}}{\pgfqpoint{2.638867in}{2.635715in}}%
\pgfpathcurveto{\pgfqpoint{2.629659in}{2.644923in}}{\pgfqpoint{2.617167in}{2.650097in}}{\pgfqpoint{2.604145in}{2.650097in}}%
\pgfpathcurveto{\pgfqpoint{2.591122in}{2.650097in}}{\pgfqpoint{2.578631in}{2.644923in}}{\pgfqpoint{2.569423in}{2.635715in}}%
\pgfpathcurveto{\pgfqpoint{2.560214in}{2.626507in}}{\pgfqpoint{2.555040in}{2.614015in}}{\pgfqpoint{2.555040in}{2.600993in}}%
\pgfpathcurveto{\pgfqpoint{2.555040in}{2.587970in}}{\pgfqpoint{2.560214in}{2.575479in}}{\pgfqpoint{2.569423in}{2.566271in}}%
\pgfpathcurveto{\pgfqpoint{2.578631in}{2.557062in}}{\pgfqpoint{2.591122in}{2.551888in}}{\pgfqpoint{2.604145in}{2.551888in}}%
\pgfpathlineto{\pgfqpoint{2.604145in}{2.551888in}}%
\pgfpathclose%
\pgfusepath{stroke,fill}%
\end{pgfscope}%
\begin{pgfscope}%
\pgfpathrectangle{\pgfqpoint{0.786164in}{0.768110in}}{\pgfqpoint{8.851069in}{7.081890in}}%
\pgfusepath{clip}%
\pgfsetbuttcap%
\pgfsetroundjoin%
\definecolor{currentfill}{rgb}{0.227802,0.326594,0.546532}%
\pgfsetfillcolor{currentfill}%
\pgfsetfillopacity{0.700000}%
\pgfsetlinewidth{0.501875pt}%
\definecolor{currentstroke}{rgb}{1.000000,1.000000,1.000000}%
\pgfsetstrokecolor{currentstroke}%
\pgfsetstrokeopacity{0.700000}%
\pgfsetdash{}{0pt}%
\pgfpathmoveto{\pgfqpoint{2.549345in}{2.420499in}}%
\pgfpathcurveto{\pgfqpoint{2.562368in}{2.420499in}}{\pgfqpoint{2.574859in}{2.425673in}}{\pgfqpoint{2.584067in}{2.434881in}}%
\pgfpathcurveto{\pgfqpoint{2.593276in}{2.444090in}}{\pgfqpoint{2.598450in}{2.456581in}}{\pgfqpoint{2.598450in}{2.469603in}}%
\pgfpathcurveto{\pgfqpoint{2.598450in}{2.482626in}}{\pgfqpoint{2.593276in}{2.495117in}}{\pgfqpoint{2.584067in}{2.504326in}}%
\pgfpathcurveto{\pgfqpoint{2.574859in}{2.513534in}}{\pgfqpoint{2.562368in}{2.518708in}}{\pgfqpoint{2.549345in}{2.518708in}}%
\pgfpathcurveto{\pgfqpoint{2.536322in}{2.518708in}}{\pgfqpoint{2.523831in}{2.513534in}}{\pgfqpoint{2.514623in}{2.504326in}}%
\pgfpathcurveto{\pgfqpoint{2.505414in}{2.495117in}}{\pgfqpoint{2.500240in}{2.482626in}}{\pgfqpoint{2.500240in}{2.469603in}}%
\pgfpathcurveto{\pgfqpoint{2.500240in}{2.456581in}}{\pgfqpoint{2.505414in}{2.444090in}}{\pgfqpoint{2.514623in}{2.434881in}}%
\pgfpathcurveto{\pgfqpoint{2.523831in}{2.425673in}}{\pgfqpoint{2.536322in}{2.420499in}}{\pgfqpoint{2.549345in}{2.420499in}}%
\pgfpathlineto{\pgfqpoint{2.549345in}{2.420499in}}%
\pgfpathclose%
\pgfusepath{stroke,fill}%
\end{pgfscope}%
\begin{pgfscope}%
\pgfpathrectangle{\pgfqpoint{0.786164in}{0.768110in}}{\pgfqpoint{8.851069in}{7.081890in}}%
\pgfusepath{clip}%
\pgfsetbuttcap%
\pgfsetroundjoin%
\definecolor{currentfill}{rgb}{0.225863,0.330805,0.547314}%
\pgfsetfillcolor{currentfill}%
\pgfsetfillopacity{0.700000}%
\pgfsetlinewidth{0.501875pt}%
\definecolor{currentstroke}{rgb}{1.000000,1.000000,1.000000}%
\pgfsetstrokecolor{currentstroke}%
\pgfsetstrokeopacity{0.700000}%
\pgfsetdash{}{0pt}%
\pgfpathmoveto{\pgfqpoint{2.412346in}{2.223415in}}%
\pgfpathcurveto{\pgfqpoint{2.425368in}{2.223415in}}{\pgfqpoint{2.437860in}{2.228589in}}{\pgfqpoint{2.447068in}{2.237797in}}%
\pgfpathcurveto{\pgfqpoint{2.456276in}{2.247005in}}{\pgfqpoint{2.461450in}{2.259497in}}{\pgfqpoint{2.461450in}{2.272519in}}%
\pgfpathcurveto{\pgfqpoint{2.461450in}{2.285542in}}{\pgfqpoint{2.456276in}{2.298033in}}{\pgfqpoint{2.447068in}{2.307241in}}%
\pgfpathcurveto{\pgfqpoint{2.437860in}{2.316450in}}{\pgfqpoint{2.425368in}{2.321624in}}{\pgfqpoint{2.412346in}{2.321624in}}%
\pgfpathcurveto{\pgfqpoint{2.399323in}{2.321624in}}{\pgfqpoint{2.386832in}{2.316450in}}{\pgfqpoint{2.377624in}{2.307241in}}%
\pgfpathcurveto{\pgfqpoint{2.368415in}{2.298033in}}{\pgfqpoint{2.363241in}{2.285542in}}{\pgfqpoint{2.363241in}{2.272519in}}%
\pgfpathcurveto{\pgfqpoint{2.363241in}{2.259497in}}{\pgfqpoint{2.368415in}{2.247005in}}{\pgfqpoint{2.377624in}{2.237797in}}%
\pgfpathcurveto{\pgfqpoint{2.386832in}{2.228589in}}{\pgfqpoint{2.399323in}{2.223415in}}{\pgfqpoint{2.412346in}{2.223415in}}%
\pgfpathlineto{\pgfqpoint{2.412346in}{2.223415in}}%
\pgfpathclose%
\pgfusepath{stroke,fill}%
\end{pgfscope}%
\begin{pgfscope}%
\pgfpathrectangle{\pgfqpoint{0.786164in}{0.768110in}}{\pgfqpoint{8.851069in}{7.081890in}}%
\pgfusepath{clip}%
\pgfsetbuttcap%
\pgfsetroundjoin%
\definecolor{currentfill}{rgb}{0.223925,0.334994,0.548053}%
\pgfsetfillcolor{currentfill}%
\pgfsetfillopacity{0.700000}%
\pgfsetlinewidth{0.501875pt}%
\definecolor{currentstroke}{rgb}{1.000000,1.000000,1.000000}%
\pgfsetstrokecolor{currentstroke}%
\pgfsetstrokeopacity{0.700000}%
\pgfsetdash{}{0pt}%
\pgfpathmoveto{\pgfqpoint{2.631545in}{2.398600in}}%
\pgfpathcurveto{\pgfqpoint{2.644567in}{2.398600in}}{\pgfqpoint{2.657058in}{2.403774in}}{\pgfqpoint{2.666267in}{2.412983in}}%
\pgfpathcurveto{\pgfqpoint{2.675475in}{2.422191in}}{\pgfqpoint{2.680649in}{2.434682in}}{\pgfqpoint{2.680649in}{2.447705in}}%
\pgfpathcurveto{\pgfqpoint{2.680649in}{2.460728in}}{\pgfqpoint{2.675475in}{2.473219in}}{\pgfqpoint{2.666267in}{2.482427in}}%
\pgfpathcurveto{\pgfqpoint{2.657058in}{2.491636in}}{\pgfqpoint{2.644567in}{2.496810in}}{\pgfqpoint{2.631545in}{2.496810in}}%
\pgfpathcurveto{\pgfqpoint{2.618522in}{2.496810in}}{\pgfqpoint{2.606031in}{2.491636in}}{\pgfqpoint{2.596822in}{2.482427in}}%
\pgfpathcurveto{\pgfqpoint{2.587614in}{2.473219in}}{\pgfqpoint{2.582440in}{2.460728in}}{\pgfqpoint{2.582440in}{2.447705in}}%
\pgfpathcurveto{\pgfqpoint{2.582440in}{2.434682in}}{\pgfqpoint{2.587614in}{2.422191in}}{\pgfqpoint{2.596822in}{2.412983in}}%
\pgfpathcurveto{\pgfqpoint{2.606031in}{2.403774in}}{\pgfqpoint{2.618522in}{2.398600in}}{\pgfqpoint{2.631545in}{2.398600in}}%
\pgfpathlineto{\pgfqpoint{2.631545in}{2.398600in}}%
\pgfpathclose%
\pgfusepath{stroke,fill}%
\end{pgfscope}%
\begin{pgfscope}%
\pgfpathrectangle{\pgfqpoint{0.786164in}{0.768110in}}{\pgfqpoint{8.851069in}{7.081890in}}%
\pgfusepath{clip}%
\pgfsetbuttcap%
\pgfsetroundjoin%
\definecolor{currentfill}{rgb}{0.223925,0.334994,0.548053}%
\pgfsetfillcolor{currentfill}%
\pgfsetfillopacity{0.700000}%
\pgfsetlinewidth{0.501875pt}%
\definecolor{currentstroke}{rgb}{1.000000,1.000000,1.000000}%
\pgfsetstrokecolor{currentstroke}%
\pgfsetstrokeopacity{0.700000}%
\pgfsetdash{}{0pt}%
\pgfpathmoveto{\pgfqpoint{2.476279in}{2.223415in}}%
\pgfpathcurveto{\pgfqpoint{2.489301in}{2.223415in}}{\pgfqpoint{2.501793in}{2.228589in}}{\pgfqpoint{2.511001in}{2.237797in}}%
\pgfpathcurveto{\pgfqpoint{2.520209in}{2.247005in}}{\pgfqpoint{2.525383in}{2.259497in}}{\pgfqpoint{2.525383in}{2.272519in}}%
\pgfpathcurveto{\pgfqpoint{2.525383in}{2.285542in}}{\pgfqpoint{2.520209in}{2.298033in}}{\pgfqpoint{2.511001in}{2.307241in}}%
\pgfpathcurveto{\pgfqpoint{2.501793in}{2.316450in}}{\pgfqpoint{2.489301in}{2.321624in}}{\pgfqpoint{2.476279in}{2.321624in}}%
\pgfpathcurveto{\pgfqpoint{2.463256in}{2.321624in}}{\pgfqpoint{2.450765in}{2.316450in}}{\pgfqpoint{2.441557in}{2.307241in}}%
\pgfpathcurveto{\pgfqpoint{2.432348in}{2.298033in}}{\pgfqpoint{2.427174in}{2.285542in}}{\pgfqpoint{2.427174in}{2.272519in}}%
\pgfpathcurveto{\pgfqpoint{2.427174in}{2.259497in}}{\pgfqpoint{2.432348in}{2.247005in}}{\pgfqpoint{2.441557in}{2.237797in}}%
\pgfpathcurveto{\pgfqpoint{2.450765in}{2.228589in}}{\pgfqpoint{2.463256in}{2.223415in}}{\pgfqpoint{2.476279in}{2.223415in}}%
\pgfpathlineto{\pgfqpoint{2.476279in}{2.223415in}}%
\pgfpathclose%
\pgfusepath{stroke,fill}%
\end{pgfscope}%
\begin{pgfscope}%
\pgfpathrectangle{\pgfqpoint{0.786164in}{0.768110in}}{\pgfqpoint{8.851069in}{7.081890in}}%
\pgfusepath{clip}%
\pgfsetrectcap%
\pgfsetroundjoin%
\pgfsetlinewidth{0.803000pt}%
\definecolor{currentstroke}{rgb}{0.690196,0.690196,0.690196}%
\pgfsetstrokecolor{currentstroke}%
\pgfsetdash{}{0pt}%
\pgfpathmoveto{\pgfqpoint{1.699949in}{0.768110in}}%
\pgfpathlineto{\pgfqpoint{1.699949in}{7.850000in}}%
\pgfusepath{stroke}%
\end{pgfscope}%
\begin{pgfscope}%
\pgfsetbuttcap%
\pgfsetroundjoin%
\definecolor{currentfill}{rgb}{0.000000,0.000000,0.000000}%
\pgfsetfillcolor{currentfill}%
\pgfsetlinewidth{0.803000pt}%
\definecolor{currentstroke}{rgb}{0.000000,0.000000,0.000000}%
\pgfsetstrokecolor{currentstroke}%
\pgfsetdash{}{0pt}%
\pgfsys@defobject{currentmarker}{\pgfqpoint{0.000000in}{-0.048611in}}{\pgfqpoint{0.000000in}{0.000000in}}{%
\pgfpathmoveto{\pgfqpoint{0.000000in}{0.000000in}}%
\pgfpathlineto{\pgfqpoint{0.000000in}{-0.048611in}}%
\pgfusepath{stroke,fill}%
}%
\begin{pgfscope}%
\pgfsys@transformshift{1.699949in}{0.768110in}%
\pgfsys@useobject{currentmarker}{}%
\end{pgfscope}%
\end{pgfscope}%
\begin{pgfscope}%
\definecolor{textcolor}{rgb}{0.000000,0.000000,0.000000}%
\pgfsetstrokecolor{textcolor}%
\pgfsetfillcolor{textcolor}%
\pgftext[x=1.699949in,y=0.670888in,,top]{\color{textcolor}{\sffamily\fontsize{15.000000}{18.000000}\selectfont\catcode`\^=\active\def^{\ifmmode\sp\else\^{}\fi}\catcode`\%=\active\def%{\%}2}}%
\end{pgfscope}%
\begin{pgfscope}%
\pgfpathrectangle{\pgfqpoint{0.786164in}{0.768110in}}{\pgfqpoint{8.851069in}{7.081890in}}%
\pgfusepath{clip}%
\pgfsetrectcap%
\pgfsetroundjoin%
\pgfsetlinewidth{0.803000pt}%
\definecolor{currentstroke}{rgb}{0.690196,0.690196,0.690196}%
\pgfsetstrokecolor{currentstroke}%
\pgfsetdash{}{0pt}%
\pgfpathmoveto{\pgfqpoint{3.526607in}{0.768110in}}%
\pgfpathlineto{\pgfqpoint{3.526607in}{7.850000in}}%
\pgfusepath{stroke}%
\end{pgfscope}%
\begin{pgfscope}%
\pgfsetbuttcap%
\pgfsetroundjoin%
\definecolor{currentfill}{rgb}{0.000000,0.000000,0.000000}%
\pgfsetfillcolor{currentfill}%
\pgfsetlinewidth{0.803000pt}%
\definecolor{currentstroke}{rgb}{0.000000,0.000000,0.000000}%
\pgfsetstrokecolor{currentstroke}%
\pgfsetdash{}{0pt}%
\pgfsys@defobject{currentmarker}{\pgfqpoint{0.000000in}{-0.048611in}}{\pgfqpoint{0.000000in}{0.000000in}}{%
\pgfpathmoveto{\pgfqpoint{0.000000in}{0.000000in}}%
\pgfpathlineto{\pgfqpoint{0.000000in}{-0.048611in}}%
\pgfusepath{stroke,fill}%
}%
\begin{pgfscope}%
\pgfsys@transformshift{3.526607in}{0.768110in}%
\pgfsys@useobject{currentmarker}{}%
\end{pgfscope}%
\end{pgfscope}%
\begin{pgfscope}%
\definecolor{textcolor}{rgb}{0.000000,0.000000,0.000000}%
\pgfsetstrokecolor{textcolor}%
\pgfsetfillcolor{textcolor}%
\pgftext[x=3.526607in,y=0.670888in,,top]{\color{textcolor}{\sffamily\fontsize{15.000000}{18.000000}\selectfont\catcode`\^=\active\def^{\ifmmode\sp\else\^{}\fi}\catcode`\%=\active\def%{\%}4}}%
\end{pgfscope}%
\begin{pgfscope}%
\pgfpathrectangle{\pgfqpoint{0.786164in}{0.768110in}}{\pgfqpoint{8.851069in}{7.081890in}}%
\pgfusepath{clip}%
\pgfsetrectcap%
\pgfsetroundjoin%
\pgfsetlinewidth{0.803000pt}%
\definecolor{currentstroke}{rgb}{0.690196,0.690196,0.690196}%
\pgfsetstrokecolor{currentstroke}%
\pgfsetdash{}{0pt}%
\pgfpathmoveto{\pgfqpoint{5.353264in}{0.768110in}}%
\pgfpathlineto{\pgfqpoint{5.353264in}{7.850000in}}%
\pgfusepath{stroke}%
\end{pgfscope}%
\begin{pgfscope}%
\pgfsetbuttcap%
\pgfsetroundjoin%
\definecolor{currentfill}{rgb}{0.000000,0.000000,0.000000}%
\pgfsetfillcolor{currentfill}%
\pgfsetlinewidth{0.803000pt}%
\definecolor{currentstroke}{rgb}{0.000000,0.000000,0.000000}%
\pgfsetstrokecolor{currentstroke}%
\pgfsetdash{}{0pt}%
\pgfsys@defobject{currentmarker}{\pgfqpoint{0.000000in}{-0.048611in}}{\pgfqpoint{0.000000in}{0.000000in}}{%
\pgfpathmoveto{\pgfqpoint{0.000000in}{0.000000in}}%
\pgfpathlineto{\pgfqpoint{0.000000in}{-0.048611in}}%
\pgfusepath{stroke,fill}%
}%
\begin{pgfscope}%
\pgfsys@transformshift{5.353264in}{0.768110in}%
\pgfsys@useobject{currentmarker}{}%
\end{pgfscope}%
\end{pgfscope}%
\begin{pgfscope}%
\definecolor{textcolor}{rgb}{0.000000,0.000000,0.000000}%
\pgfsetstrokecolor{textcolor}%
\pgfsetfillcolor{textcolor}%
\pgftext[x=5.353264in,y=0.670888in,,top]{\color{textcolor}{\sffamily\fontsize{15.000000}{18.000000}\selectfont\catcode`\^=\active\def^{\ifmmode\sp\else\^{}\fi}\catcode`\%=\active\def%{\%}6}}%
\end{pgfscope}%
\begin{pgfscope}%
\pgfpathrectangle{\pgfqpoint{0.786164in}{0.768110in}}{\pgfqpoint{8.851069in}{7.081890in}}%
\pgfusepath{clip}%
\pgfsetrectcap%
\pgfsetroundjoin%
\pgfsetlinewidth{0.803000pt}%
\definecolor{currentstroke}{rgb}{0.690196,0.690196,0.690196}%
\pgfsetstrokecolor{currentstroke}%
\pgfsetdash{}{0pt}%
\pgfpathmoveto{\pgfqpoint{7.179922in}{0.768110in}}%
\pgfpathlineto{\pgfqpoint{7.179922in}{7.850000in}}%
\pgfusepath{stroke}%
\end{pgfscope}%
\begin{pgfscope}%
\pgfsetbuttcap%
\pgfsetroundjoin%
\definecolor{currentfill}{rgb}{0.000000,0.000000,0.000000}%
\pgfsetfillcolor{currentfill}%
\pgfsetlinewidth{0.803000pt}%
\definecolor{currentstroke}{rgb}{0.000000,0.000000,0.000000}%
\pgfsetstrokecolor{currentstroke}%
\pgfsetdash{}{0pt}%
\pgfsys@defobject{currentmarker}{\pgfqpoint{0.000000in}{-0.048611in}}{\pgfqpoint{0.000000in}{0.000000in}}{%
\pgfpathmoveto{\pgfqpoint{0.000000in}{0.000000in}}%
\pgfpathlineto{\pgfqpoint{0.000000in}{-0.048611in}}%
\pgfusepath{stroke,fill}%
}%
\begin{pgfscope}%
\pgfsys@transformshift{7.179922in}{0.768110in}%
\pgfsys@useobject{currentmarker}{}%
\end{pgfscope}%
\end{pgfscope}%
\begin{pgfscope}%
\definecolor{textcolor}{rgb}{0.000000,0.000000,0.000000}%
\pgfsetstrokecolor{textcolor}%
\pgfsetfillcolor{textcolor}%
\pgftext[x=7.179922in,y=0.670888in,,top]{\color{textcolor}{\sffamily\fontsize{15.000000}{18.000000}\selectfont\catcode`\^=\active\def^{\ifmmode\sp\else\^{}\fi}\catcode`\%=\active\def%{\%}8}}%
\end{pgfscope}%
\begin{pgfscope}%
\pgfpathrectangle{\pgfqpoint{0.786164in}{0.768110in}}{\pgfqpoint{8.851069in}{7.081890in}}%
\pgfusepath{clip}%
\pgfsetrectcap%
\pgfsetroundjoin%
\pgfsetlinewidth{0.803000pt}%
\definecolor{currentstroke}{rgb}{0.690196,0.690196,0.690196}%
\pgfsetstrokecolor{currentstroke}%
\pgfsetdash{}{0pt}%
\pgfpathmoveto{\pgfqpoint{9.006579in}{0.768110in}}%
\pgfpathlineto{\pgfqpoint{9.006579in}{7.850000in}}%
\pgfusepath{stroke}%
\end{pgfscope}%
\begin{pgfscope}%
\pgfsetbuttcap%
\pgfsetroundjoin%
\definecolor{currentfill}{rgb}{0.000000,0.000000,0.000000}%
\pgfsetfillcolor{currentfill}%
\pgfsetlinewidth{0.803000pt}%
\definecolor{currentstroke}{rgb}{0.000000,0.000000,0.000000}%
\pgfsetstrokecolor{currentstroke}%
\pgfsetdash{}{0pt}%
\pgfsys@defobject{currentmarker}{\pgfqpoint{0.000000in}{-0.048611in}}{\pgfqpoint{0.000000in}{0.000000in}}{%
\pgfpathmoveto{\pgfqpoint{0.000000in}{0.000000in}}%
\pgfpathlineto{\pgfqpoint{0.000000in}{-0.048611in}}%
\pgfusepath{stroke,fill}%
}%
\begin{pgfscope}%
\pgfsys@transformshift{9.006579in}{0.768110in}%
\pgfsys@useobject{currentmarker}{}%
\end{pgfscope}%
\end{pgfscope}%
\begin{pgfscope}%
\definecolor{textcolor}{rgb}{0.000000,0.000000,0.000000}%
\pgfsetstrokecolor{textcolor}%
\pgfsetfillcolor{textcolor}%
\pgftext[x=9.006579in,y=0.670888in,,top]{\color{textcolor}{\sffamily\fontsize{15.000000}{18.000000}\selectfont\catcode`\^=\active\def^{\ifmmode\sp\else\^{}\fi}\catcode`\%=\active\def%{\%}10}}%
\end{pgfscope}%
\begin{pgfscope}%
\definecolor{textcolor}{rgb}{0.000000,0.000000,0.000000}%
\pgfsetstrokecolor{textcolor}%
\pgfsetfillcolor{textcolor}%
\pgftext[x=5.211698in,y=0.437555in,,top]{\color{textcolor}{\sffamily\fontsize{20.000000}{24.000000}\selectfont\catcode`\^=\active\def^{\ifmmode\sp\else\^{}\fi}\catcode`\%=\active\def%{\%}Energy Consumption (tonnes of oil equivalent per capita)}}%
\end{pgfscope}%
\begin{pgfscope}%
\pgfpathrectangle{\pgfqpoint{0.786164in}{0.768110in}}{\pgfqpoint{8.851069in}{7.081890in}}%
\pgfusepath{clip}%
\pgfsetrectcap%
\pgfsetroundjoin%
\pgfsetlinewidth{0.803000pt}%
\definecolor{currentstroke}{rgb}{0.690196,0.690196,0.690196}%
\pgfsetstrokecolor{currentstroke}%
\pgfsetdash{}{0pt}%
\pgfpathmoveto{\pgfqpoint{0.786164in}{1.068116in}}%
\pgfpathlineto{\pgfqpoint{9.637233in}{1.068116in}}%
\pgfusepath{stroke}%
\end{pgfscope}%
\begin{pgfscope}%
\pgfsetbuttcap%
\pgfsetroundjoin%
\definecolor{currentfill}{rgb}{0.000000,0.000000,0.000000}%
\pgfsetfillcolor{currentfill}%
\pgfsetlinewidth{0.803000pt}%
\definecolor{currentstroke}{rgb}{0.000000,0.000000,0.000000}%
\pgfsetstrokecolor{currentstroke}%
\pgfsetdash{}{0pt}%
\pgfsys@defobject{currentmarker}{\pgfqpoint{-0.048611in}{0.000000in}}{\pgfqpoint{-0.000000in}{0.000000in}}{%
\pgfpathmoveto{\pgfqpoint{-0.000000in}{0.000000in}}%
\pgfpathlineto{\pgfqpoint{-0.048611in}{0.000000in}}%
\pgfusepath{stroke,fill}%
}%
\begin{pgfscope}%
\pgfsys@transformshift{0.786164in}{1.068116in}%
\pgfsys@useobject{currentmarker}{}%
\end{pgfscope}%
\end{pgfscope}%
\begin{pgfscope}%
\definecolor{textcolor}{rgb}{0.000000,0.000000,0.000000}%
\pgfsetstrokecolor{textcolor}%
\pgfsetfillcolor{textcolor}%
\pgftext[x=0.591026in, y=0.998672in, left, base]{\color{textcolor}{\sffamily\fontsize{15.000000}{18.000000}\selectfont\catcode`\^=\active\def^{\ifmmode\sp\else\^{}\fi}\catcode`\%=\active\def%{\%}0}}%
\end{pgfscope}%
\begin{pgfscope}%
\pgfpathrectangle{\pgfqpoint{0.786164in}{0.768110in}}{\pgfqpoint{8.851069in}{7.081890in}}%
\pgfusepath{clip}%
\pgfsetrectcap%
\pgfsetroundjoin%
\pgfsetlinewidth{0.803000pt}%
\definecolor{currentstroke}{rgb}{0.690196,0.690196,0.690196}%
\pgfsetstrokecolor{currentstroke}%
\pgfsetdash{}{0pt}%
\pgfpathmoveto{\pgfqpoint{0.786164in}{2.163028in}}%
\pgfpathlineto{\pgfqpoint{9.637233in}{2.163028in}}%
\pgfusepath{stroke}%
\end{pgfscope}%
\begin{pgfscope}%
\pgfsetbuttcap%
\pgfsetroundjoin%
\definecolor{currentfill}{rgb}{0.000000,0.000000,0.000000}%
\pgfsetfillcolor{currentfill}%
\pgfsetlinewidth{0.803000pt}%
\definecolor{currentstroke}{rgb}{0.000000,0.000000,0.000000}%
\pgfsetstrokecolor{currentstroke}%
\pgfsetdash{}{0pt}%
\pgfsys@defobject{currentmarker}{\pgfqpoint{-0.048611in}{0.000000in}}{\pgfqpoint{-0.000000in}{0.000000in}}{%
\pgfpathmoveto{\pgfqpoint{-0.000000in}{0.000000in}}%
\pgfpathlineto{\pgfqpoint{-0.048611in}{0.000000in}}%
\pgfusepath{stroke,fill}%
}%
\begin{pgfscope}%
\pgfsys@transformshift{0.786164in}{2.163028in}%
\pgfsys@useobject{currentmarker}{}%
\end{pgfscope}%
\end{pgfscope}%
\begin{pgfscope}%
\definecolor{textcolor}{rgb}{0.000000,0.000000,0.000000}%
\pgfsetstrokecolor{textcolor}%
\pgfsetfillcolor{textcolor}%
\pgftext[x=0.591026in, y=2.093584in, left, base]{\color{textcolor}{\sffamily\fontsize{15.000000}{18.000000}\selectfont\catcode`\^=\active\def^{\ifmmode\sp\else\^{}\fi}\catcode`\%=\active\def%{\%}5}}%
\end{pgfscope}%
\begin{pgfscope}%
\pgfpathrectangle{\pgfqpoint{0.786164in}{0.768110in}}{\pgfqpoint{8.851069in}{7.081890in}}%
\pgfusepath{clip}%
\pgfsetrectcap%
\pgfsetroundjoin%
\pgfsetlinewidth{0.803000pt}%
\definecolor{currentstroke}{rgb}{0.690196,0.690196,0.690196}%
\pgfsetstrokecolor{currentstroke}%
\pgfsetdash{}{0pt}%
\pgfpathmoveto{\pgfqpoint{0.786164in}{3.257940in}}%
\pgfpathlineto{\pgfqpoint{9.637233in}{3.257940in}}%
\pgfusepath{stroke}%
\end{pgfscope}%
\begin{pgfscope}%
\pgfsetbuttcap%
\pgfsetroundjoin%
\definecolor{currentfill}{rgb}{0.000000,0.000000,0.000000}%
\pgfsetfillcolor{currentfill}%
\pgfsetlinewidth{0.803000pt}%
\definecolor{currentstroke}{rgb}{0.000000,0.000000,0.000000}%
\pgfsetstrokecolor{currentstroke}%
\pgfsetdash{}{0pt}%
\pgfsys@defobject{currentmarker}{\pgfqpoint{-0.048611in}{0.000000in}}{\pgfqpoint{-0.000000in}{0.000000in}}{%
\pgfpathmoveto{\pgfqpoint{-0.000000in}{0.000000in}}%
\pgfpathlineto{\pgfqpoint{-0.048611in}{0.000000in}}%
\pgfusepath{stroke,fill}%
}%
\begin{pgfscope}%
\pgfsys@transformshift{0.786164in}{3.257940in}%
\pgfsys@useobject{currentmarker}{}%
\end{pgfscope}%
\end{pgfscope}%
\begin{pgfscope}%
\definecolor{textcolor}{rgb}{0.000000,0.000000,0.000000}%
\pgfsetstrokecolor{textcolor}%
\pgfsetfillcolor{textcolor}%
\pgftext[x=0.493111in, y=3.188496in, left, base]{\color{textcolor}{\sffamily\fontsize{15.000000}{18.000000}\selectfont\catcode`\^=\active\def^{\ifmmode\sp\else\^{}\fi}\catcode`\%=\active\def%{\%}10}}%
\end{pgfscope}%
\begin{pgfscope}%
\pgfpathrectangle{\pgfqpoint{0.786164in}{0.768110in}}{\pgfqpoint{8.851069in}{7.081890in}}%
\pgfusepath{clip}%
\pgfsetrectcap%
\pgfsetroundjoin%
\pgfsetlinewidth{0.803000pt}%
\definecolor{currentstroke}{rgb}{0.690196,0.690196,0.690196}%
\pgfsetstrokecolor{currentstroke}%
\pgfsetdash{}{0pt}%
\pgfpathmoveto{\pgfqpoint{0.786164in}{4.352852in}}%
\pgfpathlineto{\pgfqpoint{9.637233in}{4.352852in}}%
\pgfusepath{stroke}%
\end{pgfscope}%
\begin{pgfscope}%
\pgfsetbuttcap%
\pgfsetroundjoin%
\definecolor{currentfill}{rgb}{0.000000,0.000000,0.000000}%
\pgfsetfillcolor{currentfill}%
\pgfsetlinewidth{0.803000pt}%
\definecolor{currentstroke}{rgb}{0.000000,0.000000,0.000000}%
\pgfsetstrokecolor{currentstroke}%
\pgfsetdash{}{0pt}%
\pgfsys@defobject{currentmarker}{\pgfqpoint{-0.048611in}{0.000000in}}{\pgfqpoint{-0.000000in}{0.000000in}}{%
\pgfpathmoveto{\pgfqpoint{-0.000000in}{0.000000in}}%
\pgfpathlineto{\pgfqpoint{-0.048611in}{0.000000in}}%
\pgfusepath{stroke,fill}%
}%
\begin{pgfscope}%
\pgfsys@transformshift{0.786164in}{4.352852in}%
\pgfsys@useobject{currentmarker}{}%
\end{pgfscope}%
\end{pgfscope}%
\begin{pgfscope}%
\definecolor{textcolor}{rgb}{0.000000,0.000000,0.000000}%
\pgfsetstrokecolor{textcolor}%
\pgfsetfillcolor{textcolor}%
\pgftext[x=0.493111in, y=4.283407in, left, base]{\color{textcolor}{\sffamily\fontsize{15.000000}{18.000000}\selectfont\catcode`\^=\active\def^{\ifmmode\sp\else\^{}\fi}\catcode`\%=\active\def%{\%}15}}%
\end{pgfscope}%
\begin{pgfscope}%
\pgfpathrectangle{\pgfqpoint{0.786164in}{0.768110in}}{\pgfqpoint{8.851069in}{7.081890in}}%
\pgfusepath{clip}%
\pgfsetrectcap%
\pgfsetroundjoin%
\pgfsetlinewidth{0.803000pt}%
\definecolor{currentstroke}{rgb}{0.690196,0.690196,0.690196}%
\pgfsetstrokecolor{currentstroke}%
\pgfsetdash{}{0pt}%
\pgfpathmoveto{\pgfqpoint{0.786164in}{5.447763in}}%
\pgfpathlineto{\pgfqpoint{9.637233in}{5.447763in}}%
\pgfusepath{stroke}%
\end{pgfscope}%
\begin{pgfscope}%
\pgfsetbuttcap%
\pgfsetroundjoin%
\definecolor{currentfill}{rgb}{0.000000,0.000000,0.000000}%
\pgfsetfillcolor{currentfill}%
\pgfsetlinewidth{0.803000pt}%
\definecolor{currentstroke}{rgb}{0.000000,0.000000,0.000000}%
\pgfsetstrokecolor{currentstroke}%
\pgfsetdash{}{0pt}%
\pgfsys@defobject{currentmarker}{\pgfqpoint{-0.048611in}{0.000000in}}{\pgfqpoint{-0.000000in}{0.000000in}}{%
\pgfpathmoveto{\pgfqpoint{-0.000000in}{0.000000in}}%
\pgfpathlineto{\pgfqpoint{-0.048611in}{0.000000in}}%
\pgfusepath{stroke,fill}%
}%
\begin{pgfscope}%
\pgfsys@transformshift{0.786164in}{5.447763in}%
\pgfsys@useobject{currentmarker}{}%
\end{pgfscope}%
\end{pgfscope}%
\begin{pgfscope}%
\definecolor{textcolor}{rgb}{0.000000,0.000000,0.000000}%
\pgfsetstrokecolor{textcolor}%
\pgfsetfillcolor{textcolor}%
\pgftext[x=0.493111in, y=5.378319in, left, base]{\color{textcolor}{\sffamily\fontsize{15.000000}{18.000000}\selectfont\catcode`\^=\active\def^{\ifmmode\sp\else\^{}\fi}\catcode`\%=\active\def%{\%}20}}%
\end{pgfscope}%
\begin{pgfscope}%
\pgfpathrectangle{\pgfqpoint{0.786164in}{0.768110in}}{\pgfqpoint{8.851069in}{7.081890in}}%
\pgfusepath{clip}%
\pgfsetrectcap%
\pgfsetroundjoin%
\pgfsetlinewidth{0.803000pt}%
\definecolor{currentstroke}{rgb}{0.690196,0.690196,0.690196}%
\pgfsetstrokecolor{currentstroke}%
\pgfsetdash{}{0pt}%
\pgfpathmoveto{\pgfqpoint{0.786164in}{6.542675in}}%
\pgfpathlineto{\pgfqpoint{9.637233in}{6.542675in}}%
\pgfusepath{stroke}%
\end{pgfscope}%
\begin{pgfscope}%
\pgfsetbuttcap%
\pgfsetroundjoin%
\definecolor{currentfill}{rgb}{0.000000,0.000000,0.000000}%
\pgfsetfillcolor{currentfill}%
\pgfsetlinewidth{0.803000pt}%
\definecolor{currentstroke}{rgb}{0.000000,0.000000,0.000000}%
\pgfsetstrokecolor{currentstroke}%
\pgfsetdash{}{0pt}%
\pgfsys@defobject{currentmarker}{\pgfqpoint{-0.048611in}{0.000000in}}{\pgfqpoint{-0.000000in}{0.000000in}}{%
\pgfpathmoveto{\pgfqpoint{-0.000000in}{0.000000in}}%
\pgfpathlineto{\pgfqpoint{-0.048611in}{0.000000in}}%
\pgfusepath{stroke,fill}%
}%
\begin{pgfscope}%
\pgfsys@transformshift{0.786164in}{6.542675in}%
\pgfsys@useobject{currentmarker}{}%
\end{pgfscope}%
\end{pgfscope}%
\begin{pgfscope}%
\definecolor{textcolor}{rgb}{0.000000,0.000000,0.000000}%
\pgfsetstrokecolor{textcolor}%
\pgfsetfillcolor{textcolor}%
\pgftext[x=0.493111in, y=6.473231in, left, base]{\color{textcolor}{\sffamily\fontsize{15.000000}{18.000000}\selectfont\catcode`\^=\active\def^{\ifmmode\sp\else\^{}\fi}\catcode`\%=\active\def%{\%}25}}%
\end{pgfscope}%
\begin{pgfscope}%
\pgfpathrectangle{\pgfqpoint{0.786164in}{0.768110in}}{\pgfqpoint{8.851069in}{7.081890in}}%
\pgfusepath{clip}%
\pgfsetrectcap%
\pgfsetroundjoin%
\pgfsetlinewidth{0.803000pt}%
\definecolor{currentstroke}{rgb}{0.690196,0.690196,0.690196}%
\pgfsetstrokecolor{currentstroke}%
\pgfsetdash{}{0pt}%
\pgfpathmoveto{\pgfqpoint{0.786164in}{7.637587in}}%
\pgfpathlineto{\pgfqpoint{9.637233in}{7.637587in}}%
\pgfusepath{stroke}%
\end{pgfscope}%
\begin{pgfscope}%
\pgfsetbuttcap%
\pgfsetroundjoin%
\definecolor{currentfill}{rgb}{0.000000,0.000000,0.000000}%
\pgfsetfillcolor{currentfill}%
\pgfsetlinewidth{0.803000pt}%
\definecolor{currentstroke}{rgb}{0.000000,0.000000,0.000000}%
\pgfsetstrokecolor{currentstroke}%
\pgfsetdash{}{0pt}%
\pgfsys@defobject{currentmarker}{\pgfqpoint{-0.048611in}{0.000000in}}{\pgfqpoint{-0.000000in}{0.000000in}}{%
\pgfpathmoveto{\pgfqpoint{-0.000000in}{0.000000in}}%
\pgfpathlineto{\pgfqpoint{-0.048611in}{0.000000in}}%
\pgfusepath{stroke,fill}%
}%
\begin{pgfscope}%
\pgfsys@transformshift{0.786164in}{7.637587in}%
\pgfsys@useobject{currentmarker}{}%
\end{pgfscope}%
\end{pgfscope}%
\begin{pgfscope}%
\definecolor{textcolor}{rgb}{0.000000,0.000000,0.000000}%
\pgfsetstrokecolor{textcolor}%
\pgfsetfillcolor{textcolor}%
\pgftext[x=0.493111in, y=7.568143in, left, base]{\color{textcolor}{\sffamily\fontsize{15.000000}{18.000000}\selectfont\catcode`\^=\active\def^{\ifmmode\sp\else\^{}\fi}\catcode`\%=\active\def%{\%}30}}%
\end{pgfscope}%
\begin{pgfscope}%
\definecolor{textcolor}{rgb}{0.000000,0.000000,0.000000}%
\pgfsetstrokecolor{textcolor}%
\pgfsetfillcolor{textcolor}%
\pgftext[x=0.437555in,y=4.309055in,,bottom,rotate=90.000000]{\color{textcolor}{\sffamily\fontsize{20.000000}{24.000000}\selectfont\catcode`\^=\active\def^{\ifmmode\sp\else\^{}\fi}\catcode`\%=\active\def%{\%}Greenhouse Gas Emissions (tonnes per capita)}}%
\end{pgfscope}%
\begin{pgfscope}%
\pgfpathrectangle{\pgfqpoint{0.786164in}{0.768110in}}{\pgfqpoint{8.851069in}{7.081890in}}%
\pgfusepath{clip}%
\pgfsetrectcap%
\pgfsetroundjoin%
\pgfsetlinewidth{1.505625pt}%
\definecolor{currentstroke}{rgb}{1.000000,0.000000,0.000000}%
\pgfsetstrokecolor{currentstroke}%
\pgfsetdash{}{0pt}%
\pgfpathmoveto{\pgfqpoint{3.362208in}{3.250135in}}%
\pgfpathlineto{\pgfqpoint{9.234911in}{6.078857in}}%
\pgfpathlineto{\pgfqpoint{1.188485in}{2.203111in}}%
\pgfpathlineto{\pgfqpoint{2.476279in}{2.823407in}}%
\pgfusepath{stroke}%
\end{pgfscope}%
\begin{pgfscope}%
\pgfsetrectcap%
\pgfsetmiterjoin%
\pgfsetlinewidth{0.803000pt}%
\definecolor{currentstroke}{rgb}{0.000000,0.000000,0.000000}%
\pgfsetstrokecolor{currentstroke}%
\pgfsetdash{}{0pt}%
\pgfpathmoveto{\pgfqpoint{0.786164in}{0.768110in}}%
\pgfpathlineto{\pgfqpoint{0.786164in}{7.850000in}}%
\pgfusepath{stroke}%
\end{pgfscope}%
\begin{pgfscope}%
\pgfsetrectcap%
\pgfsetmiterjoin%
\pgfsetlinewidth{0.803000pt}%
\definecolor{currentstroke}{rgb}{0.000000,0.000000,0.000000}%
\pgfsetstrokecolor{currentstroke}%
\pgfsetdash{}{0pt}%
\pgfpathmoveto{\pgfqpoint{9.637233in}{0.768110in}}%
\pgfpathlineto{\pgfqpoint{9.637233in}{7.850000in}}%
\pgfusepath{stroke}%
\end{pgfscope}%
\begin{pgfscope}%
\pgfsetrectcap%
\pgfsetmiterjoin%
\pgfsetlinewidth{0.803000pt}%
\definecolor{currentstroke}{rgb}{0.000000,0.000000,0.000000}%
\pgfsetstrokecolor{currentstroke}%
\pgfsetdash{}{0pt}%
\pgfpathmoveto{\pgfqpoint{0.786164in}{0.768110in}}%
\pgfpathlineto{\pgfqpoint{9.637233in}{0.768110in}}%
\pgfusepath{stroke}%
\end{pgfscope}%
\begin{pgfscope}%
\pgfsetrectcap%
\pgfsetmiterjoin%
\pgfsetlinewidth{0.803000pt}%
\definecolor{currentstroke}{rgb}{0.000000,0.000000,0.000000}%
\pgfsetstrokecolor{currentstroke}%
\pgfsetdash{}{0pt}%
\pgfpathmoveto{\pgfqpoint{0.786164in}{7.850000in}}%
\pgfpathlineto{\pgfqpoint{9.637233in}{7.850000in}}%
\pgfusepath{stroke}%
\end{pgfscope}%
\begin{pgfscope}%
\pgfsetbuttcap%
\pgfsetmiterjoin%
\definecolor{currentfill}{rgb}{1.000000,1.000000,1.000000}%
\pgfsetfillcolor{currentfill}%
\pgfsetfillopacity{0.800000}%
\pgfsetlinewidth{1.003750pt}%
\definecolor{currentstroke}{rgb}{0.800000,0.800000,0.800000}%
\pgfsetstrokecolor{currentstroke}%
\pgfsetstrokeopacity{0.800000}%
\pgfsetdash{}{0pt}%
\pgfpathmoveto{\pgfqpoint{0.980608in}{7.232821in}}%
\pgfpathlineto{\pgfqpoint{2.834606in}{7.232821in}}%
\pgfpathquadraticcurveto{\pgfqpoint{2.890162in}{7.232821in}}{\pgfqpoint{2.890162in}{7.288377in}}%
\pgfpathlineto{\pgfqpoint{2.890162in}{7.655556in}}%
\pgfpathquadraticcurveto{\pgfqpoint{2.890162in}{7.711111in}}{\pgfqpoint{2.834606in}{7.711111in}}%
\pgfpathlineto{\pgfqpoint{0.980608in}{7.711111in}}%
\pgfpathquadraticcurveto{\pgfqpoint{0.925053in}{7.711111in}}{\pgfqpoint{0.925053in}{7.655556in}}%
\pgfpathlineto{\pgfqpoint{0.925053in}{7.288377in}}%
\pgfpathquadraticcurveto{\pgfqpoint{0.925053in}{7.232821in}}{\pgfqpoint{0.980608in}{7.232821in}}%
\pgfpathlineto{\pgfqpoint{0.980608in}{7.232821in}}%
\pgfpathclose%
\pgfusepath{stroke,fill}%
\end{pgfscope}%
\begin{pgfscope}%
\pgfsetrectcap%
\pgfsetroundjoin%
\pgfsetlinewidth{1.505625pt}%
\definecolor{currentstroke}{rgb}{1.000000,0.000000,0.000000}%
\pgfsetstrokecolor{currentstroke}%
\pgfsetdash{}{0pt}%
\pgfpathmoveto{\pgfqpoint{1.036164in}{7.497184in}}%
\pgfpathlineto{\pgfqpoint{1.313942in}{7.497184in}}%
\pgfpathlineto{\pgfqpoint{1.591719in}{7.497184in}}%
\pgfusepath{stroke}%
\end{pgfscope}%
\begin{pgfscope}%
\definecolor{textcolor}{rgb}{0.000000,0.000000,0.000000}%
\pgfsetstrokecolor{textcolor}%
\pgfsetfillcolor{textcolor}%
\pgftext[x=1.813942in,y=7.399962in,left,base]{\color{textcolor}{\sffamily\fontsize{20.000000}{24.000000}\selectfont\catcode`\^=\active\def^{\ifmmode\sp\else\^{}\fi}\catcode`\%=\active\def%{\%}r = 0.64}}%
\end{pgfscope}%
\begin{pgfscope}%
\pgfsetbuttcap%
\pgfsetmiterjoin%
\definecolor{currentfill}{rgb}{1.000000,1.000000,1.000000}%
\pgfsetfillcolor{currentfill}%
\pgfsetlinewidth{0.000000pt}%
\definecolor{currentstroke}{rgb}{0.000000,0.000000,0.000000}%
\pgfsetstrokecolor{currentstroke}%
\pgfsetstrokeopacity{0.000000}%
\pgfsetdash{}{0pt}%
\pgfpathmoveto{\pgfqpoint{10.190425in}{0.768110in}}%
\pgfpathlineto{\pgfqpoint{10.544519in}{0.768110in}}%
\pgfpathlineto{\pgfqpoint{10.544519in}{7.850000in}}%
\pgfpathlineto{\pgfqpoint{10.190425in}{7.850000in}}%
\pgfpathlineto{\pgfqpoint{10.190425in}{0.768110in}}%
\pgfpathclose%
\pgfusepath{fill}%
\end{pgfscope}%
\begin{pgfscope}%
\pgfsys@transformshift{10.190000in}{0.770000in}%
\pgftext[left,bottom]{\includegraphics[interpolate=true,width=0.350000in,height=7.080000in]{plot_global_data_with_renewables-img0.png}}%
\end{pgfscope}%
\begin{pgfscope}%
\pgfsetbuttcap%
\pgfsetroundjoin%
\definecolor{currentfill}{rgb}{0.000000,0.000000,0.000000}%
\pgfsetfillcolor{currentfill}%
\pgfsetlinewidth{0.803000pt}%
\definecolor{currentstroke}{rgb}{0.000000,0.000000,0.000000}%
\pgfsetstrokecolor{currentstroke}%
\pgfsetdash{}{0pt}%
\pgfsys@defobject{currentmarker}{\pgfqpoint{0.000000in}{0.000000in}}{\pgfqpoint{0.048611in}{0.000000in}}{%
\pgfpathmoveto{\pgfqpoint{0.000000in}{0.000000in}}%
\pgfpathlineto{\pgfqpoint{0.048611in}{0.000000in}}%
\pgfusepath{stroke,fill}%
}%
\begin{pgfscope}%
\pgfsys@transformshift{10.544519in}{1.831791in}%
\pgfsys@useobject{currentmarker}{}%
\end{pgfscope}%
\end{pgfscope}%
\begin{pgfscope}%
\definecolor{textcolor}{rgb}{0.000000,0.000000,0.000000}%
\pgfsetstrokecolor{textcolor}%
\pgfsetfillcolor{textcolor}%
\pgftext[x=10.641741in, y=1.762347in, left, base]{\color{textcolor}{\sffamily\fontsize{15.000000}{18.000000}\selectfont\catcode`\^=\active\def^{\ifmmode\sp\else\^{}\fi}\catcode`\%=\active\def%{\%}10}}%
\end{pgfscope}%
\begin{pgfscope}%
\pgfsetbuttcap%
\pgfsetroundjoin%
\definecolor{currentfill}{rgb}{0.000000,0.000000,0.000000}%
\pgfsetfillcolor{currentfill}%
\pgfsetlinewidth{0.803000pt}%
\definecolor{currentstroke}{rgb}{0.000000,0.000000,0.000000}%
\pgfsetstrokecolor{currentstroke}%
\pgfsetdash{}{0pt}%
\pgfsys@defobject{currentmarker}{\pgfqpoint{0.000000in}{0.000000in}}{\pgfqpoint{0.048611in}{0.000000in}}{%
\pgfpathmoveto{\pgfqpoint{0.000000in}{0.000000in}}%
\pgfpathlineto{\pgfqpoint{0.048611in}{0.000000in}}%
\pgfusepath{stroke,fill}%
}%
\begin{pgfscope}%
\pgfsys@transformshift{10.544519in}{2.906433in}%
\pgfsys@useobject{currentmarker}{}%
\end{pgfscope}%
\end{pgfscope}%
\begin{pgfscope}%
\definecolor{textcolor}{rgb}{0.000000,0.000000,0.000000}%
\pgfsetstrokecolor{textcolor}%
\pgfsetfillcolor{textcolor}%
\pgftext[x=10.641741in, y=2.836988in, left, base]{\color{textcolor}{\sffamily\fontsize{15.000000}{18.000000}\selectfont\catcode`\^=\active\def^{\ifmmode\sp\else\^{}\fi}\catcode`\%=\active\def%{\%}20}}%
\end{pgfscope}%
\begin{pgfscope}%
\pgfsetbuttcap%
\pgfsetroundjoin%
\definecolor{currentfill}{rgb}{0.000000,0.000000,0.000000}%
\pgfsetfillcolor{currentfill}%
\pgfsetlinewidth{0.803000pt}%
\definecolor{currentstroke}{rgb}{0.000000,0.000000,0.000000}%
\pgfsetstrokecolor{currentstroke}%
\pgfsetdash{}{0pt}%
\pgfsys@defobject{currentmarker}{\pgfqpoint{0.000000in}{0.000000in}}{\pgfqpoint{0.048611in}{0.000000in}}{%
\pgfpathmoveto{\pgfqpoint{0.000000in}{0.000000in}}%
\pgfpathlineto{\pgfqpoint{0.048611in}{0.000000in}}%
\pgfusepath{stroke,fill}%
}%
\begin{pgfscope}%
\pgfsys@transformshift{10.544519in}{3.981075in}%
\pgfsys@useobject{currentmarker}{}%
\end{pgfscope}%
\end{pgfscope}%
\begin{pgfscope}%
\definecolor{textcolor}{rgb}{0.000000,0.000000,0.000000}%
\pgfsetstrokecolor{textcolor}%
\pgfsetfillcolor{textcolor}%
\pgftext[x=10.641741in, y=3.911630in, left, base]{\color{textcolor}{\sffamily\fontsize{15.000000}{18.000000}\selectfont\catcode`\^=\active\def^{\ifmmode\sp\else\^{}\fi}\catcode`\%=\active\def%{\%}30}}%
\end{pgfscope}%
\begin{pgfscope}%
\pgfsetbuttcap%
\pgfsetroundjoin%
\definecolor{currentfill}{rgb}{0.000000,0.000000,0.000000}%
\pgfsetfillcolor{currentfill}%
\pgfsetlinewidth{0.803000pt}%
\definecolor{currentstroke}{rgb}{0.000000,0.000000,0.000000}%
\pgfsetstrokecolor{currentstroke}%
\pgfsetdash{}{0pt}%
\pgfsys@defobject{currentmarker}{\pgfqpoint{0.000000in}{0.000000in}}{\pgfqpoint{0.048611in}{0.000000in}}{%
\pgfpathmoveto{\pgfqpoint{0.000000in}{0.000000in}}%
\pgfpathlineto{\pgfqpoint{0.048611in}{0.000000in}}%
\pgfusepath{stroke,fill}%
}%
\begin{pgfscope}%
\pgfsys@transformshift{10.544519in}{5.055716in}%
\pgfsys@useobject{currentmarker}{}%
\end{pgfscope}%
\end{pgfscope}%
\begin{pgfscope}%
\definecolor{textcolor}{rgb}{0.000000,0.000000,0.000000}%
\pgfsetstrokecolor{textcolor}%
\pgfsetfillcolor{textcolor}%
\pgftext[x=10.641741in, y=4.986272in, left, base]{\color{textcolor}{\sffamily\fontsize{15.000000}{18.000000}\selectfont\catcode`\^=\active\def^{\ifmmode\sp\else\^{}\fi}\catcode`\%=\active\def%{\%}40}}%
\end{pgfscope}%
\begin{pgfscope}%
\pgfsetbuttcap%
\pgfsetroundjoin%
\definecolor{currentfill}{rgb}{0.000000,0.000000,0.000000}%
\pgfsetfillcolor{currentfill}%
\pgfsetlinewidth{0.803000pt}%
\definecolor{currentstroke}{rgb}{0.000000,0.000000,0.000000}%
\pgfsetstrokecolor{currentstroke}%
\pgfsetdash{}{0pt}%
\pgfsys@defobject{currentmarker}{\pgfqpoint{0.000000in}{0.000000in}}{\pgfqpoint{0.048611in}{0.000000in}}{%
\pgfpathmoveto{\pgfqpoint{0.000000in}{0.000000in}}%
\pgfpathlineto{\pgfqpoint{0.048611in}{0.000000in}}%
\pgfusepath{stroke,fill}%
}%
\begin{pgfscope}%
\pgfsys@transformshift{10.544519in}{6.130358in}%
\pgfsys@useobject{currentmarker}{}%
\end{pgfscope}%
\end{pgfscope}%
\begin{pgfscope}%
\definecolor{textcolor}{rgb}{0.000000,0.000000,0.000000}%
\pgfsetstrokecolor{textcolor}%
\pgfsetfillcolor{textcolor}%
\pgftext[x=10.641741in, y=6.060914in, left, base]{\color{textcolor}{\sffamily\fontsize{15.000000}{18.000000}\selectfont\catcode`\^=\active\def^{\ifmmode\sp\else\^{}\fi}\catcode`\%=\active\def%{\%}50}}%
\end{pgfscope}%
\begin{pgfscope}%
\pgfsetbuttcap%
\pgfsetroundjoin%
\definecolor{currentfill}{rgb}{0.000000,0.000000,0.000000}%
\pgfsetfillcolor{currentfill}%
\pgfsetlinewidth{0.803000pt}%
\definecolor{currentstroke}{rgb}{0.000000,0.000000,0.000000}%
\pgfsetstrokecolor{currentstroke}%
\pgfsetdash{}{0pt}%
\pgfsys@defobject{currentmarker}{\pgfqpoint{0.000000in}{0.000000in}}{\pgfqpoint{0.048611in}{0.000000in}}{%
\pgfpathmoveto{\pgfqpoint{0.000000in}{0.000000in}}%
\pgfpathlineto{\pgfqpoint{0.048611in}{0.000000in}}%
\pgfusepath{stroke,fill}%
}%
\begin{pgfscope}%
\pgfsys@transformshift{10.544519in}{7.205000in}%
\pgfsys@useobject{currentmarker}{}%
\end{pgfscope}%
\end{pgfscope}%
\begin{pgfscope}%
\definecolor{textcolor}{rgb}{0.000000,0.000000,0.000000}%
\pgfsetstrokecolor{textcolor}%
\pgfsetfillcolor{textcolor}%
\pgftext[x=10.641741in, y=7.135556in, left, base]{\color{textcolor}{\sffamily\fontsize{15.000000}{18.000000}\selectfont\catcode`\^=\active\def^{\ifmmode\sp\else\^{}\fi}\catcode`\%=\active\def%{\%}60}}%
\end{pgfscope}%
\begin{pgfscope}%
\definecolor{textcolor}{rgb}{0.000000,0.000000,0.000000}%
\pgfsetstrokecolor{textcolor}%
\pgfsetfillcolor{textcolor}%
\pgftext[x=10.893128in,y=4.309055in,,top,rotate=90.000000]{\color{textcolor}{\sffamily\fontsize{20.000000}{24.000000}\selectfont\catcode`\^=\active\def^{\ifmmode\sp\else\^{}\fi}\catcode`\%=\active\def%{\%}Share of Renewable Energy (%)}}%
\end{pgfscope}%
\begin{pgfscope}%
\pgfsetrectcap%
\pgfsetmiterjoin%
\pgfsetlinewidth{0.803000pt}%
\definecolor{currentstroke}{rgb}{0.000000,0.000000,0.000000}%
\pgfsetstrokecolor{currentstroke}%
\pgfsetdash{}{0pt}%
\pgfpathmoveto{\pgfqpoint{10.190425in}{0.768110in}}%
\pgfpathlineto{\pgfqpoint{10.367472in}{0.768110in}}%
\pgfpathlineto{\pgfqpoint{10.544519in}{0.768110in}}%
\pgfpathlineto{\pgfqpoint{10.544519in}{7.850000in}}%
\pgfpathlineto{\pgfqpoint{10.367472in}{7.850000in}}%
\pgfpathlineto{\pgfqpoint{10.190425in}{7.850000in}}%
\pgfpathlineto{\pgfqpoint{10.190425in}{0.768110in}}%
\pgfpathclose%
\pgfusepath{stroke}%
\end{pgfscope}%
\end{pgfpicture}%
\makeatother%
\endgroup%
}
    \caption{Global Correlation between Energy Consumption and Greenhouse Gas Emissions}
    \label{plt:global_consumption_vs_emissions}
\end{figure}

% \begin{figure}
%     \centering
%     \begin{subfigure}[b]{0.49\textwidth}
%         \centering
%         \resizebox{\textwidth}{!}{\input{plot_country_data_with_renewables_LU.pgf}}
%         \caption{Luxembourg}
%         \label{plt:LU_consumption_vs_emissions}
%     \end{subfigure}
%     \hfill
%     \begin{subfigure}[b]{0.49\textwidth}
%         \centering
%         \resizebox{\textwidth}{!}{\input{plot_country_data_with_renewables_AT.pgf}}
%         \caption{Austria}
%         \label{plt:AT_consumption_vs_emissions}
%     \end{subfigure}
%     \caption{National Correlation between Energy Consumption and Greenhouse Gas Emissions}
% \end{figure}

\subsubsection*{Interpretation of the Findings}
The PCC of .64 for the whole EU indicates that higher energy consumption is generally associated with increased greenhouse gas emissions at the EU level.

However, the analysis of individual countries presents a more nuanced picture. While Luxembourg demonstrates a perfect positive correlation,
suggesting that its emissions are almost entirely driven by energy consumption, other countries like Austria show only a moderate correlation.
Slovenia stands out as an exception with a negative correlation, indicating other factors significantly influencing its emissions beyond energy consumption.

The median PCC across all countries is 0.86, higher than the EU-wide correlation. This disparity underscores the importance of considering national contexts
and policies when analyzing environmental impacts.

\subsection*{Correlation between Emissions and the Share of Renewables}
Secondly the report analyzes the follow-up-question considering the correlation between the share of renewable energy sources and the greenhouse
gas emissions.

\subsubsection*{Method Used}
Like with the first question, all datasets had to be extracted from the SQLite database and were merged on the country codes
and years, resulting in a unified dataset for analysis.

The method for answering the question is also strongly inflienced by the first question's method.
A scatter plot to visualize the relationship between the share of renewable energy and greenhouse gas emissions was created.
The plot includes data points for all years and countries of the EU. Each data point's color represents the level of energy consumption per capita,
which is indicated by a color bar included as a legend in the plot.

Also a trend line was generated for the plot using linear regression and the PCC was calculated for the EU as a whole.
Furthermore, the PCC was calculated for all single countries in order to assess the median of the countries PCC, as it proved to be
rather informative in analyzing the first question.

\subsubsection*{Results}
The resulting global plot in \cref{plt:global_share_vs_emissions} indicates a significant negative correlation between the share
of renewable energy and greenhouse gas emissions. The $r$-value is calculated to be -.54, suggesting a moderate to strong
negative correlation. This implies that a higher share of renewable energy is generally associated with lower greenhouse gas emissions across the EU.

In regard to the median over all national countries, the correlation becomes even far stronger with a $r$-value of -.9.
This high median value indicates that most individual countries exhibit a very strong negative correlation,
suggesting that an increased share of renewable energy is generally associated with substantial reductions in greenhouse gas emissions.

\begin{figure}
    \centering
    \resizebox{.7\textwidth}{!}{%% Creator: Matplotlib, PGF backend
%%
%% To include the figure in your LaTeX document, write
%%   \input{<filename>.pgf}
%%
%% Make sure the required packages are loaded in your preamble
%%   \usepackage{pgf}
%%
%% Also ensure that all the required font packages are loaded; for instance,
%% the lmodern package is sometimes necessary when using math font.
%%   \usepackage{lmodern}
%%
%% Figures using additional raster images can only be included by \input if
%% they are in the same directory as the main LaTeX file. For loading figures
%% from other directories you can use the `import` package
%%   \usepackage{import}
%%
%% and then include the figures with
%%   \import{<path to file>}{<filename>.pgf}
%%
%% Matplotlib used the following preamble
%%   \def\mathdefault#1{#1}
%%   \everymath=\expandafter{\the\everymath\displaystyle}
%%   
%%   \makeatletter\@ifpackageloaded{underscore}{}{\usepackage[strings]{underscore}}\makeatother
%%
\begingroup%
\makeatletter%
\begin{pgfpicture}%
\pgfpathrectangle{\pgfpointorigin}{\pgfqpoint{12.000000in}{8.000000in}}%
\pgfusepath{use as bounding box, clip}%
\begin{pgfscope}%
\pgfsetbuttcap%
\pgfsetmiterjoin%
\definecolor{currentfill}{rgb}{1.000000,1.000000,1.000000}%
\pgfsetfillcolor{currentfill}%
\pgfsetlinewidth{0.000000pt}%
\definecolor{currentstroke}{rgb}{1.000000,1.000000,1.000000}%
\pgfsetstrokecolor{currentstroke}%
\pgfsetdash{}{0pt}%
\pgfpathmoveto{\pgfqpoint{0.000000in}{0.000000in}}%
\pgfpathlineto{\pgfqpoint{12.000000in}{0.000000in}}%
\pgfpathlineto{\pgfqpoint{12.000000in}{8.000000in}}%
\pgfpathlineto{\pgfqpoint{0.000000in}{8.000000in}}%
\pgfpathlineto{\pgfqpoint{0.000000in}{0.000000in}}%
\pgfpathclose%
\pgfusepath{fill}%
\end{pgfscope}%
\begin{pgfscope}%
\pgfsetbuttcap%
\pgfsetmiterjoin%
\definecolor{currentfill}{rgb}{1.000000,1.000000,1.000000}%
\pgfsetfillcolor{currentfill}%
\pgfsetlinewidth{0.000000pt}%
\definecolor{currentstroke}{rgb}{0.000000,0.000000,0.000000}%
\pgfsetstrokecolor{currentstroke}%
\pgfsetstrokeopacity{0.000000}%
\pgfsetdash{}{0pt}%
\pgfpathmoveto{\pgfqpoint{0.786164in}{0.768110in}}%
\pgfpathlineto{\pgfqpoint{9.637233in}{0.768110in}}%
\pgfpathlineto{\pgfqpoint{9.637233in}{7.850000in}}%
\pgfpathlineto{\pgfqpoint{0.786164in}{7.850000in}}%
\pgfpathlineto{\pgfqpoint{0.786164in}{0.768110in}}%
\pgfpathclose%
\pgfusepath{fill}%
\end{pgfscope}%
\begin{pgfscope}%
\pgfpathrectangle{\pgfqpoint{0.786164in}{0.768110in}}{\pgfqpoint{8.851069in}{7.081890in}}%
\pgfusepath{clip}%
\pgfsetbuttcap%
\pgfsetroundjoin%
\definecolor{currentfill}{rgb}{0.220057,0.343307,0.549413}%
\pgfsetfillcolor{currentfill}%
\pgfsetfillopacity{0.700000}%
\pgfsetlinewidth{0.501875pt}%
\definecolor{currentstroke}{rgb}{1.000000,1.000000,1.000000}%
\pgfsetstrokecolor{currentstroke}%
\pgfsetstrokeopacity{0.700000}%
\pgfsetdash{}{0pt}%
\pgfpathmoveto{\pgfqpoint{3.929765in}{3.039820in}}%
\pgfpathcurveto{\pgfqpoint{3.942787in}{3.039820in}}{\pgfqpoint{3.955278in}{3.044994in}}{\pgfqpoint{3.964487in}{3.054202in}}%
\pgfpathcurveto{\pgfqpoint{3.973695in}{3.063411in}}{\pgfqpoint{3.978869in}{3.075902in}}{\pgfqpoint{3.978869in}{3.088924in}}%
\pgfpathcurveto{\pgfqpoint{3.978869in}{3.101947in}}{\pgfqpoint{3.973695in}{3.114438in}}{\pgfqpoint{3.964487in}{3.123647in}}%
\pgfpathcurveto{\pgfqpoint{3.955278in}{3.132855in}}{\pgfqpoint{3.942787in}{3.138029in}}{\pgfqpoint{3.929765in}{3.138029in}}%
\pgfpathcurveto{\pgfqpoint{3.916742in}{3.138029in}}{\pgfqpoint{3.904251in}{3.132855in}}{\pgfqpoint{3.895042in}{3.123647in}}%
\pgfpathcurveto{\pgfqpoint{3.885834in}{3.114438in}}{\pgfqpoint{3.880660in}{3.101947in}}{\pgfqpoint{3.880660in}{3.088924in}}%
\pgfpathcurveto{\pgfqpoint{3.880660in}{3.075902in}}{\pgfqpoint{3.885834in}{3.063411in}}{\pgfqpoint{3.895042in}{3.054202in}}%
\pgfpathcurveto{\pgfqpoint{3.904251in}{3.044994in}}{\pgfqpoint{3.916742in}{3.039820in}}{\pgfqpoint{3.929765in}{3.039820in}}%
\pgfpathlineto{\pgfqpoint{3.929765in}{3.039820in}}%
\pgfpathclose%
\pgfusepath{stroke,fill}%
\end{pgfscope}%
\begin{pgfscope}%
\pgfpathrectangle{\pgfqpoint{0.786164in}{0.768110in}}{\pgfqpoint{8.851069in}{7.081890in}}%
\pgfusepath{clip}%
\pgfsetbuttcap%
\pgfsetroundjoin%
\definecolor{currentfill}{rgb}{0.212395,0.359683,0.551710}%
\pgfsetfillcolor{currentfill}%
\pgfsetfillopacity{0.700000}%
\pgfsetlinewidth{0.501875pt}%
\definecolor{currentstroke}{rgb}{1.000000,1.000000,1.000000}%
\pgfsetstrokecolor{currentstroke}%
\pgfsetstrokeopacity{0.700000}%
\pgfsetdash{}{0pt}%
\pgfpathmoveto{\pgfqpoint{4.149546in}{3.167873in}}%
\pgfpathcurveto{\pgfqpoint{4.162568in}{3.167873in}}{\pgfqpoint{4.175059in}{3.173047in}}{\pgfqpoint{4.184268in}{3.182255in}}%
\pgfpathcurveto{\pgfqpoint{4.193476in}{3.191464in}}{\pgfqpoint{4.198650in}{3.203955in}}{\pgfqpoint{4.198650in}{3.216977in}}%
\pgfpathcurveto{\pgfqpoint{4.198650in}{3.230000in}}{\pgfqpoint{4.193476in}{3.242491in}}{\pgfqpoint{4.184268in}{3.251700in}}%
\pgfpathcurveto{\pgfqpoint{4.175059in}{3.260908in}}{\pgfqpoint{4.162568in}{3.266082in}}{\pgfqpoint{4.149546in}{3.266082in}}%
\pgfpathcurveto{\pgfqpoint{4.136523in}{3.266082in}}{\pgfqpoint{4.124032in}{3.260908in}}{\pgfqpoint{4.114823in}{3.251700in}}%
\pgfpathcurveto{\pgfqpoint{4.105615in}{3.242491in}}{\pgfqpoint{4.100441in}{3.230000in}}{\pgfqpoint{4.100441in}{3.216977in}}%
\pgfpathcurveto{\pgfqpoint{4.100441in}{3.203955in}}{\pgfqpoint{4.105615in}{3.191464in}}{\pgfqpoint{4.114823in}{3.182255in}}%
\pgfpathcurveto{\pgfqpoint{4.124032in}{3.173047in}}{\pgfqpoint{4.136523in}{3.167873in}}{\pgfqpoint{4.149546in}{3.167873in}}%
\pgfpathlineto{\pgfqpoint{4.149546in}{3.167873in}}%
\pgfpathclose%
\pgfusepath{stroke,fill}%
\end{pgfscope}%
\begin{pgfscope}%
\pgfpathrectangle{\pgfqpoint{0.786164in}{0.768110in}}{\pgfqpoint{8.851069in}{7.081890in}}%
\pgfusepath{clip}%
\pgfsetbuttcap%
\pgfsetroundjoin%
\definecolor{currentfill}{rgb}{0.214298,0.355619,0.551184}%
\pgfsetfillcolor{currentfill}%
\pgfsetfillopacity{0.700000}%
\pgfsetlinewidth{0.501875pt}%
\definecolor{currentstroke}{rgb}{1.000000,1.000000,1.000000}%
\pgfsetstrokecolor{currentstroke}%
\pgfsetstrokeopacity{0.700000}%
\pgfsetdash{}{0pt}%
\pgfpathmoveto{\pgfqpoint{4.384345in}{3.338610in}}%
\pgfpathcurveto{\pgfqpoint{4.397368in}{3.338610in}}{\pgfqpoint{4.409859in}{3.343784in}}{\pgfqpoint{4.419067in}{3.352993in}}%
\pgfpathcurveto{\pgfqpoint{4.428276in}{3.362201in}}{\pgfqpoint{4.433450in}{3.374692in}}{\pgfqpoint{4.433450in}{3.387715in}}%
\pgfpathcurveto{\pgfqpoint{4.433450in}{3.400738in}}{\pgfqpoint{4.428276in}{3.413229in}}{\pgfqpoint{4.419067in}{3.422437in}}%
\pgfpathcurveto{\pgfqpoint{4.409859in}{3.431645in}}{\pgfqpoint{4.397368in}{3.436819in}}{\pgfqpoint{4.384345in}{3.436819in}}%
\pgfpathcurveto{\pgfqpoint{4.371322in}{3.436819in}}{\pgfqpoint{4.358831in}{3.431645in}}{\pgfqpoint{4.349623in}{3.422437in}}%
\pgfpathcurveto{\pgfqpoint{4.340414in}{3.413229in}}{\pgfqpoint{4.335240in}{3.400738in}}{\pgfqpoint{4.335240in}{3.387715in}}%
\pgfpathcurveto{\pgfqpoint{4.335240in}{3.374692in}}{\pgfqpoint{4.340414in}{3.362201in}}{\pgfqpoint{4.349623in}{3.352993in}}%
\pgfpathcurveto{\pgfqpoint{4.358831in}{3.343784in}}{\pgfqpoint{4.371322in}{3.338610in}}{\pgfqpoint{4.384345in}{3.338610in}}%
\pgfpathlineto{\pgfqpoint{4.384345in}{3.338610in}}%
\pgfpathclose%
\pgfusepath{stroke,fill}%
\end{pgfscope}%
\begin{pgfscope}%
\pgfpathrectangle{\pgfqpoint{0.786164in}{0.768110in}}{\pgfqpoint{8.851069in}{7.081890in}}%
\pgfusepath{clip}%
\pgfsetbuttcap%
\pgfsetroundjoin%
\definecolor{currentfill}{rgb}{0.218130,0.347432,0.550038}%
\pgfsetfillcolor{currentfill}%
\pgfsetfillopacity{0.700000}%
\pgfsetlinewidth{0.501875pt}%
\definecolor{currentstroke}{rgb}{1.000000,1.000000,1.000000}%
\pgfsetstrokecolor{currentstroke}%
\pgfsetstrokeopacity{0.700000}%
\pgfsetdash{}{0pt}%
\pgfpathmoveto{\pgfqpoint{4.612429in}{3.359952in}}%
\pgfpathcurveto{\pgfqpoint{4.625452in}{3.359952in}}{\pgfqpoint{4.637943in}{3.365126in}}{\pgfqpoint{4.647151in}{3.374335in}}%
\pgfpathcurveto{\pgfqpoint{4.656359in}{3.383543in}}{\pgfqpoint{4.661533in}{3.396034in}}{\pgfqpoint{4.661533in}{3.409057in}}%
\pgfpathcurveto{\pgfqpoint{4.661533in}{3.422080in}}{\pgfqpoint{4.656359in}{3.434571in}}{\pgfqpoint{4.647151in}{3.443779in}}%
\pgfpathcurveto{\pgfqpoint{4.637943in}{3.452988in}}{\pgfqpoint{4.625452in}{3.458162in}}{\pgfqpoint{4.612429in}{3.458162in}}%
\pgfpathcurveto{\pgfqpoint{4.599406in}{3.458162in}}{\pgfqpoint{4.586915in}{3.452988in}}{\pgfqpoint{4.577707in}{3.443779in}}%
\pgfpathcurveto{\pgfqpoint{4.568498in}{3.434571in}}{\pgfqpoint{4.563324in}{3.422080in}}{\pgfqpoint{4.563324in}{3.409057in}}%
\pgfpathcurveto{\pgfqpoint{4.563324in}{3.396034in}}{\pgfqpoint{4.568498in}{3.383543in}}{\pgfqpoint{4.577707in}{3.374335in}}%
\pgfpathcurveto{\pgfqpoint{4.586915in}{3.365126in}}{\pgfqpoint{4.599406in}{3.359952in}}{\pgfqpoint{4.612429in}{3.359952in}}%
\pgfpathlineto{\pgfqpoint{4.612429in}{3.359952in}}%
\pgfpathclose%
\pgfusepath{stroke,fill}%
\end{pgfscope}%
\begin{pgfscope}%
\pgfpathrectangle{\pgfqpoint{0.786164in}{0.768110in}}{\pgfqpoint{8.851069in}{7.081890in}}%
\pgfusepath{clip}%
\pgfsetbuttcap%
\pgfsetroundjoin%
\definecolor{currentfill}{rgb}{0.216210,0.351535,0.550627}%
\pgfsetfillcolor{currentfill}%
\pgfsetfillopacity{0.700000}%
\pgfsetlinewidth{0.501875pt}%
\definecolor{currentstroke}{rgb}{1.000000,1.000000,1.000000}%
\pgfsetstrokecolor{currentstroke}%
\pgfsetstrokeopacity{0.700000}%
\pgfsetdash{}{0pt}%
\pgfpathmoveto{\pgfqpoint{4.691062in}{3.167873in}}%
\pgfpathcurveto{\pgfqpoint{4.704084in}{3.167873in}}{\pgfqpoint{4.716575in}{3.173047in}}{\pgfqpoint{4.725784in}{3.182255in}}%
\pgfpathcurveto{\pgfqpoint{4.734992in}{3.191464in}}{\pgfqpoint{4.740166in}{3.203955in}}{\pgfqpoint{4.740166in}{3.216977in}}%
\pgfpathcurveto{\pgfqpoint{4.740166in}{3.230000in}}{\pgfqpoint{4.734992in}{3.242491in}}{\pgfqpoint{4.725784in}{3.251700in}}%
\pgfpathcurveto{\pgfqpoint{4.716575in}{3.260908in}}{\pgfqpoint{4.704084in}{3.266082in}}{\pgfqpoint{4.691062in}{3.266082in}}%
\pgfpathcurveto{\pgfqpoint{4.678039in}{3.266082in}}{\pgfqpoint{4.665548in}{3.260908in}}{\pgfqpoint{4.656339in}{3.251700in}}%
\pgfpathcurveto{\pgfqpoint{4.647131in}{3.242491in}}{\pgfqpoint{4.641957in}{3.230000in}}{\pgfqpoint{4.641957in}{3.216977in}}%
\pgfpathcurveto{\pgfqpoint{4.641957in}{3.203955in}}{\pgfqpoint{4.647131in}{3.191464in}}{\pgfqpoint{4.656339in}{3.182255in}}%
\pgfpathcurveto{\pgfqpoint{4.665548in}{3.173047in}}{\pgfqpoint{4.678039in}{3.167873in}}{\pgfqpoint{4.691062in}{3.167873in}}%
\pgfpathlineto{\pgfqpoint{4.691062in}{3.167873in}}%
\pgfpathclose%
\pgfusepath{stroke,fill}%
\end{pgfscope}%
\begin{pgfscope}%
\pgfpathrectangle{\pgfqpoint{0.786164in}{0.768110in}}{\pgfqpoint{8.851069in}{7.081890in}}%
\pgfusepath{clip}%
\pgfsetbuttcap%
\pgfsetroundjoin%
\definecolor{currentfill}{rgb}{0.229739,0.322361,0.545706}%
\pgfsetfillcolor{currentfill}%
\pgfsetfillopacity{0.700000}%
\pgfsetlinewidth{0.501875pt}%
\definecolor{currentstroke}{rgb}{1.000000,1.000000,1.000000}%
\pgfsetstrokecolor{currentstroke}%
\pgfsetstrokeopacity{0.700000}%
\pgfsetdash{}{0pt}%
\pgfpathmoveto{\pgfqpoint{4.965910in}{3.082504in}}%
\pgfpathcurveto{\pgfqpoint{4.978933in}{3.082504in}}{\pgfqpoint{4.991424in}{3.087678in}}{\pgfqpoint{5.000632in}{3.096887in}}%
\pgfpathcurveto{\pgfqpoint{5.009841in}{3.106095in}}{\pgfqpoint{5.015015in}{3.118586in}}{\pgfqpoint{5.015015in}{3.131609in}}%
\pgfpathcurveto{\pgfqpoint{5.015015in}{3.144631in}}{\pgfqpoint{5.009841in}{3.157123in}}{\pgfqpoint{5.000632in}{3.166331in}}%
\pgfpathcurveto{\pgfqpoint{4.991424in}{3.175539in}}{\pgfqpoint{4.978933in}{3.180713in}}{\pgfqpoint{4.965910in}{3.180713in}}%
\pgfpathcurveto{\pgfqpoint{4.952887in}{3.180713in}}{\pgfqpoint{4.940396in}{3.175539in}}{\pgfqpoint{4.931188in}{3.166331in}}%
\pgfpathcurveto{\pgfqpoint{4.921979in}{3.157123in}}{\pgfqpoint{4.916805in}{3.144631in}}{\pgfqpoint{4.916805in}{3.131609in}}%
\pgfpathcurveto{\pgfqpoint{4.916805in}{3.118586in}}{\pgfqpoint{4.921979in}{3.106095in}}{\pgfqpoint{4.931188in}{3.096887in}}%
\pgfpathcurveto{\pgfqpoint{4.940396in}{3.087678in}}{\pgfqpoint{4.952887in}{3.082504in}}{\pgfqpoint{4.965910in}{3.082504in}}%
\pgfpathlineto{\pgfqpoint{4.965910in}{3.082504in}}%
\pgfpathclose%
\pgfusepath{stroke,fill}%
\end{pgfscope}%
\begin{pgfscope}%
\pgfpathrectangle{\pgfqpoint{0.786164in}{0.768110in}}{\pgfqpoint{8.851069in}{7.081890in}}%
\pgfusepath{clip}%
\pgfsetbuttcap%
\pgfsetroundjoin%
\definecolor{currentfill}{rgb}{0.214298,0.355619,0.551184}%
\pgfsetfillcolor{currentfill}%
\pgfsetfillopacity{0.700000}%
\pgfsetlinewidth{0.501875pt}%
\definecolor{currentstroke}{rgb}{1.000000,1.000000,1.000000}%
\pgfsetstrokecolor{currentstroke}%
\pgfsetstrokeopacity{0.700000}%
\pgfsetdash{}{0pt}%
\pgfpathmoveto{\pgfqpoint{4.986179in}{2.911767in}}%
\pgfpathcurveto{\pgfqpoint{4.999201in}{2.911767in}}{\pgfqpoint{5.011692in}{2.916941in}}{\pgfqpoint{5.020901in}{2.926149in}}%
\pgfpathcurveto{\pgfqpoint{5.030109in}{2.935358in}}{\pgfqpoint{5.035283in}{2.947849in}}{\pgfqpoint{5.035283in}{2.960871in}}%
\pgfpathcurveto{\pgfqpoint{5.035283in}{2.973894in}}{\pgfqpoint{5.030109in}{2.986385in}}{\pgfqpoint{5.020901in}{2.995594in}}%
\pgfpathcurveto{\pgfqpoint{5.011692in}{3.004802in}}{\pgfqpoint{4.999201in}{3.009976in}}{\pgfqpoint{4.986179in}{3.009976in}}%
\pgfpathcurveto{\pgfqpoint{4.973156in}{3.009976in}}{\pgfqpoint{4.960665in}{3.004802in}}{\pgfqpoint{4.951456in}{2.995594in}}%
\pgfpathcurveto{\pgfqpoint{4.942248in}{2.986385in}}{\pgfqpoint{4.937074in}{2.973894in}}{\pgfqpoint{4.937074in}{2.960871in}}%
\pgfpathcurveto{\pgfqpoint{4.937074in}{2.947849in}}{\pgfqpoint{4.942248in}{2.935358in}}{\pgfqpoint{4.951456in}{2.926149in}}%
\pgfpathcurveto{\pgfqpoint{4.960665in}{2.916941in}}{\pgfqpoint{4.973156in}{2.911767in}}{\pgfqpoint{4.986179in}{2.911767in}}%
\pgfpathlineto{\pgfqpoint{4.986179in}{2.911767in}}%
\pgfpathclose%
\pgfusepath{stroke,fill}%
\end{pgfscope}%
\begin{pgfscope}%
\pgfpathrectangle{\pgfqpoint{0.786164in}{0.768110in}}{\pgfqpoint{8.851069in}{7.081890in}}%
\pgfusepath{clip}%
\pgfsetbuttcap%
\pgfsetroundjoin%
\definecolor{currentfill}{rgb}{0.221989,0.339161,0.548752}%
\pgfsetfillcolor{currentfill}%
\pgfsetfillopacity{0.700000}%
\pgfsetlinewidth{0.501875pt}%
\definecolor{currentstroke}{rgb}{1.000000,1.000000,1.000000}%
\pgfsetstrokecolor{currentstroke}%
\pgfsetstrokeopacity{0.700000}%
\pgfsetdash{}{0pt}%
\pgfpathmoveto{\pgfqpoint{5.028548in}{2.954451in}}%
\pgfpathcurveto{\pgfqpoint{5.041570in}{2.954451in}}{\pgfqpoint{5.054061in}{2.959625in}}{\pgfqpoint{5.063270in}{2.968834in}}%
\pgfpathcurveto{\pgfqpoint{5.072478in}{2.978042in}}{\pgfqpoint{5.077652in}{2.990533in}}{\pgfqpoint{5.077652in}{3.003556in}}%
\pgfpathcurveto{\pgfqpoint{5.077652in}{3.016578in}}{\pgfqpoint{5.072478in}{3.029070in}}{\pgfqpoint{5.063270in}{3.038278in}}%
\pgfpathcurveto{\pgfqpoint{5.054061in}{3.047486in}}{\pgfqpoint{5.041570in}{3.052660in}}{\pgfqpoint{5.028548in}{3.052660in}}%
\pgfpathcurveto{\pgfqpoint{5.015525in}{3.052660in}}{\pgfqpoint{5.003034in}{3.047486in}}{\pgfqpoint{4.993825in}{3.038278in}}%
\pgfpathcurveto{\pgfqpoint{4.984617in}{3.029070in}}{\pgfqpoint{4.979443in}{3.016578in}}{\pgfqpoint{4.979443in}{3.003556in}}%
\pgfpathcurveto{\pgfqpoint{4.979443in}{2.990533in}}{\pgfqpoint{4.984617in}{2.978042in}}{\pgfqpoint{4.993825in}{2.968834in}}%
\pgfpathcurveto{\pgfqpoint{5.003034in}{2.959625in}}{\pgfqpoint{5.015525in}{2.954451in}}{\pgfqpoint{5.028548in}{2.954451in}}%
\pgfpathlineto{\pgfqpoint{5.028548in}{2.954451in}}%
\pgfpathclose%
\pgfusepath{stroke,fill}%
\end{pgfscope}%
\begin{pgfscope}%
\pgfpathrectangle{\pgfqpoint{0.786164in}{0.768110in}}{\pgfqpoint{8.851069in}{7.081890in}}%
\pgfusepath{clip}%
\pgfsetbuttcap%
\pgfsetroundjoin%
\definecolor{currentfill}{rgb}{0.223925,0.334994,0.548053}%
\pgfsetfillcolor{currentfill}%
\pgfsetfillopacity{0.700000}%
\pgfsetlinewidth{0.501875pt}%
\definecolor{currentstroke}{rgb}{1.000000,1.000000,1.000000}%
\pgfsetstrokecolor{currentstroke}%
\pgfsetstrokeopacity{0.700000}%
\pgfsetdash{}{0pt}%
\pgfpathmoveto{\pgfqpoint{5.172870in}{3.125188in}}%
\pgfpathcurveto{\pgfqpoint{5.185893in}{3.125188in}}{\pgfqpoint{5.198384in}{3.130362in}}{\pgfqpoint{5.207593in}{3.139571in}}%
\pgfpathcurveto{\pgfqpoint{5.216801in}{3.148779in}}{\pgfqpoint{5.221975in}{3.161270in}}{\pgfqpoint{5.221975in}{3.174293in}}%
\pgfpathcurveto{\pgfqpoint{5.221975in}{3.187316in}}{\pgfqpoint{5.216801in}{3.199807in}}{\pgfqpoint{5.207593in}{3.209015in}}%
\pgfpathcurveto{\pgfqpoint{5.198384in}{3.218224in}}{\pgfqpoint{5.185893in}{3.223398in}}{\pgfqpoint{5.172870in}{3.223398in}}%
\pgfpathcurveto{\pgfqpoint{5.159848in}{3.223398in}}{\pgfqpoint{5.147357in}{3.218224in}}{\pgfqpoint{5.138148in}{3.209015in}}%
\pgfpathcurveto{\pgfqpoint{5.128940in}{3.199807in}}{\pgfqpoint{5.123766in}{3.187316in}}{\pgfqpoint{5.123766in}{3.174293in}}%
\pgfpathcurveto{\pgfqpoint{5.123766in}{3.161270in}}{\pgfqpoint{5.128940in}{3.148779in}}{\pgfqpoint{5.138148in}{3.139571in}}%
\pgfpathcurveto{\pgfqpoint{5.147357in}{3.130362in}}{\pgfqpoint{5.159848in}{3.125188in}}{\pgfqpoint{5.172870in}{3.125188in}}%
\pgfpathlineto{\pgfqpoint{5.172870in}{3.125188in}}%
\pgfpathclose%
\pgfusepath{stroke,fill}%
\end{pgfscope}%
\begin{pgfscope}%
\pgfpathrectangle{\pgfqpoint{0.786164in}{0.768110in}}{\pgfqpoint{8.851069in}{7.081890in}}%
\pgfusepath{clip}%
\pgfsetbuttcap%
\pgfsetroundjoin%
\definecolor{currentfill}{rgb}{0.223925,0.334994,0.548053}%
\pgfsetfillcolor{currentfill}%
\pgfsetfillopacity{0.700000}%
\pgfsetlinewidth{0.501875pt}%
\definecolor{currentstroke}{rgb}{1.000000,1.000000,1.000000}%
\pgfsetstrokecolor{currentstroke}%
\pgfsetstrokeopacity{0.700000}%
\pgfsetdash{}{0pt}%
\pgfpathmoveto{\pgfqpoint{5.164445in}{3.103846in}}%
\pgfpathcurveto{\pgfqpoint{5.177468in}{3.103846in}}{\pgfqpoint{5.189959in}{3.109020in}}{\pgfqpoint{5.199168in}{3.118229in}}%
\pgfpathcurveto{\pgfqpoint{5.208376in}{3.127437in}}{\pgfqpoint{5.213550in}{3.139928in}}{\pgfqpoint{5.213550in}{3.152951in}}%
\pgfpathcurveto{\pgfqpoint{5.213550in}{3.165974in}}{\pgfqpoint{5.208376in}{3.178465in}}{\pgfqpoint{5.199168in}{3.187673in}}%
\pgfpathcurveto{\pgfqpoint{5.189959in}{3.196882in}}{\pgfqpoint{5.177468in}{3.202056in}}{\pgfqpoint{5.164445in}{3.202056in}}%
\pgfpathcurveto{\pgfqpoint{5.151423in}{3.202056in}}{\pgfqpoint{5.138932in}{3.196882in}}{\pgfqpoint{5.129723in}{3.187673in}}%
\pgfpathcurveto{\pgfqpoint{5.120515in}{3.178465in}}{\pgfqpoint{5.115341in}{3.165974in}}{\pgfqpoint{5.115341in}{3.152951in}}%
\pgfpathcurveto{\pgfqpoint{5.115341in}{3.139928in}}{\pgfqpoint{5.120515in}{3.127437in}}{\pgfqpoint{5.129723in}{3.118229in}}%
\pgfpathcurveto{\pgfqpoint{5.138932in}{3.109020in}}{\pgfqpoint{5.151423in}{3.103846in}}{\pgfqpoint{5.164445in}{3.103846in}}%
\pgfpathlineto{\pgfqpoint{5.164445in}{3.103846in}}%
\pgfpathclose%
\pgfusepath{stroke,fill}%
\end{pgfscope}%
\begin{pgfscope}%
\pgfpathrectangle{\pgfqpoint{0.786164in}{0.768110in}}{\pgfqpoint{8.851069in}{7.081890in}}%
\pgfusepath{clip}%
\pgfsetbuttcap%
\pgfsetroundjoin%
\definecolor{currentfill}{rgb}{0.233603,0.313828,0.543914}%
\pgfsetfillcolor{currentfill}%
\pgfsetfillopacity{0.700000}%
\pgfsetlinewidth{0.501875pt}%
\definecolor{currentstroke}{rgb}{1.000000,1.000000,1.000000}%
\pgfsetstrokecolor{currentstroke}%
\pgfsetstrokeopacity{0.700000}%
\pgfsetdash{}{0pt}%
\pgfpathmoveto{\pgfqpoint{5.272504in}{2.975793in}}%
\pgfpathcurveto{\pgfqpoint{5.285527in}{2.975793in}}{\pgfqpoint{5.298018in}{2.980967in}}{\pgfqpoint{5.307227in}{2.990176in}}%
\pgfpathcurveto{\pgfqpoint{5.316435in}{2.999384in}}{\pgfqpoint{5.321609in}{3.011875in}}{\pgfqpoint{5.321609in}{3.024898in}}%
\pgfpathcurveto{\pgfqpoint{5.321609in}{3.037921in}}{\pgfqpoint{5.316435in}{3.050412in}}{\pgfqpoint{5.307227in}{3.059620in}}%
\pgfpathcurveto{\pgfqpoint{5.298018in}{3.068829in}}{\pgfqpoint{5.285527in}{3.074003in}}{\pgfqpoint{5.272504in}{3.074003in}}%
\pgfpathcurveto{\pgfqpoint{5.259482in}{3.074003in}}{\pgfqpoint{5.246991in}{3.068829in}}{\pgfqpoint{5.237782in}{3.059620in}}%
\pgfpathcurveto{\pgfqpoint{5.228574in}{3.050412in}}{\pgfqpoint{5.223400in}{3.037921in}}{\pgfqpoint{5.223400in}{3.024898in}}%
\pgfpathcurveto{\pgfqpoint{5.223400in}{3.011875in}}{\pgfqpoint{5.228574in}{2.999384in}}{\pgfqpoint{5.237782in}{2.990176in}}%
\pgfpathcurveto{\pgfqpoint{5.246991in}{2.980967in}}{\pgfqpoint{5.259482in}{2.975793in}}{\pgfqpoint{5.272504in}{2.975793in}}%
\pgfpathlineto{\pgfqpoint{5.272504in}{2.975793in}}%
\pgfpathclose%
\pgfusepath{stroke,fill}%
\end{pgfscope}%
\begin{pgfscope}%
\pgfpathrectangle{\pgfqpoint{0.786164in}{0.768110in}}{\pgfqpoint{8.851069in}{7.081890in}}%
\pgfusepath{clip}%
\pgfsetbuttcap%
\pgfsetroundjoin%
\definecolor{currentfill}{rgb}{0.229739,0.322361,0.545706}%
\pgfsetfillcolor{currentfill}%
\pgfsetfillopacity{0.700000}%
\pgfsetlinewidth{0.501875pt}%
\definecolor{currentstroke}{rgb}{1.000000,1.000000,1.000000}%
\pgfsetstrokecolor{currentstroke}%
\pgfsetstrokeopacity{0.700000}%
\pgfsetdash{}{0pt}%
\pgfpathmoveto{\pgfqpoint{5.266033in}{3.039820in}}%
\pgfpathcurveto{\pgfqpoint{5.279056in}{3.039820in}}{\pgfqpoint{5.291547in}{3.044994in}}{\pgfqpoint{5.300755in}{3.054202in}}%
\pgfpathcurveto{\pgfqpoint{5.309964in}{3.063411in}}{\pgfqpoint{5.315138in}{3.075902in}}{\pgfqpoint{5.315138in}{3.088924in}}%
\pgfpathcurveto{\pgfqpoint{5.315138in}{3.101947in}}{\pgfqpoint{5.309964in}{3.114438in}}{\pgfqpoint{5.300755in}{3.123647in}}%
\pgfpathcurveto{\pgfqpoint{5.291547in}{3.132855in}}{\pgfqpoint{5.279056in}{3.138029in}}{\pgfqpoint{5.266033in}{3.138029in}}%
\pgfpathcurveto{\pgfqpoint{5.253010in}{3.138029in}}{\pgfqpoint{5.240519in}{3.132855in}}{\pgfqpoint{5.231311in}{3.123647in}}%
\pgfpathcurveto{\pgfqpoint{5.222102in}{3.114438in}}{\pgfqpoint{5.216928in}{3.101947in}}{\pgfqpoint{5.216928in}{3.088924in}}%
\pgfpathcurveto{\pgfqpoint{5.216928in}{3.075902in}}{\pgfqpoint{5.222102in}{3.063411in}}{\pgfqpoint{5.231311in}{3.054202in}}%
\pgfpathcurveto{\pgfqpoint{5.240519in}{3.044994in}}{\pgfqpoint{5.253010in}{3.039820in}}{\pgfqpoint{5.266033in}{3.039820in}}%
\pgfpathlineto{\pgfqpoint{5.266033in}{3.039820in}}%
\pgfpathclose%
\pgfusepath{stroke,fill}%
\end{pgfscope}%
\begin{pgfscope}%
\pgfpathrectangle{\pgfqpoint{0.786164in}{0.768110in}}{\pgfqpoint{8.851069in}{7.081890in}}%
\pgfusepath{clip}%
\pgfsetbuttcap%
\pgfsetroundjoin%
\definecolor{currentfill}{rgb}{0.229739,0.322361,0.545706}%
\pgfsetfillcolor{currentfill}%
\pgfsetfillopacity{0.700000}%
\pgfsetlinewidth{0.501875pt}%
\definecolor{currentstroke}{rgb}{1.000000,1.000000,1.000000}%
\pgfsetstrokecolor{currentstroke}%
\pgfsetstrokeopacity{0.700000}%
\pgfsetdash{}{0pt}%
\pgfpathmoveto{\pgfqpoint{5.250526in}{3.018478in}}%
\pgfpathcurveto{\pgfqpoint{5.263549in}{3.018478in}}{\pgfqpoint{5.276040in}{3.023652in}}{\pgfqpoint{5.285249in}{3.032860in}}%
\pgfpathcurveto{\pgfqpoint{5.294457in}{3.042068in}}{\pgfqpoint{5.299631in}{3.054560in}}{\pgfqpoint{5.299631in}{3.067582in}}%
\pgfpathcurveto{\pgfqpoint{5.299631in}{3.080605in}}{\pgfqpoint{5.294457in}{3.093096in}}{\pgfqpoint{5.285249in}{3.102305in}}%
\pgfpathcurveto{\pgfqpoint{5.276040in}{3.111513in}}{\pgfqpoint{5.263549in}{3.116687in}}{\pgfqpoint{5.250526in}{3.116687in}}%
\pgfpathcurveto{\pgfqpoint{5.237504in}{3.116687in}}{\pgfqpoint{5.225013in}{3.111513in}}{\pgfqpoint{5.215804in}{3.102305in}}%
\pgfpathcurveto{\pgfqpoint{5.206596in}{3.093096in}}{\pgfqpoint{5.201422in}{3.080605in}}{\pgfqpoint{5.201422in}{3.067582in}}%
\pgfpathcurveto{\pgfqpoint{5.201422in}{3.054560in}}{\pgfqpoint{5.206596in}{3.042068in}}{\pgfqpoint{5.215804in}{3.032860in}}%
\pgfpathcurveto{\pgfqpoint{5.225013in}{3.023652in}}{\pgfqpoint{5.237504in}{3.018478in}}{\pgfqpoint{5.250526in}{3.018478in}}%
\pgfpathlineto{\pgfqpoint{5.250526in}{3.018478in}}%
\pgfpathclose%
\pgfusepath{stroke,fill}%
\end{pgfscope}%
\begin{pgfscope}%
\pgfpathrectangle{\pgfqpoint{0.786164in}{0.768110in}}{\pgfqpoint{8.851069in}{7.081890in}}%
\pgfusepath{clip}%
\pgfsetbuttcap%
\pgfsetroundjoin%
\definecolor{currentfill}{rgb}{0.225863,0.330805,0.547314}%
\pgfsetfillcolor{currentfill}%
\pgfsetfillopacity{0.700000}%
\pgfsetlinewidth{0.501875pt}%
\definecolor{currentstroke}{rgb}{1.000000,1.000000,1.000000}%
\pgfsetstrokecolor{currentstroke}%
\pgfsetstrokeopacity{0.700000}%
\pgfsetdash{}{0pt}%
\pgfpathmoveto{\pgfqpoint{5.221955in}{3.167873in}}%
\pgfpathcurveto{\pgfqpoint{5.234977in}{3.167873in}}{\pgfqpoint{5.247469in}{3.173047in}}{\pgfqpoint{5.256677in}{3.182255in}}%
\pgfpathcurveto{\pgfqpoint{5.265885in}{3.191464in}}{\pgfqpoint{5.271059in}{3.203955in}}{\pgfqpoint{5.271059in}{3.216977in}}%
\pgfpathcurveto{\pgfqpoint{5.271059in}{3.230000in}}{\pgfqpoint{5.265885in}{3.242491in}}{\pgfqpoint{5.256677in}{3.251700in}}%
\pgfpathcurveto{\pgfqpoint{5.247469in}{3.260908in}}{\pgfqpoint{5.234977in}{3.266082in}}{\pgfqpoint{5.221955in}{3.266082in}}%
\pgfpathcurveto{\pgfqpoint{5.208932in}{3.266082in}}{\pgfqpoint{5.196441in}{3.260908in}}{\pgfqpoint{5.187233in}{3.251700in}}%
\pgfpathcurveto{\pgfqpoint{5.178024in}{3.242491in}}{\pgfqpoint{5.172850in}{3.230000in}}{\pgfqpoint{5.172850in}{3.216977in}}%
\pgfpathcurveto{\pgfqpoint{5.172850in}{3.203955in}}{\pgfqpoint{5.178024in}{3.191464in}}{\pgfqpoint{5.187233in}{3.182255in}}%
\pgfpathcurveto{\pgfqpoint{5.196441in}{3.173047in}}{\pgfqpoint{5.208932in}{3.167873in}}{\pgfqpoint{5.221955in}{3.167873in}}%
\pgfpathlineto{\pgfqpoint{5.221955in}{3.167873in}}%
\pgfpathclose%
\pgfusepath{stroke,fill}%
\end{pgfscope}%
\begin{pgfscope}%
\pgfpathrectangle{\pgfqpoint{0.786164in}{0.768110in}}{\pgfqpoint{8.851069in}{7.081890in}}%
\pgfusepath{clip}%
\pgfsetbuttcap%
\pgfsetroundjoin%
\definecolor{currentfill}{rgb}{0.233603,0.313828,0.543914}%
\pgfsetfillcolor{currentfill}%
\pgfsetfillopacity{0.700000}%
\pgfsetlinewidth{0.501875pt}%
\definecolor{currentstroke}{rgb}{1.000000,1.000000,1.000000}%
\pgfsetstrokecolor{currentstroke}%
\pgfsetstrokeopacity{0.700000}%
\pgfsetdash{}{0pt}%
\pgfpathmoveto{\pgfqpoint{5.301076in}{3.274584in}}%
\pgfpathcurveto{\pgfqpoint{5.314099in}{3.274584in}}{\pgfqpoint{5.326590in}{3.279758in}}{\pgfqpoint{5.335798in}{3.288966in}}%
\pgfpathcurveto{\pgfqpoint{5.345007in}{3.298175in}}{\pgfqpoint{5.350181in}{3.310666in}}{\pgfqpoint{5.350181in}{3.323688in}}%
\pgfpathcurveto{\pgfqpoint{5.350181in}{3.336711in}}{\pgfqpoint{5.345007in}{3.349202in}}{\pgfqpoint{5.335798in}{3.358411in}}%
\pgfpathcurveto{\pgfqpoint{5.326590in}{3.367619in}}{\pgfqpoint{5.314099in}{3.372793in}}{\pgfqpoint{5.301076in}{3.372793in}}%
\pgfpathcurveto{\pgfqpoint{5.288053in}{3.372793in}}{\pgfqpoint{5.275562in}{3.367619in}}{\pgfqpoint{5.266354in}{3.358411in}}%
\pgfpathcurveto{\pgfqpoint{5.257145in}{3.349202in}}{\pgfqpoint{5.251971in}{3.336711in}}{\pgfqpoint{5.251971in}{3.323688in}}%
\pgfpathcurveto{\pgfqpoint{5.251971in}{3.310666in}}{\pgfqpoint{5.257145in}{3.298175in}}{\pgfqpoint{5.266354in}{3.288966in}}%
\pgfpathcurveto{\pgfqpoint{5.275562in}{3.279758in}}{\pgfqpoint{5.288053in}{3.274584in}}{\pgfqpoint{5.301076in}{3.274584in}}%
\pgfpathlineto{\pgfqpoint{5.301076in}{3.274584in}}%
\pgfpathclose%
\pgfusepath{stroke,fill}%
\end{pgfscope}%
\begin{pgfscope}%
\pgfpathrectangle{\pgfqpoint{0.786164in}{0.768110in}}{\pgfqpoint{8.851069in}{7.081890in}}%
\pgfusepath{clip}%
\pgfsetbuttcap%
\pgfsetroundjoin%
\definecolor{currentfill}{rgb}{0.231674,0.318106,0.544834}%
\pgfsetfillcolor{currentfill}%
\pgfsetfillopacity{0.700000}%
\pgfsetlinewidth{0.501875pt}%
\definecolor{currentstroke}{rgb}{1.000000,1.000000,1.000000}%
\pgfsetstrokecolor{currentstroke}%
\pgfsetstrokeopacity{0.700000}%
\pgfsetdash{}{0pt}%
\pgfpathmoveto{\pgfqpoint{5.297535in}{3.231899in}}%
\pgfpathcurveto{\pgfqpoint{5.310558in}{3.231899in}}{\pgfqpoint{5.323049in}{3.237073in}}{\pgfqpoint{5.332257in}{3.246282in}}%
\pgfpathcurveto{\pgfqpoint{5.341466in}{3.255490in}}{\pgfqpoint{5.346640in}{3.267981in}}{\pgfqpoint{5.346640in}{3.281004in}}%
\pgfpathcurveto{\pgfqpoint{5.346640in}{3.294027in}}{\pgfqpoint{5.341466in}{3.306518in}}{\pgfqpoint{5.332257in}{3.315726in}}%
\pgfpathcurveto{\pgfqpoint{5.323049in}{3.324935in}}{\pgfqpoint{5.310558in}{3.330109in}}{\pgfqpoint{5.297535in}{3.330109in}}%
\pgfpathcurveto{\pgfqpoint{5.284512in}{3.330109in}}{\pgfqpoint{5.272021in}{3.324935in}}{\pgfqpoint{5.262813in}{3.315726in}}%
\pgfpathcurveto{\pgfqpoint{5.253604in}{3.306518in}}{\pgfqpoint{5.248430in}{3.294027in}}{\pgfqpoint{5.248430in}{3.281004in}}%
\pgfpathcurveto{\pgfqpoint{5.248430in}{3.267981in}}{\pgfqpoint{5.253604in}{3.255490in}}{\pgfqpoint{5.262813in}{3.246282in}}%
\pgfpathcurveto{\pgfqpoint{5.272021in}{3.237073in}}{\pgfqpoint{5.284512in}{3.231899in}}{\pgfqpoint{5.297535in}{3.231899in}}%
\pgfpathlineto{\pgfqpoint{5.297535in}{3.231899in}}%
\pgfpathclose%
\pgfusepath{stroke,fill}%
\end{pgfscope}%
\begin{pgfscope}%
\pgfpathrectangle{\pgfqpoint{0.786164in}{0.768110in}}{\pgfqpoint{8.851069in}{7.081890in}}%
\pgfusepath{clip}%
\pgfsetbuttcap%
\pgfsetroundjoin%
\definecolor{currentfill}{rgb}{0.246811,0.283237,0.535941}%
\pgfsetfillcolor{currentfill}%
\pgfsetfillopacity{0.700000}%
\pgfsetlinewidth{0.501875pt}%
\definecolor{currentstroke}{rgb}{1.000000,1.000000,1.000000}%
\pgfsetstrokecolor{currentstroke}%
\pgfsetstrokeopacity{0.700000}%
\pgfsetdash{}{0pt}%
\pgfpathmoveto{\pgfqpoint{5.638196in}{2.847740in}}%
\pgfpathcurveto{\pgfqpoint{5.651218in}{2.847740in}}{\pgfqpoint{5.663709in}{2.852914in}}{\pgfqpoint{5.672918in}{2.862123in}}%
\pgfpathcurveto{\pgfqpoint{5.682126in}{2.871331in}}{\pgfqpoint{5.687300in}{2.883822in}}{\pgfqpoint{5.687300in}{2.896845in}}%
\pgfpathcurveto{\pgfqpoint{5.687300in}{2.909868in}}{\pgfqpoint{5.682126in}{2.922359in}}{\pgfqpoint{5.672918in}{2.931567in}}%
\pgfpathcurveto{\pgfqpoint{5.663709in}{2.940776in}}{\pgfqpoint{5.651218in}{2.945950in}}{\pgfqpoint{5.638196in}{2.945950in}}%
\pgfpathcurveto{\pgfqpoint{5.625173in}{2.945950in}}{\pgfqpoint{5.612682in}{2.940776in}}{\pgfqpoint{5.603473in}{2.931567in}}%
\pgfpathcurveto{\pgfqpoint{5.594265in}{2.922359in}}{\pgfqpoint{5.589091in}{2.909868in}}{\pgfqpoint{5.589091in}{2.896845in}}%
\pgfpathcurveto{\pgfqpoint{5.589091in}{2.883822in}}{\pgfqpoint{5.594265in}{2.871331in}}{\pgfqpoint{5.603473in}{2.862123in}}%
\pgfpathcurveto{\pgfqpoint{5.612682in}{2.852914in}}{\pgfqpoint{5.625173in}{2.847740in}}{\pgfqpoint{5.638196in}{2.847740in}}%
\pgfpathlineto{\pgfqpoint{5.638196in}{2.847740in}}%
\pgfpathclose%
\pgfusepath{stroke,fill}%
\end{pgfscope}%
\begin{pgfscope}%
\pgfpathrectangle{\pgfqpoint{0.786164in}{0.768110in}}{\pgfqpoint{8.851069in}{7.081890in}}%
\pgfusepath{clip}%
\pgfsetbuttcap%
\pgfsetroundjoin%
\definecolor{currentfill}{rgb}{0.237441,0.305202,0.541921}%
\pgfsetfillcolor{currentfill}%
\pgfsetfillopacity{0.700000}%
\pgfsetlinewidth{0.501875pt}%
\definecolor{currentstroke}{rgb}{1.000000,1.000000,1.000000}%
\pgfsetstrokecolor{currentstroke}%
\pgfsetstrokeopacity{0.700000}%
\pgfsetdash{}{0pt}%
\pgfpathmoveto{\pgfqpoint{5.397413in}{2.783714in}}%
\pgfpathcurveto{\pgfqpoint{5.410436in}{2.783714in}}{\pgfqpoint{5.422927in}{2.788888in}}{\pgfqpoint{5.432135in}{2.798096in}}%
\pgfpathcurveto{\pgfqpoint{5.441344in}{2.807305in}}{\pgfqpoint{5.446518in}{2.819796in}}{\pgfqpoint{5.446518in}{2.832818in}}%
\pgfpathcurveto{\pgfqpoint{5.446518in}{2.845841in}}{\pgfqpoint{5.441344in}{2.858332in}}{\pgfqpoint{5.432135in}{2.867541in}}%
\pgfpathcurveto{\pgfqpoint{5.422927in}{2.876749in}}{\pgfqpoint{5.410436in}{2.881923in}}{\pgfqpoint{5.397413in}{2.881923in}}%
\pgfpathcurveto{\pgfqpoint{5.384391in}{2.881923in}}{\pgfqpoint{5.371899in}{2.876749in}}{\pgfqpoint{5.362691in}{2.867541in}}%
\pgfpathcurveto{\pgfqpoint{5.353483in}{2.858332in}}{\pgfqpoint{5.348309in}{2.845841in}}{\pgfqpoint{5.348309in}{2.832818in}}%
\pgfpathcurveto{\pgfqpoint{5.348309in}{2.819796in}}{\pgfqpoint{5.353483in}{2.807305in}}{\pgfqpoint{5.362691in}{2.798096in}}%
\pgfpathcurveto{\pgfqpoint{5.371899in}{2.788888in}}{\pgfqpoint{5.384391in}{2.783714in}}{\pgfqpoint{5.397413in}{2.783714in}}%
\pgfpathlineto{\pgfqpoint{5.397413in}{2.783714in}}%
\pgfpathclose%
\pgfusepath{stroke,fill}%
\end{pgfscope}%
\begin{pgfscope}%
\pgfpathrectangle{\pgfqpoint{0.786164in}{0.768110in}}{\pgfqpoint{8.851069in}{7.081890in}}%
\pgfusepath{clip}%
\pgfsetbuttcap%
\pgfsetroundjoin%
\definecolor{currentfill}{rgb}{0.246811,0.283237,0.535941}%
\pgfsetfillcolor{currentfill}%
\pgfsetfillopacity{0.700000}%
\pgfsetlinewidth{0.501875pt}%
\definecolor{currentstroke}{rgb}{1.000000,1.000000,1.000000}%
\pgfsetstrokecolor{currentstroke}%
\pgfsetstrokeopacity{0.700000}%
\pgfsetdash{}{0pt}%
\pgfpathmoveto{\pgfqpoint{5.297901in}{2.847740in}}%
\pgfpathcurveto{\pgfqpoint{5.310924in}{2.847740in}}{\pgfqpoint{5.323415in}{2.852914in}}{\pgfqpoint{5.332624in}{2.862123in}}%
\pgfpathcurveto{\pgfqpoint{5.341832in}{2.871331in}}{\pgfqpoint{5.347006in}{2.883822in}}{\pgfqpoint{5.347006in}{2.896845in}}%
\pgfpathcurveto{\pgfqpoint{5.347006in}{2.909868in}}{\pgfqpoint{5.341832in}{2.922359in}}{\pgfqpoint{5.332624in}{2.931567in}}%
\pgfpathcurveto{\pgfqpoint{5.323415in}{2.940776in}}{\pgfqpoint{5.310924in}{2.945950in}}{\pgfqpoint{5.297901in}{2.945950in}}%
\pgfpathcurveto{\pgfqpoint{5.284879in}{2.945950in}}{\pgfqpoint{5.272388in}{2.940776in}}{\pgfqpoint{5.263179in}{2.931567in}}%
\pgfpathcurveto{\pgfqpoint{5.253971in}{2.922359in}}{\pgfqpoint{5.248797in}{2.909868in}}{\pgfqpoint{5.248797in}{2.896845in}}%
\pgfpathcurveto{\pgfqpoint{5.248797in}{2.883822in}}{\pgfqpoint{5.253971in}{2.871331in}}{\pgfqpoint{5.263179in}{2.862123in}}%
\pgfpathcurveto{\pgfqpoint{5.272388in}{2.852914in}}{\pgfqpoint{5.284879in}{2.847740in}}{\pgfqpoint{5.297901in}{2.847740in}}%
\pgfpathlineto{\pgfqpoint{5.297901in}{2.847740in}}%
\pgfpathclose%
\pgfusepath{stroke,fill}%
\end{pgfscope}%
\begin{pgfscope}%
\pgfpathrectangle{\pgfqpoint{0.786164in}{0.768110in}}{\pgfqpoint{8.851069in}{7.081890in}}%
\pgfusepath{clip}%
\pgfsetbuttcap%
\pgfsetroundjoin%
\definecolor{currentfill}{rgb}{0.160665,0.478540,0.558115}%
\pgfsetfillcolor{currentfill}%
\pgfsetfillopacity{0.700000}%
\pgfsetlinewidth{0.501875pt}%
\definecolor{currentstroke}{rgb}{1.000000,1.000000,1.000000}%
\pgfsetstrokecolor{currentstroke}%
\pgfsetstrokeopacity{0.700000}%
\pgfsetdash{}{0pt}%
\pgfpathmoveto{\pgfqpoint{1.409976in}{4.277666in}}%
\pgfpathcurveto{\pgfqpoint{1.422998in}{4.277666in}}{\pgfqpoint{1.435489in}{4.282840in}}{\pgfqpoint{1.444698in}{4.292048in}}%
\pgfpathcurveto{\pgfqpoint{1.453906in}{4.301257in}}{\pgfqpoint{1.459080in}{4.313748in}}{\pgfqpoint{1.459080in}{4.326770in}}%
\pgfpathcurveto{\pgfqpoint{1.459080in}{4.339793in}}{\pgfqpoint{1.453906in}{4.352284in}}{\pgfqpoint{1.444698in}{4.361493in}}%
\pgfpathcurveto{\pgfqpoint{1.435489in}{4.370701in}}{\pgfqpoint{1.422998in}{4.375875in}}{\pgfqpoint{1.409976in}{4.375875in}}%
\pgfpathcurveto{\pgfqpoint{1.396953in}{4.375875in}}{\pgfqpoint{1.384462in}{4.370701in}}{\pgfqpoint{1.375253in}{4.361493in}}%
\pgfpathcurveto{\pgfqpoint{1.366045in}{4.352284in}}{\pgfqpoint{1.360871in}{4.339793in}}{\pgfqpoint{1.360871in}{4.326770in}}%
\pgfpathcurveto{\pgfqpoint{1.360871in}{4.313748in}}{\pgfqpoint{1.366045in}{4.301257in}}{\pgfqpoint{1.375253in}{4.292048in}}%
\pgfpathcurveto{\pgfqpoint{1.384462in}{4.282840in}}{\pgfqpoint{1.396953in}{4.277666in}}{\pgfqpoint{1.409976in}{4.277666in}}%
\pgfpathlineto{\pgfqpoint{1.409976in}{4.277666in}}%
\pgfpathclose%
\pgfusepath{stroke,fill}%
\end{pgfscope}%
\begin{pgfscope}%
\pgfpathrectangle{\pgfqpoint{0.786164in}{0.768110in}}{\pgfqpoint{8.851069in}{7.081890in}}%
\pgfusepath{clip}%
\pgfsetbuttcap%
\pgfsetroundjoin%
\definecolor{currentfill}{rgb}{0.165117,0.467423,0.558141}%
\pgfsetfillcolor{currentfill}%
\pgfsetfillopacity{0.700000}%
\pgfsetlinewidth{0.501875pt}%
\definecolor{currentstroke}{rgb}{1.000000,1.000000,1.000000}%
\pgfsetstrokecolor{currentstroke}%
\pgfsetstrokeopacity{0.700000}%
\pgfsetdash{}{0pt}%
\pgfpathmoveto{\pgfqpoint{1.459915in}{4.192297in}}%
\pgfpathcurveto{\pgfqpoint{1.472937in}{4.192297in}}{\pgfqpoint{1.485429in}{4.197471in}}{\pgfqpoint{1.494637in}{4.206679in}}%
\pgfpathcurveto{\pgfqpoint{1.503845in}{4.215888in}}{\pgfqpoint{1.509019in}{4.228379in}}{\pgfqpoint{1.509019in}{4.241402in}}%
\pgfpathcurveto{\pgfqpoint{1.509019in}{4.254424in}}{\pgfqpoint{1.503845in}{4.266915in}}{\pgfqpoint{1.494637in}{4.276124in}}%
\pgfpathcurveto{\pgfqpoint{1.485429in}{4.285332in}}{\pgfqpoint{1.472937in}{4.290506in}}{\pgfqpoint{1.459915in}{4.290506in}}%
\pgfpathcurveto{\pgfqpoint{1.446892in}{4.290506in}}{\pgfqpoint{1.434401in}{4.285332in}}{\pgfqpoint{1.425193in}{4.276124in}}%
\pgfpathcurveto{\pgfqpoint{1.415984in}{4.266915in}}{\pgfqpoint{1.410810in}{4.254424in}}{\pgfqpoint{1.410810in}{4.241402in}}%
\pgfpathcurveto{\pgfqpoint{1.410810in}{4.228379in}}{\pgfqpoint{1.415984in}{4.215888in}}{\pgfqpoint{1.425193in}{4.206679in}}%
\pgfpathcurveto{\pgfqpoint{1.434401in}{4.197471in}}{\pgfqpoint{1.446892in}{4.192297in}}{\pgfqpoint{1.459915in}{4.192297in}}%
\pgfpathlineto{\pgfqpoint{1.459915in}{4.192297in}}%
\pgfpathclose%
\pgfusepath{stroke,fill}%
\end{pgfscope}%
\begin{pgfscope}%
\pgfpathrectangle{\pgfqpoint{0.786164in}{0.768110in}}{\pgfqpoint{8.851069in}{7.081890in}}%
\pgfusepath{clip}%
\pgfsetbuttcap%
\pgfsetroundjoin%
\definecolor{currentfill}{rgb}{0.168126,0.459988,0.558082}%
\pgfsetfillcolor{currentfill}%
\pgfsetfillopacity{0.700000}%
\pgfsetlinewidth{0.501875pt}%
\definecolor{currentstroke}{rgb}{1.000000,1.000000,1.000000}%
\pgfsetstrokecolor{currentstroke}%
\pgfsetstrokeopacity{0.700000}%
\pgfsetdash{}{0pt}%
\pgfpathmoveto{\pgfqpoint{1.500574in}{4.106928in}}%
\pgfpathcurveto{\pgfqpoint{1.513597in}{4.106928in}}{\pgfqpoint{1.526088in}{4.112102in}}{\pgfqpoint{1.535296in}{4.121311in}}%
\pgfpathcurveto{\pgfqpoint{1.544505in}{4.130519in}}{\pgfqpoint{1.549679in}{4.143010in}}{\pgfqpoint{1.549679in}{4.156033in}}%
\pgfpathcurveto{\pgfqpoint{1.549679in}{4.169056in}}{\pgfqpoint{1.544505in}{4.181547in}}{\pgfqpoint{1.535296in}{4.190755in}}%
\pgfpathcurveto{\pgfqpoint{1.526088in}{4.199964in}}{\pgfqpoint{1.513597in}{4.205138in}}{\pgfqpoint{1.500574in}{4.205138in}}%
\pgfpathcurveto{\pgfqpoint{1.487552in}{4.205138in}}{\pgfqpoint{1.475060in}{4.199964in}}{\pgfqpoint{1.465852in}{4.190755in}}%
\pgfpathcurveto{\pgfqpoint{1.456644in}{4.181547in}}{\pgfqpoint{1.451470in}{4.169056in}}{\pgfqpoint{1.451470in}{4.156033in}}%
\pgfpathcurveto{\pgfqpoint{1.451470in}{4.143010in}}{\pgfqpoint{1.456644in}{4.130519in}}{\pgfqpoint{1.465852in}{4.121311in}}%
\pgfpathcurveto{\pgfqpoint{1.475060in}{4.112102in}}{\pgfqpoint{1.487552in}{4.106928in}}{\pgfqpoint{1.500574in}{4.106928in}}%
\pgfpathlineto{\pgfqpoint{1.500574in}{4.106928in}}%
\pgfpathclose%
\pgfusepath{stroke,fill}%
\end{pgfscope}%
\begin{pgfscope}%
\pgfpathrectangle{\pgfqpoint{0.786164in}{0.768110in}}{\pgfqpoint{8.851069in}{7.081890in}}%
\pgfusepath{clip}%
\pgfsetbuttcap%
\pgfsetroundjoin%
\definecolor{currentfill}{rgb}{0.174274,0.445044,0.557792}%
\pgfsetfillcolor{currentfill}%
\pgfsetfillopacity{0.700000}%
\pgfsetlinewidth{0.501875pt}%
\definecolor{currentstroke}{rgb}{1.000000,1.000000,1.000000}%
\pgfsetstrokecolor{currentstroke}%
\pgfsetstrokeopacity{0.700000}%
\pgfsetdash{}{0pt}%
\pgfpathmoveto{\pgfqpoint{1.559427in}{4.021560in}}%
\pgfpathcurveto{\pgfqpoint{1.572449in}{4.021560in}}{\pgfqpoint{1.584940in}{4.026734in}}{\pgfqpoint{1.594149in}{4.035942in}}%
\pgfpathcurveto{\pgfqpoint{1.603357in}{4.045151in}}{\pgfqpoint{1.608531in}{4.057642in}}{\pgfqpoint{1.608531in}{4.070664in}}%
\pgfpathcurveto{\pgfqpoint{1.608531in}{4.083687in}}{\pgfqpoint{1.603357in}{4.096178in}}{\pgfqpoint{1.594149in}{4.105387in}}%
\pgfpathcurveto{\pgfqpoint{1.584940in}{4.114595in}}{\pgfqpoint{1.572449in}{4.119769in}}{\pgfqpoint{1.559427in}{4.119769in}}%
\pgfpathcurveto{\pgfqpoint{1.546404in}{4.119769in}}{\pgfqpoint{1.533913in}{4.114595in}}{\pgfqpoint{1.524704in}{4.105387in}}%
\pgfpathcurveto{\pgfqpoint{1.515496in}{4.096178in}}{\pgfqpoint{1.510322in}{4.083687in}}{\pgfqpoint{1.510322in}{4.070664in}}%
\pgfpathcurveto{\pgfqpoint{1.510322in}{4.057642in}}{\pgfqpoint{1.515496in}{4.045151in}}{\pgfqpoint{1.524704in}{4.035942in}}%
\pgfpathcurveto{\pgfqpoint{1.533913in}{4.026734in}}{\pgfqpoint{1.546404in}{4.021560in}}{\pgfqpoint{1.559427in}{4.021560in}}%
\pgfpathlineto{\pgfqpoint{1.559427in}{4.021560in}}%
\pgfpathclose%
\pgfusepath{stroke,fill}%
\end{pgfscope}%
\begin{pgfscope}%
\pgfpathrectangle{\pgfqpoint{0.786164in}{0.768110in}}{\pgfqpoint{8.851069in}{7.081890in}}%
\pgfusepath{clip}%
\pgfsetbuttcap%
\pgfsetroundjoin%
\definecolor{currentfill}{rgb}{0.171176,0.452530,0.557965}%
\pgfsetfillcolor{currentfill}%
\pgfsetfillopacity{0.700000}%
\pgfsetlinewidth{0.501875pt}%
\definecolor{currentstroke}{rgb}{1.000000,1.000000,1.000000}%
\pgfsetstrokecolor{currentstroke}%
\pgfsetstrokeopacity{0.700000}%
\pgfsetdash{}{0pt}%
\pgfpathmoveto{\pgfqpoint{1.616936in}{4.021560in}}%
\pgfpathcurveto{\pgfqpoint{1.629959in}{4.021560in}}{\pgfqpoint{1.642450in}{4.026734in}}{\pgfqpoint{1.651658in}{4.035942in}}%
\pgfpathcurveto{\pgfqpoint{1.660867in}{4.045151in}}{\pgfqpoint{1.666041in}{4.057642in}}{\pgfqpoint{1.666041in}{4.070664in}}%
\pgfpathcurveto{\pgfqpoint{1.666041in}{4.083687in}}{\pgfqpoint{1.660867in}{4.096178in}}{\pgfqpoint{1.651658in}{4.105387in}}%
\pgfpathcurveto{\pgfqpoint{1.642450in}{4.114595in}}{\pgfqpoint{1.629959in}{4.119769in}}{\pgfqpoint{1.616936in}{4.119769in}}%
\pgfpathcurveto{\pgfqpoint{1.603913in}{4.119769in}}{\pgfqpoint{1.591422in}{4.114595in}}{\pgfqpoint{1.582214in}{4.105387in}}%
\pgfpathcurveto{\pgfqpoint{1.573005in}{4.096178in}}{\pgfqpoint{1.567831in}{4.083687in}}{\pgfqpoint{1.567831in}{4.070664in}}%
\pgfpathcurveto{\pgfqpoint{1.567831in}{4.057642in}}{\pgfqpoint{1.573005in}{4.045151in}}{\pgfqpoint{1.582214in}{4.035942in}}%
\pgfpathcurveto{\pgfqpoint{1.591422in}{4.026734in}}{\pgfqpoint{1.603913in}{4.021560in}}{\pgfqpoint{1.616936in}{4.021560in}}%
\pgfpathlineto{\pgfqpoint{1.616936in}{4.021560in}}%
\pgfpathclose%
\pgfusepath{stroke,fill}%
\end{pgfscope}%
\begin{pgfscope}%
\pgfpathrectangle{\pgfqpoint{0.786164in}{0.768110in}}{\pgfqpoint{8.851069in}{7.081890in}}%
\pgfusepath{clip}%
\pgfsetbuttcap%
\pgfsetroundjoin%
\definecolor{currentfill}{rgb}{0.179019,0.433756,0.557430}%
\pgfsetfillcolor{currentfill}%
\pgfsetfillopacity{0.700000}%
\pgfsetlinewidth{0.501875pt}%
\definecolor{currentstroke}{rgb}{1.000000,1.000000,1.000000}%
\pgfsetstrokecolor{currentstroke}%
\pgfsetstrokeopacity{0.700000}%
\pgfsetdash{}{0pt}%
\pgfpathmoveto{\pgfqpoint{1.755520in}{3.744111in}}%
\pgfpathcurveto{\pgfqpoint{1.768543in}{3.744111in}}{\pgfqpoint{1.781034in}{3.749285in}}{\pgfqpoint{1.790242in}{3.758494in}}%
\pgfpathcurveto{\pgfqpoint{1.799451in}{3.767702in}}{\pgfqpoint{1.804625in}{3.780193in}}{\pgfqpoint{1.804625in}{3.793216in}}%
\pgfpathcurveto{\pgfqpoint{1.804625in}{3.806239in}}{\pgfqpoint{1.799451in}{3.818730in}}{\pgfqpoint{1.790242in}{3.827938in}}%
\pgfpathcurveto{\pgfqpoint{1.781034in}{3.837147in}}{\pgfqpoint{1.768543in}{3.842321in}}{\pgfqpoint{1.755520in}{3.842321in}}%
\pgfpathcurveto{\pgfqpoint{1.742497in}{3.842321in}}{\pgfqpoint{1.730006in}{3.837147in}}{\pgfqpoint{1.720798in}{3.827938in}}%
\pgfpathcurveto{\pgfqpoint{1.711589in}{3.818730in}}{\pgfqpoint{1.706416in}{3.806239in}}{\pgfqpoint{1.706416in}{3.793216in}}%
\pgfpathcurveto{\pgfqpoint{1.706416in}{3.780193in}}{\pgfqpoint{1.711589in}{3.767702in}}{\pgfqpoint{1.720798in}{3.758494in}}%
\pgfpathcurveto{\pgfqpoint{1.730006in}{3.749285in}}{\pgfqpoint{1.742497in}{3.744111in}}{\pgfqpoint{1.755520in}{3.744111in}}%
\pgfpathlineto{\pgfqpoint{1.755520in}{3.744111in}}%
\pgfpathclose%
\pgfusepath{stroke,fill}%
\end{pgfscope}%
\begin{pgfscope}%
\pgfpathrectangle{\pgfqpoint{0.786164in}{0.768110in}}{\pgfqpoint{8.851069in}{7.081890in}}%
\pgfusepath{clip}%
\pgfsetbuttcap%
\pgfsetroundjoin%
\definecolor{currentfill}{rgb}{0.166617,0.463708,0.558119}%
\pgfsetfillcolor{currentfill}%
\pgfsetfillopacity{0.700000}%
\pgfsetlinewidth{0.501875pt}%
\definecolor{currentstroke}{rgb}{1.000000,1.000000,1.000000}%
\pgfsetstrokecolor{currentstroke}%
\pgfsetstrokeopacity{0.700000}%
\pgfsetdash{}{0pt}%
\pgfpathmoveto{\pgfqpoint{1.909123in}{3.872164in}}%
\pgfpathcurveto{\pgfqpoint{1.922145in}{3.872164in}}{\pgfqpoint{1.934636in}{3.877338in}}{\pgfqpoint{1.943845in}{3.886547in}}%
\pgfpathcurveto{\pgfqpoint{1.953053in}{3.895755in}}{\pgfqpoint{1.958227in}{3.908246in}}{\pgfqpoint{1.958227in}{3.921269in}}%
\pgfpathcurveto{\pgfqpoint{1.958227in}{3.934292in}}{\pgfqpoint{1.953053in}{3.946783in}}{\pgfqpoint{1.943845in}{3.955991in}}%
\pgfpathcurveto{\pgfqpoint{1.934636in}{3.965200in}}{\pgfqpoint{1.922145in}{3.970374in}}{\pgfqpoint{1.909123in}{3.970374in}}%
\pgfpathcurveto{\pgfqpoint{1.896100in}{3.970374in}}{\pgfqpoint{1.883609in}{3.965200in}}{\pgfqpoint{1.874400in}{3.955991in}}%
\pgfpathcurveto{\pgfqpoint{1.865192in}{3.946783in}}{\pgfqpoint{1.860018in}{3.934292in}}{\pgfqpoint{1.860018in}{3.921269in}}%
\pgfpathcurveto{\pgfqpoint{1.860018in}{3.908246in}}{\pgfqpoint{1.865192in}{3.895755in}}{\pgfqpoint{1.874400in}{3.886547in}}%
\pgfpathcurveto{\pgfqpoint{1.883609in}{3.877338in}}{\pgfqpoint{1.896100in}{3.872164in}}{\pgfqpoint{1.909123in}{3.872164in}}%
\pgfpathlineto{\pgfqpoint{1.909123in}{3.872164in}}%
\pgfpathclose%
\pgfusepath{stroke,fill}%
\end{pgfscope}%
\begin{pgfscope}%
\pgfpathrectangle{\pgfqpoint{0.786164in}{0.768110in}}{\pgfqpoint{8.851069in}{7.081890in}}%
\pgfusepath{clip}%
\pgfsetbuttcap%
\pgfsetroundjoin%
\definecolor{currentfill}{rgb}{0.185556,0.418570,0.556753}%
\pgfsetfillcolor{currentfill}%
\pgfsetfillopacity{0.700000}%
\pgfsetlinewidth{0.501875pt}%
\definecolor{currentstroke}{rgb}{1.000000,1.000000,1.000000}%
\pgfsetstrokecolor{currentstroke}%
\pgfsetstrokeopacity{0.700000}%
\pgfsetdash{}{0pt}%
\pgfpathmoveto{\pgfqpoint{1.945509in}{3.637401in}}%
\pgfpathcurveto{\pgfqpoint{1.958531in}{3.637401in}}{\pgfqpoint{1.971022in}{3.642575in}}{\pgfqpoint{1.980231in}{3.651783in}}%
\pgfpathcurveto{\pgfqpoint{1.989439in}{3.660991in}}{\pgfqpoint{1.994613in}{3.673483in}}{\pgfqpoint{1.994613in}{3.686505in}}%
\pgfpathcurveto{\pgfqpoint{1.994613in}{3.699528in}}{\pgfqpoint{1.989439in}{3.712019in}}{\pgfqpoint{1.980231in}{3.721227in}}%
\pgfpathcurveto{\pgfqpoint{1.971022in}{3.730436in}}{\pgfqpoint{1.958531in}{3.735610in}}{\pgfqpoint{1.945509in}{3.735610in}}%
\pgfpathcurveto{\pgfqpoint{1.932486in}{3.735610in}}{\pgfqpoint{1.919995in}{3.730436in}}{\pgfqpoint{1.910786in}{3.721227in}}%
\pgfpathcurveto{\pgfqpoint{1.901578in}{3.712019in}}{\pgfqpoint{1.896404in}{3.699528in}}{\pgfqpoint{1.896404in}{3.686505in}}%
\pgfpathcurveto{\pgfqpoint{1.896404in}{3.673483in}}{\pgfqpoint{1.901578in}{3.660991in}}{\pgfqpoint{1.910786in}{3.651783in}}%
\pgfpathcurveto{\pgfqpoint{1.919995in}{3.642575in}}{\pgfqpoint{1.932486in}{3.637401in}}{\pgfqpoint{1.945509in}{3.637401in}}%
\pgfpathlineto{\pgfqpoint{1.945509in}{3.637401in}}%
\pgfpathclose%
\pgfusepath{stroke,fill}%
\end{pgfscope}%
\begin{pgfscope}%
\pgfpathrectangle{\pgfqpoint{0.786164in}{0.768110in}}{\pgfqpoint{8.851069in}{7.081890in}}%
\pgfusepath{clip}%
\pgfsetbuttcap%
\pgfsetroundjoin%
\definecolor{currentfill}{rgb}{0.197636,0.391528,0.554969}%
\pgfsetfillcolor{currentfill}%
\pgfsetfillopacity{0.700000}%
\pgfsetlinewidth{0.501875pt}%
\definecolor{currentstroke}{rgb}{1.000000,1.000000,1.000000}%
\pgfsetstrokecolor{currentstroke}%
\pgfsetstrokeopacity{0.700000}%
\pgfsetdash{}{0pt}%
\pgfpathmoveto{\pgfqpoint{2.041235in}{3.573374in}}%
\pgfpathcurveto{\pgfqpoint{2.054258in}{3.573374in}}{\pgfqpoint{2.066749in}{3.578548in}}{\pgfqpoint{2.075958in}{3.587756in}}%
\pgfpathcurveto{\pgfqpoint{2.085166in}{3.596965in}}{\pgfqpoint{2.090340in}{3.609456in}}{\pgfqpoint{2.090340in}{3.622479in}}%
\pgfpathcurveto{\pgfqpoint{2.090340in}{3.635501in}}{\pgfqpoint{2.085166in}{3.647992in}}{\pgfqpoint{2.075958in}{3.657201in}}%
\pgfpathcurveto{\pgfqpoint{2.066749in}{3.666409in}}{\pgfqpoint{2.054258in}{3.671583in}}{\pgfqpoint{2.041235in}{3.671583in}}%
\pgfpathcurveto{\pgfqpoint{2.028213in}{3.671583in}}{\pgfqpoint{2.015722in}{3.666409in}}{\pgfqpoint{2.006513in}{3.657201in}}%
\pgfpathcurveto{\pgfqpoint{1.997305in}{3.647992in}}{\pgfqpoint{1.992131in}{3.635501in}}{\pgfqpoint{1.992131in}{3.622479in}}%
\pgfpathcurveto{\pgfqpoint{1.992131in}{3.609456in}}{\pgfqpoint{1.997305in}{3.596965in}}{\pgfqpoint{2.006513in}{3.587756in}}%
\pgfpathcurveto{\pgfqpoint{2.015722in}{3.578548in}}{\pgfqpoint{2.028213in}{3.573374in}}{\pgfqpoint{2.041235in}{3.573374in}}%
\pgfpathlineto{\pgfqpoint{2.041235in}{3.573374in}}%
\pgfpathclose%
\pgfusepath{stroke,fill}%
\end{pgfscope}%
\begin{pgfscope}%
\pgfpathrectangle{\pgfqpoint{0.786164in}{0.768110in}}{\pgfqpoint{8.851069in}{7.081890in}}%
\pgfusepath{clip}%
\pgfsetbuttcap%
\pgfsetroundjoin%
\definecolor{currentfill}{rgb}{0.192357,0.403199,0.555836}%
\pgfsetfillcolor{currentfill}%
\pgfsetfillopacity{0.700000}%
\pgfsetlinewidth{0.501875pt}%
\definecolor{currentstroke}{rgb}{1.000000,1.000000,1.000000}%
\pgfsetstrokecolor{currentstroke}%
\pgfsetstrokeopacity{0.700000}%
\pgfsetdash{}{0pt}%
\pgfpathmoveto{\pgfqpoint{2.112664in}{3.552032in}}%
\pgfpathcurveto{\pgfqpoint{2.125687in}{3.552032in}}{\pgfqpoint{2.138178in}{3.557206in}}{\pgfqpoint{2.147386in}{3.566414in}}%
\pgfpathcurveto{\pgfqpoint{2.156595in}{3.575623in}}{\pgfqpoint{2.161769in}{3.588114in}}{\pgfqpoint{2.161769in}{3.601137in}}%
\pgfpathcurveto{\pgfqpoint{2.161769in}{3.614159in}}{\pgfqpoint{2.156595in}{3.626650in}}{\pgfqpoint{2.147386in}{3.635859in}}%
\pgfpathcurveto{\pgfqpoint{2.138178in}{3.645067in}}{\pgfqpoint{2.125687in}{3.650241in}}{\pgfqpoint{2.112664in}{3.650241in}}%
\pgfpathcurveto{\pgfqpoint{2.099642in}{3.650241in}}{\pgfqpoint{2.087150in}{3.645067in}}{\pgfqpoint{2.077942in}{3.635859in}}%
\pgfpathcurveto{\pgfqpoint{2.068734in}{3.626650in}}{\pgfqpoint{2.063560in}{3.614159in}}{\pgfqpoint{2.063560in}{3.601137in}}%
\pgfpathcurveto{\pgfqpoint{2.063560in}{3.588114in}}{\pgfqpoint{2.068734in}{3.575623in}}{\pgfqpoint{2.077942in}{3.566414in}}%
\pgfpathcurveto{\pgfqpoint{2.087150in}{3.557206in}}{\pgfqpoint{2.099642in}{3.552032in}}{\pgfqpoint{2.112664in}{3.552032in}}%
\pgfpathlineto{\pgfqpoint{2.112664in}{3.552032in}}%
\pgfpathclose%
\pgfusepath{stroke,fill}%
\end{pgfscope}%
\begin{pgfscope}%
\pgfpathrectangle{\pgfqpoint{0.786164in}{0.768110in}}{\pgfqpoint{8.851069in}{7.081890in}}%
\pgfusepath{clip}%
\pgfsetbuttcap%
\pgfsetroundjoin%
\definecolor{currentfill}{rgb}{0.208623,0.367752,0.552675}%
\pgfsetfillcolor{currentfill}%
\pgfsetfillopacity{0.700000}%
\pgfsetlinewidth{0.501875pt}%
\definecolor{currentstroke}{rgb}{1.000000,1.000000,1.000000}%
\pgfsetstrokecolor{currentstroke}%
\pgfsetstrokeopacity{0.700000}%
\pgfsetdash{}{0pt}%
\pgfpathmoveto{\pgfqpoint{2.157475in}{3.423979in}}%
\pgfpathcurveto{\pgfqpoint{2.170498in}{3.423979in}}{\pgfqpoint{2.182989in}{3.429153in}}{\pgfqpoint{2.192197in}{3.438361in}}%
\pgfpathcurveto{\pgfqpoint{2.201406in}{3.447570in}}{\pgfqpoint{2.206580in}{3.460061in}}{\pgfqpoint{2.206580in}{3.473084in}}%
\pgfpathcurveto{\pgfqpoint{2.206580in}{3.486106in}}{\pgfqpoint{2.201406in}{3.498597in}}{\pgfqpoint{2.192197in}{3.507806in}}%
\pgfpathcurveto{\pgfqpoint{2.182989in}{3.517014in}}{\pgfqpoint{2.170498in}{3.522188in}}{\pgfqpoint{2.157475in}{3.522188in}}%
\pgfpathcurveto{\pgfqpoint{2.144452in}{3.522188in}}{\pgfqpoint{2.131961in}{3.517014in}}{\pgfqpoint{2.122753in}{3.507806in}}%
\pgfpathcurveto{\pgfqpoint{2.113545in}{3.498597in}}{\pgfqpoint{2.108371in}{3.486106in}}{\pgfqpoint{2.108371in}{3.473084in}}%
\pgfpathcurveto{\pgfqpoint{2.108371in}{3.460061in}}{\pgfqpoint{2.113545in}{3.447570in}}{\pgfqpoint{2.122753in}{3.438361in}}%
\pgfpathcurveto{\pgfqpoint{2.131961in}{3.429153in}}{\pgfqpoint{2.144452in}{3.423979in}}{\pgfqpoint{2.157475in}{3.423979in}}%
\pgfpathlineto{\pgfqpoint{2.157475in}{3.423979in}}%
\pgfpathclose%
\pgfusepath{stroke,fill}%
\end{pgfscope}%
\begin{pgfscope}%
\pgfpathrectangle{\pgfqpoint{0.786164in}{0.768110in}}{\pgfqpoint{8.851069in}{7.081890in}}%
\pgfusepath{clip}%
\pgfsetbuttcap%
\pgfsetroundjoin%
\definecolor{currentfill}{rgb}{0.208623,0.367752,0.552675}%
\pgfsetfillcolor{currentfill}%
\pgfsetfillopacity{0.700000}%
\pgfsetlinewidth{0.501875pt}%
\definecolor{currentstroke}{rgb}{1.000000,1.000000,1.000000}%
\pgfsetstrokecolor{currentstroke}%
\pgfsetstrokeopacity{0.700000}%
\pgfsetdash{}{0pt}%
\pgfpathmoveto{\pgfqpoint{2.160161in}{3.509348in}}%
\pgfpathcurveto{\pgfqpoint{2.173184in}{3.509348in}}{\pgfqpoint{2.185675in}{3.514522in}}{\pgfqpoint{2.194884in}{3.523730in}}%
\pgfpathcurveto{\pgfqpoint{2.204092in}{3.532938in}}{\pgfqpoint{2.209266in}{3.545429in}}{\pgfqpoint{2.209266in}{3.558452in}}%
\pgfpathcurveto{\pgfqpoint{2.209266in}{3.571475in}}{\pgfqpoint{2.204092in}{3.583966in}}{\pgfqpoint{2.194884in}{3.593174in}}%
\pgfpathcurveto{\pgfqpoint{2.185675in}{3.602383in}}{\pgfqpoint{2.173184in}{3.607557in}}{\pgfqpoint{2.160161in}{3.607557in}}%
\pgfpathcurveto{\pgfqpoint{2.147139in}{3.607557in}}{\pgfqpoint{2.134648in}{3.602383in}}{\pgfqpoint{2.125439in}{3.593174in}}%
\pgfpathcurveto{\pgfqpoint{2.116231in}{3.583966in}}{\pgfqpoint{2.111057in}{3.571475in}}{\pgfqpoint{2.111057in}{3.558452in}}%
\pgfpathcurveto{\pgfqpoint{2.111057in}{3.545429in}}{\pgfqpoint{2.116231in}{3.532938in}}{\pgfqpoint{2.125439in}{3.523730in}}%
\pgfpathcurveto{\pgfqpoint{2.134648in}{3.514522in}}{\pgfqpoint{2.147139in}{3.509348in}}{\pgfqpoint{2.160161in}{3.509348in}}%
\pgfpathlineto{\pgfqpoint{2.160161in}{3.509348in}}%
\pgfpathclose%
\pgfusepath{stroke,fill}%
\end{pgfscope}%
\begin{pgfscope}%
\pgfpathrectangle{\pgfqpoint{0.786164in}{0.768110in}}{\pgfqpoint{8.851069in}{7.081890in}}%
\pgfusepath{clip}%
\pgfsetbuttcap%
\pgfsetroundjoin%
\definecolor{currentfill}{rgb}{0.195860,0.395433,0.555276}%
\pgfsetfillcolor{currentfill}%
\pgfsetfillopacity{0.700000}%
\pgfsetlinewidth{0.501875pt}%
\definecolor{currentstroke}{rgb}{1.000000,1.000000,1.000000}%
\pgfsetstrokecolor{currentstroke}%
\pgfsetstrokeopacity{0.700000}%
\pgfsetdash{}{0pt}%
\pgfpathmoveto{\pgfqpoint{2.243678in}{3.466663in}}%
\pgfpathcurveto{\pgfqpoint{2.256701in}{3.466663in}}{\pgfqpoint{2.269192in}{3.471837in}}{\pgfqpoint{2.278400in}{3.481046in}}%
\pgfpathcurveto{\pgfqpoint{2.287609in}{3.490254in}}{\pgfqpoint{2.292783in}{3.502745in}}{\pgfqpoint{2.292783in}{3.515768in}}%
\pgfpathcurveto{\pgfqpoint{2.292783in}{3.528791in}}{\pgfqpoint{2.287609in}{3.541282in}}{\pgfqpoint{2.278400in}{3.550490in}}%
\pgfpathcurveto{\pgfqpoint{2.269192in}{3.559699in}}{\pgfqpoint{2.256701in}{3.564872in}}{\pgfqpoint{2.243678in}{3.564872in}}%
\pgfpathcurveto{\pgfqpoint{2.230655in}{3.564872in}}{\pgfqpoint{2.218164in}{3.559699in}}{\pgfqpoint{2.208956in}{3.550490in}}%
\pgfpathcurveto{\pgfqpoint{2.199747in}{3.541282in}}{\pgfqpoint{2.194574in}{3.528791in}}{\pgfqpoint{2.194574in}{3.515768in}}%
\pgfpathcurveto{\pgfqpoint{2.194574in}{3.502745in}}{\pgfqpoint{2.199747in}{3.490254in}}{\pgfqpoint{2.208956in}{3.481046in}}%
\pgfpathcurveto{\pgfqpoint{2.218164in}{3.471837in}}{\pgfqpoint{2.230655in}{3.466663in}}{\pgfqpoint{2.243678in}{3.466663in}}%
\pgfpathlineto{\pgfqpoint{2.243678in}{3.466663in}}%
\pgfpathclose%
\pgfusepath{stroke,fill}%
\end{pgfscope}%
\begin{pgfscope}%
\pgfpathrectangle{\pgfqpoint{0.786164in}{0.768110in}}{\pgfqpoint{8.851069in}{7.081890in}}%
\pgfusepath{clip}%
\pgfsetbuttcap%
\pgfsetroundjoin%
\definecolor{currentfill}{rgb}{0.197636,0.391528,0.554969}%
\pgfsetfillcolor{currentfill}%
\pgfsetfillopacity{0.700000}%
\pgfsetlinewidth{0.501875pt}%
\definecolor{currentstroke}{rgb}{1.000000,1.000000,1.000000}%
\pgfsetstrokecolor{currentstroke}%
\pgfsetstrokeopacity{0.700000}%
\pgfsetdash{}{0pt}%
\pgfpathmoveto{\pgfqpoint{2.291542in}{3.466663in}}%
\pgfpathcurveto{\pgfqpoint{2.304564in}{3.466663in}}{\pgfqpoint{2.317055in}{3.471837in}}{\pgfqpoint{2.326264in}{3.481046in}}%
\pgfpathcurveto{\pgfqpoint{2.335472in}{3.490254in}}{\pgfqpoint{2.340646in}{3.502745in}}{\pgfqpoint{2.340646in}{3.515768in}}%
\pgfpathcurveto{\pgfqpoint{2.340646in}{3.528791in}}{\pgfqpoint{2.335472in}{3.541282in}}{\pgfqpoint{2.326264in}{3.550490in}}%
\pgfpathcurveto{\pgfqpoint{2.317055in}{3.559699in}}{\pgfqpoint{2.304564in}{3.564872in}}{\pgfqpoint{2.291542in}{3.564872in}}%
\pgfpathcurveto{\pgfqpoint{2.278519in}{3.564872in}}{\pgfqpoint{2.266028in}{3.559699in}}{\pgfqpoint{2.256819in}{3.550490in}}%
\pgfpathcurveto{\pgfqpoint{2.247611in}{3.541282in}}{\pgfqpoint{2.242437in}{3.528791in}}{\pgfqpoint{2.242437in}{3.515768in}}%
\pgfpathcurveto{\pgfqpoint{2.242437in}{3.502745in}}{\pgfqpoint{2.247611in}{3.490254in}}{\pgfqpoint{2.256819in}{3.481046in}}%
\pgfpathcurveto{\pgfqpoint{2.266028in}{3.471837in}}{\pgfqpoint{2.278519in}{3.466663in}}{\pgfqpoint{2.291542in}{3.466663in}}%
\pgfpathlineto{\pgfqpoint{2.291542in}{3.466663in}}%
\pgfpathclose%
\pgfusepath{stroke,fill}%
\end{pgfscope}%
\begin{pgfscope}%
\pgfpathrectangle{\pgfqpoint{0.786164in}{0.768110in}}{\pgfqpoint{8.851069in}{7.081890in}}%
\pgfusepath{clip}%
\pgfsetbuttcap%
\pgfsetroundjoin%
\definecolor{currentfill}{rgb}{0.206756,0.371758,0.553117}%
\pgfsetfillcolor{currentfill}%
\pgfsetfillopacity{0.700000}%
\pgfsetlinewidth{0.501875pt}%
\definecolor{currentstroke}{rgb}{1.000000,1.000000,1.000000}%
\pgfsetstrokecolor{currentstroke}%
\pgfsetstrokeopacity{0.700000}%
\pgfsetdash{}{0pt}%
\pgfpathmoveto{\pgfqpoint{2.332567in}{3.466663in}}%
\pgfpathcurveto{\pgfqpoint{2.345590in}{3.466663in}}{\pgfqpoint{2.358081in}{3.471837in}}{\pgfqpoint{2.367290in}{3.481046in}}%
\pgfpathcurveto{\pgfqpoint{2.376498in}{3.490254in}}{\pgfqpoint{2.381672in}{3.502745in}}{\pgfqpoint{2.381672in}{3.515768in}}%
\pgfpathcurveto{\pgfqpoint{2.381672in}{3.528791in}}{\pgfqpoint{2.376498in}{3.541282in}}{\pgfqpoint{2.367290in}{3.550490in}}%
\pgfpathcurveto{\pgfqpoint{2.358081in}{3.559699in}}{\pgfqpoint{2.345590in}{3.564872in}}{\pgfqpoint{2.332567in}{3.564872in}}%
\pgfpathcurveto{\pgfqpoint{2.319545in}{3.564872in}}{\pgfqpoint{2.307054in}{3.559699in}}{\pgfqpoint{2.297845in}{3.550490in}}%
\pgfpathcurveto{\pgfqpoint{2.288637in}{3.541282in}}{\pgfqpoint{2.283463in}{3.528791in}}{\pgfqpoint{2.283463in}{3.515768in}}%
\pgfpathcurveto{\pgfqpoint{2.283463in}{3.502745in}}{\pgfqpoint{2.288637in}{3.490254in}}{\pgfqpoint{2.297845in}{3.481046in}}%
\pgfpathcurveto{\pgfqpoint{2.307054in}{3.471837in}}{\pgfqpoint{2.319545in}{3.466663in}}{\pgfqpoint{2.332567in}{3.466663in}}%
\pgfpathlineto{\pgfqpoint{2.332567in}{3.466663in}}%
\pgfpathclose%
\pgfusepath{stroke,fill}%
\end{pgfscope}%
\begin{pgfscope}%
\pgfpathrectangle{\pgfqpoint{0.786164in}{0.768110in}}{\pgfqpoint{8.851069in}{7.081890in}}%
\pgfusepath{clip}%
\pgfsetbuttcap%
\pgfsetroundjoin%
\definecolor{currentfill}{rgb}{0.199430,0.387607,0.554642}%
\pgfsetfillcolor{currentfill}%
\pgfsetfillopacity{0.700000}%
\pgfsetlinewidth{0.501875pt}%
\definecolor{currentstroke}{rgb}{1.000000,1.000000,1.000000}%
\pgfsetstrokecolor{currentstroke}%
\pgfsetstrokeopacity{0.700000}%
\pgfsetdash{}{0pt}%
\pgfpathmoveto{\pgfqpoint{2.388367in}{3.445321in}}%
\pgfpathcurveto{\pgfqpoint{2.401390in}{3.445321in}}{\pgfqpoint{2.413881in}{3.450495in}}{\pgfqpoint{2.423090in}{3.459703in}}%
\pgfpathcurveto{\pgfqpoint{2.432298in}{3.468912in}}{\pgfqpoint{2.437472in}{3.481403in}}{\pgfqpoint{2.437472in}{3.494426in}}%
\pgfpathcurveto{\pgfqpoint{2.437472in}{3.507448in}}{\pgfqpoint{2.432298in}{3.519939in}}{\pgfqpoint{2.423090in}{3.529148in}}%
\pgfpathcurveto{\pgfqpoint{2.413881in}{3.538356in}}{\pgfqpoint{2.401390in}{3.543530in}}{\pgfqpoint{2.388367in}{3.543530in}}%
\pgfpathcurveto{\pgfqpoint{2.375345in}{3.543530in}}{\pgfqpoint{2.362854in}{3.538356in}}{\pgfqpoint{2.353645in}{3.529148in}}%
\pgfpathcurveto{\pgfqpoint{2.344437in}{3.519939in}}{\pgfqpoint{2.339263in}{3.507448in}}{\pgfqpoint{2.339263in}{3.494426in}}%
\pgfpathcurveto{\pgfqpoint{2.339263in}{3.481403in}}{\pgfqpoint{2.344437in}{3.468912in}}{\pgfqpoint{2.353645in}{3.459703in}}%
\pgfpathcurveto{\pgfqpoint{2.362854in}{3.450495in}}{\pgfqpoint{2.375345in}{3.445321in}}{\pgfqpoint{2.388367in}{3.445321in}}%
\pgfpathlineto{\pgfqpoint{2.388367in}{3.445321in}}%
\pgfpathclose%
\pgfusepath{stroke,fill}%
\end{pgfscope}%
\begin{pgfscope}%
\pgfpathrectangle{\pgfqpoint{0.786164in}{0.768110in}}{\pgfqpoint{8.851069in}{7.081890in}}%
\pgfusepath{clip}%
\pgfsetbuttcap%
\pgfsetroundjoin%
\definecolor{currentfill}{rgb}{0.221989,0.339161,0.548752}%
\pgfsetfillcolor{currentfill}%
\pgfsetfillopacity{0.700000}%
\pgfsetlinewidth{0.501875pt}%
\definecolor{currentstroke}{rgb}{1.000000,1.000000,1.000000}%
\pgfsetstrokecolor{currentstroke}%
\pgfsetstrokeopacity{0.700000}%
\pgfsetdash{}{0pt}%
\pgfpathmoveto{\pgfqpoint{2.763338in}{3.231899in}}%
\pgfpathcurveto{\pgfqpoint{2.776361in}{3.231899in}}{\pgfqpoint{2.788852in}{3.237073in}}{\pgfqpoint{2.798060in}{3.246282in}}%
\pgfpathcurveto{\pgfqpoint{2.807269in}{3.255490in}}{\pgfqpoint{2.812443in}{3.267981in}}{\pgfqpoint{2.812443in}{3.281004in}}%
\pgfpathcurveto{\pgfqpoint{2.812443in}{3.294027in}}{\pgfqpoint{2.807269in}{3.306518in}}{\pgfqpoint{2.798060in}{3.315726in}}%
\pgfpathcurveto{\pgfqpoint{2.788852in}{3.324935in}}{\pgfqpoint{2.776361in}{3.330109in}}{\pgfqpoint{2.763338in}{3.330109in}}%
\pgfpathcurveto{\pgfqpoint{2.750315in}{3.330109in}}{\pgfqpoint{2.737824in}{3.324935in}}{\pgfqpoint{2.728616in}{3.315726in}}%
\pgfpathcurveto{\pgfqpoint{2.719407in}{3.306518in}}{\pgfqpoint{2.714233in}{3.294027in}}{\pgfqpoint{2.714233in}{3.281004in}}%
\pgfpathcurveto{\pgfqpoint{2.714233in}{3.267981in}}{\pgfqpoint{2.719407in}{3.255490in}}{\pgfqpoint{2.728616in}{3.246282in}}%
\pgfpathcurveto{\pgfqpoint{2.737824in}{3.237073in}}{\pgfqpoint{2.750315in}{3.231899in}}{\pgfqpoint{2.763338in}{3.231899in}}%
\pgfpathlineto{\pgfqpoint{2.763338in}{3.231899in}}%
\pgfpathclose%
\pgfusepath{stroke,fill}%
\end{pgfscope}%
\begin{pgfscope}%
\pgfpathrectangle{\pgfqpoint{0.786164in}{0.768110in}}{\pgfqpoint{8.851069in}{7.081890in}}%
\pgfusepath{clip}%
\pgfsetbuttcap%
\pgfsetroundjoin%
\definecolor{currentfill}{rgb}{0.199430,0.387607,0.554642}%
\pgfsetfillcolor{currentfill}%
\pgfsetfillopacity{0.700000}%
\pgfsetlinewidth{0.501875pt}%
\definecolor{currentstroke}{rgb}{1.000000,1.000000,1.000000}%
\pgfsetstrokecolor{currentstroke}%
\pgfsetstrokeopacity{0.700000}%
\pgfsetdash{}{0pt}%
\pgfpathmoveto{\pgfqpoint{2.764193in}{3.295926in}}%
\pgfpathcurveto{\pgfqpoint{2.777216in}{3.295926in}}{\pgfqpoint{2.789707in}{3.301100in}}{\pgfqpoint{2.798915in}{3.310308in}}%
\pgfpathcurveto{\pgfqpoint{2.808123in}{3.319517in}}{\pgfqpoint{2.813297in}{3.332008in}}{\pgfqpoint{2.813297in}{3.345030in}}%
\pgfpathcurveto{\pgfqpoint{2.813297in}{3.358053in}}{\pgfqpoint{2.808123in}{3.370544in}}{\pgfqpoint{2.798915in}{3.379753in}}%
\pgfpathcurveto{\pgfqpoint{2.789707in}{3.388961in}}{\pgfqpoint{2.777216in}{3.394135in}}{\pgfqpoint{2.764193in}{3.394135in}}%
\pgfpathcurveto{\pgfqpoint{2.751170in}{3.394135in}}{\pgfqpoint{2.738679in}{3.388961in}}{\pgfqpoint{2.729471in}{3.379753in}}%
\pgfpathcurveto{\pgfqpoint{2.720262in}{3.370544in}}{\pgfqpoint{2.715088in}{3.358053in}}{\pgfqpoint{2.715088in}{3.345030in}}%
\pgfpathcurveto{\pgfqpoint{2.715088in}{3.332008in}}{\pgfqpoint{2.720262in}{3.319517in}}{\pgfqpoint{2.729471in}{3.310308in}}%
\pgfpathcurveto{\pgfqpoint{2.738679in}{3.301100in}}{\pgfqpoint{2.751170in}{3.295926in}}{\pgfqpoint{2.764193in}{3.295926in}}%
\pgfpathlineto{\pgfqpoint{2.764193in}{3.295926in}}%
\pgfpathclose%
\pgfusepath{stroke,fill}%
\end{pgfscope}%
\begin{pgfscope}%
\pgfpathrectangle{\pgfqpoint{0.786164in}{0.768110in}}{\pgfqpoint{8.851069in}{7.081890in}}%
\pgfusepath{clip}%
\pgfsetbuttcap%
\pgfsetroundjoin%
\definecolor{currentfill}{rgb}{0.218130,0.347432,0.550038}%
\pgfsetfillcolor{currentfill}%
\pgfsetfillopacity{0.700000}%
\pgfsetlinewidth{0.501875pt}%
\definecolor{currentstroke}{rgb}{1.000000,1.000000,1.000000}%
\pgfsetstrokecolor{currentstroke}%
\pgfsetstrokeopacity{0.700000}%
\pgfsetdash{}{0pt}%
\pgfpathmoveto{\pgfqpoint{2.856012in}{3.167873in}}%
\pgfpathcurveto{\pgfqpoint{2.869035in}{3.167873in}}{\pgfqpoint{2.881526in}{3.173047in}}{\pgfqpoint{2.890735in}{3.182255in}}%
\pgfpathcurveto{\pgfqpoint{2.899943in}{3.191464in}}{\pgfqpoint{2.905117in}{3.203955in}}{\pgfqpoint{2.905117in}{3.216977in}}%
\pgfpathcurveto{\pgfqpoint{2.905117in}{3.230000in}}{\pgfqpoint{2.899943in}{3.242491in}}{\pgfqpoint{2.890735in}{3.251700in}}%
\pgfpathcurveto{\pgfqpoint{2.881526in}{3.260908in}}{\pgfqpoint{2.869035in}{3.266082in}}{\pgfqpoint{2.856012in}{3.266082in}}%
\pgfpathcurveto{\pgfqpoint{2.842990in}{3.266082in}}{\pgfqpoint{2.830499in}{3.260908in}}{\pgfqpoint{2.821290in}{3.251700in}}%
\pgfpathcurveto{\pgfqpoint{2.812082in}{3.242491in}}{\pgfqpoint{2.806908in}{3.230000in}}{\pgfqpoint{2.806908in}{3.216977in}}%
\pgfpathcurveto{\pgfqpoint{2.806908in}{3.203955in}}{\pgfqpoint{2.812082in}{3.191464in}}{\pgfqpoint{2.821290in}{3.182255in}}%
\pgfpathcurveto{\pgfqpoint{2.830499in}{3.173047in}}{\pgfqpoint{2.842990in}{3.167873in}}{\pgfqpoint{2.856012in}{3.167873in}}%
\pgfpathlineto{\pgfqpoint{2.856012in}{3.167873in}}%
\pgfpathclose%
\pgfusepath{stroke,fill}%
\end{pgfscope}%
\begin{pgfscope}%
\pgfpathrectangle{\pgfqpoint{0.786164in}{0.768110in}}{\pgfqpoint{8.851069in}{7.081890in}}%
\pgfusepath{clip}%
\pgfsetbuttcap%
\pgfsetroundjoin%
\definecolor{currentfill}{rgb}{0.282290,0.145912,0.461510}%
\pgfsetfillcolor{currentfill}%
\pgfsetfillopacity{0.700000}%
\pgfsetlinewidth{0.501875pt}%
\definecolor{currentstroke}{rgb}{1.000000,1.000000,1.000000}%
\pgfsetstrokecolor{currentstroke}%
\pgfsetstrokeopacity{0.700000}%
\pgfsetdash{}{0pt}%
\pgfpathmoveto{\pgfqpoint{2.303141in}{2.484923in}}%
\pgfpathcurveto{\pgfqpoint{2.316164in}{2.484923in}}{\pgfqpoint{2.328655in}{2.490097in}}{\pgfqpoint{2.337863in}{2.499306in}}%
\pgfpathcurveto{\pgfqpoint{2.347072in}{2.508514in}}{\pgfqpoint{2.352246in}{2.521005in}}{\pgfqpoint{2.352246in}{2.534028in}}%
\pgfpathcurveto{\pgfqpoint{2.352246in}{2.547051in}}{\pgfqpoint{2.347072in}{2.559542in}}{\pgfqpoint{2.337863in}{2.568750in}}%
\pgfpathcurveto{\pgfqpoint{2.328655in}{2.577959in}}{\pgfqpoint{2.316164in}{2.583133in}}{\pgfqpoint{2.303141in}{2.583133in}}%
\pgfpathcurveto{\pgfqpoint{2.290118in}{2.583133in}}{\pgfqpoint{2.277627in}{2.577959in}}{\pgfqpoint{2.268419in}{2.568750in}}%
\pgfpathcurveto{\pgfqpoint{2.259210in}{2.559542in}}{\pgfqpoint{2.254036in}{2.547051in}}{\pgfqpoint{2.254036in}{2.534028in}}%
\pgfpathcurveto{\pgfqpoint{2.254036in}{2.521005in}}{\pgfqpoint{2.259210in}{2.508514in}}{\pgfqpoint{2.268419in}{2.499306in}}%
\pgfpathcurveto{\pgfqpoint{2.277627in}{2.490097in}}{\pgfqpoint{2.290118in}{2.484923in}}{\pgfqpoint{2.303141in}{2.484923in}}%
\pgfpathlineto{\pgfqpoint{2.303141in}{2.484923in}}%
\pgfpathclose%
\pgfusepath{stroke,fill}%
\end{pgfscope}%
\begin{pgfscope}%
\pgfpathrectangle{\pgfqpoint{0.786164in}{0.768110in}}{\pgfqpoint{8.851069in}{7.081890in}}%
\pgfusepath{clip}%
\pgfsetbuttcap%
\pgfsetroundjoin%
\definecolor{currentfill}{rgb}{0.279574,0.170599,0.479997}%
\pgfsetfillcolor{currentfill}%
\pgfsetfillopacity{0.700000}%
\pgfsetlinewidth{0.501875pt}%
\definecolor{currentstroke}{rgb}{1.000000,1.000000,1.000000}%
\pgfsetstrokecolor{currentstroke}%
\pgfsetstrokeopacity{0.700000}%
\pgfsetdash{}{0pt}%
\pgfpathmoveto{\pgfqpoint{2.296059in}{2.484923in}}%
\pgfpathcurveto{\pgfqpoint{2.309082in}{2.484923in}}{\pgfqpoint{2.321573in}{2.490097in}}{\pgfqpoint{2.330782in}{2.499306in}}%
\pgfpathcurveto{\pgfqpoint{2.339990in}{2.508514in}}{\pgfqpoint{2.345164in}{2.521005in}}{\pgfqpoint{2.345164in}{2.534028in}}%
\pgfpathcurveto{\pgfqpoint{2.345164in}{2.547051in}}{\pgfqpoint{2.339990in}{2.559542in}}{\pgfqpoint{2.330782in}{2.568750in}}%
\pgfpathcurveto{\pgfqpoint{2.321573in}{2.577959in}}{\pgfqpoint{2.309082in}{2.583133in}}{\pgfqpoint{2.296059in}{2.583133in}}%
\pgfpathcurveto{\pgfqpoint{2.283037in}{2.583133in}}{\pgfqpoint{2.270546in}{2.577959in}}{\pgfqpoint{2.261337in}{2.568750in}}%
\pgfpathcurveto{\pgfqpoint{2.252129in}{2.559542in}}{\pgfqpoint{2.246955in}{2.547051in}}{\pgfqpoint{2.246955in}{2.534028in}}%
\pgfpathcurveto{\pgfqpoint{2.246955in}{2.521005in}}{\pgfqpoint{2.252129in}{2.508514in}}{\pgfqpoint{2.261337in}{2.499306in}}%
\pgfpathcurveto{\pgfqpoint{2.270546in}{2.490097in}}{\pgfqpoint{2.283037in}{2.484923in}}{\pgfqpoint{2.296059in}{2.484923in}}%
\pgfpathlineto{\pgfqpoint{2.296059in}{2.484923in}}%
\pgfpathclose%
\pgfusepath{stroke,fill}%
\end{pgfscope}%
\begin{pgfscope}%
\pgfpathrectangle{\pgfqpoint{0.786164in}{0.768110in}}{\pgfqpoint{8.851069in}{7.081890in}}%
\pgfusepath{clip}%
\pgfsetbuttcap%
\pgfsetroundjoin%
\definecolor{currentfill}{rgb}{0.278012,0.180367,0.486697}%
\pgfsetfillcolor{currentfill}%
\pgfsetfillopacity{0.700000}%
\pgfsetlinewidth{0.501875pt}%
\definecolor{currentstroke}{rgb}{1.000000,1.000000,1.000000}%
\pgfsetstrokecolor{currentstroke}%
\pgfsetstrokeopacity{0.700000}%
\pgfsetdash{}{0pt}%
\pgfpathmoveto{\pgfqpoint{2.325608in}{2.612976in}}%
\pgfpathcurveto{\pgfqpoint{2.338630in}{2.612976in}}{\pgfqpoint{2.351121in}{2.618150in}}{\pgfqpoint{2.360330in}{2.627359in}}%
\pgfpathcurveto{\pgfqpoint{2.369538in}{2.636567in}}{\pgfqpoint{2.374712in}{2.649058in}}{\pgfqpoint{2.374712in}{2.662081in}}%
\pgfpathcurveto{\pgfqpoint{2.374712in}{2.675104in}}{\pgfqpoint{2.369538in}{2.687595in}}{\pgfqpoint{2.360330in}{2.696803in}}%
\pgfpathcurveto{\pgfqpoint{2.351121in}{2.706012in}}{\pgfqpoint{2.338630in}{2.711186in}}{\pgfqpoint{2.325608in}{2.711186in}}%
\pgfpathcurveto{\pgfqpoint{2.312585in}{2.711186in}}{\pgfqpoint{2.300094in}{2.706012in}}{\pgfqpoint{2.290885in}{2.696803in}}%
\pgfpathcurveto{\pgfqpoint{2.281677in}{2.687595in}}{\pgfqpoint{2.276503in}{2.675104in}}{\pgfqpoint{2.276503in}{2.662081in}}%
\pgfpathcurveto{\pgfqpoint{2.276503in}{2.649058in}}{\pgfqpoint{2.281677in}{2.636567in}}{\pgfqpoint{2.290885in}{2.627359in}}%
\pgfpathcurveto{\pgfqpoint{2.300094in}{2.618150in}}{\pgfqpoint{2.312585in}{2.612976in}}{\pgfqpoint{2.325608in}{2.612976in}}%
\pgfpathlineto{\pgfqpoint{2.325608in}{2.612976in}}%
\pgfpathclose%
\pgfusepath{stroke,fill}%
\end{pgfscope}%
\begin{pgfscope}%
\pgfpathrectangle{\pgfqpoint{0.786164in}{0.768110in}}{\pgfqpoint{8.851069in}{7.081890in}}%
\pgfusepath{clip}%
\pgfsetbuttcap%
\pgfsetroundjoin%
\definecolor{currentfill}{rgb}{0.278012,0.180367,0.486697}%
\pgfsetfillcolor{currentfill}%
\pgfsetfillopacity{0.700000}%
\pgfsetlinewidth{0.501875pt}%
\definecolor{currentstroke}{rgb}{1.000000,1.000000,1.000000}%
\pgfsetstrokecolor{currentstroke}%
\pgfsetstrokeopacity{0.700000}%
\pgfsetdash{}{0pt}%
\pgfpathmoveto{\pgfqpoint{2.286902in}{2.698345in}}%
\pgfpathcurveto{\pgfqpoint{2.299924in}{2.698345in}}{\pgfqpoint{2.312416in}{2.703519in}}{\pgfqpoint{2.321624in}{2.712727in}}%
\pgfpathcurveto{\pgfqpoint{2.330832in}{2.721936in}}{\pgfqpoint{2.336006in}{2.734427in}}{\pgfqpoint{2.336006in}{2.747450in}}%
\pgfpathcurveto{\pgfqpoint{2.336006in}{2.760472in}}{\pgfqpoint{2.330832in}{2.772964in}}{\pgfqpoint{2.321624in}{2.782172in}}%
\pgfpathcurveto{\pgfqpoint{2.312416in}{2.791380in}}{\pgfqpoint{2.299924in}{2.796554in}}{\pgfqpoint{2.286902in}{2.796554in}}%
\pgfpathcurveto{\pgfqpoint{2.273879in}{2.796554in}}{\pgfqpoint{2.261388in}{2.791380in}}{\pgfqpoint{2.252180in}{2.782172in}}%
\pgfpathcurveto{\pgfqpoint{2.242971in}{2.772964in}}{\pgfqpoint{2.237797in}{2.760472in}}{\pgfqpoint{2.237797in}{2.747450in}}%
\pgfpathcurveto{\pgfqpoint{2.237797in}{2.734427in}}{\pgfqpoint{2.242971in}{2.721936in}}{\pgfqpoint{2.252180in}{2.712727in}}%
\pgfpathcurveto{\pgfqpoint{2.261388in}{2.703519in}}{\pgfqpoint{2.273879in}{2.698345in}}{\pgfqpoint{2.286902in}{2.698345in}}%
\pgfpathlineto{\pgfqpoint{2.286902in}{2.698345in}}%
\pgfpathclose%
\pgfusepath{stroke,fill}%
\end{pgfscope}%
\begin{pgfscope}%
\pgfpathrectangle{\pgfqpoint{0.786164in}{0.768110in}}{\pgfqpoint{8.851069in}{7.081890in}}%
\pgfusepath{clip}%
\pgfsetbuttcap%
\pgfsetroundjoin%
\definecolor{currentfill}{rgb}{0.279574,0.170599,0.479997}%
\pgfsetfillcolor{currentfill}%
\pgfsetfillopacity{0.700000}%
\pgfsetlinewidth{0.501875pt}%
\definecolor{currentstroke}{rgb}{1.000000,1.000000,1.000000}%
\pgfsetstrokecolor{currentstroke}%
\pgfsetstrokeopacity{0.700000}%
\pgfsetdash{}{0pt}%
\pgfpathmoveto{\pgfqpoint{2.439161in}{2.719687in}}%
\pgfpathcurveto{\pgfqpoint{2.452184in}{2.719687in}}{\pgfqpoint{2.464675in}{2.724861in}}{\pgfqpoint{2.473883in}{2.734070in}}%
\pgfpathcurveto{\pgfqpoint{2.483092in}{2.743278in}}{\pgfqpoint{2.488266in}{2.755769in}}{\pgfqpoint{2.488266in}{2.768792in}}%
\pgfpathcurveto{\pgfqpoint{2.488266in}{2.781815in}}{\pgfqpoint{2.483092in}{2.794306in}}{\pgfqpoint{2.473883in}{2.803514in}}%
\pgfpathcurveto{\pgfqpoint{2.464675in}{2.812723in}}{\pgfqpoint{2.452184in}{2.817897in}}{\pgfqpoint{2.439161in}{2.817897in}}%
\pgfpathcurveto{\pgfqpoint{2.426138in}{2.817897in}}{\pgfqpoint{2.413647in}{2.812723in}}{\pgfqpoint{2.404439in}{2.803514in}}%
\pgfpathcurveto{\pgfqpoint{2.395230in}{2.794306in}}{\pgfqpoint{2.390057in}{2.781815in}}{\pgfqpoint{2.390057in}{2.768792in}}%
\pgfpathcurveto{\pgfqpoint{2.390057in}{2.755769in}}{\pgfqpoint{2.395230in}{2.743278in}}{\pgfqpoint{2.404439in}{2.734070in}}%
\pgfpathcurveto{\pgfqpoint{2.413647in}{2.724861in}}{\pgfqpoint{2.426138in}{2.719687in}}{\pgfqpoint{2.439161in}{2.719687in}}%
\pgfpathlineto{\pgfqpoint{2.439161in}{2.719687in}}%
\pgfpathclose%
\pgfusepath{stroke,fill}%
\end{pgfscope}%
\begin{pgfscope}%
\pgfpathrectangle{\pgfqpoint{0.786164in}{0.768110in}}{\pgfqpoint{8.851069in}{7.081890in}}%
\pgfusepath{clip}%
\pgfsetbuttcap%
\pgfsetroundjoin%
\definecolor{currentfill}{rgb}{0.282884,0.135920,0.453427}%
\pgfsetfillcolor{currentfill}%
\pgfsetfillopacity{0.700000}%
\pgfsetlinewidth{0.501875pt}%
\definecolor{currentstroke}{rgb}{1.000000,1.000000,1.000000}%
\pgfsetstrokecolor{currentstroke}%
\pgfsetstrokeopacity{0.700000}%
\pgfsetdash{}{0pt}%
\pgfpathmoveto{\pgfqpoint{2.641848in}{2.463581in}}%
\pgfpathcurveto{\pgfqpoint{2.654871in}{2.463581in}}{\pgfqpoint{2.667362in}{2.468755in}}{\pgfqpoint{2.676570in}{2.477964in}}%
\pgfpathcurveto{\pgfqpoint{2.685779in}{2.487172in}}{\pgfqpoint{2.690953in}{2.499663in}}{\pgfqpoint{2.690953in}{2.512686in}}%
\pgfpathcurveto{\pgfqpoint{2.690953in}{2.525709in}}{\pgfqpoint{2.685779in}{2.538200in}}{\pgfqpoint{2.676570in}{2.547408in}}%
\pgfpathcurveto{\pgfqpoint{2.667362in}{2.556617in}}{\pgfqpoint{2.654871in}{2.561790in}}{\pgfqpoint{2.641848in}{2.561790in}}%
\pgfpathcurveto{\pgfqpoint{2.628825in}{2.561790in}}{\pgfqpoint{2.616334in}{2.556617in}}{\pgfqpoint{2.607126in}{2.547408in}}%
\pgfpathcurveto{\pgfqpoint{2.597917in}{2.538200in}}{\pgfqpoint{2.592743in}{2.525709in}}{\pgfqpoint{2.592743in}{2.512686in}}%
\pgfpathcurveto{\pgfqpoint{2.592743in}{2.499663in}}{\pgfqpoint{2.597917in}{2.487172in}}{\pgfqpoint{2.607126in}{2.477964in}}%
\pgfpathcurveto{\pgfqpoint{2.616334in}{2.468755in}}{\pgfqpoint{2.628825in}{2.463581in}}{\pgfqpoint{2.641848in}{2.463581in}}%
\pgfpathlineto{\pgfqpoint{2.641848in}{2.463581in}}%
\pgfpathclose%
\pgfusepath{stroke,fill}%
\end{pgfscope}%
\begin{pgfscope}%
\pgfpathrectangle{\pgfqpoint{0.786164in}{0.768110in}}{\pgfqpoint{8.851069in}{7.081890in}}%
\pgfusepath{clip}%
\pgfsetbuttcap%
\pgfsetroundjoin%
\definecolor{currentfill}{rgb}{0.282290,0.145912,0.461510}%
\pgfsetfillcolor{currentfill}%
\pgfsetfillopacity{0.700000}%
\pgfsetlinewidth{0.501875pt}%
\definecolor{currentstroke}{rgb}{1.000000,1.000000,1.000000}%
\pgfsetstrokecolor{currentstroke}%
\pgfsetstrokeopacity{0.700000}%
\pgfsetdash{}{0pt}%
\pgfpathmoveto{\pgfqpoint{2.876525in}{2.570292in}}%
\pgfpathcurveto{\pgfqpoint{2.889548in}{2.570292in}}{\pgfqpoint{2.902039in}{2.575466in}}{\pgfqpoint{2.911248in}{2.584674in}}%
\pgfpathcurveto{\pgfqpoint{2.920456in}{2.593883in}}{\pgfqpoint{2.925630in}{2.606374in}}{\pgfqpoint{2.925630in}{2.619397in}}%
\pgfpathcurveto{\pgfqpoint{2.925630in}{2.632419in}}{\pgfqpoint{2.920456in}{2.644910in}}{\pgfqpoint{2.911248in}{2.654119in}}%
\pgfpathcurveto{\pgfqpoint{2.902039in}{2.663327in}}{\pgfqpoint{2.889548in}{2.668501in}}{\pgfqpoint{2.876525in}{2.668501in}}%
\pgfpathcurveto{\pgfqpoint{2.863503in}{2.668501in}}{\pgfqpoint{2.851012in}{2.663327in}}{\pgfqpoint{2.841803in}{2.654119in}}%
\pgfpathcurveto{\pgfqpoint{2.832595in}{2.644910in}}{\pgfqpoint{2.827421in}{2.632419in}}{\pgfqpoint{2.827421in}{2.619397in}}%
\pgfpathcurveto{\pgfqpoint{2.827421in}{2.606374in}}{\pgfqpoint{2.832595in}{2.593883in}}{\pgfqpoint{2.841803in}{2.584674in}}%
\pgfpathcurveto{\pgfqpoint{2.851012in}{2.575466in}}{\pgfqpoint{2.863503in}{2.570292in}}{\pgfqpoint{2.876525in}{2.570292in}}%
\pgfpathlineto{\pgfqpoint{2.876525in}{2.570292in}}%
\pgfpathclose%
\pgfusepath{stroke,fill}%
\end{pgfscope}%
\begin{pgfscope}%
\pgfpathrectangle{\pgfqpoint{0.786164in}{0.768110in}}{\pgfqpoint{8.851069in}{7.081890in}}%
\pgfusepath{clip}%
\pgfsetbuttcap%
\pgfsetroundjoin%
\definecolor{currentfill}{rgb}{0.279574,0.170599,0.479997}%
\pgfsetfillcolor{currentfill}%
\pgfsetfillopacity{0.700000}%
\pgfsetlinewidth{0.501875pt}%
\definecolor{currentstroke}{rgb}{1.000000,1.000000,1.000000}%
\pgfsetstrokecolor{currentstroke}%
\pgfsetstrokeopacity{0.700000}%
\pgfsetdash{}{0pt}%
\pgfpathmoveto{\pgfqpoint{2.903998in}{2.826398in}}%
\pgfpathcurveto{\pgfqpoint{2.917021in}{2.826398in}}{\pgfqpoint{2.929512in}{2.831572in}}{\pgfqpoint{2.938720in}{2.840781in}}%
\pgfpathcurveto{\pgfqpoint{2.947929in}{2.849989in}}{\pgfqpoint{2.953103in}{2.862480in}}{\pgfqpoint{2.953103in}{2.875503in}}%
\pgfpathcurveto{\pgfqpoint{2.953103in}{2.888525in}}{\pgfqpoint{2.947929in}{2.901017in}}{\pgfqpoint{2.938720in}{2.910225in}}%
\pgfpathcurveto{\pgfqpoint{2.929512in}{2.919433in}}{\pgfqpoint{2.917021in}{2.924607in}}{\pgfqpoint{2.903998in}{2.924607in}}%
\pgfpathcurveto{\pgfqpoint{2.890975in}{2.924607in}}{\pgfqpoint{2.878484in}{2.919433in}}{\pgfqpoint{2.869276in}{2.910225in}}%
\pgfpathcurveto{\pgfqpoint{2.860067in}{2.901017in}}{\pgfqpoint{2.854893in}{2.888525in}}{\pgfqpoint{2.854893in}{2.875503in}}%
\pgfpathcurveto{\pgfqpoint{2.854893in}{2.862480in}}{\pgfqpoint{2.860067in}{2.849989in}}{\pgfqpoint{2.869276in}{2.840781in}}%
\pgfpathcurveto{\pgfqpoint{2.878484in}{2.831572in}}{\pgfqpoint{2.890975in}{2.826398in}}{\pgfqpoint{2.903998in}{2.826398in}}%
\pgfpathlineto{\pgfqpoint{2.903998in}{2.826398in}}%
\pgfpathclose%
\pgfusepath{stroke,fill}%
\end{pgfscope}%
\begin{pgfscope}%
\pgfpathrectangle{\pgfqpoint{0.786164in}{0.768110in}}{\pgfqpoint{8.851069in}{7.081890in}}%
\pgfusepath{clip}%
\pgfsetbuttcap%
\pgfsetroundjoin%
\definecolor{currentfill}{rgb}{0.280868,0.160771,0.472899}%
\pgfsetfillcolor{currentfill}%
\pgfsetfillopacity{0.700000}%
\pgfsetlinewidth{0.501875pt}%
\definecolor{currentstroke}{rgb}{1.000000,1.000000,1.000000}%
\pgfsetstrokecolor{currentstroke}%
\pgfsetstrokeopacity{0.700000}%
\pgfsetdash{}{0pt}%
\pgfpathmoveto{\pgfqpoint{3.109737in}{2.698345in}}%
\pgfpathcurveto{\pgfqpoint{3.122760in}{2.698345in}}{\pgfqpoint{3.135251in}{2.703519in}}{\pgfqpoint{3.144460in}{2.712727in}}%
\pgfpathcurveto{\pgfqpoint{3.153668in}{2.721936in}}{\pgfqpoint{3.158842in}{2.734427in}}{\pgfqpoint{3.158842in}{2.747450in}}%
\pgfpathcurveto{\pgfqpoint{3.158842in}{2.760472in}}{\pgfqpoint{3.153668in}{2.772964in}}{\pgfqpoint{3.144460in}{2.782172in}}%
\pgfpathcurveto{\pgfqpoint{3.135251in}{2.791380in}}{\pgfqpoint{3.122760in}{2.796554in}}{\pgfqpoint{3.109737in}{2.796554in}}%
\pgfpathcurveto{\pgfqpoint{3.096715in}{2.796554in}}{\pgfqpoint{3.084224in}{2.791380in}}{\pgfqpoint{3.075015in}{2.782172in}}%
\pgfpathcurveto{\pgfqpoint{3.065807in}{2.772964in}}{\pgfqpoint{3.060633in}{2.760472in}}{\pgfqpoint{3.060633in}{2.747450in}}%
\pgfpathcurveto{\pgfqpoint{3.060633in}{2.734427in}}{\pgfqpoint{3.065807in}{2.721936in}}{\pgfqpoint{3.075015in}{2.712727in}}%
\pgfpathcurveto{\pgfqpoint{3.084224in}{2.703519in}}{\pgfqpoint{3.096715in}{2.698345in}}{\pgfqpoint{3.109737in}{2.698345in}}%
\pgfpathlineto{\pgfqpoint{3.109737in}{2.698345in}}%
\pgfpathclose%
\pgfusepath{stroke,fill}%
\end{pgfscope}%
\begin{pgfscope}%
\pgfpathrectangle{\pgfqpoint{0.786164in}{0.768110in}}{\pgfqpoint{8.851069in}{7.081890in}}%
\pgfusepath{clip}%
\pgfsetbuttcap%
\pgfsetroundjoin%
\definecolor{currentfill}{rgb}{0.282884,0.135920,0.453427}%
\pgfsetfillcolor{currentfill}%
\pgfsetfillopacity{0.700000}%
\pgfsetlinewidth{0.501875pt}%
\definecolor{currentstroke}{rgb}{1.000000,1.000000,1.000000}%
\pgfsetstrokecolor{currentstroke}%
\pgfsetstrokeopacity{0.700000}%
\pgfsetdash{}{0pt}%
\pgfpathmoveto{\pgfqpoint{3.483487in}{2.591634in}}%
\pgfpathcurveto{\pgfqpoint{3.496510in}{2.591634in}}{\pgfqpoint{3.509001in}{2.596808in}}{\pgfqpoint{3.518209in}{2.606017in}}%
\pgfpathcurveto{\pgfqpoint{3.527418in}{2.615225in}}{\pgfqpoint{3.532592in}{2.627716in}}{\pgfqpoint{3.532592in}{2.640739in}}%
\pgfpathcurveto{\pgfqpoint{3.532592in}{2.653762in}}{\pgfqpoint{3.527418in}{2.666253in}}{\pgfqpoint{3.518209in}{2.675461in}}%
\pgfpathcurveto{\pgfqpoint{3.509001in}{2.684670in}}{\pgfqpoint{3.496510in}{2.689844in}}{\pgfqpoint{3.483487in}{2.689844in}}%
\pgfpathcurveto{\pgfqpoint{3.470464in}{2.689844in}}{\pgfqpoint{3.457973in}{2.684670in}}{\pgfqpoint{3.448765in}{2.675461in}}%
\pgfpathcurveto{\pgfqpoint{3.439556in}{2.666253in}}{\pgfqpoint{3.434383in}{2.653762in}}{\pgfqpoint{3.434383in}{2.640739in}}%
\pgfpathcurveto{\pgfqpoint{3.434383in}{2.627716in}}{\pgfqpoint{3.439556in}{2.615225in}}{\pgfqpoint{3.448765in}{2.606017in}}%
\pgfpathcurveto{\pgfqpoint{3.457973in}{2.596808in}}{\pgfqpoint{3.470464in}{2.591634in}}{\pgfqpoint{3.483487in}{2.591634in}}%
\pgfpathlineto{\pgfqpoint{3.483487in}{2.591634in}}%
\pgfpathclose%
\pgfusepath{stroke,fill}%
\end{pgfscope}%
\begin{pgfscope}%
\pgfpathrectangle{\pgfqpoint{0.786164in}{0.768110in}}{\pgfqpoint{8.851069in}{7.081890in}}%
\pgfusepath{clip}%
\pgfsetbuttcap%
\pgfsetroundjoin%
\definecolor{currentfill}{rgb}{0.281887,0.150881,0.465405}%
\pgfsetfillcolor{currentfill}%
\pgfsetfillopacity{0.700000}%
\pgfsetlinewidth{0.501875pt}%
\definecolor{currentstroke}{rgb}{1.000000,1.000000,1.000000}%
\pgfsetstrokecolor{currentstroke}%
\pgfsetstrokeopacity{0.700000}%
\pgfsetdash{}{0pt}%
\pgfpathmoveto{\pgfqpoint{3.379946in}{2.634319in}}%
\pgfpathcurveto{\pgfqpoint{3.392969in}{2.634319in}}{\pgfqpoint{3.405460in}{2.639493in}}{\pgfqpoint{3.414668in}{2.648701in}}%
\pgfpathcurveto{\pgfqpoint{3.423877in}{2.657909in}}{\pgfqpoint{3.429051in}{2.670401in}}{\pgfqpoint{3.429051in}{2.683423in}}%
\pgfpathcurveto{\pgfqpoint{3.429051in}{2.696446in}}{\pgfqpoint{3.423877in}{2.708937in}}{\pgfqpoint{3.414668in}{2.718145in}}%
\pgfpathcurveto{\pgfqpoint{3.405460in}{2.727354in}}{\pgfqpoint{3.392969in}{2.732528in}}{\pgfqpoint{3.379946in}{2.732528in}}%
\pgfpathcurveto{\pgfqpoint{3.366923in}{2.732528in}}{\pgfqpoint{3.354432in}{2.727354in}}{\pgfqpoint{3.345224in}{2.718145in}}%
\pgfpathcurveto{\pgfqpoint{3.336015in}{2.708937in}}{\pgfqpoint{3.330841in}{2.696446in}}{\pgfqpoint{3.330841in}{2.683423in}}%
\pgfpathcurveto{\pgfqpoint{3.330841in}{2.670401in}}{\pgfqpoint{3.336015in}{2.657909in}}{\pgfqpoint{3.345224in}{2.648701in}}%
\pgfpathcurveto{\pgfqpoint{3.354432in}{2.639493in}}{\pgfqpoint{3.366923in}{2.634319in}}{\pgfqpoint{3.379946in}{2.634319in}}%
\pgfpathlineto{\pgfqpoint{3.379946in}{2.634319in}}%
\pgfpathclose%
\pgfusepath{stroke,fill}%
\end{pgfscope}%
\begin{pgfscope}%
\pgfpathrectangle{\pgfqpoint{0.786164in}{0.768110in}}{\pgfqpoint{8.851069in}{7.081890in}}%
\pgfusepath{clip}%
\pgfsetbuttcap%
\pgfsetroundjoin%
\definecolor{currentfill}{rgb}{0.280255,0.165693,0.476498}%
\pgfsetfillcolor{currentfill}%
\pgfsetfillopacity{0.700000}%
\pgfsetlinewidth{0.501875pt}%
\definecolor{currentstroke}{rgb}{1.000000,1.000000,1.000000}%
\pgfsetstrokecolor{currentstroke}%
\pgfsetstrokeopacity{0.700000}%
\pgfsetdash{}{0pt}%
\pgfpathmoveto{\pgfqpoint{3.405709in}{2.762372in}}%
\pgfpathcurveto{\pgfqpoint{3.418732in}{2.762372in}}{\pgfqpoint{3.431223in}{2.767546in}}{\pgfqpoint{3.440431in}{2.776754in}}%
\pgfpathcurveto{\pgfqpoint{3.449640in}{2.785962in}}{\pgfqpoint{3.454814in}{2.798454in}}{\pgfqpoint{3.454814in}{2.811476in}}%
\pgfpathcurveto{\pgfqpoint{3.454814in}{2.824499in}}{\pgfqpoint{3.449640in}{2.836990in}}{\pgfqpoint{3.440431in}{2.846198in}}%
\pgfpathcurveto{\pgfqpoint{3.431223in}{2.855407in}}{\pgfqpoint{3.418732in}{2.860581in}}{\pgfqpoint{3.405709in}{2.860581in}}%
\pgfpathcurveto{\pgfqpoint{3.392686in}{2.860581in}}{\pgfqpoint{3.380195in}{2.855407in}}{\pgfqpoint{3.370987in}{2.846198in}}%
\pgfpathcurveto{\pgfqpoint{3.361778in}{2.836990in}}{\pgfqpoint{3.356604in}{2.824499in}}{\pgfqpoint{3.356604in}{2.811476in}}%
\pgfpathcurveto{\pgfqpoint{3.356604in}{2.798454in}}{\pgfqpoint{3.361778in}{2.785962in}}{\pgfqpoint{3.370987in}{2.776754in}}%
\pgfpathcurveto{\pgfqpoint{3.380195in}{2.767546in}}{\pgfqpoint{3.392686in}{2.762372in}}{\pgfqpoint{3.405709in}{2.762372in}}%
\pgfpathlineto{\pgfqpoint{3.405709in}{2.762372in}}%
\pgfpathclose%
\pgfusepath{stroke,fill}%
\end{pgfscope}%
\begin{pgfscope}%
\pgfpathrectangle{\pgfqpoint{0.786164in}{0.768110in}}{\pgfqpoint{8.851069in}{7.081890in}}%
\pgfusepath{clip}%
\pgfsetbuttcap%
\pgfsetroundjoin%
\definecolor{currentfill}{rgb}{0.280255,0.165693,0.476498}%
\pgfsetfillcolor{currentfill}%
\pgfsetfillopacity{0.700000}%
\pgfsetlinewidth{0.501875pt}%
\definecolor{currentstroke}{rgb}{1.000000,1.000000,1.000000}%
\pgfsetstrokecolor{currentstroke}%
\pgfsetstrokeopacity{0.700000}%
\pgfsetdash{}{0pt}%
\pgfpathmoveto{\pgfqpoint{3.466637in}{2.634319in}}%
\pgfpathcurveto{\pgfqpoint{3.479660in}{2.634319in}}{\pgfqpoint{3.492151in}{2.639493in}}{\pgfqpoint{3.501359in}{2.648701in}}%
\pgfpathcurveto{\pgfqpoint{3.510568in}{2.657909in}}{\pgfqpoint{3.515742in}{2.670401in}}{\pgfqpoint{3.515742in}{2.683423in}}%
\pgfpathcurveto{\pgfqpoint{3.515742in}{2.696446in}}{\pgfqpoint{3.510568in}{2.708937in}}{\pgfqpoint{3.501359in}{2.718145in}}%
\pgfpathcurveto{\pgfqpoint{3.492151in}{2.727354in}}{\pgfqpoint{3.479660in}{2.732528in}}{\pgfqpoint{3.466637in}{2.732528in}}%
\pgfpathcurveto{\pgfqpoint{3.453615in}{2.732528in}}{\pgfqpoint{3.441123in}{2.727354in}}{\pgfqpoint{3.431915in}{2.718145in}}%
\pgfpathcurveto{\pgfqpoint{3.422707in}{2.708937in}}{\pgfqpoint{3.417533in}{2.696446in}}{\pgfqpoint{3.417533in}{2.683423in}}%
\pgfpathcurveto{\pgfqpoint{3.417533in}{2.670401in}}{\pgfqpoint{3.422707in}{2.657909in}}{\pgfqpoint{3.431915in}{2.648701in}}%
\pgfpathcurveto{\pgfqpoint{3.441123in}{2.639493in}}{\pgfqpoint{3.453615in}{2.634319in}}{\pgfqpoint{3.466637in}{2.634319in}}%
\pgfpathlineto{\pgfqpoint{3.466637in}{2.634319in}}%
\pgfpathclose%
\pgfusepath{stroke,fill}%
\end{pgfscope}%
\begin{pgfscope}%
\pgfpathrectangle{\pgfqpoint{0.786164in}{0.768110in}}{\pgfqpoint{8.851069in}{7.081890in}}%
\pgfusepath{clip}%
\pgfsetbuttcap%
\pgfsetroundjoin%
\definecolor{currentfill}{rgb}{0.278012,0.180367,0.486697}%
\pgfsetfillcolor{currentfill}%
\pgfsetfillopacity{0.700000}%
\pgfsetlinewidth{0.501875pt}%
\definecolor{currentstroke}{rgb}{1.000000,1.000000,1.000000}%
\pgfsetstrokecolor{currentstroke}%
\pgfsetstrokeopacity{0.700000}%
\pgfsetdash{}{0pt}%
\pgfpathmoveto{\pgfqpoint{3.458701in}{2.719687in}}%
\pgfpathcurveto{\pgfqpoint{3.471723in}{2.719687in}}{\pgfqpoint{3.484215in}{2.724861in}}{\pgfqpoint{3.493423in}{2.734070in}}%
\pgfpathcurveto{\pgfqpoint{3.502631in}{2.743278in}}{\pgfqpoint{3.507805in}{2.755769in}}{\pgfqpoint{3.507805in}{2.768792in}}%
\pgfpathcurveto{\pgfqpoint{3.507805in}{2.781815in}}{\pgfqpoint{3.502631in}{2.794306in}}{\pgfqpoint{3.493423in}{2.803514in}}%
\pgfpathcurveto{\pgfqpoint{3.484215in}{2.812723in}}{\pgfqpoint{3.471723in}{2.817897in}}{\pgfqpoint{3.458701in}{2.817897in}}%
\pgfpathcurveto{\pgfqpoint{3.445678in}{2.817897in}}{\pgfqpoint{3.433187in}{2.812723in}}{\pgfqpoint{3.423979in}{2.803514in}}%
\pgfpathcurveto{\pgfqpoint{3.414770in}{2.794306in}}{\pgfqpoint{3.409596in}{2.781815in}}{\pgfqpoint{3.409596in}{2.768792in}}%
\pgfpathcurveto{\pgfqpoint{3.409596in}{2.755769in}}{\pgfqpoint{3.414770in}{2.743278in}}{\pgfqpoint{3.423979in}{2.734070in}}%
\pgfpathcurveto{\pgfqpoint{3.433187in}{2.724861in}}{\pgfqpoint{3.445678in}{2.719687in}}{\pgfqpoint{3.458701in}{2.719687in}}%
\pgfpathlineto{\pgfqpoint{3.458701in}{2.719687in}}%
\pgfpathclose%
\pgfusepath{stroke,fill}%
\end{pgfscope}%
\begin{pgfscope}%
\pgfpathrectangle{\pgfqpoint{0.786164in}{0.768110in}}{\pgfqpoint{8.851069in}{7.081890in}}%
\pgfusepath{clip}%
\pgfsetbuttcap%
\pgfsetroundjoin%
\definecolor{currentfill}{rgb}{0.278012,0.180367,0.486697}%
\pgfsetfillcolor{currentfill}%
\pgfsetfillopacity{0.700000}%
\pgfsetlinewidth{0.501875pt}%
\definecolor{currentstroke}{rgb}{1.000000,1.000000,1.000000}%
\pgfsetstrokecolor{currentstroke}%
\pgfsetstrokeopacity{0.700000}%
\pgfsetdash{}{0pt}%
\pgfpathmoveto{\pgfqpoint{3.688982in}{2.591634in}}%
\pgfpathcurveto{\pgfqpoint{3.702005in}{2.591634in}}{\pgfqpoint{3.714496in}{2.596808in}}{\pgfqpoint{3.723705in}{2.606017in}}%
\pgfpathcurveto{\pgfqpoint{3.732913in}{2.615225in}}{\pgfqpoint{3.738087in}{2.627716in}}{\pgfqpoint{3.738087in}{2.640739in}}%
\pgfpathcurveto{\pgfqpoint{3.738087in}{2.653762in}}{\pgfqpoint{3.732913in}{2.666253in}}{\pgfqpoint{3.723705in}{2.675461in}}%
\pgfpathcurveto{\pgfqpoint{3.714496in}{2.684670in}}{\pgfqpoint{3.702005in}{2.689844in}}{\pgfqpoint{3.688982in}{2.689844in}}%
\pgfpathcurveto{\pgfqpoint{3.675960in}{2.689844in}}{\pgfqpoint{3.663469in}{2.684670in}}{\pgfqpoint{3.654260in}{2.675461in}}%
\pgfpathcurveto{\pgfqpoint{3.645052in}{2.666253in}}{\pgfqpoint{3.639878in}{2.653762in}}{\pgfqpoint{3.639878in}{2.640739in}}%
\pgfpathcurveto{\pgfqpoint{3.639878in}{2.627716in}}{\pgfqpoint{3.645052in}{2.615225in}}{\pgfqpoint{3.654260in}{2.606017in}}%
\pgfpathcurveto{\pgfqpoint{3.663469in}{2.596808in}}{\pgfqpoint{3.675960in}{2.591634in}}{\pgfqpoint{3.688982in}{2.591634in}}%
\pgfpathlineto{\pgfqpoint{3.688982in}{2.591634in}}%
\pgfpathclose%
\pgfusepath{stroke,fill}%
\end{pgfscope}%
\begin{pgfscope}%
\pgfpathrectangle{\pgfqpoint{0.786164in}{0.768110in}}{\pgfqpoint{8.851069in}{7.081890in}}%
\pgfusepath{clip}%
\pgfsetbuttcap%
\pgfsetroundjoin%
\definecolor{currentfill}{rgb}{0.278012,0.180367,0.486697}%
\pgfsetfillcolor{currentfill}%
\pgfsetfillopacity{0.700000}%
\pgfsetlinewidth{0.501875pt}%
\definecolor{currentstroke}{rgb}{1.000000,1.000000,1.000000}%
\pgfsetstrokecolor{currentstroke}%
\pgfsetstrokeopacity{0.700000}%
\pgfsetdash{}{0pt}%
\pgfpathmoveto{\pgfqpoint{3.806687in}{2.570292in}}%
\pgfpathcurveto{\pgfqpoint{3.819710in}{2.570292in}}{\pgfqpoint{3.832201in}{2.575466in}}{\pgfqpoint{3.841410in}{2.584674in}}%
\pgfpathcurveto{\pgfqpoint{3.850618in}{2.593883in}}{\pgfqpoint{3.855792in}{2.606374in}}{\pgfqpoint{3.855792in}{2.619397in}}%
\pgfpathcurveto{\pgfqpoint{3.855792in}{2.632419in}}{\pgfqpoint{3.850618in}{2.644910in}}{\pgfqpoint{3.841410in}{2.654119in}}%
\pgfpathcurveto{\pgfqpoint{3.832201in}{2.663327in}}{\pgfqpoint{3.819710in}{2.668501in}}{\pgfqpoint{3.806687in}{2.668501in}}%
\pgfpathcurveto{\pgfqpoint{3.793665in}{2.668501in}}{\pgfqpoint{3.781174in}{2.663327in}}{\pgfqpoint{3.771965in}{2.654119in}}%
\pgfpathcurveto{\pgfqpoint{3.762757in}{2.644910in}}{\pgfqpoint{3.757583in}{2.632419in}}{\pgfqpoint{3.757583in}{2.619397in}}%
\pgfpathcurveto{\pgfqpoint{3.757583in}{2.606374in}}{\pgfqpoint{3.762757in}{2.593883in}}{\pgfqpoint{3.771965in}{2.584674in}}%
\pgfpathcurveto{\pgfqpoint{3.781174in}{2.575466in}}{\pgfqpoint{3.793665in}{2.570292in}}{\pgfqpoint{3.806687in}{2.570292in}}%
\pgfpathlineto{\pgfqpoint{3.806687in}{2.570292in}}%
\pgfpathclose%
\pgfusepath{stroke,fill}%
\end{pgfscope}%
\begin{pgfscope}%
\pgfpathrectangle{\pgfqpoint{0.786164in}{0.768110in}}{\pgfqpoint{8.851069in}{7.081890in}}%
\pgfusepath{clip}%
\pgfsetbuttcap%
\pgfsetroundjoin%
\definecolor{currentfill}{rgb}{0.280868,0.160771,0.472899}%
\pgfsetfillcolor{currentfill}%
\pgfsetfillopacity{0.700000}%
\pgfsetlinewidth{0.501875pt}%
\definecolor{currentstroke}{rgb}{1.000000,1.000000,1.000000}%
\pgfsetstrokecolor{currentstroke}%
\pgfsetstrokeopacity{0.700000}%
\pgfsetdash{}{0pt}%
\pgfpathmoveto{\pgfqpoint{4.023294in}{2.378213in}}%
\pgfpathcurveto{\pgfqpoint{4.036316in}{2.378213in}}{\pgfqpoint{4.048807in}{2.383386in}}{\pgfqpoint{4.058016in}{2.392595in}}%
\pgfpathcurveto{\pgfqpoint{4.067224in}{2.401803in}}{\pgfqpoint{4.072398in}{2.414294in}}{\pgfqpoint{4.072398in}{2.427317in}}%
\pgfpathcurveto{\pgfqpoint{4.072398in}{2.440340in}}{\pgfqpoint{4.067224in}{2.452831in}}{\pgfqpoint{4.058016in}{2.462039in}}%
\pgfpathcurveto{\pgfqpoint{4.048807in}{2.471248in}}{\pgfqpoint{4.036316in}{2.476422in}}{\pgfqpoint{4.023294in}{2.476422in}}%
\pgfpathcurveto{\pgfqpoint{4.010271in}{2.476422in}}{\pgfqpoint{3.997780in}{2.471248in}}{\pgfqpoint{3.988571in}{2.462039in}}%
\pgfpathcurveto{\pgfqpoint{3.979363in}{2.452831in}}{\pgfqpoint{3.974189in}{2.440340in}}{\pgfqpoint{3.974189in}{2.427317in}}%
\pgfpathcurveto{\pgfqpoint{3.974189in}{2.414294in}}{\pgfqpoint{3.979363in}{2.401803in}}{\pgfqpoint{3.988571in}{2.392595in}}%
\pgfpathcurveto{\pgfqpoint{3.997780in}{2.383386in}}{\pgfqpoint{4.010271in}{2.378213in}}{\pgfqpoint{4.023294in}{2.378213in}}%
\pgfpathlineto{\pgfqpoint{4.023294in}{2.378213in}}%
\pgfpathclose%
\pgfusepath{stroke,fill}%
\end{pgfscope}%
\begin{pgfscope}%
\pgfpathrectangle{\pgfqpoint{0.786164in}{0.768110in}}{\pgfqpoint{8.851069in}{7.081890in}}%
\pgfusepath{clip}%
\pgfsetbuttcap%
\pgfsetroundjoin%
\definecolor{currentfill}{rgb}{0.275191,0.194905,0.496005}%
\pgfsetfillcolor{currentfill}%
\pgfsetfillopacity{0.700000}%
\pgfsetlinewidth{0.501875pt}%
\definecolor{currentstroke}{rgb}{1.000000,1.000000,1.000000}%
\pgfsetstrokecolor{currentstroke}%
\pgfsetstrokeopacity{0.700000}%
\pgfsetdash{}{0pt}%
\pgfpathmoveto{\pgfqpoint{3.550520in}{2.570292in}}%
\pgfpathcurveto{\pgfqpoint{3.563543in}{2.570292in}}{\pgfqpoint{3.576034in}{2.575466in}}{\pgfqpoint{3.585243in}{2.584674in}}%
\pgfpathcurveto{\pgfqpoint{3.594451in}{2.593883in}}{\pgfqpoint{3.599625in}{2.606374in}}{\pgfqpoint{3.599625in}{2.619397in}}%
\pgfpathcurveto{\pgfqpoint{3.599625in}{2.632419in}}{\pgfqpoint{3.594451in}{2.644910in}}{\pgfqpoint{3.585243in}{2.654119in}}%
\pgfpathcurveto{\pgfqpoint{3.576034in}{2.663327in}}{\pgfqpoint{3.563543in}{2.668501in}}{\pgfqpoint{3.550520in}{2.668501in}}%
\pgfpathcurveto{\pgfqpoint{3.537498in}{2.668501in}}{\pgfqpoint{3.525007in}{2.663327in}}{\pgfqpoint{3.515798in}{2.654119in}}%
\pgfpathcurveto{\pgfqpoint{3.506590in}{2.644910in}}{\pgfqpoint{3.501416in}{2.632419in}}{\pgfqpoint{3.501416in}{2.619397in}}%
\pgfpathcurveto{\pgfqpoint{3.501416in}{2.606374in}}{\pgfqpoint{3.506590in}{2.593883in}}{\pgfqpoint{3.515798in}{2.584674in}}%
\pgfpathcurveto{\pgfqpoint{3.525007in}{2.575466in}}{\pgfqpoint{3.537498in}{2.570292in}}{\pgfqpoint{3.550520in}{2.570292in}}%
\pgfpathlineto{\pgfqpoint{3.550520in}{2.570292in}}%
\pgfpathclose%
\pgfusepath{stroke,fill}%
\end{pgfscope}%
\begin{pgfscope}%
\pgfpathrectangle{\pgfqpoint{0.786164in}{0.768110in}}{\pgfqpoint{8.851069in}{7.081890in}}%
\pgfusepath{clip}%
\pgfsetbuttcap%
\pgfsetroundjoin%
\definecolor{currentfill}{rgb}{0.270595,0.214069,0.507052}%
\pgfsetfillcolor{currentfill}%
\pgfsetfillopacity{0.700000}%
\pgfsetlinewidth{0.501875pt}%
\definecolor{currentstroke}{rgb}{1.000000,1.000000,1.000000}%
\pgfsetstrokecolor{currentstroke}%
\pgfsetstrokeopacity{0.700000}%
\pgfsetdash{}{0pt}%
\pgfpathmoveto{\pgfqpoint{3.507541in}{2.783714in}}%
\pgfpathcurveto{\pgfqpoint{3.520564in}{2.783714in}}{\pgfqpoint{3.533055in}{2.788888in}}{\pgfqpoint{3.542263in}{2.798096in}}%
\pgfpathcurveto{\pgfqpoint{3.551472in}{2.807305in}}{\pgfqpoint{3.556646in}{2.819796in}}{\pgfqpoint{3.556646in}{2.832818in}}%
\pgfpathcurveto{\pgfqpoint{3.556646in}{2.845841in}}{\pgfqpoint{3.551472in}{2.858332in}}{\pgfqpoint{3.542263in}{2.867541in}}%
\pgfpathcurveto{\pgfqpoint{3.533055in}{2.876749in}}{\pgfqpoint{3.520564in}{2.881923in}}{\pgfqpoint{3.507541in}{2.881923in}}%
\pgfpathcurveto{\pgfqpoint{3.494518in}{2.881923in}}{\pgfqpoint{3.482027in}{2.876749in}}{\pgfqpoint{3.472819in}{2.867541in}}%
\pgfpathcurveto{\pgfqpoint{3.463610in}{2.858332in}}{\pgfqpoint{3.458436in}{2.845841in}}{\pgfqpoint{3.458436in}{2.832818in}}%
\pgfpathcurveto{\pgfqpoint{3.458436in}{2.819796in}}{\pgfqpoint{3.463610in}{2.807305in}}{\pgfqpoint{3.472819in}{2.798096in}}%
\pgfpathcurveto{\pgfqpoint{3.482027in}{2.788888in}}{\pgfqpoint{3.494518in}{2.783714in}}{\pgfqpoint{3.507541in}{2.783714in}}%
\pgfpathlineto{\pgfqpoint{3.507541in}{2.783714in}}%
\pgfpathclose%
\pgfusepath{stroke,fill}%
\end{pgfscope}%
\begin{pgfscope}%
\pgfpathrectangle{\pgfqpoint{0.786164in}{0.768110in}}{\pgfqpoint{8.851069in}{7.081890in}}%
\pgfusepath{clip}%
\pgfsetbuttcap%
\pgfsetroundjoin%
\definecolor{currentfill}{rgb}{0.246811,0.283237,0.535941}%
\pgfsetfillcolor{currentfill}%
\pgfsetfillopacity{0.700000}%
\pgfsetlinewidth{0.501875pt}%
\definecolor{currentstroke}{rgb}{1.000000,1.000000,1.000000}%
\pgfsetstrokecolor{currentstroke}%
\pgfsetstrokeopacity{0.700000}%
\pgfsetdash{}{0pt}%
\pgfpathmoveto{\pgfqpoint{1.551002in}{4.064244in}}%
\pgfpathcurveto{\pgfqpoint{1.564024in}{4.064244in}}{\pgfqpoint{1.576516in}{4.069418in}}{\pgfqpoint{1.585724in}{4.078626in}}%
\pgfpathcurveto{\pgfqpoint{1.594932in}{4.087835in}}{\pgfqpoint{1.600106in}{4.100326in}}{\pgfqpoint{1.600106in}{4.113349in}}%
\pgfpathcurveto{\pgfqpoint{1.600106in}{4.126371in}}{\pgfqpoint{1.594932in}{4.138862in}}{\pgfqpoint{1.585724in}{4.148071in}}%
\pgfpathcurveto{\pgfqpoint{1.576516in}{4.157279in}}{\pgfqpoint{1.564024in}{4.162453in}}{\pgfqpoint{1.551002in}{4.162453in}}%
\pgfpathcurveto{\pgfqpoint{1.537979in}{4.162453in}}{\pgfqpoint{1.525488in}{4.157279in}}{\pgfqpoint{1.516280in}{4.148071in}}%
\pgfpathcurveto{\pgfqpoint{1.507071in}{4.138862in}}{\pgfqpoint{1.501897in}{4.126371in}}{\pgfqpoint{1.501897in}{4.113349in}}%
\pgfpathcurveto{\pgfqpoint{1.501897in}{4.100326in}}{\pgfqpoint{1.507071in}{4.087835in}}{\pgfqpoint{1.516280in}{4.078626in}}%
\pgfpathcurveto{\pgfqpoint{1.525488in}{4.069418in}}{\pgfqpoint{1.537979in}{4.064244in}}{\pgfqpoint{1.551002in}{4.064244in}}%
\pgfpathlineto{\pgfqpoint{1.551002in}{4.064244in}}%
\pgfpathclose%
\pgfusepath{stroke,fill}%
\end{pgfscope}%
\begin{pgfscope}%
\pgfpathrectangle{\pgfqpoint{0.786164in}{0.768110in}}{\pgfqpoint{8.851069in}{7.081890in}}%
\pgfusepath{clip}%
\pgfsetbuttcap%
\pgfsetroundjoin%
\definecolor{currentfill}{rgb}{0.246811,0.283237,0.535941}%
\pgfsetfillcolor{currentfill}%
\pgfsetfillopacity{0.700000}%
\pgfsetlinewidth{0.501875pt}%
\definecolor{currentstroke}{rgb}{1.000000,1.000000,1.000000}%
\pgfsetstrokecolor{currentstroke}%
\pgfsetstrokeopacity{0.700000}%
\pgfsetdash{}{0pt}%
\pgfpathmoveto{\pgfqpoint{1.558328in}{4.021560in}}%
\pgfpathcurveto{\pgfqpoint{1.571350in}{4.021560in}}{\pgfqpoint{1.583842in}{4.026734in}}{\pgfqpoint{1.593050in}{4.035942in}}%
\pgfpathcurveto{\pgfqpoint{1.602258in}{4.045151in}}{\pgfqpoint{1.607432in}{4.057642in}}{\pgfqpoint{1.607432in}{4.070664in}}%
\pgfpathcurveto{\pgfqpoint{1.607432in}{4.083687in}}{\pgfqpoint{1.602258in}{4.096178in}}{\pgfqpoint{1.593050in}{4.105387in}}%
\pgfpathcurveto{\pgfqpoint{1.583842in}{4.114595in}}{\pgfqpoint{1.571350in}{4.119769in}}{\pgfqpoint{1.558328in}{4.119769in}}%
\pgfpathcurveto{\pgfqpoint{1.545305in}{4.119769in}}{\pgfqpoint{1.532814in}{4.114595in}}{\pgfqpoint{1.523606in}{4.105387in}}%
\pgfpathcurveto{\pgfqpoint{1.514397in}{4.096178in}}{\pgfqpoint{1.509223in}{4.083687in}}{\pgfqpoint{1.509223in}{4.070664in}}%
\pgfpathcurveto{\pgfqpoint{1.509223in}{4.057642in}}{\pgfqpoint{1.514397in}{4.045151in}}{\pgfqpoint{1.523606in}{4.035942in}}%
\pgfpathcurveto{\pgfqpoint{1.532814in}{4.026734in}}{\pgfqpoint{1.545305in}{4.021560in}}{\pgfqpoint{1.558328in}{4.021560in}}%
\pgfpathlineto{\pgfqpoint{1.558328in}{4.021560in}}%
\pgfpathclose%
\pgfusepath{stroke,fill}%
\end{pgfscope}%
\begin{pgfscope}%
\pgfpathrectangle{\pgfqpoint{0.786164in}{0.768110in}}{\pgfqpoint{8.851069in}{7.081890in}}%
\pgfusepath{clip}%
\pgfsetbuttcap%
\pgfsetroundjoin%
\definecolor{currentfill}{rgb}{0.243113,0.292092,0.538516}%
\pgfsetfillcolor{currentfill}%
\pgfsetfillopacity{0.700000}%
\pgfsetlinewidth{0.501875pt}%
\definecolor{currentstroke}{rgb}{1.000000,1.000000,1.000000}%
\pgfsetstrokecolor{currentstroke}%
\pgfsetstrokeopacity{0.700000}%
\pgfsetdash{}{0pt}%
\pgfpathmoveto{\pgfqpoint{1.574445in}{4.064244in}}%
\pgfpathcurveto{\pgfqpoint{1.587468in}{4.064244in}}{\pgfqpoint{1.599959in}{4.069418in}}{\pgfqpoint{1.609167in}{4.078626in}}%
\pgfpathcurveto{\pgfqpoint{1.618376in}{4.087835in}}{\pgfqpoint{1.623550in}{4.100326in}}{\pgfqpoint{1.623550in}{4.113349in}}%
\pgfpathcurveto{\pgfqpoint{1.623550in}{4.126371in}}{\pgfqpoint{1.618376in}{4.138862in}}{\pgfqpoint{1.609167in}{4.148071in}}%
\pgfpathcurveto{\pgfqpoint{1.599959in}{4.157279in}}{\pgfqpoint{1.587468in}{4.162453in}}{\pgfqpoint{1.574445in}{4.162453in}}%
\pgfpathcurveto{\pgfqpoint{1.561422in}{4.162453in}}{\pgfqpoint{1.548931in}{4.157279in}}{\pgfqpoint{1.539723in}{4.148071in}}%
\pgfpathcurveto{\pgfqpoint{1.530514in}{4.138862in}}{\pgfqpoint{1.525340in}{4.126371in}}{\pgfqpoint{1.525340in}{4.113349in}}%
\pgfpathcurveto{\pgfqpoint{1.525340in}{4.100326in}}{\pgfqpoint{1.530514in}{4.087835in}}{\pgfqpoint{1.539723in}{4.078626in}}%
\pgfpathcurveto{\pgfqpoint{1.548931in}{4.069418in}}{\pgfqpoint{1.561422in}{4.064244in}}{\pgfqpoint{1.574445in}{4.064244in}}%
\pgfpathlineto{\pgfqpoint{1.574445in}{4.064244in}}%
\pgfpathclose%
\pgfusepath{stroke,fill}%
\end{pgfscope}%
\begin{pgfscope}%
\pgfpathrectangle{\pgfqpoint{0.786164in}{0.768110in}}{\pgfqpoint{8.851069in}{7.081890in}}%
\pgfusepath{clip}%
\pgfsetbuttcap%
\pgfsetroundjoin%
\definecolor{currentfill}{rgb}{0.237441,0.305202,0.541921}%
\pgfsetfillcolor{currentfill}%
\pgfsetfillopacity{0.700000}%
\pgfsetlinewidth{0.501875pt}%
\definecolor{currentstroke}{rgb}{1.000000,1.000000,1.000000}%
\pgfsetstrokecolor{currentstroke}%
\pgfsetstrokeopacity{0.700000}%
\pgfsetdash{}{0pt}%
\pgfpathmoveto{\pgfqpoint{1.664555in}{4.106928in}}%
\pgfpathcurveto{\pgfqpoint{1.677578in}{4.106928in}}{\pgfqpoint{1.690069in}{4.112102in}}{\pgfqpoint{1.699277in}{4.121311in}}%
\pgfpathcurveto{\pgfqpoint{1.708486in}{4.130519in}}{\pgfqpoint{1.713660in}{4.143010in}}{\pgfqpoint{1.713660in}{4.156033in}}%
\pgfpathcurveto{\pgfqpoint{1.713660in}{4.169056in}}{\pgfqpoint{1.708486in}{4.181547in}}{\pgfqpoint{1.699277in}{4.190755in}}%
\pgfpathcurveto{\pgfqpoint{1.690069in}{4.199964in}}{\pgfqpoint{1.677578in}{4.205138in}}{\pgfqpoint{1.664555in}{4.205138in}}%
\pgfpathcurveto{\pgfqpoint{1.651533in}{4.205138in}}{\pgfqpoint{1.639041in}{4.199964in}}{\pgfqpoint{1.629833in}{4.190755in}}%
\pgfpathcurveto{\pgfqpoint{1.620625in}{4.181547in}}{\pgfqpoint{1.615451in}{4.169056in}}{\pgfqpoint{1.615451in}{4.156033in}}%
\pgfpathcurveto{\pgfqpoint{1.615451in}{4.143010in}}{\pgfqpoint{1.620625in}{4.130519in}}{\pgfqpoint{1.629833in}{4.121311in}}%
\pgfpathcurveto{\pgfqpoint{1.639041in}{4.112102in}}{\pgfqpoint{1.651533in}{4.106928in}}{\pgfqpoint{1.664555in}{4.106928in}}%
\pgfpathlineto{\pgfqpoint{1.664555in}{4.106928in}}%
\pgfpathclose%
\pgfusepath{stroke,fill}%
\end{pgfscope}%
\begin{pgfscope}%
\pgfpathrectangle{\pgfqpoint{0.786164in}{0.768110in}}{\pgfqpoint{8.851069in}{7.081890in}}%
\pgfusepath{clip}%
\pgfsetbuttcap%
\pgfsetroundjoin%
\definecolor{currentfill}{rgb}{0.231674,0.318106,0.544834}%
\pgfsetfillcolor{currentfill}%
\pgfsetfillopacity{0.700000}%
\pgfsetlinewidth{0.501875pt}%
\definecolor{currentstroke}{rgb}{1.000000,1.000000,1.000000}%
\pgfsetstrokecolor{currentstroke}%
\pgfsetstrokeopacity{0.700000}%
\pgfsetdash{}{0pt}%
\pgfpathmoveto{\pgfqpoint{1.802529in}{4.064244in}}%
\pgfpathcurveto{\pgfqpoint{1.815552in}{4.064244in}}{\pgfqpoint{1.828043in}{4.069418in}}{\pgfqpoint{1.837251in}{4.078626in}}%
\pgfpathcurveto{\pgfqpoint{1.846460in}{4.087835in}}{\pgfqpoint{1.851634in}{4.100326in}}{\pgfqpoint{1.851634in}{4.113349in}}%
\pgfpathcurveto{\pgfqpoint{1.851634in}{4.126371in}}{\pgfqpoint{1.846460in}{4.138862in}}{\pgfqpoint{1.837251in}{4.148071in}}%
\pgfpathcurveto{\pgfqpoint{1.828043in}{4.157279in}}{\pgfqpoint{1.815552in}{4.162453in}}{\pgfqpoint{1.802529in}{4.162453in}}%
\pgfpathcurveto{\pgfqpoint{1.789506in}{4.162453in}}{\pgfqpoint{1.777015in}{4.157279in}}{\pgfqpoint{1.767807in}{4.148071in}}%
\pgfpathcurveto{\pgfqpoint{1.758598in}{4.138862in}}{\pgfqpoint{1.753424in}{4.126371in}}{\pgfqpoint{1.753424in}{4.113349in}}%
\pgfpathcurveto{\pgfqpoint{1.753424in}{4.100326in}}{\pgfqpoint{1.758598in}{4.087835in}}{\pgfqpoint{1.767807in}{4.078626in}}%
\pgfpathcurveto{\pgfqpoint{1.777015in}{4.069418in}}{\pgfqpoint{1.789506in}{4.064244in}}{\pgfqpoint{1.802529in}{4.064244in}}%
\pgfpathlineto{\pgfqpoint{1.802529in}{4.064244in}}%
\pgfpathclose%
\pgfusepath{stroke,fill}%
\end{pgfscope}%
\begin{pgfscope}%
\pgfpathrectangle{\pgfqpoint{0.786164in}{0.768110in}}{\pgfqpoint{8.851069in}{7.081890in}}%
\pgfusepath{clip}%
\pgfsetbuttcap%
\pgfsetroundjoin%
\definecolor{currentfill}{rgb}{0.243113,0.292092,0.538516}%
\pgfsetfillcolor{currentfill}%
\pgfsetfillopacity{0.700000}%
\pgfsetlinewidth{0.501875pt}%
\definecolor{currentstroke}{rgb}{1.000000,1.000000,1.000000}%
\pgfsetstrokecolor{currentstroke}%
\pgfsetstrokeopacity{0.700000}%
\pgfsetdash{}{0pt}%
\pgfpathmoveto{\pgfqpoint{1.898866in}{3.914849in}}%
\pgfpathcurveto{\pgfqpoint{1.911889in}{3.914849in}}{\pgfqpoint{1.924380in}{3.920023in}}{\pgfqpoint{1.933588in}{3.929231in}}%
\pgfpathcurveto{\pgfqpoint{1.942797in}{3.938440in}}{\pgfqpoint{1.947971in}{3.950931in}}{\pgfqpoint{1.947971in}{3.963953in}}%
\pgfpathcurveto{\pgfqpoint{1.947971in}{3.976976in}}{\pgfqpoint{1.942797in}{3.989467in}}{\pgfqpoint{1.933588in}{3.998676in}}%
\pgfpathcurveto{\pgfqpoint{1.924380in}{4.007884in}}{\pgfqpoint{1.911889in}{4.013058in}}{\pgfqpoint{1.898866in}{4.013058in}}%
\pgfpathcurveto{\pgfqpoint{1.885844in}{4.013058in}}{\pgfqpoint{1.873352in}{4.007884in}}{\pgfqpoint{1.864144in}{3.998676in}}%
\pgfpathcurveto{\pgfqpoint{1.854936in}{3.989467in}}{\pgfqpoint{1.849762in}{3.976976in}}{\pgfqpoint{1.849762in}{3.963953in}}%
\pgfpathcurveto{\pgfqpoint{1.849762in}{3.950931in}}{\pgfqpoint{1.854936in}{3.938440in}}{\pgfqpoint{1.864144in}{3.929231in}}%
\pgfpathcurveto{\pgfqpoint{1.873352in}{3.920023in}}{\pgfqpoint{1.885844in}{3.914849in}}{\pgfqpoint{1.898866in}{3.914849in}}%
\pgfpathlineto{\pgfqpoint{1.898866in}{3.914849in}}%
\pgfpathclose%
\pgfusepath{stroke,fill}%
\end{pgfscope}%
\begin{pgfscope}%
\pgfpathrectangle{\pgfqpoint{0.786164in}{0.768110in}}{\pgfqpoint{8.851069in}{7.081890in}}%
\pgfusepath{clip}%
\pgfsetbuttcap%
\pgfsetroundjoin%
\definecolor{currentfill}{rgb}{0.253935,0.265254,0.529983}%
\pgfsetfillcolor{currentfill}%
\pgfsetfillopacity{0.700000}%
\pgfsetlinewidth{0.501875pt}%
\definecolor{currentstroke}{rgb}{1.000000,1.000000,1.000000}%
\pgfsetstrokecolor{currentstroke}%
\pgfsetstrokeopacity{0.700000}%
\pgfsetdash{}{0pt}%
\pgfpathmoveto{\pgfqpoint{1.928292in}{3.765454in}}%
\pgfpathcurveto{\pgfqpoint{1.941315in}{3.765454in}}{\pgfqpoint{1.953806in}{3.770628in}}{\pgfqpoint{1.963015in}{3.779836in}}%
\pgfpathcurveto{\pgfqpoint{1.972223in}{3.789044in}}{\pgfqpoint{1.977397in}{3.801536in}}{\pgfqpoint{1.977397in}{3.814558in}}%
\pgfpathcurveto{\pgfqpoint{1.977397in}{3.827581in}}{\pgfqpoint{1.972223in}{3.840072in}}{\pgfqpoint{1.963015in}{3.849280in}}%
\pgfpathcurveto{\pgfqpoint{1.953806in}{3.858489in}}{\pgfqpoint{1.941315in}{3.863663in}}{\pgfqpoint{1.928292in}{3.863663in}}%
\pgfpathcurveto{\pgfqpoint{1.915270in}{3.863663in}}{\pgfqpoint{1.902779in}{3.858489in}}{\pgfqpoint{1.893570in}{3.849280in}}%
\pgfpathcurveto{\pgfqpoint{1.884362in}{3.840072in}}{\pgfqpoint{1.879188in}{3.827581in}}{\pgfqpoint{1.879188in}{3.814558in}}%
\pgfpathcurveto{\pgfqpoint{1.879188in}{3.801536in}}{\pgfqpoint{1.884362in}{3.789044in}}{\pgfqpoint{1.893570in}{3.779836in}}%
\pgfpathcurveto{\pgfqpoint{1.902779in}{3.770628in}}{\pgfqpoint{1.915270in}{3.765454in}}{\pgfqpoint{1.928292in}{3.765454in}}%
\pgfpathlineto{\pgfqpoint{1.928292in}{3.765454in}}%
\pgfpathclose%
\pgfusepath{stroke,fill}%
\end{pgfscope}%
\begin{pgfscope}%
\pgfpathrectangle{\pgfqpoint{0.786164in}{0.768110in}}{\pgfqpoint{8.851069in}{7.081890in}}%
\pgfusepath{clip}%
\pgfsetbuttcap%
\pgfsetroundjoin%
\definecolor{currentfill}{rgb}{0.258965,0.251537,0.524736}%
\pgfsetfillcolor{currentfill}%
\pgfsetfillopacity{0.700000}%
\pgfsetlinewidth{0.501875pt}%
\definecolor{currentstroke}{rgb}{1.000000,1.000000,1.000000}%
\pgfsetstrokecolor{currentstroke}%
\pgfsetstrokeopacity{0.700000}%
\pgfsetdash{}{0pt}%
\pgfpathmoveto{\pgfqpoint{1.938549in}{3.616058in}}%
\pgfpathcurveto{\pgfqpoint{1.951572in}{3.616058in}}{\pgfqpoint{1.964063in}{3.621232in}}{\pgfqpoint{1.973271in}{3.630441in}}%
\pgfpathcurveto{\pgfqpoint{1.982480in}{3.639649in}}{\pgfqpoint{1.987654in}{3.652140in}}{\pgfqpoint{1.987654in}{3.665163in}}%
\pgfpathcurveto{\pgfqpoint{1.987654in}{3.678186in}}{\pgfqpoint{1.982480in}{3.690677in}}{\pgfqpoint{1.973271in}{3.699885in}}%
\pgfpathcurveto{\pgfqpoint{1.964063in}{3.709094in}}{\pgfqpoint{1.951572in}{3.714268in}}{\pgfqpoint{1.938549in}{3.714268in}}%
\pgfpathcurveto{\pgfqpoint{1.925526in}{3.714268in}}{\pgfqpoint{1.913035in}{3.709094in}}{\pgfqpoint{1.903827in}{3.699885in}}%
\pgfpathcurveto{\pgfqpoint{1.894618in}{3.690677in}}{\pgfqpoint{1.889444in}{3.678186in}}{\pgfqpoint{1.889444in}{3.665163in}}%
\pgfpathcurveto{\pgfqpoint{1.889444in}{3.652140in}}{\pgfqpoint{1.894618in}{3.639649in}}{\pgfqpoint{1.903827in}{3.630441in}}%
\pgfpathcurveto{\pgfqpoint{1.913035in}{3.621232in}}{\pgfqpoint{1.925526in}{3.616058in}}{\pgfqpoint{1.938549in}{3.616058in}}%
\pgfpathlineto{\pgfqpoint{1.938549in}{3.616058in}}%
\pgfpathclose%
\pgfusepath{stroke,fill}%
\end{pgfscope}%
\begin{pgfscope}%
\pgfpathrectangle{\pgfqpoint{0.786164in}{0.768110in}}{\pgfqpoint{8.851069in}{7.081890in}}%
\pgfusepath{clip}%
\pgfsetbuttcap%
\pgfsetroundjoin%
\definecolor{currentfill}{rgb}{0.267968,0.223549,0.512008}%
\pgfsetfillcolor{currentfill}%
\pgfsetfillopacity{0.700000}%
\pgfsetlinewidth{0.501875pt}%
\definecolor{currentstroke}{rgb}{1.000000,1.000000,1.000000}%
\pgfsetstrokecolor{currentstroke}%
\pgfsetstrokeopacity{0.700000}%
\pgfsetdash{}{0pt}%
\pgfpathmoveto{\pgfqpoint{2.044288in}{3.445321in}}%
\pgfpathcurveto{\pgfqpoint{2.057311in}{3.445321in}}{\pgfqpoint{2.069802in}{3.450495in}}{\pgfqpoint{2.079010in}{3.459703in}}%
\pgfpathcurveto{\pgfqpoint{2.088219in}{3.468912in}}{\pgfqpoint{2.093393in}{3.481403in}}{\pgfqpoint{2.093393in}{3.494426in}}%
\pgfpathcurveto{\pgfqpoint{2.093393in}{3.507448in}}{\pgfqpoint{2.088219in}{3.519939in}}{\pgfqpoint{2.079010in}{3.529148in}}%
\pgfpathcurveto{\pgfqpoint{2.069802in}{3.538356in}}{\pgfqpoint{2.057311in}{3.543530in}}{\pgfqpoint{2.044288in}{3.543530in}}%
\pgfpathcurveto{\pgfqpoint{2.031265in}{3.543530in}}{\pgfqpoint{2.018774in}{3.538356in}}{\pgfqpoint{2.009566in}{3.529148in}}%
\pgfpathcurveto{\pgfqpoint{2.000357in}{3.519939in}}{\pgfqpoint{1.995183in}{3.507448in}}{\pgfqpoint{1.995183in}{3.494426in}}%
\pgfpathcurveto{\pgfqpoint{1.995183in}{3.481403in}}{\pgfqpoint{2.000357in}{3.468912in}}{\pgfqpoint{2.009566in}{3.459703in}}%
\pgfpathcurveto{\pgfqpoint{2.018774in}{3.450495in}}{\pgfqpoint{2.031265in}{3.445321in}}{\pgfqpoint{2.044288in}{3.445321in}}%
\pgfpathlineto{\pgfqpoint{2.044288in}{3.445321in}}%
\pgfpathclose%
\pgfusepath{stroke,fill}%
\end{pgfscope}%
\begin{pgfscope}%
\pgfpathrectangle{\pgfqpoint{0.786164in}{0.768110in}}{\pgfqpoint{8.851069in}{7.081890in}}%
\pgfusepath{clip}%
\pgfsetbuttcap%
\pgfsetroundjoin%
\definecolor{currentfill}{rgb}{0.279574,0.170599,0.479997}%
\pgfsetfillcolor{currentfill}%
\pgfsetfillopacity{0.700000}%
\pgfsetlinewidth{0.501875pt}%
\definecolor{currentstroke}{rgb}{1.000000,1.000000,1.000000}%
\pgfsetstrokecolor{currentstroke}%
\pgfsetstrokeopacity{0.700000}%
\pgfsetdash{}{0pt}%
\pgfpathmoveto{\pgfqpoint{2.205094in}{3.274584in}}%
\pgfpathcurveto{\pgfqpoint{2.218117in}{3.274584in}}{\pgfqpoint{2.230608in}{3.279758in}}{\pgfqpoint{2.239817in}{3.288966in}}%
\pgfpathcurveto{\pgfqpoint{2.249025in}{3.298175in}}{\pgfqpoint{2.254199in}{3.310666in}}{\pgfqpoint{2.254199in}{3.323688in}}%
\pgfpathcurveto{\pgfqpoint{2.254199in}{3.336711in}}{\pgfqpoint{2.249025in}{3.349202in}}{\pgfqpoint{2.239817in}{3.358411in}}%
\pgfpathcurveto{\pgfqpoint{2.230608in}{3.367619in}}{\pgfqpoint{2.218117in}{3.372793in}}{\pgfqpoint{2.205094in}{3.372793in}}%
\pgfpathcurveto{\pgfqpoint{2.192072in}{3.372793in}}{\pgfqpoint{2.179581in}{3.367619in}}{\pgfqpoint{2.170372in}{3.358411in}}%
\pgfpathcurveto{\pgfqpoint{2.161164in}{3.349202in}}{\pgfqpoint{2.155990in}{3.336711in}}{\pgfqpoint{2.155990in}{3.323688in}}%
\pgfpathcurveto{\pgfqpoint{2.155990in}{3.310666in}}{\pgfqpoint{2.161164in}{3.298175in}}{\pgfqpoint{2.170372in}{3.288966in}}%
\pgfpathcurveto{\pgfqpoint{2.179581in}{3.279758in}}{\pgfqpoint{2.192072in}{3.274584in}}{\pgfqpoint{2.205094in}{3.274584in}}%
\pgfpathlineto{\pgfqpoint{2.205094in}{3.274584in}}%
\pgfpathclose%
\pgfusepath{stroke,fill}%
\end{pgfscope}%
\begin{pgfscope}%
\pgfpathrectangle{\pgfqpoint{0.786164in}{0.768110in}}{\pgfqpoint{8.851069in}{7.081890in}}%
\pgfusepath{clip}%
\pgfsetbuttcap%
\pgfsetroundjoin%
\definecolor{currentfill}{rgb}{0.278012,0.180367,0.486697}%
\pgfsetfillcolor{currentfill}%
\pgfsetfillopacity{0.700000}%
\pgfsetlinewidth{0.501875pt}%
\definecolor{currentstroke}{rgb}{1.000000,1.000000,1.000000}%
\pgfsetstrokecolor{currentstroke}%
\pgfsetstrokeopacity{0.700000}%
\pgfsetdash{}{0pt}%
\pgfpathmoveto{\pgfqpoint{2.292518in}{3.381295in}}%
\pgfpathcurveto{\pgfqpoint{2.305541in}{3.381295in}}{\pgfqpoint{2.318032in}{3.386469in}}{\pgfqpoint{2.327241in}{3.395677in}}%
\pgfpathcurveto{\pgfqpoint{2.336449in}{3.404885in}}{\pgfqpoint{2.341623in}{3.417376in}}{\pgfqpoint{2.341623in}{3.430399in}}%
\pgfpathcurveto{\pgfqpoint{2.341623in}{3.443422in}}{\pgfqpoint{2.336449in}{3.455913in}}{\pgfqpoint{2.327241in}{3.465121in}}%
\pgfpathcurveto{\pgfqpoint{2.318032in}{3.474330in}}{\pgfqpoint{2.305541in}{3.479504in}}{\pgfqpoint{2.292518in}{3.479504in}}%
\pgfpathcurveto{\pgfqpoint{2.279496in}{3.479504in}}{\pgfqpoint{2.267005in}{3.474330in}}{\pgfqpoint{2.257796in}{3.465121in}}%
\pgfpathcurveto{\pgfqpoint{2.248588in}{3.455913in}}{\pgfqpoint{2.243414in}{3.443422in}}{\pgfqpoint{2.243414in}{3.430399in}}%
\pgfpathcurveto{\pgfqpoint{2.243414in}{3.417376in}}{\pgfqpoint{2.248588in}{3.404885in}}{\pgfqpoint{2.257796in}{3.395677in}}%
\pgfpathcurveto{\pgfqpoint{2.267005in}{3.386469in}}{\pgfqpoint{2.279496in}{3.381295in}}{\pgfqpoint{2.292518in}{3.381295in}}%
\pgfpathlineto{\pgfqpoint{2.292518in}{3.381295in}}%
\pgfpathclose%
\pgfusepath{stroke,fill}%
\end{pgfscope}%
\begin{pgfscope}%
\pgfpathrectangle{\pgfqpoint{0.786164in}{0.768110in}}{\pgfqpoint{8.851069in}{7.081890in}}%
\pgfusepath{clip}%
\pgfsetbuttcap%
\pgfsetroundjoin%
\definecolor{currentfill}{rgb}{0.275191,0.194905,0.496005}%
\pgfsetfillcolor{currentfill}%
\pgfsetfillopacity{0.700000}%
\pgfsetlinewidth{0.501875pt}%
\definecolor{currentstroke}{rgb}{1.000000,1.000000,1.000000}%
\pgfsetstrokecolor{currentstroke}%
\pgfsetstrokeopacity{0.700000}%
\pgfsetdash{}{0pt}%
\pgfpathmoveto{\pgfqpoint{2.385193in}{3.402637in}}%
\pgfpathcurveto{\pgfqpoint{2.398215in}{3.402637in}}{\pgfqpoint{2.410706in}{3.407811in}}{\pgfqpoint{2.419915in}{3.417019in}}%
\pgfpathcurveto{\pgfqpoint{2.429123in}{3.426228in}}{\pgfqpoint{2.434297in}{3.438719in}}{\pgfqpoint{2.434297in}{3.451741in}}%
\pgfpathcurveto{\pgfqpoint{2.434297in}{3.464764in}}{\pgfqpoint{2.429123in}{3.477255in}}{\pgfqpoint{2.419915in}{3.486464in}}%
\pgfpathcurveto{\pgfqpoint{2.410706in}{3.495672in}}{\pgfqpoint{2.398215in}{3.500846in}}{\pgfqpoint{2.385193in}{3.500846in}}%
\pgfpathcurveto{\pgfqpoint{2.372170in}{3.500846in}}{\pgfqpoint{2.359679in}{3.495672in}}{\pgfqpoint{2.350470in}{3.486464in}}%
\pgfpathcurveto{\pgfqpoint{2.341262in}{3.477255in}}{\pgfqpoint{2.336088in}{3.464764in}}{\pgfqpoint{2.336088in}{3.451741in}}%
\pgfpathcurveto{\pgfqpoint{2.336088in}{3.438719in}}{\pgfqpoint{2.341262in}{3.426228in}}{\pgfqpoint{2.350470in}{3.417019in}}%
\pgfpathcurveto{\pgfqpoint{2.359679in}{3.407811in}}{\pgfqpoint{2.372170in}{3.402637in}}{\pgfqpoint{2.385193in}{3.402637in}}%
\pgfpathlineto{\pgfqpoint{2.385193in}{3.402637in}}%
\pgfpathclose%
\pgfusepath{stroke,fill}%
\end{pgfscope}%
\begin{pgfscope}%
\pgfpathrectangle{\pgfqpoint{0.786164in}{0.768110in}}{\pgfqpoint{8.851069in}{7.081890in}}%
\pgfusepath{clip}%
\pgfsetbuttcap%
\pgfsetroundjoin%
\definecolor{currentfill}{rgb}{0.270595,0.214069,0.507052}%
\pgfsetfillcolor{currentfill}%
\pgfsetfillopacity{0.700000}%
\pgfsetlinewidth{0.501875pt}%
\definecolor{currentstroke}{rgb}{1.000000,1.000000,1.000000}%
\pgfsetstrokecolor{currentstroke}%
\pgfsetstrokeopacity{0.700000}%
\pgfsetdash{}{0pt}%
\pgfpathmoveto{\pgfqpoint{2.376646in}{3.552032in}}%
\pgfpathcurveto{\pgfqpoint{2.389668in}{3.552032in}}{\pgfqpoint{2.402159in}{3.557206in}}{\pgfqpoint{2.411368in}{3.566414in}}%
\pgfpathcurveto{\pgfqpoint{2.420576in}{3.575623in}}{\pgfqpoint{2.425750in}{3.588114in}}{\pgfqpoint{2.425750in}{3.601137in}}%
\pgfpathcurveto{\pgfqpoint{2.425750in}{3.614159in}}{\pgfqpoint{2.420576in}{3.626650in}}{\pgfqpoint{2.411368in}{3.635859in}}%
\pgfpathcurveto{\pgfqpoint{2.402159in}{3.645067in}}{\pgfqpoint{2.389668in}{3.650241in}}{\pgfqpoint{2.376646in}{3.650241in}}%
\pgfpathcurveto{\pgfqpoint{2.363623in}{3.650241in}}{\pgfqpoint{2.351132in}{3.645067in}}{\pgfqpoint{2.341923in}{3.635859in}}%
\pgfpathcurveto{\pgfqpoint{2.332715in}{3.626650in}}{\pgfqpoint{2.327541in}{3.614159in}}{\pgfqpoint{2.327541in}{3.601137in}}%
\pgfpathcurveto{\pgfqpoint{2.327541in}{3.588114in}}{\pgfqpoint{2.332715in}{3.575623in}}{\pgfqpoint{2.341923in}{3.566414in}}%
\pgfpathcurveto{\pgfqpoint{2.351132in}{3.557206in}}{\pgfqpoint{2.363623in}{3.552032in}}{\pgfqpoint{2.376646in}{3.552032in}}%
\pgfpathlineto{\pgfqpoint{2.376646in}{3.552032in}}%
\pgfpathclose%
\pgfusepath{stroke,fill}%
\end{pgfscope}%
\begin{pgfscope}%
\pgfpathrectangle{\pgfqpoint{0.786164in}{0.768110in}}{\pgfqpoint{8.851069in}{7.081890in}}%
\pgfusepath{clip}%
\pgfsetbuttcap%
\pgfsetroundjoin%
\definecolor{currentfill}{rgb}{0.266580,0.228262,0.514349}%
\pgfsetfillcolor{currentfill}%
\pgfsetfillopacity{0.700000}%
\pgfsetlinewidth{0.501875pt}%
\definecolor{currentstroke}{rgb}{1.000000,1.000000,1.000000}%
\pgfsetstrokecolor{currentstroke}%
\pgfsetstrokeopacity{0.700000}%
\pgfsetdash{}{0pt}%
\pgfpathmoveto{\pgfqpoint{2.455401in}{3.573374in}}%
\pgfpathcurveto{\pgfqpoint{2.468423in}{3.573374in}}{\pgfqpoint{2.480914in}{3.578548in}}{\pgfqpoint{2.490123in}{3.587756in}}%
\pgfpathcurveto{\pgfqpoint{2.499331in}{3.596965in}}{\pgfqpoint{2.504505in}{3.609456in}}{\pgfqpoint{2.504505in}{3.622479in}}%
\pgfpathcurveto{\pgfqpoint{2.504505in}{3.635501in}}{\pgfqpoint{2.499331in}{3.647992in}}{\pgfqpoint{2.490123in}{3.657201in}}%
\pgfpathcurveto{\pgfqpoint{2.480914in}{3.666409in}}{\pgfqpoint{2.468423in}{3.671583in}}{\pgfqpoint{2.455401in}{3.671583in}}%
\pgfpathcurveto{\pgfqpoint{2.442378in}{3.671583in}}{\pgfqpoint{2.429887in}{3.666409in}}{\pgfqpoint{2.420678in}{3.657201in}}%
\pgfpathcurveto{\pgfqpoint{2.411470in}{3.647992in}}{\pgfqpoint{2.406296in}{3.635501in}}{\pgfqpoint{2.406296in}{3.622479in}}%
\pgfpathcurveto{\pgfqpoint{2.406296in}{3.609456in}}{\pgfqpoint{2.411470in}{3.596965in}}{\pgfqpoint{2.420678in}{3.587756in}}%
\pgfpathcurveto{\pgfqpoint{2.429887in}{3.578548in}}{\pgfqpoint{2.442378in}{3.573374in}}{\pgfqpoint{2.455401in}{3.573374in}}%
\pgfpathlineto{\pgfqpoint{2.455401in}{3.573374in}}%
\pgfpathclose%
\pgfusepath{stroke,fill}%
\end{pgfscope}%
\begin{pgfscope}%
\pgfpathrectangle{\pgfqpoint{0.786164in}{0.768110in}}{\pgfqpoint{8.851069in}{7.081890in}}%
\pgfusepath{clip}%
\pgfsetbuttcap%
\pgfsetroundjoin%
\definecolor{currentfill}{rgb}{0.266580,0.228262,0.514349}%
\pgfsetfillcolor{currentfill}%
\pgfsetfillopacity{0.700000}%
\pgfsetlinewidth{0.501875pt}%
\definecolor{currentstroke}{rgb}{1.000000,1.000000,1.000000}%
\pgfsetstrokecolor{currentstroke}%
\pgfsetstrokeopacity{0.700000}%
\pgfsetdash{}{0pt}%
\pgfpathmoveto{\pgfqpoint{2.869932in}{3.530690in}}%
\pgfpathcurveto{\pgfqpoint{2.882955in}{3.530690in}}{\pgfqpoint{2.895446in}{3.535864in}}{\pgfqpoint{2.904654in}{3.545072in}}%
\pgfpathcurveto{\pgfqpoint{2.913863in}{3.554281in}}{\pgfqpoint{2.919037in}{3.566772in}}{\pgfqpoint{2.919037in}{3.579794in}}%
\pgfpathcurveto{\pgfqpoint{2.919037in}{3.592817in}}{\pgfqpoint{2.913863in}{3.605308in}}{\pgfqpoint{2.904654in}{3.614517in}}%
\pgfpathcurveto{\pgfqpoint{2.895446in}{3.623725in}}{\pgfqpoint{2.882955in}{3.628899in}}{\pgfqpoint{2.869932in}{3.628899in}}%
\pgfpathcurveto{\pgfqpoint{2.856909in}{3.628899in}}{\pgfqpoint{2.844418in}{3.623725in}}{\pgfqpoint{2.835210in}{3.614517in}}%
\pgfpathcurveto{\pgfqpoint{2.826001in}{3.605308in}}{\pgfqpoint{2.820827in}{3.592817in}}{\pgfqpoint{2.820827in}{3.579794in}}%
\pgfpathcurveto{\pgfqpoint{2.820827in}{3.566772in}}{\pgfqpoint{2.826001in}{3.554281in}}{\pgfqpoint{2.835210in}{3.545072in}}%
\pgfpathcurveto{\pgfqpoint{2.844418in}{3.535864in}}{\pgfqpoint{2.856909in}{3.530690in}}{\pgfqpoint{2.869932in}{3.530690in}}%
\pgfpathlineto{\pgfqpoint{2.869932in}{3.530690in}}%
\pgfpathclose%
\pgfusepath{stroke,fill}%
\end{pgfscope}%
\begin{pgfscope}%
\pgfpathrectangle{\pgfqpoint{0.786164in}{0.768110in}}{\pgfqpoint{8.851069in}{7.081890in}}%
\pgfusepath{clip}%
\pgfsetbuttcap%
\pgfsetroundjoin%
\definecolor{currentfill}{rgb}{0.269308,0.218818,0.509577}%
\pgfsetfillcolor{currentfill}%
\pgfsetfillopacity{0.700000}%
\pgfsetlinewidth{0.501875pt}%
\definecolor{currentstroke}{rgb}{1.000000,1.000000,1.000000}%
\pgfsetstrokecolor{currentstroke}%
\pgfsetstrokeopacity{0.700000}%
\pgfsetdash{}{0pt}%
\pgfpathmoveto{\pgfqpoint{2.858210in}{3.509348in}}%
\pgfpathcurveto{\pgfqpoint{2.871233in}{3.509348in}}{\pgfqpoint{2.883724in}{3.514522in}}{\pgfqpoint{2.892932in}{3.523730in}}%
\pgfpathcurveto{\pgfqpoint{2.902141in}{3.532938in}}{\pgfqpoint{2.907315in}{3.545429in}}{\pgfqpoint{2.907315in}{3.558452in}}%
\pgfpathcurveto{\pgfqpoint{2.907315in}{3.571475in}}{\pgfqpoint{2.902141in}{3.583966in}}{\pgfqpoint{2.892932in}{3.593174in}}%
\pgfpathcurveto{\pgfqpoint{2.883724in}{3.602383in}}{\pgfqpoint{2.871233in}{3.607557in}}{\pgfqpoint{2.858210in}{3.607557in}}%
\pgfpathcurveto{\pgfqpoint{2.845188in}{3.607557in}}{\pgfqpoint{2.832696in}{3.602383in}}{\pgfqpoint{2.823488in}{3.593174in}}%
\pgfpathcurveto{\pgfqpoint{2.814280in}{3.583966in}}{\pgfqpoint{2.809106in}{3.571475in}}{\pgfqpoint{2.809106in}{3.558452in}}%
\pgfpathcurveto{\pgfqpoint{2.809106in}{3.545429in}}{\pgfqpoint{2.814280in}{3.532938in}}{\pgfqpoint{2.823488in}{3.523730in}}%
\pgfpathcurveto{\pgfqpoint{2.832696in}{3.514522in}}{\pgfqpoint{2.845188in}{3.509348in}}{\pgfqpoint{2.858210in}{3.509348in}}%
\pgfpathlineto{\pgfqpoint{2.858210in}{3.509348in}}%
\pgfpathclose%
\pgfusepath{stroke,fill}%
\end{pgfscope}%
\begin{pgfscope}%
\pgfpathrectangle{\pgfqpoint{0.786164in}{0.768110in}}{\pgfqpoint{8.851069in}{7.081890in}}%
\pgfusepath{clip}%
\pgfsetbuttcap%
\pgfsetroundjoin%
\definecolor{currentfill}{rgb}{0.280868,0.160771,0.472899}%
\pgfsetfillcolor{currentfill}%
\pgfsetfillopacity{0.700000}%
\pgfsetlinewidth{0.501875pt}%
\definecolor{currentstroke}{rgb}{1.000000,1.000000,1.000000}%
\pgfsetstrokecolor{currentstroke}%
\pgfsetstrokeopacity{0.700000}%
\pgfsetdash{}{0pt}%
\pgfpathmoveto{\pgfqpoint{3.236966in}{3.231899in}}%
\pgfpathcurveto{\pgfqpoint{3.249989in}{3.231899in}}{\pgfqpoint{3.262480in}{3.237073in}}{\pgfqpoint{3.271688in}{3.246282in}}%
\pgfpathcurveto{\pgfqpoint{3.280897in}{3.255490in}}{\pgfqpoint{3.286071in}{3.267981in}}{\pgfqpoint{3.286071in}{3.281004in}}%
\pgfpathcurveto{\pgfqpoint{3.286071in}{3.294027in}}{\pgfqpoint{3.280897in}{3.306518in}}{\pgfqpoint{3.271688in}{3.315726in}}%
\pgfpathcurveto{\pgfqpoint{3.262480in}{3.324935in}}{\pgfqpoint{3.249989in}{3.330109in}}{\pgfqpoint{3.236966in}{3.330109in}}%
\pgfpathcurveto{\pgfqpoint{3.223943in}{3.330109in}}{\pgfqpoint{3.211452in}{3.324935in}}{\pgfqpoint{3.202244in}{3.315726in}}%
\pgfpathcurveto{\pgfqpoint{3.193035in}{3.306518in}}{\pgfqpoint{3.187861in}{3.294027in}}{\pgfqpoint{3.187861in}{3.281004in}}%
\pgfpathcurveto{\pgfqpoint{3.187861in}{3.267981in}}{\pgfqpoint{3.193035in}{3.255490in}}{\pgfqpoint{3.202244in}{3.246282in}}%
\pgfpathcurveto{\pgfqpoint{3.211452in}{3.237073in}}{\pgfqpoint{3.223943in}{3.231899in}}{\pgfqpoint{3.236966in}{3.231899in}}%
\pgfpathlineto{\pgfqpoint{3.236966in}{3.231899in}}%
\pgfpathclose%
\pgfusepath{stroke,fill}%
\end{pgfscope}%
\begin{pgfscope}%
\pgfpathrectangle{\pgfqpoint{0.786164in}{0.768110in}}{\pgfqpoint{8.851069in}{7.081890in}}%
\pgfusepath{clip}%
\pgfsetbuttcap%
\pgfsetroundjoin%
\definecolor{currentfill}{rgb}{0.278826,0.175490,0.483397}%
\pgfsetfillcolor{currentfill}%
\pgfsetfillopacity{0.700000}%
\pgfsetlinewidth{0.501875pt}%
\definecolor{currentstroke}{rgb}{1.000000,1.000000,1.000000}%
\pgfsetstrokecolor{currentstroke}%
\pgfsetstrokeopacity{0.700000}%
\pgfsetdash{}{0pt}%
\pgfpathmoveto{\pgfqpoint{3.504366in}{3.338610in}}%
\pgfpathcurveto{\pgfqpoint{3.517389in}{3.338610in}}{\pgfqpoint{3.529880in}{3.343784in}}{\pgfqpoint{3.539089in}{3.352993in}}%
\pgfpathcurveto{\pgfqpoint{3.548297in}{3.362201in}}{\pgfqpoint{3.553471in}{3.374692in}}{\pgfqpoint{3.553471in}{3.387715in}}%
\pgfpathcurveto{\pgfqpoint{3.553471in}{3.400738in}}{\pgfqpoint{3.548297in}{3.413229in}}{\pgfqpoint{3.539089in}{3.422437in}}%
\pgfpathcurveto{\pgfqpoint{3.529880in}{3.431645in}}{\pgfqpoint{3.517389in}{3.436819in}}{\pgfqpoint{3.504366in}{3.436819in}}%
\pgfpathcurveto{\pgfqpoint{3.491344in}{3.436819in}}{\pgfqpoint{3.478853in}{3.431645in}}{\pgfqpoint{3.469644in}{3.422437in}}%
\pgfpathcurveto{\pgfqpoint{3.460436in}{3.413229in}}{\pgfqpoint{3.455262in}{3.400738in}}{\pgfqpoint{3.455262in}{3.387715in}}%
\pgfpathcurveto{\pgfqpoint{3.455262in}{3.374692in}}{\pgfqpoint{3.460436in}{3.362201in}}{\pgfqpoint{3.469644in}{3.352993in}}%
\pgfpathcurveto{\pgfqpoint{3.478853in}{3.343784in}}{\pgfqpoint{3.491344in}{3.338610in}}{\pgfqpoint{3.504366in}{3.338610in}}%
\pgfpathlineto{\pgfqpoint{3.504366in}{3.338610in}}%
\pgfpathclose%
\pgfusepath{stroke,fill}%
\end{pgfscope}%
\begin{pgfscope}%
\pgfpathrectangle{\pgfqpoint{0.786164in}{0.768110in}}{\pgfqpoint{8.851069in}{7.081890in}}%
\pgfusepath{clip}%
\pgfsetbuttcap%
\pgfsetroundjoin%
\definecolor{currentfill}{rgb}{0.274128,0.199721,0.498911}%
\pgfsetfillcolor{currentfill}%
\pgfsetfillopacity{0.700000}%
\pgfsetlinewidth{0.501875pt}%
\definecolor{currentstroke}{rgb}{1.000000,1.000000,1.000000}%
\pgfsetstrokecolor{currentstroke}%
\pgfsetstrokeopacity{0.700000}%
\pgfsetdash{}{0pt}%
\pgfpathmoveto{\pgfqpoint{3.548323in}{3.359952in}}%
\pgfpathcurveto{\pgfqpoint{3.561345in}{3.359952in}}{\pgfqpoint{3.573836in}{3.365126in}}{\pgfqpoint{3.583045in}{3.374335in}}%
\pgfpathcurveto{\pgfqpoint{3.592253in}{3.383543in}}{\pgfqpoint{3.597427in}{3.396034in}}{\pgfqpoint{3.597427in}{3.409057in}}%
\pgfpathcurveto{\pgfqpoint{3.597427in}{3.422080in}}{\pgfqpoint{3.592253in}{3.434571in}}{\pgfqpoint{3.583045in}{3.443779in}}%
\pgfpathcurveto{\pgfqpoint{3.573836in}{3.452988in}}{\pgfqpoint{3.561345in}{3.458162in}}{\pgfqpoint{3.548323in}{3.458162in}}%
\pgfpathcurveto{\pgfqpoint{3.535300in}{3.458162in}}{\pgfqpoint{3.522809in}{3.452988in}}{\pgfqpoint{3.513600in}{3.443779in}}%
\pgfpathcurveto{\pgfqpoint{3.504392in}{3.434571in}}{\pgfqpoint{3.499218in}{3.422080in}}{\pgfqpoint{3.499218in}{3.409057in}}%
\pgfpathcurveto{\pgfqpoint{3.499218in}{3.396034in}}{\pgfqpoint{3.504392in}{3.383543in}}{\pgfqpoint{3.513600in}{3.374335in}}%
\pgfpathcurveto{\pgfqpoint{3.522809in}{3.365126in}}{\pgfqpoint{3.535300in}{3.359952in}}{\pgfqpoint{3.548323in}{3.359952in}}%
\pgfpathlineto{\pgfqpoint{3.548323in}{3.359952in}}%
\pgfpathclose%
\pgfusepath{stroke,fill}%
\end{pgfscope}%
\begin{pgfscope}%
\pgfpathrectangle{\pgfqpoint{0.786164in}{0.768110in}}{\pgfqpoint{8.851069in}{7.081890in}}%
\pgfusepath{clip}%
\pgfsetbuttcap%
\pgfsetroundjoin%
\definecolor{currentfill}{rgb}{0.199430,0.387607,0.554642}%
\pgfsetfillcolor{currentfill}%
\pgfsetfillopacity{0.700000}%
\pgfsetlinewidth{0.501875pt}%
\definecolor{currentstroke}{rgb}{1.000000,1.000000,1.000000}%
\pgfsetstrokecolor{currentstroke}%
\pgfsetstrokeopacity{0.700000}%
\pgfsetdash{}{0pt}%
\pgfpathmoveto{\pgfqpoint{2.003018in}{4.213639in}}%
\pgfpathcurveto{\pgfqpoint{2.016041in}{4.213639in}}{\pgfqpoint{2.028532in}{4.218813in}}{\pgfqpoint{2.037740in}{4.228022in}}%
\pgfpathcurveto{\pgfqpoint{2.046949in}{4.237230in}}{\pgfqpoint{2.052123in}{4.249721in}}{\pgfqpoint{2.052123in}{4.262744in}}%
\pgfpathcurveto{\pgfqpoint{2.052123in}{4.275767in}}{\pgfqpoint{2.046949in}{4.288258in}}{\pgfqpoint{2.037740in}{4.297466in}}%
\pgfpathcurveto{\pgfqpoint{2.028532in}{4.306674in}}{\pgfqpoint{2.016041in}{4.311848in}}{\pgfqpoint{2.003018in}{4.311848in}}%
\pgfpathcurveto{\pgfqpoint{1.989995in}{4.311848in}}{\pgfqpoint{1.977504in}{4.306674in}}{\pgfqpoint{1.968296in}{4.297466in}}%
\pgfpathcurveto{\pgfqpoint{1.959087in}{4.288258in}}{\pgfqpoint{1.953913in}{4.275767in}}{\pgfqpoint{1.953913in}{4.262744in}}%
\pgfpathcurveto{\pgfqpoint{1.953913in}{4.249721in}}{\pgfqpoint{1.959087in}{4.237230in}}{\pgfqpoint{1.968296in}{4.228022in}}%
\pgfpathcurveto{\pgfqpoint{1.977504in}{4.218813in}}{\pgfqpoint{1.989995in}{4.213639in}}{\pgfqpoint{2.003018in}{4.213639in}}%
\pgfpathlineto{\pgfqpoint{2.003018in}{4.213639in}}%
\pgfpathclose%
\pgfusepath{stroke,fill}%
\end{pgfscope}%
\begin{pgfscope}%
\pgfpathrectangle{\pgfqpoint{0.786164in}{0.768110in}}{\pgfqpoint{8.851069in}{7.081890in}}%
\pgfusepath{clip}%
\pgfsetbuttcap%
\pgfsetroundjoin%
\definecolor{currentfill}{rgb}{0.201239,0.383670,0.554294}%
\pgfsetfillcolor{currentfill}%
\pgfsetfillopacity{0.700000}%
\pgfsetlinewidth{0.501875pt}%
\definecolor{currentstroke}{rgb}{1.000000,1.000000,1.000000}%
\pgfsetstrokecolor{currentstroke}%
\pgfsetstrokeopacity{0.700000}%
\pgfsetdash{}{0pt}%
\pgfpathmoveto{\pgfqpoint{2.044532in}{4.170955in}}%
\pgfpathcurveto{\pgfqpoint{2.057555in}{4.170955in}}{\pgfqpoint{2.070046in}{4.176129in}}{\pgfqpoint{2.079254in}{4.185337in}}%
\pgfpathcurveto{\pgfqpoint{2.088463in}{4.194546in}}{\pgfqpoint{2.093637in}{4.207037in}}{\pgfqpoint{2.093637in}{4.220059in}}%
\pgfpathcurveto{\pgfqpoint{2.093637in}{4.233082in}}{\pgfqpoint{2.088463in}{4.245573in}}{\pgfqpoint{2.079254in}{4.254782in}}%
\pgfpathcurveto{\pgfqpoint{2.070046in}{4.263990in}}{\pgfqpoint{2.057555in}{4.269164in}}{\pgfqpoint{2.044532in}{4.269164in}}%
\pgfpathcurveto{\pgfqpoint{2.031509in}{4.269164in}}{\pgfqpoint{2.019018in}{4.263990in}}{\pgfqpoint{2.009810in}{4.254782in}}%
\pgfpathcurveto{\pgfqpoint{2.000601in}{4.245573in}}{\pgfqpoint{1.995428in}{4.233082in}}{\pgfqpoint{1.995428in}{4.220059in}}%
\pgfpathcurveto{\pgfqpoint{1.995428in}{4.207037in}}{\pgfqpoint{2.000601in}{4.194546in}}{\pgfqpoint{2.009810in}{4.185337in}}%
\pgfpathcurveto{\pgfqpoint{2.019018in}{4.176129in}}{\pgfqpoint{2.031509in}{4.170955in}}{\pgfqpoint{2.044532in}{4.170955in}}%
\pgfpathlineto{\pgfqpoint{2.044532in}{4.170955in}}%
\pgfpathclose%
\pgfusepath{stroke,fill}%
\end{pgfscope}%
\begin{pgfscope}%
\pgfpathrectangle{\pgfqpoint{0.786164in}{0.768110in}}{\pgfqpoint{8.851069in}{7.081890in}}%
\pgfusepath{clip}%
\pgfsetbuttcap%
\pgfsetroundjoin%
\definecolor{currentfill}{rgb}{0.197636,0.391528,0.554969}%
\pgfsetfillcolor{currentfill}%
\pgfsetfillopacity{0.700000}%
\pgfsetlinewidth{0.501875pt}%
\definecolor{currentstroke}{rgb}{1.000000,1.000000,1.000000}%
\pgfsetstrokecolor{currentstroke}%
\pgfsetstrokeopacity{0.700000}%
\pgfsetdash{}{0pt}%
\pgfpathmoveto{\pgfqpoint{2.074935in}{4.213639in}}%
\pgfpathcurveto{\pgfqpoint{2.087958in}{4.213639in}}{\pgfqpoint{2.100449in}{4.218813in}}{\pgfqpoint{2.109657in}{4.228022in}}%
\pgfpathcurveto{\pgfqpoint{2.118866in}{4.237230in}}{\pgfqpoint{2.124040in}{4.249721in}}{\pgfqpoint{2.124040in}{4.262744in}}%
\pgfpathcurveto{\pgfqpoint{2.124040in}{4.275767in}}{\pgfqpoint{2.118866in}{4.288258in}}{\pgfqpoint{2.109657in}{4.297466in}}%
\pgfpathcurveto{\pgfqpoint{2.100449in}{4.306674in}}{\pgfqpoint{2.087958in}{4.311848in}}{\pgfqpoint{2.074935in}{4.311848in}}%
\pgfpathcurveto{\pgfqpoint{2.061912in}{4.311848in}}{\pgfqpoint{2.049421in}{4.306674in}}{\pgfqpoint{2.040213in}{4.297466in}}%
\pgfpathcurveto{\pgfqpoint{2.031005in}{4.288258in}}{\pgfqpoint{2.025831in}{4.275767in}}{\pgfqpoint{2.025831in}{4.262744in}}%
\pgfpathcurveto{\pgfqpoint{2.025831in}{4.249721in}}{\pgfqpoint{2.031005in}{4.237230in}}{\pgfqpoint{2.040213in}{4.228022in}}%
\pgfpathcurveto{\pgfqpoint{2.049421in}{4.218813in}}{\pgfqpoint{2.061912in}{4.213639in}}{\pgfqpoint{2.074935in}{4.213639in}}%
\pgfpathlineto{\pgfqpoint{2.074935in}{4.213639in}}%
\pgfpathclose%
\pgfusepath{stroke,fill}%
\end{pgfscope}%
\begin{pgfscope}%
\pgfpathrectangle{\pgfqpoint{0.786164in}{0.768110in}}{\pgfqpoint{8.851069in}{7.081890in}}%
\pgfusepath{clip}%
\pgfsetbuttcap%
\pgfsetroundjoin%
\definecolor{currentfill}{rgb}{0.197636,0.391528,0.554969}%
\pgfsetfillcolor{currentfill}%
\pgfsetfillopacity{0.700000}%
\pgfsetlinewidth{0.501875pt}%
\definecolor{currentstroke}{rgb}{1.000000,1.000000,1.000000}%
\pgfsetstrokecolor{currentstroke}%
\pgfsetstrokeopacity{0.700000}%
\pgfsetdash{}{0pt}%
\pgfpathmoveto{\pgfqpoint{2.140015in}{4.234981in}}%
\pgfpathcurveto{\pgfqpoint{2.153037in}{4.234981in}}{\pgfqpoint{2.165529in}{4.240155in}}{\pgfqpoint{2.174737in}{4.249364in}}%
\pgfpathcurveto{\pgfqpoint{2.183945in}{4.258572in}}{\pgfqpoint{2.189119in}{4.271063in}}{\pgfqpoint{2.189119in}{4.284086in}}%
\pgfpathcurveto{\pgfqpoint{2.189119in}{4.297109in}}{\pgfqpoint{2.183945in}{4.309600in}}{\pgfqpoint{2.174737in}{4.318808in}}%
\pgfpathcurveto{\pgfqpoint{2.165529in}{4.328017in}}{\pgfqpoint{2.153037in}{4.333191in}}{\pgfqpoint{2.140015in}{4.333191in}}%
\pgfpathcurveto{\pgfqpoint{2.126992in}{4.333191in}}{\pgfqpoint{2.114501in}{4.328017in}}{\pgfqpoint{2.105293in}{4.318808in}}%
\pgfpathcurveto{\pgfqpoint{2.096084in}{4.309600in}}{\pgfqpoint{2.090910in}{4.297109in}}{\pgfqpoint{2.090910in}{4.284086in}}%
\pgfpathcurveto{\pgfqpoint{2.090910in}{4.271063in}}{\pgfqpoint{2.096084in}{4.258572in}}{\pgfqpoint{2.105293in}{4.249364in}}%
\pgfpathcurveto{\pgfqpoint{2.114501in}{4.240155in}}{\pgfqpoint{2.126992in}{4.234981in}}{\pgfqpoint{2.140015in}{4.234981in}}%
\pgfpathlineto{\pgfqpoint{2.140015in}{4.234981in}}%
\pgfpathclose%
\pgfusepath{stroke,fill}%
\end{pgfscope}%
\begin{pgfscope}%
\pgfpathrectangle{\pgfqpoint{0.786164in}{0.768110in}}{\pgfqpoint{8.851069in}{7.081890in}}%
\pgfusepath{clip}%
\pgfsetbuttcap%
\pgfsetroundjoin%
\definecolor{currentfill}{rgb}{0.204903,0.375746,0.553533}%
\pgfsetfillcolor{currentfill}%
\pgfsetfillopacity{0.700000}%
\pgfsetlinewidth{0.501875pt}%
\definecolor{currentstroke}{rgb}{1.000000,1.000000,1.000000}%
\pgfsetstrokecolor{currentstroke}%
\pgfsetstrokeopacity{0.700000}%
\pgfsetdash{}{0pt}%
\pgfpathmoveto{\pgfqpoint{2.235131in}{4.085586in}}%
\pgfpathcurveto{\pgfqpoint{2.248154in}{4.085586in}}{\pgfqpoint{2.260645in}{4.090760in}}{\pgfqpoint{2.269853in}{4.099969in}}%
\pgfpathcurveto{\pgfqpoint{2.279062in}{4.109177in}}{\pgfqpoint{2.284236in}{4.121668in}}{\pgfqpoint{2.284236in}{4.134691in}}%
\pgfpathcurveto{\pgfqpoint{2.284236in}{4.147714in}}{\pgfqpoint{2.279062in}{4.160205in}}{\pgfqpoint{2.269853in}{4.169413in}}%
\pgfpathcurveto{\pgfqpoint{2.260645in}{4.178621in}}{\pgfqpoint{2.248154in}{4.183795in}}{\pgfqpoint{2.235131in}{4.183795in}}%
\pgfpathcurveto{\pgfqpoint{2.222108in}{4.183795in}}{\pgfqpoint{2.209617in}{4.178621in}}{\pgfqpoint{2.200409in}{4.169413in}}%
\pgfpathcurveto{\pgfqpoint{2.191200in}{4.160205in}}{\pgfqpoint{2.186026in}{4.147714in}}{\pgfqpoint{2.186026in}{4.134691in}}%
\pgfpathcurveto{\pgfqpoint{2.186026in}{4.121668in}}{\pgfqpoint{2.191200in}{4.109177in}}{\pgfqpoint{2.200409in}{4.099969in}}%
\pgfpathcurveto{\pgfqpoint{2.209617in}{4.090760in}}{\pgfqpoint{2.222108in}{4.085586in}}{\pgfqpoint{2.235131in}{4.085586in}}%
\pgfpathlineto{\pgfqpoint{2.235131in}{4.085586in}}%
\pgfpathclose%
\pgfusepath{stroke,fill}%
\end{pgfscope}%
\begin{pgfscope}%
\pgfpathrectangle{\pgfqpoint{0.786164in}{0.768110in}}{\pgfqpoint{8.851069in}{7.081890in}}%
\pgfusepath{clip}%
\pgfsetbuttcap%
\pgfsetroundjoin%
\definecolor{currentfill}{rgb}{0.218130,0.347432,0.550038}%
\pgfsetfillcolor{currentfill}%
\pgfsetfillopacity{0.700000}%
\pgfsetlinewidth{0.501875pt}%
\definecolor{currentstroke}{rgb}{1.000000,1.000000,1.000000}%
\pgfsetstrokecolor{currentstroke}%
\pgfsetstrokeopacity{0.700000}%
\pgfsetdash{}{0pt}%
\pgfpathmoveto{\pgfqpoint{2.394228in}{3.872164in}}%
\pgfpathcurveto{\pgfqpoint{2.407251in}{3.872164in}}{\pgfqpoint{2.419742in}{3.877338in}}{\pgfqpoint{2.428950in}{3.886547in}}%
\pgfpathcurveto{\pgfqpoint{2.438159in}{3.895755in}}{\pgfqpoint{2.443333in}{3.908246in}}{\pgfqpoint{2.443333in}{3.921269in}}%
\pgfpathcurveto{\pgfqpoint{2.443333in}{3.934292in}}{\pgfqpoint{2.438159in}{3.946783in}}{\pgfqpoint{2.428950in}{3.955991in}}%
\pgfpathcurveto{\pgfqpoint{2.419742in}{3.965200in}}{\pgfqpoint{2.407251in}{3.970374in}}{\pgfqpoint{2.394228in}{3.970374in}}%
\pgfpathcurveto{\pgfqpoint{2.381205in}{3.970374in}}{\pgfqpoint{2.368714in}{3.965200in}}{\pgfqpoint{2.359506in}{3.955991in}}%
\pgfpathcurveto{\pgfqpoint{2.350297in}{3.946783in}}{\pgfqpoint{2.345123in}{3.934292in}}{\pgfqpoint{2.345123in}{3.921269in}}%
\pgfpathcurveto{\pgfqpoint{2.345123in}{3.908246in}}{\pgfqpoint{2.350297in}{3.895755in}}{\pgfqpoint{2.359506in}{3.886547in}}%
\pgfpathcurveto{\pgfqpoint{2.368714in}{3.877338in}}{\pgfqpoint{2.381205in}{3.872164in}}{\pgfqpoint{2.394228in}{3.872164in}}%
\pgfpathlineto{\pgfqpoint{2.394228in}{3.872164in}}%
\pgfpathclose%
\pgfusepath{stroke,fill}%
\end{pgfscope}%
\begin{pgfscope}%
\pgfpathrectangle{\pgfqpoint{0.786164in}{0.768110in}}{\pgfqpoint{8.851069in}{7.081890in}}%
\pgfusepath{clip}%
\pgfsetbuttcap%
\pgfsetroundjoin%
\definecolor{currentfill}{rgb}{0.206756,0.371758,0.553117}%
\pgfsetfillcolor{currentfill}%
\pgfsetfillopacity{0.700000}%
\pgfsetlinewidth{0.501875pt}%
\definecolor{currentstroke}{rgb}{1.000000,1.000000,1.000000}%
\pgfsetstrokecolor{currentstroke}%
\pgfsetstrokeopacity{0.700000}%
\pgfsetdash{}{0pt}%
\pgfpathmoveto{\pgfqpoint{2.459674in}{3.936191in}}%
\pgfpathcurveto{\pgfqpoint{2.472697in}{3.936191in}}{\pgfqpoint{2.485188in}{3.941365in}}{\pgfqpoint{2.494396in}{3.950573in}}%
\pgfpathcurveto{\pgfqpoint{2.503605in}{3.959782in}}{\pgfqpoint{2.508779in}{3.972273in}}{\pgfqpoint{2.508779in}{3.985296in}}%
\pgfpathcurveto{\pgfqpoint{2.508779in}{3.998318in}}{\pgfqpoint{2.503605in}{4.010809in}}{\pgfqpoint{2.494396in}{4.020018in}}%
\pgfpathcurveto{\pgfqpoint{2.485188in}{4.029226in}}{\pgfqpoint{2.472697in}{4.034400in}}{\pgfqpoint{2.459674in}{4.034400in}}%
\pgfpathcurveto{\pgfqpoint{2.446651in}{4.034400in}}{\pgfqpoint{2.434160in}{4.029226in}}{\pgfqpoint{2.424952in}{4.020018in}}%
\pgfpathcurveto{\pgfqpoint{2.415743in}{4.010809in}}{\pgfqpoint{2.410569in}{3.998318in}}{\pgfqpoint{2.410569in}{3.985296in}}%
\pgfpathcurveto{\pgfqpoint{2.410569in}{3.972273in}}{\pgfqpoint{2.415743in}{3.959782in}}{\pgfqpoint{2.424952in}{3.950573in}}%
\pgfpathcurveto{\pgfqpoint{2.434160in}{3.941365in}}{\pgfqpoint{2.446651in}{3.936191in}}{\pgfqpoint{2.459674in}{3.936191in}}%
\pgfpathlineto{\pgfqpoint{2.459674in}{3.936191in}}%
\pgfpathclose%
\pgfusepath{stroke,fill}%
\end{pgfscope}%
\begin{pgfscope}%
\pgfpathrectangle{\pgfqpoint{0.786164in}{0.768110in}}{\pgfqpoint{8.851069in}{7.081890in}}%
\pgfusepath{clip}%
\pgfsetbuttcap%
\pgfsetroundjoin%
\definecolor{currentfill}{rgb}{0.216210,0.351535,0.550627}%
\pgfsetfillcolor{currentfill}%
\pgfsetfillopacity{0.700000}%
\pgfsetlinewidth{0.501875pt}%
\definecolor{currentstroke}{rgb}{1.000000,1.000000,1.000000}%
\pgfsetstrokecolor{currentstroke}%
\pgfsetstrokeopacity{0.700000}%
\pgfsetdash{}{0pt}%
\pgfpathmoveto{\pgfqpoint{2.512421in}{3.872164in}}%
\pgfpathcurveto{\pgfqpoint{2.525444in}{3.872164in}}{\pgfqpoint{2.537935in}{3.877338in}}{\pgfqpoint{2.547144in}{3.886547in}}%
\pgfpathcurveto{\pgfqpoint{2.556352in}{3.895755in}}{\pgfqpoint{2.561526in}{3.908246in}}{\pgfqpoint{2.561526in}{3.921269in}}%
\pgfpathcurveto{\pgfqpoint{2.561526in}{3.934292in}}{\pgfqpoint{2.556352in}{3.946783in}}{\pgfqpoint{2.547144in}{3.955991in}}%
\pgfpathcurveto{\pgfqpoint{2.537935in}{3.965200in}}{\pgfqpoint{2.525444in}{3.970374in}}{\pgfqpoint{2.512421in}{3.970374in}}%
\pgfpathcurveto{\pgfqpoint{2.499399in}{3.970374in}}{\pgfqpoint{2.486908in}{3.965200in}}{\pgfqpoint{2.477699in}{3.955991in}}%
\pgfpathcurveto{\pgfqpoint{2.468491in}{3.946783in}}{\pgfqpoint{2.463317in}{3.934292in}}{\pgfqpoint{2.463317in}{3.921269in}}%
\pgfpathcurveto{\pgfqpoint{2.463317in}{3.908246in}}{\pgfqpoint{2.468491in}{3.895755in}}{\pgfqpoint{2.477699in}{3.886547in}}%
\pgfpathcurveto{\pgfqpoint{2.486908in}{3.877338in}}{\pgfqpoint{2.499399in}{3.872164in}}{\pgfqpoint{2.512421in}{3.872164in}}%
\pgfpathlineto{\pgfqpoint{2.512421in}{3.872164in}}%
\pgfpathclose%
\pgfusepath{stroke,fill}%
\end{pgfscope}%
\begin{pgfscope}%
\pgfpathrectangle{\pgfqpoint{0.786164in}{0.768110in}}{\pgfqpoint{8.851069in}{7.081890in}}%
\pgfusepath{clip}%
\pgfsetbuttcap%
\pgfsetroundjoin%
\definecolor{currentfill}{rgb}{0.220057,0.343307,0.549413}%
\pgfsetfillcolor{currentfill}%
\pgfsetfillopacity{0.700000}%
\pgfsetlinewidth{0.501875pt}%
\definecolor{currentstroke}{rgb}{1.000000,1.000000,1.000000}%
\pgfsetstrokecolor{currentstroke}%
\pgfsetstrokeopacity{0.700000}%
\pgfsetdash{}{0pt}%
\pgfpathmoveto{\pgfqpoint{2.740627in}{3.786796in}}%
\pgfpathcurveto{\pgfqpoint{2.753650in}{3.786796in}}{\pgfqpoint{2.766141in}{3.791970in}}{\pgfqpoint{2.775350in}{3.801178in}}%
\pgfpathcurveto{\pgfqpoint{2.784558in}{3.810387in}}{\pgfqpoint{2.789732in}{3.822878in}}{\pgfqpoint{2.789732in}{3.835900in}}%
\pgfpathcurveto{\pgfqpoint{2.789732in}{3.848923in}}{\pgfqpoint{2.784558in}{3.861414in}}{\pgfqpoint{2.775350in}{3.870623in}}%
\pgfpathcurveto{\pgfqpoint{2.766141in}{3.879831in}}{\pgfqpoint{2.753650in}{3.885005in}}{\pgfqpoint{2.740627in}{3.885005in}}%
\pgfpathcurveto{\pgfqpoint{2.727605in}{3.885005in}}{\pgfqpoint{2.715114in}{3.879831in}}{\pgfqpoint{2.705905in}{3.870623in}}%
\pgfpathcurveto{\pgfqpoint{2.696697in}{3.861414in}}{\pgfqpoint{2.691523in}{3.848923in}}{\pgfqpoint{2.691523in}{3.835900in}}%
\pgfpathcurveto{\pgfqpoint{2.691523in}{3.822878in}}{\pgfqpoint{2.696697in}{3.810387in}}{\pgfqpoint{2.705905in}{3.801178in}}%
\pgfpathcurveto{\pgfqpoint{2.715114in}{3.791970in}}{\pgfqpoint{2.727605in}{3.786796in}}{\pgfqpoint{2.740627in}{3.786796in}}%
\pgfpathlineto{\pgfqpoint{2.740627in}{3.786796in}}%
\pgfpathclose%
\pgfusepath{stroke,fill}%
\end{pgfscope}%
\begin{pgfscope}%
\pgfpathrectangle{\pgfqpoint{0.786164in}{0.768110in}}{\pgfqpoint{8.851069in}{7.081890in}}%
\pgfusepath{clip}%
\pgfsetbuttcap%
\pgfsetroundjoin%
\definecolor{currentfill}{rgb}{0.218130,0.347432,0.550038}%
\pgfsetfillcolor{currentfill}%
\pgfsetfillopacity{0.700000}%
\pgfsetlinewidth{0.501875pt}%
\definecolor{currentstroke}{rgb}{1.000000,1.000000,1.000000}%
\pgfsetstrokecolor{currentstroke}%
\pgfsetstrokeopacity{0.700000}%
\pgfsetdash{}{0pt}%
\pgfpathmoveto{\pgfqpoint{2.876525in}{3.701427in}}%
\pgfpathcurveto{\pgfqpoint{2.889548in}{3.701427in}}{\pgfqpoint{2.902039in}{3.706601in}}{\pgfqpoint{2.911248in}{3.715810in}}%
\pgfpathcurveto{\pgfqpoint{2.920456in}{3.725018in}}{\pgfqpoint{2.925630in}{3.737509in}}{\pgfqpoint{2.925630in}{3.750532in}}%
\pgfpathcurveto{\pgfqpoint{2.925630in}{3.763554in}}{\pgfqpoint{2.920456in}{3.776046in}}{\pgfqpoint{2.911248in}{3.785254in}}%
\pgfpathcurveto{\pgfqpoint{2.902039in}{3.794462in}}{\pgfqpoint{2.889548in}{3.799636in}}{\pgfqpoint{2.876525in}{3.799636in}}%
\pgfpathcurveto{\pgfqpoint{2.863503in}{3.799636in}}{\pgfqpoint{2.851012in}{3.794462in}}{\pgfqpoint{2.841803in}{3.785254in}}%
\pgfpathcurveto{\pgfqpoint{2.832595in}{3.776046in}}{\pgfqpoint{2.827421in}{3.763554in}}{\pgfqpoint{2.827421in}{3.750532in}}%
\pgfpathcurveto{\pgfqpoint{2.827421in}{3.737509in}}{\pgfqpoint{2.832595in}{3.725018in}}{\pgfqpoint{2.841803in}{3.715810in}}%
\pgfpathcurveto{\pgfqpoint{2.851012in}{3.706601in}}{\pgfqpoint{2.863503in}{3.701427in}}{\pgfqpoint{2.876525in}{3.701427in}}%
\pgfpathlineto{\pgfqpoint{2.876525in}{3.701427in}}%
\pgfpathclose%
\pgfusepath{stroke,fill}%
\end{pgfscope}%
\begin{pgfscope}%
\pgfpathrectangle{\pgfqpoint{0.786164in}{0.768110in}}{\pgfqpoint{8.851069in}{7.081890in}}%
\pgfusepath{clip}%
\pgfsetbuttcap%
\pgfsetroundjoin%
\definecolor{currentfill}{rgb}{0.227802,0.326594,0.546532}%
\pgfsetfillcolor{currentfill}%
\pgfsetfillopacity{0.700000}%
\pgfsetlinewidth{0.501875pt}%
\definecolor{currentstroke}{rgb}{1.000000,1.000000,1.000000}%
\pgfsetstrokecolor{currentstroke}%
\pgfsetstrokeopacity{0.700000}%
\pgfsetdash{}{0pt}%
\pgfpathmoveto{\pgfqpoint{3.016575in}{3.637401in}}%
\pgfpathcurveto{\pgfqpoint{3.029597in}{3.637401in}}{\pgfqpoint{3.042088in}{3.642575in}}{\pgfqpoint{3.051297in}{3.651783in}}%
\pgfpathcurveto{\pgfqpoint{3.060505in}{3.660991in}}{\pgfqpoint{3.065679in}{3.673483in}}{\pgfqpoint{3.065679in}{3.686505in}}%
\pgfpathcurveto{\pgfqpoint{3.065679in}{3.699528in}}{\pgfqpoint{3.060505in}{3.712019in}}{\pgfqpoint{3.051297in}{3.721227in}}%
\pgfpathcurveto{\pgfqpoint{3.042088in}{3.730436in}}{\pgfqpoint{3.029597in}{3.735610in}}{\pgfqpoint{3.016575in}{3.735610in}}%
\pgfpathcurveto{\pgfqpoint{3.003552in}{3.735610in}}{\pgfqpoint{2.991061in}{3.730436in}}{\pgfqpoint{2.981852in}{3.721227in}}%
\pgfpathcurveto{\pgfqpoint{2.972644in}{3.712019in}}{\pgfqpoint{2.967470in}{3.699528in}}{\pgfqpoint{2.967470in}{3.686505in}}%
\pgfpathcurveto{\pgfqpoint{2.967470in}{3.673483in}}{\pgfqpoint{2.972644in}{3.660991in}}{\pgfqpoint{2.981852in}{3.651783in}}%
\pgfpathcurveto{\pgfqpoint{2.991061in}{3.642575in}}{\pgfqpoint{3.003552in}{3.637401in}}{\pgfqpoint{3.016575in}{3.637401in}}%
\pgfpathlineto{\pgfqpoint{3.016575in}{3.637401in}}%
\pgfpathclose%
\pgfusepath{stroke,fill}%
\end{pgfscope}%
\begin{pgfscope}%
\pgfpathrectangle{\pgfqpoint{0.786164in}{0.768110in}}{\pgfqpoint{8.851069in}{7.081890in}}%
\pgfusepath{clip}%
\pgfsetbuttcap%
\pgfsetroundjoin%
\definecolor{currentfill}{rgb}{0.225863,0.330805,0.547314}%
\pgfsetfillcolor{currentfill}%
\pgfsetfillopacity{0.700000}%
\pgfsetlinewidth{0.501875pt}%
\definecolor{currentstroke}{rgb}{1.000000,1.000000,1.000000}%
\pgfsetstrokecolor{currentstroke}%
\pgfsetstrokeopacity{0.700000}%
\pgfsetdash{}{0pt}%
\pgfpathmoveto{\pgfqpoint{3.016086in}{3.680085in}}%
\pgfpathcurveto{\pgfqpoint{3.029109in}{3.680085in}}{\pgfqpoint{3.041600in}{3.685259in}}{\pgfqpoint{3.050808in}{3.694467in}}%
\pgfpathcurveto{\pgfqpoint{3.060017in}{3.703676in}}{\pgfqpoint{3.065191in}{3.716167in}}{\pgfqpoint{3.065191in}{3.729190in}}%
\pgfpathcurveto{\pgfqpoint{3.065191in}{3.742212in}}{\pgfqpoint{3.060017in}{3.754703in}}{\pgfqpoint{3.050808in}{3.763912in}}%
\pgfpathcurveto{\pgfqpoint{3.041600in}{3.773120in}}{\pgfqpoint{3.029109in}{3.778294in}}{\pgfqpoint{3.016086in}{3.778294in}}%
\pgfpathcurveto{\pgfqpoint{3.003064in}{3.778294in}}{\pgfqpoint{2.990572in}{3.773120in}}{\pgfqpoint{2.981364in}{3.763912in}}%
\pgfpathcurveto{\pgfqpoint{2.972156in}{3.754703in}}{\pgfqpoint{2.966982in}{3.742212in}}{\pgfqpoint{2.966982in}{3.729190in}}%
\pgfpathcurveto{\pgfqpoint{2.966982in}{3.716167in}}{\pgfqpoint{2.972156in}{3.703676in}}{\pgfqpoint{2.981364in}{3.694467in}}%
\pgfpathcurveto{\pgfqpoint{2.990572in}{3.685259in}}{\pgfqpoint{3.003064in}{3.680085in}}{\pgfqpoint{3.016086in}{3.680085in}}%
\pgfpathlineto{\pgfqpoint{3.016086in}{3.680085in}}%
\pgfpathclose%
\pgfusepath{stroke,fill}%
\end{pgfscope}%
\begin{pgfscope}%
\pgfpathrectangle{\pgfqpoint{0.786164in}{0.768110in}}{\pgfqpoint{8.851069in}{7.081890in}}%
\pgfusepath{clip}%
\pgfsetbuttcap%
\pgfsetroundjoin%
\definecolor{currentfill}{rgb}{0.223925,0.334994,0.548053}%
\pgfsetfillcolor{currentfill}%
\pgfsetfillopacity{0.700000}%
\pgfsetlinewidth{0.501875pt}%
\definecolor{currentstroke}{rgb}{1.000000,1.000000,1.000000}%
\pgfsetstrokecolor{currentstroke}%
\pgfsetstrokeopacity{0.700000}%
\pgfsetdash{}{0pt}%
\pgfpathmoveto{\pgfqpoint{2.998504in}{3.722769in}}%
\pgfpathcurveto{\pgfqpoint{3.011526in}{3.722769in}}{\pgfqpoint{3.024018in}{3.727943in}}{\pgfqpoint{3.033226in}{3.737152in}}%
\pgfpathcurveto{\pgfqpoint{3.042434in}{3.746360in}}{\pgfqpoint{3.047608in}{3.758851in}}{\pgfqpoint{3.047608in}{3.771874in}}%
\pgfpathcurveto{\pgfqpoint{3.047608in}{3.784897in}}{\pgfqpoint{3.042434in}{3.797388in}}{\pgfqpoint{3.033226in}{3.806596in}}%
\pgfpathcurveto{\pgfqpoint{3.024018in}{3.815805in}}{\pgfqpoint{3.011526in}{3.820979in}}{\pgfqpoint{2.998504in}{3.820979in}}%
\pgfpathcurveto{\pgfqpoint{2.985481in}{3.820979in}}{\pgfqpoint{2.972990in}{3.815805in}}{\pgfqpoint{2.963782in}{3.806596in}}%
\pgfpathcurveto{\pgfqpoint{2.954573in}{3.797388in}}{\pgfqpoint{2.949399in}{3.784897in}}{\pgfqpoint{2.949399in}{3.771874in}}%
\pgfpathcurveto{\pgfqpoint{2.949399in}{3.758851in}}{\pgfqpoint{2.954573in}{3.746360in}}{\pgfqpoint{2.963782in}{3.737152in}}%
\pgfpathcurveto{\pgfqpoint{2.972990in}{3.727943in}}{\pgfqpoint{2.985481in}{3.722769in}}{\pgfqpoint{2.998504in}{3.722769in}}%
\pgfpathlineto{\pgfqpoint{2.998504in}{3.722769in}}%
\pgfpathclose%
\pgfusepath{stroke,fill}%
\end{pgfscope}%
\begin{pgfscope}%
\pgfpathrectangle{\pgfqpoint{0.786164in}{0.768110in}}{\pgfqpoint{8.851069in}{7.081890in}}%
\pgfusepath{clip}%
\pgfsetbuttcap%
\pgfsetroundjoin%
\definecolor{currentfill}{rgb}{0.221989,0.339161,0.548752}%
\pgfsetfillcolor{currentfill}%
\pgfsetfillopacity{0.700000}%
\pgfsetlinewidth{0.501875pt}%
\definecolor{currentstroke}{rgb}{1.000000,1.000000,1.000000}%
\pgfsetstrokecolor{currentstroke}%
\pgfsetstrokeopacity{0.700000}%
\pgfsetdash{}{0pt}%
\pgfpathmoveto{\pgfqpoint{2.982997in}{3.765454in}}%
\pgfpathcurveto{\pgfqpoint{2.996020in}{3.765454in}}{\pgfqpoint{3.008511in}{3.770628in}}{\pgfqpoint{3.017719in}{3.779836in}}%
\pgfpathcurveto{\pgfqpoint{3.026928in}{3.789044in}}{\pgfqpoint{3.032102in}{3.801536in}}{\pgfqpoint{3.032102in}{3.814558in}}%
\pgfpathcurveto{\pgfqpoint{3.032102in}{3.827581in}}{\pgfqpoint{3.026928in}{3.840072in}}{\pgfqpoint{3.017719in}{3.849280in}}%
\pgfpathcurveto{\pgfqpoint{3.008511in}{3.858489in}}{\pgfqpoint{2.996020in}{3.863663in}}{\pgfqpoint{2.982997in}{3.863663in}}%
\pgfpathcurveto{\pgfqpoint{2.969974in}{3.863663in}}{\pgfqpoint{2.957483in}{3.858489in}}{\pgfqpoint{2.948275in}{3.849280in}}%
\pgfpathcurveto{\pgfqpoint{2.939066in}{3.840072in}}{\pgfqpoint{2.933892in}{3.827581in}}{\pgfqpoint{2.933892in}{3.814558in}}%
\pgfpathcurveto{\pgfqpoint{2.933892in}{3.801536in}}{\pgfqpoint{2.939066in}{3.789044in}}{\pgfqpoint{2.948275in}{3.779836in}}%
\pgfpathcurveto{\pgfqpoint{2.957483in}{3.770628in}}{\pgfqpoint{2.969974in}{3.765454in}}{\pgfqpoint{2.982997in}{3.765454in}}%
\pgfpathlineto{\pgfqpoint{2.982997in}{3.765454in}}%
\pgfpathclose%
\pgfusepath{stroke,fill}%
\end{pgfscope}%
\begin{pgfscope}%
\pgfpathrectangle{\pgfqpoint{0.786164in}{0.768110in}}{\pgfqpoint{8.851069in}{7.081890in}}%
\pgfusepath{clip}%
\pgfsetbuttcap%
\pgfsetroundjoin%
\definecolor{currentfill}{rgb}{0.221989,0.339161,0.548752}%
\pgfsetfillcolor{currentfill}%
\pgfsetfillopacity{0.700000}%
\pgfsetlinewidth{0.501875pt}%
\definecolor{currentstroke}{rgb}{1.000000,1.000000,1.000000}%
\pgfsetstrokecolor{currentstroke}%
\pgfsetstrokeopacity{0.700000}%
\pgfsetdash{}{0pt}%
\pgfpathmoveto{\pgfqpoint{3.024511in}{3.829480in}}%
\pgfpathcurveto{\pgfqpoint{3.037534in}{3.829480in}}{\pgfqpoint{3.050025in}{3.834654in}}{\pgfqpoint{3.059233in}{3.843863in}}%
\pgfpathcurveto{\pgfqpoint{3.068442in}{3.853071in}}{\pgfqpoint{3.073616in}{3.865562in}}{\pgfqpoint{3.073616in}{3.878585in}}%
\pgfpathcurveto{\pgfqpoint{3.073616in}{3.891607in}}{\pgfqpoint{3.068442in}{3.904099in}}{\pgfqpoint{3.059233in}{3.913307in}}%
\pgfpathcurveto{\pgfqpoint{3.050025in}{3.922515in}}{\pgfqpoint{3.037534in}{3.927689in}}{\pgfqpoint{3.024511in}{3.927689in}}%
\pgfpathcurveto{\pgfqpoint{3.011488in}{3.927689in}}{\pgfqpoint{2.998997in}{3.922515in}}{\pgfqpoint{2.989789in}{3.913307in}}%
\pgfpathcurveto{\pgfqpoint{2.980581in}{3.904099in}}{\pgfqpoint{2.975407in}{3.891607in}}{\pgfqpoint{2.975407in}{3.878585in}}%
\pgfpathcurveto{\pgfqpoint{2.975407in}{3.865562in}}{\pgfqpoint{2.980581in}{3.853071in}}{\pgfqpoint{2.989789in}{3.843863in}}%
\pgfpathcurveto{\pgfqpoint{2.998997in}{3.834654in}}{\pgfqpoint{3.011488in}{3.829480in}}{\pgfqpoint{3.024511in}{3.829480in}}%
\pgfpathlineto{\pgfqpoint{3.024511in}{3.829480in}}%
\pgfpathclose%
\pgfusepath{stroke,fill}%
\end{pgfscope}%
\begin{pgfscope}%
\pgfpathrectangle{\pgfqpoint{0.786164in}{0.768110in}}{\pgfqpoint{8.851069in}{7.081890in}}%
\pgfusepath{clip}%
\pgfsetbuttcap%
\pgfsetroundjoin%
\definecolor{currentfill}{rgb}{0.225863,0.330805,0.547314}%
\pgfsetfillcolor{currentfill}%
\pgfsetfillopacity{0.700000}%
\pgfsetlinewidth{0.501875pt}%
\definecolor{currentstroke}{rgb}{1.000000,1.000000,1.000000}%
\pgfsetstrokecolor{currentstroke}%
\pgfsetstrokeopacity{0.700000}%
\pgfsetdash{}{0pt}%
\pgfpathmoveto{\pgfqpoint{3.158822in}{3.829480in}}%
\pgfpathcurveto{\pgfqpoint{3.171844in}{3.829480in}}{\pgfqpoint{3.184336in}{3.834654in}}{\pgfqpoint{3.193544in}{3.843863in}}%
\pgfpathcurveto{\pgfqpoint{3.202752in}{3.853071in}}{\pgfqpoint{3.207926in}{3.865562in}}{\pgfqpoint{3.207926in}{3.878585in}}%
\pgfpathcurveto{\pgfqpoint{3.207926in}{3.891607in}}{\pgfqpoint{3.202752in}{3.904099in}}{\pgfqpoint{3.193544in}{3.913307in}}%
\pgfpathcurveto{\pgfqpoint{3.184336in}{3.922515in}}{\pgfqpoint{3.171844in}{3.927689in}}{\pgfqpoint{3.158822in}{3.927689in}}%
\pgfpathcurveto{\pgfqpoint{3.145799in}{3.927689in}}{\pgfqpoint{3.133308in}{3.922515in}}{\pgfqpoint{3.124100in}{3.913307in}}%
\pgfpathcurveto{\pgfqpoint{3.114891in}{3.904099in}}{\pgfqpoint{3.109717in}{3.891607in}}{\pgfqpoint{3.109717in}{3.878585in}}%
\pgfpathcurveto{\pgfqpoint{3.109717in}{3.865562in}}{\pgfqpoint{3.114891in}{3.853071in}}{\pgfqpoint{3.124100in}{3.843863in}}%
\pgfpathcurveto{\pgfqpoint{3.133308in}{3.834654in}}{\pgfqpoint{3.145799in}{3.829480in}}{\pgfqpoint{3.158822in}{3.829480in}}%
\pgfpathlineto{\pgfqpoint{3.158822in}{3.829480in}}%
\pgfpathclose%
\pgfusepath{stroke,fill}%
\end{pgfscope}%
\begin{pgfscope}%
\pgfpathrectangle{\pgfqpoint{0.786164in}{0.768110in}}{\pgfqpoint{8.851069in}{7.081890in}}%
\pgfusepath{clip}%
\pgfsetbuttcap%
\pgfsetroundjoin%
\definecolor{currentfill}{rgb}{0.237441,0.305202,0.541921}%
\pgfsetfillcolor{currentfill}%
\pgfsetfillopacity{0.700000}%
\pgfsetlinewidth{0.501875pt}%
\definecolor{currentstroke}{rgb}{1.000000,1.000000,1.000000}%
\pgfsetstrokecolor{currentstroke}%
\pgfsetstrokeopacity{0.700000}%
\pgfsetdash{}{0pt}%
\pgfpathmoveto{\pgfqpoint{3.288737in}{3.658743in}}%
\pgfpathcurveto{\pgfqpoint{3.301759in}{3.658743in}}{\pgfqpoint{3.314251in}{3.663917in}}{\pgfqpoint{3.323459in}{3.673125in}}%
\pgfpathcurveto{\pgfqpoint{3.332667in}{3.682334in}}{\pgfqpoint{3.337841in}{3.694825in}}{\pgfqpoint{3.337841in}{3.707847in}}%
\pgfpathcurveto{\pgfqpoint{3.337841in}{3.720870in}}{\pgfqpoint{3.332667in}{3.733361in}}{\pgfqpoint{3.323459in}{3.742570in}}%
\pgfpathcurveto{\pgfqpoint{3.314251in}{3.751778in}}{\pgfqpoint{3.301759in}{3.756952in}}{\pgfqpoint{3.288737in}{3.756952in}}%
\pgfpathcurveto{\pgfqpoint{3.275714in}{3.756952in}}{\pgfqpoint{3.263223in}{3.751778in}}{\pgfqpoint{3.254015in}{3.742570in}}%
\pgfpathcurveto{\pgfqpoint{3.244806in}{3.733361in}}{\pgfqpoint{3.239632in}{3.720870in}}{\pgfqpoint{3.239632in}{3.707847in}}%
\pgfpathcurveto{\pgfqpoint{3.239632in}{3.694825in}}{\pgfqpoint{3.244806in}{3.682334in}}{\pgfqpoint{3.254015in}{3.673125in}}%
\pgfpathcurveto{\pgfqpoint{3.263223in}{3.663917in}}{\pgfqpoint{3.275714in}{3.658743in}}{\pgfqpoint{3.288737in}{3.658743in}}%
\pgfpathlineto{\pgfqpoint{3.288737in}{3.658743in}}%
\pgfpathclose%
\pgfusepath{stroke,fill}%
\end{pgfscope}%
\begin{pgfscope}%
\pgfpathrectangle{\pgfqpoint{0.786164in}{0.768110in}}{\pgfqpoint{8.851069in}{7.081890in}}%
\pgfusepath{clip}%
\pgfsetbuttcap%
\pgfsetroundjoin%
\definecolor{currentfill}{rgb}{0.223925,0.334994,0.548053}%
\pgfsetfillcolor{currentfill}%
\pgfsetfillopacity{0.700000}%
\pgfsetlinewidth{0.501875pt}%
\definecolor{currentstroke}{rgb}{1.000000,1.000000,1.000000}%
\pgfsetstrokecolor{currentstroke}%
\pgfsetstrokeopacity{0.700000}%
\pgfsetdash{}{0pt}%
\pgfpathmoveto{\pgfqpoint{3.333670in}{3.744111in}}%
\pgfpathcurveto{\pgfqpoint{3.346692in}{3.744111in}}{\pgfqpoint{3.359184in}{3.749285in}}{\pgfqpoint{3.368392in}{3.758494in}}%
\pgfpathcurveto{\pgfqpoint{3.377600in}{3.767702in}}{\pgfqpoint{3.382774in}{3.780193in}}{\pgfqpoint{3.382774in}{3.793216in}}%
\pgfpathcurveto{\pgfqpoint{3.382774in}{3.806239in}}{\pgfqpoint{3.377600in}{3.818730in}}{\pgfqpoint{3.368392in}{3.827938in}}%
\pgfpathcurveto{\pgfqpoint{3.359184in}{3.837147in}}{\pgfqpoint{3.346692in}{3.842321in}}{\pgfqpoint{3.333670in}{3.842321in}}%
\pgfpathcurveto{\pgfqpoint{3.320647in}{3.842321in}}{\pgfqpoint{3.308156in}{3.837147in}}{\pgfqpoint{3.298948in}{3.827938in}}%
\pgfpathcurveto{\pgfqpoint{3.289739in}{3.818730in}}{\pgfqpoint{3.284565in}{3.806239in}}{\pgfqpoint{3.284565in}{3.793216in}}%
\pgfpathcurveto{\pgfqpoint{3.284565in}{3.780193in}}{\pgfqpoint{3.289739in}{3.767702in}}{\pgfqpoint{3.298948in}{3.758494in}}%
\pgfpathcurveto{\pgfqpoint{3.308156in}{3.749285in}}{\pgfqpoint{3.320647in}{3.744111in}}{\pgfqpoint{3.333670in}{3.744111in}}%
\pgfpathlineto{\pgfqpoint{3.333670in}{3.744111in}}%
\pgfpathclose%
\pgfusepath{stroke,fill}%
\end{pgfscope}%
\begin{pgfscope}%
\pgfpathrectangle{\pgfqpoint{0.786164in}{0.768110in}}{\pgfqpoint{8.851069in}{7.081890in}}%
\pgfusepath{clip}%
\pgfsetbuttcap%
\pgfsetroundjoin%
\definecolor{currentfill}{rgb}{0.231674,0.318106,0.544834}%
\pgfsetfillcolor{currentfill}%
\pgfsetfillopacity{0.700000}%
\pgfsetlinewidth{0.501875pt}%
\definecolor{currentstroke}{rgb}{1.000000,1.000000,1.000000}%
\pgfsetstrokecolor{currentstroke}%
\pgfsetstrokeopacity{0.700000}%
\pgfsetdash{}{0pt}%
\pgfpathmoveto{\pgfqpoint{3.397650in}{3.616058in}}%
\pgfpathcurveto{\pgfqpoint{3.410673in}{3.616058in}}{\pgfqpoint{3.423164in}{3.621232in}}{\pgfqpoint{3.432373in}{3.630441in}}%
\pgfpathcurveto{\pgfqpoint{3.441581in}{3.639649in}}{\pgfqpoint{3.446755in}{3.652140in}}{\pgfqpoint{3.446755in}{3.665163in}}%
\pgfpathcurveto{\pgfqpoint{3.446755in}{3.678186in}}{\pgfqpoint{3.441581in}{3.690677in}}{\pgfqpoint{3.432373in}{3.699885in}}%
\pgfpathcurveto{\pgfqpoint{3.423164in}{3.709094in}}{\pgfqpoint{3.410673in}{3.714268in}}{\pgfqpoint{3.397650in}{3.714268in}}%
\pgfpathcurveto{\pgfqpoint{3.384628in}{3.714268in}}{\pgfqpoint{3.372137in}{3.709094in}}{\pgfqpoint{3.362928in}{3.699885in}}%
\pgfpathcurveto{\pgfqpoint{3.353720in}{3.690677in}}{\pgfqpoint{3.348546in}{3.678186in}}{\pgfqpoint{3.348546in}{3.665163in}}%
\pgfpathcurveto{\pgfqpoint{3.348546in}{3.652140in}}{\pgfqpoint{3.353720in}{3.639649in}}{\pgfqpoint{3.362928in}{3.630441in}}%
\pgfpathcurveto{\pgfqpoint{3.372137in}{3.621232in}}{\pgfqpoint{3.384628in}{3.616058in}}{\pgfqpoint{3.397650in}{3.616058in}}%
\pgfpathlineto{\pgfqpoint{3.397650in}{3.616058in}}%
\pgfpathclose%
\pgfusepath{stroke,fill}%
\end{pgfscope}%
\begin{pgfscope}%
\pgfpathrectangle{\pgfqpoint{0.786164in}{0.768110in}}{\pgfqpoint{8.851069in}{7.081890in}}%
\pgfusepath{clip}%
\pgfsetbuttcap%
\pgfsetroundjoin%
\definecolor{currentfill}{rgb}{0.214298,0.355619,0.551184}%
\pgfsetfillcolor{currentfill}%
\pgfsetfillopacity{0.700000}%
\pgfsetlinewidth{0.501875pt}%
\definecolor{currentstroke}{rgb}{1.000000,1.000000,1.000000}%
\pgfsetstrokecolor{currentstroke}%
\pgfsetstrokeopacity{0.700000}%
\pgfsetdash{}{0pt}%
\pgfpathmoveto{\pgfqpoint{1.933909in}{3.850822in}}%
\pgfpathcurveto{\pgfqpoint{1.946932in}{3.850822in}}{\pgfqpoint{1.959423in}{3.855996in}}{\pgfqpoint{1.968631in}{3.865205in}}%
\pgfpathcurveto{\pgfqpoint{1.977840in}{3.874413in}}{\pgfqpoint{1.983014in}{3.886904in}}{\pgfqpoint{1.983014in}{3.899927in}}%
\pgfpathcurveto{\pgfqpoint{1.983014in}{3.912950in}}{\pgfqpoint{1.977840in}{3.925441in}}{\pgfqpoint{1.968631in}{3.934649in}}%
\pgfpathcurveto{\pgfqpoint{1.959423in}{3.943858in}}{\pgfqpoint{1.946932in}{3.949032in}}{\pgfqpoint{1.933909in}{3.949032in}}%
\pgfpathcurveto{\pgfqpoint{1.920886in}{3.949032in}}{\pgfqpoint{1.908395in}{3.943858in}}{\pgfqpoint{1.899187in}{3.934649in}}%
\pgfpathcurveto{\pgfqpoint{1.889978in}{3.925441in}}{\pgfqpoint{1.884804in}{3.912950in}}{\pgfqpoint{1.884804in}{3.899927in}}%
\pgfpathcurveto{\pgfqpoint{1.884804in}{3.886904in}}{\pgfqpoint{1.889978in}{3.874413in}}{\pgfqpoint{1.899187in}{3.865205in}}%
\pgfpathcurveto{\pgfqpoint{1.908395in}{3.855996in}}{\pgfqpoint{1.920886in}{3.850822in}}{\pgfqpoint{1.933909in}{3.850822in}}%
\pgfpathlineto{\pgfqpoint{1.933909in}{3.850822in}}%
\pgfpathclose%
\pgfusepath{stroke,fill}%
\end{pgfscope}%
\begin{pgfscope}%
\pgfpathrectangle{\pgfqpoint{0.786164in}{0.768110in}}{\pgfqpoint{8.851069in}{7.081890in}}%
\pgfusepath{clip}%
\pgfsetbuttcap%
\pgfsetroundjoin%
\definecolor{currentfill}{rgb}{0.216210,0.351535,0.550627}%
\pgfsetfillcolor{currentfill}%
\pgfsetfillopacity{0.700000}%
\pgfsetlinewidth{0.501875pt}%
\definecolor{currentstroke}{rgb}{1.000000,1.000000,1.000000}%
\pgfsetstrokecolor{currentstroke}%
\pgfsetstrokeopacity{0.700000}%
\pgfsetdash{}{0pt}%
\pgfpathmoveto{\pgfqpoint{2.051126in}{3.829480in}}%
\pgfpathcurveto{\pgfqpoint{2.064148in}{3.829480in}}{\pgfqpoint{2.076639in}{3.834654in}}{\pgfqpoint{2.085848in}{3.843863in}}%
\pgfpathcurveto{\pgfqpoint{2.095056in}{3.853071in}}{\pgfqpoint{2.100230in}{3.865562in}}{\pgfqpoint{2.100230in}{3.878585in}}%
\pgfpathcurveto{\pgfqpoint{2.100230in}{3.891607in}}{\pgfqpoint{2.095056in}{3.904099in}}{\pgfqpoint{2.085848in}{3.913307in}}%
\pgfpathcurveto{\pgfqpoint{2.076639in}{3.922515in}}{\pgfqpoint{2.064148in}{3.927689in}}{\pgfqpoint{2.051126in}{3.927689in}}%
\pgfpathcurveto{\pgfqpoint{2.038103in}{3.927689in}}{\pgfqpoint{2.025612in}{3.922515in}}{\pgfqpoint{2.016403in}{3.913307in}}%
\pgfpathcurveto{\pgfqpoint{2.007195in}{3.904099in}}{\pgfqpoint{2.002021in}{3.891607in}}{\pgfqpoint{2.002021in}{3.878585in}}%
\pgfpathcurveto{\pgfqpoint{2.002021in}{3.865562in}}{\pgfqpoint{2.007195in}{3.853071in}}{\pgfqpoint{2.016403in}{3.843863in}}%
\pgfpathcurveto{\pgfqpoint{2.025612in}{3.834654in}}{\pgfqpoint{2.038103in}{3.829480in}}{\pgfqpoint{2.051126in}{3.829480in}}%
\pgfpathlineto{\pgfqpoint{2.051126in}{3.829480in}}%
\pgfpathclose%
\pgfusepath{stroke,fill}%
\end{pgfscope}%
\begin{pgfscope}%
\pgfpathrectangle{\pgfqpoint{0.786164in}{0.768110in}}{\pgfqpoint{8.851069in}{7.081890in}}%
\pgfusepath{clip}%
\pgfsetbuttcap%
\pgfsetroundjoin%
\definecolor{currentfill}{rgb}{0.208623,0.367752,0.552675}%
\pgfsetfillcolor{currentfill}%
\pgfsetfillopacity{0.700000}%
\pgfsetlinewidth{0.501875pt}%
\definecolor{currentstroke}{rgb}{1.000000,1.000000,1.000000}%
\pgfsetstrokecolor{currentstroke}%
\pgfsetstrokeopacity{0.700000}%
\pgfsetdash{}{0pt}%
\pgfpathmoveto{\pgfqpoint{2.209734in}{3.850822in}}%
\pgfpathcurveto{\pgfqpoint{2.222757in}{3.850822in}}{\pgfqpoint{2.235248in}{3.855996in}}{\pgfqpoint{2.244456in}{3.865205in}}%
\pgfpathcurveto{\pgfqpoint{2.253665in}{3.874413in}}{\pgfqpoint{2.258839in}{3.886904in}}{\pgfqpoint{2.258839in}{3.899927in}}%
\pgfpathcurveto{\pgfqpoint{2.258839in}{3.912950in}}{\pgfqpoint{2.253665in}{3.925441in}}{\pgfqpoint{2.244456in}{3.934649in}}%
\pgfpathcurveto{\pgfqpoint{2.235248in}{3.943858in}}{\pgfqpoint{2.222757in}{3.949032in}}{\pgfqpoint{2.209734in}{3.949032in}}%
\pgfpathcurveto{\pgfqpoint{2.196712in}{3.949032in}}{\pgfqpoint{2.184220in}{3.943858in}}{\pgfqpoint{2.175012in}{3.934649in}}%
\pgfpathcurveto{\pgfqpoint{2.165804in}{3.925441in}}{\pgfqpoint{2.160630in}{3.912950in}}{\pgfqpoint{2.160630in}{3.899927in}}%
\pgfpathcurveto{\pgfqpoint{2.160630in}{3.886904in}}{\pgfqpoint{2.165804in}{3.874413in}}{\pgfqpoint{2.175012in}{3.865205in}}%
\pgfpathcurveto{\pgfqpoint{2.184220in}{3.855996in}}{\pgfqpoint{2.196712in}{3.850822in}}{\pgfqpoint{2.209734in}{3.850822in}}%
\pgfpathlineto{\pgfqpoint{2.209734in}{3.850822in}}%
\pgfpathclose%
\pgfusepath{stroke,fill}%
\end{pgfscope}%
\begin{pgfscope}%
\pgfpathrectangle{\pgfqpoint{0.786164in}{0.768110in}}{\pgfqpoint{8.851069in}{7.081890in}}%
\pgfusepath{clip}%
\pgfsetbuttcap%
\pgfsetroundjoin%
\definecolor{currentfill}{rgb}{0.220057,0.343307,0.549413}%
\pgfsetfillcolor{currentfill}%
\pgfsetfillopacity{0.700000}%
\pgfsetlinewidth{0.501875pt}%
\definecolor{currentstroke}{rgb}{1.000000,1.000000,1.000000}%
\pgfsetstrokecolor{currentstroke}%
\pgfsetstrokeopacity{0.700000}%
\pgfsetdash{}{0pt}%
\pgfpathmoveto{\pgfqpoint{2.401798in}{3.765454in}}%
\pgfpathcurveto{\pgfqpoint{2.414821in}{3.765454in}}{\pgfqpoint{2.427312in}{3.770628in}}{\pgfqpoint{2.436521in}{3.779836in}}%
\pgfpathcurveto{\pgfqpoint{2.445729in}{3.789044in}}{\pgfqpoint{2.450903in}{3.801536in}}{\pgfqpoint{2.450903in}{3.814558in}}%
\pgfpathcurveto{\pgfqpoint{2.450903in}{3.827581in}}{\pgfqpoint{2.445729in}{3.840072in}}{\pgfqpoint{2.436521in}{3.849280in}}%
\pgfpathcurveto{\pgfqpoint{2.427312in}{3.858489in}}{\pgfqpoint{2.414821in}{3.863663in}}{\pgfqpoint{2.401798in}{3.863663in}}%
\pgfpathcurveto{\pgfqpoint{2.388776in}{3.863663in}}{\pgfqpoint{2.376285in}{3.858489in}}{\pgfqpoint{2.367076in}{3.849280in}}%
\pgfpathcurveto{\pgfqpoint{2.357868in}{3.840072in}}{\pgfqpoint{2.352694in}{3.827581in}}{\pgfqpoint{2.352694in}{3.814558in}}%
\pgfpathcurveto{\pgfqpoint{2.352694in}{3.801536in}}{\pgfqpoint{2.357868in}{3.789044in}}{\pgfqpoint{2.367076in}{3.779836in}}%
\pgfpathcurveto{\pgfqpoint{2.376285in}{3.770628in}}{\pgfqpoint{2.388776in}{3.765454in}}{\pgfqpoint{2.401798in}{3.765454in}}%
\pgfpathlineto{\pgfqpoint{2.401798in}{3.765454in}}%
\pgfpathclose%
\pgfusepath{stroke,fill}%
\end{pgfscope}%
\begin{pgfscope}%
\pgfpathrectangle{\pgfqpoint{0.786164in}{0.768110in}}{\pgfqpoint{8.851069in}{7.081890in}}%
\pgfusepath{clip}%
\pgfsetbuttcap%
\pgfsetroundjoin%
\definecolor{currentfill}{rgb}{0.216210,0.351535,0.550627}%
\pgfsetfillcolor{currentfill}%
\pgfsetfillopacity{0.700000}%
\pgfsetlinewidth{0.501875pt}%
\definecolor{currentstroke}{rgb}{1.000000,1.000000,1.000000}%
\pgfsetstrokecolor{currentstroke}%
\pgfsetstrokeopacity{0.700000}%
\pgfsetdash{}{0pt}%
\pgfpathmoveto{\pgfqpoint{2.405828in}{3.744111in}}%
\pgfpathcurveto{\pgfqpoint{2.418850in}{3.744111in}}{\pgfqpoint{2.431341in}{3.749285in}}{\pgfqpoint{2.440550in}{3.758494in}}%
\pgfpathcurveto{\pgfqpoint{2.449758in}{3.767702in}}{\pgfqpoint{2.454932in}{3.780193in}}{\pgfqpoint{2.454932in}{3.793216in}}%
\pgfpathcurveto{\pgfqpoint{2.454932in}{3.806239in}}{\pgfqpoint{2.449758in}{3.818730in}}{\pgfqpoint{2.440550in}{3.827938in}}%
\pgfpathcurveto{\pgfqpoint{2.431341in}{3.837147in}}{\pgfqpoint{2.418850in}{3.842321in}}{\pgfqpoint{2.405828in}{3.842321in}}%
\pgfpathcurveto{\pgfqpoint{2.392805in}{3.842321in}}{\pgfqpoint{2.380314in}{3.837147in}}{\pgfqpoint{2.371105in}{3.827938in}}%
\pgfpathcurveto{\pgfqpoint{2.361897in}{3.818730in}}{\pgfqpoint{2.356723in}{3.806239in}}{\pgfqpoint{2.356723in}{3.793216in}}%
\pgfpathcurveto{\pgfqpoint{2.356723in}{3.780193in}}{\pgfqpoint{2.361897in}{3.767702in}}{\pgfqpoint{2.371105in}{3.758494in}}%
\pgfpathcurveto{\pgfqpoint{2.380314in}{3.749285in}}{\pgfqpoint{2.392805in}{3.744111in}}{\pgfqpoint{2.405828in}{3.744111in}}%
\pgfpathlineto{\pgfqpoint{2.405828in}{3.744111in}}%
\pgfpathclose%
\pgfusepath{stroke,fill}%
\end{pgfscope}%
\begin{pgfscope}%
\pgfpathrectangle{\pgfqpoint{0.786164in}{0.768110in}}{\pgfqpoint{8.851069in}{7.081890in}}%
\pgfusepath{clip}%
\pgfsetbuttcap%
\pgfsetroundjoin%
\definecolor{currentfill}{rgb}{0.229739,0.322361,0.545706}%
\pgfsetfillcolor{currentfill}%
\pgfsetfillopacity{0.700000}%
\pgfsetlinewidth{0.501875pt}%
\definecolor{currentstroke}{rgb}{1.000000,1.000000,1.000000}%
\pgfsetstrokecolor{currentstroke}%
\pgfsetstrokeopacity{0.700000}%
\pgfsetdash{}{0pt}%
\pgfpathmoveto{\pgfqpoint{2.500944in}{3.573374in}}%
\pgfpathcurveto{\pgfqpoint{2.513967in}{3.573374in}}{\pgfqpoint{2.526458in}{3.578548in}}{\pgfqpoint{2.535666in}{3.587756in}}%
\pgfpathcurveto{\pgfqpoint{2.544875in}{3.596965in}}{\pgfqpoint{2.550049in}{3.609456in}}{\pgfqpoint{2.550049in}{3.622479in}}%
\pgfpathcurveto{\pgfqpoint{2.550049in}{3.635501in}}{\pgfqpoint{2.544875in}{3.647992in}}{\pgfqpoint{2.535666in}{3.657201in}}%
\pgfpathcurveto{\pgfqpoint{2.526458in}{3.666409in}}{\pgfqpoint{2.513967in}{3.671583in}}{\pgfqpoint{2.500944in}{3.671583in}}%
\pgfpathcurveto{\pgfqpoint{2.487921in}{3.671583in}}{\pgfqpoint{2.475430in}{3.666409in}}{\pgfqpoint{2.466222in}{3.657201in}}%
\pgfpathcurveto{\pgfqpoint{2.457013in}{3.647992in}}{\pgfqpoint{2.451839in}{3.635501in}}{\pgfqpoint{2.451839in}{3.622479in}}%
\pgfpathcurveto{\pgfqpoint{2.451839in}{3.609456in}}{\pgfqpoint{2.457013in}{3.596965in}}{\pgfqpoint{2.466222in}{3.587756in}}%
\pgfpathcurveto{\pgfqpoint{2.475430in}{3.578548in}}{\pgfqpoint{2.487921in}{3.573374in}}{\pgfqpoint{2.500944in}{3.573374in}}%
\pgfpathlineto{\pgfqpoint{2.500944in}{3.573374in}}%
\pgfpathclose%
\pgfusepath{stroke,fill}%
\end{pgfscope}%
\begin{pgfscope}%
\pgfpathrectangle{\pgfqpoint{0.786164in}{0.768110in}}{\pgfqpoint{8.851069in}{7.081890in}}%
\pgfusepath{clip}%
\pgfsetbuttcap%
\pgfsetroundjoin%
\definecolor{currentfill}{rgb}{0.218130,0.347432,0.550038}%
\pgfsetfillcolor{currentfill}%
\pgfsetfillopacity{0.700000}%
\pgfsetlinewidth{0.501875pt}%
\definecolor{currentstroke}{rgb}{1.000000,1.000000,1.000000}%
\pgfsetstrokecolor{currentstroke}%
\pgfsetstrokeopacity{0.700000}%
\pgfsetdash{}{0pt}%
\pgfpathmoveto{\pgfqpoint{2.600578in}{3.658743in}}%
\pgfpathcurveto{\pgfqpoint{2.613601in}{3.658743in}}{\pgfqpoint{2.626092in}{3.663917in}}{\pgfqpoint{2.635300in}{3.673125in}}%
\pgfpathcurveto{\pgfqpoint{2.644509in}{3.682334in}}{\pgfqpoint{2.649683in}{3.694825in}}{\pgfqpoint{2.649683in}{3.707847in}}%
\pgfpathcurveto{\pgfqpoint{2.649683in}{3.720870in}}{\pgfqpoint{2.644509in}{3.733361in}}{\pgfqpoint{2.635300in}{3.742570in}}%
\pgfpathcurveto{\pgfqpoint{2.626092in}{3.751778in}}{\pgfqpoint{2.613601in}{3.756952in}}{\pgfqpoint{2.600578in}{3.756952in}}%
\pgfpathcurveto{\pgfqpoint{2.587555in}{3.756952in}}{\pgfqpoint{2.575064in}{3.751778in}}{\pgfqpoint{2.565856in}{3.742570in}}%
\pgfpathcurveto{\pgfqpoint{2.556647in}{3.733361in}}{\pgfqpoint{2.551473in}{3.720870in}}{\pgfqpoint{2.551473in}{3.707847in}}%
\pgfpathcurveto{\pgfqpoint{2.551473in}{3.694825in}}{\pgfqpoint{2.556647in}{3.682334in}}{\pgfqpoint{2.565856in}{3.673125in}}%
\pgfpathcurveto{\pgfqpoint{2.575064in}{3.663917in}}{\pgfqpoint{2.587555in}{3.658743in}}{\pgfqpoint{2.600578in}{3.658743in}}%
\pgfpathlineto{\pgfqpoint{2.600578in}{3.658743in}}%
\pgfpathclose%
\pgfusepath{stroke,fill}%
\end{pgfscope}%
\begin{pgfscope}%
\pgfpathrectangle{\pgfqpoint{0.786164in}{0.768110in}}{\pgfqpoint{8.851069in}{7.081890in}}%
\pgfusepath{clip}%
\pgfsetbuttcap%
\pgfsetroundjoin%
\definecolor{currentfill}{rgb}{0.227802,0.326594,0.546532}%
\pgfsetfillcolor{currentfill}%
\pgfsetfillopacity{0.700000}%
\pgfsetlinewidth{0.501875pt}%
\definecolor{currentstroke}{rgb}{1.000000,1.000000,1.000000}%
\pgfsetstrokecolor{currentstroke}%
\pgfsetstrokeopacity{0.700000}%
\pgfsetdash{}{0pt}%
\pgfpathmoveto{\pgfqpoint{2.698625in}{3.637401in}}%
\pgfpathcurveto{\pgfqpoint{2.711648in}{3.637401in}}{\pgfqpoint{2.724139in}{3.642575in}}{\pgfqpoint{2.733347in}{3.651783in}}%
\pgfpathcurveto{\pgfqpoint{2.742555in}{3.660991in}}{\pgfqpoint{2.747729in}{3.673483in}}{\pgfqpoint{2.747729in}{3.686505in}}%
\pgfpathcurveto{\pgfqpoint{2.747729in}{3.699528in}}{\pgfqpoint{2.742555in}{3.712019in}}{\pgfqpoint{2.733347in}{3.721227in}}%
\pgfpathcurveto{\pgfqpoint{2.724139in}{3.730436in}}{\pgfqpoint{2.711648in}{3.735610in}}{\pgfqpoint{2.698625in}{3.735610in}}%
\pgfpathcurveto{\pgfqpoint{2.685602in}{3.735610in}}{\pgfqpoint{2.673111in}{3.730436in}}{\pgfqpoint{2.663903in}{3.721227in}}%
\pgfpathcurveto{\pgfqpoint{2.654694in}{3.712019in}}{\pgfqpoint{2.649520in}{3.699528in}}{\pgfqpoint{2.649520in}{3.686505in}}%
\pgfpathcurveto{\pgfqpoint{2.649520in}{3.673483in}}{\pgfqpoint{2.654694in}{3.660991in}}{\pgfqpoint{2.663903in}{3.651783in}}%
\pgfpathcurveto{\pgfqpoint{2.673111in}{3.642575in}}{\pgfqpoint{2.685602in}{3.637401in}}{\pgfqpoint{2.698625in}{3.637401in}}%
\pgfpathlineto{\pgfqpoint{2.698625in}{3.637401in}}%
\pgfpathclose%
\pgfusepath{stroke,fill}%
\end{pgfscope}%
\begin{pgfscope}%
\pgfpathrectangle{\pgfqpoint{0.786164in}{0.768110in}}{\pgfqpoint{8.851069in}{7.081890in}}%
\pgfusepath{clip}%
\pgfsetbuttcap%
\pgfsetroundjoin%
\definecolor{currentfill}{rgb}{0.225863,0.330805,0.547314}%
\pgfsetfillcolor{currentfill}%
\pgfsetfillopacity{0.700000}%
\pgfsetlinewidth{0.501875pt}%
\definecolor{currentstroke}{rgb}{1.000000,1.000000,1.000000}%
\pgfsetstrokecolor{currentstroke}%
\pgfsetstrokeopacity{0.700000}%
\pgfsetdash{}{0pt}%
\pgfpathmoveto{\pgfqpoint{2.830371in}{3.637401in}}%
\pgfpathcurveto{\pgfqpoint{2.843394in}{3.637401in}}{\pgfqpoint{2.855885in}{3.642575in}}{\pgfqpoint{2.865094in}{3.651783in}}%
\pgfpathcurveto{\pgfqpoint{2.874302in}{3.660991in}}{\pgfqpoint{2.879476in}{3.673483in}}{\pgfqpoint{2.879476in}{3.686505in}}%
\pgfpathcurveto{\pgfqpoint{2.879476in}{3.699528in}}{\pgfqpoint{2.874302in}{3.712019in}}{\pgfqpoint{2.865094in}{3.721227in}}%
\pgfpathcurveto{\pgfqpoint{2.855885in}{3.730436in}}{\pgfqpoint{2.843394in}{3.735610in}}{\pgfqpoint{2.830371in}{3.735610in}}%
\pgfpathcurveto{\pgfqpoint{2.817349in}{3.735610in}}{\pgfqpoint{2.804858in}{3.730436in}}{\pgfqpoint{2.795649in}{3.721227in}}%
\pgfpathcurveto{\pgfqpoint{2.786441in}{3.712019in}}{\pgfqpoint{2.781267in}{3.699528in}}{\pgfqpoint{2.781267in}{3.686505in}}%
\pgfpathcurveto{\pgfqpoint{2.781267in}{3.673483in}}{\pgfqpoint{2.786441in}{3.660991in}}{\pgfqpoint{2.795649in}{3.651783in}}%
\pgfpathcurveto{\pgfqpoint{2.804858in}{3.642575in}}{\pgfqpoint{2.817349in}{3.637401in}}{\pgfqpoint{2.830371in}{3.637401in}}%
\pgfpathlineto{\pgfqpoint{2.830371in}{3.637401in}}%
\pgfpathclose%
\pgfusepath{stroke,fill}%
\end{pgfscope}%
\begin{pgfscope}%
\pgfpathrectangle{\pgfqpoint{0.786164in}{0.768110in}}{\pgfqpoint{8.851069in}{7.081890in}}%
\pgfusepath{clip}%
\pgfsetbuttcap%
\pgfsetroundjoin%
\definecolor{currentfill}{rgb}{0.220057,0.343307,0.549413}%
\pgfsetfillcolor{currentfill}%
\pgfsetfillopacity{0.700000}%
\pgfsetlinewidth{0.501875pt}%
\definecolor{currentstroke}{rgb}{1.000000,1.000000,1.000000}%
\pgfsetstrokecolor{currentstroke}%
\pgfsetstrokeopacity{0.700000}%
\pgfsetdash{}{0pt}%
\pgfpathmoveto{\pgfqpoint{2.855768in}{3.680085in}}%
\pgfpathcurveto{\pgfqpoint{2.868791in}{3.680085in}}{\pgfqpoint{2.881282in}{3.685259in}}{\pgfqpoint{2.890490in}{3.694467in}}%
\pgfpathcurveto{\pgfqpoint{2.899699in}{3.703676in}}{\pgfqpoint{2.904873in}{3.716167in}}{\pgfqpoint{2.904873in}{3.729190in}}%
\pgfpathcurveto{\pgfqpoint{2.904873in}{3.742212in}}{\pgfqpoint{2.899699in}{3.754703in}}{\pgfqpoint{2.890490in}{3.763912in}}%
\pgfpathcurveto{\pgfqpoint{2.881282in}{3.773120in}}{\pgfqpoint{2.868791in}{3.778294in}}{\pgfqpoint{2.855768in}{3.778294in}}%
\pgfpathcurveto{\pgfqpoint{2.842746in}{3.778294in}}{\pgfqpoint{2.830254in}{3.773120in}}{\pgfqpoint{2.821046in}{3.763912in}}%
\pgfpathcurveto{\pgfqpoint{2.811838in}{3.754703in}}{\pgfqpoint{2.806664in}{3.742212in}}{\pgfqpoint{2.806664in}{3.729190in}}%
\pgfpathcurveto{\pgfqpoint{2.806664in}{3.716167in}}{\pgfqpoint{2.811838in}{3.703676in}}{\pgfqpoint{2.821046in}{3.694467in}}%
\pgfpathcurveto{\pgfqpoint{2.830254in}{3.685259in}}{\pgfqpoint{2.842746in}{3.680085in}}{\pgfqpoint{2.855768in}{3.680085in}}%
\pgfpathlineto{\pgfqpoint{2.855768in}{3.680085in}}%
\pgfpathclose%
\pgfusepath{stroke,fill}%
\end{pgfscope}%
\begin{pgfscope}%
\pgfpathrectangle{\pgfqpoint{0.786164in}{0.768110in}}{\pgfqpoint{8.851069in}{7.081890in}}%
\pgfusepath{clip}%
\pgfsetbuttcap%
\pgfsetroundjoin%
\definecolor{currentfill}{rgb}{0.231674,0.318106,0.544834}%
\pgfsetfillcolor{currentfill}%
\pgfsetfillopacity{0.700000}%
\pgfsetlinewidth{0.501875pt}%
\definecolor{currentstroke}{rgb}{1.000000,1.000000,1.000000}%
\pgfsetstrokecolor{currentstroke}%
\pgfsetstrokeopacity{0.700000}%
\pgfsetdash{}{0pt}%
\pgfpathmoveto{\pgfqpoint{2.931959in}{3.573374in}}%
\pgfpathcurveto{\pgfqpoint{2.944982in}{3.573374in}}{\pgfqpoint{2.957473in}{3.578548in}}{\pgfqpoint{2.966681in}{3.587756in}}%
\pgfpathcurveto{\pgfqpoint{2.975890in}{3.596965in}}{\pgfqpoint{2.981064in}{3.609456in}}{\pgfqpoint{2.981064in}{3.622479in}}%
\pgfpathcurveto{\pgfqpoint{2.981064in}{3.635501in}}{\pgfqpoint{2.975890in}{3.647992in}}{\pgfqpoint{2.966681in}{3.657201in}}%
\pgfpathcurveto{\pgfqpoint{2.957473in}{3.666409in}}{\pgfqpoint{2.944982in}{3.671583in}}{\pgfqpoint{2.931959in}{3.671583in}}%
\pgfpathcurveto{\pgfqpoint{2.918936in}{3.671583in}}{\pgfqpoint{2.906445in}{3.666409in}}{\pgfqpoint{2.897237in}{3.657201in}}%
\pgfpathcurveto{\pgfqpoint{2.888028in}{3.647992in}}{\pgfqpoint{2.882854in}{3.635501in}}{\pgfqpoint{2.882854in}{3.622479in}}%
\pgfpathcurveto{\pgfqpoint{2.882854in}{3.609456in}}{\pgfqpoint{2.888028in}{3.596965in}}{\pgfqpoint{2.897237in}{3.587756in}}%
\pgfpathcurveto{\pgfqpoint{2.906445in}{3.578548in}}{\pgfqpoint{2.918936in}{3.573374in}}{\pgfqpoint{2.931959in}{3.573374in}}%
\pgfpathlineto{\pgfqpoint{2.931959in}{3.573374in}}%
\pgfpathclose%
\pgfusepath{stroke,fill}%
\end{pgfscope}%
\begin{pgfscope}%
\pgfpathrectangle{\pgfqpoint{0.786164in}{0.768110in}}{\pgfqpoint{8.851069in}{7.081890in}}%
\pgfusepath{clip}%
\pgfsetbuttcap%
\pgfsetroundjoin%
\definecolor{currentfill}{rgb}{0.231674,0.318106,0.544834}%
\pgfsetfillcolor{currentfill}%
\pgfsetfillopacity{0.700000}%
\pgfsetlinewidth{0.501875pt}%
\definecolor{currentstroke}{rgb}{1.000000,1.000000,1.000000}%
\pgfsetstrokecolor{currentstroke}%
\pgfsetstrokeopacity{0.700000}%
\pgfsetdash{}{0pt}%
\pgfpathmoveto{\pgfqpoint{2.995451in}{3.573374in}}%
\pgfpathcurveto{\pgfqpoint{3.008474in}{3.573374in}}{\pgfqpoint{3.020965in}{3.578548in}}{\pgfqpoint{3.030173in}{3.587756in}}%
\pgfpathcurveto{\pgfqpoint{3.039382in}{3.596965in}}{\pgfqpoint{3.044556in}{3.609456in}}{\pgfqpoint{3.044556in}{3.622479in}}%
\pgfpathcurveto{\pgfqpoint{3.044556in}{3.635501in}}{\pgfqpoint{3.039382in}{3.647992in}}{\pgfqpoint{3.030173in}{3.657201in}}%
\pgfpathcurveto{\pgfqpoint{3.020965in}{3.666409in}}{\pgfqpoint{3.008474in}{3.671583in}}{\pgfqpoint{2.995451in}{3.671583in}}%
\pgfpathcurveto{\pgfqpoint{2.982429in}{3.671583in}}{\pgfqpoint{2.969937in}{3.666409in}}{\pgfqpoint{2.960729in}{3.657201in}}%
\pgfpathcurveto{\pgfqpoint{2.951521in}{3.647992in}}{\pgfqpoint{2.946347in}{3.635501in}}{\pgfqpoint{2.946347in}{3.622479in}}%
\pgfpathcurveto{\pgfqpoint{2.946347in}{3.609456in}}{\pgfqpoint{2.951521in}{3.596965in}}{\pgfqpoint{2.960729in}{3.587756in}}%
\pgfpathcurveto{\pgfqpoint{2.969937in}{3.578548in}}{\pgfqpoint{2.982429in}{3.573374in}}{\pgfqpoint{2.995451in}{3.573374in}}%
\pgfpathlineto{\pgfqpoint{2.995451in}{3.573374in}}%
\pgfpathclose%
\pgfusepath{stroke,fill}%
\end{pgfscope}%
\begin{pgfscope}%
\pgfpathrectangle{\pgfqpoint{0.786164in}{0.768110in}}{\pgfqpoint{8.851069in}{7.081890in}}%
\pgfusepath{clip}%
\pgfsetbuttcap%
\pgfsetroundjoin%
\definecolor{currentfill}{rgb}{0.231674,0.318106,0.544834}%
\pgfsetfillcolor{currentfill}%
\pgfsetfillopacity{0.700000}%
\pgfsetlinewidth{0.501875pt}%
\definecolor{currentstroke}{rgb}{1.000000,1.000000,1.000000}%
\pgfsetstrokecolor{currentstroke}%
\pgfsetstrokeopacity{0.700000}%
\pgfsetdash{}{0pt}%
\pgfpathmoveto{\pgfqpoint{2.993498in}{3.552032in}}%
\pgfpathcurveto{\pgfqpoint{3.006520in}{3.552032in}}{\pgfqpoint{3.019011in}{3.557206in}}{\pgfqpoint{3.028220in}{3.566414in}}%
\pgfpathcurveto{\pgfqpoint{3.037428in}{3.575623in}}{\pgfqpoint{3.042602in}{3.588114in}}{\pgfqpoint{3.042602in}{3.601137in}}%
\pgfpathcurveto{\pgfqpoint{3.042602in}{3.614159in}}{\pgfqpoint{3.037428in}{3.626650in}}{\pgfqpoint{3.028220in}{3.635859in}}%
\pgfpathcurveto{\pgfqpoint{3.019011in}{3.645067in}}{\pgfqpoint{3.006520in}{3.650241in}}{\pgfqpoint{2.993498in}{3.650241in}}%
\pgfpathcurveto{\pgfqpoint{2.980475in}{3.650241in}}{\pgfqpoint{2.967984in}{3.645067in}}{\pgfqpoint{2.958775in}{3.635859in}}%
\pgfpathcurveto{\pgfqpoint{2.949567in}{3.626650in}}{\pgfqpoint{2.944393in}{3.614159in}}{\pgfqpoint{2.944393in}{3.601137in}}%
\pgfpathcurveto{\pgfqpoint{2.944393in}{3.588114in}}{\pgfqpoint{2.949567in}{3.575623in}}{\pgfqpoint{2.958775in}{3.566414in}}%
\pgfpathcurveto{\pgfqpoint{2.967984in}{3.557206in}}{\pgfqpoint{2.980475in}{3.552032in}}{\pgfqpoint{2.993498in}{3.552032in}}%
\pgfpathlineto{\pgfqpoint{2.993498in}{3.552032in}}%
\pgfpathclose%
\pgfusepath{stroke,fill}%
\end{pgfscope}%
\begin{pgfscope}%
\pgfpathrectangle{\pgfqpoint{0.786164in}{0.768110in}}{\pgfqpoint{8.851069in}{7.081890in}}%
\pgfusepath{clip}%
\pgfsetbuttcap%
\pgfsetroundjoin%
\definecolor{currentfill}{rgb}{0.231674,0.318106,0.544834}%
\pgfsetfillcolor{currentfill}%
\pgfsetfillopacity{0.700000}%
\pgfsetlinewidth{0.501875pt}%
\definecolor{currentstroke}{rgb}{1.000000,1.000000,1.000000}%
\pgfsetstrokecolor{currentstroke}%
\pgfsetstrokeopacity{0.700000}%
\pgfsetdash{}{0pt}%
\pgfpathmoveto{\pgfqpoint{3.065171in}{3.509348in}}%
\pgfpathcurveto{\pgfqpoint{3.078193in}{3.509348in}}{\pgfqpoint{3.090684in}{3.514522in}}{\pgfqpoint{3.099893in}{3.523730in}}%
\pgfpathcurveto{\pgfqpoint{3.109101in}{3.532938in}}{\pgfqpoint{3.114275in}{3.545429in}}{\pgfqpoint{3.114275in}{3.558452in}}%
\pgfpathcurveto{\pgfqpoint{3.114275in}{3.571475in}}{\pgfqpoint{3.109101in}{3.583966in}}{\pgfqpoint{3.099893in}{3.593174in}}%
\pgfpathcurveto{\pgfqpoint{3.090684in}{3.602383in}}{\pgfqpoint{3.078193in}{3.607557in}}{\pgfqpoint{3.065171in}{3.607557in}}%
\pgfpathcurveto{\pgfqpoint{3.052148in}{3.607557in}}{\pgfqpoint{3.039657in}{3.602383in}}{\pgfqpoint{3.030448in}{3.593174in}}%
\pgfpathcurveto{\pgfqpoint{3.021240in}{3.583966in}}{\pgfqpoint{3.016066in}{3.571475in}}{\pgfqpoint{3.016066in}{3.558452in}}%
\pgfpathcurveto{\pgfqpoint{3.016066in}{3.545429in}}{\pgfqpoint{3.021240in}{3.532938in}}{\pgfqpoint{3.030448in}{3.523730in}}%
\pgfpathcurveto{\pgfqpoint{3.039657in}{3.514522in}}{\pgfqpoint{3.052148in}{3.509348in}}{\pgfqpoint{3.065171in}{3.509348in}}%
\pgfpathlineto{\pgfqpoint{3.065171in}{3.509348in}}%
\pgfpathclose%
\pgfusepath{stroke,fill}%
\end{pgfscope}%
\begin{pgfscope}%
\pgfpathrectangle{\pgfqpoint{0.786164in}{0.768110in}}{\pgfqpoint{8.851069in}{7.081890in}}%
\pgfusepath{clip}%
\pgfsetbuttcap%
\pgfsetroundjoin%
\definecolor{currentfill}{rgb}{0.237441,0.305202,0.541921}%
\pgfsetfillcolor{currentfill}%
\pgfsetfillopacity{0.700000}%
\pgfsetlinewidth{0.501875pt}%
\definecolor{currentstroke}{rgb}{1.000000,1.000000,1.000000}%
\pgfsetstrokecolor{currentstroke}%
\pgfsetstrokeopacity{0.700000}%
\pgfsetdash{}{0pt}%
\pgfpathmoveto{\pgfqpoint{3.210226in}{3.445321in}}%
\pgfpathcurveto{\pgfqpoint{3.223249in}{3.445321in}}{\pgfqpoint{3.235740in}{3.450495in}}{\pgfqpoint{3.244948in}{3.459703in}}%
\pgfpathcurveto{\pgfqpoint{3.254157in}{3.468912in}}{\pgfqpoint{3.259331in}{3.481403in}}{\pgfqpoint{3.259331in}{3.494426in}}%
\pgfpathcurveto{\pgfqpoint{3.259331in}{3.507448in}}{\pgfqpoint{3.254157in}{3.519939in}}{\pgfqpoint{3.244948in}{3.529148in}}%
\pgfpathcurveto{\pgfqpoint{3.235740in}{3.538356in}}{\pgfqpoint{3.223249in}{3.543530in}}{\pgfqpoint{3.210226in}{3.543530in}}%
\pgfpathcurveto{\pgfqpoint{3.197203in}{3.543530in}}{\pgfqpoint{3.184712in}{3.538356in}}{\pgfqpoint{3.175504in}{3.529148in}}%
\pgfpathcurveto{\pgfqpoint{3.166295in}{3.519939in}}{\pgfqpoint{3.161121in}{3.507448in}}{\pgfqpoint{3.161121in}{3.494426in}}%
\pgfpathcurveto{\pgfqpoint{3.161121in}{3.481403in}}{\pgfqpoint{3.166295in}{3.468912in}}{\pgfqpoint{3.175504in}{3.459703in}}%
\pgfpathcurveto{\pgfqpoint{3.184712in}{3.450495in}}{\pgfqpoint{3.197203in}{3.445321in}}{\pgfqpoint{3.210226in}{3.445321in}}%
\pgfpathlineto{\pgfqpoint{3.210226in}{3.445321in}}%
\pgfpathclose%
\pgfusepath{stroke,fill}%
\end{pgfscope}%
\begin{pgfscope}%
\pgfpathrectangle{\pgfqpoint{0.786164in}{0.768110in}}{\pgfqpoint{8.851069in}{7.081890in}}%
\pgfusepath{clip}%
\pgfsetbuttcap%
\pgfsetroundjoin%
\definecolor{currentfill}{rgb}{0.243113,0.292092,0.538516}%
\pgfsetfillcolor{currentfill}%
\pgfsetfillopacity{0.700000}%
\pgfsetlinewidth{0.501875pt}%
\definecolor{currentstroke}{rgb}{1.000000,1.000000,1.000000}%
\pgfsetstrokecolor{currentstroke}%
\pgfsetstrokeopacity{0.700000}%
\pgfsetdash{}{0pt}%
\pgfpathmoveto{\pgfqpoint{3.284219in}{3.295926in}}%
\pgfpathcurveto{\pgfqpoint{3.297242in}{3.295926in}}{\pgfqpoint{3.309733in}{3.301100in}}{\pgfqpoint{3.318941in}{3.310308in}}%
\pgfpathcurveto{\pgfqpoint{3.328150in}{3.319517in}}{\pgfqpoint{3.333324in}{3.332008in}}{\pgfqpoint{3.333324in}{3.345030in}}%
\pgfpathcurveto{\pgfqpoint{3.333324in}{3.358053in}}{\pgfqpoint{3.328150in}{3.370544in}}{\pgfqpoint{3.318941in}{3.379753in}}%
\pgfpathcurveto{\pgfqpoint{3.309733in}{3.388961in}}{\pgfqpoint{3.297242in}{3.394135in}}{\pgfqpoint{3.284219in}{3.394135in}}%
\pgfpathcurveto{\pgfqpoint{3.271196in}{3.394135in}}{\pgfqpoint{3.258705in}{3.388961in}}{\pgfqpoint{3.249497in}{3.379753in}}%
\pgfpathcurveto{\pgfqpoint{3.240288in}{3.370544in}}{\pgfqpoint{3.235114in}{3.358053in}}{\pgfqpoint{3.235114in}{3.345030in}}%
\pgfpathcurveto{\pgfqpoint{3.235114in}{3.332008in}}{\pgfqpoint{3.240288in}{3.319517in}}{\pgfqpoint{3.249497in}{3.310308in}}%
\pgfpathcurveto{\pgfqpoint{3.258705in}{3.301100in}}{\pgfqpoint{3.271196in}{3.295926in}}{\pgfqpoint{3.284219in}{3.295926in}}%
\pgfpathlineto{\pgfqpoint{3.284219in}{3.295926in}}%
\pgfpathclose%
\pgfusepath{stroke,fill}%
\end{pgfscope}%
\begin{pgfscope}%
\pgfpathrectangle{\pgfqpoint{0.786164in}{0.768110in}}{\pgfqpoint{8.851069in}{7.081890in}}%
\pgfusepath{clip}%
\pgfsetbuttcap%
\pgfsetroundjoin%
\definecolor{currentfill}{rgb}{0.257322,0.256130,0.526563}%
\pgfsetfillcolor{currentfill}%
\pgfsetfillopacity{0.700000}%
\pgfsetlinewidth{0.501875pt}%
\definecolor{currentstroke}{rgb}{1.000000,1.000000,1.000000}%
\pgfsetstrokecolor{currentstroke}%
\pgfsetstrokeopacity{0.700000}%
\pgfsetdash{}{0pt}%
\pgfpathmoveto{\pgfqpoint{3.506930in}{3.103846in}}%
\pgfpathcurveto{\pgfqpoint{3.519953in}{3.103846in}}{\pgfqpoint{3.532444in}{3.109020in}}{\pgfqpoint{3.541653in}{3.118229in}}%
\pgfpathcurveto{\pgfqpoint{3.550861in}{3.127437in}}{\pgfqpoint{3.556035in}{3.139928in}}{\pgfqpoint{3.556035in}{3.152951in}}%
\pgfpathcurveto{\pgfqpoint{3.556035in}{3.165974in}}{\pgfqpoint{3.550861in}{3.178465in}}{\pgfqpoint{3.541653in}{3.187673in}}%
\pgfpathcurveto{\pgfqpoint{3.532444in}{3.196882in}}{\pgfqpoint{3.519953in}{3.202056in}}{\pgfqpoint{3.506930in}{3.202056in}}%
\pgfpathcurveto{\pgfqpoint{3.493908in}{3.202056in}}{\pgfqpoint{3.481417in}{3.196882in}}{\pgfqpoint{3.472208in}{3.187673in}}%
\pgfpathcurveto{\pgfqpoint{3.463000in}{3.178465in}}{\pgfqpoint{3.457826in}{3.165974in}}{\pgfqpoint{3.457826in}{3.152951in}}%
\pgfpathcurveto{\pgfqpoint{3.457826in}{3.139928in}}{\pgfqpoint{3.463000in}{3.127437in}}{\pgfqpoint{3.472208in}{3.118229in}}%
\pgfpathcurveto{\pgfqpoint{3.481417in}{3.109020in}}{\pgfqpoint{3.493908in}{3.103846in}}{\pgfqpoint{3.506930in}{3.103846in}}%
\pgfpathlineto{\pgfqpoint{3.506930in}{3.103846in}}%
\pgfpathclose%
\pgfusepath{stroke,fill}%
\end{pgfscope}%
\begin{pgfscope}%
\pgfpathrectangle{\pgfqpoint{0.786164in}{0.768110in}}{\pgfqpoint{8.851069in}{7.081890in}}%
\pgfusepath{clip}%
\pgfsetbuttcap%
\pgfsetroundjoin%
\definecolor{currentfill}{rgb}{0.252194,0.269783,0.531579}%
\pgfsetfillcolor{currentfill}%
\pgfsetfillopacity{0.700000}%
\pgfsetlinewidth{0.501875pt}%
\definecolor{currentstroke}{rgb}{1.000000,1.000000,1.000000}%
\pgfsetstrokecolor{currentstroke}%
\pgfsetstrokeopacity{0.700000}%
\pgfsetdash{}{0pt}%
\pgfpathmoveto{\pgfqpoint{3.544171in}{3.189215in}}%
\pgfpathcurveto{\pgfqpoint{3.557194in}{3.189215in}}{\pgfqpoint{3.569685in}{3.194389in}}{\pgfqpoint{3.578893in}{3.203597in}}%
\pgfpathcurveto{\pgfqpoint{3.588102in}{3.212806in}}{\pgfqpoint{3.593276in}{3.225297in}}{\pgfqpoint{3.593276in}{3.238320in}}%
\pgfpathcurveto{\pgfqpoint{3.593276in}{3.251342in}}{\pgfqpoint{3.588102in}{3.263833in}}{\pgfqpoint{3.578893in}{3.273042in}}%
\pgfpathcurveto{\pgfqpoint{3.569685in}{3.282250in}}{\pgfqpoint{3.557194in}{3.287424in}}{\pgfqpoint{3.544171in}{3.287424in}}%
\pgfpathcurveto{\pgfqpoint{3.531148in}{3.287424in}}{\pgfqpoint{3.518657in}{3.282250in}}{\pgfqpoint{3.509449in}{3.273042in}}%
\pgfpathcurveto{\pgfqpoint{3.500240in}{3.263833in}}{\pgfqpoint{3.495066in}{3.251342in}}{\pgfqpoint{3.495066in}{3.238320in}}%
\pgfpathcurveto{\pgfqpoint{3.495066in}{3.225297in}}{\pgfqpoint{3.500240in}{3.212806in}}{\pgfqpoint{3.509449in}{3.203597in}}%
\pgfpathcurveto{\pgfqpoint{3.518657in}{3.194389in}}{\pgfqpoint{3.531148in}{3.189215in}}{\pgfqpoint{3.544171in}{3.189215in}}%
\pgfpathlineto{\pgfqpoint{3.544171in}{3.189215in}}%
\pgfpathclose%
\pgfusepath{stroke,fill}%
\end{pgfscope}%
\begin{pgfscope}%
\pgfpathrectangle{\pgfqpoint{0.786164in}{0.768110in}}{\pgfqpoint{8.851069in}{7.081890in}}%
\pgfusepath{clip}%
\pgfsetbuttcap%
\pgfsetroundjoin%
\definecolor{currentfill}{rgb}{0.258965,0.251537,0.524736}%
\pgfsetfillcolor{currentfill}%
\pgfsetfillopacity{0.700000}%
\pgfsetlinewidth{0.501875pt}%
\definecolor{currentstroke}{rgb}{1.000000,1.000000,1.000000}%
\pgfsetstrokecolor{currentstroke}%
\pgfsetstrokeopacity{0.700000}%
\pgfsetdash{}{0pt}%
\pgfpathmoveto{\pgfqpoint{3.715234in}{3.167873in}}%
\pgfpathcurveto{\pgfqpoint{3.728257in}{3.167873in}}{\pgfqpoint{3.740748in}{3.173047in}}{\pgfqpoint{3.749956in}{3.182255in}}%
\pgfpathcurveto{\pgfqpoint{3.759165in}{3.191464in}}{\pgfqpoint{3.764339in}{3.203955in}}{\pgfqpoint{3.764339in}{3.216977in}}%
\pgfpathcurveto{\pgfqpoint{3.764339in}{3.230000in}}{\pgfqpoint{3.759165in}{3.242491in}}{\pgfqpoint{3.749956in}{3.251700in}}%
\pgfpathcurveto{\pgfqpoint{3.740748in}{3.260908in}}{\pgfqpoint{3.728257in}{3.266082in}}{\pgfqpoint{3.715234in}{3.266082in}}%
\pgfpathcurveto{\pgfqpoint{3.702211in}{3.266082in}}{\pgfqpoint{3.689720in}{3.260908in}}{\pgfqpoint{3.680512in}{3.251700in}}%
\pgfpathcurveto{\pgfqpoint{3.671303in}{3.242491in}}{\pgfqpoint{3.666129in}{3.230000in}}{\pgfqpoint{3.666129in}{3.216977in}}%
\pgfpathcurveto{\pgfqpoint{3.666129in}{3.203955in}}{\pgfqpoint{3.671303in}{3.191464in}}{\pgfqpoint{3.680512in}{3.182255in}}%
\pgfpathcurveto{\pgfqpoint{3.689720in}{3.173047in}}{\pgfqpoint{3.702211in}{3.167873in}}{\pgfqpoint{3.715234in}{3.167873in}}%
\pgfpathlineto{\pgfqpoint{3.715234in}{3.167873in}}%
\pgfpathclose%
\pgfusepath{stroke,fill}%
\end{pgfscope}%
\begin{pgfscope}%
\pgfpathrectangle{\pgfqpoint{0.786164in}{0.768110in}}{\pgfqpoint{8.851069in}{7.081890in}}%
\pgfusepath{clip}%
\pgfsetbuttcap%
\pgfsetroundjoin%
\definecolor{currentfill}{rgb}{0.227802,0.326594,0.546532}%
\pgfsetfillcolor{currentfill}%
\pgfsetfillopacity{0.700000}%
\pgfsetlinewidth{0.501875pt}%
\definecolor{currentstroke}{rgb}{1.000000,1.000000,1.000000}%
\pgfsetstrokecolor{currentstroke}%
\pgfsetstrokeopacity{0.700000}%
\pgfsetdash{}{0pt}%
\pgfpathmoveto{\pgfqpoint{2.987881in}{4.341692in}}%
\pgfpathcurveto{\pgfqpoint{3.000904in}{4.341692in}}{\pgfqpoint{3.013395in}{4.346866in}}{\pgfqpoint{3.022603in}{4.356075in}}%
\pgfpathcurveto{\pgfqpoint{3.031812in}{4.365283in}}{\pgfqpoint{3.036986in}{4.377774in}}{\pgfqpoint{3.036986in}{4.390797in}}%
\pgfpathcurveto{\pgfqpoint{3.036986in}{4.403820in}}{\pgfqpoint{3.031812in}{4.416311in}}{\pgfqpoint{3.022603in}{4.425519in}}%
\pgfpathcurveto{\pgfqpoint{3.013395in}{4.434728in}}{\pgfqpoint{3.000904in}{4.439901in}}{\pgfqpoint{2.987881in}{4.439901in}}%
\pgfpathcurveto{\pgfqpoint{2.974858in}{4.439901in}}{\pgfqpoint{2.962367in}{4.434728in}}{\pgfqpoint{2.953159in}{4.425519in}}%
\pgfpathcurveto{\pgfqpoint{2.943950in}{4.416311in}}{\pgfqpoint{2.938776in}{4.403820in}}{\pgfqpoint{2.938776in}{4.390797in}}%
\pgfpathcurveto{\pgfqpoint{2.938776in}{4.377774in}}{\pgfqpoint{2.943950in}{4.365283in}}{\pgfqpoint{2.953159in}{4.356075in}}%
\pgfpathcurveto{\pgfqpoint{2.962367in}{4.346866in}}{\pgfqpoint{2.974858in}{4.341692in}}{\pgfqpoint{2.987881in}{4.341692in}}%
\pgfpathlineto{\pgfqpoint{2.987881in}{4.341692in}}%
\pgfpathclose%
\pgfusepath{stroke,fill}%
\end{pgfscope}%
\begin{pgfscope}%
\pgfpathrectangle{\pgfqpoint{0.786164in}{0.768110in}}{\pgfqpoint{8.851069in}{7.081890in}}%
\pgfusepath{clip}%
\pgfsetbuttcap%
\pgfsetroundjoin%
\definecolor{currentfill}{rgb}{0.233603,0.313828,0.543914}%
\pgfsetfillcolor{currentfill}%
\pgfsetfillopacity{0.700000}%
\pgfsetlinewidth{0.501875pt}%
\definecolor{currentstroke}{rgb}{1.000000,1.000000,1.000000}%
\pgfsetstrokecolor{currentstroke}%
\pgfsetstrokeopacity{0.700000}%
\pgfsetdash{}{0pt}%
\pgfpathmoveto{\pgfqpoint{3.124145in}{4.149613in}}%
\pgfpathcurveto{\pgfqpoint{3.137168in}{4.149613in}}{\pgfqpoint{3.149659in}{4.154787in}}{\pgfqpoint{3.158867in}{4.163995in}}%
\pgfpathcurveto{\pgfqpoint{3.168076in}{4.173204in}}{\pgfqpoint{3.173250in}{4.185695in}}{\pgfqpoint{3.173250in}{4.198717in}}%
\pgfpathcurveto{\pgfqpoint{3.173250in}{4.211740in}}{\pgfqpoint{3.168076in}{4.224231in}}{\pgfqpoint{3.158867in}{4.233440in}}%
\pgfpathcurveto{\pgfqpoint{3.149659in}{4.242648in}}{\pgfqpoint{3.137168in}{4.247822in}}{\pgfqpoint{3.124145in}{4.247822in}}%
\pgfpathcurveto{\pgfqpoint{3.111123in}{4.247822in}}{\pgfqpoint{3.098631in}{4.242648in}}{\pgfqpoint{3.089423in}{4.233440in}}%
\pgfpathcurveto{\pgfqpoint{3.080215in}{4.224231in}}{\pgfqpoint{3.075041in}{4.211740in}}{\pgfqpoint{3.075041in}{4.198717in}}%
\pgfpathcurveto{\pgfqpoint{3.075041in}{4.185695in}}{\pgfqpoint{3.080215in}{4.173204in}}{\pgfqpoint{3.089423in}{4.163995in}}%
\pgfpathcurveto{\pgfqpoint{3.098631in}{4.154787in}}{\pgfqpoint{3.111123in}{4.149613in}}{\pgfqpoint{3.124145in}{4.149613in}}%
\pgfpathlineto{\pgfqpoint{3.124145in}{4.149613in}}%
\pgfpathclose%
\pgfusepath{stroke,fill}%
\end{pgfscope}%
\begin{pgfscope}%
\pgfpathrectangle{\pgfqpoint{0.786164in}{0.768110in}}{\pgfqpoint{8.851069in}{7.081890in}}%
\pgfusepath{clip}%
\pgfsetbuttcap%
\pgfsetroundjoin%
\definecolor{currentfill}{rgb}{0.220057,0.343307,0.549413}%
\pgfsetfillcolor{currentfill}%
\pgfsetfillopacity{0.700000}%
\pgfsetlinewidth{0.501875pt}%
\definecolor{currentstroke}{rgb}{1.000000,1.000000,1.000000}%
\pgfsetstrokecolor{currentstroke}%
\pgfsetstrokeopacity{0.700000}%
\pgfsetdash{}{0pt}%
\pgfpathmoveto{\pgfqpoint{3.170177in}{4.469745in}}%
\pgfpathcurveto{\pgfqpoint{3.183200in}{4.469745in}}{\pgfqpoint{3.195691in}{4.474919in}}{\pgfqpoint{3.204899in}{4.484128in}}%
\pgfpathcurveto{\pgfqpoint{3.214108in}{4.493336in}}{\pgfqpoint{3.219282in}{4.505827in}}{\pgfqpoint{3.219282in}{4.518850in}}%
\pgfpathcurveto{\pgfqpoint{3.219282in}{4.531873in}}{\pgfqpoint{3.214108in}{4.544364in}}{\pgfqpoint{3.204899in}{4.553572in}}%
\pgfpathcurveto{\pgfqpoint{3.195691in}{4.562781in}}{\pgfqpoint{3.183200in}{4.567955in}}{\pgfqpoint{3.170177in}{4.567955in}}%
\pgfpathcurveto{\pgfqpoint{3.157154in}{4.567955in}}{\pgfqpoint{3.144663in}{4.562781in}}{\pgfqpoint{3.135455in}{4.553572in}}%
\pgfpathcurveto{\pgfqpoint{3.126246in}{4.544364in}}{\pgfqpoint{3.121072in}{4.531873in}}{\pgfqpoint{3.121072in}{4.518850in}}%
\pgfpathcurveto{\pgfqpoint{3.121072in}{4.505827in}}{\pgfqpoint{3.126246in}{4.493336in}}{\pgfqpoint{3.135455in}{4.484128in}}%
\pgfpathcurveto{\pgfqpoint{3.144663in}{4.474919in}}{\pgfqpoint{3.157154in}{4.469745in}}{\pgfqpoint{3.170177in}{4.469745in}}%
\pgfpathlineto{\pgfqpoint{3.170177in}{4.469745in}}%
\pgfpathclose%
\pgfusepath{stroke,fill}%
\end{pgfscope}%
\begin{pgfscope}%
\pgfpathrectangle{\pgfqpoint{0.786164in}{0.768110in}}{\pgfqpoint{8.851069in}{7.081890in}}%
\pgfusepath{clip}%
\pgfsetbuttcap%
\pgfsetroundjoin%
\definecolor{currentfill}{rgb}{0.225863,0.330805,0.547314}%
\pgfsetfillcolor{currentfill}%
\pgfsetfillopacity{0.700000}%
\pgfsetlinewidth{0.501875pt}%
\definecolor{currentstroke}{rgb}{1.000000,1.000000,1.000000}%
\pgfsetstrokecolor{currentstroke}%
\pgfsetstrokeopacity{0.700000}%
\pgfsetdash{}{0pt}%
\pgfpathmoveto{\pgfqpoint{3.342949in}{4.277666in}}%
\pgfpathcurveto{\pgfqpoint{3.355972in}{4.277666in}}{\pgfqpoint{3.368463in}{4.282840in}}{\pgfqpoint{3.377672in}{4.292048in}}%
\pgfpathcurveto{\pgfqpoint{3.386880in}{4.301257in}}{\pgfqpoint{3.392054in}{4.313748in}}{\pgfqpoint{3.392054in}{4.326770in}}%
\pgfpathcurveto{\pgfqpoint{3.392054in}{4.339793in}}{\pgfqpoint{3.386880in}{4.352284in}}{\pgfqpoint{3.377672in}{4.361493in}}%
\pgfpathcurveto{\pgfqpoint{3.368463in}{4.370701in}}{\pgfqpoint{3.355972in}{4.375875in}}{\pgfqpoint{3.342949in}{4.375875in}}%
\pgfpathcurveto{\pgfqpoint{3.329927in}{4.375875in}}{\pgfqpoint{3.317436in}{4.370701in}}{\pgfqpoint{3.308227in}{4.361493in}}%
\pgfpathcurveto{\pgfqpoint{3.299019in}{4.352284in}}{\pgfqpoint{3.293845in}{4.339793in}}{\pgfqpoint{3.293845in}{4.326770in}}%
\pgfpathcurveto{\pgfqpoint{3.293845in}{4.313748in}}{\pgfqpoint{3.299019in}{4.301257in}}{\pgfqpoint{3.308227in}{4.292048in}}%
\pgfpathcurveto{\pgfqpoint{3.317436in}{4.282840in}}{\pgfqpoint{3.329927in}{4.277666in}}{\pgfqpoint{3.342949in}{4.277666in}}%
\pgfpathlineto{\pgfqpoint{3.342949in}{4.277666in}}%
\pgfpathclose%
\pgfusepath{stroke,fill}%
\end{pgfscope}%
\begin{pgfscope}%
\pgfpathrectangle{\pgfqpoint{0.786164in}{0.768110in}}{\pgfqpoint{8.851069in}{7.081890in}}%
\pgfusepath{clip}%
\pgfsetbuttcap%
\pgfsetroundjoin%
\definecolor{currentfill}{rgb}{0.231674,0.318106,0.544834}%
\pgfsetfillcolor{currentfill}%
\pgfsetfillopacity{0.700000}%
\pgfsetlinewidth{0.501875pt}%
\definecolor{currentstroke}{rgb}{1.000000,1.000000,1.000000}%
\pgfsetstrokecolor{currentstroke}%
\pgfsetstrokeopacity{0.700000}%
\pgfsetdash{}{0pt}%
\pgfpathmoveto{\pgfqpoint{3.440141in}{4.064244in}}%
\pgfpathcurveto{\pgfqpoint{3.453164in}{4.064244in}}{\pgfqpoint{3.465655in}{4.069418in}}{\pgfqpoint{3.474864in}{4.078626in}}%
\pgfpathcurveto{\pgfqpoint{3.484072in}{4.087835in}}{\pgfqpoint{3.489246in}{4.100326in}}{\pgfqpoint{3.489246in}{4.113349in}}%
\pgfpathcurveto{\pgfqpoint{3.489246in}{4.126371in}}{\pgfqpoint{3.484072in}{4.138862in}}{\pgfqpoint{3.474864in}{4.148071in}}%
\pgfpathcurveto{\pgfqpoint{3.465655in}{4.157279in}}{\pgfqpoint{3.453164in}{4.162453in}}{\pgfqpoint{3.440141in}{4.162453in}}%
\pgfpathcurveto{\pgfqpoint{3.427119in}{4.162453in}}{\pgfqpoint{3.414628in}{4.157279in}}{\pgfqpoint{3.405419in}{4.148071in}}%
\pgfpathcurveto{\pgfqpoint{3.396211in}{4.138862in}}{\pgfqpoint{3.391037in}{4.126371in}}{\pgfqpoint{3.391037in}{4.113349in}}%
\pgfpathcurveto{\pgfqpoint{3.391037in}{4.100326in}}{\pgfqpoint{3.396211in}{4.087835in}}{\pgfqpoint{3.405419in}{4.078626in}}%
\pgfpathcurveto{\pgfqpoint{3.414628in}{4.069418in}}{\pgfqpoint{3.427119in}{4.064244in}}{\pgfqpoint{3.440141in}{4.064244in}}%
\pgfpathlineto{\pgfqpoint{3.440141in}{4.064244in}}%
\pgfpathclose%
\pgfusepath{stroke,fill}%
\end{pgfscope}%
\begin{pgfscope}%
\pgfpathrectangle{\pgfqpoint{0.786164in}{0.768110in}}{\pgfqpoint{8.851069in}{7.081890in}}%
\pgfusepath{clip}%
\pgfsetbuttcap%
\pgfsetroundjoin%
\definecolor{currentfill}{rgb}{0.241237,0.296485,0.539709}%
\pgfsetfillcolor{currentfill}%
\pgfsetfillopacity{0.700000}%
\pgfsetlinewidth{0.501875pt}%
\definecolor{currentstroke}{rgb}{1.000000,1.000000,1.000000}%
\pgfsetstrokecolor{currentstroke}%
\pgfsetstrokeopacity{0.700000}%
\pgfsetdash{}{0pt}%
\pgfpathmoveto{\pgfqpoint{3.611693in}{3.893507in}}%
\pgfpathcurveto{\pgfqpoint{3.624715in}{3.893507in}}{\pgfqpoint{3.637207in}{3.898681in}}{\pgfqpoint{3.646415in}{3.907889in}}%
\pgfpathcurveto{\pgfqpoint{3.655623in}{3.917097in}}{\pgfqpoint{3.660797in}{3.929589in}}{\pgfqpoint{3.660797in}{3.942611in}}%
\pgfpathcurveto{\pgfqpoint{3.660797in}{3.955634in}}{\pgfqpoint{3.655623in}{3.968125in}}{\pgfqpoint{3.646415in}{3.977333in}}%
\pgfpathcurveto{\pgfqpoint{3.637207in}{3.986542in}}{\pgfqpoint{3.624715in}{3.991716in}}{\pgfqpoint{3.611693in}{3.991716in}}%
\pgfpathcurveto{\pgfqpoint{3.598670in}{3.991716in}}{\pgfqpoint{3.586179in}{3.986542in}}{\pgfqpoint{3.576971in}{3.977333in}}%
\pgfpathcurveto{\pgfqpoint{3.567762in}{3.968125in}}{\pgfqpoint{3.562588in}{3.955634in}}{\pgfqpoint{3.562588in}{3.942611in}}%
\pgfpathcurveto{\pgfqpoint{3.562588in}{3.929589in}}{\pgfqpoint{3.567762in}{3.917097in}}{\pgfqpoint{3.576971in}{3.907889in}}%
\pgfpathcurveto{\pgfqpoint{3.586179in}{3.898681in}}{\pgfqpoint{3.598670in}{3.893507in}}{\pgfqpoint{3.611693in}{3.893507in}}%
\pgfpathlineto{\pgfqpoint{3.611693in}{3.893507in}}%
\pgfpathclose%
\pgfusepath{stroke,fill}%
\end{pgfscope}%
\begin{pgfscope}%
\pgfpathrectangle{\pgfqpoint{0.786164in}{0.768110in}}{\pgfqpoint{8.851069in}{7.081890in}}%
\pgfusepath{clip}%
\pgfsetbuttcap%
\pgfsetroundjoin%
\definecolor{currentfill}{rgb}{0.233603,0.313828,0.543914}%
\pgfsetfillcolor{currentfill}%
\pgfsetfillopacity{0.700000}%
\pgfsetlinewidth{0.501875pt}%
\definecolor{currentstroke}{rgb}{1.000000,1.000000,1.000000}%
\pgfsetstrokecolor{currentstroke}%
\pgfsetstrokeopacity{0.700000}%
\pgfsetdash{}{0pt}%
\pgfpathmoveto{\pgfqpoint{3.848568in}{3.850822in}}%
\pgfpathcurveto{\pgfqpoint{3.861590in}{3.850822in}}{\pgfqpoint{3.874082in}{3.855996in}}{\pgfqpoint{3.883290in}{3.865205in}}%
\pgfpathcurveto{\pgfqpoint{3.892498in}{3.874413in}}{\pgfqpoint{3.897672in}{3.886904in}}{\pgfqpoint{3.897672in}{3.899927in}}%
\pgfpathcurveto{\pgfqpoint{3.897672in}{3.912950in}}{\pgfqpoint{3.892498in}{3.925441in}}{\pgfqpoint{3.883290in}{3.934649in}}%
\pgfpathcurveto{\pgfqpoint{3.874082in}{3.943858in}}{\pgfqpoint{3.861590in}{3.949032in}}{\pgfqpoint{3.848568in}{3.949032in}}%
\pgfpathcurveto{\pgfqpoint{3.835545in}{3.949032in}}{\pgfqpoint{3.823054in}{3.943858in}}{\pgfqpoint{3.813846in}{3.934649in}}%
\pgfpathcurveto{\pgfqpoint{3.804637in}{3.925441in}}{\pgfqpoint{3.799463in}{3.912950in}}{\pgfqpoint{3.799463in}{3.899927in}}%
\pgfpathcurveto{\pgfqpoint{3.799463in}{3.886904in}}{\pgfqpoint{3.804637in}{3.874413in}}{\pgfqpoint{3.813846in}{3.865205in}}%
\pgfpathcurveto{\pgfqpoint{3.823054in}{3.855996in}}{\pgfqpoint{3.835545in}{3.850822in}}{\pgfqpoint{3.848568in}{3.850822in}}%
\pgfpathlineto{\pgfqpoint{3.848568in}{3.850822in}}%
\pgfpathclose%
\pgfusepath{stroke,fill}%
\end{pgfscope}%
\begin{pgfscope}%
\pgfpathrectangle{\pgfqpoint{0.786164in}{0.768110in}}{\pgfqpoint{8.851069in}{7.081890in}}%
\pgfusepath{clip}%
\pgfsetbuttcap%
\pgfsetroundjoin%
\definecolor{currentfill}{rgb}{0.248629,0.278775,0.534556}%
\pgfsetfillcolor{currentfill}%
\pgfsetfillopacity{0.700000}%
\pgfsetlinewidth{0.501875pt}%
\definecolor{currentstroke}{rgb}{1.000000,1.000000,1.000000}%
\pgfsetstrokecolor{currentstroke}%
\pgfsetstrokeopacity{0.700000}%
\pgfsetdash{}{0pt}%
\pgfpathmoveto{\pgfqpoint{4.031841in}{3.616058in}}%
\pgfpathcurveto{\pgfqpoint{4.044863in}{3.616058in}}{\pgfqpoint{4.057354in}{3.621232in}}{\pgfqpoint{4.066563in}{3.630441in}}%
\pgfpathcurveto{\pgfqpoint{4.075771in}{3.639649in}}{\pgfqpoint{4.080945in}{3.652140in}}{\pgfqpoint{4.080945in}{3.665163in}}%
\pgfpathcurveto{\pgfqpoint{4.080945in}{3.678186in}}{\pgfqpoint{4.075771in}{3.690677in}}{\pgfqpoint{4.066563in}{3.699885in}}%
\pgfpathcurveto{\pgfqpoint{4.057354in}{3.709094in}}{\pgfqpoint{4.044863in}{3.714268in}}{\pgfqpoint{4.031841in}{3.714268in}}%
\pgfpathcurveto{\pgfqpoint{4.018818in}{3.714268in}}{\pgfqpoint{4.006327in}{3.709094in}}{\pgfqpoint{3.997118in}{3.699885in}}%
\pgfpathcurveto{\pgfqpoint{3.987910in}{3.690677in}}{\pgfqpoint{3.982736in}{3.678186in}}{\pgfqpoint{3.982736in}{3.665163in}}%
\pgfpathcurveto{\pgfqpoint{3.982736in}{3.652140in}}{\pgfqpoint{3.987910in}{3.639649in}}{\pgfqpoint{3.997118in}{3.630441in}}%
\pgfpathcurveto{\pgfqpoint{4.006327in}{3.621232in}}{\pgfqpoint{4.018818in}{3.616058in}}{\pgfqpoint{4.031841in}{3.616058in}}%
\pgfpathlineto{\pgfqpoint{4.031841in}{3.616058in}}%
\pgfpathclose%
\pgfusepath{stroke,fill}%
\end{pgfscope}%
\begin{pgfscope}%
\pgfpathrectangle{\pgfqpoint{0.786164in}{0.768110in}}{\pgfqpoint{8.851069in}{7.081890in}}%
\pgfusepath{clip}%
\pgfsetbuttcap%
\pgfsetroundjoin%
\definecolor{currentfill}{rgb}{0.255645,0.260703,0.528312}%
\pgfsetfillcolor{currentfill}%
\pgfsetfillopacity{0.700000}%
\pgfsetlinewidth{0.501875pt}%
\definecolor{currentstroke}{rgb}{1.000000,1.000000,1.000000}%
\pgfsetstrokecolor{currentstroke}%
\pgfsetstrokeopacity{0.700000}%
\pgfsetdash{}{0pt}%
\pgfpathmoveto{\pgfqpoint{4.285321in}{3.402637in}}%
\pgfpathcurveto{\pgfqpoint{4.298344in}{3.402637in}}{\pgfqpoint{4.310835in}{3.407811in}}{\pgfqpoint{4.320044in}{3.417019in}}%
\pgfpathcurveto{\pgfqpoint{4.329252in}{3.426228in}}{\pgfqpoint{4.334426in}{3.438719in}}{\pgfqpoint{4.334426in}{3.451741in}}%
\pgfpathcurveto{\pgfqpoint{4.334426in}{3.464764in}}{\pgfqpoint{4.329252in}{3.477255in}}{\pgfqpoint{4.320044in}{3.486464in}}%
\pgfpathcurveto{\pgfqpoint{4.310835in}{3.495672in}}{\pgfqpoint{4.298344in}{3.500846in}}{\pgfqpoint{4.285321in}{3.500846in}}%
\pgfpathcurveto{\pgfqpoint{4.272299in}{3.500846in}}{\pgfqpoint{4.259808in}{3.495672in}}{\pgfqpoint{4.250599in}{3.486464in}}%
\pgfpathcurveto{\pgfqpoint{4.241391in}{3.477255in}}{\pgfqpoint{4.236217in}{3.464764in}}{\pgfqpoint{4.236217in}{3.451741in}}%
\pgfpathcurveto{\pgfqpoint{4.236217in}{3.438719in}}{\pgfqpoint{4.241391in}{3.426228in}}{\pgfqpoint{4.250599in}{3.417019in}}%
\pgfpathcurveto{\pgfqpoint{4.259808in}{3.407811in}}{\pgfqpoint{4.272299in}{3.402637in}}{\pgfqpoint{4.285321in}{3.402637in}}%
\pgfpathlineto{\pgfqpoint{4.285321in}{3.402637in}}%
\pgfpathclose%
\pgfusepath{stroke,fill}%
\end{pgfscope}%
\begin{pgfscope}%
\pgfpathrectangle{\pgfqpoint{0.786164in}{0.768110in}}{\pgfqpoint{8.851069in}{7.081890in}}%
\pgfusepath{clip}%
\pgfsetbuttcap%
\pgfsetroundjoin%
\definecolor{currentfill}{rgb}{0.255645,0.260703,0.528312}%
\pgfsetfillcolor{currentfill}%
\pgfsetfillopacity{0.700000}%
\pgfsetlinewidth{0.501875pt}%
\definecolor{currentstroke}{rgb}{1.000000,1.000000,1.000000}%
\pgfsetstrokecolor{currentstroke}%
\pgfsetstrokeopacity{0.700000}%
\pgfsetdash{}{0pt}%
\pgfpathmoveto{\pgfqpoint{4.493869in}{3.445321in}}%
\pgfpathcurveto{\pgfqpoint{4.506892in}{3.445321in}}{\pgfqpoint{4.519383in}{3.450495in}}{\pgfqpoint{4.528591in}{3.459703in}}%
\pgfpathcurveto{\pgfqpoint{4.537800in}{3.468912in}}{\pgfqpoint{4.542974in}{3.481403in}}{\pgfqpoint{4.542974in}{3.494426in}}%
\pgfpathcurveto{\pgfqpoint{4.542974in}{3.507448in}}{\pgfqpoint{4.537800in}{3.519939in}}{\pgfqpoint{4.528591in}{3.529148in}}%
\pgfpathcurveto{\pgfqpoint{4.519383in}{3.538356in}}{\pgfqpoint{4.506892in}{3.543530in}}{\pgfqpoint{4.493869in}{3.543530in}}%
\pgfpathcurveto{\pgfqpoint{4.480846in}{3.543530in}}{\pgfqpoint{4.468355in}{3.538356in}}{\pgfqpoint{4.459147in}{3.529148in}}%
\pgfpathcurveto{\pgfqpoint{4.449939in}{3.519939in}}{\pgfqpoint{4.444765in}{3.507448in}}{\pgfqpoint{4.444765in}{3.494426in}}%
\pgfpathcurveto{\pgfqpoint{4.444765in}{3.481403in}}{\pgfqpoint{4.449939in}{3.468912in}}{\pgfqpoint{4.459147in}{3.459703in}}%
\pgfpathcurveto{\pgfqpoint{4.468355in}{3.450495in}}{\pgfqpoint{4.480846in}{3.445321in}}{\pgfqpoint{4.493869in}{3.445321in}}%
\pgfpathlineto{\pgfqpoint{4.493869in}{3.445321in}}%
\pgfpathclose%
\pgfusepath{stroke,fill}%
\end{pgfscope}%
\begin{pgfscope}%
\pgfpathrectangle{\pgfqpoint{0.786164in}{0.768110in}}{\pgfqpoint{8.851069in}{7.081890in}}%
\pgfusepath{clip}%
\pgfsetbuttcap%
\pgfsetroundjoin%
\definecolor{currentfill}{rgb}{0.263663,0.237631,0.518762}%
\pgfsetfillcolor{currentfill}%
\pgfsetfillopacity{0.700000}%
\pgfsetlinewidth{0.501875pt}%
\definecolor{currentstroke}{rgb}{1.000000,1.000000,1.000000}%
\pgfsetstrokecolor{currentstroke}%
\pgfsetstrokeopacity{0.700000}%
\pgfsetdash{}{0pt}%
\pgfpathmoveto{\pgfqpoint{4.754798in}{3.295926in}}%
\pgfpathcurveto{\pgfqpoint{4.767821in}{3.295926in}}{\pgfqpoint{4.780312in}{3.301100in}}{\pgfqpoint{4.789520in}{3.310308in}}%
\pgfpathcurveto{\pgfqpoint{4.798729in}{3.319517in}}{\pgfqpoint{4.803903in}{3.332008in}}{\pgfqpoint{4.803903in}{3.345030in}}%
\pgfpathcurveto{\pgfqpoint{4.803903in}{3.358053in}}{\pgfqpoint{4.798729in}{3.370544in}}{\pgfqpoint{4.789520in}{3.379753in}}%
\pgfpathcurveto{\pgfqpoint{4.780312in}{3.388961in}}{\pgfqpoint{4.767821in}{3.394135in}}{\pgfqpoint{4.754798in}{3.394135in}}%
\pgfpathcurveto{\pgfqpoint{4.741775in}{3.394135in}}{\pgfqpoint{4.729284in}{3.388961in}}{\pgfqpoint{4.720076in}{3.379753in}}%
\pgfpathcurveto{\pgfqpoint{4.710867in}{3.370544in}}{\pgfqpoint{4.705693in}{3.358053in}}{\pgfqpoint{4.705693in}{3.345030in}}%
\pgfpathcurveto{\pgfqpoint{4.705693in}{3.332008in}}{\pgfqpoint{4.710867in}{3.319517in}}{\pgfqpoint{4.720076in}{3.310308in}}%
\pgfpathcurveto{\pgfqpoint{4.729284in}{3.301100in}}{\pgfqpoint{4.741775in}{3.295926in}}{\pgfqpoint{4.754798in}{3.295926in}}%
\pgfpathlineto{\pgfqpoint{4.754798in}{3.295926in}}%
\pgfpathclose%
\pgfusepath{stroke,fill}%
\end{pgfscope}%
\begin{pgfscope}%
\pgfpathrectangle{\pgfqpoint{0.786164in}{0.768110in}}{\pgfqpoint{8.851069in}{7.081890in}}%
\pgfusepath{clip}%
\pgfsetbuttcap%
\pgfsetroundjoin%
\definecolor{currentfill}{rgb}{0.265145,0.232956,0.516599}%
\pgfsetfillcolor{currentfill}%
\pgfsetfillopacity{0.700000}%
\pgfsetlinewidth{0.501875pt}%
\definecolor{currentstroke}{rgb}{1.000000,1.000000,1.000000}%
\pgfsetstrokecolor{currentstroke}%
\pgfsetstrokeopacity{0.700000}%
\pgfsetdash{}{0pt}%
\pgfpathmoveto{\pgfqpoint{4.896313in}{3.146531in}}%
\pgfpathcurveto{\pgfqpoint{4.909335in}{3.146531in}}{\pgfqpoint{4.921826in}{3.151705in}}{\pgfqpoint{4.931035in}{3.160913in}}%
\pgfpathcurveto{\pgfqpoint{4.940243in}{3.170122in}}{\pgfqpoint{4.945417in}{3.182613in}}{\pgfqpoint{4.945417in}{3.195635in}}%
\pgfpathcurveto{\pgfqpoint{4.945417in}{3.208658in}}{\pgfqpoint{4.940243in}{3.221149in}}{\pgfqpoint{4.931035in}{3.230358in}}%
\pgfpathcurveto{\pgfqpoint{4.921826in}{3.239566in}}{\pgfqpoint{4.909335in}{3.244740in}}{\pgfqpoint{4.896313in}{3.244740in}}%
\pgfpathcurveto{\pgfqpoint{4.883290in}{3.244740in}}{\pgfqpoint{4.870799in}{3.239566in}}{\pgfqpoint{4.861590in}{3.230358in}}%
\pgfpathcurveto{\pgfqpoint{4.852382in}{3.221149in}}{\pgfqpoint{4.847208in}{3.208658in}}{\pgfqpoint{4.847208in}{3.195635in}}%
\pgfpathcurveto{\pgfqpoint{4.847208in}{3.182613in}}{\pgfqpoint{4.852382in}{3.170122in}}{\pgfqpoint{4.861590in}{3.160913in}}%
\pgfpathcurveto{\pgfqpoint{4.870799in}{3.151705in}}{\pgfqpoint{4.883290in}{3.146531in}}{\pgfqpoint{4.896313in}{3.146531in}}%
\pgfpathlineto{\pgfqpoint{4.896313in}{3.146531in}}%
\pgfpathclose%
\pgfusepath{stroke,fill}%
\end{pgfscope}%
\begin{pgfscope}%
\pgfpathrectangle{\pgfqpoint{0.786164in}{0.768110in}}{\pgfqpoint{8.851069in}{7.081890in}}%
\pgfusepath{clip}%
\pgfsetbuttcap%
\pgfsetroundjoin%
\definecolor{currentfill}{rgb}{0.263663,0.237631,0.518762}%
\pgfsetfillcolor{currentfill}%
\pgfsetfillopacity{0.700000}%
\pgfsetlinewidth{0.501875pt}%
\definecolor{currentstroke}{rgb}{1.000000,1.000000,1.000000}%
\pgfsetstrokecolor{currentstroke}%
\pgfsetstrokeopacity{0.700000}%
\pgfsetdash{}{0pt}%
\pgfpathmoveto{\pgfqpoint{5.048450in}{3.253242in}}%
\pgfpathcurveto{\pgfqpoint{5.061473in}{3.253242in}}{\pgfqpoint{5.073964in}{3.258415in}}{\pgfqpoint{5.083172in}{3.267624in}}%
\pgfpathcurveto{\pgfqpoint{5.092381in}{3.276832in}}{\pgfqpoint{5.097555in}{3.289323in}}{\pgfqpoint{5.097555in}{3.302346in}}%
\pgfpathcurveto{\pgfqpoint{5.097555in}{3.315369in}}{\pgfqpoint{5.092381in}{3.327860in}}{\pgfqpoint{5.083172in}{3.337068in}}%
\pgfpathcurveto{\pgfqpoint{5.073964in}{3.346277in}}{\pgfqpoint{5.061473in}{3.351451in}}{\pgfqpoint{5.048450in}{3.351451in}}%
\pgfpathcurveto{\pgfqpoint{5.035427in}{3.351451in}}{\pgfqpoint{5.022936in}{3.346277in}}{\pgfqpoint{5.013728in}{3.337068in}}%
\pgfpathcurveto{\pgfqpoint{5.004519in}{3.327860in}}{\pgfqpoint{4.999345in}{3.315369in}}{\pgfqpoint{4.999345in}{3.302346in}}%
\pgfpathcurveto{\pgfqpoint{4.999345in}{3.289323in}}{\pgfqpoint{5.004519in}{3.276832in}}{\pgfqpoint{5.013728in}{3.267624in}}%
\pgfpathcurveto{\pgfqpoint{5.022936in}{3.258415in}}{\pgfqpoint{5.035427in}{3.253242in}}{\pgfqpoint{5.048450in}{3.253242in}}%
\pgfpathlineto{\pgfqpoint{5.048450in}{3.253242in}}%
\pgfpathclose%
\pgfusepath{stroke,fill}%
\end{pgfscope}%
\begin{pgfscope}%
\pgfpathrectangle{\pgfqpoint{0.786164in}{0.768110in}}{\pgfqpoint{8.851069in}{7.081890in}}%
\pgfusepath{clip}%
\pgfsetbuttcap%
\pgfsetroundjoin%
\definecolor{currentfill}{rgb}{0.263663,0.237631,0.518762}%
\pgfsetfillcolor{currentfill}%
\pgfsetfillopacity{0.700000}%
\pgfsetlinewidth{0.501875pt}%
\definecolor{currentstroke}{rgb}{1.000000,1.000000,1.000000}%
\pgfsetstrokecolor{currentstroke}%
\pgfsetstrokeopacity{0.700000}%
\pgfsetdash{}{0pt}%
\pgfpathmoveto{\pgfqpoint{5.374703in}{3.146531in}}%
\pgfpathcurveto{\pgfqpoint{5.387725in}{3.146531in}}{\pgfqpoint{5.400216in}{3.151705in}}{\pgfqpoint{5.409425in}{3.160913in}}%
\pgfpathcurveto{\pgfqpoint{5.418633in}{3.170122in}}{\pgfqpoint{5.423807in}{3.182613in}}{\pgfqpoint{5.423807in}{3.195635in}}%
\pgfpathcurveto{\pgfqpoint{5.423807in}{3.208658in}}{\pgfqpoint{5.418633in}{3.221149in}}{\pgfqpoint{5.409425in}{3.230358in}}%
\pgfpathcurveto{\pgfqpoint{5.400216in}{3.239566in}}{\pgfqpoint{5.387725in}{3.244740in}}{\pgfqpoint{5.374703in}{3.244740in}}%
\pgfpathcurveto{\pgfqpoint{5.361680in}{3.244740in}}{\pgfqpoint{5.349189in}{3.239566in}}{\pgfqpoint{5.339980in}{3.230358in}}%
\pgfpathcurveto{\pgfqpoint{5.330772in}{3.221149in}}{\pgfqpoint{5.325598in}{3.208658in}}{\pgfqpoint{5.325598in}{3.195635in}}%
\pgfpathcurveto{\pgfqpoint{5.325598in}{3.182613in}}{\pgfqpoint{5.330772in}{3.170122in}}{\pgfqpoint{5.339980in}{3.160913in}}%
\pgfpathcurveto{\pgfqpoint{5.349189in}{3.151705in}}{\pgfqpoint{5.361680in}{3.146531in}}{\pgfqpoint{5.374703in}{3.146531in}}%
\pgfpathlineto{\pgfqpoint{5.374703in}{3.146531in}}%
\pgfpathclose%
\pgfusepath{stroke,fill}%
\end{pgfscope}%
\begin{pgfscope}%
\pgfpathrectangle{\pgfqpoint{0.786164in}{0.768110in}}{\pgfqpoint{8.851069in}{7.081890in}}%
\pgfusepath{clip}%
\pgfsetbuttcap%
\pgfsetroundjoin%
\definecolor{currentfill}{rgb}{0.263663,0.237631,0.518762}%
\pgfsetfillcolor{currentfill}%
\pgfsetfillopacity{0.700000}%
\pgfsetlinewidth{0.501875pt}%
\definecolor{currentstroke}{rgb}{1.000000,1.000000,1.000000}%
\pgfsetstrokecolor{currentstroke}%
\pgfsetstrokeopacity{0.700000}%
\pgfsetdash{}{0pt}%
\pgfpathmoveto{\pgfqpoint{5.468964in}{3.189215in}}%
\pgfpathcurveto{\pgfqpoint{5.481987in}{3.189215in}}{\pgfqpoint{5.494478in}{3.194389in}}{\pgfqpoint{5.503686in}{3.203597in}}%
\pgfpathcurveto{\pgfqpoint{5.512895in}{3.212806in}}{\pgfqpoint{5.518069in}{3.225297in}}{\pgfqpoint{5.518069in}{3.238320in}}%
\pgfpathcurveto{\pgfqpoint{5.518069in}{3.251342in}}{\pgfqpoint{5.512895in}{3.263833in}}{\pgfqpoint{5.503686in}{3.273042in}}%
\pgfpathcurveto{\pgfqpoint{5.494478in}{3.282250in}}{\pgfqpoint{5.481987in}{3.287424in}}{\pgfqpoint{5.468964in}{3.287424in}}%
\pgfpathcurveto{\pgfqpoint{5.455941in}{3.287424in}}{\pgfqpoint{5.443450in}{3.282250in}}{\pgfqpoint{5.434242in}{3.273042in}}%
\pgfpathcurveto{\pgfqpoint{5.425034in}{3.263833in}}{\pgfqpoint{5.419860in}{3.251342in}}{\pgfqpoint{5.419860in}{3.238320in}}%
\pgfpathcurveto{\pgfqpoint{5.419860in}{3.225297in}}{\pgfqpoint{5.425034in}{3.212806in}}{\pgfqpoint{5.434242in}{3.203597in}}%
\pgfpathcurveto{\pgfqpoint{5.443450in}{3.194389in}}{\pgfqpoint{5.455941in}{3.189215in}}{\pgfqpoint{5.468964in}{3.189215in}}%
\pgfpathlineto{\pgfqpoint{5.468964in}{3.189215in}}%
\pgfpathclose%
\pgfusepath{stroke,fill}%
\end{pgfscope}%
\begin{pgfscope}%
\pgfpathrectangle{\pgfqpoint{0.786164in}{0.768110in}}{\pgfqpoint{8.851069in}{7.081890in}}%
\pgfusepath{clip}%
\pgfsetbuttcap%
\pgfsetroundjoin%
\definecolor{currentfill}{rgb}{0.267968,0.223549,0.512008}%
\pgfsetfillcolor{currentfill}%
\pgfsetfillopacity{0.700000}%
\pgfsetlinewidth{0.501875pt}%
\definecolor{currentstroke}{rgb}{1.000000,1.000000,1.000000}%
\pgfsetstrokecolor{currentstroke}%
\pgfsetstrokeopacity{0.700000}%
\pgfsetdash{}{0pt}%
\pgfpathmoveto{\pgfqpoint{5.696193in}{3.018478in}}%
\pgfpathcurveto{\pgfqpoint{5.709216in}{3.018478in}}{\pgfqpoint{5.721707in}{3.023652in}}{\pgfqpoint{5.730916in}{3.032860in}}%
\pgfpathcurveto{\pgfqpoint{5.740124in}{3.042068in}}{\pgfqpoint{5.745298in}{3.054560in}}{\pgfqpoint{5.745298in}{3.067582in}}%
\pgfpathcurveto{\pgfqpoint{5.745298in}{3.080605in}}{\pgfqpoint{5.740124in}{3.093096in}}{\pgfqpoint{5.730916in}{3.102305in}}%
\pgfpathcurveto{\pgfqpoint{5.721707in}{3.111513in}}{\pgfqpoint{5.709216in}{3.116687in}}{\pgfqpoint{5.696193in}{3.116687in}}%
\pgfpathcurveto{\pgfqpoint{5.683171in}{3.116687in}}{\pgfqpoint{5.670680in}{3.111513in}}{\pgfqpoint{5.661471in}{3.102305in}}%
\pgfpathcurveto{\pgfqpoint{5.652263in}{3.093096in}}{\pgfqpoint{5.647089in}{3.080605in}}{\pgfqpoint{5.647089in}{3.067582in}}%
\pgfpathcurveto{\pgfqpoint{5.647089in}{3.054560in}}{\pgfqpoint{5.652263in}{3.042068in}}{\pgfqpoint{5.661471in}{3.032860in}}%
\pgfpathcurveto{\pgfqpoint{5.670680in}{3.023652in}}{\pgfqpoint{5.683171in}{3.018478in}}{\pgfqpoint{5.696193in}{3.018478in}}%
\pgfpathlineto{\pgfqpoint{5.696193in}{3.018478in}}%
\pgfpathclose%
\pgfusepath{stroke,fill}%
\end{pgfscope}%
\begin{pgfscope}%
\pgfpathrectangle{\pgfqpoint{0.786164in}{0.768110in}}{\pgfqpoint{8.851069in}{7.081890in}}%
\pgfusepath{clip}%
\pgfsetbuttcap%
\pgfsetroundjoin%
\definecolor{currentfill}{rgb}{0.276194,0.190074,0.493001}%
\pgfsetfillcolor{currentfill}%
\pgfsetfillopacity{0.700000}%
\pgfsetlinewidth{0.501875pt}%
\definecolor{currentstroke}{rgb}{1.000000,1.000000,1.000000}%
\pgfsetstrokecolor{currentstroke}%
\pgfsetstrokeopacity{0.700000}%
\pgfsetdash{}{0pt}%
\pgfpathmoveto{\pgfqpoint{5.044298in}{2.826398in}}%
\pgfpathcurveto{\pgfqpoint{5.057321in}{2.826398in}}{\pgfqpoint{5.069812in}{2.831572in}}{\pgfqpoint{5.079021in}{2.840781in}}%
\pgfpathcurveto{\pgfqpoint{5.088229in}{2.849989in}}{\pgfqpoint{5.093403in}{2.862480in}}{\pgfqpoint{5.093403in}{2.875503in}}%
\pgfpathcurveto{\pgfqpoint{5.093403in}{2.888525in}}{\pgfqpoint{5.088229in}{2.901017in}}{\pgfqpoint{5.079021in}{2.910225in}}%
\pgfpathcurveto{\pgfqpoint{5.069812in}{2.919433in}}{\pgfqpoint{5.057321in}{2.924607in}}{\pgfqpoint{5.044298in}{2.924607in}}%
\pgfpathcurveto{\pgfqpoint{5.031276in}{2.924607in}}{\pgfqpoint{5.018785in}{2.919433in}}{\pgfqpoint{5.009576in}{2.910225in}}%
\pgfpathcurveto{\pgfqpoint{5.000368in}{2.901017in}}{\pgfqpoint{4.995194in}{2.888525in}}{\pgfqpoint{4.995194in}{2.875503in}}%
\pgfpathcurveto{\pgfqpoint{4.995194in}{2.862480in}}{\pgfqpoint{5.000368in}{2.849989in}}{\pgfqpoint{5.009576in}{2.840781in}}%
\pgfpathcurveto{\pgfqpoint{5.018785in}{2.831572in}}{\pgfqpoint{5.031276in}{2.826398in}}{\pgfqpoint{5.044298in}{2.826398in}}%
\pgfpathlineto{\pgfqpoint{5.044298in}{2.826398in}}%
\pgfpathclose%
\pgfusepath{stroke,fill}%
\end{pgfscope}%
\begin{pgfscope}%
\pgfpathrectangle{\pgfqpoint{0.786164in}{0.768110in}}{\pgfqpoint{8.851069in}{7.081890in}}%
\pgfusepath{clip}%
\pgfsetbuttcap%
\pgfsetroundjoin%
\definecolor{currentfill}{rgb}{0.271828,0.209303,0.504434}%
\pgfsetfillcolor{currentfill}%
\pgfsetfillopacity{0.700000}%
\pgfsetlinewidth{0.501875pt}%
\definecolor{currentstroke}{rgb}{1.000000,1.000000,1.000000}%
\pgfsetstrokecolor{currentstroke}%
\pgfsetstrokeopacity{0.700000}%
\pgfsetdash{}{0pt}%
\pgfpathmoveto{\pgfqpoint{6.183252in}{2.826398in}}%
\pgfpathcurveto{\pgfqpoint{6.196275in}{2.826398in}}{\pgfqpoint{6.208766in}{2.831572in}}{\pgfqpoint{6.217975in}{2.840781in}}%
\pgfpathcurveto{\pgfqpoint{6.227183in}{2.849989in}}{\pgfqpoint{6.232357in}{2.862480in}}{\pgfqpoint{6.232357in}{2.875503in}}%
\pgfpathcurveto{\pgfqpoint{6.232357in}{2.888525in}}{\pgfqpoint{6.227183in}{2.901017in}}{\pgfqpoint{6.217975in}{2.910225in}}%
\pgfpathcurveto{\pgfqpoint{6.208766in}{2.919433in}}{\pgfqpoint{6.196275in}{2.924607in}}{\pgfqpoint{6.183252in}{2.924607in}}%
\pgfpathcurveto{\pgfqpoint{6.170230in}{2.924607in}}{\pgfqpoint{6.157739in}{2.919433in}}{\pgfqpoint{6.148530in}{2.910225in}}%
\pgfpathcurveto{\pgfqpoint{6.139322in}{2.901017in}}{\pgfqpoint{6.134148in}{2.888525in}}{\pgfqpoint{6.134148in}{2.875503in}}%
\pgfpathcurveto{\pgfqpoint{6.134148in}{2.862480in}}{\pgfqpoint{6.139322in}{2.849989in}}{\pgfqpoint{6.148530in}{2.840781in}}%
\pgfpathcurveto{\pgfqpoint{6.157739in}{2.831572in}}{\pgfqpoint{6.170230in}{2.826398in}}{\pgfqpoint{6.183252in}{2.826398in}}%
\pgfpathlineto{\pgfqpoint{6.183252in}{2.826398in}}%
\pgfpathclose%
\pgfusepath{stroke,fill}%
\end{pgfscope}%
\begin{pgfscope}%
\pgfpathrectangle{\pgfqpoint{0.786164in}{0.768110in}}{\pgfqpoint{8.851069in}{7.081890in}}%
\pgfusepath{clip}%
\pgfsetbuttcap%
\pgfsetroundjoin%
\definecolor{currentfill}{rgb}{0.275191,0.194905,0.496005}%
\pgfsetfillcolor{currentfill}%
\pgfsetfillopacity{0.700000}%
\pgfsetlinewidth{0.501875pt}%
\definecolor{currentstroke}{rgb}{1.000000,1.000000,1.000000}%
\pgfsetstrokecolor{currentstroke}%
\pgfsetstrokeopacity{0.700000}%
\pgfsetdash{}{0pt}%
\pgfpathmoveto{\pgfqpoint{6.255536in}{2.762372in}}%
\pgfpathcurveto{\pgfqpoint{6.268559in}{2.762372in}}{\pgfqpoint{6.281050in}{2.767546in}}{\pgfqpoint{6.290258in}{2.776754in}}%
\pgfpathcurveto{\pgfqpoint{6.299467in}{2.785962in}}{\pgfqpoint{6.304641in}{2.798454in}}{\pgfqpoint{6.304641in}{2.811476in}}%
\pgfpathcurveto{\pgfqpoint{6.304641in}{2.824499in}}{\pgfqpoint{6.299467in}{2.836990in}}{\pgfqpoint{6.290258in}{2.846198in}}%
\pgfpathcurveto{\pgfqpoint{6.281050in}{2.855407in}}{\pgfqpoint{6.268559in}{2.860581in}}{\pgfqpoint{6.255536in}{2.860581in}}%
\pgfpathcurveto{\pgfqpoint{6.242513in}{2.860581in}}{\pgfqpoint{6.230022in}{2.855407in}}{\pgfqpoint{6.220814in}{2.846198in}}%
\pgfpathcurveto{\pgfqpoint{6.211605in}{2.836990in}}{\pgfqpoint{6.206431in}{2.824499in}}{\pgfqpoint{6.206431in}{2.811476in}}%
\pgfpathcurveto{\pgfqpoint{6.206431in}{2.798454in}}{\pgfqpoint{6.211605in}{2.785962in}}{\pgfqpoint{6.220814in}{2.776754in}}%
\pgfpathcurveto{\pgfqpoint{6.230022in}{2.767546in}}{\pgfqpoint{6.242513in}{2.762372in}}{\pgfqpoint{6.255536in}{2.762372in}}%
\pgfpathlineto{\pgfqpoint{6.255536in}{2.762372in}}%
\pgfpathclose%
\pgfusepath{stroke,fill}%
\end{pgfscope}%
\begin{pgfscope}%
\pgfpathrectangle{\pgfqpoint{0.786164in}{0.768110in}}{\pgfqpoint{8.851069in}{7.081890in}}%
\pgfusepath{clip}%
\pgfsetbuttcap%
\pgfsetroundjoin%
\definecolor{currentfill}{rgb}{0.220057,0.343307,0.549413}%
\pgfsetfillcolor{currentfill}%
\pgfsetfillopacity{0.700000}%
\pgfsetlinewidth{0.501875pt}%
\definecolor{currentstroke}{rgb}{1.000000,1.000000,1.000000}%
\pgfsetstrokecolor{currentstroke}%
\pgfsetstrokeopacity{0.700000}%
\pgfsetdash{}{0pt}%
\pgfpathmoveto{\pgfqpoint{3.425123in}{3.978875in}}%
\pgfpathcurveto{\pgfqpoint{3.438146in}{3.978875in}}{\pgfqpoint{3.450637in}{3.984049in}}{\pgfqpoint{3.459845in}{3.993258in}}%
\pgfpathcurveto{\pgfqpoint{3.469054in}{4.002466in}}{\pgfqpoint{3.474228in}{4.014957in}}{\pgfqpoint{3.474228in}{4.027980in}}%
\pgfpathcurveto{\pgfqpoint{3.474228in}{4.041003in}}{\pgfqpoint{3.469054in}{4.053494in}}{\pgfqpoint{3.459845in}{4.062702in}}%
\pgfpathcurveto{\pgfqpoint{3.450637in}{4.071911in}}{\pgfqpoint{3.438146in}{4.077085in}}{\pgfqpoint{3.425123in}{4.077085in}}%
\pgfpathcurveto{\pgfqpoint{3.412100in}{4.077085in}}{\pgfqpoint{3.399609in}{4.071911in}}{\pgfqpoint{3.390401in}{4.062702in}}%
\pgfpathcurveto{\pgfqpoint{3.381192in}{4.053494in}}{\pgfqpoint{3.376018in}{4.041003in}}{\pgfqpoint{3.376018in}{4.027980in}}%
\pgfpathcurveto{\pgfqpoint{3.376018in}{4.014957in}}{\pgfqpoint{3.381192in}{4.002466in}}{\pgfqpoint{3.390401in}{3.993258in}}%
\pgfpathcurveto{\pgfqpoint{3.399609in}{3.984049in}}{\pgfqpoint{3.412100in}{3.978875in}}{\pgfqpoint{3.425123in}{3.978875in}}%
\pgfpathlineto{\pgfqpoint{3.425123in}{3.978875in}}%
\pgfpathclose%
\pgfusepath{stroke,fill}%
\end{pgfscope}%
\begin{pgfscope}%
\pgfpathrectangle{\pgfqpoint{0.786164in}{0.768110in}}{\pgfqpoint{8.851069in}{7.081890in}}%
\pgfusepath{clip}%
\pgfsetbuttcap%
\pgfsetroundjoin%
\definecolor{currentfill}{rgb}{0.216210,0.351535,0.550627}%
\pgfsetfillcolor{currentfill}%
\pgfsetfillopacity{0.700000}%
\pgfsetlinewidth{0.501875pt}%
\definecolor{currentstroke}{rgb}{1.000000,1.000000,1.000000}%
\pgfsetstrokecolor{currentstroke}%
\pgfsetstrokeopacity{0.700000}%
\pgfsetdash{}{0pt}%
\pgfpathmoveto{\pgfqpoint{3.310104in}{3.765454in}}%
\pgfpathcurveto{\pgfqpoint{3.323127in}{3.765454in}}{\pgfqpoint{3.335618in}{3.770628in}}{\pgfqpoint{3.344827in}{3.779836in}}%
\pgfpathcurveto{\pgfqpoint{3.354035in}{3.789044in}}{\pgfqpoint{3.359209in}{3.801536in}}{\pgfqpoint{3.359209in}{3.814558in}}%
\pgfpathcurveto{\pgfqpoint{3.359209in}{3.827581in}}{\pgfqpoint{3.354035in}{3.840072in}}{\pgfqpoint{3.344827in}{3.849280in}}%
\pgfpathcurveto{\pgfqpoint{3.335618in}{3.858489in}}{\pgfqpoint{3.323127in}{3.863663in}}{\pgfqpoint{3.310104in}{3.863663in}}%
\pgfpathcurveto{\pgfqpoint{3.297082in}{3.863663in}}{\pgfqpoint{3.284591in}{3.858489in}}{\pgfqpoint{3.275382in}{3.849280in}}%
\pgfpathcurveto{\pgfqpoint{3.266174in}{3.840072in}}{\pgfqpoint{3.261000in}{3.827581in}}{\pgfqpoint{3.261000in}{3.814558in}}%
\pgfpathcurveto{\pgfqpoint{3.261000in}{3.801536in}}{\pgfqpoint{3.266174in}{3.789044in}}{\pgfqpoint{3.275382in}{3.779836in}}%
\pgfpathcurveto{\pgfqpoint{3.284591in}{3.770628in}}{\pgfqpoint{3.297082in}{3.765454in}}{\pgfqpoint{3.310104in}{3.765454in}}%
\pgfpathlineto{\pgfqpoint{3.310104in}{3.765454in}}%
\pgfpathclose%
\pgfusepath{stroke,fill}%
\end{pgfscope}%
\begin{pgfscope}%
\pgfpathrectangle{\pgfqpoint{0.786164in}{0.768110in}}{\pgfqpoint{8.851069in}{7.081890in}}%
\pgfusepath{clip}%
\pgfsetbuttcap%
\pgfsetroundjoin%
\definecolor{currentfill}{rgb}{0.218130,0.347432,0.550038}%
\pgfsetfillcolor{currentfill}%
\pgfsetfillopacity{0.700000}%
\pgfsetlinewidth{0.501875pt}%
\definecolor{currentstroke}{rgb}{1.000000,1.000000,1.000000}%
\pgfsetstrokecolor{currentstroke}%
\pgfsetstrokeopacity{0.700000}%
\pgfsetdash{}{0pt}%
\pgfpathmoveto{\pgfqpoint{3.130983in}{3.509348in}}%
\pgfpathcurveto{\pgfqpoint{3.144006in}{3.509348in}}{\pgfqpoint{3.156497in}{3.514522in}}{\pgfqpoint{3.165705in}{3.523730in}}%
\pgfpathcurveto{\pgfqpoint{3.174914in}{3.532938in}}{\pgfqpoint{3.180087in}{3.545429in}}{\pgfqpoint{3.180087in}{3.558452in}}%
\pgfpathcurveto{\pgfqpoint{3.180087in}{3.571475in}}{\pgfqpoint{3.174914in}{3.583966in}}{\pgfqpoint{3.165705in}{3.593174in}}%
\pgfpathcurveto{\pgfqpoint{3.156497in}{3.602383in}}{\pgfqpoint{3.144006in}{3.607557in}}{\pgfqpoint{3.130983in}{3.607557in}}%
\pgfpathcurveto{\pgfqpoint{3.117960in}{3.607557in}}{\pgfqpoint{3.105469in}{3.602383in}}{\pgfqpoint{3.096261in}{3.593174in}}%
\pgfpathcurveto{\pgfqpoint{3.087052in}{3.583966in}}{\pgfqpoint{3.081878in}{3.571475in}}{\pgfqpoint{3.081878in}{3.558452in}}%
\pgfpathcurveto{\pgfqpoint{3.081878in}{3.545429in}}{\pgfqpoint{3.087052in}{3.532938in}}{\pgfqpoint{3.096261in}{3.523730in}}%
\pgfpathcurveto{\pgfqpoint{3.105469in}{3.514522in}}{\pgfqpoint{3.117960in}{3.509348in}}{\pgfqpoint{3.130983in}{3.509348in}}%
\pgfpathlineto{\pgfqpoint{3.130983in}{3.509348in}}%
\pgfpathclose%
\pgfusepath{stroke,fill}%
\end{pgfscope}%
\begin{pgfscope}%
\pgfpathrectangle{\pgfqpoint{0.786164in}{0.768110in}}{\pgfqpoint{8.851069in}{7.081890in}}%
\pgfusepath{clip}%
\pgfsetbuttcap%
\pgfsetroundjoin%
\definecolor{currentfill}{rgb}{0.180629,0.429975,0.557282}%
\pgfsetfillcolor{currentfill}%
\pgfsetfillopacity{0.700000}%
\pgfsetlinewidth{0.501875pt}%
\definecolor{currentstroke}{rgb}{1.000000,1.000000,1.000000}%
\pgfsetstrokecolor{currentstroke}%
\pgfsetstrokeopacity{0.700000}%
\pgfsetdash{}{0pt}%
\pgfpathmoveto{\pgfqpoint{3.268712in}{3.872164in}}%
\pgfpathcurveto{\pgfqpoint{3.281735in}{3.872164in}}{\pgfqpoint{3.294226in}{3.877338in}}{\pgfqpoint{3.303434in}{3.886547in}}%
\pgfpathcurveto{\pgfqpoint{3.312643in}{3.895755in}}{\pgfqpoint{3.317817in}{3.908246in}}{\pgfqpoint{3.317817in}{3.921269in}}%
\pgfpathcurveto{\pgfqpoint{3.317817in}{3.934292in}}{\pgfqpoint{3.312643in}{3.946783in}}{\pgfqpoint{3.303434in}{3.955991in}}%
\pgfpathcurveto{\pgfqpoint{3.294226in}{3.965200in}}{\pgfqpoint{3.281735in}{3.970374in}}{\pgfqpoint{3.268712in}{3.970374in}}%
\pgfpathcurveto{\pgfqpoint{3.255690in}{3.970374in}}{\pgfqpoint{3.243198in}{3.965200in}}{\pgfqpoint{3.233990in}{3.955991in}}%
\pgfpathcurveto{\pgfqpoint{3.224782in}{3.946783in}}{\pgfqpoint{3.219608in}{3.934292in}}{\pgfqpoint{3.219608in}{3.921269in}}%
\pgfpathcurveto{\pgfqpoint{3.219608in}{3.908246in}}{\pgfqpoint{3.224782in}{3.895755in}}{\pgfqpoint{3.233990in}{3.886547in}}%
\pgfpathcurveto{\pgfqpoint{3.243198in}{3.877338in}}{\pgfqpoint{3.255690in}{3.872164in}}{\pgfqpoint{3.268712in}{3.872164in}}%
\pgfpathlineto{\pgfqpoint{3.268712in}{3.872164in}}%
\pgfpathclose%
\pgfusepath{stroke,fill}%
\end{pgfscope}%
\begin{pgfscope}%
\pgfpathrectangle{\pgfqpoint{0.786164in}{0.768110in}}{\pgfqpoint{8.851069in}{7.081890in}}%
\pgfusepath{clip}%
\pgfsetbuttcap%
\pgfsetroundjoin%
\definecolor{currentfill}{rgb}{0.210503,0.363727,0.552206}%
\pgfsetfillcolor{currentfill}%
\pgfsetfillopacity{0.700000}%
\pgfsetlinewidth{0.501875pt}%
\definecolor{currentstroke}{rgb}{1.000000,1.000000,1.000000}%
\pgfsetstrokecolor{currentstroke}%
\pgfsetstrokeopacity{0.700000}%
\pgfsetdash{}{0pt}%
\pgfpathmoveto{\pgfqpoint{3.472864in}{3.509348in}}%
\pgfpathcurveto{\pgfqpoint{3.485887in}{3.509348in}}{\pgfqpoint{3.498378in}{3.514522in}}{\pgfqpoint{3.507587in}{3.523730in}}%
\pgfpathcurveto{\pgfqpoint{3.516795in}{3.532938in}}{\pgfqpoint{3.521969in}{3.545429in}}{\pgfqpoint{3.521969in}{3.558452in}}%
\pgfpathcurveto{\pgfqpoint{3.521969in}{3.571475in}}{\pgfqpoint{3.516795in}{3.583966in}}{\pgfqpoint{3.507587in}{3.593174in}}%
\pgfpathcurveto{\pgfqpoint{3.498378in}{3.602383in}}{\pgfqpoint{3.485887in}{3.607557in}}{\pgfqpoint{3.472864in}{3.607557in}}%
\pgfpathcurveto{\pgfqpoint{3.459842in}{3.607557in}}{\pgfqpoint{3.447351in}{3.602383in}}{\pgfqpoint{3.438142in}{3.593174in}}%
\pgfpathcurveto{\pgfqpoint{3.428934in}{3.583966in}}{\pgfqpoint{3.423760in}{3.571475in}}{\pgfqpoint{3.423760in}{3.558452in}}%
\pgfpathcurveto{\pgfqpoint{3.423760in}{3.545429in}}{\pgfqpoint{3.428934in}{3.532938in}}{\pgfqpoint{3.438142in}{3.523730in}}%
\pgfpathcurveto{\pgfqpoint{3.447351in}{3.514522in}}{\pgfqpoint{3.459842in}{3.509348in}}{\pgfqpoint{3.472864in}{3.509348in}}%
\pgfpathlineto{\pgfqpoint{3.472864in}{3.509348in}}%
\pgfpathclose%
\pgfusepath{stroke,fill}%
\end{pgfscope}%
\begin{pgfscope}%
\pgfpathrectangle{\pgfqpoint{0.786164in}{0.768110in}}{\pgfqpoint{8.851069in}{7.081890in}}%
\pgfusepath{clip}%
\pgfsetbuttcap%
\pgfsetroundjoin%
\definecolor{currentfill}{rgb}{0.252194,0.269783,0.531579}%
\pgfsetfillcolor{currentfill}%
\pgfsetfillopacity{0.700000}%
\pgfsetlinewidth{0.501875pt}%
\definecolor{currentstroke}{rgb}{1.000000,1.000000,1.000000}%
\pgfsetstrokecolor{currentstroke}%
\pgfsetstrokeopacity{0.700000}%
\pgfsetdash{}{0pt}%
\pgfpathmoveto{\pgfqpoint{3.985443in}{2.826398in}}%
\pgfpathcurveto{\pgfqpoint{3.998465in}{2.826398in}}{\pgfqpoint{4.010956in}{2.831572in}}{\pgfqpoint{4.020165in}{2.840781in}}%
\pgfpathcurveto{\pgfqpoint{4.029373in}{2.849989in}}{\pgfqpoint{4.034547in}{2.862480in}}{\pgfqpoint{4.034547in}{2.875503in}}%
\pgfpathcurveto{\pgfqpoint{4.034547in}{2.888525in}}{\pgfqpoint{4.029373in}{2.901017in}}{\pgfqpoint{4.020165in}{2.910225in}}%
\pgfpathcurveto{\pgfqpoint{4.010956in}{2.919433in}}{\pgfqpoint{3.998465in}{2.924607in}}{\pgfqpoint{3.985443in}{2.924607in}}%
\pgfpathcurveto{\pgfqpoint{3.972420in}{2.924607in}}{\pgfqpoint{3.959929in}{2.919433in}}{\pgfqpoint{3.950720in}{2.910225in}}%
\pgfpathcurveto{\pgfqpoint{3.941512in}{2.901017in}}{\pgfqpoint{3.936338in}{2.888525in}}{\pgfqpoint{3.936338in}{2.875503in}}%
\pgfpathcurveto{\pgfqpoint{3.936338in}{2.862480in}}{\pgfqpoint{3.941512in}{2.849989in}}{\pgfqpoint{3.950720in}{2.840781in}}%
\pgfpathcurveto{\pgfqpoint{3.959929in}{2.831572in}}{\pgfqpoint{3.972420in}{2.826398in}}{\pgfqpoint{3.985443in}{2.826398in}}%
\pgfpathlineto{\pgfqpoint{3.985443in}{2.826398in}}%
\pgfpathclose%
\pgfusepath{stroke,fill}%
\end{pgfscope}%
\begin{pgfscope}%
\pgfpathrectangle{\pgfqpoint{0.786164in}{0.768110in}}{\pgfqpoint{8.851069in}{7.081890in}}%
\pgfusepath{clip}%
\pgfsetbuttcap%
\pgfsetroundjoin%
\definecolor{currentfill}{rgb}{0.190631,0.407061,0.556089}%
\pgfsetfillcolor{currentfill}%
\pgfsetfillopacity{0.700000}%
\pgfsetlinewidth{0.501875pt}%
\definecolor{currentstroke}{rgb}{1.000000,1.000000,1.000000}%
\pgfsetstrokecolor{currentstroke}%
\pgfsetstrokeopacity{0.700000}%
\pgfsetdash{}{0pt}%
\pgfpathmoveto{\pgfqpoint{4.176652in}{3.829480in}}%
\pgfpathcurveto{\pgfqpoint{4.189675in}{3.829480in}}{\pgfqpoint{4.202166in}{3.834654in}}{\pgfqpoint{4.211374in}{3.843863in}}%
\pgfpathcurveto{\pgfqpoint{4.220583in}{3.853071in}}{\pgfqpoint{4.225757in}{3.865562in}}{\pgfqpoint{4.225757in}{3.878585in}}%
\pgfpathcurveto{\pgfqpoint{4.225757in}{3.891607in}}{\pgfqpoint{4.220583in}{3.904099in}}{\pgfqpoint{4.211374in}{3.913307in}}%
\pgfpathcurveto{\pgfqpoint{4.202166in}{3.922515in}}{\pgfqpoint{4.189675in}{3.927689in}}{\pgfqpoint{4.176652in}{3.927689in}}%
\pgfpathcurveto{\pgfqpoint{4.163629in}{3.927689in}}{\pgfqpoint{4.151138in}{3.922515in}}{\pgfqpoint{4.141930in}{3.913307in}}%
\pgfpathcurveto{\pgfqpoint{4.132721in}{3.904099in}}{\pgfqpoint{4.127547in}{3.891607in}}{\pgfqpoint{4.127547in}{3.878585in}}%
\pgfpathcurveto{\pgfqpoint{4.127547in}{3.865562in}}{\pgfqpoint{4.132721in}{3.853071in}}{\pgfqpoint{4.141930in}{3.843863in}}%
\pgfpathcurveto{\pgfqpoint{4.151138in}{3.834654in}}{\pgfqpoint{4.163629in}{3.829480in}}{\pgfqpoint{4.176652in}{3.829480in}}%
\pgfpathlineto{\pgfqpoint{4.176652in}{3.829480in}}%
\pgfpathclose%
\pgfusepath{stroke,fill}%
\end{pgfscope}%
\begin{pgfscope}%
\pgfpathrectangle{\pgfqpoint{0.786164in}{0.768110in}}{\pgfqpoint{8.851069in}{7.081890in}}%
\pgfusepath{clip}%
\pgfsetbuttcap%
\pgfsetroundjoin%
\definecolor{currentfill}{rgb}{0.195860,0.395433,0.555276}%
\pgfsetfillcolor{currentfill}%
\pgfsetfillopacity{0.700000}%
\pgfsetlinewidth{0.501875pt}%
\definecolor{currentstroke}{rgb}{1.000000,1.000000,1.000000}%
\pgfsetstrokecolor{currentstroke}%
\pgfsetstrokeopacity{0.700000}%
\pgfsetdash{}{0pt}%
\pgfpathmoveto{\pgfqpoint{4.291426in}{4.085586in}}%
\pgfpathcurveto{\pgfqpoint{4.304449in}{4.085586in}}{\pgfqpoint{4.316940in}{4.090760in}}{\pgfqpoint{4.326149in}{4.099969in}}%
\pgfpathcurveto{\pgfqpoint{4.335357in}{4.109177in}}{\pgfqpoint{4.340531in}{4.121668in}}{\pgfqpoint{4.340531in}{4.134691in}}%
\pgfpathcurveto{\pgfqpoint{4.340531in}{4.147714in}}{\pgfqpoint{4.335357in}{4.160205in}}{\pgfqpoint{4.326149in}{4.169413in}}%
\pgfpathcurveto{\pgfqpoint{4.316940in}{4.178621in}}{\pgfqpoint{4.304449in}{4.183795in}}{\pgfqpoint{4.291426in}{4.183795in}}%
\pgfpathcurveto{\pgfqpoint{4.278404in}{4.183795in}}{\pgfqpoint{4.265913in}{4.178621in}}{\pgfqpoint{4.256704in}{4.169413in}}%
\pgfpathcurveto{\pgfqpoint{4.247496in}{4.160205in}}{\pgfqpoint{4.242322in}{4.147714in}}{\pgfqpoint{4.242322in}{4.134691in}}%
\pgfpathcurveto{\pgfqpoint{4.242322in}{4.121668in}}{\pgfqpoint{4.247496in}{4.109177in}}{\pgfqpoint{4.256704in}{4.099969in}}%
\pgfpathcurveto{\pgfqpoint{4.265913in}{4.090760in}}{\pgfqpoint{4.278404in}{4.085586in}}{\pgfqpoint{4.291426in}{4.085586in}}%
\pgfpathlineto{\pgfqpoint{4.291426in}{4.085586in}}%
\pgfpathclose%
\pgfusepath{stroke,fill}%
\end{pgfscope}%
\begin{pgfscope}%
\pgfpathrectangle{\pgfqpoint{0.786164in}{0.768110in}}{\pgfqpoint{8.851069in}{7.081890in}}%
\pgfusepath{clip}%
\pgfsetbuttcap%
\pgfsetroundjoin%
\definecolor{currentfill}{rgb}{0.214298,0.355619,0.551184}%
\pgfsetfillcolor{currentfill}%
\pgfsetfillopacity{0.700000}%
\pgfsetlinewidth{0.501875pt}%
\definecolor{currentstroke}{rgb}{1.000000,1.000000,1.000000}%
\pgfsetstrokecolor{currentstroke}%
\pgfsetstrokeopacity{0.700000}%
\pgfsetdash{}{0pt}%
\pgfpathmoveto{\pgfqpoint{4.300096in}{4.170955in}}%
\pgfpathcurveto{\pgfqpoint{4.313118in}{4.170955in}}{\pgfqpoint{4.325609in}{4.176129in}}{\pgfqpoint{4.334818in}{4.185337in}}%
\pgfpathcurveto{\pgfqpoint{4.344026in}{4.194546in}}{\pgfqpoint{4.349200in}{4.207037in}}{\pgfqpoint{4.349200in}{4.220059in}}%
\pgfpathcurveto{\pgfqpoint{4.349200in}{4.233082in}}{\pgfqpoint{4.344026in}{4.245573in}}{\pgfqpoint{4.334818in}{4.254782in}}%
\pgfpathcurveto{\pgfqpoint{4.325609in}{4.263990in}}{\pgfqpoint{4.313118in}{4.269164in}}{\pgfqpoint{4.300096in}{4.269164in}}%
\pgfpathcurveto{\pgfqpoint{4.287073in}{4.269164in}}{\pgfqpoint{4.274582in}{4.263990in}}{\pgfqpoint{4.265373in}{4.254782in}}%
\pgfpathcurveto{\pgfqpoint{4.256165in}{4.245573in}}{\pgfqpoint{4.250991in}{4.233082in}}{\pgfqpoint{4.250991in}{4.220059in}}%
\pgfpathcurveto{\pgfqpoint{4.250991in}{4.207037in}}{\pgfqpoint{4.256165in}{4.194546in}}{\pgfqpoint{4.265373in}{4.185337in}}%
\pgfpathcurveto{\pgfqpoint{4.274582in}{4.176129in}}{\pgfqpoint{4.287073in}{4.170955in}}{\pgfqpoint{4.300096in}{4.170955in}}%
\pgfpathlineto{\pgfqpoint{4.300096in}{4.170955in}}%
\pgfpathclose%
\pgfusepath{stroke,fill}%
\end{pgfscope}%
\begin{pgfscope}%
\pgfpathrectangle{\pgfqpoint{0.786164in}{0.768110in}}{\pgfqpoint{8.851069in}{7.081890in}}%
\pgfusepath{clip}%
\pgfsetbuttcap%
\pgfsetroundjoin%
\definecolor{currentfill}{rgb}{0.192357,0.403199,0.555836}%
\pgfsetfillcolor{currentfill}%
\pgfsetfillopacity{0.700000}%
\pgfsetlinewidth{0.501875pt}%
\definecolor{currentstroke}{rgb}{1.000000,1.000000,1.000000}%
\pgfsetstrokecolor{currentstroke}%
\pgfsetstrokeopacity{0.700000}%
\pgfsetdash{}{0pt}%
\pgfpathmoveto{\pgfqpoint{4.272012in}{4.661825in}}%
\pgfpathcurveto{\pgfqpoint{4.285035in}{4.661825in}}{\pgfqpoint{4.297526in}{4.666999in}}{\pgfqpoint{4.306735in}{4.676207in}}%
\pgfpathcurveto{\pgfqpoint{4.315943in}{4.685416in}}{\pgfqpoint{4.321117in}{4.697907in}}{\pgfqpoint{4.321117in}{4.710929in}}%
\pgfpathcurveto{\pgfqpoint{4.321117in}{4.723952in}}{\pgfqpoint{4.315943in}{4.736443in}}{\pgfqpoint{4.306735in}{4.745652in}}%
\pgfpathcurveto{\pgfqpoint{4.297526in}{4.754860in}}{\pgfqpoint{4.285035in}{4.760034in}}{\pgfqpoint{4.272012in}{4.760034in}}%
\pgfpathcurveto{\pgfqpoint{4.258990in}{4.760034in}}{\pgfqpoint{4.246499in}{4.754860in}}{\pgfqpoint{4.237290in}{4.745652in}}%
\pgfpathcurveto{\pgfqpoint{4.228082in}{4.736443in}}{\pgfqpoint{4.222908in}{4.723952in}}{\pgfqpoint{4.222908in}{4.710929in}}%
\pgfpathcurveto{\pgfqpoint{4.222908in}{4.697907in}}{\pgfqpoint{4.228082in}{4.685416in}}{\pgfqpoint{4.237290in}{4.676207in}}%
\pgfpathcurveto{\pgfqpoint{4.246499in}{4.666999in}}{\pgfqpoint{4.258990in}{4.661825in}}{\pgfqpoint{4.272012in}{4.661825in}}%
\pgfpathlineto{\pgfqpoint{4.272012in}{4.661825in}}%
\pgfpathclose%
\pgfusepath{stroke,fill}%
\end{pgfscope}%
\begin{pgfscope}%
\pgfpathrectangle{\pgfqpoint{0.786164in}{0.768110in}}{\pgfqpoint{8.851069in}{7.081890in}}%
\pgfusepath{clip}%
\pgfsetbuttcap%
\pgfsetroundjoin%
\definecolor{currentfill}{rgb}{0.201239,0.383670,0.554294}%
\pgfsetfillcolor{currentfill}%
\pgfsetfillopacity{0.700000}%
\pgfsetlinewidth{0.501875pt}%
\definecolor{currentstroke}{rgb}{1.000000,1.000000,1.000000}%
\pgfsetstrokecolor{currentstroke}%
\pgfsetstrokeopacity{0.700000}%
\pgfsetdash{}{0pt}%
\pgfpathmoveto{\pgfqpoint{4.366518in}{4.661825in}}%
\pgfpathcurveto{\pgfqpoint{4.379541in}{4.661825in}}{\pgfqpoint{4.392032in}{4.666999in}}{\pgfqpoint{4.401241in}{4.676207in}}%
\pgfpathcurveto{\pgfqpoint{4.410449in}{4.685416in}}{\pgfqpoint{4.415623in}{4.697907in}}{\pgfqpoint{4.415623in}{4.710929in}}%
\pgfpathcurveto{\pgfqpoint{4.415623in}{4.723952in}}{\pgfqpoint{4.410449in}{4.736443in}}{\pgfqpoint{4.401241in}{4.745652in}}%
\pgfpathcurveto{\pgfqpoint{4.392032in}{4.754860in}}{\pgfqpoint{4.379541in}{4.760034in}}{\pgfqpoint{4.366518in}{4.760034in}}%
\pgfpathcurveto{\pgfqpoint{4.353496in}{4.760034in}}{\pgfqpoint{4.341005in}{4.754860in}}{\pgfqpoint{4.331796in}{4.745652in}}%
\pgfpathcurveto{\pgfqpoint{4.322588in}{4.736443in}}{\pgfqpoint{4.317414in}{4.723952in}}{\pgfqpoint{4.317414in}{4.710929in}}%
\pgfpathcurveto{\pgfqpoint{4.317414in}{4.697907in}}{\pgfqpoint{4.322588in}{4.685416in}}{\pgfqpoint{4.331796in}{4.676207in}}%
\pgfpathcurveto{\pgfqpoint{4.341005in}{4.666999in}}{\pgfqpoint{4.353496in}{4.661825in}}{\pgfqpoint{4.366518in}{4.661825in}}%
\pgfpathlineto{\pgfqpoint{4.366518in}{4.661825in}}%
\pgfpathclose%
\pgfusepath{stroke,fill}%
\end{pgfscope}%
\begin{pgfscope}%
\pgfpathrectangle{\pgfqpoint{0.786164in}{0.768110in}}{\pgfqpoint{8.851069in}{7.081890in}}%
\pgfusepath{clip}%
\pgfsetbuttcap%
\pgfsetroundjoin%
\definecolor{currentfill}{rgb}{0.231674,0.318106,0.544834}%
\pgfsetfillcolor{currentfill}%
\pgfsetfillopacity{0.700000}%
\pgfsetlinewidth{0.501875pt}%
\definecolor{currentstroke}{rgb}{1.000000,1.000000,1.000000}%
\pgfsetstrokecolor{currentstroke}%
\pgfsetstrokeopacity{0.700000}%
\pgfsetdash{}{0pt}%
\pgfpathmoveto{\pgfqpoint{4.715360in}{4.256324in}}%
\pgfpathcurveto{\pgfqpoint{4.728382in}{4.256324in}}{\pgfqpoint{4.740873in}{4.261498in}}{\pgfqpoint{4.750082in}{4.270706in}}%
\pgfpathcurveto{\pgfqpoint{4.759290in}{4.279914in}}{\pgfqpoint{4.764464in}{4.292405in}}{\pgfqpoint{4.764464in}{4.305428in}}%
\pgfpathcurveto{\pgfqpoint{4.764464in}{4.318451in}}{\pgfqpoint{4.759290in}{4.330942in}}{\pgfqpoint{4.750082in}{4.340150in}}%
\pgfpathcurveto{\pgfqpoint{4.740873in}{4.349359in}}{\pgfqpoint{4.728382in}{4.354533in}}{\pgfqpoint{4.715360in}{4.354533in}}%
\pgfpathcurveto{\pgfqpoint{4.702337in}{4.354533in}}{\pgfqpoint{4.689846in}{4.349359in}}{\pgfqpoint{4.680637in}{4.340150in}}%
\pgfpathcurveto{\pgfqpoint{4.671429in}{4.330942in}}{\pgfqpoint{4.666255in}{4.318451in}}{\pgfqpoint{4.666255in}{4.305428in}}%
\pgfpathcurveto{\pgfqpoint{4.666255in}{4.292405in}}{\pgfqpoint{4.671429in}{4.279914in}}{\pgfqpoint{4.680637in}{4.270706in}}%
\pgfpathcurveto{\pgfqpoint{4.689846in}{4.261498in}}{\pgfqpoint{4.702337in}{4.256324in}}{\pgfqpoint{4.715360in}{4.256324in}}%
\pgfpathlineto{\pgfqpoint{4.715360in}{4.256324in}}%
\pgfpathclose%
\pgfusepath{stroke,fill}%
\end{pgfscope}%
\begin{pgfscope}%
\pgfpathrectangle{\pgfqpoint{0.786164in}{0.768110in}}{\pgfqpoint{8.851069in}{7.081890in}}%
\pgfusepath{clip}%
\pgfsetbuttcap%
\pgfsetroundjoin%
\definecolor{currentfill}{rgb}{0.183898,0.422383,0.556944}%
\pgfsetfillcolor{currentfill}%
\pgfsetfillopacity{0.700000}%
\pgfsetlinewidth{0.501875pt}%
\definecolor{currentstroke}{rgb}{1.000000,1.000000,1.000000}%
\pgfsetstrokecolor{currentstroke}%
\pgfsetstrokeopacity{0.700000}%
\pgfsetdash{}{0pt}%
\pgfpathmoveto{\pgfqpoint{4.745274in}{4.683167in}}%
\pgfpathcurveto{\pgfqpoint{4.758297in}{4.683167in}}{\pgfqpoint{4.770788in}{4.688341in}}{\pgfqpoint{4.779996in}{4.697549in}}%
\pgfpathcurveto{\pgfqpoint{4.789205in}{4.706758in}}{\pgfqpoint{4.794379in}{4.719249in}}{\pgfqpoint{4.794379in}{4.732272in}}%
\pgfpathcurveto{\pgfqpoint{4.794379in}{4.745294in}}{\pgfqpoint{4.789205in}{4.757785in}}{\pgfqpoint{4.779996in}{4.766994in}}%
\pgfpathcurveto{\pgfqpoint{4.770788in}{4.776202in}}{\pgfqpoint{4.758297in}{4.781376in}}{\pgfqpoint{4.745274in}{4.781376in}}%
\pgfpathcurveto{\pgfqpoint{4.732252in}{4.781376in}}{\pgfqpoint{4.719760in}{4.776202in}}{\pgfqpoint{4.710552in}{4.766994in}}%
\pgfpathcurveto{\pgfqpoint{4.701344in}{4.757785in}}{\pgfqpoint{4.696170in}{4.745294in}}{\pgfqpoint{4.696170in}{4.732272in}}%
\pgfpathcurveto{\pgfqpoint{4.696170in}{4.719249in}}{\pgfqpoint{4.701344in}{4.706758in}}{\pgfqpoint{4.710552in}{4.697549in}}%
\pgfpathcurveto{\pgfqpoint{4.719760in}{4.688341in}}{\pgfqpoint{4.732252in}{4.683167in}}{\pgfqpoint{4.745274in}{4.683167in}}%
\pgfpathlineto{\pgfqpoint{4.745274in}{4.683167in}}%
\pgfpathclose%
\pgfusepath{stroke,fill}%
\end{pgfscope}%
\begin{pgfscope}%
\pgfpathrectangle{\pgfqpoint{0.786164in}{0.768110in}}{\pgfqpoint{8.851069in}{7.081890in}}%
\pgfusepath{clip}%
\pgfsetbuttcap%
\pgfsetroundjoin%
\definecolor{currentfill}{rgb}{0.190631,0.407061,0.556089}%
\pgfsetfillcolor{currentfill}%
\pgfsetfillopacity{0.700000}%
\pgfsetlinewidth{0.501875pt}%
\definecolor{currentstroke}{rgb}{1.000000,1.000000,1.000000}%
\pgfsetstrokecolor{currentstroke}%
\pgfsetstrokeopacity{0.700000}%
\pgfsetdash{}{0pt}%
\pgfpathmoveto{\pgfqpoint{4.782637in}{4.853904in}}%
\pgfpathcurveto{\pgfqpoint{4.795660in}{4.853904in}}{\pgfqpoint{4.808151in}{4.859078in}}{\pgfqpoint{4.817359in}{4.868287in}}%
\pgfpathcurveto{\pgfqpoint{4.826568in}{4.877495in}}{\pgfqpoint{4.831742in}{4.889986in}}{\pgfqpoint{4.831742in}{4.903009in}}%
\pgfpathcurveto{\pgfqpoint{4.831742in}{4.916032in}}{\pgfqpoint{4.826568in}{4.928523in}}{\pgfqpoint{4.817359in}{4.937731in}}%
\pgfpathcurveto{\pgfqpoint{4.808151in}{4.946940in}}{\pgfqpoint{4.795660in}{4.952114in}}{\pgfqpoint{4.782637in}{4.952114in}}%
\pgfpathcurveto{\pgfqpoint{4.769614in}{4.952114in}}{\pgfqpoint{4.757123in}{4.946940in}}{\pgfqpoint{4.747915in}{4.937731in}}%
\pgfpathcurveto{\pgfqpoint{4.738706in}{4.928523in}}{\pgfqpoint{4.733532in}{4.916032in}}{\pgfqpoint{4.733532in}{4.903009in}}%
\pgfpathcurveto{\pgfqpoint{4.733532in}{4.889986in}}{\pgfqpoint{4.738706in}{4.877495in}}{\pgfqpoint{4.747915in}{4.868287in}}%
\pgfpathcurveto{\pgfqpoint{4.757123in}{4.859078in}}{\pgfqpoint{4.769614in}{4.853904in}}{\pgfqpoint{4.782637in}{4.853904in}}%
\pgfpathlineto{\pgfqpoint{4.782637in}{4.853904in}}%
\pgfpathclose%
\pgfusepath{stroke,fill}%
\end{pgfscope}%
\begin{pgfscope}%
\pgfpathrectangle{\pgfqpoint{0.786164in}{0.768110in}}{\pgfqpoint{8.851069in}{7.081890in}}%
\pgfusepath{clip}%
\pgfsetbuttcap%
\pgfsetroundjoin%
\definecolor{currentfill}{rgb}{0.197636,0.391528,0.554969}%
\pgfsetfillcolor{currentfill}%
\pgfsetfillopacity{0.700000}%
\pgfsetlinewidth{0.501875pt}%
\definecolor{currentstroke}{rgb}{1.000000,1.000000,1.000000}%
\pgfsetstrokecolor{currentstroke}%
\pgfsetstrokeopacity{0.700000}%
\pgfsetdash{}{0pt}%
\pgfpathmoveto{\pgfqpoint{4.835384in}{5.003300in}}%
\pgfpathcurveto{\pgfqpoint{4.848407in}{5.003300in}}{\pgfqpoint{4.860898in}{5.008473in}}{\pgfqpoint{4.870107in}{5.017682in}}%
\pgfpathcurveto{\pgfqpoint{4.879315in}{5.026890in}}{\pgfqpoint{4.884489in}{5.039381in}}{\pgfqpoint{4.884489in}{5.052404in}}%
\pgfpathcurveto{\pgfqpoint{4.884489in}{5.065427in}}{\pgfqpoint{4.879315in}{5.077918in}}{\pgfqpoint{4.870107in}{5.087126in}}%
\pgfpathcurveto{\pgfqpoint{4.860898in}{5.096335in}}{\pgfqpoint{4.848407in}{5.101509in}}{\pgfqpoint{4.835384in}{5.101509in}}%
\pgfpathcurveto{\pgfqpoint{4.822362in}{5.101509in}}{\pgfqpoint{4.809871in}{5.096335in}}{\pgfqpoint{4.800662in}{5.087126in}}%
\pgfpathcurveto{\pgfqpoint{4.791454in}{5.077918in}}{\pgfqpoint{4.786280in}{5.065427in}}{\pgfqpoint{4.786280in}{5.052404in}}%
\pgfpathcurveto{\pgfqpoint{4.786280in}{5.039381in}}{\pgfqpoint{4.791454in}{5.026890in}}{\pgfqpoint{4.800662in}{5.017682in}}%
\pgfpathcurveto{\pgfqpoint{4.809871in}{5.008473in}}{\pgfqpoint{4.822362in}{5.003300in}}{\pgfqpoint{4.835384in}{5.003300in}}%
\pgfpathlineto{\pgfqpoint{4.835384in}{5.003300in}}%
\pgfpathclose%
\pgfusepath{stroke,fill}%
\end{pgfscope}%
\begin{pgfscope}%
\pgfpathrectangle{\pgfqpoint{0.786164in}{0.768110in}}{\pgfqpoint{8.851069in}{7.081890in}}%
\pgfusepath{clip}%
\pgfsetbuttcap%
\pgfsetroundjoin%
\definecolor{currentfill}{rgb}{0.231674,0.318106,0.544834}%
\pgfsetfillcolor{currentfill}%
\pgfsetfillopacity{0.700000}%
\pgfsetlinewidth{0.501875pt}%
\definecolor{currentstroke}{rgb}{1.000000,1.000000,1.000000}%
\pgfsetstrokecolor{currentstroke}%
\pgfsetstrokeopacity{0.700000}%
\pgfsetdash{}{0pt}%
\pgfpathmoveto{\pgfqpoint{5.050281in}{4.085586in}}%
\pgfpathcurveto{\pgfqpoint{5.063304in}{4.085586in}}{\pgfqpoint{5.075795in}{4.090760in}}{\pgfqpoint{5.085004in}{4.099969in}}%
\pgfpathcurveto{\pgfqpoint{5.094212in}{4.109177in}}{\pgfqpoint{5.099386in}{4.121668in}}{\pgfqpoint{5.099386in}{4.134691in}}%
\pgfpathcurveto{\pgfqpoint{5.099386in}{4.147714in}}{\pgfqpoint{5.094212in}{4.160205in}}{\pgfqpoint{5.085004in}{4.169413in}}%
\pgfpathcurveto{\pgfqpoint{5.075795in}{4.178621in}}{\pgfqpoint{5.063304in}{4.183795in}}{\pgfqpoint{5.050281in}{4.183795in}}%
\pgfpathcurveto{\pgfqpoint{5.037259in}{4.183795in}}{\pgfqpoint{5.024768in}{4.178621in}}{\pgfqpoint{5.015559in}{4.169413in}}%
\pgfpathcurveto{\pgfqpoint{5.006351in}{4.160205in}}{\pgfqpoint{5.001177in}{4.147714in}}{\pgfqpoint{5.001177in}{4.134691in}}%
\pgfpathcurveto{\pgfqpoint{5.001177in}{4.121668in}}{\pgfqpoint{5.006351in}{4.109177in}}{\pgfqpoint{5.015559in}{4.099969in}}%
\pgfpathcurveto{\pgfqpoint{5.024768in}{4.090760in}}{\pgfqpoint{5.037259in}{4.085586in}}{\pgfqpoint{5.050281in}{4.085586in}}%
\pgfpathlineto{\pgfqpoint{5.050281in}{4.085586in}}%
\pgfpathclose%
\pgfusepath{stroke,fill}%
\end{pgfscope}%
\begin{pgfscope}%
\pgfpathrectangle{\pgfqpoint{0.786164in}{0.768110in}}{\pgfqpoint{8.851069in}{7.081890in}}%
\pgfusepath{clip}%
\pgfsetbuttcap%
\pgfsetroundjoin%
\definecolor{currentfill}{rgb}{0.252194,0.269783,0.531579}%
\pgfsetfillcolor{currentfill}%
\pgfsetfillopacity{0.700000}%
\pgfsetlinewidth{0.501875pt}%
\definecolor{currentstroke}{rgb}{1.000000,1.000000,1.000000}%
\pgfsetstrokecolor{currentstroke}%
\pgfsetstrokeopacity{0.700000}%
\pgfsetdash{}{0pt}%
\pgfpathmoveto{\pgfqpoint{4.847472in}{3.210557in}}%
\pgfpathcurveto{\pgfqpoint{4.860495in}{3.210557in}}{\pgfqpoint{4.872986in}{3.215731in}}{\pgfqpoint{4.882195in}{3.224940in}}%
\pgfpathcurveto{\pgfqpoint{4.891403in}{3.234148in}}{\pgfqpoint{4.896577in}{3.246639in}}{\pgfqpoint{4.896577in}{3.259662in}}%
\pgfpathcurveto{\pgfqpoint{4.896577in}{3.272685in}}{\pgfqpoint{4.891403in}{3.285176in}}{\pgfqpoint{4.882195in}{3.294384in}}%
\pgfpathcurveto{\pgfqpoint{4.872986in}{3.303592in}}{\pgfqpoint{4.860495in}{3.308766in}}{\pgfqpoint{4.847472in}{3.308766in}}%
\pgfpathcurveto{\pgfqpoint{4.834450in}{3.308766in}}{\pgfqpoint{4.821959in}{3.303592in}}{\pgfqpoint{4.812750in}{3.294384in}}%
\pgfpathcurveto{\pgfqpoint{4.803542in}{3.285176in}}{\pgfqpoint{4.798368in}{3.272685in}}{\pgfqpoint{4.798368in}{3.259662in}}%
\pgfpathcurveto{\pgfqpoint{4.798368in}{3.246639in}}{\pgfqpoint{4.803542in}{3.234148in}}{\pgfqpoint{4.812750in}{3.224940in}}%
\pgfpathcurveto{\pgfqpoint{4.821959in}{3.215731in}}{\pgfqpoint{4.834450in}{3.210557in}}{\pgfqpoint{4.847472in}{3.210557in}}%
\pgfpathlineto{\pgfqpoint{4.847472in}{3.210557in}}%
\pgfpathclose%
\pgfusepath{stroke,fill}%
\end{pgfscope}%
\begin{pgfscope}%
\pgfpathrectangle{\pgfqpoint{0.786164in}{0.768110in}}{\pgfqpoint{8.851069in}{7.081890in}}%
\pgfusepath{clip}%
\pgfsetbuttcap%
\pgfsetroundjoin%
\definecolor{currentfill}{rgb}{0.246811,0.283237,0.535941}%
\pgfsetfillcolor{currentfill}%
\pgfsetfillopacity{0.700000}%
\pgfsetlinewidth{0.501875pt}%
\definecolor{currentstroke}{rgb}{1.000000,1.000000,1.000000}%
\pgfsetstrokecolor{currentstroke}%
\pgfsetstrokeopacity{0.700000}%
\pgfsetdash{}{0pt}%
\pgfpathmoveto{\pgfqpoint{5.747720in}{3.359952in}}%
\pgfpathcurveto{\pgfqpoint{5.760742in}{3.359952in}}{\pgfqpoint{5.773234in}{3.365126in}}{\pgfqpoint{5.782442in}{3.374335in}}%
\pgfpathcurveto{\pgfqpoint{5.791650in}{3.383543in}}{\pgfqpoint{5.796824in}{3.396034in}}{\pgfqpoint{5.796824in}{3.409057in}}%
\pgfpathcurveto{\pgfqpoint{5.796824in}{3.422080in}}{\pgfqpoint{5.791650in}{3.434571in}}{\pgfqpoint{5.782442in}{3.443779in}}%
\pgfpathcurveto{\pgfqpoint{5.773234in}{3.452988in}}{\pgfqpoint{5.760742in}{3.458162in}}{\pgfqpoint{5.747720in}{3.458162in}}%
\pgfpathcurveto{\pgfqpoint{5.734697in}{3.458162in}}{\pgfqpoint{5.722206in}{3.452988in}}{\pgfqpoint{5.712998in}{3.443779in}}%
\pgfpathcurveto{\pgfqpoint{5.703789in}{3.434571in}}{\pgfqpoint{5.698615in}{3.422080in}}{\pgfqpoint{5.698615in}{3.409057in}}%
\pgfpathcurveto{\pgfqpoint{5.698615in}{3.396034in}}{\pgfqpoint{5.703789in}{3.383543in}}{\pgfqpoint{5.712998in}{3.374335in}}%
\pgfpathcurveto{\pgfqpoint{5.722206in}{3.365126in}}{\pgfqpoint{5.734697in}{3.359952in}}{\pgfqpoint{5.747720in}{3.359952in}}%
\pgfpathlineto{\pgfqpoint{5.747720in}{3.359952in}}%
\pgfpathclose%
\pgfusepath{stroke,fill}%
\end{pgfscope}%
\begin{pgfscope}%
\pgfpathrectangle{\pgfqpoint{0.786164in}{0.768110in}}{\pgfqpoint{8.851069in}{7.081890in}}%
\pgfusepath{clip}%
\pgfsetbuttcap%
\pgfsetroundjoin%
\definecolor{currentfill}{rgb}{0.239346,0.300855,0.540844}%
\pgfsetfillcolor{currentfill}%
\pgfsetfillopacity{0.700000}%
\pgfsetlinewidth{0.501875pt}%
\definecolor{currentstroke}{rgb}{1.000000,1.000000,1.000000}%
\pgfsetstrokecolor{currentstroke}%
\pgfsetstrokeopacity{0.700000}%
\pgfsetdash{}{0pt}%
\pgfpathmoveto{\pgfqpoint{5.873483in}{3.466663in}}%
\pgfpathcurveto{\pgfqpoint{5.886506in}{3.466663in}}{\pgfqpoint{5.898997in}{3.471837in}}{\pgfqpoint{5.908206in}{3.481046in}}%
\pgfpathcurveto{\pgfqpoint{5.917414in}{3.490254in}}{\pgfqpoint{5.922588in}{3.502745in}}{\pgfqpoint{5.922588in}{3.515768in}}%
\pgfpathcurveto{\pgfqpoint{5.922588in}{3.528791in}}{\pgfqpoint{5.917414in}{3.541282in}}{\pgfqpoint{5.908206in}{3.550490in}}%
\pgfpathcurveto{\pgfqpoint{5.898997in}{3.559699in}}{\pgfqpoint{5.886506in}{3.564872in}}{\pgfqpoint{5.873483in}{3.564872in}}%
\pgfpathcurveto{\pgfqpoint{5.860461in}{3.564872in}}{\pgfqpoint{5.847970in}{3.559699in}}{\pgfqpoint{5.838761in}{3.550490in}}%
\pgfpathcurveto{\pgfqpoint{5.829553in}{3.541282in}}{\pgfqpoint{5.824379in}{3.528791in}}{\pgfqpoint{5.824379in}{3.515768in}}%
\pgfpathcurveto{\pgfqpoint{5.824379in}{3.502745in}}{\pgfqpoint{5.829553in}{3.490254in}}{\pgfqpoint{5.838761in}{3.481046in}}%
\pgfpathcurveto{\pgfqpoint{5.847970in}{3.471837in}}{\pgfqpoint{5.860461in}{3.466663in}}{\pgfqpoint{5.873483in}{3.466663in}}%
\pgfpathlineto{\pgfqpoint{5.873483in}{3.466663in}}%
\pgfpathclose%
\pgfusepath{stroke,fill}%
\end{pgfscope}%
\begin{pgfscope}%
\pgfpathrectangle{\pgfqpoint{0.786164in}{0.768110in}}{\pgfqpoint{8.851069in}{7.081890in}}%
\pgfusepath{clip}%
\pgfsetbuttcap%
\pgfsetroundjoin%
\definecolor{currentfill}{rgb}{0.275191,0.194905,0.496005}%
\pgfsetfillcolor{currentfill}%
\pgfsetfillopacity{0.700000}%
\pgfsetlinewidth{0.501875pt}%
\definecolor{currentstroke}{rgb}{1.000000,1.000000,1.000000}%
\pgfsetstrokecolor{currentstroke}%
\pgfsetstrokeopacity{0.700000}%
\pgfsetdash{}{0pt}%
\pgfpathmoveto{\pgfqpoint{2.050393in}{3.765454in}}%
\pgfpathcurveto{\pgfqpoint{2.063416in}{3.765454in}}{\pgfqpoint{2.075907in}{3.770628in}}{\pgfqpoint{2.085115in}{3.779836in}}%
\pgfpathcurveto{\pgfqpoint{2.094324in}{3.789044in}}{\pgfqpoint{2.099498in}{3.801536in}}{\pgfqpoint{2.099498in}{3.814558in}}%
\pgfpathcurveto{\pgfqpoint{2.099498in}{3.827581in}}{\pgfqpoint{2.094324in}{3.840072in}}{\pgfqpoint{2.085115in}{3.849280in}}%
\pgfpathcurveto{\pgfqpoint{2.075907in}{3.858489in}}{\pgfqpoint{2.063416in}{3.863663in}}{\pgfqpoint{2.050393in}{3.863663in}}%
\pgfpathcurveto{\pgfqpoint{2.037370in}{3.863663in}}{\pgfqpoint{2.024879in}{3.858489in}}{\pgfqpoint{2.015671in}{3.849280in}}%
\pgfpathcurveto{\pgfqpoint{2.006462in}{3.840072in}}{\pgfqpoint{2.001288in}{3.827581in}}{\pgfqpoint{2.001288in}{3.814558in}}%
\pgfpathcurveto{\pgfqpoint{2.001288in}{3.801536in}}{\pgfqpoint{2.006462in}{3.789044in}}{\pgfqpoint{2.015671in}{3.779836in}}%
\pgfpathcurveto{\pgfqpoint{2.024879in}{3.770628in}}{\pgfqpoint{2.037370in}{3.765454in}}{\pgfqpoint{2.050393in}{3.765454in}}%
\pgfpathlineto{\pgfqpoint{2.050393in}{3.765454in}}%
\pgfpathclose%
\pgfusepath{stroke,fill}%
\end{pgfscope}%
\begin{pgfscope}%
\pgfpathrectangle{\pgfqpoint{0.786164in}{0.768110in}}{\pgfqpoint{8.851069in}{7.081890in}}%
\pgfusepath{clip}%
\pgfsetbuttcap%
\pgfsetroundjoin%
\definecolor{currentfill}{rgb}{0.273006,0.204520,0.501721}%
\pgfsetfillcolor{currentfill}%
\pgfsetfillopacity{0.700000}%
\pgfsetlinewidth{0.501875pt}%
\definecolor{currentstroke}{rgb}{1.000000,1.000000,1.000000}%
\pgfsetstrokecolor{currentstroke}%
\pgfsetstrokeopacity{0.700000}%
\pgfsetdash{}{0pt}%
\pgfpathmoveto{\pgfqpoint{2.064557in}{3.829480in}}%
\pgfpathcurveto{\pgfqpoint{2.077579in}{3.829480in}}{\pgfqpoint{2.090070in}{3.834654in}}{\pgfqpoint{2.099279in}{3.843863in}}%
\pgfpathcurveto{\pgfqpoint{2.108487in}{3.853071in}}{\pgfqpoint{2.113661in}{3.865562in}}{\pgfqpoint{2.113661in}{3.878585in}}%
\pgfpathcurveto{\pgfqpoint{2.113661in}{3.891607in}}{\pgfqpoint{2.108487in}{3.904099in}}{\pgfqpoint{2.099279in}{3.913307in}}%
\pgfpathcurveto{\pgfqpoint{2.090070in}{3.922515in}}{\pgfqpoint{2.077579in}{3.927689in}}{\pgfqpoint{2.064557in}{3.927689in}}%
\pgfpathcurveto{\pgfqpoint{2.051534in}{3.927689in}}{\pgfqpoint{2.039043in}{3.922515in}}{\pgfqpoint{2.029834in}{3.913307in}}%
\pgfpathcurveto{\pgfqpoint{2.020626in}{3.904099in}}{\pgfqpoint{2.015452in}{3.891607in}}{\pgfqpoint{2.015452in}{3.878585in}}%
\pgfpathcurveto{\pgfqpoint{2.015452in}{3.865562in}}{\pgfqpoint{2.020626in}{3.853071in}}{\pgfqpoint{2.029834in}{3.843863in}}%
\pgfpathcurveto{\pgfqpoint{2.039043in}{3.834654in}}{\pgfqpoint{2.051534in}{3.829480in}}{\pgfqpoint{2.064557in}{3.829480in}}%
\pgfpathlineto{\pgfqpoint{2.064557in}{3.829480in}}%
\pgfpathclose%
\pgfusepath{stroke,fill}%
\end{pgfscope}%
\begin{pgfscope}%
\pgfpathrectangle{\pgfqpoint{0.786164in}{0.768110in}}{\pgfqpoint{8.851069in}{7.081890in}}%
\pgfusepath{clip}%
\pgfsetbuttcap%
\pgfsetroundjoin%
\definecolor{currentfill}{rgb}{0.273006,0.204520,0.501721}%
\pgfsetfillcolor{currentfill}%
\pgfsetfillopacity{0.700000}%
\pgfsetlinewidth{0.501875pt}%
\definecolor{currentstroke}{rgb}{1.000000,1.000000,1.000000}%
\pgfsetstrokecolor{currentstroke}%
\pgfsetstrokeopacity{0.700000}%
\pgfsetdash{}{0pt}%
\pgfpathmoveto{\pgfqpoint{2.086657in}{3.744111in}}%
\pgfpathcurveto{\pgfqpoint{2.099680in}{3.744111in}}{\pgfqpoint{2.112171in}{3.749285in}}{\pgfqpoint{2.121379in}{3.758494in}}%
\pgfpathcurveto{\pgfqpoint{2.130588in}{3.767702in}}{\pgfqpoint{2.135761in}{3.780193in}}{\pgfqpoint{2.135761in}{3.793216in}}%
\pgfpathcurveto{\pgfqpoint{2.135761in}{3.806239in}}{\pgfqpoint{2.130588in}{3.818730in}}{\pgfqpoint{2.121379in}{3.827938in}}%
\pgfpathcurveto{\pgfqpoint{2.112171in}{3.837147in}}{\pgfqpoint{2.099680in}{3.842321in}}{\pgfqpoint{2.086657in}{3.842321in}}%
\pgfpathcurveto{\pgfqpoint{2.073634in}{3.842321in}}{\pgfqpoint{2.061143in}{3.837147in}}{\pgfqpoint{2.051935in}{3.827938in}}%
\pgfpathcurveto{\pgfqpoint{2.042726in}{3.818730in}}{\pgfqpoint{2.037552in}{3.806239in}}{\pgfqpoint{2.037552in}{3.793216in}}%
\pgfpathcurveto{\pgfqpoint{2.037552in}{3.780193in}}{\pgfqpoint{2.042726in}{3.767702in}}{\pgfqpoint{2.051935in}{3.758494in}}%
\pgfpathcurveto{\pgfqpoint{2.061143in}{3.749285in}}{\pgfqpoint{2.073634in}{3.744111in}}{\pgfqpoint{2.086657in}{3.744111in}}%
\pgfpathlineto{\pgfqpoint{2.086657in}{3.744111in}}%
\pgfpathclose%
\pgfusepath{stroke,fill}%
\end{pgfscope}%
\begin{pgfscope}%
\pgfpathrectangle{\pgfqpoint{0.786164in}{0.768110in}}{\pgfqpoint{8.851069in}{7.081890in}}%
\pgfusepath{clip}%
\pgfsetbuttcap%
\pgfsetroundjoin%
\definecolor{currentfill}{rgb}{0.274128,0.199721,0.498911}%
\pgfsetfillcolor{currentfill}%
\pgfsetfillopacity{0.700000}%
\pgfsetlinewidth{0.501875pt}%
\definecolor{currentstroke}{rgb}{1.000000,1.000000,1.000000}%
\pgfsetstrokecolor{currentstroke}%
\pgfsetstrokeopacity{0.700000}%
\pgfsetdash{}{0pt}%
\pgfpathmoveto{\pgfqpoint{2.183238in}{3.829480in}}%
\pgfpathcurveto{\pgfqpoint{2.196261in}{3.829480in}}{\pgfqpoint{2.208752in}{3.834654in}}{\pgfqpoint{2.217961in}{3.843863in}}%
\pgfpathcurveto{\pgfqpoint{2.227169in}{3.853071in}}{\pgfqpoint{2.232343in}{3.865562in}}{\pgfqpoint{2.232343in}{3.878585in}}%
\pgfpathcurveto{\pgfqpoint{2.232343in}{3.891607in}}{\pgfqpoint{2.227169in}{3.904099in}}{\pgfqpoint{2.217961in}{3.913307in}}%
\pgfpathcurveto{\pgfqpoint{2.208752in}{3.922515in}}{\pgfqpoint{2.196261in}{3.927689in}}{\pgfqpoint{2.183238in}{3.927689in}}%
\pgfpathcurveto{\pgfqpoint{2.170216in}{3.927689in}}{\pgfqpoint{2.157725in}{3.922515in}}{\pgfqpoint{2.148516in}{3.913307in}}%
\pgfpathcurveto{\pgfqpoint{2.139308in}{3.904099in}}{\pgfqpoint{2.134134in}{3.891607in}}{\pgfqpoint{2.134134in}{3.878585in}}%
\pgfpathcurveto{\pgfqpoint{2.134134in}{3.865562in}}{\pgfqpoint{2.139308in}{3.853071in}}{\pgfqpoint{2.148516in}{3.843863in}}%
\pgfpathcurveto{\pgfqpoint{2.157725in}{3.834654in}}{\pgfqpoint{2.170216in}{3.829480in}}{\pgfqpoint{2.183238in}{3.829480in}}%
\pgfpathlineto{\pgfqpoint{2.183238in}{3.829480in}}%
\pgfpathclose%
\pgfusepath{stroke,fill}%
\end{pgfscope}%
\begin{pgfscope}%
\pgfpathrectangle{\pgfqpoint{0.786164in}{0.768110in}}{\pgfqpoint{8.851069in}{7.081890in}}%
\pgfusepath{clip}%
\pgfsetbuttcap%
\pgfsetroundjoin%
\definecolor{currentfill}{rgb}{0.273006,0.204520,0.501721}%
\pgfsetfillcolor{currentfill}%
\pgfsetfillopacity{0.700000}%
\pgfsetlinewidth{0.501875pt}%
\definecolor{currentstroke}{rgb}{1.000000,1.000000,1.000000}%
\pgfsetstrokecolor{currentstroke}%
\pgfsetstrokeopacity{0.700000}%
\pgfsetdash{}{0pt}%
\pgfpathmoveto{\pgfqpoint{2.175180in}{3.722769in}}%
\pgfpathcurveto{\pgfqpoint{2.188202in}{3.722769in}}{\pgfqpoint{2.200694in}{3.727943in}}{\pgfqpoint{2.209902in}{3.737152in}}%
\pgfpathcurveto{\pgfqpoint{2.219110in}{3.746360in}}{\pgfqpoint{2.224284in}{3.758851in}}{\pgfqpoint{2.224284in}{3.771874in}}%
\pgfpathcurveto{\pgfqpoint{2.224284in}{3.784897in}}{\pgfqpoint{2.219110in}{3.797388in}}{\pgfqpoint{2.209902in}{3.806596in}}%
\pgfpathcurveto{\pgfqpoint{2.200694in}{3.815805in}}{\pgfqpoint{2.188202in}{3.820979in}}{\pgfqpoint{2.175180in}{3.820979in}}%
\pgfpathcurveto{\pgfqpoint{2.162157in}{3.820979in}}{\pgfqpoint{2.149666in}{3.815805in}}{\pgfqpoint{2.140458in}{3.806596in}}%
\pgfpathcurveto{\pgfqpoint{2.131249in}{3.797388in}}{\pgfqpoint{2.126075in}{3.784897in}}{\pgfqpoint{2.126075in}{3.771874in}}%
\pgfpathcurveto{\pgfqpoint{2.126075in}{3.758851in}}{\pgfqpoint{2.131249in}{3.746360in}}{\pgfqpoint{2.140458in}{3.737152in}}%
\pgfpathcurveto{\pgfqpoint{2.149666in}{3.727943in}}{\pgfqpoint{2.162157in}{3.722769in}}{\pgfqpoint{2.175180in}{3.722769in}}%
\pgfpathlineto{\pgfqpoint{2.175180in}{3.722769in}}%
\pgfpathclose%
\pgfusepath{stroke,fill}%
\end{pgfscope}%
\begin{pgfscope}%
\pgfpathrectangle{\pgfqpoint{0.786164in}{0.768110in}}{\pgfqpoint{8.851069in}{7.081890in}}%
\pgfusepath{clip}%
\pgfsetbuttcap%
\pgfsetroundjoin%
\definecolor{currentfill}{rgb}{0.276194,0.190074,0.493001}%
\pgfsetfillcolor{currentfill}%
\pgfsetfillopacity{0.700000}%
\pgfsetlinewidth{0.501875pt}%
\definecolor{currentstroke}{rgb}{1.000000,1.000000,1.000000}%
\pgfsetstrokecolor{currentstroke}%
\pgfsetstrokeopacity{0.700000}%
\pgfsetdash{}{0pt}%
\pgfpathmoveto{\pgfqpoint{2.242091in}{3.573374in}}%
\pgfpathcurveto{\pgfqpoint{2.255114in}{3.573374in}}{\pgfqpoint{2.267605in}{3.578548in}}{\pgfqpoint{2.276813in}{3.587756in}}%
\pgfpathcurveto{\pgfqpoint{2.286022in}{3.596965in}}{\pgfqpoint{2.291195in}{3.609456in}}{\pgfqpoint{2.291195in}{3.622479in}}%
\pgfpathcurveto{\pgfqpoint{2.291195in}{3.635501in}}{\pgfqpoint{2.286022in}{3.647992in}}{\pgfqpoint{2.276813in}{3.657201in}}%
\pgfpathcurveto{\pgfqpoint{2.267605in}{3.666409in}}{\pgfqpoint{2.255114in}{3.671583in}}{\pgfqpoint{2.242091in}{3.671583in}}%
\pgfpathcurveto{\pgfqpoint{2.229068in}{3.671583in}}{\pgfqpoint{2.216577in}{3.666409in}}{\pgfqpoint{2.207369in}{3.657201in}}%
\pgfpathcurveto{\pgfqpoint{2.198160in}{3.647992in}}{\pgfqpoint{2.192986in}{3.635501in}}{\pgfqpoint{2.192986in}{3.622479in}}%
\pgfpathcurveto{\pgfqpoint{2.192986in}{3.609456in}}{\pgfqpoint{2.198160in}{3.596965in}}{\pgfqpoint{2.207369in}{3.587756in}}%
\pgfpathcurveto{\pgfqpoint{2.216577in}{3.578548in}}{\pgfqpoint{2.229068in}{3.573374in}}{\pgfqpoint{2.242091in}{3.573374in}}%
\pgfpathlineto{\pgfqpoint{2.242091in}{3.573374in}}%
\pgfpathclose%
\pgfusepath{stroke,fill}%
\end{pgfscope}%
\begin{pgfscope}%
\pgfpathrectangle{\pgfqpoint{0.786164in}{0.768110in}}{\pgfqpoint{8.851069in}{7.081890in}}%
\pgfusepath{clip}%
\pgfsetbuttcap%
\pgfsetroundjoin%
\definecolor{currentfill}{rgb}{0.280868,0.160771,0.472899}%
\pgfsetfillcolor{currentfill}%
\pgfsetfillopacity{0.700000}%
\pgfsetlinewidth{0.501875pt}%
\definecolor{currentstroke}{rgb}{1.000000,1.000000,1.000000}%
\pgfsetstrokecolor{currentstroke}%
\pgfsetstrokeopacity{0.700000}%
\pgfsetdash{}{0pt}%
\pgfpathmoveto{\pgfqpoint{2.406438in}{3.445321in}}%
\pgfpathcurveto{\pgfqpoint{2.419461in}{3.445321in}}{\pgfqpoint{2.431952in}{3.450495in}}{\pgfqpoint{2.441160in}{3.459703in}}%
\pgfpathcurveto{\pgfqpoint{2.450369in}{3.468912in}}{\pgfqpoint{2.455543in}{3.481403in}}{\pgfqpoint{2.455543in}{3.494426in}}%
\pgfpathcurveto{\pgfqpoint{2.455543in}{3.507448in}}{\pgfqpoint{2.450369in}{3.519939in}}{\pgfqpoint{2.441160in}{3.529148in}}%
\pgfpathcurveto{\pgfqpoint{2.431952in}{3.538356in}}{\pgfqpoint{2.419461in}{3.543530in}}{\pgfqpoint{2.406438in}{3.543530in}}%
\pgfpathcurveto{\pgfqpoint{2.393415in}{3.543530in}}{\pgfqpoint{2.380924in}{3.538356in}}{\pgfqpoint{2.371716in}{3.529148in}}%
\pgfpathcurveto{\pgfqpoint{2.362508in}{3.519939in}}{\pgfqpoint{2.357334in}{3.507448in}}{\pgfqpoint{2.357334in}{3.494426in}}%
\pgfpathcurveto{\pgfqpoint{2.357334in}{3.481403in}}{\pgfqpoint{2.362508in}{3.468912in}}{\pgfqpoint{2.371716in}{3.459703in}}%
\pgfpathcurveto{\pgfqpoint{2.380924in}{3.450495in}}{\pgfqpoint{2.393415in}{3.445321in}}{\pgfqpoint{2.406438in}{3.445321in}}%
\pgfpathlineto{\pgfqpoint{2.406438in}{3.445321in}}%
\pgfpathclose%
\pgfusepath{stroke,fill}%
\end{pgfscope}%
\begin{pgfscope}%
\pgfpathrectangle{\pgfqpoint{0.786164in}{0.768110in}}{\pgfqpoint{8.851069in}{7.081890in}}%
\pgfusepath{clip}%
\pgfsetbuttcap%
\pgfsetroundjoin%
\definecolor{currentfill}{rgb}{0.281887,0.150881,0.465405}%
\pgfsetfillcolor{currentfill}%
\pgfsetfillopacity{0.700000}%
\pgfsetlinewidth{0.501875pt}%
\definecolor{currentstroke}{rgb}{1.000000,1.000000,1.000000}%
\pgfsetstrokecolor{currentstroke}%
\pgfsetstrokeopacity{0.700000}%
\pgfsetdash{}{0pt}%
\pgfpathmoveto{\pgfqpoint{2.537818in}{3.402637in}}%
\pgfpathcurveto{\pgfqpoint{2.550841in}{3.402637in}}{\pgfqpoint{2.563332in}{3.407811in}}{\pgfqpoint{2.572541in}{3.417019in}}%
\pgfpathcurveto{\pgfqpoint{2.581749in}{3.426228in}}{\pgfqpoint{2.586923in}{3.438719in}}{\pgfqpoint{2.586923in}{3.451741in}}%
\pgfpathcurveto{\pgfqpoint{2.586923in}{3.464764in}}{\pgfqpoint{2.581749in}{3.477255in}}{\pgfqpoint{2.572541in}{3.486464in}}%
\pgfpathcurveto{\pgfqpoint{2.563332in}{3.495672in}}{\pgfqpoint{2.550841in}{3.500846in}}{\pgfqpoint{2.537818in}{3.500846in}}%
\pgfpathcurveto{\pgfqpoint{2.524796in}{3.500846in}}{\pgfqpoint{2.512305in}{3.495672in}}{\pgfqpoint{2.503096in}{3.486464in}}%
\pgfpathcurveto{\pgfqpoint{2.493888in}{3.477255in}}{\pgfqpoint{2.488714in}{3.464764in}}{\pgfqpoint{2.488714in}{3.451741in}}%
\pgfpathcurveto{\pgfqpoint{2.488714in}{3.438719in}}{\pgfqpoint{2.493888in}{3.426228in}}{\pgfqpoint{2.503096in}{3.417019in}}%
\pgfpathcurveto{\pgfqpoint{2.512305in}{3.407811in}}{\pgfqpoint{2.524796in}{3.402637in}}{\pgfqpoint{2.537818in}{3.402637in}}%
\pgfpathlineto{\pgfqpoint{2.537818in}{3.402637in}}%
\pgfpathclose%
\pgfusepath{stroke,fill}%
\end{pgfscope}%
\begin{pgfscope}%
\pgfpathrectangle{\pgfqpoint{0.786164in}{0.768110in}}{\pgfqpoint{8.851069in}{7.081890in}}%
\pgfusepath{clip}%
\pgfsetbuttcap%
\pgfsetroundjoin%
\definecolor{currentfill}{rgb}{0.281887,0.150881,0.465405}%
\pgfsetfillcolor{currentfill}%
\pgfsetfillopacity{0.700000}%
\pgfsetlinewidth{0.501875pt}%
\definecolor{currentstroke}{rgb}{1.000000,1.000000,1.000000}%
\pgfsetstrokecolor{currentstroke}%
\pgfsetstrokeopacity{0.700000}%
\pgfsetdash{}{0pt}%
\pgfpathmoveto{\pgfqpoint{2.853815in}{3.338610in}}%
\pgfpathcurveto{\pgfqpoint{2.866837in}{3.338610in}}{\pgfqpoint{2.879328in}{3.343784in}}{\pgfqpoint{2.888537in}{3.352993in}}%
\pgfpathcurveto{\pgfqpoint{2.897745in}{3.362201in}}{\pgfqpoint{2.902919in}{3.374692in}}{\pgfqpoint{2.902919in}{3.387715in}}%
\pgfpathcurveto{\pgfqpoint{2.902919in}{3.400738in}}{\pgfqpoint{2.897745in}{3.413229in}}{\pgfqpoint{2.888537in}{3.422437in}}%
\pgfpathcurveto{\pgfqpoint{2.879328in}{3.431645in}}{\pgfqpoint{2.866837in}{3.436819in}}{\pgfqpoint{2.853815in}{3.436819in}}%
\pgfpathcurveto{\pgfqpoint{2.840792in}{3.436819in}}{\pgfqpoint{2.828301in}{3.431645in}}{\pgfqpoint{2.819092in}{3.422437in}}%
\pgfpathcurveto{\pgfqpoint{2.809884in}{3.413229in}}{\pgfqpoint{2.804710in}{3.400738in}}{\pgfqpoint{2.804710in}{3.387715in}}%
\pgfpathcurveto{\pgfqpoint{2.804710in}{3.374692in}}{\pgfqpoint{2.809884in}{3.362201in}}{\pgfqpoint{2.819092in}{3.352993in}}%
\pgfpathcurveto{\pgfqpoint{2.828301in}{3.343784in}}{\pgfqpoint{2.840792in}{3.338610in}}{\pgfqpoint{2.853815in}{3.338610in}}%
\pgfpathlineto{\pgfqpoint{2.853815in}{3.338610in}}%
\pgfpathclose%
\pgfusepath{stroke,fill}%
\end{pgfscope}%
\begin{pgfscope}%
\pgfpathrectangle{\pgfqpoint{0.786164in}{0.768110in}}{\pgfqpoint{8.851069in}{7.081890in}}%
\pgfusepath{clip}%
\pgfsetbuttcap%
\pgfsetroundjoin%
\definecolor{currentfill}{rgb}{0.283197,0.115680,0.436115}%
\pgfsetfillcolor{currentfill}%
\pgfsetfillopacity{0.700000}%
\pgfsetlinewidth{0.501875pt}%
\definecolor{currentstroke}{rgb}{1.000000,1.000000,1.000000}%
\pgfsetstrokecolor{currentstroke}%
\pgfsetstrokeopacity{0.700000}%
\pgfsetdash{}{0pt}%
\pgfpathmoveto{\pgfqpoint{3.047344in}{3.210557in}}%
\pgfpathcurveto{\pgfqpoint{3.060367in}{3.210557in}}{\pgfqpoint{3.072858in}{3.215731in}}{\pgfqpoint{3.082066in}{3.224940in}}%
\pgfpathcurveto{\pgfqpoint{3.091275in}{3.234148in}}{\pgfqpoint{3.096449in}{3.246639in}}{\pgfqpoint{3.096449in}{3.259662in}}%
\pgfpathcurveto{\pgfqpoint{3.096449in}{3.272685in}}{\pgfqpoint{3.091275in}{3.285176in}}{\pgfqpoint{3.082066in}{3.294384in}}%
\pgfpathcurveto{\pgfqpoint{3.072858in}{3.303592in}}{\pgfqpoint{3.060367in}{3.308766in}}{\pgfqpoint{3.047344in}{3.308766in}}%
\pgfpathcurveto{\pgfqpoint{3.034321in}{3.308766in}}{\pgfqpoint{3.021830in}{3.303592in}}{\pgfqpoint{3.012622in}{3.294384in}}%
\pgfpathcurveto{\pgfqpoint{3.003413in}{3.285176in}}{\pgfqpoint{2.998239in}{3.272685in}}{\pgfqpoint{2.998239in}{3.259662in}}%
\pgfpathcurveto{\pgfqpoint{2.998239in}{3.246639in}}{\pgfqpoint{3.003413in}{3.234148in}}{\pgfqpoint{3.012622in}{3.224940in}}%
\pgfpathcurveto{\pgfqpoint{3.021830in}{3.215731in}}{\pgfqpoint{3.034321in}{3.210557in}}{\pgfqpoint{3.047344in}{3.210557in}}%
\pgfpathlineto{\pgfqpoint{3.047344in}{3.210557in}}%
\pgfpathclose%
\pgfusepath{stroke,fill}%
\end{pgfscope}%
\begin{pgfscope}%
\pgfpathrectangle{\pgfqpoint{0.786164in}{0.768110in}}{\pgfqpoint{8.851069in}{7.081890in}}%
\pgfusepath{clip}%
\pgfsetbuttcap%
\pgfsetroundjoin%
\definecolor{currentfill}{rgb}{0.283197,0.115680,0.436115}%
\pgfsetfillcolor{currentfill}%
\pgfsetfillopacity{0.700000}%
\pgfsetlinewidth{0.501875pt}%
\definecolor{currentstroke}{rgb}{1.000000,1.000000,1.000000}%
\pgfsetstrokecolor{currentstroke}%
\pgfsetstrokeopacity{0.700000}%
\pgfsetdash{}{0pt}%
\pgfpathmoveto{\pgfqpoint{3.090934in}{3.189215in}}%
\pgfpathcurveto{\pgfqpoint{3.103957in}{3.189215in}}{\pgfqpoint{3.116448in}{3.194389in}}{\pgfqpoint{3.125656in}{3.203597in}}%
\pgfpathcurveto{\pgfqpoint{3.134865in}{3.212806in}}{\pgfqpoint{3.140039in}{3.225297in}}{\pgfqpoint{3.140039in}{3.238320in}}%
\pgfpathcurveto{\pgfqpoint{3.140039in}{3.251342in}}{\pgfqpoint{3.134865in}{3.263833in}}{\pgfqpoint{3.125656in}{3.273042in}}%
\pgfpathcurveto{\pgfqpoint{3.116448in}{3.282250in}}{\pgfqpoint{3.103957in}{3.287424in}}{\pgfqpoint{3.090934in}{3.287424in}}%
\pgfpathcurveto{\pgfqpoint{3.077911in}{3.287424in}}{\pgfqpoint{3.065420in}{3.282250in}}{\pgfqpoint{3.056212in}{3.273042in}}%
\pgfpathcurveto{\pgfqpoint{3.047003in}{3.263833in}}{\pgfqpoint{3.041829in}{3.251342in}}{\pgfqpoint{3.041829in}{3.238320in}}%
\pgfpathcurveto{\pgfqpoint{3.041829in}{3.225297in}}{\pgfqpoint{3.047003in}{3.212806in}}{\pgfqpoint{3.056212in}{3.203597in}}%
\pgfpathcurveto{\pgfqpoint{3.065420in}{3.194389in}}{\pgfqpoint{3.077911in}{3.189215in}}{\pgfqpoint{3.090934in}{3.189215in}}%
\pgfpathlineto{\pgfqpoint{3.090934in}{3.189215in}}%
\pgfpathclose%
\pgfusepath{stroke,fill}%
\end{pgfscope}%
\begin{pgfscope}%
\pgfpathrectangle{\pgfqpoint{0.786164in}{0.768110in}}{\pgfqpoint{8.851069in}{7.081890in}}%
\pgfusepath{clip}%
\pgfsetbuttcap%
\pgfsetroundjoin%
\definecolor{currentfill}{rgb}{0.283197,0.115680,0.436115}%
\pgfsetfillcolor{currentfill}%
\pgfsetfillopacity{0.700000}%
\pgfsetlinewidth{0.501875pt}%
\definecolor{currentstroke}{rgb}{1.000000,1.000000,1.000000}%
\pgfsetstrokecolor{currentstroke}%
\pgfsetstrokeopacity{0.700000}%
\pgfsetdash{}{0pt}%
\pgfpathmoveto{\pgfqpoint{3.091789in}{3.061162in}}%
\pgfpathcurveto{\pgfqpoint{3.104811in}{3.061162in}}{\pgfqpoint{3.117302in}{3.066336in}}{\pgfqpoint{3.126511in}{3.075544in}}%
\pgfpathcurveto{\pgfqpoint{3.135719in}{3.084753in}}{\pgfqpoint{3.140893in}{3.097244in}}{\pgfqpoint{3.140893in}{3.110267in}}%
\pgfpathcurveto{\pgfqpoint{3.140893in}{3.123289in}}{\pgfqpoint{3.135719in}{3.135780in}}{\pgfqpoint{3.126511in}{3.144989in}}%
\pgfpathcurveto{\pgfqpoint{3.117302in}{3.154197in}}{\pgfqpoint{3.104811in}{3.159371in}}{\pgfqpoint{3.091789in}{3.159371in}}%
\pgfpathcurveto{\pgfqpoint{3.078766in}{3.159371in}}{\pgfqpoint{3.066275in}{3.154197in}}{\pgfqpoint{3.057066in}{3.144989in}}%
\pgfpathcurveto{\pgfqpoint{3.047858in}{3.135780in}}{\pgfqpoint{3.042684in}{3.123289in}}{\pgfqpoint{3.042684in}{3.110267in}}%
\pgfpathcurveto{\pgfqpoint{3.042684in}{3.097244in}}{\pgfqpoint{3.047858in}{3.084753in}}{\pgfqpoint{3.057066in}{3.075544in}}%
\pgfpathcurveto{\pgfqpoint{3.066275in}{3.066336in}}{\pgfqpoint{3.078766in}{3.061162in}}{\pgfqpoint{3.091789in}{3.061162in}}%
\pgfpathlineto{\pgfqpoint{3.091789in}{3.061162in}}%
\pgfpathclose%
\pgfusepath{stroke,fill}%
\end{pgfscope}%
\begin{pgfscope}%
\pgfpathrectangle{\pgfqpoint{0.786164in}{0.768110in}}{\pgfqpoint{8.851069in}{7.081890in}}%
\pgfusepath{clip}%
\pgfsetbuttcap%
\pgfsetroundjoin%
\definecolor{currentfill}{rgb}{0.283197,0.115680,0.436115}%
\pgfsetfillcolor{currentfill}%
\pgfsetfillopacity{0.700000}%
\pgfsetlinewidth{0.501875pt}%
\definecolor{currentstroke}{rgb}{1.000000,1.000000,1.000000}%
\pgfsetstrokecolor{currentstroke}%
\pgfsetstrokeopacity{0.700000}%
\pgfsetdash{}{0pt}%
\pgfpathmoveto{\pgfqpoint{3.055158in}{2.997135in}}%
\pgfpathcurveto{\pgfqpoint{3.068181in}{2.997135in}}{\pgfqpoint{3.080672in}{3.002309in}}{\pgfqpoint{3.089881in}{3.011518in}}%
\pgfpathcurveto{\pgfqpoint{3.099089in}{3.020726in}}{\pgfqpoint{3.104263in}{3.033217in}}{\pgfqpoint{3.104263in}{3.046240in}}%
\pgfpathcurveto{\pgfqpoint{3.104263in}{3.059263in}}{\pgfqpoint{3.099089in}{3.071754in}}{\pgfqpoint{3.089881in}{3.080962in}}%
\pgfpathcurveto{\pgfqpoint{3.080672in}{3.090171in}}{\pgfqpoint{3.068181in}{3.095345in}}{\pgfqpoint{3.055158in}{3.095345in}}%
\pgfpathcurveto{\pgfqpoint{3.042136in}{3.095345in}}{\pgfqpoint{3.029645in}{3.090171in}}{\pgfqpoint{3.020436in}{3.080962in}}%
\pgfpathcurveto{\pgfqpoint{3.011228in}{3.071754in}}{\pgfqpoint{3.006054in}{3.059263in}}{\pgfqpoint{3.006054in}{3.046240in}}%
\pgfpathcurveto{\pgfqpoint{3.006054in}{3.033217in}}{\pgfqpoint{3.011228in}{3.020726in}}{\pgfqpoint{3.020436in}{3.011518in}}%
\pgfpathcurveto{\pgfqpoint{3.029645in}{3.002309in}}{\pgfqpoint{3.042136in}{2.997135in}}{\pgfqpoint{3.055158in}{2.997135in}}%
\pgfpathlineto{\pgfqpoint{3.055158in}{2.997135in}}%
\pgfpathclose%
\pgfusepath{stroke,fill}%
\end{pgfscope}%
\begin{pgfscope}%
\pgfpathrectangle{\pgfqpoint{0.786164in}{0.768110in}}{\pgfqpoint{8.851069in}{7.081890in}}%
\pgfusepath{clip}%
\pgfsetbuttcap%
\pgfsetroundjoin%
\definecolor{currentfill}{rgb}{0.283197,0.115680,0.436115}%
\pgfsetfillcolor{currentfill}%
\pgfsetfillopacity{0.700000}%
\pgfsetlinewidth{0.501875pt}%
\definecolor{currentstroke}{rgb}{1.000000,1.000000,1.000000}%
\pgfsetstrokecolor{currentstroke}%
\pgfsetstrokeopacity{0.700000}%
\pgfsetdash{}{0pt}%
\pgfpathmoveto{\pgfqpoint{3.288370in}{3.082504in}}%
\pgfpathcurveto{\pgfqpoint{3.301393in}{3.082504in}}{\pgfqpoint{3.313884in}{3.087678in}}{\pgfqpoint{3.323093in}{3.096887in}}%
\pgfpathcurveto{\pgfqpoint{3.332301in}{3.106095in}}{\pgfqpoint{3.337475in}{3.118586in}}{\pgfqpoint{3.337475in}{3.131609in}}%
\pgfpathcurveto{\pgfqpoint{3.337475in}{3.144631in}}{\pgfqpoint{3.332301in}{3.157123in}}{\pgfqpoint{3.323093in}{3.166331in}}%
\pgfpathcurveto{\pgfqpoint{3.313884in}{3.175539in}}{\pgfqpoint{3.301393in}{3.180713in}}{\pgfqpoint{3.288370in}{3.180713in}}%
\pgfpathcurveto{\pgfqpoint{3.275348in}{3.180713in}}{\pgfqpoint{3.262857in}{3.175539in}}{\pgfqpoint{3.253648in}{3.166331in}}%
\pgfpathcurveto{\pgfqpoint{3.244440in}{3.157123in}}{\pgfqpoint{3.239266in}{3.144631in}}{\pgfqpoint{3.239266in}{3.131609in}}%
\pgfpathcurveto{\pgfqpoint{3.239266in}{3.118586in}}{\pgfqpoint{3.244440in}{3.106095in}}{\pgfqpoint{3.253648in}{3.096887in}}%
\pgfpathcurveto{\pgfqpoint{3.262857in}{3.087678in}}{\pgfqpoint{3.275348in}{3.082504in}}{\pgfqpoint{3.288370in}{3.082504in}}%
\pgfpathlineto{\pgfqpoint{3.288370in}{3.082504in}}%
\pgfpathclose%
\pgfusepath{stroke,fill}%
\end{pgfscope}%
\begin{pgfscope}%
\pgfpathrectangle{\pgfqpoint{0.786164in}{0.768110in}}{\pgfqpoint{8.851069in}{7.081890in}}%
\pgfusepath{clip}%
\pgfsetbuttcap%
\pgfsetroundjoin%
\definecolor{currentfill}{rgb}{0.283091,0.110553,0.431554}%
\pgfsetfillcolor{currentfill}%
\pgfsetfillopacity{0.700000}%
\pgfsetlinewidth{0.501875pt}%
\definecolor{currentstroke}{rgb}{1.000000,1.000000,1.000000}%
\pgfsetstrokecolor{currentstroke}%
\pgfsetstrokeopacity{0.700000}%
\pgfsetdash{}{0pt}%
\pgfpathmoveto{\pgfqpoint{3.373963in}{3.018478in}}%
\pgfpathcurveto{\pgfqpoint{3.386986in}{3.018478in}}{\pgfqpoint{3.399477in}{3.023652in}}{\pgfqpoint{3.408685in}{3.032860in}}%
\pgfpathcurveto{\pgfqpoint{3.417894in}{3.042068in}}{\pgfqpoint{3.423068in}{3.054560in}}{\pgfqpoint{3.423068in}{3.067582in}}%
\pgfpathcurveto{\pgfqpoint{3.423068in}{3.080605in}}{\pgfqpoint{3.417894in}{3.093096in}}{\pgfqpoint{3.408685in}{3.102305in}}%
\pgfpathcurveto{\pgfqpoint{3.399477in}{3.111513in}}{\pgfqpoint{3.386986in}{3.116687in}}{\pgfqpoint{3.373963in}{3.116687in}}%
\pgfpathcurveto{\pgfqpoint{3.360940in}{3.116687in}}{\pgfqpoint{3.348449in}{3.111513in}}{\pgfqpoint{3.339241in}{3.102305in}}%
\pgfpathcurveto{\pgfqpoint{3.330032in}{3.093096in}}{\pgfqpoint{3.324858in}{3.080605in}}{\pgfqpoint{3.324858in}{3.067582in}}%
\pgfpathcurveto{\pgfqpoint{3.324858in}{3.054560in}}{\pgfqpoint{3.330032in}{3.042068in}}{\pgfqpoint{3.339241in}{3.032860in}}%
\pgfpathcurveto{\pgfqpoint{3.348449in}{3.023652in}}{\pgfqpoint{3.360940in}{3.018478in}}{\pgfqpoint{3.373963in}{3.018478in}}%
\pgfpathlineto{\pgfqpoint{3.373963in}{3.018478in}}%
\pgfpathclose%
\pgfusepath{stroke,fill}%
\end{pgfscope}%
\begin{pgfscope}%
\pgfpathrectangle{\pgfqpoint{0.786164in}{0.768110in}}{\pgfqpoint{8.851069in}{7.081890in}}%
\pgfusepath{clip}%
\pgfsetbuttcap%
\pgfsetroundjoin%
\definecolor{currentfill}{rgb}{0.282910,0.105393,0.426902}%
\pgfsetfillcolor{currentfill}%
\pgfsetfillopacity{0.700000}%
\pgfsetlinewidth{0.501875pt}%
\definecolor{currentstroke}{rgb}{1.000000,1.000000,1.000000}%
\pgfsetstrokecolor{currentstroke}%
\pgfsetstrokeopacity{0.700000}%
\pgfsetdash{}{0pt}%
\pgfpathmoveto{\pgfqpoint{3.573231in}{2.869082in}}%
\pgfpathcurveto{\pgfqpoint{3.586254in}{2.869082in}}{\pgfqpoint{3.598745in}{2.874256in}}{\pgfqpoint{3.607953in}{2.883465in}}%
\pgfpathcurveto{\pgfqpoint{3.617162in}{2.892673in}}{\pgfqpoint{3.622336in}{2.905164in}}{\pgfqpoint{3.622336in}{2.918187in}}%
\pgfpathcurveto{\pgfqpoint{3.622336in}{2.931210in}}{\pgfqpoint{3.617162in}{2.943701in}}{\pgfqpoint{3.607953in}{2.952909in}}%
\pgfpathcurveto{\pgfqpoint{3.598745in}{2.962118in}}{\pgfqpoint{3.586254in}{2.967292in}}{\pgfqpoint{3.573231in}{2.967292in}}%
\pgfpathcurveto{\pgfqpoint{3.560208in}{2.967292in}}{\pgfqpoint{3.547717in}{2.962118in}}{\pgfqpoint{3.538509in}{2.952909in}}%
\pgfpathcurveto{\pgfqpoint{3.529300in}{2.943701in}}{\pgfqpoint{3.524126in}{2.931210in}}{\pgfqpoint{3.524126in}{2.918187in}}%
\pgfpathcurveto{\pgfqpoint{3.524126in}{2.905164in}}{\pgfqpoint{3.529300in}{2.892673in}}{\pgfqpoint{3.538509in}{2.883465in}}%
\pgfpathcurveto{\pgfqpoint{3.547717in}{2.874256in}}{\pgfqpoint{3.560208in}{2.869082in}}{\pgfqpoint{3.573231in}{2.869082in}}%
\pgfpathlineto{\pgfqpoint{3.573231in}{2.869082in}}%
\pgfpathclose%
\pgfusepath{stroke,fill}%
\end{pgfscope}%
\begin{pgfscope}%
\pgfpathrectangle{\pgfqpoint{0.786164in}{0.768110in}}{\pgfqpoint{8.851069in}{7.081890in}}%
\pgfusepath{clip}%
\pgfsetbuttcap%
\pgfsetroundjoin%
\definecolor{currentfill}{rgb}{0.278791,0.062145,0.386592}%
\pgfsetfillcolor{currentfill}%
\pgfsetfillopacity{0.700000}%
\pgfsetlinewidth{0.501875pt}%
\definecolor{currentstroke}{rgb}{1.000000,1.000000,1.000000}%
\pgfsetstrokecolor{currentstroke}%
\pgfsetstrokeopacity{0.700000}%
\pgfsetdash{}{0pt}%
\pgfpathmoveto{\pgfqpoint{3.831596in}{2.612976in}}%
\pgfpathcurveto{\pgfqpoint{3.844619in}{2.612976in}}{\pgfqpoint{3.857110in}{2.618150in}}{\pgfqpoint{3.866318in}{2.627359in}}%
\pgfpathcurveto{\pgfqpoint{3.875526in}{2.636567in}}{\pgfqpoint{3.880700in}{2.649058in}}{\pgfqpoint{3.880700in}{2.662081in}}%
\pgfpathcurveto{\pgfqpoint{3.880700in}{2.675104in}}{\pgfqpoint{3.875526in}{2.687595in}}{\pgfqpoint{3.866318in}{2.696803in}}%
\pgfpathcurveto{\pgfqpoint{3.857110in}{2.706012in}}{\pgfqpoint{3.844619in}{2.711186in}}{\pgfqpoint{3.831596in}{2.711186in}}%
\pgfpathcurveto{\pgfqpoint{3.818573in}{2.711186in}}{\pgfqpoint{3.806082in}{2.706012in}}{\pgfqpoint{3.796874in}{2.696803in}}%
\pgfpathcurveto{\pgfqpoint{3.787665in}{2.687595in}}{\pgfqpoint{3.782491in}{2.675104in}}{\pgfqpoint{3.782491in}{2.662081in}}%
\pgfpathcurveto{\pgfqpoint{3.782491in}{2.649058in}}{\pgfqpoint{3.787665in}{2.636567in}}{\pgfqpoint{3.796874in}{2.627359in}}%
\pgfpathcurveto{\pgfqpoint{3.806082in}{2.618150in}}{\pgfqpoint{3.818573in}{2.612976in}}{\pgfqpoint{3.831596in}{2.612976in}}%
\pgfpathlineto{\pgfqpoint{3.831596in}{2.612976in}}%
\pgfpathclose%
\pgfusepath{stroke,fill}%
\end{pgfscope}%
\begin{pgfscope}%
\pgfpathrectangle{\pgfqpoint{0.786164in}{0.768110in}}{\pgfqpoint{8.851069in}{7.081890in}}%
\pgfusepath{clip}%
\pgfsetbuttcap%
\pgfsetroundjoin%
\definecolor{currentfill}{rgb}{0.280894,0.078907,0.402329}%
\pgfsetfillcolor{currentfill}%
\pgfsetfillopacity{0.700000}%
\pgfsetlinewidth{0.501875pt}%
\definecolor{currentstroke}{rgb}{1.000000,1.000000,1.000000}%
\pgfsetstrokecolor{currentstroke}%
\pgfsetstrokeopacity{0.700000}%
\pgfsetdash{}{0pt}%
\pgfpathmoveto{\pgfqpoint{3.864319in}{2.698345in}}%
\pgfpathcurveto{\pgfqpoint{3.877341in}{2.698345in}}{\pgfqpoint{3.889833in}{2.703519in}}{\pgfqpoint{3.899041in}{2.712727in}}%
\pgfpathcurveto{\pgfqpoint{3.908249in}{2.721936in}}{\pgfqpoint{3.913423in}{2.734427in}}{\pgfqpoint{3.913423in}{2.747450in}}%
\pgfpathcurveto{\pgfqpoint{3.913423in}{2.760472in}}{\pgfqpoint{3.908249in}{2.772964in}}{\pgfqpoint{3.899041in}{2.782172in}}%
\pgfpathcurveto{\pgfqpoint{3.889833in}{2.791380in}}{\pgfqpoint{3.877341in}{2.796554in}}{\pgfqpoint{3.864319in}{2.796554in}}%
\pgfpathcurveto{\pgfqpoint{3.851296in}{2.796554in}}{\pgfqpoint{3.838805in}{2.791380in}}{\pgfqpoint{3.829597in}{2.782172in}}%
\pgfpathcurveto{\pgfqpoint{3.820388in}{2.772964in}}{\pgfqpoint{3.815214in}{2.760472in}}{\pgfqpoint{3.815214in}{2.747450in}}%
\pgfpathcurveto{\pgfqpoint{3.815214in}{2.734427in}}{\pgfqpoint{3.820388in}{2.721936in}}{\pgfqpoint{3.829597in}{2.712727in}}%
\pgfpathcurveto{\pgfqpoint{3.838805in}{2.703519in}}{\pgfqpoint{3.851296in}{2.698345in}}{\pgfqpoint{3.864319in}{2.698345in}}%
\pgfpathlineto{\pgfqpoint{3.864319in}{2.698345in}}%
\pgfpathclose%
\pgfusepath{stroke,fill}%
\end{pgfscope}%
\begin{pgfscope}%
\pgfpathrectangle{\pgfqpoint{0.786164in}{0.768110in}}{\pgfqpoint{8.851069in}{7.081890in}}%
\pgfusepath{clip}%
\pgfsetbuttcap%
\pgfsetroundjoin%
\definecolor{currentfill}{rgb}{0.282327,0.094955,0.417331}%
\pgfsetfillcolor{currentfill}%
\pgfsetfillopacity{0.700000}%
\pgfsetlinewidth{0.501875pt}%
\definecolor{currentstroke}{rgb}{1.000000,1.000000,1.000000}%
\pgfsetstrokecolor{currentstroke}%
\pgfsetstrokeopacity{0.700000}%
\pgfsetdash{}{0pt}%
\pgfpathmoveto{\pgfqpoint{3.945027in}{2.762372in}}%
\pgfpathcurveto{\pgfqpoint{3.958050in}{2.762372in}}{\pgfqpoint{3.970541in}{2.767546in}}{\pgfqpoint{3.979749in}{2.776754in}}%
\pgfpathcurveto{\pgfqpoint{3.988958in}{2.785962in}}{\pgfqpoint{3.994132in}{2.798454in}}{\pgfqpoint{3.994132in}{2.811476in}}%
\pgfpathcurveto{\pgfqpoint{3.994132in}{2.824499in}}{\pgfqpoint{3.988958in}{2.836990in}}{\pgfqpoint{3.979749in}{2.846198in}}%
\pgfpathcurveto{\pgfqpoint{3.970541in}{2.855407in}}{\pgfqpoint{3.958050in}{2.860581in}}{\pgfqpoint{3.945027in}{2.860581in}}%
\pgfpathcurveto{\pgfqpoint{3.932005in}{2.860581in}}{\pgfqpoint{3.919513in}{2.855407in}}{\pgfqpoint{3.910305in}{2.846198in}}%
\pgfpathcurveto{\pgfqpoint{3.901097in}{2.836990in}}{\pgfqpoint{3.895923in}{2.824499in}}{\pgfqpoint{3.895923in}{2.811476in}}%
\pgfpathcurveto{\pgfqpoint{3.895923in}{2.798454in}}{\pgfqpoint{3.901097in}{2.785962in}}{\pgfqpoint{3.910305in}{2.776754in}}%
\pgfpathcurveto{\pgfqpoint{3.919513in}{2.767546in}}{\pgfqpoint{3.932005in}{2.762372in}}{\pgfqpoint{3.945027in}{2.762372in}}%
\pgfpathlineto{\pgfqpoint{3.945027in}{2.762372in}}%
\pgfpathclose%
\pgfusepath{stroke,fill}%
\end{pgfscope}%
\begin{pgfscope}%
\pgfpathrectangle{\pgfqpoint{0.786164in}{0.768110in}}{\pgfqpoint{8.851069in}{7.081890in}}%
\pgfusepath{clip}%
\pgfsetbuttcap%
\pgfsetroundjoin%
\definecolor{currentfill}{rgb}{0.260571,0.246922,0.522828}%
\pgfsetfillcolor{currentfill}%
\pgfsetfillopacity{0.700000}%
\pgfsetlinewidth{0.501875pt}%
\definecolor{currentstroke}{rgb}{1.000000,1.000000,1.000000}%
\pgfsetstrokecolor{currentstroke}%
\pgfsetstrokeopacity{0.700000}%
\pgfsetdash{}{0pt}%
\pgfpathmoveto{\pgfqpoint{2.194960in}{3.125188in}}%
\pgfpathcurveto{\pgfqpoint{2.207983in}{3.125188in}}{\pgfqpoint{2.220474in}{3.130362in}}{\pgfqpoint{2.229682in}{3.139571in}}%
\pgfpathcurveto{\pgfqpoint{2.238891in}{3.148779in}}{\pgfqpoint{2.244065in}{3.161270in}}{\pgfqpoint{2.244065in}{3.174293in}}%
\pgfpathcurveto{\pgfqpoint{2.244065in}{3.187316in}}{\pgfqpoint{2.238891in}{3.199807in}}{\pgfqpoint{2.229682in}{3.209015in}}%
\pgfpathcurveto{\pgfqpoint{2.220474in}{3.218224in}}{\pgfqpoint{2.207983in}{3.223398in}}{\pgfqpoint{2.194960in}{3.223398in}}%
\pgfpathcurveto{\pgfqpoint{2.181937in}{3.223398in}}{\pgfqpoint{2.169446in}{3.218224in}}{\pgfqpoint{2.160238in}{3.209015in}}%
\pgfpathcurveto{\pgfqpoint{2.151029in}{3.199807in}}{\pgfqpoint{2.145855in}{3.187316in}}{\pgfqpoint{2.145855in}{3.174293in}}%
\pgfpathcurveto{\pgfqpoint{2.145855in}{3.161270in}}{\pgfqpoint{2.151029in}{3.148779in}}{\pgfqpoint{2.160238in}{3.139571in}}%
\pgfpathcurveto{\pgfqpoint{2.169446in}{3.130362in}}{\pgfqpoint{2.181937in}{3.125188in}}{\pgfqpoint{2.194960in}{3.125188in}}%
\pgfpathlineto{\pgfqpoint{2.194960in}{3.125188in}}%
\pgfpathclose%
\pgfusepath{stroke,fill}%
\end{pgfscope}%
\begin{pgfscope}%
\pgfpathrectangle{\pgfqpoint{0.786164in}{0.768110in}}{\pgfqpoint{8.851069in}{7.081890in}}%
\pgfusepath{clip}%
\pgfsetbuttcap%
\pgfsetroundjoin%
\definecolor{currentfill}{rgb}{0.258965,0.251537,0.524736}%
\pgfsetfillcolor{currentfill}%
\pgfsetfillopacity{0.700000}%
\pgfsetlinewidth{0.501875pt}%
\definecolor{currentstroke}{rgb}{1.000000,1.000000,1.000000}%
\pgfsetstrokecolor{currentstroke}%
\pgfsetstrokeopacity{0.700000}%
\pgfsetdash{}{0pt}%
\pgfpathmoveto{\pgfqpoint{2.207048in}{3.146531in}}%
\pgfpathcurveto{\pgfqpoint{2.220071in}{3.146531in}}{\pgfqpoint{2.232562in}{3.151705in}}{\pgfqpoint{2.241770in}{3.160913in}}%
\pgfpathcurveto{\pgfqpoint{2.250979in}{3.170122in}}{\pgfqpoint{2.256153in}{3.182613in}}{\pgfqpoint{2.256153in}{3.195635in}}%
\pgfpathcurveto{\pgfqpoint{2.256153in}{3.208658in}}{\pgfqpoint{2.250979in}{3.221149in}}{\pgfqpoint{2.241770in}{3.230358in}}%
\pgfpathcurveto{\pgfqpoint{2.232562in}{3.239566in}}{\pgfqpoint{2.220071in}{3.244740in}}{\pgfqpoint{2.207048in}{3.244740in}}%
\pgfpathcurveto{\pgfqpoint{2.194025in}{3.244740in}}{\pgfqpoint{2.181534in}{3.239566in}}{\pgfqpoint{2.172326in}{3.230358in}}%
\pgfpathcurveto{\pgfqpoint{2.163117in}{3.221149in}}{\pgfqpoint{2.157943in}{3.208658in}}{\pgfqpoint{2.157943in}{3.195635in}}%
\pgfpathcurveto{\pgfqpoint{2.157943in}{3.182613in}}{\pgfqpoint{2.163117in}{3.170122in}}{\pgfqpoint{2.172326in}{3.160913in}}%
\pgfpathcurveto{\pgfqpoint{2.181534in}{3.151705in}}{\pgfqpoint{2.194025in}{3.146531in}}{\pgfqpoint{2.207048in}{3.146531in}}%
\pgfpathlineto{\pgfqpoint{2.207048in}{3.146531in}}%
\pgfpathclose%
\pgfusepath{stroke,fill}%
\end{pgfscope}%
\begin{pgfscope}%
\pgfpathrectangle{\pgfqpoint{0.786164in}{0.768110in}}{\pgfqpoint{8.851069in}{7.081890in}}%
\pgfusepath{clip}%
\pgfsetbuttcap%
\pgfsetroundjoin%
\definecolor{currentfill}{rgb}{0.260571,0.246922,0.522828}%
\pgfsetfillcolor{currentfill}%
\pgfsetfillopacity{0.700000}%
\pgfsetlinewidth{0.501875pt}%
\definecolor{currentstroke}{rgb}{1.000000,1.000000,1.000000}%
\pgfsetstrokecolor{currentstroke}%
\pgfsetstrokeopacity{0.700000}%
\pgfsetdash{}{0pt}%
\pgfpathmoveto{\pgfqpoint{2.293984in}{3.082504in}}%
\pgfpathcurveto{\pgfqpoint{2.307006in}{3.082504in}}{\pgfqpoint{2.319497in}{3.087678in}}{\pgfqpoint{2.328706in}{3.096887in}}%
\pgfpathcurveto{\pgfqpoint{2.337914in}{3.106095in}}{\pgfqpoint{2.343088in}{3.118586in}}{\pgfqpoint{2.343088in}{3.131609in}}%
\pgfpathcurveto{\pgfqpoint{2.343088in}{3.144631in}}{\pgfqpoint{2.337914in}{3.157123in}}{\pgfqpoint{2.328706in}{3.166331in}}%
\pgfpathcurveto{\pgfqpoint{2.319497in}{3.175539in}}{\pgfqpoint{2.307006in}{3.180713in}}{\pgfqpoint{2.293984in}{3.180713in}}%
\pgfpathcurveto{\pgfqpoint{2.280961in}{3.180713in}}{\pgfqpoint{2.268470in}{3.175539in}}{\pgfqpoint{2.259261in}{3.166331in}}%
\pgfpathcurveto{\pgfqpoint{2.250053in}{3.157123in}}{\pgfqpoint{2.244879in}{3.144631in}}{\pgfqpoint{2.244879in}{3.131609in}}%
\pgfpathcurveto{\pgfqpoint{2.244879in}{3.118586in}}{\pgfqpoint{2.250053in}{3.106095in}}{\pgfqpoint{2.259261in}{3.096887in}}%
\pgfpathcurveto{\pgfqpoint{2.268470in}{3.087678in}}{\pgfqpoint{2.280961in}{3.082504in}}{\pgfqpoint{2.293984in}{3.082504in}}%
\pgfpathlineto{\pgfqpoint{2.293984in}{3.082504in}}%
\pgfpathclose%
\pgfusepath{stroke,fill}%
\end{pgfscope}%
\begin{pgfscope}%
\pgfpathrectangle{\pgfqpoint{0.786164in}{0.768110in}}{\pgfqpoint{8.851069in}{7.081890in}}%
\pgfusepath{clip}%
\pgfsetbuttcap%
\pgfsetroundjoin%
\definecolor{currentfill}{rgb}{0.260571,0.246922,0.522828}%
\pgfsetfillcolor{currentfill}%
\pgfsetfillopacity{0.700000}%
\pgfsetlinewidth{0.501875pt}%
\definecolor{currentstroke}{rgb}{1.000000,1.000000,1.000000}%
\pgfsetstrokecolor{currentstroke}%
\pgfsetstrokeopacity{0.700000}%
\pgfsetdash{}{0pt}%
\pgfpathmoveto{\pgfqpoint{2.356377in}{3.125188in}}%
\pgfpathcurveto{\pgfqpoint{2.369400in}{3.125188in}}{\pgfqpoint{2.381891in}{3.130362in}}{\pgfqpoint{2.391099in}{3.139571in}}%
\pgfpathcurveto{\pgfqpoint{2.400308in}{3.148779in}}{\pgfqpoint{2.405482in}{3.161270in}}{\pgfqpoint{2.405482in}{3.174293in}}%
\pgfpathcurveto{\pgfqpoint{2.405482in}{3.187316in}}{\pgfqpoint{2.400308in}{3.199807in}}{\pgfqpoint{2.391099in}{3.209015in}}%
\pgfpathcurveto{\pgfqpoint{2.381891in}{3.218224in}}{\pgfqpoint{2.369400in}{3.223398in}}{\pgfqpoint{2.356377in}{3.223398in}}%
\pgfpathcurveto{\pgfqpoint{2.343354in}{3.223398in}}{\pgfqpoint{2.330863in}{3.218224in}}{\pgfqpoint{2.321655in}{3.209015in}}%
\pgfpathcurveto{\pgfqpoint{2.312446in}{3.199807in}}{\pgfqpoint{2.307272in}{3.187316in}}{\pgfqpoint{2.307272in}{3.174293in}}%
\pgfpathcurveto{\pgfqpoint{2.307272in}{3.161270in}}{\pgfqpoint{2.312446in}{3.148779in}}{\pgfqpoint{2.321655in}{3.139571in}}%
\pgfpathcurveto{\pgfqpoint{2.330863in}{3.130362in}}{\pgfqpoint{2.343354in}{3.125188in}}{\pgfqpoint{2.356377in}{3.125188in}}%
\pgfpathlineto{\pgfqpoint{2.356377in}{3.125188in}}%
\pgfpathclose%
\pgfusepath{stroke,fill}%
\end{pgfscope}%
\begin{pgfscope}%
\pgfpathrectangle{\pgfqpoint{0.786164in}{0.768110in}}{\pgfqpoint{8.851069in}{7.081890in}}%
\pgfusepath{clip}%
\pgfsetbuttcap%
\pgfsetroundjoin%
\definecolor{currentfill}{rgb}{0.267968,0.223549,0.512008}%
\pgfsetfillcolor{currentfill}%
\pgfsetfillopacity{0.700000}%
\pgfsetlinewidth{0.501875pt}%
\definecolor{currentstroke}{rgb}{1.000000,1.000000,1.000000}%
\pgfsetstrokecolor{currentstroke}%
\pgfsetstrokeopacity{0.700000}%
\pgfsetdash{}{0pt}%
\pgfpathmoveto{\pgfqpoint{2.487879in}{2.933109in}}%
\pgfpathcurveto{\pgfqpoint{2.500902in}{2.933109in}}{\pgfqpoint{2.513393in}{2.938283in}}{\pgfqpoint{2.522601in}{2.947491in}}%
\pgfpathcurveto{\pgfqpoint{2.531810in}{2.956700in}}{\pgfqpoint{2.536984in}{2.969191in}}{\pgfqpoint{2.536984in}{2.982214in}}%
\pgfpathcurveto{\pgfqpoint{2.536984in}{2.995236in}}{\pgfqpoint{2.531810in}{3.007727in}}{\pgfqpoint{2.522601in}{3.016936in}}%
\pgfpathcurveto{\pgfqpoint{2.513393in}{3.026144in}}{\pgfqpoint{2.500902in}{3.031318in}}{\pgfqpoint{2.487879in}{3.031318in}}%
\pgfpathcurveto{\pgfqpoint{2.474857in}{3.031318in}}{\pgfqpoint{2.462365in}{3.026144in}}{\pgfqpoint{2.453157in}{3.016936in}}%
\pgfpathcurveto{\pgfqpoint{2.443949in}{3.007727in}}{\pgfqpoint{2.438775in}{2.995236in}}{\pgfqpoint{2.438775in}{2.982214in}}%
\pgfpathcurveto{\pgfqpoint{2.438775in}{2.969191in}}{\pgfqpoint{2.443949in}{2.956700in}}{\pgfqpoint{2.453157in}{2.947491in}}%
\pgfpathcurveto{\pgfqpoint{2.462365in}{2.938283in}}{\pgfqpoint{2.474857in}{2.933109in}}{\pgfqpoint{2.487879in}{2.933109in}}%
\pgfpathlineto{\pgfqpoint{2.487879in}{2.933109in}}%
\pgfpathclose%
\pgfusepath{stroke,fill}%
\end{pgfscope}%
\begin{pgfscope}%
\pgfpathrectangle{\pgfqpoint{0.786164in}{0.768110in}}{\pgfqpoint{8.851069in}{7.081890in}}%
\pgfusepath{clip}%
\pgfsetbuttcap%
\pgfsetroundjoin%
\definecolor{currentfill}{rgb}{0.276194,0.190074,0.493001}%
\pgfsetfillcolor{currentfill}%
\pgfsetfillopacity{0.700000}%
\pgfsetlinewidth{0.501875pt}%
\definecolor{currentstroke}{rgb}{1.000000,1.000000,1.000000}%
\pgfsetstrokecolor{currentstroke}%
\pgfsetstrokeopacity{0.700000}%
\pgfsetdash{}{0pt}%
\pgfpathmoveto{\pgfqpoint{2.758210in}{2.741029in}}%
\pgfpathcurveto{\pgfqpoint{2.771233in}{2.741029in}}{\pgfqpoint{2.783724in}{2.746203in}}{\pgfqpoint{2.792932in}{2.755412in}}%
\pgfpathcurveto{\pgfqpoint{2.802141in}{2.764620in}}{\pgfqpoint{2.807315in}{2.777111in}}{\pgfqpoint{2.807315in}{2.790134in}}%
\pgfpathcurveto{\pgfqpoint{2.807315in}{2.803157in}}{\pgfqpoint{2.802141in}{2.815648in}}{\pgfqpoint{2.792932in}{2.824856in}}%
\pgfpathcurveto{\pgfqpoint{2.783724in}{2.834065in}}{\pgfqpoint{2.771233in}{2.839239in}}{\pgfqpoint{2.758210in}{2.839239in}}%
\pgfpathcurveto{\pgfqpoint{2.745187in}{2.839239in}}{\pgfqpoint{2.732696in}{2.834065in}}{\pgfqpoint{2.723488in}{2.824856in}}%
\pgfpathcurveto{\pgfqpoint{2.714279in}{2.815648in}}{\pgfqpoint{2.709105in}{2.803157in}}{\pgfqpoint{2.709105in}{2.790134in}}%
\pgfpathcurveto{\pgfqpoint{2.709105in}{2.777111in}}{\pgfqpoint{2.714279in}{2.764620in}}{\pgfqpoint{2.723488in}{2.755412in}}%
\pgfpathcurveto{\pgfqpoint{2.732696in}{2.746203in}}{\pgfqpoint{2.745187in}{2.741029in}}{\pgfqpoint{2.758210in}{2.741029in}}%
\pgfpathlineto{\pgfqpoint{2.758210in}{2.741029in}}%
\pgfpathclose%
\pgfusepath{stroke,fill}%
\end{pgfscope}%
\begin{pgfscope}%
\pgfpathrectangle{\pgfqpoint{0.786164in}{0.768110in}}{\pgfqpoint{8.851069in}{7.081890in}}%
\pgfusepath{clip}%
\pgfsetbuttcap%
\pgfsetroundjoin%
\definecolor{currentfill}{rgb}{0.277134,0.185228,0.489898}%
\pgfsetfillcolor{currentfill}%
\pgfsetfillopacity{0.700000}%
\pgfsetlinewidth{0.501875pt}%
\definecolor{currentstroke}{rgb}{1.000000,1.000000,1.000000}%
\pgfsetstrokecolor{currentstroke}%
\pgfsetstrokeopacity{0.700000}%
\pgfsetdash{}{0pt}%
\pgfpathmoveto{\pgfqpoint{2.858821in}{2.655661in}}%
\pgfpathcurveto{\pgfqpoint{2.871843in}{2.655661in}}{\pgfqpoint{2.884335in}{2.660835in}}{\pgfqpoint{2.893543in}{2.670043in}}%
\pgfpathcurveto{\pgfqpoint{2.902751in}{2.679252in}}{\pgfqpoint{2.907925in}{2.691743in}}{\pgfqpoint{2.907925in}{2.704765in}}%
\pgfpathcurveto{\pgfqpoint{2.907925in}{2.717788in}}{\pgfqpoint{2.902751in}{2.730279in}}{\pgfqpoint{2.893543in}{2.739488in}}%
\pgfpathcurveto{\pgfqpoint{2.884335in}{2.748696in}}{\pgfqpoint{2.871843in}{2.753870in}}{\pgfqpoint{2.858821in}{2.753870in}}%
\pgfpathcurveto{\pgfqpoint{2.845798in}{2.753870in}}{\pgfqpoint{2.833307in}{2.748696in}}{\pgfqpoint{2.824099in}{2.739488in}}%
\pgfpathcurveto{\pgfqpoint{2.814890in}{2.730279in}}{\pgfqpoint{2.809716in}{2.717788in}}{\pgfqpoint{2.809716in}{2.704765in}}%
\pgfpathcurveto{\pgfqpoint{2.809716in}{2.691743in}}{\pgfqpoint{2.814890in}{2.679252in}}{\pgfqpoint{2.824099in}{2.670043in}}%
\pgfpathcurveto{\pgfqpoint{2.833307in}{2.660835in}}{\pgfqpoint{2.845798in}{2.655661in}}{\pgfqpoint{2.858821in}{2.655661in}}%
\pgfpathlineto{\pgfqpoint{2.858821in}{2.655661in}}%
\pgfpathclose%
\pgfusepath{stroke,fill}%
\end{pgfscope}%
\begin{pgfscope}%
\pgfpathrectangle{\pgfqpoint{0.786164in}{0.768110in}}{\pgfqpoint{8.851069in}{7.081890in}}%
\pgfusepath{clip}%
\pgfsetbuttcap%
\pgfsetroundjoin%
\definecolor{currentfill}{rgb}{0.277134,0.185228,0.489898}%
\pgfsetfillcolor{currentfill}%
\pgfsetfillopacity{0.700000}%
\pgfsetlinewidth{0.501875pt}%
\definecolor{currentstroke}{rgb}{1.000000,1.000000,1.000000}%
\pgfsetstrokecolor{currentstroke}%
\pgfsetstrokeopacity{0.700000}%
\pgfsetdash{}{0pt}%
\pgfpathmoveto{\pgfqpoint{2.784828in}{2.655661in}}%
\pgfpathcurveto{\pgfqpoint{2.797850in}{2.655661in}}{\pgfqpoint{2.810342in}{2.660835in}}{\pgfqpoint{2.819550in}{2.670043in}}%
\pgfpathcurveto{\pgfqpoint{2.828758in}{2.679252in}}{\pgfqpoint{2.833932in}{2.691743in}}{\pgfqpoint{2.833932in}{2.704765in}}%
\pgfpathcurveto{\pgfqpoint{2.833932in}{2.717788in}}{\pgfqpoint{2.828758in}{2.730279in}}{\pgfqpoint{2.819550in}{2.739488in}}%
\pgfpathcurveto{\pgfqpoint{2.810342in}{2.748696in}}{\pgfqpoint{2.797850in}{2.753870in}}{\pgfqpoint{2.784828in}{2.753870in}}%
\pgfpathcurveto{\pgfqpoint{2.771805in}{2.753870in}}{\pgfqpoint{2.759314in}{2.748696in}}{\pgfqpoint{2.750106in}{2.739488in}}%
\pgfpathcurveto{\pgfqpoint{2.740897in}{2.730279in}}{\pgfqpoint{2.735723in}{2.717788in}}{\pgfqpoint{2.735723in}{2.704765in}}%
\pgfpathcurveto{\pgfqpoint{2.735723in}{2.691743in}}{\pgfqpoint{2.740897in}{2.679252in}}{\pgfqpoint{2.750106in}{2.670043in}}%
\pgfpathcurveto{\pgfqpoint{2.759314in}{2.660835in}}{\pgfqpoint{2.771805in}{2.655661in}}{\pgfqpoint{2.784828in}{2.655661in}}%
\pgfpathlineto{\pgfqpoint{2.784828in}{2.655661in}}%
\pgfpathclose%
\pgfusepath{stroke,fill}%
\end{pgfscope}%
\begin{pgfscope}%
\pgfpathrectangle{\pgfqpoint{0.786164in}{0.768110in}}{\pgfqpoint{8.851069in}{7.081890in}}%
\pgfusepath{clip}%
\pgfsetbuttcap%
\pgfsetroundjoin%
\definecolor{currentfill}{rgb}{0.277134,0.185228,0.489898}%
\pgfsetfillcolor{currentfill}%
\pgfsetfillopacity{0.700000}%
\pgfsetlinewidth{0.501875pt}%
\definecolor{currentstroke}{rgb}{1.000000,1.000000,1.000000}%
\pgfsetstrokecolor{currentstroke}%
\pgfsetstrokeopacity{0.700000}%
\pgfsetdash{}{0pt}%
\pgfpathmoveto{\pgfqpoint{2.914621in}{2.634319in}}%
\pgfpathcurveto{\pgfqpoint{2.927643in}{2.634319in}}{\pgfqpoint{2.940134in}{2.639493in}}{\pgfqpoint{2.949343in}{2.648701in}}%
\pgfpathcurveto{\pgfqpoint{2.958551in}{2.657909in}}{\pgfqpoint{2.963725in}{2.670401in}}{\pgfqpoint{2.963725in}{2.683423in}}%
\pgfpathcurveto{\pgfqpoint{2.963725in}{2.696446in}}{\pgfqpoint{2.958551in}{2.708937in}}{\pgfqpoint{2.949343in}{2.718145in}}%
\pgfpathcurveto{\pgfqpoint{2.940134in}{2.727354in}}{\pgfqpoint{2.927643in}{2.732528in}}{\pgfqpoint{2.914621in}{2.732528in}}%
\pgfpathcurveto{\pgfqpoint{2.901598in}{2.732528in}}{\pgfqpoint{2.889107in}{2.727354in}}{\pgfqpoint{2.879898in}{2.718145in}}%
\pgfpathcurveto{\pgfqpoint{2.870690in}{2.708937in}}{\pgfqpoint{2.865516in}{2.696446in}}{\pgfqpoint{2.865516in}{2.683423in}}%
\pgfpathcurveto{\pgfqpoint{2.865516in}{2.670401in}}{\pgfqpoint{2.870690in}{2.657909in}}{\pgfqpoint{2.879898in}{2.648701in}}%
\pgfpathcurveto{\pgfqpoint{2.889107in}{2.639493in}}{\pgfqpoint{2.901598in}{2.634319in}}{\pgfqpoint{2.914621in}{2.634319in}}%
\pgfpathlineto{\pgfqpoint{2.914621in}{2.634319in}}%
\pgfpathclose%
\pgfusepath{stroke,fill}%
\end{pgfscope}%
\begin{pgfscope}%
\pgfpathrectangle{\pgfqpoint{0.786164in}{0.768110in}}{\pgfqpoint{8.851069in}{7.081890in}}%
\pgfusepath{clip}%
\pgfsetbuttcap%
\pgfsetroundjoin%
\definecolor{currentfill}{rgb}{0.280255,0.165693,0.476498}%
\pgfsetfillcolor{currentfill}%
\pgfsetfillopacity{0.700000}%
\pgfsetlinewidth{0.501875pt}%
\definecolor{currentstroke}{rgb}{1.000000,1.000000,1.000000}%
\pgfsetstrokecolor{currentstroke}%
\pgfsetstrokeopacity{0.700000}%
\pgfsetdash{}{0pt}%
\pgfpathmoveto{\pgfqpoint{3.017429in}{2.527608in}}%
\pgfpathcurveto{\pgfqpoint{3.030452in}{2.527608in}}{\pgfqpoint{3.042943in}{2.532782in}}{\pgfqpoint{3.052152in}{2.541990in}}%
\pgfpathcurveto{\pgfqpoint{3.061360in}{2.551199in}}{\pgfqpoint{3.066534in}{2.563690in}}{\pgfqpoint{3.066534in}{2.576712in}}%
\pgfpathcurveto{\pgfqpoint{3.066534in}{2.589735in}}{\pgfqpoint{3.061360in}{2.602226in}}{\pgfqpoint{3.052152in}{2.611435in}}%
\pgfpathcurveto{\pgfqpoint{3.042943in}{2.620643in}}{\pgfqpoint{3.030452in}{2.625817in}}{\pgfqpoint{3.017429in}{2.625817in}}%
\pgfpathcurveto{\pgfqpoint{3.004407in}{2.625817in}}{\pgfqpoint{2.991916in}{2.620643in}}{\pgfqpoint{2.982707in}{2.611435in}}%
\pgfpathcurveto{\pgfqpoint{2.973499in}{2.602226in}}{\pgfqpoint{2.968325in}{2.589735in}}{\pgfqpoint{2.968325in}{2.576712in}}%
\pgfpathcurveto{\pgfqpoint{2.968325in}{2.563690in}}{\pgfqpoint{2.973499in}{2.551199in}}{\pgfqpoint{2.982707in}{2.541990in}}%
\pgfpathcurveto{\pgfqpoint{2.991916in}{2.532782in}}{\pgfqpoint{3.004407in}{2.527608in}}{\pgfqpoint{3.017429in}{2.527608in}}%
\pgfpathlineto{\pgfqpoint{3.017429in}{2.527608in}}%
\pgfpathclose%
\pgfusepath{stroke,fill}%
\end{pgfscope}%
\begin{pgfscope}%
\pgfpathrectangle{\pgfqpoint{0.786164in}{0.768110in}}{\pgfqpoint{8.851069in}{7.081890in}}%
\pgfusepath{clip}%
\pgfsetbuttcap%
\pgfsetroundjoin%
\definecolor{currentfill}{rgb}{0.280868,0.160771,0.472899}%
\pgfsetfillcolor{currentfill}%
\pgfsetfillopacity{0.700000}%
\pgfsetlinewidth{0.501875pt}%
\definecolor{currentstroke}{rgb}{1.000000,1.000000,1.000000}%
\pgfsetstrokecolor{currentstroke}%
\pgfsetstrokeopacity{0.700000}%
\pgfsetdash{}{0pt}%
\pgfpathmoveto{\pgfqpoint{3.114866in}{2.527608in}}%
\pgfpathcurveto{\pgfqpoint{3.127888in}{2.527608in}}{\pgfqpoint{3.140379in}{2.532782in}}{\pgfqpoint{3.149588in}{2.541990in}}%
\pgfpathcurveto{\pgfqpoint{3.158796in}{2.551199in}}{\pgfqpoint{3.163970in}{2.563690in}}{\pgfqpoint{3.163970in}{2.576712in}}%
\pgfpathcurveto{\pgfqpoint{3.163970in}{2.589735in}}{\pgfqpoint{3.158796in}{2.602226in}}{\pgfqpoint{3.149588in}{2.611435in}}%
\pgfpathcurveto{\pgfqpoint{3.140379in}{2.620643in}}{\pgfqpoint{3.127888in}{2.625817in}}{\pgfqpoint{3.114866in}{2.625817in}}%
\pgfpathcurveto{\pgfqpoint{3.101843in}{2.625817in}}{\pgfqpoint{3.089352in}{2.620643in}}{\pgfqpoint{3.080143in}{2.611435in}}%
\pgfpathcurveto{\pgfqpoint{3.070935in}{2.602226in}}{\pgfqpoint{3.065761in}{2.589735in}}{\pgfqpoint{3.065761in}{2.576712in}}%
\pgfpathcurveto{\pgfqpoint{3.065761in}{2.563690in}}{\pgfqpoint{3.070935in}{2.551199in}}{\pgfqpoint{3.080143in}{2.541990in}}%
\pgfpathcurveto{\pgfqpoint{3.089352in}{2.532782in}}{\pgfqpoint{3.101843in}{2.527608in}}{\pgfqpoint{3.114866in}{2.527608in}}%
\pgfpathlineto{\pgfqpoint{3.114866in}{2.527608in}}%
\pgfpathclose%
\pgfusepath{stroke,fill}%
\end{pgfscope}%
\begin{pgfscope}%
\pgfpathrectangle{\pgfqpoint{0.786164in}{0.768110in}}{\pgfqpoint{8.851069in}{7.081890in}}%
\pgfusepath{clip}%
\pgfsetbuttcap%
\pgfsetroundjoin%
\definecolor{currentfill}{rgb}{0.279574,0.170599,0.479997}%
\pgfsetfillcolor{currentfill}%
\pgfsetfillopacity{0.700000}%
\pgfsetlinewidth{0.501875pt}%
\definecolor{currentstroke}{rgb}{1.000000,1.000000,1.000000}%
\pgfsetstrokecolor{currentstroke}%
\pgfsetstrokeopacity{0.700000}%
\pgfsetdash{}{0pt}%
\pgfpathmoveto{\pgfqpoint{3.156624in}{2.570292in}}%
\pgfpathcurveto{\pgfqpoint{3.169647in}{2.570292in}}{\pgfqpoint{3.182138in}{2.575466in}}{\pgfqpoint{3.191346in}{2.584674in}}%
\pgfpathcurveto{\pgfqpoint{3.200555in}{2.593883in}}{\pgfqpoint{3.205729in}{2.606374in}}{\pgfqpoint{3.205729in}{2.619397in}}%
\pgfpathcurveto{\pgfqpoint{3.205729in}{2.632419in}}{\pgfqpoint{3.200555in}{2.644910in}}{\pgfqpoint{3.191346in}{2.654119in}}%
\pgfpathcurveto{\pgfqpoint{3.182138in}{2.663327in}}{\pgfqpoint{3.169647in}{2.668501in}}{\pgfqpoint{3.156624in}{2.668501in}}%
\pgfpathcurveto{\pgfqpoint{3.143601in}{2.668501in}}{\pgfqpoint{3.131110in}{2.663327in}}{\pgfqpoint{3.121902in}{2.654119in}}%
\pgfpathcurveto{\pgfqpoint{3.112693in}{2.644910in}}{\pgfqpoint{3.107519in}{2.632419in}}{\pgfqpoint{3.107519in}{2.619397in}}%
\pgfpathcurveto{\pgfqpoint{3.107519in}{2.606374in}}{\pgfqpoint{3.112693in}{2.593883in}}{\pgfqpoint{3.121902in}{2.584674in}}%
\pgfpathcurveto{\pgfqpoint{3.131110in}{2.575466in}}{\pgfqpoint{3.143601in}{2.570292in}}{\pgfqpoint{3.156624in}{2.570292in}}%
\pgfpathlineto{\pgfqpoint{3.156624in}{2.570292in}}%
\pgfpathclose%
\pgfusepath{stroke,fill}%
\end{pgfscope}%
\begin{pgfscope}%
\pgfpathrectangle{\pgfqpoint{0.786164in}{0.768110in}}{\pgfqpoint{8.851069in}{7.081890in}}%
\pgfusepath{clip}%
\pgfsetbuttcap%
\pgfsetroundjoin%
\definecolor{currentfill}{rgb}{0.278826,0.175490,0.483397}%
\pgfsetfillcolor{currentfill}%
\pgfsetfillopacity{0.700000}%
\pgfsetlinewidth{0.501875pt}%
\definecolor{currentstroke}{rgb}{1.000000,1.000000,1.000000}%
\pgfsetstrokecolor{currentstroke}%
\pgfsetstrokeopacity{0.700000}%
\pgfsetdash{}{0pt}%
\pgfpathmoveto{\pgfqpoint{3.253572in}{2.506266in}}%
\pgfpathcurveto{\pgfqpoint{3.266595in}{2.506266in}}{\pgfqpoint{3.279086in}{2.511440in}}{\pgfqpoint{3.288294in}{2.520648in}}%
\pgfpathcurveto{\pgfqpoint{3.297502in}{2.529856in}}{\pgfqpoint{3.302676in}{2.542347in}}{\pgfqpoint{3.302676in}{2.555370in}}%
\pgfpathcurveto{\pgfqpoint{3.302676in}{2.568393in}}{\pgfqpoint{3.297502in}{2.580884in}}{\pgfqpoint{3.288294in}{2.590092in}}%
\pgfpathcurveto{\pgfqpoint{3.279086in}{2.599301in}}{\pgfqpoint{3.266595in}{2.604475in}}{\pgfqpoint{3.253572in}{2.604475in}}%
\pgfpathcurveto{\pgfqpoint{3.240549in}{2.604475in}}{\pgfqpoint{3.228058in}{2.599301in}}{\pgfqpoint{3.218850in}{2.590092in}}%
\pgfpathcurveto{\pgfqpoint{3.209641in}{2.580884in}}{\pgfqpoint{3.204467in}{2.568393in}}{\pgfqpoint{3.204467in}{2.555370in}}%
\pgfpathcurveto{\pgfqpoint{3.204467in}{2.542347in}}{\pgfqpoint{3.209641in}{2.529856in}}{\pgfqpoint{3.218850in}{2.520648in}}%
\pgfpathcurveto{\pgfqpoint{3.228058in}{2.511440in}}{\pgfqpoint{3.240549in}{2.506266in}}{\pgfqpoint{3.253572in}{2.506266in}}%
\pgfpathlineto{\pgfqpoint{3.253572in}{2.506266in}}%
\pgfpathclose%
\pgfusepath{stroke,fill}%
\end{pgfscope}%
\begin{pgfscope}%
\pgfpathrectangle{\pgfqpoint{0.786164in}{0.768110in}}{\pgfqpoint{8.851069in}{7.081890in}}%
\pgfusepath{clip}%
\pgfsetbuttcap%
\pgfsetroundjoin%
\definecolor{currentfill}{rgb}{0.275191,0.194905,0.496005}%
\pgfsetfillcolor{currentfill}%
\pgfsetfillopacity{0.700000}%
\pgfsetlinewidth{0.501875pt}%
\definecolor{currentstroke}{rgb}{1.000000,1.000000,1.000000}%
\pgfsetstrokecolor{currentstroke}%
\pgfsetstrokeopacity{0.700000}%
\pgfsetdash{}{0pt}%
\pgfpathmoveto{\pgfqpoint{3.266148in}{2.570292in}}%
\pgfpathcurveto{\pgfqpoint{3.279171in}{2.570292in}}{\pgfqpoint{3.291662in}{2.575466in}}{\pgfqpoint{3.300870in}{2.584674in}}%
\pgfpathcurveto{\pgfqpoint{3.310079in}{2.593883in}}{\pgfqpoint{3.315253in}{2.606374in}}{\pgfqpoint{3.315253in}{2.619397in}}%
\pgfpathcurveto{\pgfqpoint{3.315253in}{2.632419in}}{\pgfqpoint{3.310079in}{2.644910in}}{\pgfqpoint{3.300870in}{2.654119in}}%
\pgfpathcurveto{\pgfqpoint{3.291662in}{2.663327in}}{\pgfqpoint{3.279171in}{2.668501in}}{\pgfqpoint{3.266148in}{2.668501in}}%
\pgfpathcurveto{\pgfqpoint{3.253125in}{2.668501in}}{\pgfqpoint{3.240634in}{2.663327in}}{\pgfqpoint{3.231426in}{2.654119in}}%
\pgfpathcurveto{\pgfqpoint{3.222218in}{2.644910in}}{\pgfqpoint{3.217044in}{2.632419in}}{\pgfqpoint{3.217044in}{2.619397in}}%
\pgfpathcurveto{\pgfqpoint{3.217044in}{2.606374in}}{\pgfqpoint{3.222218in}{2.593883in}}{\pgfqpoint{3.231426in}{2.584674in}}%
\pgfpathcurveto{\pgfqpoint{3.240634in}{2.575466in}}{\pgfqpoint{3.253125in}{2.570292in}}{\pgfqpoint{3.266148in}{2.570292in}}%
\pgfpathlineto{\pgfqpoint{3.266148in}{2.570292in}}%
\pgfpathclose%
\pgfusepath{stroke,fill}%
\end{pgfscope}%
\begin{pgfscope}%
\pgfpathrectangle{\pgfqpoint{0.786164in}{0.768110in}}{\pgfqpoint{8.851069in}{7.081890in}}%
\pgfusepath{clip}%
\pgfsetbuttcap%
\pgfsetroundjoin%
\definecolor{currentfill}{rgb}{0.276194,0.190074,0.493001}%
\pgfsetfillcolor{currentfill}%
\pgfsetfillopacity{0.700000}%
\pgfsetlinewidth{0.501875pt}%
\definecolor{currentstroke}{rgb}{1.000000,1.000000,1.000000}%
\pgfsetstrokecolor{currentstroke}%
\pgfsetstrokeopacity{0.700000}%
\pgfsetdash{}{0pt}%
\pgfpathmoveto{\pgfqpoint{3.254549in}{2.548950in}}%
\pgfpathcurveto{\pgfqpoint{3.267571in}{2.548950in}}{\pgfqpoint{3.280062in}{2.554124in}}{\pgfqpoint{3.289271in}{2.563332in}}%
\pgfpathcurveto{\pgfqpoint{3.298479in}{2.572541in}}{\pgfqpoint{3.303653in}{2.585032in}}{\pgfqpoint{3.303653in}{2.598055in}}%
\pgfpathcurveto{\pgfqpoint{3.303653in}{2.611077in}}{\pgfqpoint{3.298479in}{2.623568in}}{\pgfqpoint{3.289271in}{2.632777in}}%
\pgfpathcurveto{\pgfqpoint{3.280062in}{2.641985in}}{\pgfqpoint{3.267571in}{2.647159in}}{\pgfqpoint{3.254549in}{2.647159in}}%
\pgfpathcurveto{\pgfqpoint{3.241526in}{2.647159in}}{\pgfqpoint{3.229035in}{2.641985in}}{\pgfqpoint{3.219826in}{2.632777in}}%
\pgfpathcurveto{\pgfqpoint{3.210618in}{2.623568in}}{\pgfqpoint{3.205444in}{2.611077in}}{\pgfqpoint{3.205444in}{2.598055in}}%
\pgfpathcurveto{\pgfqpoint{3.205444in}{2.585032in}}{\pgfqpoint{3.210618in}{2.572541in}}{\pgfqpoint{3.219826in}{2.563332in}}%
\pgfpathcurveto{\pgfqpoint{3.229035in}{2.554124in}}{\pgfqpoint{3.241526in}{2.548950in}}{\pgfqpoint{3.254549in}{2.548950in}}%
\pgfpathlineto{\pgfqpoint{3.254549in}{2.548950in}}%
\pgfpathclose%
\pgfusepath{stroke,fill}%
\end{pgfscope}%
\begin{pgfscope}%
\pgfpathrectangle{\pgfqpoint{0.786164in}{0.768110in}}{\pgfqpoint{8.851069in}{7.081890in}}%
\pgfusepath{clip}%
\pgfsetbuttcap%
\pgfsetroundjoin%
\definecolor{currentfill}{rgb}{0.278826,0.175490,0.483397}%
\pgfsetfillcolor{currentfill}%
\pgfsetfillopacity{0.700000}%
\pgfsetlinewidth{0.501875pt}%
\definecolor{currentstroke}{rgb}{1.000000,1.000000,1.000000}%
\pgfsetstrokecolor{currentstroke}%
\pgfsetstrokeopacity{0.700000}%
\pgfsetdash{}{0pt}%
\pgfpathmoveto{\pgfqpoint{3.355770in}{2.463581in}}%
\pgfpathcurveto{\pgfqpoint{3.368793in}{2.463581in}}{\pgfqpoint{3.381284in}{2.468755in}}{\pgfqpoint{3.390492in}{2.477964in}}%
\pgfpathcurveto{\pgfqpoint{3.399701in}{2.487172in}}{\pgfqpoint{3.404875in}{2.499663in}}{\pgfqpoint{3.404875in}{2.512686in}}%
\pgfpathcurveto{\pgfqpoint{3.404875in}{2.525709in}}{\pgfqpoint{3.399701in}{2.538200in}}{\pgfqpoint{3.390492in}{2.547408in}}%
\pgfpathcurveto{\pgfqpoint{3.381284in}{2.556617in}}{\pgfqpoint{3.368793in}{2.561790in}}{\pgfqpoint{3.355770in}{2.561790in}}%
\pgfpathcurveto{\pgfqpoint{3.342747in}{2.561790in}}{\pgfqpoint{3.330256in}{2.556617in}}{\pgfqpoint{3.321048in}{2.547408in}}%
\pgfpathcurveto{\pgfqpoint{3.311839in}{2.538200in}}{\pgfqpoint{3.306665in}{2.525709in}}{\pgfqpoint{3.306665in}{2.512686in}}%
\pgfpathcurveto{\pgfqpoint{3.306665in}{2.499663in}}{\pgfqpoint{3.311839in}{2.487172in}}{\pgfqpoint{3.321048in}{2.477964in}}%
\pgfpathcurveto{\pgfqpoint{3.330256in}{2.468755in}}{\pgfqpoint{3.342747in}{2.463581in}}{\pgfqpoint{3.355770in}{2.463581in}}%
\pgfpathlineto{\pgfqpoint{3.355770in}{2.463581in}}%
\pgfpathclose%
\pgfusepath{stroke,fill}%
\end{pgfscope}%
\begin{pgfscope}%
\pgfpathrectangle{\pgfqpoint{0.786164in}{0.768110in}}{\pgfqpoint{8.851069in}{7.081890in}}%
\pgfusepath{clip}%
\pgfsetbuttcap%
\pgfsetroundjoin%
\definecolor{currentfill}{rgb}{0.283187,0.125848,0.444960}%
\pgfsetfillcolor{currentfill}%
\pgfsetfillopacity{0.700000}%
\pgfsetlinewidth{0.501875pt}%
\definecolor{currentstroke}{rgb}{1.000000,1.000000,1.000000}%
\pgfsetstrokecolor{currentstroke}%
\pgfsetstrokeopacity{0.700000}%
\pgfsetdash{}{0pt}%
\pgfpathmoveto{\pgfqpoint{3.767005in}{2.228817in}}%
\pgfpathcurveto{\pgfqpoint{3.780027in}{2.228817in}}{\pgfqpoint{3.792518in}{2.233991in}}{\pgfqpoint{3.801727in}{2.243200in}}%
\pgfpathcurveto{\pgfqpoint{3.810935in}{2.252408in}}{\pgfqpoint{3.816109in}{2.264899in}}{\pgfqpoint{3.816109in}{2.277922in}}%
\pgfpathcurveto{\pgfqpoint{3.816109in}{2.290945in}}{\pgfqpoint{3.810935in}{2.303436in}}{\pgfqpoint{3.801727in}{2.312644in}}%
\pgfpathcurveto{\pgfqpoint{3.792518in}{2.321853in}}{\pgfqpoint{3.780027in}{2.327027in}}{\pgfqpoint{3.767005in}{2.327027in}}%
\pgfpathcurveto{\pgfqpoint{3.753982in}{2.327027in}}{\pgfqpoint{3.741491in}{2.321853in}}{\pgfqpoint{3.732282in}{2.312644in}}%
\pgfpathcurveto{\pgfqpoint{3.723074in}{2.303436in}}{\pgfqpoint{3.717900in}{2.290945in}}{\pgfqpoint{3.717900in}{2.277922in}}%
\pgfpathcurveto{\pgfqpoint{3.717900in}{2.264899in}}{\pgfqpoint{3.723074in}{2.252408in}}{\pgfqpoint{3.732282in}{2.243200in}}%
\pgfpathcurveto{\pgfqpoint{3.741491in}{2.233991in}}{\pgfqpoint{3.753982in}{2.228817in}}{\pgfqpoint{3.767005in}{2.228817in}}%
\pgfpathlineto{\pgfqpoint{3.767005in}{2.228817in}}%
\pgfpathclose%
\pgfusepath{stroke,fill}%
\end{pgfscope}%
\begin{pgfscope}%
\pgfpathrectangle{\pgfqpoint{0.786164in}{0.768110in}}{\pgfqpoint{8.851069in}{7.081890in}}%
\pgfusepath{clip}%
\pgfsetbuttcap%
\pgfsetroundjoin%
\definecolor{currentfill}{rgb}{0.282290,0.145912,0.461510}%
\pgfsetfillcolor{currentfill}%
\pgfsetfillopacity{0.700000}%
\pgfsetlinewidth{0.501875pt}%
\definecolor{currentstroke}{rgb}{1.000000,1.000000,1.000000}%
\pgfsetstrokecolor{currentstroke}%
\pgfsetstrokeopacity{0.700000}%
\pgfsetdash{}{0pt}%
\pgfpathmoveto{\pgfqpoint{3.707908in}{2.314186in}}%
\pgfpathcurveto{\pgfqpoint{3.720931in}{2.314186in}}{\pgfqpoint{3.733422in}{2.319360in}}{\pgfqpoint{3.742630in}{2.328568in}}%
\pgfpathcurveto{\pgfqpoint{3.751839in}{2.337777in}}{\pgfqpoint{3.757013in}{2.350268in}}{\pgfqpoint{3.757013in}{2.363291in}}%
\pgfpathcurveto{\pgfqpoint{3.757013in}{2.376313in}}{\pgfqpoint{3.751839in}{2.388804in}}{\pgfqpoint{3.742630in}{2.398013in}}%
\pgfpathcurveto{\pgfqpoint{3.733422in}{2.407221in}}{\pgfqpoint{3.720931in}{2.412395in}}{\pgfqpoint{3.707908in}{2.412395in}}%
\pgfpathcurveto{\pgfqpoint{3.694885in}{2.412395in}}{\pgfqpoint{3.682394in}{2.407221in}}{\pgfqpoint{3.673186in}{2.398013in}}%
\pgfpathcurveto{\pgfqpoint{3.663977in}{2.388804in}}{\pgfqpoint{3.658803in}{2.376313in}}{\pgfqpoint{3.658803in}{2.363291in}}%
\pgfpathcurveto{\pgfqpoint{3.658803in}{2.350268in}}{\pgfqpoint{3.663977in}{2.337777in}}{\pgfqpoint{3.673186in}{2.328568in}}%
\pgfpathcurveto{\pgfqpoint{3.682394in}{2.319360in}}{\pgfqpoint{3.694885in}{2.314186in}}{\pgfqpoint{3.707908in}{2.314186in}}%
\pgfpathlineto{\pgfqpoint{3.707908in}{2.314186in}}%
\pgfpathclose%
\pgfusepath{stroke,fill}%
\end{pgfscope}%
\begin{pgfscope}%
\pgfpathrectangle{\pgfqpoint{0.786164in}{0.768110in}}{\pgfqpoint{8.851069in}{7.081890in}}%
\pgfusepath{clip}%
\pgfsetbuttcap%
\pgfsetroundjoin%
\definecolor{currentfill}{rgb}{0.281887,0.150881,0.465405}%
\pgfsetfillcolor{currentfill}%
\pgfsetfillopacity{0.700000}%
\pgfsetlinewidth{0.501875pt}%
\definecolor{currentstroke}{rgb}{1.000000,1.000000,1.000000}%
\pgfsetstrokecolor{currentstroke}%
\pgfsetstrokeopacity{0.700000}%
\pgfsetdash{}{0pt}%
\pgfpathmoveto{\pgfqpoint{3.876407in}{2.356870in}}%
\pgfpathcurveto{\pgfqpoint{3.889429in}{2.356870in}}{\pgfqpoint{3.901920in}{2.362044in}}{\pgfqpoint{3.911129in}{2.371253in}}%
\pgfpathcurveto{\pgfqpoint{3.920337in}{2.380461in}}{\pgfqpoint{3.925511in}{2.392952in}}{\pgfqpoint{3.925511in}{2.405975in}}%
\pgfpathcurveto{\pgfqpoint{3.925511in}{2.418998in}}{\pgfqpoint{3.920337in}{2.431489in}}{\pgfqpoint{3.911129in}{2.440697in}}%
\pgfpathcurveto{\pgfqpoint{3.901920in}{2.449906in}}{\pgfqpoint{3.889429in}{2.455080in}}{\pgfqpoint{3.876407in}{2.455080in}}%
\pgfpathcurveto{\pgfqpoint{3.863384in}{2.455080in}}{\pgfqpoint{3.850893in}{2.449906in}}{\pgfqpoint{3.841684in}{2.440697in}}%
\pgfpathcurveto{\pgfqpoint{3.832476in}{2.431489in}}{\pgfqpoint{3.827302in}{2.418998in}}{\pgfqpoint{3.827302in}{2.405975in}}%
\pgfpathcurveto{\pgfqpoint{3.827302in}{2.392952in}}{\pgfqpoint{3.832476in}{2.380461in}}{\pgfqpoint{3.841684in}{2.371253in}}%
\pgfpathcurveto{\pgfqpoint{3.850893in}{2.362044in}}{\pgfqpoint{3.863384in}{2.356870in}}{\pgfqpoint{3.876407in}{2.356870in}}%
\pgfpathlineto{\pgfqpoint{3.876407in}{2.356870in}}%
\pgfpathclose%
\pgfusepath{stroke,fill}%
\end{pgfscope}%
\begin{pgfscope}%
\pgfpathrectangle{\pgfqpoint{0.786164in}{0.768110in}}{\pgfqpoint{8.851069in}{7.081890in}}%
\pgfusepath{clip}%
\pgfsetbuttcap%
\pgfsetroundjoin%
\definecolor{currentfill}{rgb}{0.162016,0.687316,0.499129}%
\pgfsetfillcolor{currentfill}%
\pgfsetfillopacity{0.700000}%
\pgfsetlinewidth{0.501875pt}%
\definecolor{currentstroke}{rgb}{1.000000,1.000000,1.000000}%
\pgfsetstrokecolor{currentstroke}%
\pgfsetstrokeopacity{0.700000}%
\pgfsetdash{}{0pt}%
\pgfpathmoveto{\pgfqpoint{4.745274in}{3.552032in}}%
\pgfpathcurveto{\pgfqpoint{4.758297in}{3.552032in}}{\pgfqpoint{4.770788in}{3.557206in}}{\pgfqpoint{4.779996in}{3.566414in}}%
\pgfpathcurveto{\pgfqpoint{4.789205in}{3.575623in}}{\pgfqpoint{4.794379in}{3.588114in}}{\pgfqpoint{4.794379in}{3.601137in}}%
\pgfpathcurveto{\pgfqpoint{4.794379in}{3.614159in}}{\pgfqpoint{4.789205in}{3.626650in}}{\pgfqpoint{4.779996in}{3.635859in}}%
\pgfpathcurveto{\pgfqpoint{4.770788in}{3.645067in}}{\pgfqpoint{4.758297in}{3.650241in}}{\pgfqpoint{4.745274in}{3.650241in}}%
\pgfpathcurveto{\pgfqpoint{4.732252in}{3.650241in}}{\pgfqpoint{4.719760in}{3.645067in}}{\pgfqpoint{4.710552in}{3.635859in}}%
\pgfpathcurveto{\pgfqpoint{4.701344in}{3.626650in}}{\pgfqpoint{4.696170in}{3.614159in}}{\pgfqpoint{4.696170in}{3.601137in}}%
\pgfpathcurveto{\pgfqpoint{4.696170in}{3.588114in}}{\pgfqpoint{4.701344in}{3.575623in}}{\pgfqpoint{4.710552in}{3.566414in}}%
\pgfpathcurveto{\pgfqpoint{4.719760in}{3.557206in}}{\pgfqpoint{4.732252in}{3.552032in}}{\pgfqpoint{4.745274in}{3.552032in}}%
\pgfpathlineto{\pgfqpoint{4.745274in}{3.552032in}}%
\pgfpathclose%
\pgfusepath{stroke,fill}%
\end{pgfscope}%
\begin{pgfscope}%
\pgfpathrectangle{\pgfqpoint{0.786164in}{0.768110in}}{\pgfqpoint{8.851069in}{7.081890in}}%
\pgfusepath{clip}%
\pgfsetbuttcap%
\pgfsetroundjoin%
\definecolor{currentfill}{rgb}{0.120638,0.625828,0.533488}%
\pgfsetfillcolor{currentfill}%
\pgfsetfillopacity{0.700000}%
\pgfsetlinewidth{0.501875pt}%
\definecolor{currentstroke}{rgb}{1.000000,1.000000,1.000000}%
\pgfsetstrokecolor{currentstroke}%
\pgfsetstrokeopacity{0.700000}%
\pgfsetdash{}{0pt}%
\pgfpathmoveto{\pgfqpoint{4.694236in}{3.061162in}}%
\pgfpathcurveto{\pgfqpoint{4.707259in}{3.061162in}}{\pgfqpoint{4.719750in}{3.066336in}}{\pgfqpoint{4.728958in}{3.075544in}}%
\pgfpathcurveto{\pgfqpoint{4.738167in}{3.084753in}}{\pgfqpoint{4.743341in}{3.097244in}}{\pgfqpoint{4.743341in}{3.110267in}}%
\pgfpathcurveto{\pgfqpoint{4.743341in}{3.123289in}}{\pgfqpoint{4.738167in}{3.135780in}}{\pgfqpoint{4.728958in}{3.144989in}}%
\pgfpathcurveto{\pgfqpoint{4.719750in}{3.154197in}}{\pgfqpoint{4.707259in}{3.159371in}}{\pgfqpoint{4.694236in}{3.159371in}}%
\pgfpathcurveto{\pgfqpoint{4.681213in}{3.159371in}}{\pgfqpoint{4.668722in}{3.154197in}}{\pgfqpoint{4.659514in}{3.144989in}}%
\pgfpathcurveto{\pgfqpoint{4.650306in}{3.135780in}}{\pgfqpoint{4.645132in}{3.123289in}}{\pgfqpoint{4.645132in}{3.110267in}}%
\pgfpathcurveto{\pgfqpoint{4.645132in}{3.097244in}}{\pgfqpoint{4.650306in}{3.084753in}}{\pgfqpoint{4.659514in}{3.075544in}}%
\pgfpathcurveto{\pgfqpoint{4.668722in}{3.066336in}}{\pgfqpoint{4.681213in}{3.061162in}}{\pgfqpoint{4.694236in}{3.061162in}}%
\pgfpathlineto{\pgfqpoint{4.694236in}{3.061162in}}%
\pgfpathclose%
\pgfusepath{stroke,fill}%
\end{pgfscope}%
\begin{pgfscope}%
\pgfpathrectangle{\pgfqpoint{0.786164in}{0.768110in}}{\pgfqpoint{8.851069in}{7.081890in}}%
\pgfusepath{clip}%
\pgfsetbuttcap%
\pgfsetroundjoin%
\definecolor{currentfill}{rgb}{0.157851,0.683765,0.501686}%
\pgfsetfillcolor{currentfill}%
\pgfsetfillopacity{0.700000}%
\pgfsetlinewidth{0.501875pt}%
\definecolor{currentstroke}{rgb}{1.000000,1.000000,1.000000}%
\pgfsetstrokecolor{currentstroke}%
\pgfsetstrokeopacity{0.700000}%
\pgfsetdash{}{0pt}%
\pgfpathmoveto{\pgfqpoint{4.844298in}{3.359952in}}%
\pgfpathcurveto{\pgfqpoint{4.857320in}{3.359952in}}{\pgfqpoint{4.869812in}{3.365126in}}{\pgfqpoint{4.879020in}{3.374335in}}%
\pgfpathcurveto{\pgfqpoint{4.888228in}{3.383543in}}{\pgfqpoint{4.893402in}{3.396034in}}{\pgfqpoint{4.893402in}{3.409057in}}%
\pgfpathcurveto{\pgfqpoint{4.893402in}{3.422080in}}{\pgfqpoint{4.888228in}{3.434571in}}{\pgfqpoint{4.879020in}{3.443779in}}%
\pgfpathcurveto{\pgfqpoint{4.869812in}{3.452988in}}{\pgfqpoint{4.857320in}{3.458162in}}{\pgfqpoint{4.844298in}{3.458162in}}%
\pgfpathcurveto{\pgfqpoint{4.831275in}{3.458162in}}{\pgfqpoint{4.818784in}{3.452988in}}{\pgfqpoint{4.809576in}{3.443779in}}%
\pgfpathcurveto{\pgfqpoint{4.800367in}{3.434571in}}{\pgfqpoint{4.795193in}{3.422080in}}{\pgfqpoint{4.795193in}{3.409057in}}%
\pgfpathcurveto{\pgfqpoint{4.795193in}{3.396034in}}{\pgfqpoint{4.800367in}{3.383543in}}{\pgfqpoint{4.809576in}{3.374335in}}%
\pgfpathcurveto{\pgfqpoint{4.818784in}{3.365126in}}{\pgfqpoint{4.831275in}{3.359952in}}{\pgfqpoint{4.844298in}{3.359952in}}%
\pgfpathlineto{\pgfqpoint{4.844298in}{3.359952in}}%
\pgfpathclose%
\pgfusepath{stroke,fill}%
\end{pgfscope}%
\begin{pgfscope}%
\pgfpathrectangle{\pgfqpoint{0.786164in}{0.768110in}}{\pgfqpoint{8.851069in}{7.081890in}}%
\pgfusepath{clip}%
\pgfsetbuttcap%
\pgfsetroundjoin%
\definecolor{currentfill}{rgb}{0.143303,0.669459,0.511215}%
\pgfsetfillcolor{currentfill}%
\pgfsetfillopacity{0.700000}%
\pgfsetlinewidth{0.501875pt}%
\definecolor{currentstroke}{rgb}{1.000000,1.000000,1.000000}%
\pgfsetstrokecolor{currentstroke}%
\pgfsetstrokeopacity{0.700000}%
\pgfsetdash{}{0pt}%
\pgfpathmoveto{\pgfqpoint{4.785445in}{3.680085in}}%
\pgfpathcurveto{\pgfqpoint{4.798468in}{3.680085in}}{\pgfqpoint{4.810959in}{3.685259in}}{\pgfqpoint{4.820168in}{3.694467in}}%
\pgfpathcurveto{\pgfqpoint{4.829376in}{3.703676in}}{\pgfqpoint{4.834550in}{3.716167in}}{\pgfqpoint{4.834550in}{3.729190in}}%
\pgfpathcurveto{\pgfqpoint{4.834550in}{3.742212in}}{\pgfqpoint{4.829376in}{3.754703in}}{\pgfqpoint{4.820168in}{3.763912in}}%
\pgfpathcurveto{\pgfqpoint{4.810959in}{3.773120in}}{\pgfqpoint{4.798468in}{3.778294in}}{\pgfqpoint{4.785445in}{3.778294in}}%
\pgfpathcurveto{\pgfqpoint{4.772423in}{3.778294in}}{\pgfqpoint{4.759932in}{3.773120in}}{\pgfqpoint{4.750723in}{3.763912in}}%
\pgfpathcurveto{\pgfqpoint{4.741515in}{3.754703in}}{\pgfqpoint{4.736341in}{3.742212in}}{\pgfqpoint{4.736341in}{3.729190in}}%
\pgfpathcurveto{\pgfqpoint{4.736341in}{3.716167in}}{\pgfqpoint{4.741515in}{3.703676in}}{\pgfqpoint{4.750723in}{3.694467in}}%
\pgfpathcurveto{\pgfqpoint{4.759932in}{3.685259in}}{\pgfqpoint{4.772423in}{3.680085in}}{\pgfqpoint{4.785445in}{3.680085in}}%
\pgfpathlineto{\pgfqpoint{4.785445in}{3.680085in}}%
\pgfpathclose%
\pgfusepath{stroke,fill}%
\end{pgfscope}%
\begin{pgfscope}%
\pgfpathrectangle{\pgfqpoint{0.786164in}{0.768110in}}{\pgfqpoint{8.851069in}{7.081890in}}%
\pgfusepath{clip}%
\pgfsetbuttcap%
\pgfsetroundjoin%
\definecolor{currentfill}{rgb}{0.123444,0.636809,0.528763}%
\pgfsetfillcolor{currentfill}%
\pgfsetfillopacity{0.700000}%
\pgfsetlinewidth{0.501875pt}%
\definecolor{currentstroke}{rgb}{1.000000,1.000000,1.000000}%
\pgfsetstrokecolor{currentstroke}%
\pgfsetstrokeopacity{0.700000}%
\pgfsetdash{}{0pt}%
\pgfpathmoveto{\pgfqpoint{4.969817in}{3.253242in}}%
\pgfpathcurveto{\pgfqpoint{4.982840in}{3.253242in}}{\pgfqpoint{4.995331in}{3.258415in}}{\pgfqpoint{5.004539in}{3.267624in}}%
\pgfpathcurveto{\pgfqpoint{5.013748in}{3.276832in}}{\pgfqpoint{5.018922in}{3.289323in}}{\pgfqpoint{5.018922in}{3.302346in}}%
\pgfpathcurveto{\pgfqpoint{5.018922in}{3.315369in}}{\pgfqpoint{5.013748in}{3.327860in}}{\pgfqpoint{5.004539in}{3.337068in}}%
\pgfpathcurveto{\pgfqpoint{4.995331in}{3.346277in}}{\pgfqpoint{4.982840in}{3.351451in}}{\pgfqpoint{4.969817in}{3.351451in}}%
\pgfpathcurveto{\pgfqpoint{4.956794in}{3.351451in}}{\pgfqpoint{4.944303in}{3.346277in}}{\pgfqpoint{4.935095in}{3.337068in}}%
\pgfpathcurveto{\pgfqpoint{4.925886in}{3.327860in}}{\pgfqpoint{4.920713in}{3.315369in}}{\pgfqpoint{4.920713in}{3.302346in}}%
\pgfpathcurveto{\pgfqpoint{4.920713in}{3.289323in}}{\pgfqpoint{4.925886in}{3.276832in}}{\pgfqpoint{4.935095in}{3.267624in}}%
\pgfpathcurveto{\pgfqpoint{4.944303in}{3.258415in}}{\pgfqpoint{4.956794in}{3.253242in}}{\pgfqpoint{4.969817in}{3.253242in}}%
\pgfpathlineto{\pgfqpoint{4.969817in}{3.253242in}}%
\pgfpathclose%
\pgfusepath{stroke,fill}%
\end{pgfscope}%
\begin{pgfscope}%
\pgfpathrectangle{\pgfqpoint{0.786164in}{0.768110in}}{\pgfqpoint{8.851069in}{7.081890in}}%
\pgfusepath{clip}%
\pgfsetbuttcap%
\pgfsetroundjoin%
\definecolor{currentfill}{rgb}{0.121831,0.589055,0.545623}%
\pgfsetfillcolor{currentfill}%
\pgfsetfillopacity{0.700000}%
\pgfsetlinewidth{0.501875pt}%
\definecolor{currentstroke}{rgb}{1.000000,1.000000,1.000000}%
\pgfsetstrokecolor{currentstroke}%
\pgfsetstrokeopacity{0.700000}%
\pgfsetdash{}{0pt}%
\pgfpathmoveto{\pgfqpoint{4.966643in}{2.591634in}}%
\pgfpathcurveto{\pgfqpoint{4.979665in}{2.591634in}}{\pgfqpoint{4.992156in}{2.596808in}}{\pgfqpoint{5.001365in}{2.606017in}}%
\pgfpathcurveto{\pgfqpoint{5.010573in}{2.615225in}}{\pgfqpoint{5.015747in}{2.627716in}}{\pgfqpoint{5.015747in}{2.640739in}}%
\pgfpathcurveto{\pgfqpoint{5.015747in}{2.653762in}}{\pgfqpoint{5.010573in}{2.666253in}}{\pgfqpoint{5.001365in}{2.675461in}}%
\pgfpathcurveto{\pgfqpoint{4.992156in}{2.684670in}}{\pgfqpoint{4.979665in}{2.689844in}}{\pgfqpoint{4.966643in}{2.689844in}}%
\pgfpathcurveto{\pgfqpoint{4.953620in}{2.689844in}}{\pgfqpoint{4.941129in}{2.684670in}}{\pgfqpoint{4.931920in}{2.675461in}}%
\pgfpathcurveto{\pgfqpoint{4.922712in}{2.666253in}}{\pgfqpoint{4.917538in}{2.653762in}}{\pgfqpoint{4.917538in}{2.640739in}}%
\pgfpathcurveto{\pgfqpoint{4.917538in}{2.627716in}}{\pgfqpoint{4.922712in}{2.615225in}}{\pgfqpoint{4.931920in}{2.606017in}}%
\pgfpathcurveto{\pgfqpoint{4.941129in}{2.596808in}}{\pgfqpoint{4.953620in}{2.591634in}}{\pgfqpoint{4.966643in}{2.591634in}}%
\pgfpathlineto{\pgfqpoint{4.966643in}{2.591634in}}%
\pgfpathclose%
\pgfusepath{stroke,fill}%
\end{pgfscope}%
\begin{pgfscope}%
\pgfpathrectangle{\pgfqpoint{0.786164in}{0.768110in}}{\pgfqpoint{8.851069in}{7.081890in}}%
\pgfusepath{clip}%
\pgfsetbuttcap%
\pgfsetroundjoin%
\definecolor{currentfill}{rgb}{0.128087,0.647749,0.523491}%
\pgfsetfillcolor{currentfill}%
\pgfsetfillopacity{0.700000}%
\pgfsetlinewidth{0.501875pt}%
\definecolor{currentstroke}{rgb}{1.000000,1.000000,1.000000}%
\pgfsetstrokecolor{currentstroke}%
\pgfsetstrokeopacity{0.700000}%
\pgfsetdash{}{0pt}%
\pgfpathmoveto{\pgfqpoint{5.103517in}{3.359952in}}%
\pgfpathcurveto{\pgfqpoint{5.116540in}{3.359952in}}{\pgfqpoint{5.129031in}{3.365126in}}{\pgfqpoint{5.138239in}{3.374335in}}%
\pgfpathcurveto{\pgfqpoint{5.147448in}{3.383543in}}{\pgfqpoint{5.152622in}{3.396034in}}{\pgfqpoint{5.152622in}{3.409057in}}%
\pgfpathcurveto{\pgfqpoint{5.152622in}{3.422080in}}{\pgfqpoint{5.147448in}{3.434571in}}{\pgfqpoint{5.138239in}{3.443779in}}%
\pgfpathcurveto{\pgfqpoint{5.129031in}{3.452988in}}{\pgfqpoint{5.116540in}{3.458162in}}{\pgfqpoint{5.103517in}{3.458162in}}%
\pgfpathcurveto{\pgfqpoint{5.090495in}{3.458162in}}{\pgfqpoint{5.078003in}{3.452988in}}{\pgfqpoint{5.068795in}{3.443779in}}%
\pgfpathcurveto{\pgfqpoint{5.059587in}{3.434571in}}{\pgfqpoint{5.054413in}{3.422080in}}{\pgfqpoint{5.054413in}{3.409057in}}%
\pgfpathcurveto{\pgfqpoint{5.054413in}{3.396034in}}{\pgfqpoint{5.059587in}{3.383543in}}{\pgfqpoint{5.068795in}{3.374335in}}%
\pgfpathcurveto{\pgfqpoint{5.078003in}{3.365126in}}{\pgfqpoint{5.090495in}{3.359952in}}{\pgfqpoint{5.103517in}{3.359952in}}%
\pgfpathlineto{\pgfqpoint{5.103517in}{3.359952in}}%
\pgfpathclose%
\pgfusepath{stroke,fill}%
\end{pgfscope}%
\begin{pgfscope}%
\pgfpathrectangle{\pgfqpoint{0.786164in}{0.768110in}}{\pgfqpoint{8.851069in}{7.081890in}}%
\pgfusepath{clip}%
\pgfsetbuttcap%
\pgfsetroundjoin%
\definecolor{currentfill}{rgb}{0.119699,0.618490,0.536347}%
\pgfsetfillcolor{currentfill}%
\pgfsetfillopacity{0.700000}%
\pgfsetlinewidth{0.501875pt}%
\definecolor{currentstroke}{rgb}{1.000000,1.000000,1.000000}%
\pgfsetstrokecolor{currentstroke}%
\pgfsetstrokeopacity{0.700000}%
\pgfsetdash{}{0pt}%
\pgfpathmoveto{\pgfqpoint{5.148206in}{3.082504in}}%
\pgfpathcurveto{\pgfqpoint{5.161229in}{3.082504in}}{\pgfqpoint{5.173720in}{3.087678in}}{\pgfqpoint{5.182928in}{3.096887in}}%
\pgfpathcurveto{\pgfqpoint{5.192137in}{3.106095in}}{\pgfqpoint{5.197311in}{3.118586in}}{\pgfqpoint{5.197311in}{3.131609in}}%
\pgfpathcurveto{\pgfqpoint{5.197311in}{3.144631in}}{\pgfqpoint{5.192137in}{3.157123in}}{\pgfqpoint{5.182928in}{3.166331in}}%
\pgfpathcurveto{\pgfqpoint{5.173720in}{3.175539in}}{\pgfqpoint{5.161229in}{3.180713in}}{\pgfqpoint{5.148206in}{3.180713in}}%
\pgfpathcurveto{\pgfqpoint{5.135183in}{3.180713in}}{\pgfqpoint{5.122692in}{3.175539in}}{\pgfqpoint{5.113484in}{3.166331in}}%
\pgfpathcurveto{\pgfqpoint{5.104275in}{3.157123in}}{\pgfqpoint{5.099101in}{3.144631in}}{\pgfqpoint{5.099101in}{3.131609in}}%
\pgfpathcurveto{\pgfqpoint{5.099101in}{3.118586in}}{\pgfqpoint{5.104275in}{3.106095in}}{\pgfqpoint{5.113484in}{3.096887in}}%
\pgfpathcurveto{\pgfqpoint{5.122692in}{3.087678in}}{\pgfqpoint{5.135183in}{3.082504in}}{\pgfqpoint{5.148206in}{3.082504in}}%
\pgfpathlineto{\pgfqpoint{5.148206in}{3.082504in}}%
\pgfpathclose%
\pgfusepath{stroke,fill}%
\end{pgfscope}%
\begin{pgfscope}%
\pgfpathrectangle{\pgfqpoint{0.786164in}{0.768110in}}{\pgfqpoint{8.851069in}{7.081890in}}%
\pgfusepath{clip}%
\pgfsetbuttcap%
\pgfsetroundjoin%
\definecolor{currentfill}{rgb}{0.121148,0.592739,0.544641}%
\pgfsetfillcolor{currentfill}%
\pgfsetfillopacity{0.700000}%
\pgfsetlinewidth{0.501875pt}%
\definecolor{currentstroke}{rgb}{1.000000,1.000000,1.000000}%
\pgfsetstrokecolor{currentstroke}%
\pgfsetstrokeopacity{0.700000}%
\pgfsetdash{}{0pt}%
\pgfpathmoveto{\pgfqpoint{5.354556in}{2.783714in}}%
\pgfpathcurveto{\pgfqpoint{5.367579in}{2.783714in}}{\pgfqpoint{5.380070in}{2.788888in}}{\pgfqpoint{5.389278in}{2.798096in}}%
\pgfpathcurveto{\pgfqpoint{5.398487in}{2.807305in}}{\pgfqpoint{5.403661in}{2.819796in}}{\pgfqpoint{5.403661in}{2.832818in}}%
\pgfpathcurveto{\pgfqpoint{5.403661in}{2.845841in}}{\pgfqpoint{5.398487in}{2.858332in}}{\pgfqpoint{5.389278in}{2.867541in}}%
\pgfpathcurveto{\pgfqpoint{5.380070in}{2.876749in}}{\pgfqpoint{5.367579in}{2.881923in}}{\pgfqpoint{5.354556in}{2.881923in}}%
\pgfpathcurveto{\pgfqpoint{5.341533in}{2.881923in}}{\pgfqpoint{5.329042in}{2.876749in}}{\pgfqpoint{5.319834in}{2.867541in}}%
\pgfpathcurveto{\pgfqpoint{5.310625in}{2.858332in}}{\pgfqpoint{5.305451in}{2.845841in}}{\pgfqpoint{5.305451in}{2.832818in}}%
\pgfpathcurveto{\pgfqpoint{5.305451in}{2.819796in}}{\pgfqpoint{5.310625in}{2.807305in}}{\pgfqpoint{5.319834in}{2.798096in}}%
\pgfpathcurveto{\pgfqpoint{5.329042in}{2.788888in}}{\pgfqpoint{5.341533in}{2.783714in}}{\pgfqpoint{5.354556in}{2.783714in}}%
\pgfpathlineto{\pgfqpoint{5.354556in}{2.783714in}}%
\pgfpathclose%
\pgfusepath{stroke,fill}%
\end{pgfscope}%
\begin{pgfscope}%
\pgfpathrectangle{\pgfqpoint{0.786164in}{0.768110in}}{\pgfqpoint{8.851069in}{7.081890in}}%
\pgfusepath{clip}%
\pgfsetbuttcap%
\pgfsetroundjoin%
\definecolor{currentfill}{rgb}{0.126453,0.570633,0.549841}%
\pgfsetfillcolor{currentfill}%
\pgfsetfillopacity{0.700000}%
\pgfsetlinewidth{0.501875pt}%
\definecolor{currentstroke}{rgb}{1.000000,1.000000,1.000000}%
\pgfsetstrokecolor{currentstroke}%
\pgfsetstrokeopacity{0.700000}%
\pgfsetdash{}{0pt}%
\pgfpathmoveto{\pgfqpoint{5.648574in}{3.061162in}}%
\pgfpathcurveto{\pgfqpoint{5.661597in}{3.061162in}}{\pgfqpoint{5.674088in}{3.066336in}}{\pgfqpoint{5.683296in}{3.075544in}}%
\pgfpathcurveto{\pgfqpoint{5.692505in}{3.084753in}}{\pgfqpoint{5.697679in}{3.097244in}}{\pgfqpoint{5.697679in}{3.110267in}}%
\pgfpathcurveto{\pgfqpoint{5.697679in}{3.123289in}}{\pgfqpoint{5.692505in}{3.135780in}}{\pgfqpoint{5.683296in}{3.144989in}}%
\pgfpathcurveto{\pgfqpoint{5.674088in}{3.154197in}}{\pgfqpoint{5.661597in}{3.159371in}}{\pgfqpoint{5.648574in}{3.159371in}}%
\pgfpathcurveto{\pgfqpoint{5.635551in}{3.159371in}}{\pgfqpoint{5.623060in}{3.154197in}}{\pgfqpoint{5.613852in}{3.144989in}}%
\pgfpathcurveto{\pgfqpoint{5.604643in}{3.135780in}}{\pgfqpoint{5.599469in}{3.123289in}}{\pgfqpoint{5.599469in}{3.110267in}}%
\pgfpathcurveto{\pgfqpoint{5.599469in}{3.097244in}}{\pgfqpoint{5.604643in}{3.084753in}}{\pgfqpoint{5.613852in}{3.075544in}}%
\pgfpathcurveto{\pgfqpoint{5.623060in}{3.066336in}}{\pgfqpoint{5.635551in}{3.061162in}}{\pgfqpoint{5.648574in}{3.061162in}}%
\pgfpathlineto{\pgfqpoint{5.648574in}{3.061162in}}%
\pgfpathclose%
\pgfusepath{stroke,fill}%
\end{pgfscope}%
\begin{pgfscope}%
\pgfpathrectangle{\pgfqpoint{0.786164in}{0.768110in}}{\pgfqpoint{8.851069in}{7.081890in}}%
\pgfusepath{clip}%
\pgfsetbuttcap%
\pgfsetroundjoin%
\definecolor{currentfill}{rgb}{0.124395,0.578002,0.548287}%
\pgfsetfillcolor{currentfill}%
\pgfsetfillopacity{0.700000}%
\pgfsetlinewidth{0.501875pt}%
\definecolor{currentstroke}{rgb}{1.000000,1.000000,1.000000}%
\pgfsetstrokecolor{currentstroke}%
\pgfsetstrokeopacity{0.700000}%
\pgfsetdash{}{0pt}%
\pgfpathmoveto{\pgfqpoint{5.893142in}{2.847740in}}%
\pgfpathcurveto{\pgfqpoint{5.906164in}{2.847740in}}{\pgfqpoint{5.918655in}{2.852914in}}{\pgfqpoint{5.927864in}{2.862123in}}%
\pgfpathcurveto{\pgfqpoint{5.937072in}{2.871331in}}{\pgfqpoint{5.942246in}{2.883822in}}{\pgfqpoint{5.942246in}{2.896845in}}%
\pgfpathcurveto{\pgfqpoint{5.942246in}{2.909868in}}{\pgfqpoint{5.937072in}{2.922359in}}{\pgfqpoint{5.927864in}{2.931567in}}%
\pgfpathcurveto{\pgfqpoint{5.918655in}{2.940776in}}{\pgfqpoint{5.906164in}{2.945950in}}{\pgfqpoint{5.893142in}{2.945950in}}%
\pgfpathcurveto{\pgfqpoint{5.880119in}{2.945950in}}{\pgfqpoint{5.867628in}{2.940776in}}{\pgfqpoint{5.858419in}{2.931567in}}%
\pgfpathcurveto{\pgfqpoint{5.849211in}{2.922359in}}{\pgfqpoint{5.844037in}{2.909868in}}{\pgfqpoint{5.844037in}{2.896845in}}%
\pgfpathcurveto{\pgfqpoint{5.844037in}{2.883822in}}{\pgfqpoint{5.849211in}{2.871331in}}{\pgfqpoint{5.858419in}{2.862123in}}%
\pgfpathcurveto{\pgfqpoint{5.867628in}{2.852914in}}{\pgfqpoint{5.880119in}{2.847740in}}{\pgfqpoint{5.893142in}{2.847740in}}%
\pgfpathlineto{\pgfqpoint{5.893142in}{2.847740in}}%
\pgfpathclose%
\pgfusepath{stroke,fill}%
\end{pgfscope}%
\begin{pgfscope}%
\pgfpathrectangle{\pgfqpoint{0.786164in}{0.768110in}}{\pgfqpoint{8.851069in}{7.081890in}}%
\pgfusepath{clip}%
\pgfsetbuttcap%
\pgfsetroundjoin%
\definecolor{currentfill}{rgb}{0.133743,0.548535,0.553541}%
\pgfsetfillcolor{currentfill}%
\pgfsetfillopacity{0.700000}%
\pgfsetlinewidth{0.501875pt}%
\definecolor{currentstroke}{rgb}{1.000000,1.000000,1.000000}%
\pgfsetstrokecolor{currentstroke}%
\pgfsetstrokeopacity{0.700000}%
\pgfsetdash{}{0pt}%
\pgfpathmoveto{\pgfqpoint{5.966036in}{2.890425in}}%
\pgfpathcurveto{\pgfqpoint{5.979058in}{2.890425in}}{\pgfqpoint{5.991549in}{2.895599in}}{\pgfqpoint{6.000758in}{2.904807in}}%
\pgfpathcurveto{\pgfqpoint{6.009966in}{2.914015in}}{\pgfqpoint{6.015140in}{2.926507in}}{\pgfqpoint{6.015140in}{2.939529in}}%
\pgfpathcurveto{\pgfqpoint{6.015140in}{2.952552in}}{\pgfqpoint{6.009966in}{2.965043in}}{\pgfqpoint{6.000758in}{2.974251in}}%
\pgfpathcurveto{\pgfqpoint{5.991549in}{2.983460in}}{\pgfqpoint{5.979058in}{2.988634in}}{\pgfqpoint{5.966036in}{2.988634in}}%
\pgfpathcurveto{\pgfqpoint{5.953013in}{2.988634in}}{\pgfqpoint{5.940522in}{2.983460in}}{\pgfqpoint{5.931313in}{2.974251in}}%
\pgfpathcurveto{\pgfqpoint{5.922105in}{2.965043in}}{\pgfqpoint{5.916931in}{2.952552in}}{\pgfqpoint{5.916931in}{2.939529in}}%
\pgfpathcurveto{\pgfqpoint{5.916931in}{2.926507in}}{\pgfqpoint{5.922105in}{2.914015in}}{\pgfqpoint{5.931313in}{2.904807in}}%
\pgfpathcurveto{\pgfqpoint{5.940522in}{2.895599in}}{\pgfqpoint{5.953013in}{2.890425in}}{\pgfqpoint{5.966036in}{2.890425in}}%
\pgfpathlineto{\pgfqpoint{5.966036in}{2.890425in}}%
\pgfpathclose%
\pgfusepath{stroke,fill}%
\end{pgfscope}%
\begin{pgfscope}%
\pgfpathrectangle{\pgfqpoint{0.786164in}{0.768110in}}{\pgfqpoint{8.851069in}{7.081890in}}%
\pgfusepath{clip}%
\pgfsetbuttcap%
\pgfsetroundjoin%
\definecolor{currentfill}{rgb}{0.127568,0.566949,0.550556}%
\pgfsetfillcolor{currentfill}%
\pgfsetfillopacity{0.700000}%
\pgfsetlinewidth{0.501875pt}%
\definecolor{currentstroke}{rgb}{1.000000,1.000000,1.000000}%
\pgfsetstrokecolor{currentstroke}%
\pgfsetstrokeopacity{0.700000}%
\pgfsetdash{}{0pt}%
\pgfpathmoveto{\pgfqpoint{5.930993in}{3.146531in}}%
\pgfpathcurveto{\pgfqpoint{5.944015in}{3.146531in}}{\pgfqpoint{5.956506in}{3.151705in}}{\pgfqpoint{5.965715in}{3.160913in}}%
\pgfpathcurveto{\pgfqpoint{5.974923in}{3.170122in}}{\pgfqpoint{5.980097in}{3.182613in}}{\pgfqpoint{5.980097in}{3.195635in}}%
\pgfpathcurveto{\pgfqpoint{5.980097in}{3.208658in}}{\pgfqpoint{5.974923in}{3.221149in}}{\pgfqpoint{5.965715in}{3.230358in}}%
\pgfpathcurveto{\pgfqpoint{5.956506in}{3.239566in}}{\pgfqpoint{5.944015in}{3.244740in}}{\pgfqpoint{5.930993in}{3.244740in}}%
\pgfpathcurveto{\pgfqpoint{5.917970in}{3.244740in}}{\pgfqpoint{5.905479in}{3.239566in}}{\pgfqpoint{5.896270in}{3.230358in}}%
\pgfpathcurveto{\pgfqpoint{5.887062in}{3.221149in}}{\pgfqpoint{5.881888in}{3.208658in}}{\pgfqpoint{5.881888in}{3.195635in}}%
\pgfpathcurveto{\pgfqpoint{5.881888in}{3.182613in}}{\pgfqpoint{5.887062in}{3.170122in}}{\pgfqpoint{5.896270in}{3.160913in}}%
\pgfpathcurveto{\pgfqpoint{5.905479in}{3.151705in}}{\pgfqpoint{5.917970in}{3.146531in}}{\pgfqpoint{5.930993in}{3.146531in}}%
\pgfpathlineto{\pgfqpoint{5.930993in}{3.146531in}}%
\pgfpathclose%
\pgfusepath{stroke,fill}%
\end{pgfscope}%
\begin{pgfscope}%
\pgfpathrectangle{\pgfqpoint{0.786164in}{0.768110in}}{\pgfqpoint{8.851069in}{7.081890in}}%
\pgfusepath{clip}%
\pgfsetbuttcap%
\pgfsetroundjoin%
\definecolor{currentfill}{rgb}{0.128729,0.563265,0.551229}%
\pgfsetfillcolor{currentfill}%
\pgfsetfillopacity{0.700000}%
\pgfsetlinewidth{0.501875pt}%
\definecolor{currentstroke}{rgb}{1.000000,1.000000,1.000000}%
\pgfsetstrokecolor{currentstroke}%
\pgfsetstrokeopacity{0.700000}%
\pgfsetdash{}{0pt}%
\pgfpathmoveto{\pgfqpoint{6.164693in}{3.125188in}}%
\pgfpathcurveto{\pgfqpoint{6.177716in}{3.125188in}}{\pgfqpoint{6.190207in}{3.130362in}}{\pgfqpoint{6.199415in}{3.139571in}}%
\pgfpathcurveto{\pgfqpoint{6.208624in}{3.148779in}}{\pgfqpoint{6.213798in}{3.161270in}}{\pgfqpoint{6.213798in}{3.174293in}}%
\pgfpathcurveto{\pgfqpoint{6.213798in}{3.187316in}}{\pgfqpoint{6.208624in}{3.199807in}}{\pgfqpoint{6.199415in}{3.209015in}}%
\pgfpathcurveto{\pgfqpoint{6.190207in}{3.218224in}}{\pgfqpoint{6.177716in}{3.223398in}}{\pgfqpoint{6.164693in}{3.223398in}}%
\pgfpathcurveto{\pgfqpoint{6.151670in}{3.223398in}}{\pgfqpoint{6.139179in}{3.218224in}}{\pgfqpoint{6.129971in}{3.209015in}}%
\pgfpathcurveto{\pgfqpoint{6.120762in}{3.199807in}}{\pgfqpoint{6.115588in}{3.187316in}}{\pgfqpoint{6.115588in}{3.174293in}}%
\pgfpathcurveto{\pgfqpoint{6.115588in}{3.161270in}}{\pgfqpoint{6.120762in}{3.148779in}}{\pgfqpoint{6.129971in}{3.139571in}}%
\pgfpathcurveto{\pgfqpoint{6.139179in}{3.130362in}}{\pgfqpoint{6.151670in}{3.125188in}}{\pgfqpoint{6.164693in}{3.125188in}}%
\pgfpathlineto{\pgfqpoint{6.164693in}{3.125188in}}%
\pgfpathclose%
\pgfusepath{stroke,fill}%
\end{pgfscope}%
\begin{pgfscope}%
\pgfpathrectangle{\pgfqpoint{0.786164in}{0.768110in}}{\pgfqpoint{8.851069in}{7.081890in}}%
\pgfusepath{clip}%
\pgfsetbuttcap%
\pgfsetroundjoin%
\definecolor{currentfill}{rgb}{0.124395,0.578002,0.548287}%
\pgfsetfillcolor{currentfill}%
\pgfsetfillopacity{0.700000}%
\pgfsetlinewidth{0.501875pt}%
\definecolor{currentstroke}{rgb}{1.000000,1.000000,1.000000}%
\pgfsetstrokecolor{currentstroke}%
\pgfsetstrokeopacity{0.700000}%
\pgfsetdash{}{0pt}%
\pgfpathmoveto{\pgfqpoint{6.204742in}{3.530690in}}%
\pgfpathcurveto{\pgfqpoint{6.217765in}{3.530690in}}{\pgfqpoint{6.230256in}{3.535864in}}{\pgfqpoint{6.239464in}{3.545072in}}%
\pgfpathcurveto{\pgfqpoint{6.248673in}{3.554281in}}{\pgfqpoint{6.253847in}{3.566772in}}{\pgfqpoint{6.253847in}{3.579794in}}%
\pgfpathcurveto{\pgfqpoint{6.253847in}{3.592817in}}{\pgfqpoint{6.248673in}{3.605308in}}{\pgfqpoint{6.239464in}{3.614517in}}%
\pgfpathcurveto{\pgfqpoint{6.230256in}{3.623725in}}{\pgfqpoint{6.217765in}{3.628899in}}{\pgfqpoint{6.204742in}{3.628899in}}%
\pgfpathcurveto{\pgfqpoint{6.191719in}{3.628899in}}{\pgfqpoint{6.179228in}{3.623725in}}{\pgfqpoint{6.170020in}{3.614517in}}%
\pgfpathcurveto{\pgfqpoint{6.160811in}{3.605308in}}{\pgfqpoint{6.155637in}{3.592817in}}{\pgfqpoint{6.155637in}{3.579794in}}%
\pgfpathcurveto{\pgfqpoint{6.155637in}{3.566772in}}{\pgfqpoint{6.160811in}{3.554281in}}{\pgfqpoint{6.170020in}{3.545072in}}%
\pgfpathcurveto{\pgfqpoint{6.179228in}{3.535864in}}{\pgfqpoint{6.191719in}{3.530690in}}{\pgfqpoint{6.204742in}{3.530690in}}%
\pgfpathlineto{\pgfqpoint{6.204742in}{3.530690in}}%
\pgfpathclose%
\pgfusepath{stroke,fill}%
\end{pgfscope}%
\begin{pgfscope}%
\pgfpathrectangle{\pgfqpoint{0.786164in}{0.768110in}}{\pgfqpoint{8.851069in}{7.081890in}}%
\pgfusepath{clip}%
\pgfsetbuttcap%
\pgfsetroundjoin%
\definecolor{currentfill}{rgb}{0.129933,0.559582,0.551864}%
\pgfsetfillcolor{currentfill}%
\pgfsetfillopacity{0.700000}%
\pgfsetlinewidth{0.501875pt}%
\definecolor{currentstroke}{rgb}{1.000000,1.000000,1.000000}%
\pgfsetstrokecolor{currentstroke}%
\pgfsetstrokeopacity{0.700000}%
\pgfsetdash{}{0pt}%
\pgfpathmoveto{\pgfqpoint{6.402789in}{3.189215in}}%
\pgfpathcurveto{\pgfqpoint{6.415812in}{3.189215in}}{\pgfqpoint{6.428303in}{3.194389in}}{\pgfqpoint{6.437511in}{3.203597in}}%
\pgfpathcurveto{\pgfqpoint{6.446720in}{3.212806in}}{\pgfqpoint{6.451894in}{3.225297in}}{\pgfqpoint{6.451894in}{3.238320in}}%
\pgfpathcurveto{\pgfqpoint{6.451894in}{3.251342in}}{\pgfqpoint{6.446720in}{3.263833in}}{\pgfqpoint{6.437511in}{3.273042in}}%
\pgfpathcurveto{\pgfqpoint{6.428303in}{3.282250in}}{\pgfqpoint{6.415812in}{3.287424in}}{\pgfqpoint{6.402789in}{3.287424in}}%
\pgfpathcurveto{\pgfqpoint{6.389766in}{3.287424in}}{\pgfqpoint{6.377275in}{3.282250in}}{\pgfqpoint{6.368067in}{3.273042in}}%
\pgfpathcurveto{\pgfqpoint{6.358859in}{3.263833in}}{\pgfqpoint{6.353685in}{3.251342in}}{\pgfqpoint{6.353685in}{3.238320in}}%
\pgfpathcurveto{\pgfqpoint{6.353685in}{3.225297in}}{\pgfqpoint{6.358859in}{3.212806in}}{\pgfqpoint{6.368067in}{3.203597in}}%
\pgfpathcurveto{\pgfqpoint{6.377275in}{3.194389in}}{\pgfqpoint{6.389766in}{3.189215in}}{\pgfqpoint{6.402789in}{3.189215in}}%
\pgfpathlineto{\pgfqpoint{6.402789in}{3.189215in}}%
\pgfpathclose%
\pgfusepath{stroke,fill}%
\end{pgfscope}%
\begin{pgfscope}%
\pgfpathrectangle{\pgfqpoint{0.786164in}{0.768110in}}{\pgfqpoint{8.851069in}{7.081890in}}%
\pgfusepath{clip}%
\pgfsetbuttcap%
\pgfsetroundjoin%
\definecolor{currentfill}{rgb}{0.144759,0.519093,0.556572}%
\pgfsetfillcolor{currentfill}%
\pgfsetfillopacity{0.700000}%
\pgfsetlinewidth{0.501875pt}%
\definecolor{currentstroke}{rgb}{1.000000,1.000000,1.000000}%
\pgfsetstrokecolor{currentstroke}%
\pgfsetstrokeopacity{0.700000}%
\pgfsetdash{}{0pt}%
\pgfpathmoveto{\pgfqpoint{6.541007in}{2.847740in}}%
\pgfpathcurveto{\pgfqpoint{6.554030in}{2.847740in}}{\pgfqpoint{6.566521in}{2.852914in}}{\pgfqpoint{6.575729in}{2.862123in}}%
\pgfpathcurveto{\pgfqpoint{6.584938in}{2.871331in}}{\pgfqpoint{6.590112in}{2.883822in}}{\pgfqpoint{6.590112in}{2.896845in}}%
\pgfpathcurveto{\pgfqpoint{6.590112in}{2.909868in}}{\pgfqpoint{6.584938in}{2.922359in}}{\pgfqpoint{6.575729in}{2.931567in}}%
\pgfpathcurveto{\pgfqpoint{6.566521in}{2.940776in}}{\pgfqpoint{6.554030in}{2.945950in}}{\pgfqpoint{6.541007in}{2.945950in}}%
\pgfpathcurveto{\pgfqpoint{6.527984in}{2.945950in}}{\pgfqpoint{6.515493in}{2.940776in}}{\pgfqpoint{6.506285in}{2.931567in}}%
\pgfpathcurveto{\pgfqpoint{6.497076in}{2.922359in}}{\pgfqpoint{6.491902in}{2.909868in}}{\pgfqpoint{6.491902in}{2.896845in}}%
\pgfpathcurveto{\pgfqpoint{6.491902in}{2.883822in}}{\pgfqpoint{6.497076in}{2.871331in}}{\pgfqpoint{6.506285in}{2.862123in}}%
\pgfpathcurveto{\pgfqpoint{6.515493in}{2.852914in}}{\pgfqpoint{6.527984in}{2.847740in}}{\pgfqpoint{6.541007in}{2.847740in}}%
\pgfpathlineto{\pgfqpoint{6.541007in}{2.847740in}}%
\pgfpathclose%
\pgfusepath{stroke,fill}%
\end{pgfscope}%
\begin{pgfscope}%
\pgfpathrectangle{\pgfqpoint{0.786164in}{0.768110in}}{\pgfqpoint{8.851069in}{7.081890in}}%
\pgfusepath{clip}%
\pgfsetbuttcap%
\pgfsetroundjoin%
\definecolor{currentfill}{rgb}{0.133743,0.548535,0.553541}%
\pgfsetfillcolor{currentfill}%
\pgfsetfillopacity{0.700000}%
\pgfsetlinewidth{0.501875pt}%
\definecolor{currentstroke}{rgb}{1.000000,1.000000,1.000000}%
\pgfsetstrokecolor{currentstroke}%
\pgfsetstrokeopacity{0.700000}%
\pgfsetdash{}{0pt}%
\pgfpathmoveto{\pgfqpoint{6.408528in}{3.189215in}}%
\pgfpathcurveto{\pgfqpoint{6.421551in}{3.189215in}}{\pgfqpoint{6.434042in}{3.194389in}}{\pgfqpoint{6.443250in}{3.203597in}}%
\pgfpathcurveto{\pgfqpoint{6.452459in}{3.212806in}}{\pgfqpoint{6.457633in}{3.225297in}}{\pgfqpoint{6.457633in}{3.238320in}}%
\pgfpathcurveto{\pgfqpoint{6.457633in}{3.251342in}}{\pgfqpoint{6.452459in}{3.263833in}}{\pgfqpoint{6.443250in}{3.273042in}}%
\pgfpathcurveto{\pgfqpoint{6.434042in}{3.282250in}}{\pgfqpoint{6.421551in}{3.287424in}}{\pgfqpoint{6.408528in}{3.287424in}}%
\pgfpathcurveto{\pgfqpoint{6.395505in}{3.287424in}}{\pgfqpoint{6.383014in}{3.282250in}}{\pgfqpoint{6.373806in}{3.273042in}}%
\pgfpathcurveto{\pgfqpoint{6.364597in}{3.263833in}}{\pgfqpoint{6.359423in}{3.251342in}}{\pgfqpoint{6.359423in}{3.238320in}}%
\pgfpathcurveto{\pgfqpoint{6.359423in}{3.225297in}}{\pgfqpoint{6.364597in}{3.212806in}}{\pgfqpoint{6.373806in}{3.203597in}}%
\pgfpathcurveto{\pgfqpoint{6.383014in}{3.194389in}}{\pgfqpoint{6.395505in}{3.189215in}}{\pgfqpoint{6.408528in}{3.189215in}}%
\pgfpathlineto{\pgfqpoint{6.408528in}{3.189215in}}%
\pgfpathclose%
\pgfusepath{stroke,fill}%
\end{pgfscope}%
\begin{pgfscope}%
\pgfpathrectangle{\pgfqpoint{0.786164in}{0.768110in}}{\pgfqpoint{8.851069in}{7.081890in}}%
\pgfusepath{clip}%
\pgfsetbuttcap%
\pgfsetroundjoin%
\definecolor{currentfill}{rgb}{0.144759,0.519093,0.556572}%
\pgfsetfillcolor{currentfill}%
\pgfsetfillopacity{0.700000}%
\pgfsetlinewidth{0.501875pt}%
\definecolor{currentstroke}{rgb}{1.000000,1.000000,1.000000}%
\pgfsetstrokecolor{currentstroke}%
\pgfsetstrokeopacity{0.700000}%
\pgfsetdash{}{0pt}%
\pgfpathmoveto{\pgfqpoint{7.022938in}{3.167873in}}%
\pgfpathcurveto{\pgfqpoint{7.035961in}{3.167873in}}{\pgfqpoint{7.048452in}{3.173047in}}{\pgfqpoint{7.057660in}{3.182255in}}%
\pgfpathcurveto{\pgfqpoint{7.066869in}{3.191464in}}{\pgfqpoint{7.072043in}{3.203955in}}{\pgfqpoint{7.072043in}{3.216977in}}%
\pgfpathcurveto{\pgfqpoint{7.072043in}{3.230000in}}{\pgfqpoint{7.066869in}{3.242491in}}{\pgfqpoint{7.057660in}{3.251700in}}%
\pgfpathcurveto{\pgfqpoint{7.048452in}{3.260908in}}{\pgfqpoint{7.035961in}{3.266082in}}{\pgfqpoint{7.022938in}{3.266082in}}%
\pgfpathcurveto{\pgfqpoint{7.009915in}{3.266082in}}{\pgfqpoint{6.997424in}{3.260908in}}{\pgfqpoint{6.988216in}{3.251700in}}%
\pgfpathcurveto{\pgfqpoint{6.979007in}{3.242491in}}{\pgfqpoint{6.973833in}{3.230000in}}{\pgfqpoint{6.973833in}{3.216977in}}%
\pgfpathcurveto{\pgfqpoint{6.973833in}{3.203955in}}{\pgfqpoint{6.979007in}{3.191464in}}{\pgfqpoint{6.988216in}{3.182255in}}%
\pgfpathcurveto{\pgfqpoint{6.997424in}{3.173047in}}{\pgfqpoint{7.009915in}{3.167873in}}{\pgfqpoint{7.022938in}{3.167873in}}%
\pgfpathlineto{\pgfqpoint{7.022938in}{3.167873in}}%
\pgfpathclose%
\pgfusepath{stroke,fill}%
\end{pgfscope}%
\begin{pgfscope}%
\pgfpathrectangle{\pgfqpoint{0.786164in}{0.768110in}}{\pgfqpoint{8.851069in}{7.081890in}}%
\pgfusepath{clip}%
\pgfsetbuttcap%
\pgfsetroundjoin%
\definecolor{currentfill}{rgb}{0.203063,0.379716,0.553925}%
\pgfsetfillcolor{currentfill}%
\pgfsetfillopacity{0.700000}%
\pgfsetlinewidth{0.501875pt}%
\definecolor{currentstroke}{rgb}{1.000000,1.000000,1.000000}%
\pgfsetstrokecolor{currentstroke}%
\pgfsetstrokeopacity{0.700000}%
\pgfsetdash{}{0pt}%
\pgfpathmoveto{\pgfqpoint{2.313886in}{2.933109in}}%
\pgfpathcurveto{\pgfqpoint{2.326909in}{2.933109in}}{\pgfqpoint{2.339400in}{2.938283in}}{\pgfqpoint{2.348608in}{2.947491in}}%
\pgfpathcurveto{\pgfqpoint{2.357817in}{2.956700in}}{\pgfqpoint{2.362991in}{2.969191in}}{\pgfqpoint{2.362991in}{2.982214in}}%
\pgfpathcurveto{\pgfqpoint{2.362991in}{2.995236in}}{\pgfqpoint{2.357817in}{3.007727in}}{\pgfqpoint{2.348608in}{3.016936in}}%
\pgfpathcurveto{\pgfqpoint{2.339400in}{3.026144in}}{\pgfqpoint{2.326909in}{3.031318in}}{\pgfqpoint{2.313886in}{3.031318in}}%
\pgfpathcurveto{\pgfqpoint{2.300863in}{3.031318in}}{\pgfqpoint{2.288372in}{3.026144in}}{\pgfqpoint{2.279164in}{3.016936in}}%
\pgfpathcurveto{\pgfqpoint{2.269955in}{3.007727in}}{\pgfqpoint{2.264781in}{2.995236in}}{\pgfqpoint{2.264781in}{2.982214in}}%
\pgfpathcurveto{\pgfqpoint{2.264781in}{2.969191in}}{\pgfqpoint{2.269955in}{2.956700in}}{\pgfqpoint{2.279164in}{2.947491in}}%
\pgfpathcurveto{\pgfqpoint{2.288372in}{2.938283in}}{\pgfqpoint{2.300863in}{2.933109in}}{\pgfqpoint{2.313886in}{2.933109in}}%
\pgfpathlineto{\pgfqpoint{2.313886in}{2.933109in}}%
\pgfpathclose%
\pgfusepath{stroke,fill}%
\end{pgfscope}%
\begin{pgfscope}%
\pgfpathrectangle{\pgfqpoint{0.786164in}{0.768110in}}{\pgfqpoint{8.851069in}{7.081890in}}%
\pgfusepath{clip}%
\pgfsetbuttcap%
\pgfsetroundjoin%
\definecolor{currentfill}{rgb}{0.203063,0.379716,0.553925}%
\pgfsetfillcolor{currentfill}%
\pgfsetfillopacity{0.700000}%
\pgfsetlinewidth{0.501875pt}%
\definecolor{currentstroke}{rgb}{1.000000,1.000000,1.000000}%
\pgfsetstrokecolor{currentstroke}%
\pgfsetstrokeopacity{0.700000}%
\pgfsetdash{}{0pt}%
\pgfpathmoveto{\pgfqpoint{2.308147in}{2.911767in}}%
\pgfpathcurveto{\pgfqpoint{2.321170in}{2.911767in}}{\pgfqpoint{2.333661in}{2.916941in}}{\pgfqpoint{2.342869in}{2.926149in}}%
\pgfpathcurveto{\pgfqpoint{2.352078in}{2.935358in}}{\pgfqpoint{2.357252in}{2.947849in}}{\pgfqpoint{2.357252in}{2.960871in}}%
\pgfpathcurveto{\pgfqpoint{2.357252in}{2.973894in}}{\pgfqpoint{2.352078in}{2.986385in}}{\pgfqpoint{2.342869in}{2.995594in}}%
\pgfpathcurveto{\pgfqpoint{2.333661in}{3.004802in}}{\pgfqpoint{2.321170in}{3.009976in}}{\pgfqpoint{2.308147in}{3.009976in}}%
\pgfpathcurveto{\pgfqpoint{2.295125in}{3.009976in}}{\pgfqpoint{2.282633in}{3.004802in}}{\pgfqpoint{2.273425in}{2.995594in}}%
\pgfpathcurveto{\pgfqpoint{2.264217in}{2.986385in}}{\pgfqpoint{2.259043in}{2.973894in}}{\pgfqpoint{2.259043in}{2.960871in}}%
\pgfpathcurveto{\pgfqpoint{2.259043in}{2.947849in}}{\pgfqpoint{2.264217in}{2.935358in}}{\pgfqpoint{2.273425in}{2.926149in}}%
\pgfpathcurveto{\pgfqpoint{2.282633in}{2.916941in}}{\pgfqpoint{2.295125in}{2.911767in}}{\pgfqpoint{2.308147in}{2.911767in}}%
\pgfpathlineto{\pgfqpoint{2.308147in}{2.911767in}}%
\pgfpathclose%
\pgfusepath{stroke,fill}%
\end{pgfscope}%
\begin{pgfscope}%
\pgfpathrectangle{\pgfqpoint{0.786164in}{0.768110in}}{\pgfqpoint{8.851069in}{7.081890in}}%
\pgfusepath{clip}%
\pgfsetbuttcap%
\pgfsetroundjoin%
\definecolor{currentfill}{rgb}{0.208623,0.367752,0.552675}%
\pgfsetfillcolor{currentfill}%
\pgfsetfillopacity{0.700000}%
\pgfsetlinewidth{0.501875pt}%
\definecolor{currentstroke}{rgb}{1.000000,1.000000,1.000000}%
\pgfsetstrokecolor{currentstroke}%
\pgfsetstrokeopacity{0.700000}%
\pgfsetdash{}{0pt}%
\pgfpathmoveto{\pgfqpoint{2.267121in}{2.869082in}}%
\pgfpathcurveto{\pgfqpoint{2.280144in}{2.869082in}}{\pgfqpoint{2.292635in}{2.874256in}}{\pgfqpoint{2.301844in}{2.883465in}}%
\pgfpathcurveto{\pgfqpoint{2.311052in}{2.892673in}}{\pgfqpoint{2.316226in}{2.905164in}}{\pgfqpoint{2.316226in}{2.918187in}}%
\pgfpathcurveto{\pgfqpoint{2.316226in}{2.931210in}}{\pgfqpoint{2.311052in}{2.943701in}}{\pgfqpoint{2.301844in}{2.952909in}}%
\pgfpathcurveto{\pgfqpoint{2.292635in}{2.962118in}}{\pgfqpoint{2.280144in}{2.967292in}}{\pgfqpoint{2.267121in}{2.967292in}}%
\pgfpathcurveto{\pgfqpoint{2.254099in}{2.967292in}}{\pgfqpoint{2.241608in}{2.962118in}}{\pgfqpoint{2.232399in}{2.952909in}}%
\pgfpathcurveto{\pgfqpoint{2.223191in}{2.943701in}}{\pgfqpoint{2.218017in}{2.931210in}}{\pgfqpoint{2.218017in}{2.918187in}}%
\pgfpathcurveto{\pgfqpoint{2.218017in}{2.905164in}}{\pgfqpoint{2.223191in}{2.892673in}}{\pgfqpoint{2.232399in}{2.883465in}}%
\pgfpathcurveto{\pgfqpoint{2.241608in}{2.874256in}}{\pgfqpoint{2.254099in}{2.869082in}}{\pgfqpoint{2.267121in}{2.869082in}}%
\pgfpathlineto{\pgfqpoint{2.267121in}{2.869082in}}%
\pgfpathclose%
\pgfusepath{stroke,fill}%
\end{pgfscope}%
\begin{pgfscope}%
\pgfpathrectangle{\pgfqpoint{0.786164in}{0.768110in}}{\pgfqpoint{8.851069in}{7.081890in}}%
\pgfusepath{clip}%
\pgfsetbuttcap%
\pgfsetroundjoin%
\definecolor{currentfill}{rgb}{0.212395,0.359683,0.551710}%
\pgfsetfillcolor{currentfill}%
\pgfsetfillopacity{0.700000}%
\pgfsetlinewidth{0.501875pt}%
\definecolor{currentstroke}{rgb}{1.000000,1.000000,1.000000}%
\pgfsetstrokecolor{currentstroke}%
\pgfsetstrokeopacity{0.700000}%
\pgfsetdash{}{0pt}%
\pgfpathmoveto{\pgfqpoint{2.326951in}{2.847740in}}%
\pgfpathcurveto{\pgfqpoint{2.339973in}{2.847740in}}{\pgfqpoint{2.352465in}{2.852914in}}{\pgfqpoint{2.361673in}{2.862123in}}%
\pgfpathcurveto{\pgfqpoint{2.370881in}{2.871331in}}{\pgfqpoint{2.376055in}{2.883822in}}{\pgfqpoint{2.376055in}{2.896845in}}%
\pgfpathcurveto{\pgfqpoint{2.376055in}{2.909868in}}{\pgfqpoint{2.370881in}{2.922359in}}{\pgfqpoint{2.361673in}{2.931567in}}%
\pgfpathcurveto{\pgfqpoint{2.352465in}{2.940776in}}{\pgfqpoint{2.339973in}{2.945950in}}{\pgfqpoint{2.326951in}{2.945950in}}%
\pgfpathcurveto{\pgfqpoint{2.313928in}{2.945950in}}{\pgfqpoint{2.301437in}{2.940776in}}{\pgfqpoint{2.292229in}{2.931567in}}%
\pgfpathcurveto{\pgfqpoint{2.283020in}{2.922359in}}{\pgfqpoint{2.277846in}{2.909868in}}{\pgfqpoint{2.277846in}{2.896845in}}%
\pgfpathcurveto{\pgfqpoint{2.277846in}{2.883822in}}{\pgfqpoint{2.283020in}{2.871331in}}{\pgfqpoint{2.292229in}{2.862123in}}%
\pgfpathcurveto{\pgfqpoint{2.301437in}{2.852914in}}{\pgfqpoint{2.313928in}{2.847740in}}{\pgfqpoint{2.326951in}{2.847740in}}%
\pgfpathlineto{\pgfqpoint{2.326951in}{2.847740in}}%
\pgfpathclose%
\pgfusepath{stroke,fill}%
\end{pgfscope}%
\begin{pgfscope}%
\pgfpathrectangle{\pgfqpoint{0.786164in}{0.768110in}}{\pgfqpoint{8.851069in}{7.081890in}}%
\pgfusepath{clip}%
\pgfsetbuttcap%
\pgfsetroundjoin%
\definecolor{currentfill}{rgb}{0.212395,0.359683,0.551710}%
\pgfsetfillcolor{currentfill}%
\pgfsetfillopacity{0.700000}%
\pgfsetlinewidth{0.501875pt}%
\definecolor{currentstroke}{rgb}{1.000000,1.000000,1.000000}%
\pgfsetstrokecolor{currentstroke}%
\pgfsetstrokeopacity{0.700000}%
\pgfsetdash{}{0pt}%
\pgfpathmoveto{\pgfqpoint{2.542092in}{2.826398in}}%
\pgfpathcurveto{\pgfqpoint{2.555115in}{2.826398in}}{\pgfqpoint{2.567606in}{2.831572in}}{\pgfqpoint{2.576814in}{2.840781in}}%
\pgfpathcurveto{\pgfqpoint{2.586023in}{2.849989in}}{\pgfqpoint{2.591197in}{2.862480in}}{\pgfqpoint{2.591197in}{2.875503in}}%
\pgfpathcurveto{\pgfqpoint{2.591197in}{2.888525in}}{\pgfqpoint{2.586023in}{2.901017in}}{\pgfqpoint{2.576814in}{2.910225in}}%
\pgfpathcurveto{\pgfqpoint{2.567606in}{2.919433in}}{\pgfqpoint{2.555115in}{2.924607in}}{\pgfqpoint{2.542092in}{2.924607in}}%
\pgfpathcurveto{\pgfqpoint{2.529069in}{2.924607in}}{\pgfqpoint{2.516578in}{2.919433in}}{\pgfqpoint{2.507370in}{2.910225in}}%
\pgfpathcurveto{\pgfqpoint{2.498161in}{2.901017in}}{\pgfqpoint{2.492987in}{2.888525in}}{\pgfqpoint{2.492987in}{2.875503in}}%
\pgfpathcurveto{\pgfqpoint{2.492987in}{2.862480in}}{\pgfqpoint{2.498161in}{2.849989in}}{\pgfqpoint{2.507370in}{2.840781in}}%
\pgfpathcurveto{\pgfqpoint{2.516578in}{2.831572in}}{\pgfqpoint{2.529069in}{2.826398in}}{\pgfqpoint{2.542092in}{2.826398in}}%
\pgfpathlineto{\pgfqpoint{2.542092in}{2.826398in}}%
\pgfpathclose%
\pgfusepath{stroke,fill}%
\end{pgfscope}%
\begin{pgfscope}%
\pgfpathrectangle{\pgfqpoint{0.786164in}{0.768110in}}{\pgfqpoint{8.851069in}{7.081890in}}%
\pgfusepath{clip}%
\pgfsetbuttcap%
\pgfsetroundjoin%
\definecolor{currentfill}{rgb}{0.220057,0.343307,0.549413}%
\pgfsetfillcolor{currentfill}%
\pgfsetfillopacity{0.700000}%
\pgfsetlinewidth{0.501875pt}%
\definecolor{currentstroke}{rgb}{1.000000,1.000000,1.000000}%
\pgfsetstrokecolor{currentstroke}%
\pgfsetstrokeopacity{0.700000}%
\pgfsetdash{}{0pt}%
\pgfpathmoveto{\pgfqpoint{2.667489in}{2.783714in}}%
\pgfpathcurveto{\pgfqpoint{2.680512in}{2.783714in}}{\pgfqpoint{2.693003in}{2.788888in}}{\pgfqpoint{2.702211in}{2.798096in}}%
\pgfpathcurveto{\pgfqpoint{2.711420in}{2.807305in}}{\pgfqpoint{2.716594in}{2.819796in}}{\pgfqpoint{2.716594in}{2.832818in}}%
\pgfpathcurveto{\pgfqpoint{2.716594in}{2.845841in}}{\pgfqpoint{2.711420in}{2.858332in}}{\pgfqpoint{2.702211in}{2.867541in}}%
\pgfpathcurveto{\pgfqpoint{2.693003in}{2.876749in}}{\pgfqpoint{2.680512in}{2.881923in}}{\pgfqpoint{2.667489in}{2.881923in}}%
\pgfpathcurveto{\pgfqpoint{2.654466in}{2.881923in}}{\pgfqpoint{2.641975in}{2.876749in}}{\pgfqpoint{2.632767in}{2.867541in}}%
\pgfpathcurveto{\pgfqpoint{2.623559in}{2.858332in}}{\pgfqpoint{2.618385in}{2.845841in}}{\pgfqpoint{2.618385in}{2.832818in}}%
\pgfpathcurveto{\pgfqpoint{2.618385in}{2.819796in}}{\pgfqpoint{2.623559in}{2.807305in}}{\pgfqpoint{2.632767in}{2.798096in}}%
\pgfpathcurveto{\pgfqpoint{2.641975in}{2.788888in}}{\pgfqpoint{2.654466in}{2.783714in}}{\pgfqpoint{2.667489in}{2.783714in}}%
\pgfpathlineto{\pgfqpoint{2.667489in}{2.783714in}}%
\pgfpathclose%
\pgfusepath{stroke,fill}%
\end{pgfscope}%
\begin{pgfscope}%
\pgfpathrectangle{\pgfqpoint{0.786164in}{0.768110in}}{\pgfqpoint{8.851069in}{7.081890in}}%
\pgfusepath{clip}%
\pgfsetbuttcap%
\pgfsetroundjoin%
\definecolor{currentfill}{rgb}{0.214298,0.355619,0.551184}%
\pgfsetfillcolor{currentfill}%
\pgfsetfillopacity{0.700000}%
\pgfsetlinewidth{0.501875pt}%
\definecolor{currentstroke}{rgb}{1.000000,1.000000,1.000000}%
\pgfsetstrokecolor{currentstroke}%
\pgfsetstrokeopacity{0.700000}%
\pgfsetdash{}{0pt}%
\pgfpathmoveto{\pgfqpoint{2.723167in}{2.783714in}}%
\pgfpathcurveto{\pgfqpoint{2.736190in}{2.783714in}}{\pgfqpoint{2.748681in}{2.788888in}}{\pgfqpoint{2.757889in}{2.798096in}}%
\pgfpathcurveto{\pgfqpoint{2.767098in}{2.807305in}}{\pgfqpoint{2.772272in}{2.819796in}}{\pgfqpoint{2.772272in}{2.832818in}}%
\pgfpathcurveto{\pgfqpoint{2.772272in}{2.845841in}}{\pgfqpoint{2.767098in}{2.858332in}}{\pgfqpoint{2.757889in}{2.867541in}}%
\pgfpathcurveto{\pgfqpoint{2.748681in}{2.876749in}}{\pgfqpoint{2.736190in}{2.881923in}}{\pgfqpoint{2.723167in}{2.881923in}}%
\pgfpathcurveto{\pgfqpoint{2.710144in}{2.881923in}}{\pgfqpoint{2.697653in}{2.876749in}}{\pgfqpoint{2.688445in}{2.867541in}}%
\pgfpathcurveto{\pgfqpoint{2.679236in}{2.858332in}}{\pgfqpoint{2.674062in}{2.845841in}}{\pgfqpoint{2.674062in}{2.832818in}}%
\pgfpathcurveto{\pgfqpoint{2.674062in}{2.819796in}}{\pgfqpoint{2.679236in}{2.807305in}}{\pgfqpoint{2.688445in}{2.798096in}}%
\pgfpathcurveto{\pgfqpoint{2.697653in}{2.788888in}}{\pgfqpoint{2.710144in}{2.783714in}}{\pgfqpoint{2.723167in}{2.783714in}}%
\pgfpathlineto{\pgfqpoint{2.723167in}{2.783714in}}%
\pgfpathclose%
\pgfusepath{stroke,fill}%
\end{pgfscope}%
\begin{pgfscope}%
\pgfpathrectangle{\pgfqpoint{0.786164in}{0.768110in}}{\pgfqpoint{8.851069in}{7.081890in}}%
\pgfusepath{clip}%
\pgfsetbuttcap%
\pgfsetroundjoin%
\definecolor{currentfill}{rgb}{0.220057,0.343307,0.549413}%
\pgfsetfillcolor{currentfill}%
\pgfsetfillopacity{0.700000}%
\pgfsetlinewidth{0.501875pt}%
\definecolor{currentstroke}{rgb}{1.000000,1.000000,1.000000}%
\pgfsetstrokecolor{currentstroke}%
\pgfsetstrokeopacity{0.700000}%
\pgfsetdash{}{0pt}%
\pgfpathmoveto{\pgfqpoint{2.496304in}{2.698345in}}%
\pgfpathcurveto{\pgfqpoint{2.509327in}{2.698345in}}{\pgfqpoint{2.521818in}{2.703519in}}{\pgfqpoint{2.531026in}{2.712727in}}%
\pgfpathcurveto{\pgfqpoint{2.540235in}{2.721936in}}{\pgfqpoint{2.545409in}{2.734427in}}{\pgfqpoint{2.545409in}{2.747450in}}%
\pgfpathcurveto{\pgfqpoint{2.545409in}{2.760472in}}{\pgfqpoint{2.540235in}{2.772964in}}{\pgfqpoint{2.531026in}{2.782172in}}%
\pgfpathcurveto{\pgfqpoint{2.521818in}{2.791380in}}{\pgfqpoint{2.509327in}{2.796554in}}{\pgfqpoint{2.496304in}{2.796554in}}%
\pgfpathcurveto{\pgfqpoint{2.483281in}{2.796554in}}{\pgfqpoint{2.470790in}{2.791380in}}{\pgfqpoint{2.461582in}{2.782172in}}%
\pgfpathcurveto{\pgfqpoint{2.452374in}{2.772964in}}{\pgfqpoint{2.447200in}{2.760472in}}{\pgfqpoint{2.447200in}{2.747450in}}%
\pgfpathcurveto{\pgfqpoint{2.447200in}{2.734427in}}{\pgfqpoint{2.452374in}{2.721936in}}{\pgfqpoint{2.461582in}{2.712727in}}%
\pgfpathcurveto{\pgfqpoint{2.470790in}{2.703519in}}{\pgfqpoint{2.483281in}{2.698345in}}{\pgfqpoint{2.496304in}{2.698345in}}%
\pgfpathlineto{\pgfqpoint{2.496304in}{2.698345in}}%
\pgfpathclose%
\pgfusepath{stroke,fill}%
\end{pgfscope}%
\begin{pgfscope}%
\pgfpathrectangle{\pgfqpoint{0.786164in}{0.768110in}}{\pgfqpoint{8.851069in}{7.081890in}}%
\pgfusepath{clip}%
\pgfsetbuttcap%
\pgfsetroundjoin%
\definecolor{currentfill}{rgb}{0.221989,0.339161,0.548752}%
\pgfsetfillcolor{currentfill}%
\pgfsetfillopacity{0.700000}%
\pgfsetlinewidth{0.501875pt}%
\definecolor{currentstroke}{rgb}{1.000000,1.000000,1.000000}%
\pgfsetstrokecolor{currentstroke}%
\pgfsetstrokeopacity{0.700000}%
\pgfsetdash{}{0pt}%
\pgfpathmoveto{\pgfqpoint{2.792520in}{2.677003in}}%
\pgfpathcurveto{\pgfqpoint{2.805543in}{2.677003in}}{\pgfqpoint{2.818034in}{2.682177in}}{\pgfqpoint{2.827242in}{2.691385in}}%
\pgfpathcurveto{\pgfqpoint{2.836451in}{2.700594in}}{\pgfqpoint{2.841625in}{2.713085in}}{\pgfqpoint{2.841625in}{2.726108in}}%
\pgfpathcurveto{\pgfqpoint{2.841625in}{2.739130in}}{\pgfqpoint{2.836451in}{2.751621in}}{\pgfqpoint{2.827242in}{2.760830in}}%
\pgfpathcurveto{\pgfqpoint{2.818034in}{2.770038in}}{\pgfqpoint{2.805543in}{2.775212in}}{\pgfqpoint{2.792520in}{2.775212in}}%
\pgfpathcurveto{\pgfqpoint{2.779497in}{2.775212in}}{\pgfqpoint{2.767006in}{2.770038in}}{\pgfqpoint{2.757798in}{2.760830in}}%
\pgfpathcurveto{\pgfqpoint{2.748589in}{2.751621in}}{\pgfqpoint{2.743415in}{2.739130in}}{\pgfqpoint{2.743415in}{2.726108in}}%
\pgfpathcurveto{\pgfqpoint{2.743415in}{2.713085in}}{\pgfqpoint{2.748589in}{2.700594in}}{\pgfqpoint{2.757798in}{2.691385in}}%
\pgfpathcurveto{\pgfqpoint{2.767006in}{2.682177in}}{\pgfqpoint{2.779497in}{2.677003in}}{\pgfqpoint{2.792520in}{2.677003in}}%
\pgfpathlineto{\pgfqpoint{2.792520in}{2.677003in}}%
\pgfpathclose%
\pgfusepath{stroke,fill}%
\end{pgfscope}%
\begin{pgfscope}%
\pgfpathrectangle{\pgfqpoint{0.786164in}{0.768110in}}{\pgfqpoint{8.851069in}{7.081890in}}%
\pgfusepath{clip}%
\pgfsetbuttcap%
\pgfsetroundjoin%
\definecolor{currentfill}{rgb}{0.221989,0.339161,0.548752}%
\pgfsetfillcolor{currentfill}%
\pgfsetfillopacity{0.700000}%
\pgfsetlinewidth{0.501875pt}%
\definecolor{currentstroke}{rgb}{1.000000,1.000000,1.000000}%
\pgfsetstrokecolor{currentstroke}%
\pgfsetstrokeopacity{0.700000}%
\pgfsetdash{}{0pt}%
\pgfpathmoveto{\pgfqpoint{2.870787in}{2.655661in}}%
\pgfpathcurveto{\pgfqpoint{2.883809in}{2.655661in}}{\pgfqpoint{2.896300in}{2.660835in}}{\pgfqpoint{2.905509in}{2.670043in}}%
\pgfpathcurveto{\pgfqpoint{2.914717in}{2.679252in}}{\pgfqpoint{2.919891in}{2.691743in}}{\pgfqpoint{2.919891in}{2.704765in}}%
\pgfpathcurveto{\pgfqpoint{2.919891in}{2.717788in}}{\pgfqpoint{2.914717in}{2.730279in}}{\pgfqpoint{2.905509in}{2.739488in}}%
\pgfpathcurveto{\pgfqpoint{2.896300in}{2.748696in}}{\pgfqpoint{2.883809in}{2.753870in}}{\pgfqpoint{2.870787in}{2.753870in}}%
\pgfpathcurveto{\pgfqpoint{2.857764in}{2.753870in}}{\pgfqpoint{2.845273in}{2.748696in}}{\pgfqpoint{2.836064in}{2.739488in}}%
\pgfpathcurveto{\pgfqpoint{2.826856in}{2.730279in}}{\pgfqpoint{2.821682in}{2.717788in}}{\pgfqpoint{2.821682in}{2.704765in}}%
\pgfpathcurveto{\pgfqpoint{2.821682in}{2.691743in}}{\pgfqpoint{2.826856in}{2.679252in}}{\pgfqpoint{2.836064in}{2.670043in}}%
\pgfpathcurveto{\pgfqpoint{2.845273in}{2.660835in}}{\pgfqpoint{2.857764in}{2.655661in}}{\pgfqpoint{2.870787in}{2.655661in}}%
\pgfpathlineto{\pgfqpoint{2.870787in}{2.655661in}}%
\pgfpathclose%
\pgfusepath{stroke,fill}%
\end{pgfscope}%
\begin{pgfscope}%
\pgfpathrectangle{\pgfqpoint{0.786164in}{0.768110in}}{\pgfqpoint{8.851069in}{7.081890in}}%
\pgfusepath{clip}%
\pgfsetbuttcap%
\pgfsetroundjoin%
\definecolor{currentfill}{rgb}{0.231674,0.318106,0.544834}%
\pgfsetfillcolor{currentfill}%
\pgfsetfillopacity{0.700000}%
\pgfsetlinewidth{0.501875pt}%
\definecolor{currentstroke}{rgb}{1.000000,1.000000,1.000000}%
\pgfsetstrokecolor{currentstroke}%
\pgfsetstrokeopacity{0.700000}%
\pgfsetdash{}{0pt}%
\pgfpathmoveto{\pgfqpoint{2.929639in}{2.548950in}}%
\pgfpathcurveto{\pgfqpoint{2.942662in}{2.548950in}}{\pgfqpoint{2.955153in}{2.554124in}}{\pgfqpoint{2.964361in}{2.563332in}}%
\pgfpathcurveto{\pgfqpoint{2.973570in}{2.572541in}}{\pgfqpoint{2.978744in}{2.585032in}}{\pgfqpoint{2.978744in}{2.598055in}}%
\pgfpathcurveto{\pgfqpoint{2.978744in}{2.611077in}}{\pgfqpoint{2.973570in}{2.623568in}}{\pgfqpoint{2.964361in}{2.632777in}}%
\pgfpathcurveto{\pgfqpoint{2.955153in}{2.641985in}}{\pgfqpoint{2.942662in}{2.647159in}}{\pgfqpoint{2.929639in}{2.647159in}}%
\pgfpathcurveto{\pgfqpoint{2.916616in}{2.647159in}}{\pgfqpoint{2.904125in}{2.641985in}}{\pgfqpoint{2.894917in}{2.632777in}}%
\pgfpathcurveto{\pgfqpoint{2.885708in}{2.623568in}}{\pgfqpoint{2.880534in}{2.611077in}}{\pgfqpoint{2.880534in}{2.598055in}}%
\pgfpathcurveto{\pgfqpoint{2.880534in}{2.585032in}}{\pgfqpoint{2.885708in}{2.572541in}}{\pgfqpoint{2.894917in}{2.563332in}}%
\pgfpathcurveto{\pgfqpoint{2.904125in}{2.554124in}}{\pgfqpoint{2.916616in}{2.548950in}}{\pgfqpoint{2.929639in}{2.548950in}}%
\pgfpathlineto{\pgfqpoint{2.929639in}{2.548950in}}%
\pgfpathclose%
\pgfusepath{stroke,fill}%
\end{pgfscope}%
\begin{pgfscope}%
\pgfpathrectangle{\pgfqpoint{0.786164in}{0.768110in}}{\pgfqpoint{8.851069in}{7.081890in}}%
\pgfusepath{clip}%
\pgfsetbuttcap%
\pgfsetroundjoin%
\definecolor{currentfill}{rgb}{0.229739,0.322361,0.545706}%
\pgfsetfillcolor{currentfill}%
\pgfsetfillopacity{0.700000}%
\pgfsetlinewidth{0.501875pt}%
\definecolor{currentstroke}{rgb}{1.000000,1.000000,1.000000}%
\pgfsetstrokecolor{currentstroke}%
\pgfsetstrokeopacity{0.700000}%
\pgfsetdash{}{0pt}%
\pgfpathmoveto{\pgfqpoint{2.983485in}{2.570292in}}%
\pgfpathcurveto{\pgfqpoint{2.996508in}{2.570292in}}{\pgfqpoint{3.008999in}{2.575466in}}{\pgfqpoint{3.018208in}{2.584674in}}%
\pgfpathcurveto{\pgfqpoint{3.027416in}{2.593883in}}{\pgfqpoint{3.032590in}{2.606374in}}{\pgfqpoint{3.032590in}{2.619397in}}%
\pgfpathcurveto{\pgfqpoint{3.032590in}{2.632419in}}{\pgfqpoint{3.027416in}{2.644910in}}{\pgfqpoint{3.018208in}{2.654119in}}%
\pgfpathcurveto{\pgfqpoint{3.008999in}{2.663327in}}{\pgfqpoint{2.996508in}{2.668501in}}{\pgfqpoint{2.983485in}{2.668501in}}%
\pgfpathcurveto{\pgfqpoint{2.970463in}{2.668501in}}{\pgfqpoint{2.957972in}{2.663327in}}{\pgfqpoint{2.948763in}{2.654119in}}%
\pgfpathcurveto{\pgfqpoint{2.939555in}{2.644910in}}{\pgfqpoint{2.934381in}{2.632419in}}{\pgfqpoint{2.934381in}{2.619397in}}%
\pgfpathcurveto{\pgfqpoint{2.934381in}{2.606374in}}{\pgfqpoint{2.939555in}{2.593883in}}{\pgfqpoint{2.948763in}{2.584674in}}%
\pgfpathcurveto{\pgfqpoint{2.957972in}{2.575466in}}{\pgfqpoint{2.970463in}{2.570292in}}{\pgfqpoint{2.983485in}{2.570292in}}%
\pgfpathlineto{\pgfqpoint{2.983485in}{2.570292in}}%
\pgfpathclose%
\pgfusepath{stroke,fill}%
\end{pgfscope}%
\begin{pgfscope}%
\pgfpathrectangle{\pgfqpoint{0.786164in}{0.768110in}}{\pgfqpoint{8.851069in}{7.081890in}}%
\pgfusepath{clip}%
\pgfsetbuttcap%
\pgfsetroundjoin%
\definecolor{currentfill}{rgb}{0.233603,0.313828,0.543914}%
\pgfsetfillcolor{currentfill}%
\pgfsetfillopacity{0.700000}%
\pgfsetlinewidth{0.501875pt}%
\definecolor{currentstroke}{rgb}{1.000000,1.000000,1.000000}%
\pgfsetstrokecolor{currentstroke}%
\pgfsetstrokeopacity{0.700000}%
\pgfsetdash{}{0pt}%
\pgfpathmoveto{\pgfqpoint{3.062607in}{2.591634in}}%
\pgfpathcurveto{\pgfqpoint{3.075629in}{2.591634in}}{\pgfqpoint{3.088120in}{2.596808in}}{\pgfqpoint{3.097329in}{2.606017in}}%
\pgfpathcurveto{\pgfqpoint{3.106537in}{2.615225in}}{\pgfqpoint{3.111711in}{2.627716in}}{\pgfqpoint{3.111711in}{2.640739in}}%
\pgfpathcurveto{\pgfqpoint{3.111711in}{2.653762in}}{\pgfqpoint{3.106537in}{2.666253in}}{\pgfqpoint{3.097329in}{2.675461in}}%
\pgfpathcurveto{\pgfqpoint{3.088120in}{2.684670in}}{\pgfqpoint{3.075629in}{2.689844in}}{\pgfqpoint{3.062607in}{2.689844in}}%
\pgfpathcurveto{\pgfqpoint{3.049584in}{2.689844in}}{\pgfqpoint{3.037093in}{2.684670in}}{\pgfqpoint{3.027884in}{2.675461in}}%
\pgfpathcurveto{\pgfqpoint{3.018676in}{2.666253in}}{\pgfqpoint{3.013502in}{2.653762in}}{\pgfqpoint{3.013502in}{2.640739in}}%
\pgfpathcurveto{\pgfqpoint{3.013502in}{2.627716in}}{\pgfqpoint{3.018676in}{2.615225in}}{\pgfqpoint{3.027884in}{2.606017in}}%
\pgfpathcurveto{\pgfqpoint{3.037093in}{2.596808in}}{\pgfqpoint{3.049584in}{2.591634in}}{\pgfqpoint{3.062607in}{2.591634in}}%
\pgfpathlineto{\pgfqpoint{3.062607in}{2.591634in}}%
\pgfpathclose%
\pgfusepath{stroke,fill}%
\end{pgfscope}%
\begin{pgfscope}%
\pgfpathrectangle{\pgfqpoint{0.786164in}{0.768110in}}{\pgfqpoint{8.851069in}{7.081890in}}%
\pgfusepath{clip}%
\pgfsetbuttcap%
\pgfsetroundjoin%
\definecolor{currentfill}{rgb}{0.233603,0.313828,0.543914}%
\pgfsetfillcolor{currentfill}%
\pgfsetfillopacity{0.700000}%
\pgfsetlinewidth{0.501875pt}%
\definecolor{currentstroke}{rgb}{1.000000,1.000000,1.000000}%
\pgfsetstrokecolor{currentstroke}%
\pgfsetstrokeopacity{0.700000}%
\pgfsetdash{}{0pt}%
\pgfpathmoveto{\pgfqpoint{3.110958in}{2.634319in}}%
\pgfpathcurveto{\pgfqpoint{3.123981in}{2.634319in}}{\pgfqpoint{3.136472in}{2.639493in}}{\pgfqpoint{3.145681in}{2.648701in}}%
\pgfpathcurveto{\pgfqpoint{3.154889in}{2.657909in}}{\pgfqpoint{3.160063in}{2.670401in}}{\pgfqpoint{3.160063in}{2.683423in}}%
\pgfpathcurveto{\pgfqpoint{3.160063in}{2.696446in}}{\pgfqpoint{3.154889in}{2.708937in}}{\pgfqpoint{3.145681in}{2.718145in}}%
\pgfpathcurveto{\pgfqpoint{3.136472in}{2.727354in}}{\pgfqpoint{3.123981in}{2.732528in}}{\pgfqpoint{3.110958in}{2.732528in}}%
\pgfpathcurveto{\pgfqpoint{3.097936in}{2.732528in}}{\pgfqpoint{3.085445in}{2.727354in}}{\pgfqpoint{3.076236in}{2.718145in}}%
\pgfpathcurveto{\pgfqpoint{3.067028in}{2.708937in}}{\pgfqpoint{3.061854in}{2.696446in}}{\pgfqpoint{3.061854in}{2.683423in}}%
\pgfpathcurveto{\pgfqpoint{3.061854in}{2.670401in}}{\pgfqpoint{3.067028in}{2.657909in}}{\pgfqpoint{3.076236in}{2.648701in}}%
\pgfpathcurveto{\pgfqpoint{3.085445in}{2.639493in}}{\pgfqpoint{3.097936in}{2.634319in}}{\pgfqpoint{3.110958in}{2.634319in}}%
\pgfpathlineto{\pgfqpoint{3.110958in}{2.634319in}}%
\pgfpathclose%
\pgfusepath{stroke,fill}%
\end{pgfscope}%
\begin{pgfscope}%
\pgfpathrectangle{\pgfqpoint{0.786164in}{0.768110in}}{\pgfqpoint{8.851069in}{7.081890in}}%
\pgfusepath{clip}%
\pgfsetbuttcap%
\pgfsetroundjoin%
\definecolor{currentfill}{rgb}{0.235526,0.309527,0.542944}%
\pgfsetfillcolor{currentfill}%
\pgfsetfillopacity{0.700000}%
\pgfsetlinewidth{0.501875pt}%
\definecolor{currentstroke}{rgb}{1.000000,1.000000,1.000000}%
\pgfsetstrokecolor{currentstroke}%
\pgfsetstrokeopacity{0.700000}%
\pgfsetdash{}{0pt}%
\pgfpathmoveto{\pgfqpoint{3.176526in}{2.548950in}}%
\pgfpathcurveto{\pgfqpoint{3.189549in}{2.548950in}}{\pgfqpoint{3.202040in}{2.554124in}}{\pgfqpoint{3.211249in}{2.563332in}}%
\pgfpathcurveto{\pgfqpoint{3.220457in}{2.572541in}}{\pgfqpoint{3.225631in}{2.585032in}}{\pgfqpoint{3.225631in}{2.598055in}}%
\pgfpathcurveto{\pgfqpoint{3.225631in}{2.611077in}}{\pgfqpoint{3.220457in}{2.623568in}}{\pgfqpoint{3.211249in}{2.632777in}}%
\pgfpathcurveto{\pgfqpoint{3.202040in}{2.641985in}}{\pgfqpoint{3.189549in}{2.647159in}}{\pgfqpoint{3.176526in}{2.647159in}}%
\pgfpathcurveto{\pgfqpoint{3.163504in}{2.647159in}}{\pgfqpoint{3.151013in}{2.641985in}}{\pgfqpoint{3.141804in}{2.632777in}}%
\pgfpathcurveto{\pgfqpoint{3.132596in}{2.623568in}}{\pgfqpoint{3.127422in}{2.611077in}}{\pgfqpoint{3.127422in}{2.598055in}}%
\pgfpathcurveto{\pgfqpoint{3.127422in}{2.585032in}}{\pgfqpoint{3.132596in}{2.572541in}}{\pgfqpoint{3.141804in}{2.563332in}}%
\pgfpathcurveto{\pgfqpoint{3.151013in}{2.554124in}}{\pgfqpoint{3.163504in}{2.548950in}}{\pgfqpoint{3.176526in}{2.548950in}}%
\pgfpathlineto{\pgfqpoint{3.176526in}{2.548950in}}%
\pgfpathclose%
\pgfusepath{stroke,fill}%
\end{pgfscope}%
\begin{pgfscope}%
\pgfpathrectangle{\pgfqpoint{0.786164in}{0.768110in}}{\pgfqpoint{8.851069in}{7.081890in}}%
\pgfusepath{clip}%
\pgfsetbuttcap%
\pgfsetroundjoin%
\definecolor{currentfill}{rgb}{0.239346,0.300855,0.540844}%
\pgfsetfillcolor{currentfill}%
\pgfsetfillopacity{0.700000}%
\pgfsetlinewidth{0.501875pt}%
\definecolor{currentstroke}{rgb}{1.000000,1.000000,1.000000}%
\pgfsetstrokecolor{currentstroke}%
\pgfsetstrokeopacity{0.700000}%
\pgfsetdash{}{0pt}%
\pgfpathmoveto{\pgfqpoint{3.272986in}{2.548950in}}%
\pgfpathcurveto{\pgfqpoint{3.286008in}{2.548950in}}{\pgfqpoint{3.298500in}{2.554124in}}{\pgfqpoint{3.307708in}{2.563332in}}%
\pgfpathcurveto{\pgfqpoint{3.316916in}{2.572541in}}{\pgfqpoint{3.322090in}{2.585032in}}{\pgfqpoint{3.322090in}{2.598055in}}%
\pgfpathcurveto{\pgfqpoint{3.322090in}{2.611077in}}{\pgfqpoint{3.316916in}{2.623568in}}{\pgfqpoint{3.307708in}{2.632777in}}%
\pgfpathcurveto{\pgfqpoint{3.298500in}{2.641985in}}{\pgfqpoint{3.286008in}{2.647159in}}{\pgfqpoint{3.272986in}{2.647159in}}%
\pgfpathcurveto{\pgfqpoint{3.259963in}{2.647159in}}{\pgfqpoint{3.247472in}{2.641985in}}{\pgfqpoint{3.238264in}{2.632777in}}%
\pgfpathcurveto{\pgfqpoint{3.229055in}{2.623568in}}{\pgfqpoint{3.223881in}{2.611077in}}{\pgfqpoint{3.223881in}{2.598055in}}%
\pgfpathcurveto{\pgfqpoint{3.223881in}{2.585032in}}{\pgfqpoint{3.229055in}{2.572541in}}{\pgfqpoint{3.238264in}{2.563332in}}%
\pgfpathcurveto{\pgfqpoint{3.247472in}{2.554124in}}{\pgfqpoint{3.259963in}{2.548950in}}{\pgfqpoint{3.272986in}{2.548950in}}%
\pgfpathlineto{\pgfqpoint{3.272986in}{2.548950in}}%
\pgfpathclose%
\pgfusepath{stroke,fill}%
\end{pgfscope}%
\begin{pgfscope}%
\pgfpathrectangle{\pgfqpoint{0.786164in}{0.768110in}}{\pgfqpoint{8.851069in}{7.081890in}}%
\pgfusepath{clip}%
\pgfsetbuttcap%
\pgfsetroundjoin%
\definecolor{currentfill}{rgb}{0.260571,0.246922,0.522828}%
\pgfsetfillcolor{currentfill}%
\pgfsetfillopacity{0.700000}%
\pgfsetlinewidth{0.501875pt}%
\definecolor{currentstroke}{rgb}{1.000000,1.000000,1.000000}%
\pgfsetstrokecolor{currentstroke}%
\pgfsetstrokeopacity{0.700000}%
\pgfsetdash{}{0pt}%
\pgfpathmoveto{\pgfqpoint{3.509250in}{2.378213in}}%
\pgfpathcurveto{\pgfqpoint{3.522273in}{2.378213in}}{\pgfqpoint{3.534764in}{2.383386in}}{\pgfqpoint{3.543973in}{2.392595in}}%
\pgfpathcurveto{\pgfqpoint{3.553181in}{2.401803in}}{\pgfqpoint{3.558355in}{2.414294in}}{\pgfqpoint{3.558355in}{2.427317in}}%
\pgfpathcurveto{\pgfqpoint{3.558355in}{2.440340in}}{\pgfqpoint{3.553181in}{2.452831in}}{\pgfqpoint{3.543973in}{2.462039in}}%
\pgfpathcurveto{\pgfqpoint{3.534764in}{2.471248in}}{\pgfqpoint{3.522273in}{2.476422in}}{\pgfqpoint{3.509250in}{2.476422in}}%
\pgfpathcurveto{\pgfqpoint{3.496228in}{2.476422in}}{\pgfqpoint{3.483737in}{2.471248in}}{\pgfqpoint{3.474528in}{2.462039in}}%
\pgfpathcurveto{\pgfqpoint{3.465320in}{2.452831in}}{\pgfqpoint{3.460146in}{2.440340in}}{\pgfqpoint{3.460146in}{2.427317in}}%
\pgfpathcurveto{\pgfqpoint{3.460146in}{2.414294in}}{\pgfqpoint{3.465320in}{2.401803in}}{\pgfqpoint{3.474528in}{2.392595in}}%
\pgfpathcurveto{\pgfqpoint{3.483737in}{2.383386in}}{\pgfqpoint{3.496228in}{2.378213in}}{\pgfqpoint{3.509250in}{2.378213in}}%
\pgfpathlineto{\pgfqpoint{3.509250in}{2.378213in}}%
\pgfpathclose%
\pgfusepath{stroke,fill}%
\end{pgfscope}%
\begin{pgfscope}%
\pgfpathrectangle{\pgfqpoint{0.786164in}{0.768110in}}{\pgfqpoint{8.851069in}{7.081890in}}%
\pgfusepath{clip}%
\pgfsetbuttcap%
\pgfsetroundjoin%
\definecolor{currentfill}{rgb}{0.250425,0.274290,0.533103}%
\pgfsetfillcolor{currentfill}%
\pgfsetfillopacity{0.700000}%
\pgfsetlinewidth{0.501875pt}%
\definecolor{currentstroke}{rgb}{1.000000,1.000000,1.000000}%
\pgfsetstrokecolor{currentstroke}%
\pgfsetstrokeopacity{0.700000}%
\pgfsetdash{}{0pt}%
\pgfpathmoveto{\pgfqpoint{3.520850in}{2.442239in}}%
\pgfpathcurveto{\pgfqpoint{3.533873in}{2.442239in}}{\pgfqpoint{3.546364in}{2.447413in}}{\pgfqpoint{3.555572in}{2.456621in}}%
\pgfpathcurveto{\pgfqpoint{3.564781in}{2.465830in}}{\pgfqpoint{3.569955in}{2.478321in}}{\pgfqpoint{3.569955in}{2.491344in}}%
\pgfpathcurveto{\pgfqpoint{3.569955in}{2.504366in}}{\pgfqpoint{3.564781in}{2.516857in}}{\pgfqpoint{3.555572in}{2.526066in}}%
\pgfpathcurveto{\pgfqpoint{3.546364in}{2.535274in}}{\pgfqpoint{3.533873in}{2.540448in}}{\pgfqpoint{3.520850in}{2.540448in}}%
\pgfpathcurveto{\pgfqpoint{3.507827in}{2.540448in}}{\pgfqpoint{3.495336in}{2.535274in}}{\pgfqpoint{3.486128in}{2.526066in}}%
\pgfpathcurveto{\pgfqpoint{3.476919in}{2.516857in}}{\pgfqpoint{3.471745in}{2.504366in}}{\pgfqpoint{3.471745in}{2.491344in}}%
\pgfpathcurveto{\pgfqpoint{3.471745in}{2.478321in}}{\pgfqpoint{3.476919in}{2.465830in}}{\pgfqpoint{3.486128in}{2.456621in}}%
\pgfpathcurveto{\pgfqpoint{3.495336in}{2.447413in}}{\pgfqpoint{3.507827in}{2.442239in}}{\pgfqpoint{3.520850in}{2.442239in}}%
\pgfpathlineto{\pgfqpoint{3.520850in}{2.442239in}}%
\pgfpathclose%
\pgfusepath{stroke,fill}%
\end{pgfscope}%
\begin{pgfscope}%
\pgfpathrectangle{\pgfqpoint{0.786164in}{0.768110in}}{\pgfqpoint{8.851069in}{7.081890in}}%
\pgfusepath{clip}%
\pgfsetbuttcap%
\pgfsetroundjoin%
\definecolor{currentfill}{rgb}{0.263663,0.237631,0.518762}%
\pgfsetfillcolor{currentfill}%
\pgfsetfillopacity{0.700000}%
\pgfsetlinewidth{0.501875pt}%
\definecolor{currentstroke}{rgb}{1.000000,1.000000,1.000000}%
\pgfsetstrokecolor{currentstroke}%
\pgfsetstrokeopacity{0.700000}%
\pgfsetdash{}{0pt}%
\pgfpathmoveto{\pgfqpoint{3.649666in}{2.399555in}}%
\pgfpathcurveto{\pgfqpoint{3.662689in}{2.399555in}}{\pgfqpoint{3.675180in}{2.404729in}}{\pgfqpoint{3.684388in}{2.413937in}}%
\pgfpathcurveto{\pgfqpoint{3.693597in}{2.423146in}}{\pgfqpoint{3.698771in}{2.435637in}}{\pgfqpoint{3.698771in}{2.448659in}}%
\pgfpathcurveto{\pgfqpoint{3.698771in}{2.461682in}}{\pgfqpoint{3.693597in}{2.474173in}}{\pgfqpoint{3.684388in}{2.483382in}}%
\pgfpathcurveto{\pgfqpoint{3.675180in}{2.492590in}}{\pgfqpoint{3.662689in}{2.497764in}}{\pgfqpoint{3.649666in}{2.497764in}}%
\pgfpathcurveto{\pgfqpoint{3.636643in}{2.497764in}}{\pgfqpoint{3.624152in}{2.492590in}}{\pgfqpoint{3.614944in}{2.483382in}}%
\pgfpathcurveto{\pgfqpoint{3.605735in}{2.474173in}}{\pgfqpoint{3.600561in}{2.461682in}}{\pgfqpoint{3.600561in}{2.448659in}}%
\pgfpathcurveto{\pgfqpoint{3.600561in}{2.435637in}}{\pgfqpoint{3.605735in}{2.423146in}}{\pgfqpoint{3.614944in}{2.413937in}}%
\pgfpathcurveto{\pgfqpoint{3.624152in}{2.404729in}}{\pgfqpoint{3.636643in}{2.399555in}}{\pgfqpoint{3.649666in}{2.399555in}}%
\pgfpathlineto{\pgfqpoint{3.649666in}{2.399555in}}%
\pgfpathclose%
\pgfusepath{stroke,fill}%
\end{pgfscope}%
\begin{pgfscope}%
\pgfpathrectangle{\pgfqpoint{0.786164in}{0.768110in}}{\pgfqpoint{8.851069in}{7.081890in}}%
\pgfusepath{clip}%
\pgfsetbuttcap%
\pgfsetroundjoin%
\definecolor{currentfill}{rgb}{0.282910,0.105393,0.426902}%
\pgfsetfillcolor{currentfill}%
\pgfsetfillopacity{0.700000}%
\pgfsetlinewidth{0.501875pt}%
\definecolor{currentstroke}{rgb}{1.000000,1.000000,1.000000}%
\pgfsetstrokecolor{currentstroke}%
\pgfsetstrokeopacity{0.700000}%
\pgfsetdash{}{0pt}%
\pgfpathmoveto{\pgfqpoint{4.033672in}{2.314186in}}%
\pgfpathcurveto{\pgfqpoint{4.046695in}{2.314186in}}{\pgfqpoint{4.059186in}{2.319360in}}{\pgfqpoint{4.068394in}{2.328568in}}%
\pgfpathcurveto{\pgfqpoint{4.077603in}{2.337777in}}{\pgfqpoint{4.082777in}{2.350268in}}{\pgfqpoint{4.082777in}{2.363291in}}%
\pgfpathcurveto{\pgfqpoint{4.082777in}{2.376313in}}{\pgfqpoint{4.077603in}{2.388804in}}{\pgfqpoint{4.068394in}{2.398013in}}%
\pgfpathcurveto{\pgfqpoint{4.059186in}{2.407221in}}{\pgfqpoint{4.046695in}{2.412395in}}{\pgfqpoint{4.033672in}{2.412395in}}%
\pgfpathcurveto{\pgfqpoint{4.020650in}{2.412395in}}{\pgfqpoint{4.008158in}{2.407221in}}{\pgfqpoint{3.998950in}{2.398013in}}%
\pgfpathcurveto{\pgfqpoint{3.989742in}{2.388804in}}{\pgfqpoint{3.984568in}{2.376313in}}{\pgfqpoint{3.984568in}{2.363291in}}%
\pgfpathcurveto{\pgfqpoint{3.984568in}{2.350268in}}{\pgfqpoint{3.989742in}{2.337777in}}{\pgfqpoint{3.998950in}{2.328568in}}%
\pgfpathcurveto{\pgfqpoint{4.008158in}{2.319360in}}{\pgfqpoint{4.020650in}{2.314186in}}{\pgfqpoint{4.033672in}{2.314186in}}%
\pgfpathlineto{\pgfqpoint{4.033672in}{2.314186in}}%
\pgfpathclose%
\pgfusepath{stroke,fill}%
\end{pgfscope}%
\begin{pgfscope}%
\pgfpathrectangle{\pgfqpoint{0.786164in}{0.768110in}}{\pgfqpoint{8.851069in}{7.081890in}}%
\pgfusepath{clip}%
\pgfsetbuttcap%
\pgfsetroundjoin%
\definecolor{currentfill}{rgb}{0.283091,0.110553,0.431554}%
\pgfsetfillcolor{currentfill}%
\pgfsetfillopacity{0.700000}%
\pgfsetlinewidth{0.501875pt}%
\definecolor{currentstroke}{rgb}{1.000000,1.000000,1.000000}%
\pgfsetstrokecolor{currentstroke}%
\pgfsetstrokeopacity{0.700000}%
\pgfsetdash{}{0pt}%
\pgfpathmoveto{\pgfqpoint{4.068715in}{2.335528in}}%
\pgfpathcurveto{\pgfqpoint{4.081738in}{2.335528in}}{\pgfqpoint{4.094229in}{2.340702in}}{\pgfqpoint{4.103437in}{2.349911in}}%
\pgfpathcurveto{\pgfqpoint{4.112646in}{2.359119in}}{\pgfqpoint{4.117820in}{2.371610in}}{\pgfqpoint{4.117820in}{2.384633in}}%
\pgfpathcurveto{\pgfqpoint{4.117820in}{2.397656in}}{\pgfqpoint{4.112646in}{2.410147in}}{\pgfqpoint{4.103437in}{2.419355in}}%
\pgfpathcurveto{\pgfqpoint{4.094229in}{2.428563in}}{\pgfqpoint{4.081738in}{2.433737in}}{\pgfqpoint{4.068715in}{2.433737in}}%
\pgfpathcurveto{\pgfqpoint{4.055692in}{2.433737in}}{\pgfqpoint{4.043201in}{2.428563in}}{\pgfqpoint{4.033993in}{2.419355in}}%
\pgfpathcurveto{\pgfqpoint{4.024784in}{2.410147in}}{\pgfqpoint{4.019610in}{2.397656in}}{\pgfqpoint{4.019610in}{2.384633in}}%
\pgfpathcurveto{\pgfqpoint{4.019610in}{2.371610in}}{\pgfqpoint{4.024784in}{2.359119in}}{\pgfqpoint{4.033993in}{2.349911in}}%
\pgfpathcurveto{\pgfqpoint{4.043201in}{2.340702in}}{\pgfqpoint{4.055692in}{2.335528in}}{\pgfqpoint{4.068715in}{2.335528in}}%
\pgfpathlineto{\pgfqpoint{4.068715in}{2.335528in}}%
\pgfpathclose%
\pgfusepath{stroke,fill}%
\end{pgfscope}%
\begin{pgfscope}%
\pgfpathrectangle{\pgfqpoint{0.786164in}{0.768110in}}{\pgfqpoint{8.851069in}{7.081890in}}%
\pgfusepath{clip}%
\pgfsetbuttcap%
\pgfsetroundjoin%
\definecolor{currentfill}{rgb}{0.283091,0.110553,0.431554}%
\pgfsetfillcolor{currentfill}%
\pgfsetfillopacity{0.700000}%
\pgfsetlinewidth{0.501875pt}%
\definecolor{currentstroke}{rgb}{1.000000,1.000000,1.000000}%
\pgfsetstrokecolor{currentstroke}%
\pgfsetstrokeopacity{0.700000}%
\pgfsetdash{}{0pt}%
\pgfpathmoveto{\pgfqpoint{3.943806in}{2.356870in}}%
\pgfpathcurveto{\pgfqpoint{3.956829in}{2.356870in}}{\pgfqpoint{3.969320in}{2.362044in}}{\pgfqpoint{3.978528in}{2.371253in}}%
\pgfpathcurveto{\pgfqpoint{3.987737in}{2.380461in}}{\pgfqpoint{3.992911in}{2.392952in}}{\pgfqpoint{3.992911in}{2.405975in}}%
\pgfpathcurveto{\pgfqpoint{3.992911in}{2.418998in}}{\pgfqpoint{3.987737in}{2.431489in}}{\pgfqpoint{3.978528in}{2.440697in}}%
\pgfpathcurveto{\pgfqpoint{3.969320in}{2.449906in}}{\pgfqpoint{3.956829in}{2.455080in}}{\pgfqpoint{3.943806in}{2.455080in}}%
\pgfpathcurveto{\pgfqpoint{3.930784in}{2.455080in}}{\pgfqpoint{3.918292in}{2.449906in}}{\pgfqpoint{3.909084in}{2.440697in}}%
\pgfpathcurveto{\pgfqpoint{3.899876in}{2.431489in}}{\pgfqpoint{3.894702in}{2.418998in}}{\pgfqpoint{3.894702in}{2.405975in}}%
\pgfpathcurveto{\pgfqpoint{3.894702in}{2.392952in}}{\pgfqpoint{3.899876in}{2.380461in}}{\pgfqpoint{3.909084in}{2.371253in}}%
\pgfpathcurveto{\pgfqpoint{3.918292in}{2.362044in}}{\pgfqpoint{3.930784in}{2.356870in}}{\pgfqpoint{3.943806in}{2.356870in}}%
\pgfpathlineto{\pgfqpoint{3.943806in}{2.356870in}}%
\pgfpathclose%
\pgfusepath{stroke,fill}%
\end{pgfscope}%
\begin{pgfscope}%
\pgfpathrectangle{\pgfqpoint{0.786164in}{0.768110in}}{\pgfqpoint{8.851069in}{7.081890in}}%
\pgfusepath{clip}%
\pgfsetbuttcap%
\pgfsetroundjoin%
\definecolor{currentfill}{rgb}{0.283229,0.120777,0.440584}%
\pgfsetfillcolor{currentfill}%
\pgfsetfillopacity{0.700000}%
\pgfsetlinewidth{0.501875pt}%
\definecolor{currentstroke}{rgb}{1.000000,1.000000,1.000000}%
\pgfsetstrokecolor{currentstroke}%
\pgfsetstrokeopacity{0.700000}%
\pgfsetdash{}{0pt}%
\pgfpathmoveto{\pgfqpoint{3.881901in}{2.484923in}}%
\pgfpathcurveto{\pgfqpoint{3.894924in}{2.484923in}}{\pgfqpoint{3.907415in}{2.490097in}}{\pgfqpoint{3.916623in}{2.499306in}}%
\pgfpathcurveto{\pgfqpoint{3.925832in}{2.508514in}}{\pgfqpoint{3.931006in}{2.521005in}}{\pgfqpoint{3.931006in}{2.534028in}}%
\pgfpathcurveto{\pgfqpoint{3.931006in}{2.547051in}}{\pgfqpoint{3.925832in}{2.559542in}}{\pgfqpoint{3.916623in}{2.568750in}}%
\pgfpathcurveto{\pgfqpoint{3.907415in}{2.577959in}}{\pgfqpoint{3.894924in}{2.583133in}}{\pgfqpoint{3.881901in}{2.583133in}}%
\pgfpathcurveto{\pgfqpoint{3.868879in}{2.583133in}}{\pgfqpoint{3.856387in}{2.577959in}}{\pgfqpoint{3.847179in}{2.568750in}}%
\pgfpathcurveto{\pgfqpoint{3.837971in}{2.559542in}}{\pgfqpoint{3.832797in}{2.547051in}}{\pgfqpoint{3.832797in}{2.534028in}}%
\pgfpathcurveto{\pgfqpoint{3.832797in}{2.521005in}}{\pgfqpoint{3.837971in}{2.508514in}}{\pgfqpoint{3.847179in}{2.499306in}}%
\pgfpathcurveto{\pgfqpoint{3.856387in}{2.490097in}}{\pgfqpoint{3.868879in}{2.484923in}}{\pgfqpoint{3.881901in}{2.484923in}}%
\pgfpathlineto{\pgfqpoint{3.881901in}{2.484923in}}%
\pgfpathclose%
\pgfusepath{stroke,fill}%
\end{pgfscope}%
\begin{pgfscope}%
\pgfpathrectangle{\pgfqpoint{0.786164in}{0.768110in}}{\pgfqpoint{8.851069in}{7.081890in}}%
\pgfusepath{clip}%
\pgfsetbuttcap%
\pgfsetroundjoin%
\definecolor{currentfill}{rgb}{0.283197,0.115680,0.436115}%
\pgfsetfillcolor{currentfill}%
\pgfsetfillopacity{0.700000}%
\pgfsetlinewidth{0.501875pt}%
\definecolor{currentstroke}{rgb}{1.000000,1.000000,1.000000}%
\pgfsetstrokecolor{currentstroke}%
\pgfsetstrokeopacity{0.700000}%
\pgfsetdash{}{0pt}%
\pgfpathmoveto{\pgfqpoint{3.860534in}{2.442239in}}%
\pgfpathcurveto{\pgfqpoint{3.873556in}{2.442239in}}{\pgfqpoint{3.886047in}{2.447413in}}{\pgfqpoint{3.895256in}{2.456621in}}%
\pgfpathcurveto{\pgfqpoint{3.904464in}{2.465830in}}{\pgfqpoint{3.909638in}{2.478321in}}{\pgfqpoint{3.909638in}{2.491344in}}%
\pgfpathcurveto{\pgfqpoint{3.909638in}{2.504366in}}{\pgfqpoint{3.904464in}{2.516857in}}{\pgfqpoint{3.895256in}{2.526066in}}%
\pgfpathcurveto{\pgfqpoint{3.886047in}{2.535274in}}{\pgfqpoint{3.873556in}{2.540448in}}{\pgfqpoint{3.860534in}{2.540448in}}%
\pgfpathcurveto{\pgfqpoint{3.847511in}{2.540448in}}{\pgfqpoint{3.835020in}{2.535274in}}{\pgfqpoint{3.825811in}{2.526066in}}%
\pgfpathcurveto{\pgfqpoint{3.816603in}{2.516857in}}{\pgfqpoint{3.811429in}{2.504366in}}{\pgfqpoint{3.811429in}{2.491344in}}%
\pgfpathcurveto{\pgfqpoint{3.811429in}{2.478321in}}{\pgfqpoint{3.816603in}{2.465830in}}{\pgfqpoint{3.825811in}{2.456621in}}%
\pgfpathcurveto{\pgfqpoint{3.835020in}{2.447413in}}{\pgfqpoint{3.847511in}{2.442239in}}{\pgfqpoint{3.860534in}{2.442239in}}%
\pgfpathlineto{\pgfqpoint{3.860534in}{2.442239in}}%
\pgfpathclose%
\pgfusepath{stroke,fill}%
\end{pgfscope}%
\begin{pgfscope}%
\pgfpathrectangle{\pgfqpoint{0.786164in}{0.768110in}}{\pgfqpoint{8.851069in}{7.081890in}}%
\pgfusepath{clip}%
\pgfsetbuttcap%
\pgfsetroundjoin%
\definecolor{currentfill}{rgb}{0.282910,0.105393,0.426902}%
\pgfsetfillcolor{currentfill}%
\pgfsetfillopacity{0.700000}%
\pgfsetlinewidth{0.501875pt}%
\definecolor{currentstroke}{rgb}{1.000000,1.000000,1.000000}%
\pgfsetstrokecolor{currentstroke}%
\pgfsetstrokeopacity{0.700000}%
\pgfsetdash{}{0pt}%
\pgfpathmoveto{\pgfqpoint{4.057116in}{2.335528in}}%
\pgfpathcurveto{\pgfqpoint{4.070138in}{2.335528in}}{\pgfqpoint{4.082629in}{2.340702in}}{\pgfqpoint{4.091838in}{2.349911in}}%
\pgfpathcurveto{\pgfqpoint{4.101046in}{2.359119in}}{\pgfqpoint{4.106220in}{2.371610in}}{\pgfqpoint{4.106220in}{2.384633in}}%
\pgfpathcurveto{\pgfqpoint{4.106220in}{2.397656in}}{\pgfqpoint{4.101046in}{2.410147in}}{\pgfqpoint{4.091838in}{2.419355in}}%
\pgfpathcurveto{\pgfqpoint{4.082629in}{2.428563in}}{\pgfqpoint{4.070138in}{2.433737in}}{\pgfqpoint{4.057116in}{2.433737in}}%
\pgfpathcurveto{\pgfqpoint{4.044093in}{2.433737in}}{\pgfqpoint{4.031602in}{2.428563in}}{\pgfqpoint{4.022393in}{2.419355in}}%
\pgfpathcurveto{\pgfqpoint{4.013185in}{2.410147in}}{\pgfqpoint{4.008011in}{2.397656in}}{\pgfqpoint{4.008011in}{2.384633in}}%
\pgfpathcurveto{\pgfqpoint{4.008011in}{2.371610in}}{\pgfqpoint{4.013185in}{2.359119in}}{\pgfqpoint{4.022393in}{2.349911in}}%
\pgfpathcurveto{\pgfqpoint{4.031602in}{2.340702in}}{\pgfqpoint{4.044093in}{2.335528in}}{\pgfqpoint{4.057116in}{2.335528in}}%
\pgfpathlineto{\pgfqpoint{4.057116in}{2.335528in}}%
\pgfpathclose%
\pgfusepath{stroke,fill}%
\end{pgfscope}%
\begin{pgfscope}%
\pgfpathrectangle{\pgfqpoint{0.786164in}{0.768110in}}{\pgfqpoint{8.851069in}{7.081890in}}%
\pgfusepath{clip}%
\pgfsetbuttcap%
\pgfsetroundjoin%
\definecolor{currentfill}{rgb}{0.282910,0.105393,0.426902}%
\pgfsetfillcolor{currentfill}%
\pgfsetfillopacity{0.700000}%
\pgfsetlinewidth{0.501875pt}%
\definecolor{currentstroke}{rgb}{1.000000,1.000000,1.000000}%
\pgfsetstrokecolor{currentstroke}%
\pgfsetstrokeopacity{0.700000}%
\pgfsetdash{}{0pt}%
\pgfpathmoveto{\pgfqpoint{4.241121in}{2.314186in}}%
\pgfpathcurveto{\pgfqpoint{4.254144in}{2.314186in}}{\pgfqpoint{4.266635in}{2.319360in}}{\pgfqpoint{4.275843in}{2.328568in}}%
\pgfpathcurveto{\pgfqpoint{4.285052in}{2.337777in}}{\pgfqpoint{4.290226in}{2.350268in}}{\pgfqpoint{4.290226in}{2.363291in}}%
\pgfpathcurveto{\pgfqpoint{4.290226in}{2.376313in}}{\pgfqpoint{4.285052in}{2.388804in}}{\pgfqpoint{4.275843in}{2.398013in}}%
\pgfpathcurveto{\pgfqpoint{4.266635in}{2.407221in}}{\pgfqpoint{4.254144in}{2.412395in}}{\pgfqpoint{4.241121in}{2.412395in}}%
\pgfpathcurveto{\pgfqpoint{4.228098in}{2.412395in}}{\pgfqpoint{4.215607in}{2.407221in}}{\pgfqpoint{4.206399in}{2.398013in}}%
\pgfpathcurveto{\pgfqpoint{4.197190in}{2.388804in}}{\pgfqpoint{4.192016in}{2.376313in}}{\pgfqpoint{4.192016in}{2.363291in}}%
\pgfpathcurveto{\pgfqpoint{4.192016in}{2.350268in}}{\pgfqpoint{4.197190in}{2.337777in}}{\pgfqpoint{4.206399in}{2.328568in}}%
\pgfpathcurveto{\pgfqpoint{4.215607in}{2.319360in}}{\pgfqpoint{4.228098in}{2.314186in}}{\pgfqpoint{4.241121in}{2.314186in}}%
\pgfpathlineto{\pgfqpoint{4.241121in}{2.314186in}}%
\pgfpathclose%
\pgfusepath{stroke,fill}%
\end{pgfscope}%
\begin{pgfscope}%
\pgfpathrectangle{\pgfqpoint{0.786164in}{0.768110in}}{\pgfqpoint{8.851069in}{7.081890in}}%
\pgfusepath{clip}%
\pgfsetbuttcap%
\pgfsetroundjoin%
\definecolor{currentfill}{rgb}{0.282327,0.094955,0.417331}%
\pgfsetfillcolor{currentfill}%
\pgfsetfillopacity{0.700000}%
\pgfsetlinewidth{0.501875pt}%
\definecolor{currentstroke}{rgb}{1.000000,1.000000,1.000000}%
\pgfsetstrokecolor{currentstroke}%
\pgfsetstrokeopacity{0.700000}%
\pgfsetdash{}{0pt}%
\pgfpathmoveto{\pgfqpoint{4.276042in}{2.378213in}}%
\pgfpathcurveto{\pgfqpoint{4.289065in}{2.378213in}}{\pgfqpoint{4.301556in}{2.383386in}}{\pgfqpoint{4.310764in}{2.392595in}}%
\pgfpathcurveto{\pgfqpoint{4.319972in}{2.401803in}}{\pgfqpoint{4.325146in}{2.414294in}}{\pgfqpoint{4.325146in}{2.427317in}}%
\pgfpathcurveto{\pgfqpoint{4.325146in}{2.440340in}}{\pgfqpoint{4.319972in}{2.452831in}}{\pgfqpoint{4.310764in}{2.462039in}}%
\pgfpathcurveto{\pgfqpoint{4.301556in}{2.471248in}}{\pgfqpoint{4.289065in}{2.476422in}}{\pgfqpoint{4.276042in}{2.476422in}}%
\pgfpathcurveto{\pgfqpoint{4.263019in}{2.476422in}}{\pgfqpoint{4.250528in}{2.471248in}}{\pgfqpoint{4.241320in}{2.462039in}}%
\pgfpathcurveto{\pgfqpoint{4.232111in}{2.452831in}}{\pgfqpoint{4.226937in}{2.440340in}}{\pgfqpoint{4.226937in}{2.427317in}}%
\pgfpathcurveto{\pgfqpoint{4.226937in}{2.414294in}}{\pgfqpoint{4.232111in}{2.401803in}}{\pgfqpoint{4.241320in}{2.392595in}}%
\pgfpathcurveto{\pgfqpoint{4.250528in}{2.383386in}}{\pgfqpoint{4.263019in}{2.378213in}}{\pgfqpoint{4.276042in}{2.378213in}}%
\pgfpathlineto{\pgfqpoint{4.276042in}{2.378213in}}%
\pgfpathclose%
\pgfusepath{stroke,fill}%
\end{pgfscope}%
\begin{pgfscope}%
\pgfpathrectangle{\pgfqpoint{0.786164in}{0.768110in}}{\pgfqpoint{8.851069in}{7.081890in}}%
\pgfusepath{clip}%
\pgfsetbuttcap%
\pgfsetroundjoin%
\definecolor{currentfill}{rgb}{0.280894,0.078907,0.402329}%
\pgfsetfillcolor{currentfill}%
\pgfsetfillopacity{0.700000}%
\pgfsetlinewidth{0.501875pt}%
\definecolor{currentstroke}{rgb}{1.000000,1.000000,1.000000}%
\pgfsetstrokecolor{currentstroke}%
\pgfsetstrokeopacity{0.700000}%
\pgfsetdash{}{0pt}%
\pgfpathmoveto{\pgfqpoint{4.443075in}{2.314186in}}%
\pgfpathcurveto{\pgfqpoint{4.456098in}{2.314186in}}{\pgfqpoint{4.468589in}{2.319360in}}{\pgfqpoint{4.477798in}{2.328568in}}%
\pgfpathcurveto{\pgfqpoint{4.487006in}{2.337777in}}{\pgfqpoint{4.492180in}{2.350268in}}{\pgfqpoint{4.492180in}{2.363291in}}%
\pgfpathcurveto{\pgfqpoint{4.492180in}{2.376313in}}{\pgfqpoint{4.487006in}{2.388804in}}{\pgfqpoint{4.477798in}{2.398013in}}%
\pgfpathcurveto{\pgfqpoint{4.468589in}{2.407221in}}{\pgfqpoint{4.456098in}{2.412395in}}{\pgfqpoint{4.443075in}{2.412395in}}%
\pgfpathcurveto{\pgfqpoint{4.430053in}{2.412395in}}{\pgfqpoint{4.417562in}{2.407221in}}{\pgfqpoint{4.408353in}{2.398013in}}%
\pgfpathcurveto{\pgfqpoint{4.399145in}{2.388804in}}{\pgfqpoint{4.393971in}{2.376313in}}{\pgfqpoint{4.393971in}{2.363291in}}%
\pgfpathcurveto{\pgfqpoint{4.393971in}{2.350268in}}{\pgfqpoint{4.399145in}{2.337777in}}{\pgfqpoint{4.408353in}{2.328568in}}%
\pgfpathcurveto{\pgfqpoint{4.417562in}{2.319360in}}{\pgfqpoint{4.430053in}{2.314186in}}{\pgfqpoint{4.443075in}{2.314186in}}%
\pgfpathlineto{\pgfqpoint{4.443075in}{2.314186in}}%
\pgfpathclose%
\pgfusepath{stroke,fill}%
\end{pgfscope}%
\begin{pgfscope}%
\pgfpathrectangle{\pgfqpoint{0.786164in}{0.768110in}}{\pgfqpoint{8.851069in}{7.081890in}}%
\pgfusepath{clip}%
\pgfsetbuttcap%
\pgfsetroundjoin%
\definecolor{currentfill}{rgb}{0.280267,0.073417,0.397163}%
\pgfsetfillcolor{currentfill}%
\pgfsetfillopacity{0.700000}%
\pgfsetlinewidth{0.501875pt}%
\definecolor{currentstroke}{rgb}{1.000000,1.000000,1.000000}%
\pgfsetstrokecolor{currentstroke}%
\pgfsetstrokeopacity{0.700000}%
\pgfsetdash{}{0pt}%
\pgfpathmoveto{\pgfqpoint{4.599730in}{2.228817in}}%
\pgfpathcurveto{\pgfqpoint{4.612753in}{2.228817in}}{\pgfqpoint{4.625244in}{2.233991in}}{\pgfqpoint{4.634453in}{2.243200in}}%
\pgfpathcurveto{\pgfqpoint{4.643661in}{2.252408in}}{\pgfqpoint{4.648835in}{2.264899in}}{\pgfqpoint{4.648835in}{2.277922in}}%
\pgfpathcurveto{\pgfqpoint{4.648835in}{2.290945in}}{\pgfqpoint{4.643661in}{2.303436in}}{\pgfqpoint{4.634453in}{2.312644in}}%
\pgfpathcurveto{\pgfqpoint{4.625244in}{2.321853in}}{\pgfqpoint{4.612753in}{2.327027in}}{\pgfqpoint{4.599730in}{2.327027in}}%
\pgfpathcurveto{\pgfqpoint{4.586708in}{2.327027in}}{\pgfqpoint{4.574217in}{2.321853in}}{\pgfqpoint{4.565008in}{2.312644in}}%
\pgfpathcurveto{\pgfqpoint{4.555800in}{2.303436in}}{\pgfqpoint{4.550626in}{2.290945in}}{\pgfqpoint{4.550626in}{2.277922in}}%
\pgfpathcurveto{\pgfqpoint{4.550626in}{2.264899in}}{\pgfqpoint{4.555800in}{2.252408in}}{\pgfqpoint{4.565008in}{2.243200in}}%
\pgfpathcurveto{\pgfqpoint{4.574217in}{2.233991in}}{\pgfqpoint{4.586708in}{2.228817in}}{\pgfqpoint{4.599730in}{2.228817in}}%
\pgfpathlineto{\pgfqpoint{4.599730in}{2.228817in}}%
\pgfpathclose%
\pgfusepath{stroke,fill}%
\end{pgfscope}%
\begin{pgfscope}%
\pgfpathrectangle{\pgfqpoint{0.786164in}{0.768110in}}{\pgfqpoint{8.851069in}{7.081890in}}%
\pgfusepath{clip}%
\pgfsetbuttcap%
\pgfsetroundjoin%
\definecolor{currentfill}{rgb}{0.278791,0.062145,0.386592}%
\pgfsetfillcolor{currentfill}%
\pgfsetfillopacity{0.700000}%
\pgfsetlinewidth{0.501875pt}%
\definecolor{currentstroke}{rgb}{1.000000,1.000000,1.000000}%
\pgfsetstrokecolor{currentstroke}%
\pgfsetstrokeopacity{0.700000}%
\pgfsetdash{}{0pt}%
\pgfpathmoveto{\pgfqpoint{4.572502in}{2.207475in}}%
\pgfpathcurveto{\pgfqpoint{4.585525in}{2.207475in}}{\pgfqpoint{4.598016in}{2.212649in}}{\pgfqpoint{4.607224in}{2.221858in}}%
\pgfpathcurveto{\pgfqpoint{4.616433in}{2.231066in}}{\pgfqpoint{4.621607in}{2.243557in}}{\pgfqpoint{4.621607in}{2.256580in}}%
\pgfpathcurveto{\pgfqpoint{4.621607in}{2.269602in}}{\pgfqpoint{4.616433in}{2.282094in}}{\pgfqpoint{4.607224in}{2.291302in}}%
\pgfpathcurveto{\pgfqpoint{4.598016in}{2.300510in}}{\pgfqpoint{4.585525in}{2.305684in}}{\pgfqpoint{4.572502in}{2.305684in}}%
\pgfpathcurveto{\pgfqpoint{4.559479in}{2.305684in}}{\pgfqpoint{4.546988in}{2.300510in}}{\pgfqpoint{4.537780in}{2.291302in}}%
\pgfpathcurveto{\pgfqpoint{4.528571in}{2.282094in}}{\pgfqpoint{4.523397in}{2.269602in}}{\pgfqpoint{4.523397in}{2.256580in}}%
\pgfpathcurveto{\pgfqpoint{4.523397in}{2.243557in}}{\pgfqpoint{4.528571in}{2.231066in}}{\pgfqpoint{4.537780in}{2.221858in}}%
\pgfpathcurveto{\pgfqpoint{4.546988in}{2.212649in}}{\pgfqpoint{4.559479in}{2.207475in}}{\pgfqpoint{4.572502in}{2.207475in}}%
\pgfpathlineto{\pgfqpoint{4.572502in}{2.207475in}}%
\pgfpathclose%
\pgfusepath{stroke,fill}%
\end{pgfscope}%
\begin{pgfscope}%
\pgfpathrectangle{\pgfqpoint{0.786164in}{0.768110in}}{\pgfqpoint{8.851069in}{7.081890in}}%
\pgfusepath{clip}%
\pgfsetbuttcap%
\pgfsetroundjoin%
\definecolor{currentfill}{rgb}{0.280894,0.078907,0.402329}%
\pgfsetfillcolor{currentfill}%
\pgfsetfillopacity{0.700000}%
\pgfsetlinewidth{0.501875pt}%
\definecolor{currentstroke}{rgb}{1.000000,1.000000,1.000000}%
\pgfsetstrokecolor{currentstroke}%
\pgfsetstrokeopacity{0.700000}%
\pgfsetdash{}{0pt}%
\pgfpathmoveto{\pgfqpoint{4.713162in}{2.250159in}}%
\pgfpathcurveto{\pgfqpoint{4.726184in}{2.250159in}}{\pgfqpoint{4.738676in}{2.255333in}}{\pgfqpoint{4.747884in}{2.264542in}}%
\pgfpathcurveto{\pgfqpoint{4.757092in}{2.273750in}}{\pgfqpoint{4.762266in}{2.286241in}}{\pgfqpoint{4.762266in}{2.299264in}}%
\pgfpathcurveto{\pgfqpoint{4.762266in}{2.312287in}}{\pgfqpoint{4.757092in}{2.324778in}}{\pgfqpoint{4.747884in}{2.333986in}}%
\pgfpathcurveto{\pgfqpoint{4.738676in}{2.343195in}}{\pgfqpoint{4.726184in}{2.348369in}}{\pgfqpoint{4.713162in}{2.348369in}}%
\pgfpathcurveto{\pgfqpoint{4.700139in}{2.348369in}}{\pgfqpoint{4.687648in}{2.343195in}}{\pgfqpoint{4.678440in}{2.333986in}}%
\pgfpathcurveto{\pgfqpoint{4.669231in}{2.324778in}}{\pgfqpoint{4.664057in}{2.312287in}}{\pgfqpoint{4.664057in}{2.299264in}}%
\pgfpathcurveto{\pgfqpoint{4.664057in}{2.286241in}}{\pgfqpoint{4.669231in}{2.273750in}}{\pgfqpoint{4.678440in}{2.264542in}}%
\pgfpathcurveto{\pgfqpoint{4.687648in}{2.255333in}}{\pgfqpoint{4.700139in}{2.250159in}}{\pgfqpoint{4.713162in}{2.250159in}}%
\pgfpathlineto{\pgfqpoint{4.713162in}{2.250159in}}%
\pgfpathclose%
\pgfusepath{stroke,fill}%
\end{pgfscope}%
\begin{pgfscope}%
\pgfpathrectangle{\pgfqpoint{0.786164in}{0.768110in}}{\pgfqpoint{8.851069in}{7.081890in}}%
\pgfusepath{clip}%
\pgfsetbuttcap%
\pgfsetroundjoin%
\definecolor{currentfill}{rgb}{0.281446,0.084320,0.407414}%
\pgfsetfillcolor{currentfill}%
\pgfsetfillopacity{0.700000}%
\pgfsetlinewidth{0.501875pt}%
\definecolor{currentstroke}{rgb}{1.000000,1.000000,1.000000}%
\pgfsetstrokecolor{currentstroke}%
\pgfsetstrokeopacity{0.700000}%
\pgfsetdash{}{0pt}%
\pgfpathmoveto{\pgfqpoint{4.627325in}{2.271502in}}%
\pgfpathcurveto{\pgfqpoint{4.640348in}{2.271502in}}{\pgfqpoint{4.652839in}{2.276676in}}{\pgfqpoint{4.662047in}{2.285884in}}%
\pgfpathcurveto{\pgfqpoint{4.671256in}{2.295093in}}{\pgfqpoint{4.676430in}{2.307584in}}{\pgfqpoint{4.676430in}{2.320606in}}%
\pgfpathcurveto{\pgfqpoint{4.676430in}{2.333629in}}{\pgfqpoint{4.671256in}{2.346120in}}{\pgfqpoint{4.662047in}{2.355329in}}%
\pgfpathcurveto{\pgfqpoint{4.652839in}{2.364537in}}{\pgfqpoint{4.640348in}{2.369711in}}{\pgfqpoint{4.627325in}{2.369711in}}%
\pgfpathcurveto{\pgfqpoint{4.614302in}{2.369711in}}{\pgfqpoint{4.601811in}{2.364537in}}{\pgfqpoint{4.592603in}{2.355329in}}%
\pgfpathcurveto{\pgfqpoint{4.583394in}{2.346120in}}{\pgfqpoint{4.578220in}{2.333629in}}{\pgfqpoint{4.578220in}{2.320606in}}%
\pgfpathcurveto{\pgfqpoint{4.578220in}{2.307584in}}{\pgfqpoint{4.583394in}{2.295093in}}{\pgfqpoint{4.592603in}{2.285884in}}%
\pgfpathcurveto{\pgfqpoint{4.601811in}{2.276676in}}{\pgfqpoint{4.614302in}{2.271502in}}{\pgfqpoint{4.627325in}{2.271502in}}%
\pgfpathlineto{\pgfqpoint{4.627325in}{2.271502in}}%
\pgfpathclose%
\pgfusepath{stroke,fill}%
\end{pgfscope}%
\begin{pgfscope}%
\pgfpathrectangle{\pgfqpoint{0.786164in}{0.768110in}}{\pgfqpoint{8.851069in}{7.081890in}}%
\pgfusepath{clip}%
\pgfsetbuttcap%
\pgfsetroundjoin%
\definecolor{currentfill}{rgb}{0.282327,0.094955,0.417331}%
\pgfsetfillcolor{currentfill}%
\pgfsetfillopacity{0.700000}%
\pgfsetlinewidth{0.501875pt}%
\definecolor{currentstroke}{rgb}{1.000000,1.000000,1.000000}%
\pgfsetstrokecolor{currentstroke}%
\pgfsetstrokeopacity{0.700000}%
\pgfsetdash{}{0pt}%
\pgfpathmoveto{\pgfqpoint{4.506934in}{2.378213in}}%
\pgfpathcurveto{\pgfqpoint{4.519957in}{2.378213in}}{\pgfqpoint{4.532448in}{2.383386in}}{\pgfqpoint{4.541656in}{2.392595in}}%
\pgfpathcurveto{\pgfqpoint{4.550865in}{2.401803in}}{\pgfqpoint{4.556039in}{2.414294in}}{\pgfqpoint{4.556039in}{2.427317in}}%
\pgfpathcurveto{\pgfqpoint{4.556039in}{2.440340in}}{\pgfqpoint{4.550865in}{2.452831in}}{\pgfqpoint{4.541656in}{2.462039in}}%
\pgfpathcurveto{\pgfqpoint{4.532448in}{2.471248in}}{\pgfqpoint{4.519957in}{2.476422in}}{\pgfqpoint{4.506934in}{2.476422in}}%
\pgfpathcurveto{\pgfqpoint{4.493911in}{2.476422in}}{\pgfqpoint{4.481420in}{2.471248in}}{\pgfqpoint{4.472212in}{2.462039in}}%
\pgfpathcurveto{\pgfqpoint{4.463003in}{2.452831in}}{\pgfqpoint{4.457829in}{2.440340in}}{\pgfqpoint{4.457829in}{2.427317in}}%
\pgfpathcurveto{\pgfqpoint{4.457829in}{2.414294in}}{\pgfqpoint{4.463003in}{2.401803in}}{\pgfqpoint{4.472212in}{2.392595in}}%
\pgfpathcurveto{\pgfqpoint{4.481420in}{2.383386in}}{\pgfqpoint{4.493911in}{2.378213in}}{\pgfqpoint{4.506934in}{2.378213in}}%
\pgfpathlineto{\pgfqpoint{4.506934in}{2.378213in}}%
\pgfpathclose%
\pgfusepath{stroke,fill}%
\end{pgfscope}%
\begin{pgfscope}%
\pgfpathrectangle{\pgfqpoint{0.786164in}{0.768110in}}{\pgfqpoint{8.851069in}{7.081890in}}%
\pgfusepath{clip}%
\pgfsetbuttcap%
\pgfsetroundjoin%
\definecolor{currentfill}{rgb}{0.282327,0.094955,0.417331}%
\pgfsetfillcolor{currentfill}%
\pgfsetfillopacity{0.700000}%
\pgfsetlinewidth{0.501875pt}%
\definecolor{currentstroke}{rgb}{1.000000,1.000000,1.000000}%
\pgfsetstrokecolor{currentstroke}%
\pgfsetstrokeopacity{0.700000}%
\pgfsetdash{}{0pt}%
\pgfpathmoveto{\pgfqpoint{4.600585in}{2.292844in}}%
\pgfpathcurveto{\pgfqpoint{4.613608in}{2.292844in}}{\pgfqpoint{4.626099in}{2.298018in}}{\pgfqpoint{4.635307in}{2.307226in}}%
\pgfpathcurveto{\pgfqpoint{4.644516in}{2.316435in}}{\pgfqpoint{4.649690in}{2.328926in}}{\pgfqpoint{4.649690in}{2.341948in}}%
\pgfpathcurveto{\pgfqpoint{4.649690in}{2.354971in}}{\pgfqpoint{4.644516in}{2.367462in}}{\pgfqpoint{4.635307in}{2.376671in}}%
\pgfpathcurveto{\pgfqpoint{4.626099in}{2.385879in}}{\pgfqpoint{4.613608in}{2.391053in}}{\pgfqpoint{4.600585in}{2.391053in}}%
\pgfpathcurveto{\pgfqpoint{4.587562in}{2.391053in}}{\pgfqpoint{4.575071in}{2.385879in}}{\pgfqpoint{4.565863in}{2.376671in}}%
\pgfpathcurveto{\pgfqpoint{4.556654in}{2.367462in}}{\pgfqpoint{4.551480in}{2.354971in}}{\pgfqpoint{4.551480in}{2.341948in}}%
\pgfpathcurveto{\pgfqpoint{4.551480in}{2.328926in}}{\pgfqpoint{4.556654in}{2.316435in}}{\pgfqpoint{4.565863in}{2.307226in}}%
\pgfpathcurveto{\pgfqpoint{4.575071in}{2.298018in}}{\pgfqpoint{4.587562in}{2.292844in}}{\pgfqpoint{4.600585in}{2.292844in}}%
\pgfpathlineto{\pgfqpoint{4.600585in}{2.292844in}}%
\pgfpathclose%
\pgfusepath{stroke,fill}%
\end{pgfscope}%
\begin{pgfscope}%
\pgfpathrectangle{\pgfqpoint{0.786164in}{0.768110in}}{\pgfqpoint{8.851069in}{7.081890in}}%
\pgfusepath{clip}%
\pgfsetbuttcap%
\pgfsetroundjoin%
\definecolor{currentfill}{rgb}{0.282327,0.094955,0.417331}%
\pgfsetfillcolor{currentfill}%
\pgfsetfillopacity{0.700000}%
\pgfsetlinewidth{0.501875pt}%
\definecolor{currentstroke}{rgb}{1.000000,1.000000,1.000000}%
\pgfsetstrokecolor{currentstroke}%
\pgfsetstrokeopacity{0.700000}%
\pgfsetdash{}{0pt}%
\pgfpathmoveto{\pgfqpoint{4.651745in}{2.292844in}}%
\pgfpathcurveto{\pgfqpoint{4.664768in}{2.292844in}}{\pgfqpoint{4.677259in}{2.298018in}}{\pgfqpoint{4.686467in}{2.307226in}}%
\pgfpathcurveto{\pgfqpoint{4.695676in}{2.316435in}}{\pgfqpoint{4.700850in}{2.328926in}}{\pgfqpoint{4.700850in}{2.341948in}}%
\pgfpathcurveto{\pgfqpoint{4.700850in}{2.354971in}}{\pgfqpoint{4.695676in}{2.367462in}}{\pgfqpoint{4.686467in}{2.376671in}}%
\pgfpathcurveto{\pgfqpoint{4.677259in}{2.385879in}}{\pgfqpoint{4.664768in}{2.391053in}}{\pgfqpoint{4.651745in}{2.391053in}}%
\pgfpathcurveto{\pgfqpoint{4.638723in}{2.391053in}}{\pgfqpoint{4.626231in}{2.385879in}}{\pgfqpoint{4.617023in}{2.376671in}}%
\pgfpathcurveto{\pgfqpoint{4.607815in}{2.367462in}}{\pgfqpoint{4.602641in}{2.354971in}}{\pgfqpoint{4.602641in}{2.341948in}}%
\pgfpathcurveto{\pgfqpoint{4.602641in}{2.328926in}}{\pgfqpoint{4.607815in}{2.316435in}}{\pgfqpoint{4.617023in}{2.307226in}}%
\pgfpathcurveto{\pgfqpoint{4.626231in}{2.298018in}}{\pgfqpoint{4.638723in}{2.292844in}}{\pgfqpoint{4.651745in}{2.292844in}}%
\pgfpathlineto{\pgfqpoint{4.651745in}{2.292844in}}%
\pgfpathclose%
\pgfusepath{stroke,fill}%
\end{pgfscope}%
\begin{pgfscope}%
\pgfpathrectangle{\pgfqpoint{0.786164in}{0.768110in}}{\pgfqpoint{8.851069in}{7.081890in}}%
\pgfusepath{clip}%
\pgfsetbuttcap%
\pgfsetroundjoin%
\definecolor{currentfill}{rgb}{0.280894,0.078907,0.402329}%
\pgfsetfillcolor{currentfill}%
\pgfsetfillopacity{0.700000}%
\pgfsetlinewidth{0.501875pt}%
\definecolor{currentstroke}{rgb}{1.000000,1.000000,1.000000}%
\pgfsetstrokecolor{currentstroke}%
\pgfsetstrokeopacity{0.700000}%
\pgfsetdash{}{0pt}%
\pgfpathmoveto{\pgfqpoint{4.963956in}{2.228817in}}%
\pgfpathcurveto{\pgfqpoint{4.976979in}{2.228817in}}{\pgfqpoint{4.989470in}{2.233991in}}{\pgfqpoint{4.998679in}{2.243200in}}%
\pgfpathcurveto{\pgfqpoint{5.007887in}{2.252408in}}{\pgfqpoint{5.013061in}{2.264899in}}{\pgfqpoint{5.013061in}{2.277922in}}%
\pgfpathcurveto{\pgfqpoint{5.013061in}{2.290945in}}{\pgfqpoint{5.007887in}{2.303436in}}{\pgfqpoint{4.998679in}{2.312644in}}%
\pgfpathcurveto{\pgfqpoint{4.989470in}{2.321853in}}{\pgfqpoint{4.976979in}{2.327027in}}{\pgfqpoint{4.963956in}{2.327027in}}%
\pgfpathcurveto{\pgfqpoint{4.950934in}{2.327027in}}{\pgfqpoint{4.938443in}{2.321853in}}{\pgfqpoint{4.929234in}{2.312644in}}%
\pgfpathcurveto{\pgfqpoint{4.920026in}{2.303436in}}{\pgfqpoint{4.914852in}{2.290945in}}{\pgfqpoint{4.914852in}{2.277922in}}%
\pgfpathcurveto{\pgfqpoint{4.914852in}{2.264899in}}{\pgfqpoint{4.920026in}{2.252408in}}{\pgfqpoint{4.929234in}{2.243200in}}%
\pgfpathcurveto{\pgfqpoint{4.938443in}{2.233991in}}{\pgfqpoint{4.950934in}{2.228817in}}{\pgfqpoint{4.963956in}{2.228817in}}%
\pgfpathlineto{\pgfqpoint{4.963956in}{2.228817in}}%
\pgfpathclose%
\pgfusepath{stroke,fill}%
\end{pgfscope}%
\begin{pgfscope}%
\pgfpathrectangle{\pgfqpoint{0.786164in}{0.768110in}}{\pgfqpoint{8.851069in}{7.081890in}}%
\pgfusepath{clip}%
\pgfsetbuttcap%
\pgfsetroundjoin%
\definecolor{currentfill}{rgb}{0.282910,0.105393,0.426902}%
\pgfsetfillcolor{currentfill}%
\pgfsetfillopacity{0.700000}%
\pgfsetlinewidth{0.501875pt}%
\definecolor{currentstroke}{rgb}{1.000000,1.000000,1.000000}%
\pgfsetstrokecolor{currentstroke}%
\pgfsetstrokeopacity{0.700000}%
\pgfsetdash{}{0pt}%
\pgfpathmoveto{\pgfqpoint{4.995947in}{2.271502in}}%
\pgfpathcurveto{\pgfqpoint{5.008969in}{2.271502in}}{\pgfqpoint{5.021460in}{2.276676in}}{\pgfqpoint{5.030669in}{2.285884in}}%
\pgfpathcurveto{\pgfqpoint{5.039877in}{2.295093in}}{\pgfqpoint{5.045051in}{2.307584in}}{\pgfqpoint{5.045051in}{2.320606in}}%
\pgfpathcurveto{\pgfqpoint{5.045051in}{2.333629in}}{\pgfqpoint{5.039877in}{2.346120in}}{\pgfqpoint{5.030669in}{2.355329in}}%
\pgfpathcurveto{\pgfqpoint{5.021460in}{2.364537in}}{\pgfqpoint{5.008969in}{2.369711in}}{\pgfqpoint{4.995947in}{2.369711in}}%
\pgfpathcurveto{\pgfqpoint{4.982924in}{2.369711in}}{\pgfqpoint{4.970433in}{2.364537in}}{\pgfqpoint{4.961224in}{2.355329in}}%
\pgfpathcurveto{\pgfqpoint{4.952016in}{2.346120in}}{\pgfqpoint{4.946842in}{2.333629in}}{\pgfqpoint{4.946842in}{2.320606in}}%
\pgfpathcurveto{\pgfqpoint{4.946842in}{2.307584in}}{\pgfqpoint{4.952016in}{2.295093in}}{\pgfqpoint{4.961224in}{2.285884in}}%
\pgfpathcurveto{\pgfqpoint{4.970433in}{2.276676in}}{\pgfqpoint{4.982924in}{2.271502in}}{\pgfqpoint{4.995947in}{2.271502in}}%
\pgfpathlineto{\pgfqpoint{4.995947in}{2.271502in}}%
\pgfpathclose%
\pgfusepath{stroke,fill}%
\end{pgfscope}%
\begin{pgfscope}%
\pgfpathrectangle{\pgfqpoint{0.786164in}{0.768110in}}{\pgfqpoint{8.851069in}{7.081890in}}%
\pgfusepath{clip}%
\pgfsetbuttcap%
\pgfsetroundjoin%
\definecolor{currentfill}{rgb}{0.283197,0.115680,0.436115}%
\pgfsetfillcolor{currentfill}%
\pgfsetfillopacity{0.700000}%
\pgfsetlinewidth{0.501875pt}%
\definecolor{currentstroke}{rgb}{1.000000,1.000000,1.000000}%
\pgfsetstrokecolor{currentstroke}%
\pgfsetstrokeopacity{0.700000}%
\pgfsetdash{}{0pt}%
\pgfpathmoveto{\pgfqpoint{4.760170in}{2.356870in}}%
\pgfpathcurveto{\pgfqpoint{4.773193in}{2.356870in}}{\pgfqpoint{4.785684in}{2.362044in}}{\pgfqpoint{4.794893in}{2.371253in}}%
\pgfpathcurveto{\pgfqpoint{4.804101in}{2.380461in}}{\pgfqpoint{4.809275in}{2.392952in}}{\pgfqpoint{4.809275in}{2.405975in}}%
\pgfpathcurveto{\pgfqpoint{4.809275in}{2.418998in}}{\pgfqpoint{4.804101in}{2.431489in}}{\pgfqpoint{4.794893in}{2.440697in}}%
\pgfpathcurveto{\pgfqpoint{4.785684in}{2.449906in}}{\pgfqpoint{4.773193in}{2.455080in}}{\pgfqpoint{4.760170in}{2.455080in}}%
\pgfpathcurveto{\pgfqpoint{4.747148in}{2.455080in}}{\pgfqpoint{4.734657in}{2.449906in}}{\pgfqpoint{4.725448in}{2.440697in}}%
\pgfpathcurveto{\pgfqpoint{4.716240in}{2.431489in}}{\pgfqpoint{4.711066in}{2.418998in}}{\pgfqpoint{4.711066in}{2.405975in}}%
\pgfpathcurveto{\pgfqpoint{4.711066in}{2.392952in}}{\pgfqpoint{4.716240in}{2.380461in}}{\pgfqpoint{4.725448in}{2.371253in}}%
\pgfpathcurveto{\pgfqpoint{4.734657in}{2.362044in}}{\pgfqpoint{4.747148in}{2.356870in}}{\pgfqpoint{4.760170in}{2.356870in}}%
\pgfpathlineto{\pgfqpoint{4.760170in}{2.356870in}}%
\pgfpathclose%
\pgfusepath{stroke,fill}%
\end{pgfscope}%
\begin{pgfscope}%
\pgfpathrectangle{\pgfqpoint{0.786164in}{0.768110in}}{\pgfqpoint{8.851069in}{7.081890in}}%
\pgfusepath{clip}%
\pgfsetbuttcap%
\pgfsetroundjoin%
\definecolor{currentfill}{rgb}{0.280868,0.160771,0.472899}%
\pgfsetfillcolor{currentfill}%
\pgfsetfillopacity{0.700000}%
\pgfsetlinewidth{0.501875pt}%
\definecolor{currentstroke}{rgb}{1.000000,1.000000,1.000000}%
\pgfsetstrokecolor{currentstroke}%
\pgfsetstrokeopacity{0.700000}%
\pgfsetdash{}{0pt}%
\pgfpathmoveto{\pgfqpoint{1.708878in}{2.762372in}}%
\pgfpathcurveto{\pgfqpoint{1.721900in}{2.762372in}}{\pgfqpoint{1.734392in}{2.767546in}}{\pgfqpoint{1.743600in}{2.776754in}}%
\pgfpathcurveto{\pgfqpoint{1.752808in}{2.785962in}}{\pgfqpoint{1.757982in}{2.798454in}}{\pgfqpoint{1.757982in}{2.811476in}}%
\pgfpathcurveto{\pgfqpoint{1.757982in}{2.824499in}}{\pgfqpoint{1.752808in}{2.836990in}}{\pgfqpoint{1.743600in}{2.846198in}}%
\pgfpathcurveto{\pgfqpoint{1.734392in}{2.855407in}}{\pgfqpoint{1.721900in}{2.860581in}}{\pgfqpoint{1.708878in}{2.860581in}}%
\pgfpathcurveto{\pgfqpoint{1.695855in}{2.860581in}}{\pgfqpoint{1.683364in}{2.855407in}}{\pgfqpoint{1.674156in}{2.846198in}}%
\pgfpathcurveto{\pgfqpoint{1.664947in}{2.836990in}}{\pgfqpoint{1.659773in}{2.824499in}}{\pgfqpoint{1.659773in}{2.811476in}}%
\pgfpathcurveto{\pgfqpoint{1.659773in}{2.798454in}}{\pgfqpoint{1.664947in}{2.785962in}}{\pgfqpoint{1.674156in}{2.776754in}}%
\pgfpathcurveto{\pgfqpoint{1.683364in}{2.767546in}}{\pgfqpoint{1.695855in}{2.762372in}}{\pgfqpoint{1.708878in}{2.762372in}}%
\pgfpathlineto{\pgfqpoint{1.708878in}{2.762372in}}%
\pgfpathclose%
\pgfusepath{stroke,fill}%
\end{pgfscope}%
\begin{pgfscope}%
\pgfpathrectangle{\pgfqpoint{0.786164in}{0.768110in}}{\pgfqpoint{8.851069in}{7.081890in}}%
\pgfusepath{clip}%
\pgfsetbuttcap%
\pgfsetroundjoin%
\definecolor{currentfill}{rgb}{0.278012,0.180367,0.486697}%
\pgfsetfillcolor{currentfill}%
\pgfsetfillopacity{0.700000}%
\pgfsetlinewidth{0.501875pt}%
\definecolor{currentstroke}{rgb}{1.000000,1.000000,1.000000}%
\pgfsetstrokecolor{currentstroke}%
\pgfsetstrokeopacity{0.700000}%
\pgfsetdash{}{0pt}%
\pgfpathmoveto{\pgfqpoint{2.022310in}{2.698345in}}%
\pgfpathcurveto{\pgfqpoint{2.035333in}{2.698345in}}{\pgfqpoint{2.047824in}{2.703519in}}{\pgfqpoint{2.057032in}{2.712727in}}%
\pgfpathcurveto{\pgfqpoint{2.066241in}{2.721936in}}{\pgfqpoint{2.071414in}{2.734427in}}{\pgfqpoint{2.071414in}{2.747450in}}%
\pgfpathcurveto{\pgfqpoint{2.071414in}{2.760472in}}{\pgfqpoint{2.066241in}{2.772964in}}{\pgfqpoint{2.057032in}{2.782172in}}%
\pgfpathcurveto{\pgfqpoint{2.047824in}{2.791380in}}{\pgfqpoint{2.035333in}{2.796554in}}{\pgfqpoint{2.022310in}{2.796554in}}%
\pgfpathcurveto{\pgfqpoint{2.009287in}{2.796554in}}{\pgfqpoint{1.996796in}{2.791380in}}{\pgfqpoint{1.987588in}{2.782172in}}%
\pgfpathcurveto{\pgfqpoint{1.978379in}{2.772964in}}{\pgfqpoint{1.973205in}{2.760472in}}{\pgfqpoint{1.973205in}{2.747450in}}%
\pgfpathcurveto{\pgfqpoint{1.973205in}{2.734427in}}{\pgfqpoint{1.978379in}{2.721936in}}{\pgfqpoint{1.987588in}{2.712727in}}%
\pgfpathcurveto{\pgfqpoint{1.996796in}{2.703519in}}{\pgfqpoint{2.009287in}{2.698345in}}{\pgfqpoint{2.022310in}{2.698345in}}%
\pgfpathlineto{\pgfqpoint{2.022310in}{2.698345in}}%
\pgfpathclose%
\pgfusepath{stroke,fill}%
\end{pgfscope}%
\begin{pgfscope}%
\pgfpathrectangle{\pgfqpoint{0.786164in}{0.768110in}}{\pgfqpoint{8.851069in}{7.081890in}}%
\pgfusepath{clip}%
\pgfsetbuttcap%
\pgfsetroundjoin%
\definecolor{currentfill}{rgb}{0.278012,0.180367,0.486697}%
\pgfsetfillcolor{currentfill}%
\pgfsetfillopacity{0.700000}%
\pgfsetlinewidth{0.501875pt}%
\definecolor{currentstroke}{rgb}{1.000000,1.000000,1.000000}%
\pgfsetstrokecolor{currentstroke}%
\pgfsetstrokeopacity{0.700000}%
\pgfsetdash{}{0pt}%
\pgfpathmoveto{\pgfqpoint{2.083604in}{2.719687in}}%
\pgfpathcurveto{\pgfqpoint{2.096627in}{2.719687in}}{\pgfqpoint{2.109118in}{2.724861in}}{\pgfqpoint{2.118327in}{2.734070in}}%
\pgfpathcurveto{\pgfqpoint{2.127535in}{2.743278in}}{\pgfqpoint{2.132709in}{2.755769in}}{\pgfqpoint{2.132709in}{2.768792in}}%
\pgfpathcurveto{\pgfqpoint{2.132709in}{2.781815in}}{\pgfqpoint{2.127535in}{2.794306in}}{\pgfqpoint{2.118327in}{2.803514in}}%
\pgfpathcurveto{\pgfqpoint{2.109118in}{2.812723in}}{\pgfqpoint{2.096627in}{2.817897in}}{\pgfqpoint{2.083604in}{2.817897in}}%
\pgfpathcurveto{\pgfqpoint{2.070582in}{2.817897in}}{\pgfqpoint{2.058091in}{2.812723in}}{\pgfqpoint{2.048882in}{2.803514in}}%
\pgfpathcurveto{\pgfqpoint{2.039674in}{2.794306in}}{\pgfqpoint{2.034500in}{2.781815in}}{\pgfqpoint{2.034500in}{2.768792in}}%
\pgfpathcurveto{\pgfqpoint{2.034500in}{2.755769in}}{\pgfqpoint{2.039674in}{2.743278in}}{\pgfqpoint{2.048882in}{2.734070in}}%
\pgfpathcurveto{\pgfqpoint{2.058091in}{2.724861in}}{\pgfqpoint{2.070582in}{2.719687in}}{\pgfqpoint{2.083604in}{2.719687in}}%
\pgfpathlineto{\pgfqpoint{2.083604in}{2.719687in}}%
\pgfpathclose%
\pgfusepath{stroke,fill}%
\end{pgfscope}%
\begin{pgfscope}%
\pgfpathrectangle{\pgfqpoint{0.786164in}{0.768110in}}{\pgfqpoint{8.851069in}{7.081890in}}%
\pgfusepath{clip}%
\pgfsetbuttcap%
\pgfsetroundjoin%
\definecolor{currentfill}{rgb}{0.279574,0.170599,0.479997}%
\pgfsetfillcolor{currentfill}%
\pgfsetfillopacity{0.700000}%
\pgfsetlinewidth{0.501875pt}%
\definecolor{currentstroke}{rgb}{1.000000,1.000000,1.000000}%
\pgfsetstrokecolor{currentstroke}%
\pgfsetstrokeopacity{0.700000}%
\pgfsetdash{}{0pt}%
\pgfpathmoveto{\pgfqpoint{2.223043in}{2.677003in}}%
\pgfpathcurveto{\pgfqpoint{2.236066in}{2.677003in}}{\pgfqpoint{2.248557in}{2.682177in}}{\pgfqpoint{2.257765in}{2.691385in}}%
\pgfpathcurveto{\pgfqpoint{2.266974in}{2.700594in}}{\pgfqpoint{2.272148in}{2.713085in}}{\pgfqpoint{2.272148in}{2.726108in}}%
\pgfpathcurveto{\pgfqpoint{2.272148in}{2.739130in}}{\pgfqpoint{2.266974in}{2.751621in}}{\pgfqpoint{2.257765in}{2.760830in}}%
\pgfpathcurveto{\pgfqpoint{2.248557in}{2.770038in}}{\pgfqpoint{2.236066in}{2.775212in}}{\pgfqpoint{2.223043in}{2.775212in}}%
\pgfpathcurveto{\pgfqpoint{2.210020in}{2.775212in}}{\pgfqpoint{2.197529in}{2.770038in}}{\pgfqpoint{2.188321in}{2.760830in}}%
\pgfpathcurveto{\pgfqpoint{2.179113in}{2.751621in}}{\pgfqpoint{2.173939in}{2.739130in}}{\pgfqpoint{2.173939in}{2.726108in}}%
\pgfpathcurveto{\pgfqpoint{2.173939in}{2.713085in}}{\pgfqpoint{2.179113in}{2.700594in}}{\pgfqpoint{2.188321in}{2.691385in}}%
\pgfpathcurveto{\pgfqpoint{2.197529in}{2.682177in}}{\pgfqpoint{2.210020in}{2.677003in}}{\pgfqpoint{2.223043in}{2.677003in}}%
\pgfpathlineto{\pgfqpoint{2.223043in}{2.677003in}}%
\pgfpathclose%
\pgfusepath{stroke,fill}%
\end{pgfscope}%
\begin{pgfscope}%
\pgfpathrectangle{\pgfqpoint{0.786164in}{0.768110in}}{\pgfqpoint{8.851069in}{7.081890in}}%
\pgfusepath{clip}%
\pgfsetbuttcap%
\pgfsetroundjoin%
\definecolor{currentfill}{rgb}{0.279574,0.170599,0.479997}%
\pgfsetfillcolor{currentfill}%
\pgfsetfillopacity{0.700000}%
\pgfsetlinewidth{0.501875pt}%
\definecolor{currentstroke}{rgb}{1.000000,1.000000,1.000000}%
\pgfsetstrokecolor{currentstroke}%
\pgfsetstrokeopacity{0.700000}%
\pgfsetdash{}{0pt}%
\pgfpathmoveto{\pgfqpoint{2.221700in}{2.591634in}}%
\pgfpathcurveto{\pgfqpoint{2.234723in}{2.591634in}}{\pgfqpoint{2.247214in}{2.596808in}}{\pgfqpoint{2.256422in}{2.606017in}}%
\pgfpathcurveto{\pgfqpoint{2.265631in}{2.615225in}}{\pgfqpoint{2.270805in}{2.627716in}}{\pgfqpoint{2.270805in}{2.640739in}}%
\pgfpathcurveto{\pgfqpoint{2.270805in}{2.653762in}}{\pgfqpoint{2.265631in}{2.666253in}}{\pgfqpoint{2.256422in}{2.675461in}}%
\pgfpathcurveto{\pgfqpoint{2.247214in}{2.684670in}}{\pgfqpoint{2.234723in}{2.689844in}}{\pgfqpoint{2.221700in}{2.689844in}}%
\pgfpathcurveto{\pgfqpoint{2.208677in}{2.689844in}}{\pgfqpoint{2.196186in}{2.684670in}}{\pgfqpoint{2.186978in}{2.675461in}}%
\pgfpathcurveto{\pgfqpoint{2.177769in}{2.666253in}}{\pgfqpoint{2.172595in}{2.653762in}}{\pgfqpoint{2.172595in}{2.640739in}}%
\pgfpathcurveto{\pgfqpoint{2.172595in}{2.627716in}}{\pgfqpoint{2.177769in}{2.615225in}}{\pgfqpoint{2.186978in}{2.606017in}}%
\pgfpathcurveto{\pgfqpoint{2.196186in}{2.596808in}}{\pgfqpoint{2.208677in}{2.591634in}}{\pgfqpoint{2.221700in}{2.591634in}}%
\pgfpathlineto{\pgfqpoint{2.221700in}{2.591634in}}%
\pgfpathclose%
\pgfusepath{stroke,fill}%
\end{pgfscope}%
\begin{pgfscope}%
\pgfpathrectangle{\pgfqpoint{0.786164in}{0.768110in}}{\pgfqpoint{8.851069in}{7.081890in}}%
\pgfusepath{clip}%
\pgfsetbuttcap%
\pgfsetroundjoin%
\definecolor{currentfill}{rgb}{0.281887,0.150881,0.465405}%
\pgfsetfillcolor{currentfill}%
\pgfsetfillopacity{0.700000}%
\pgfsetlinewidth{0.501875pt}%
\definecolor{currentstroke}{rgb}{1.000000,1.000000,1.000000}%
\pgfsetstrokecolor{currentstroke}%
\pgfsetstrokeopacity{0.700000}%
\pgfsetdash{}{0pt}%
\pgfpathmoveto{\pgfqpoint{2.601311in}{2.506266in}}%
\pgfpathcurveto{\pgfqpoint{2.614333in}{2.506266in}}{\pgfqpoint{2.626824in}{2.511440in}}{\pgfqpoint{2.636033in}{2.520648in}}%
\pgfpathcurveto{\pgfqpoint{2.645241in}{2.529856in}}{\pgfqpoint{2.650415in}{2.542347in}}{\pgfqpoint{2.650415in}{2.555370in}}%
\pgfpathcurveto{\pgfqpoint{2.650415in}{2.568393in}}{\pgfqpoint{2.645241in}{2.580884in}}{\pgfqpoint{2.636033in}{2.590092in}}%
\pgfpathcurveto{\pgfqpoint{2.626824in}{2.599301in}}{\pgfqpoint{2.614333in}{2.604475in}}{\pgfqpoint{2.601311in}{2.604475in}}%
\pgfpathcurveto{\pgfqpoint{2.588288in}{2.604475in}}{\pgfqpoint{2.575797in}{2.599301in}}{\pgfqpoint{2.566588in}{2.590092in}}%
\pgfpathcurveto{\pgfqpoint{2.557380in}{2.580884in}}{\pgfqpoint{2.552206in}{2.568393in}}{\pgfqpoint{2.552206in}{2.555370in}}%
\pgfpathcurveto{\pgfqpoint{2.552206in}{2.542347in}}{\pgfqpoint{2.557380in}{2.529856in}}{\pgfqpoint{2.566588in}{2.520648in}}%
\pgfpathcurveto{\pgfqpoint{2.575797in}{2.511440in}}{\pgfqpoint{2.588288in}{2.506266in}}{\pgfqpoint{2.601311in}{2.506266in}}%
\pgfpathlineto{\pgfqpoint{2.601311in}{2.506266in}}%
\pgfpathclose%
\pgfusepath{stroke,fill}%
\end{pgfscope}%
\begin{pgfscope}%
\pgfpathrectangle{\pgfqpoint{0.786164in}{0.768110in}}{\pgfqpoint{8.851069in}{7.081890in}}%
\pgfusepath{clip}%
\pgfsetbuttcap%
\pgfsetroundjoin%
\definecolor{currentfill}{rgb}{0.280868,0.160771,0.472899}%
\pgfsetfillcolor{currentfill}%
\pgfsetfillopacity{0.700000}%
\pgfsetlinewidth{0.501875pt}%
\definecolor{currentstroke}{rgb}{1.000000,1.000000,1.000000}%
\pgfsetstrokecolor{currentstroke}%
\pgfsetstrokeopacity{0.700000}%
\pgfsetdash{}{0pt}%
\pgfpathmoveto{\pgfqpoint{2.731836in}{2.506266in}}%
\pgfpathcurveto{\pgfqpoint{2.744859in}{2.506266in}}{\pgfqpoint{2.757350in}{2.511440in}}{\pgfqpoint{2.766558in}{2.520648in}}%
\pgfpathcurveto{\pgfqpoint{2.775767in}{2.529856in}}{\pgfqpoint{2.780941in}{2.542347in}}{\pgfqpoint{2.780941in}{2.555370in}}%
\pgfpathcurveto{\pgfqpoint{2.780941in}{2.568393in}}{\pgfqpoint{2.775767in}{2.580884in}}{\pgfqpoint{2.766558in}{2.590092in}}%
\pgfpathcurveto{\pgfqpoint{2.757350in}{2.599301in}}{\pgfqpoint{2.744859in}{2.604475in}}{\pgfqpoint{2.731836in}{2.604475in}}%
\pgfpathcurveto{\pgfqpoint{2.718813in}{2.604475in}}{\pgfqpoint{2.706322in}{2.599301in}}{\pgfqpoint{2.697114in}{2.590092in}}%
\pgfpathcurveto{\pgfqpoint{2.687905in}{2.580884in}}{\pgfqpoint{2.682732in}{2.568393in}}{\pgfqpoint{2.682732in}{2.555370in}}%
\pgfpathcurveto{\pgfqpoint{2.682732in}{2.542347in}}{\pgfqpoint{2.687905in}{2.529856in}}{\pgfqpoint{2.697114in}{2.520648in}}%
\pgfpathcurveto{\pgfqpoint{2.706322in}{2.511440in}}{\pgfqpoint{2.718813in}{2.506266in}}{\pgfqpoint{2.731836in}{2.506266in}}%
\pgfpathlineto{\pgfqpoint{2.731836in}{2.506266in}}%
\pgfpathclose%
\pgfusepath{stroke,fill}%
\end{pgfscope}%
\begin{pgfscope}%
\pgfpathrectangle{\pgfqpoint{0.786164in}{0.768110in}}{\pgfqpoint{8.851069in}{7.081890in}}%
\pgfusepath{clip}%
\pgfsetbuttcap%
\pgfsetroundjoin%
\definecolor{currentfill}{rgb}{0.280868,0.160771,0.472899}%
\pgfsetfillcolor{currentfill}%
\pgfsetfillopacity{0.700000}%
\pgfsetlinewidth{0.501875pt}%
\definecolor{currentstroke}{rgb}{1.000000,1.000000,1.000000}%
\pgfsetstrokecolor{currentstroke}%
\pgfsetstrokeopacity{0.700000}%
\pgfsetdash{}{0pt}%
\pgfpathmoveto{\pgfqpoint{2.882020in}{2.506266in}}%
\pgfpathcurveto{\pgfqpoint{2.895043in}{2.506266in}}{\pgfqpoint{2.907534in}{2.511440in}}{\pgfqpoint{2.916742in}{2.520648in}}%
\pgfpathcurveto{\pgfqpoint{2.925950in}{2.529856in}}{\pgfqpoint{2.931124in}{2.542347in}}{\pgfqpoint{2.931124in}{2.555370in}}%
\pgfpathcurveto{\pgfqpoint{2.931124in}{2.568393in}}{\pgfqpoint{2.925950in}{2.580884in}}{\pgfqpoint{2.916742in}{2.590092in}}%
\pgfpathcurveto{\pgfqpoint{2.907534in}{2.599301in}}{\pgfqpoint{2.895043in}{2.604475in}}{\pgfqpoint{2.882020in}{2.604475in}}%
\pgfpathcurveto{\pgfqpoint{2.868997in}{2.604475in}}{\pgfqpoint{2.856506in}{2.599301in}}{\pgfqpoint{2.847298in}{2.590092in}}%
\pgfpathcurveto{\pgfqpoint{2.838089in}{2.580884in}}{\pgfqpoint{2.832915in}{2.568393in}}{\pgfqpoint{2.832915in}{2.555370in}}%
\pgfpathcurveto{\pgfqpoint{2.832915in}{2.542347in}}{\pgfqpoint{2.838089in}{2.529856in}}{\pgfqpoint{2.847298in}{2.520648in}}%
\pgfpathcurveto{\pgfqpoint{2.856506in}{2.511440in}}{\pgfqpoint{2.868997in}{2.506266in}}{\pgfqpoint{2.882020in}{2.506266in}}%
\pgfpathlineto{\pgfqpoint{2.882020in}{2.506266in}}%
\pgfpathclose%
\pgfusepath{stroke,fill}%
\end{pgfscope}%
\begin{pgfscope}%
\pgfpathrectangle{\pgfqpoint{0.786164in}{0.768110in}}{\pgfqpoint{8.851069in}{7.081890in}}%
\pgfusepath{clip}%
\pgfsetbuttcap%
\pgfsetroundjoin%
\definecolor{currentfill}{rgb}{0.282623,0.140926,0.457517}%
\pgfsetfillcolor{currentfill}%
\pgfsetfillopacity{0.700000}%
\pgfsetlinewidth{0.501875pt}%
\definecolor{currentstroke}{rgb}{1.000000,1.000000,1.000000}%
\pgfsetstrokecolor{currentstroke}%
\pgfsetstrokeopacity{0.700000}%
\pgfsetdash{}{0pt}%
\pgfpathmoveto{\pgfqpoint{3.072252in}{2.420897in}}%
\pgfpathcurveto{\pgfqpoint{3.085275in}{2.420897in}}{\pgfqpoint{3.097766in}{2.426071in}}{\pgfqpoint{3.106975in}{2.435279in}}%
\pgfpathcurveto{\pgfqpoint{3.116183in}{2.444488in}}{\pgfqpoint{3.121357in}{2.456979in}}{\pgfqpoint{3.121357in}{2.470002in}}%
\pgfpathcurveto{\pgfqpoint{3.121357in}{2.483024in}}{\pgfqpoint{3.116183in}{2.495515in}}{\pgfqpoint{3.106975in}{2.504724in}}%
\pgfpathcurveto{\pgfqpoint{3.097766in}{2.513932in}}{\pgfqpoint{3.085275in}{2.519106in}}{\pgfqpoint{3.072252in}{2.519106in}}%
\pgfpathcurveto{\pgfqpoint{3.059230in}{2.519106in}}{\pgfqpoint{3.046739in}{2.513932in}}{\pgfqpoint{3.037530in}{2.504724in}}%
\pgfpathcurveto{\pgfqpoint{3.028322in}{2.495515in}}{\pgfqpoint{3.023148in}{2.483024in}}{\pgfqpoint{3.023148in}{2.470002in}}%
\pgfpathcurveto{\pgfqpoint{3.023148in}{2.456979in}}{\pgfqpoint{3.028322in}{2.444488in}}{\pgfqpoint{3.037530in}{2.435279in}}%
\pgfpathcurveto{\pgfqpoint{3.046739in}{2.426071in}}{\pgfqpoint{3.059230in}{2.420897in}}{\pgfqpoint{3.072252in}{2.420897in}}%
\pgfpathlineto{\pgfqpoint{3.072252in}{2.420897in}}%
\pgfpathclose%
\pgfusepath{stroke,fill}%
\end{pgfscope}%
\begin{pgfscope}%
\pgfpathrectangle{\pgfqpoint{0.786164in}{0.768110in}}{\pgfqpoint{8.851069in}{7.081890in}}%
\pgfusepath{clip}%
\pgfsetbuttcap%
\pgfsetroundjoin%
\definecolor{currentfill}{rgb}{0.282884,0.135920,0.453427}%
\pgfsetfillcolor{currentfill}%
\pgfsetfillopacity{0.700000}%
\pgfsetlinewidth{0.501875pt}%
\definecolor{currentstroke}{rgb}{1.000000,1.000000,1.000000}%
\pgfsetstrokecolor{currentstroke}%
\pgfsetstrokeopacity{0.700000}%
\pgfsetdash{}{0pt}%
\pgfpathmoveto{\pgfqpoint{3.154670in}{2.378213in}}%
\pgfpathcurveto{\pgfqpoint{3.167693in}{2.378213in}}{\pgfqpoint{3.180184in}{2.383386in}}{\pgfqpoint{3.189393in}{2.392595in}}%
\pgfpathcurveto{\pgfqpoint{3.198601in}{2.401803in}}{\pgfqpoint{3.203775in}{2.414294in}}{\pgfqpoint{3.203775in}{2.427317in}}%
\pgfpathcurveto{\pgfqpoint{3.203775in}{2.440340in}}{\pgfqpoint{3.198601in}{2.452831in}}{\pgfqpoint{3.189393in}{2.462039in}}%
\pgfpathcurveto{\pgfqpoint{3.180184in}{2.471248in}}{\pgfqpoint{3.167693in}{2.476422in}}{\pgfqpoint{3.154670in}{2.476422in}}%
\pgfpathcurveto{\pgfqpoint{3.141648in}{2.476422in}}{\pgfqpoint{3.129157in}{2.471248in}}{\pgfqpoint{3.119948in}{2.462039in}}%
\pgfpathcurveto{\pgfqpoint{3.110740in}{2.452831in}}{\pgfqpoint{3.105566in}{2.440340in}}{\pgfqpoint{3.105566in}{2.427317in}}%
\pgfpathcurveto{\pgfqpoint{3.105566in}{2.414294in}}{\pgfqpoint{3.110740in}{2.401803in}}{\pgfqpoint{3.119948in}{2.392595in}}%
\pgfpathcurveto{\pgfqpoint{3.129157in}{2.383386in}}{\pgfqpoint{3.141648in}{2.378213in}}{\pgfqpoint{3.154670in}{2.378213in}}%
\pgfpathlineto{\pgfqpoint{3.154670in}{2.378213in}}%
\pgfpathclose%
\pgfusepath{stroke,fill}%
\end{pgfscope}%
\begin{pgfscope}%
\pgfpathrectangle{\pgfqpoint{0.786164in}{0.768110in}}{\pgfqpoint{8.851069in}{7.081890in}}%
\pgfusepath{clip}%
\pgfsetbuttcap%
\pgfsetroundjoin%
\definecolor{currentfill}{rgb}{0.283187,0.125848,0.444960}%
\pgfsetfillcolor{currentfill}%
\pgfsetfillopacity{0.700000}%
\pgfsetlinewidth{0.501875pt}%
\definecolor{currentstroke}{rgb}{1.000000,1.000000,1.000000}%
\pgfsetstrokecolor{currentstroke}%
\pgfsetstrokeopacity{0.700000}%
\pgfsetdash{}{0pt}%
\pgfpathmoveto{\pgfqpoint{2.960897in}{2.356870in}}%
\pgfpathcurveto{\pgfqpoint{2.973919in}{2.356870in}}{\pgfqpoint{2.986411in}{2.362044in}}{\pgfqpoint{2.995619in}{2.371253in}}%
\pgfpathcurveto{\pgfqpoint{3.004827in}{2.380461in}}{\pgfqpoint{3.010001in}{2.392952in}}{\pgfqpoint{3.010001in}{2.405975in}}%
\pgfpathcurveto{\pgfqpoint{3.010001in}{2.418998in}}{\pgfqpoint{3.004827in}{2.431489in}}{\pgfqpoint{2.995619in}{2.440697in}}%
\pgfpathcurveto{\pgfqpoint{2.986411in}{2.449906in}}{\pgfqpoint{2.973919in}{2.455080in}}{\pgfqpoint{2.960897in}{2.455080in}}%
\pgfpathcurveto{\pgfqpoint{2.947874in}{2.455080in}}{\pgfqpoint{2.935383in}{2.449906in}}{\pgfqpoint{2.926175in}{2.440697in}}%
\pgfpathcurveto{\pgfqpoint{2.916966in}{2.431489in}}{\pgfqpoint{2.911792in}{2.418998in}}{\pgfqpoint{2.911792in}{2.405975in}}%
\pgfpathcurveto{\pgfqpoint{2.911792in}{2.392952in}}{\pgfqpoint{2.916966in}{2.380461in}}{\pgfqpoint{2.926175in}{2.371253in}}%
\pgfpathcurveto{\pgfqpoint{2.935383in}{2.362044in}}{\pgfqpoint{2.947874in}{2.356870in}}{\pgfqpoint{2.960897in}{2.356870in}}%
\pgfpathlineto{\pgfqpoint{2.960897in}{2.356870in}}%
\pgfpathclose%
\pgfusepath{stroke,fill}%
\end{pgfscope}%
\begin{pgfscope}%
\pgfpathrectangle{\pgfqpoint{0.786164in}{0.768110in}}{\pgfqpoint{8.851069in}{7.081890in}}%
\pgfusepath{clip}%
\pgfsetbuttcap%
\pgfsetroundjoin%
\definecolor{currentfill}{rgb}{0.281887,0.150881,0.465405}%
\pgfsetfillcolor{currentfill}%
\pgfsetfillopacity{0.700000}%
\pgfsetlinewidth{0.501875pt}%
\definecolor{currentstroke}{rgb}{1.000000,1.000000,1.000000}%
\pgfsetstrokecolor{currentstroke}%
\pgfsetstrokeopacity{0.700000}%
\pgfsetdash{}{0pt}%
\pgfpathmoveto{\pgfqpoint{2.945878in}{2.420897in}}%
\pgfpathcurveto{\pgfqpoint{2.958901in}{2.420897in}}{\pgfqpoint{2.971392in}{2.426071in}}{\pgfqpoint{2.980601in}{2.435279in}}%
\pgfpathcurveto{\pgfqpoint{2.989809in}{2.444488in}}{\pgfqpoint{2.994983in}{2.456979in}}{\pgfqpoint{2.994983in}{2.470002in}}%
\pgfpathcurveto{\pgfqpoint{2.994983in}{2.483024in}}{\pgfqpoint{2.989809in}{2.495515in}}{\pgfqpoint{2.980601in}{2.504724in}}%
\pgfpathcurveto{\pgfqpoint{2.971392in}{2.513932in}}{\pgfqpoint{2.958901in}{2.519106in}}{\pgfqpoint{2.945878in}{2.519106in}}%
\pgfpathcurveto{\pgfqpoint{2.932856in}{2.519106in}}{\pgfqpoint{2.920365in}{2.513932in}}{\pgfqpoint{2.911156in}{2.504724in}}%
\pgfpathcurveto{\pgfqpoint{2.901948in}{2.495515in}}{\pgfqpoint{2.896774in}{2.483024in}}{\pgfqpoint{2.896774in}{2.470002in}}%
\pgfpathcurveto{\pgfqpoint{2.896774in}{2.456979in}}{\pgfqpoint{2.901948in}{2.444488in}}{\pgfqpoint{2.911156in}{2.435279in}}%
\pgfpathcurveto{\pgfqpoint{2.920365in}{2.426071in}}{\pgfqpoint{2.932856in}{2.420897in}}{\pgfqpoint{2.945878in}{2.420897in}}%
\pgfpathlineto{\pgfqpoint{2.945878in}{2.420897in}}%
\pgfpathclose%
\pgfusepath{stroke,fill}%
\end{pgfscope}%
\begin{pgfscope}%
\pgfpathrectangle{\pgfqpoint{0.786164in}{0.768110in}}{\pgfqpoint{8.851069in}{7.081890in}}%
\pgfusepath{clip}%
\pgfsetbuttcap%
\pgfsetroundjoin%
\definecolor{currentfill}{rgb}{0.281412,0.155834,0.469201}%
\pgfsetfillcolor{currentfill}%
\pgfsetfillopacity{0.700000}%
\pgfsetlinewidth{0.501875pt}%
\definecolor{currentstroke}{rgb}{1.000000,1.000000,1.000000}%
\pgfsetstrokecolor{currentstroke}%
\pgfsetstrokeopacity{0.700000}%
\pgfsetdash{}{0pt}%
\pgfpathmoveto{\pgfqpoint{2.931471in}{2.463581in}}%
\pgfpathcurveto{\pgfqpoint{2.944493in}{2.463581in}}{\pgfqpoint{2.956984in}{2.468755in}}{\pgfqpoint{2.966193in}{2.477964in}}%
\pgfpathcurveto{\pgfqpoint{2.975401in}{2.487172in}}{\pgfqpoint{2.980575in}{2.499663in}}{\pgfqpoint{2.980575in}{2.512686in}}%
\pgfpathcurveto{\pgfqpoint{2.980575in}{2.525709in}}{\pgfqpoint{2.975401in}{2.538200in}}{\pgfqpoint{2.966193in}{2.547408in}}%
\pgfpathcurveto{\pgfqpoint{2.956984in}{2.556617in}}{\pgfqpoint{2.944493in}{2.561790in}}{\pgfqpoint{2.931471in}{2.561790in}}%
\pgfpathcurveto{\pgfqpoint{2.918448in}{2.561790in}}{\pgfqpoint{2.905957in}{2.556617in}}{\pgfqpoint{2.896748in}{2.547408in}}%
\pgfpathcurveto{\pgfqpoint{2.887540in}{2.538200in}}{\pgfqpoint{2.882366in}{2.525709in}}{\pgfqpoint{2.882366in}{2.512686in}}%
\pgfpathcurveto{\pgfqpoint{2.882366in}{2.499663in}}{\pgfqpoint{2.887540in}{2.487172in}}{\pgfqpoint{2.896748in}{2.477964in}}%
\pgfpathcurveto{\pgfqpoint{2.905957in}{2.468755in}}{\pgfqpoint{2.918448in}{2.463581in}}{\pgfqpoint{2.931471in}{2.463581in}}%
\pgfpathlineto{\pgfqpoint{2.931471in}{2.463581in}}%
\pgfpathclose%
\pgfusepath{stroke,fill}%
\end{pgfscope}%
\begin{pgfscope}%
\pgfpathrectangle{\pgfqpoint{0.786164in}{0.768110in}}{\pgfqpoint{8.851069in}{7.081890in}}%
\pgfusepath{clip}%
\pgfsetbuttcap%
\pgfsetroundjoin%
\definecolor{currentfill}{rgb}{0.280255,0.165693,0.476498}%
\pgfsetfillcolor{currentfill}%
\pgfsetfillopacity{0.700000}%
\pgfsetlinewidth{0.501875pt}%
\definecolor{currentstroke}{rgb}{1.000000,1.000000,1.000000}%
\pgfsetstrokecolor{currentstroke}%
\pgfsetstrokeopacity{0.700000}%
\pgfsetdash{}{0pt}%
\pgfpathmoveto{\pgfqpoint{2.831226in}{2.506266in}}%
\pgfpathcurveto{\pgfqpoint{2.844249in}{2.506266in}}{\pgfqpoint{2.856740in}{2.511440in}}{\pgfqpoint{2.865948in}{2.520648in}}%
\pgfpathcurveto{\pgfqpoint{2.875157in}{2.529856in}}{\pgfqpoint{2.880331in}{2.542347in}}{\pgfqpoint{2.880331in}{2.555370in}}%
\pgfpathcurveto{\pgfqpoint{2.880331in}{2.568393in}}{\pgfqpoint{2.875157in}{2.580884in}}{\pgfqpoint{2.865948in}{2.590092in}}%
\pgfpathcurveto{\pgfqpoint{2.856740in}{2.599301in}}{\pgfqpoint{2.844249in}{2.604475in}}{\pgfqpoint{2.831226in}{2.604475in}}%
\pgfpathcurveto{\pgfqpoint{2.818203in}{2.604475in}}{\pgfqpoint{2.805712in}{2.599301in}}{\pgfqpoint{2.796504in}{2.590092in}}%
\pgfpathcurveto{\pgfqpoint{2.787295in}{2.580884in}}{\pgfqpoint{2.782121in}{2.568393in}}{\pgfqpoint{2.782121in}{2.555370in}}%
\pgfpathcurveto{\pgfqpoint{2.782121in}{2.542347in}}{\pgfqpoint{2.787295in}{2.529856in}}{\pgfqpoint{2.796504in}{2.520648in}}%
\pgfpathcurveto{\pgfqpoint{2.805712in}{2.511440in}}{\pgfqpoint{2.818203in}{2.506266in}}{\pgfqpoint{2.831226in}{2.506266in}}%
\pgfpathlineto{\pgfqpoint{2.831226in}{2.506266in}}%
\pgfpathclose%
\pgfusepath{stroke,fill}%
\end{pgfscope}%
\begin{pgfscope}%
\pgfpathrectangle{\pgfqpoint{0.786164in}{0.768110in}}{\pgfqpoint{8.851069in}{7.081890in}}%
\pgfusepath{clip}%
\pgfsetbuttcap%
\pgfsetroundjoin%
\definecolor{currentfill}{rgb}{0.280255,0.165693,0.476498}%
\pgfsetfillcolor{currentfill}%
\pgfsetfillopacity{0.700000}%
\pgfsetlinewidth{0.501875pt}%
\definecolor{currentstroke}{rgb}{1.000000,1.000000,1.000000}%
\pgfsetstrokecolor{currentstroke}%
\pgfsetstrokeopacity{0.700000}%
\pgfsetdash{}{0pt}%
\pgfpathmoveto{\pgfqpoint{2.708149in}{2.506266in}}%
\pgfpathcurveto{\pgfqpoint{2.721171in}{2.506266in}}{\pgfqpoint{2.733662in}{2.511440in}}{\pgfqpoint{2.742871in}{2.520648in}}%
\pgfpathcurveto{\pgfqpoint{2.752079in}{2.529856in}}{\pgfqpoint{2.757253in}{2.542347in}}{\pgfqpoint{2.757253in}{2.555370in}}%
\pgfpathcurveto{\pgfqpoint{2.757253in}{2.568393in}}{\pgfqpoint{2.752079in}{2.580884in}}{\pgfqpoint{2.742871in}{2.590092in}}%
\pgfpathcurveto{\pgfqpoint{2.733662in}{2.599301in}}{\pgfqpoint{2.721171in}{2.604475in}}{\pgfqpoint{2.708149in}{2.604475in}}%
\pgfpathcurveto{\pgfqpoint{2.695126in}{2.604475in}}{\pgfqpoint{2.682635in}{2.599301in}}{\pgfqpoint{2.673426in}{2.590092in}}%
\pgfpathcurveto{\pgfqpoint{2.664218in}{2.580884in}}{\pgfqpoint{2.659044in}{2.568393in}}{\pgfqpoint{2.659044in}{2.555370in}}%
\pgfpathcurveto{\pgfqpoint{2.659044in}{2.542347in}}{\pgfqpoint{2.664218in}{2.529856in}}{\pgfqpoint{2.673426in}{2.520648in}}%
\pgfpathcurveto{\pgfqpoint{2.682635in}{2.511440in}}{\pgfqpoint{2.695126in}{2.506266in}}{\pgfqpoint{2.708149in}{2.506266in}}%
\pgfpathlineto{\pgfqpoint{2.708149in}{2.506266in}}%
\pgfpathclose%
\pgfusepath{stroke,fill}%
\end{pgfscope}%
\begin{pgfscope}%
\pgfpathrectangle{\pgfqpoint{0.786164in}{0.768110in}}{\pgfqpoint{8.851069in}{7.081890in}}%
\pgfusepath{clip}%
\pgfsetbuttcap%
\pgfsetroundjoin%
\definecolor{currentfill}{rgb}{0.279574,0.170599,0.479997}%
\pgfsetfillcolor{currentfill}%
\pgfsetfillopacity{0.700000}%
\pgfsetlinewidth{0.501875pt}%
\definecolor{currentstroke}{rgb}{1.000000,1.000000,1.000000}%
\pgfsetstrokecolor{currentstroke}%
\pgfsetstrokeopacity{0.700000}%
\pgfsetdash{}{0pt}%
\pgfpathmoveto{\pgfqpoint{2.718649in}{2.506266in}}%
\pgfpathcurveto{\pgfqpoint{2.731672in}{2.506266in}}{\pgfqpoint{2.744163in}{2.511440in}}{\pgfqpoint{2.753372in}{2.520648in}}%
\pgfpathcurveto{\pgfqpoint{2.762580in}{2.529856in}}{\pgfqpoint{2.767754in}{2.542347in}}{\pgfqpoint{2.767754in}{2.555370in}}%
\pgfpathcurveto{\pgfqpoint{2.767754in}{2.568393in}}{\pgfqpoint{2.762580in}{2.580884in}}{\pgfqpoint{2.753372in}{2.590092in}}%
\pgfpathcurveto{\pgfqpoint{2.744163in}{2.599301in}}{\pgfqpoint{2.731672in}{2.604475in}}{\pgfqpoint{2.718649in}{2.604475in}}%
\pgfpathcurveto{\pgfqpoint{2.705627in}{2.604475in}}{\pgfqpoint{2.693136in}{2.599301in}}{\pgfqpoint{2.683927in}{2.590092in}}%
\pgfpathcurveto{\pgfqpoint{2.674719in}{2.580884in}}{\pgfqpoint{2.669545in}{2.568393in}}{\pgfqpoint{2.669545in}{2.555370in}}%
\pgfpathcurveto{\pgfqpoint{2.669545in}{2.542347in}}{\pgfqpoint{2.674719in}{2.529856in}}{\pgfqpoint{2.683927in}{2.520648in}}%
\pgfpathcurveto{\pgfqpoint{2.693136in}{2.511440in}}{\pgfqpoint{2.705627in}{2.506266in}}{\pgfqpoint{2.718649in}{2.506266in}}%
\pgfpathlineto{\pgfqpoint{2.718649in}{2.506266in}}%
\pgfpathclose%
\pgfusepath{stroke,fill}%
\end{pgfscope}%
\begin{pgfscope}%
\pgfpathrectangle{\pgfqpoint{0.786164in}{0.768110in}}{\pgfqpoint{8.851069in}{7.081890in}}%
\pgfusepath{clip}%
\pgfsetbuttcap%
\pgfsetroundjoin%
\definecolor{currentfill}{rgb}{0.280868,0.160771,0.472899}%
\pgfsetfillcolor{currentfill}%
\pgfsetfillopacity{0.700000}%
\pgfsetlinewidth{0.501875pt}%
\definecolor{currentstroke}{rgb}{1.000000,1.000000,1.000000}%
\pgfsetstrokecolor{currentstroke}%
\pgfsetstrokeopacity{0.700000}%
\pgfsetdash{}{0pt}%
\pgfpathmoveto{\pgfqpoint{2.867124in}{2.399555in}}%
\pgfpathcurveto{\pgfqpoint{2.880146in}{2.399555in}}{\pgfqpoint{2.892637in}{2.404729in}}{\pgfqpoint{2.901846in}{2.413937in}}%
\pgfpathcurveto{\pgfqpoint{2.911054in}{2.423146in}}{\pgfqpoint{2.916228in}{2.435637in}}{\pgfqpoint{2.916228in}{2.448659in}}%
\pgfpathcurveto{\pgfqpoint{2.916228in}{2.461682in}}{\pgfqpoint{2.911054in}{2.474173in}}{\pgfqpoint{2.901846in}{2.483382in}}%
\pgfpathcurveto{\pgfqpoint{2.892637in}{2.492590in}}{\pgfqpoint{2.880146in}{2.497764in}}{\pgfqpoint{2.867124in}{2.497764in}}%
\pgfpathcurveto{\pgfqpoint{2.854101in}{2.497764in}}{\pgfqpoint{2.841610in}{2.492590in}}{\pgfqpoint{2.832401in}{2.483382in}}%
\pgfpathcurveto{\pgfqpoint{2.823193in}{2.474173in}}{\pgfqpoint{2.818019in}{2.461682in}}{\pgfqpoint{2.818019in}{2.448659in}}%
\pgfpathcurveto{\pgfqpoint{2.818019in}{2.435637in}}{\pgfqpoint{2.823193in}{2.423146in}}{\pgfqpoint{2.832401in}{2.413937in}}%
\pgfpathcurveto{\pgfqpoint{2.841610in}{2.404729in}}{\pgfqpoint{2.854101in}{2.399555in}}{\pgfqpoint{2.867124in}{2.399555in}}%
\pgfpathlineto{\pgfqpoint{2.867124in}{2.399555in}}%
\pgfpathclose%
\pgfusepath{stroke,fill}%
\end{pgfscope}%
\begin{pgfscope}%
\pgfpathrectangle{\pgfqpoint{0.786164in}{0.768110in}}{\pgfqpoint{8.851069in}{7.081890in}}%
\pgfusepath{clip}%
\pgfsetbuttcap%
\pgfsetroundjoin%
\definecolor{currentfill}{rgb}{0.278826,0.175490,0.483397}%
\pgfsetfillcolor{currentfill}%
\pgfsetfillopacity{0.700000}%
\pgfsetlinewidth{0.501875pt}%
\definecolor{currentstroke}{rgb}{1.000000,1.000000,1.000000}%
\pgfsetstrokecolor{currentstroke}%
\pgfsetstrokeopacity{0.700000}%
\pgfsetdash{}{0pt}%
\pgfpathmoveto{\pgfqpoint{2.901800in}{2.442239in}}%
\pgfpathcurveto{\pgfqpoint{2.914823in}{2.442239in}}{\pgfqpoint{2.927314in}{2.447413in}}{\pgfqpoint{2.936522in}{2.456621in}}%
\pgfpathcurveto{\pgfqpoint{2.945731in}{2.465830in}}{\pgfqpoint{2.950905in}{2.478321in}}{\pgfqpoint{2.950905in}{2.491344in}}%
\pgfpathcurveto{\pgfqpoint{2.950905in}{2.504366in}}{\pgfqpoint{2.945731in}{2.516857in}}{\pgfqpoint{2.936522in}{2.526066in}}%
\pgfpathcurveto{\pgfqpoint{2.927314in}{2.535274in}}{\pgfqpoint{2.914823in}{2.540448in}}{\pgfqpoint{2.901800in}{2.540448in}}%
\pgfpathcurveto{\pgfqpoint{2.888777in}{2.540448in}}{\pgfqpoint{2.876286in}{2.535274in}}{\pgfqpoint{2.867078in}{2.526066in}}%
\pgfpathcurveto{\pgfqpoint{2.857869in}{2.516857in}}{\pgfqpoint{2.852695in}{2.504366in}}{\pgfqpoint{2.852695in}{2.491344in}}%
\pgfpathcurveto{\pgfqpoint{2.852695in}{2.478321in}}{\pgfqpoint{2.857869in}{2.465830in}}{\pgfqpoint{2.867078in}{2.456621in}}%
\pgfpathcurveto{\pgfqpoint{2.876286in}{2.447413in}}{\pgfqpoint{2.888777in}{2.442239in}}{\pgfqpoint{2.901800in}{2.442239in}}%
\pgfpathlineto{\pgfqpoint{2.901800in}{2.442239in}}%
\pgfpathclose%
\pgfusepath{stroke,fill}%
\end{pgfscope}%
\begin{pgfscope}%
\pgfpathrectangle{\pgfqpoint{0.786164in}{0.768110in}}{\pgfqpoint{8.851069in}{7.081890in}}%
\pgfusepath{clip}%
\pgfsetbuttcap%
\pgfsetroundjoin%
\definecolor{currentfill}{rgb}{0.280255,0.165693,0.476498}%
\pgfsetfillcolor{currentfill}%
\pgfsetfillopacity{0.700000}%
\pgfsetlinewidth{0.501875pt}%
\definecolor{currentstroke}{rgb}{1.000000,1.000000,1.000000}%
\pgfsetstrokecolor{currentstroke}%
\pgfsetstrokeopacity{0.700000}%
\pgfsetdash{}{0pt}%
\pgfpathmoveto{\pgfqpoint{3.030738in}{2.378213in}}%
\pgfpathcurveto{\pgfqpoint{3.043761in}{2.378213in}}{\pgfqpoint{3.056252in}{2.383386in}}{\pgfqpoint{3.065461in}{2.392595in}}%
\pgfpathcurveto{\pgfqpoint{3.074669in}{2.401803in}}{\pgfqpoint{3.079843in}{2.414294in}}{\pgfqpoint{3.079843in}{2.427317in}}%
\pgfpathcurveto{\pgfqpoint{3.079843in}{2.440340in}}{\pgfqpoint{3.074669in}{2.452831in}}{\pgfqpoint{3.065461in}{2.462039in}}%
\pgfpathcurveto{\pgfqpoint{3.056252in}{2.471248in}}{\pgfqpoint{3.043761in}{2.476422in}}{\pgfqpoint{3.030738in}{2.476422in}}%
\pgfpathcurveto{\pgfqpoint{3.017716in}{2.476422in}}{\pgfqpoint{3.005225in}{2.471248in}}{\pgfqpoint{2.996016in}{2.462039in}}%
\pgfpathcurveto{\pgfqpoint{2.986808in}{2.452831in}}{\pgfqpoint{2.981634in}{2.440340in}}{\pgfqpoint{2.981634in}{2.427317in}}%
\pgfpathcurveto{\pgfqpoint{2.981634in}{2.414294in}}{\pgfqpoint{2.986808in}{2.401803in}}{\pgfqpoint{2.996016in}{2.392595in}}%
\pgfpathcurveto{\pgfqpoint{3.005225in}{2.383386in}}{\pgfqpoint{3.017716in}{2.378213in}}{\pgfqpoint{3.030738in}{2.378213in}}%
\pgfpathlineto{\pgfqpoint{3.030738in}{2.378213in}}%
\pgfpathclose%
\pgfusepath{stroke,fill}%
\end{pgfscope}%
\begin{pgfscope}%
\pgfpathrectangle{\pgfqpoint{0.786164in}{0.768110in}}{\pgfqpoint{8.851069in}{7.081890in}}%
\pgfusepath{clip}%
\pgfsetbuttcap%
\pgfsetroundjoin%
\definecolor{currentfill}{rgb}{0.231674,0.318106,0.544834}%
\pgfsetfillcolor{currentfill}%
\pgfsetfillopacity{0.700000}%
\pgfsetlinewidth{0.501875pt}%
\definecolor{currentstroke}{rgb}{1.000000,1.000000,1.000000}%
\pgfsetstrokecolor{currentstroke}%
\pgfsetstrokeopacity{0.700000}%
\pgfsetdash{}{0pt}%
\pgfpathmoveto{\pgfqpoint{1.466386in}{5.259406in}}%
\pgfpathcurveto{\pgfqpoint{1.479409in}{5.259406in}}{\pgfqpoint{1.491900in}{5.264580in}}{\pgfqpoint{1.501108in}{5.273788in}}%
\pgfpathcurveto{\pgfqpoint{1.510317in}{5.282996in}}{\pgfqpoint{1.515491in}{5.295487in}}{\pgfqpoint{1.515491in}{5.308510in}}%
\pgfpathcurveto{\pgfqpoint{1.515491in}{5.321533in}}{\pgfqpoint{1.510317in}{5.334024in}}{\pgfqpoint{1.501108in}{5.343232in}}%
\pgfpathcurveto{\pgfqpoint{1.491900in}{5.352441in}}{\pgfqpoint{1.479409in}{5.357615in}}{\pgfqpoint{1.466386in}{5.357615in}}%
\pgfpathcurveto{\pgfqpoint{1.453363in}{5.357615in}}{\pgfqpoint{1.440872in}{5.352441in}}{\pgfqpoint{1.431664in}{5.343232in}}%
\pgfpathcurveto{\pgfqpoint{1.422455in}{5.334024in}}{\pgfqpoint{1.417281in}{5.321533in}}{\pgfqpoint{1.417281in}{5.308510in}}%
\pgfpathcurveto{\pgfqpoint{1.417281in}{5.295487in}}{\pgfqpoint{1.422455in}{5.282996in}}{\pgfqpoint{1.431664in}{5.273788in}}%
\pgfpathcurveto{\pgfqpoint{1.440872in}{5.264580in}}{\pgfqpoint{1.453363in}{5.259406in}}{\pgfqpoint{1.466386in}{5.259406in}}%
\pgfpathlineto{\pgfqpoint{1.466386in}{5.259406in}}%
\pgfpathclose%
\pgfusepath{stroke,fill}%
\end{pgfscope}%
\begin{pgfscope}%
\pgfpathrectangle{\pgfqpoint{0.786164in}{0.768110in}}{\pgfqpoint{8.851069in}{7.081890in}}%
\pgfusepath{clip}%
\pgfsetbuttcap%
\pgfsetroundjoin%
\definecolor{currentfill}{rgb}{0.233603,0.313828,0.543914}%
\pgfsetfillcolor{currentfill}%
\pgfsetfillopacity{0.700000}%
\pgfsetlinewidth{0.501875pt}%
\definecolor{currentstroke}{rgb}{1.000000,1.000000,1.000000}%
\pgfsetstrokecolor{currentstroke}%
\pgfsetstrokeopacity{0.700000}%
\pgfsetdash{}{0pt}%
\pgfpathmoveto{\pgfqpoint{1.520599in}{5.280748in}}%
\pgfpathcurveto{\pgfqpoint{1.533621in}{5.280748in}}{\pgfqpoint{1.546112in}{5.285922in}}{\pgfqpoint{1.555321in}{5.295130in}}%
\pgfpathcurveto{\pgfqpoint{1.564529in}{5.304339in}}{\pgfqpoint{1.569703in}{5.316830in}}{\pgfqpoint{1.569703in}{5.329852in}}%
\pgfpathcurveto{\pgfqpoint{1.569703in}{5.342875in}}{\pgfqpoint{1.564529in}{5.355366in}}{\pgfqpoint{1.555321in}{5.364575in}}%
\pgfpathcurveto{\pgfqpoint{1.546112in}{5.373783in}}{\pgfqpoint{1.533621in}{5.378957in}}{\pgfqpoint{1.520599in}{5.378957in}}%
\pgfpathcurveto{\pgfqpoint{1.507576in}{5.378957in}}{\pgfqpoint{1.495085in}{5.373783in}}{\pgfqpoint{1.485876in}{5.364575in}}%
\pgfpathcurveto{\pgfqpoint{1.476668in}{5.355366in}}{\pgfqpoint{1.471494in}{5.342875in}}{\pgfqpoint{1.471494in}{5.329852in}}%
\pgfpathcurveto{\pgfqpoint{1.471494in}{5.316830in}}{\pgfqpoint{1.476668in}{5.304339in}}{\pgfqpoint{1.485876in}{5.295130in}}%
\pgfpathcurveto{\pgfqpoint{1.495085in}{5.285922in}}{\pgfqpoint{1.507576in}{5.280748in}}{\pgfqpoint{1.520599in}{5.280748in}}%
\pgfpathlineto{\pgfqpoint{1.520599in}{5.280748in}}%
\pgfpathclose%
\pgfusepath{stroke,fill}%
\end{pgfscope}%
\begin{pgfscope}%
\pgfpathrectangle{\pgfqpoint{0.786164in}{0.768110in}}{\pgfqpoint{8.851069in}{7.081890in}}%
\pgfusepath{clip}%
\pgfsetbuttcap%
\pgfsetroundjoin%
\definecolor{currentfill}{rgb}{0.235526,0.309527,0.542944}%
\pgfsetfillcolor{currentfill}%
\pgfsetfillopacity{0.700000}%
\pgfsetlinewidth{0.501875pt}%
\definecolor{currentstroke}{rgb}{1.000000,1.000000,1.000000}%
\pgfsetstrokecolor{currentstroke}%
\pgfsetstrokeopacity{0.700000}%
\pgfsetdash{}{0pt}%
\pgfpathmoveto{\pgfqpoint{1.551246in}{5.174037in}}%
\pgfpathcurveto{\pgfqpoint{1.564269in}{5.174037in}}{\pgfqpoint{1.576760in}{5.179211in}}{\pgfqpoint{1.585968in}{5.188419in}}%
\pgfpathcurveto{\pgfqpoint{1.595177in}{5.197628in}}{\pgfqpoint{1.600351in}{5.210119in}}{\pgfqpoint{1.600351in}{5.223142in}}%
\pgfpathcurveto{\pgfqpoint{1.600351in}{5.236164in}}{\pgfqpoint{1.595177in}{5.248655in}}{\pgfqpoint{1.585968in}{5.257864in}}%
\pgfpathcurveto{\pgfqpoint{1.576760in}{5.267072in}}{\pgfqpoint{1.564269in}{5.272246in}}{\pgfqpoint{1.551246in}{5.272246in}}%
\pgfpathcurveto{\pgfqpoint{1.538223in}{5.272246in}}{\pgfqpoint{1.525732in}{5.267072in}}{\pgfqpoint{1.516524in}{5.257864in}}%
\pgfpathcurveto{\pgfqpoint{1.507315in}{5.248655in}}{\pgfqpoint{1.502141in}{5.236164in}}{\pgfqpoint{1.502141in}{5.223142in}}%
\pgfpathcurveto{\pgfqpoint{1.502141in}{5.210119in}}{\pgfqpoint{1.507315in}{5.197628in}}{\pgfqpoint{1.516524in}{5.188419in}}%
\pgfpathcurveto{\pgfqpoint{1.525732in}{5.179211in}}{\pgfqpoint{1.538223in}{5.174037in}}{\pgfqpoint{1.551246in}{5.174037in}}%
\pgfpathlineto{\pgfqpoint{1.551246in}{5.174037in}}%
\pgfpathclose%
\pgfusepath{stroke,fill}%
\end{pgfscope}%
\begin{pgfscope}%
\pgfpathrectangle{\pgfqpoint{0.786164in}{0.768110in}}{\pgfqpoint{8.851069in}{7.081890in}}%
\pgfusepath{clip}%
\pgfsetbuttcap%
\pgfsetroundjoin%
\definecolor{currentfill}{rgb}{0.231674,0.318106,0.544834}%
\pgfsetfillcolor{currentfill}%
\pgfsetfillopacity{0.700000}%
\pgfsetlinewidth{0.501875pt}%
\definecolor{currentstroke}{rgb}{1.000000,1.000000,1.000000}%
\pgfsetstrokecolor{currentstroke}%
\pgfsetstrokeopacity{0.700000}%
\pgfsetdash{}{0pt}%
\pgfpathmoveto{\pgfqpoint{1.603017in}{4.960615in}}%
\pgfpathcurveto{\pgfqpoint{1.616039in}{4.960615in}}{\pgfqpoint{1.628530in}{4.965789in}}{\pgfqpoint{1.637739in}{4.974998in}}%
\pgfpathcurveto{\pgfqpoint{1.646947in}{4.984206in}}{\pgfqpoint{1.652121in}{4.996697in}}{\pgfqpoint{1.652121in}{5.009720in}}%
\pgfpathcurveto{\pgfqpoint{1.652121in}{5.022743in}}{\pgfqpoint{1.646947in}{5.035234in}}{\pgfqpoint{1.637739in}{5.044442in}}%
\pgfpathcurveto{\pgfqpoint{1.628530in}{5.053650in}}{\pgfqpoint{1.616039in}{5.058824in}}{\pgfqpoint{1.603017in}{5.058824in}}%
\pgfpathcurveto{\pgfqpoint{1.589994in}{5.058824in}}{\pgfqpoint{1.577503in}{5.053650in}}{\pgfqpoint{1.568294in}{5.044442in}}%
\pgfpathcurveto{\pgfqpoint{1.559086in}{5.035234in}}{\pgfqpoint{1.553912in}{5.022743in}}{\pgfqpoint{1.553912in}{5.009720in}}%
\pgfpathcurveto{\pgfqpoint{1.553912in}{4.996697in}}{\pgfqpoint{1.559086in}{4.984206in}}{\pgfqpoint{1.568294in}{4.974998in}}%
\pgfpathcurveto{\pgfqpoint{1.577503in}{4.965789in}}{\pgfqpoint{1.589994in}{4.960615in}}{\pgfqpoint{1.603017in}{4.960615in}}%
\pgfpathlineto{\pgfqpoint{1.603017in}{4.960615in}}%
\pgfpathclose%
\pgfusepath{stroke,fill}%
\end{pgfscope}%
\begin{pgfscope}%
\pgfpathrectangle{\pgfqpoint{0.786164in}{0.768110in}}{\pgfqpoint{8.851069in}{7.081890in}}%
\pgfusepath{clip}%
\pgfsetbuttcap%
\pgfsetroundjoin%
\definecolor{currentfill}{rgb}{0.239346,0.300855,0.540844}%
\pgfsetfillcolor{currentfill}%
\pgfsetfillopacity{0.700000}%
\pgfsetlinewidth{0.501875pt}%
\definecolor{currentstroke}{rgb}{1.000000,1.000000,1.000000}%
\pgfsetstrokecolor{currentstroke}%
\pgfsetstrokeopacity{0.700000}%
\pgfsetdash{}{0pt}%
\pgfpathmoveto{\pgfqpoint{1.661869in}{4.811220in}}%
\pgfpathcurveto{\pgfqpoint{1.674892in}{4.811220in}}{\pgfqpoint{1.687383in}{4.816394in}}{\pgfqpoint{1.696591in}{4.825602in}}%
\pgfpathcurveto{\pgfqpoint{1.705800in}{4.834811in}}{\pgfqpoint{1.710974in}{4.847302in}}{\pgfqpoint{1.710974in}{4.860325in}}%
\pgfpathcurveto{\pgfqpoint{1.710974in}{4.873347in}}{\pgfqpoint{1.705800in}{4.885838in}}{\pgfqpoint{1.696591in}{4.895047in}}%
\pgfpathcurveto{\pgfqpoint{1.687383in}{4.904255in}}{\pgfqpoint{1.674892in}{4.909429in}}{\pgfqpoint{1.661869in}{4.909429in}}%
\pgfpathcurveto{\pgfqpoint{1.648846in}{4.909429in}}{\pgfqpoint{1.636355in}{4.904255in}}{\pgfqpoint{1.627147in}{4.895047in}}%
\pgfpathcurveto{\pgfqpoint{1.617938in}{4.885838in}}{\pgfqpoint{1.612764in}{4.873347in}}{\pgfqpoint{1.612764in}{4.860325in}}%
\pgfpathcurveto{\pgfqpoint{1.612764in}{4.847302in}}{\pgfqpoint{1.617938in}{4.834811in}}{\pgfqpoint{1.627147in}{4.825602in}}%
\pgfpathcurveto{\pgfqpoint{1.636355in}{4.816394in}}{\pgfqpoint{1.648846in}{4.811220in}}{\pgfqpoint{1.661869in}{4.811220in}}%
\pgfpathlineto{\pgfqpoint{1.661869in}{4.811220in}}%
\pgfpathclose%
\pgfusepath{stroke,fill}%
\end{pgfscope}%
\begin{pgfscope}%
\pgfpathrectangle{\pgfqpoint{0.786164in}{0.768110in}}{\pgfqpoint{8.851069in}{7.081890in}}%
\pgfusepath{clip}%
\pgfsetbuttcap%
\pgfsetroundjoin%
\definecolor{currentfill}{rgb}{0.250425,0.274290,0.533103}%
\pgfsetfillcolor{currentfill}%
\pgfsetfillopacity{0.700000}%
\pgfsetlinewidth{0.501875pt}%
\definecolor{currentstroke}{rgb}{1.000000,1.000000,1.000000}%
\pgfsetstrokecolor{currentstroke}%
\pgfsetstrokeopacity{0.700000}%
\pgfsetdash{}{0pt}%
\pgfpathmoveto{\pgfqpoint{1.816204in}{4.448403in}}%
\pgfpathcurveto{\pgfqpoint{1.829227in}{4.448403in}}{\pgfqpoint{1.841718in}{4.453577in}}{\pgfqpoint{1.850926in}{4.462785in}}%
\pgfpathcurveto{\pgfqpoint{1.860135in}{4.471994in}}{\pgfqpoint{1.865309in}{4.484485in}}{\pgfqpoint{1.865309in}{4.497508in}}%
\pgfpathcurveto{\pgfqpoint{1.865309in}{4.510530in}}{\pgfqpoint{1.860135in}{4.523021in}}{\pgfqpoint{1.850926in}{4.532230in}}%
\pgfpathcurveto{\pgfqpoint{1.841718in}{4.541438in}}{\pgfqpoint{1.829227in}{4.546612in}}{\pgfqpoint{1.816204in}{4.546612in}}%
\pgfpathcurveto{\pgfqpoint{1.803181in}{4.546612in}}{\pgfqpoint{1.790690in}{4.541438in}}{\pgfqpoint{1.781482in}{4.532230in}}%
\pgfpathcurveto{\pgfqpoint{1.772273in}{4.523021in}}{\pgfqpoint{1.767099in}{4.510530in}}{\pgfqpoint{1.767099in}{4.497508in}}%
\pgfpathcurveto{\pgfqpoint{1.767099in}{4.484485in}}{\pgfqpoint{1.772273in}{4.471994in}}{\pgfqpoint{1.781482in}{4.462785in}}%
\pgfpathcurveto{\pgfqpoint{1.790690in}{4.453577in}}{\pgfqpoint{1.803181in}{4.448403in}}{\pgfqpoint{1.816204in}{4.448403in}}%
\pgfpathlineto{\pgfqpoint{1.816204in}{4.448403in}}%
\pgfpathclose%
\pgfusepath{stroke,fill}%
\end{pgfscope}%
\begin{pgfscope}%
\pgfpathrectangle{\pgfqpoint{0.786164in}{0.768110in}}{\pgfqpoint{8.851069in}{7.081890in}}%
\pgfusepath{clip}%
\pgfsetbuttcap%
\pgfsetroundjoin%
\definecolor{currentfill}{rgb}{0.252194,0.269783,0.531579}%
\pgfsetfillcolor{currentfill}%
\pgfsetfillopacity{0.700000}%
\pgfsetlinewidth{0.501875pt}%
\definecolor{currentstroke}{rgb}{1.000000,1.000000,1.000000}%
\pgfsetstrokecolor{currentstroke}%
\pgfsetstrokeopacity{0.700000}%
\pgfsetdash{}{0pt}%
\pgfpathmoveto{\pgfqpoint{1.878720in}{4.469745in}}%
\pgfpathcurveto{\pgfqpoint{1.891742in}{4.469745in}}{\pgfqpoint{1.904233in}{4.474919in}}{\pgfqpoint{1.913442in}{4.484128in}}%
\pgfpathcurveto{\pgfqpoint{1.922650in}{4.493336in}}{\pgfqpoint{1.927824in}{4.505827in}}{\pgfqpoint{1.927824in}{4.518850in}}%
\pgfpathcurveto{\pgfqpoint{1.927824in}{4.531873in}}{\pgfqpoint{1.922650in}{4.544364in}}{\pgfqpoint{1.913442in}{4.553572in}}%
\pgfpathcurveto{\pgfqpoint{1.904233in}{4.562781in}}{\pgfqpoint{1.891742in}{4.567955in}}{\pgfqpoint{1.878720in}{4.567955in}}%
\pgfpathcurveto{\pgfqpoint{1.865697in}{4.567955in}}{\pgfqpoint{1.853206in}{4.562781in}}{\pgfqpoint{1.843997in}{4.553572in}}%
\pgfpathcurveto{\pgfqpoint{1.834789in}{4.544364in}}{\pgfqpoint{1.829615in}{4.531873in}}{\pgfqpoint{1.829615in}{4.518850in}}%
\pgfpathcurveto{\pgfqpoint{1.829615in}{4.505827in}}{\pgfqpoint{1.834789in}{4.493336in}}{\pgfqpoint{1.843997in}{4.484128in}}%
\pgfpathcurveto{\pgfqpoint{1.853206in}{4.474919in}}{\pgfqpoint{1.865697in}{4.469745in}}{\pgfqpoint{1.878720in}{4.469745in}}%
\pgfpathlineto{\pgfqpoint{1.878720in}{4.469745in}}%
\pgfpathclose%
\pgfusepath{stroke,fill}%
\end{pgfscope}%
\begin{pgfscope}%
\pgfpathrectangle{\pgfqpoint{0.786164in}{0.768110in}}{\pgfqpoint{8.851069in}{7.081890in}}%
\pgfusepath{clip}%
\pgfsetbuttcap%
\pgfsetroundjoin%
\definecolor{currentfill}{rgb}{0.265145,0.232956,0.516599}%
\pgfsetfillcolor{currentfill}%
\pgfsetfillopacity{0.700000}%
\pgfsetlinewidth{0.501875pt}%
\definecolor{currentstroke}{rgb}{1.000000,1.000000,1.000000}%
\pgfsetstrokecolor{currentstroke}%
\pgfsetstrokeopacity{0.700000}%
\pgfsetdash{}{0pt}%
\pgfpathmoveto{\pgfqpoint{1.982505in}{4.213639in}}%
\pgfpathcurveto{\pgfqpoint{1.995528in}{4.213639in}}{\pgfqpoint{2.008019in}{4.218813in}}{\pgfqpoint{2.017227in}{4.228022in}}%
\pgfpathcurveto{\pgfqpoint{2.026436in}{4.237230in}}{\pgfqpoint{2.031610in}{4.249721in}}{\pgfqpoint{2.031610in}{4.262744in}}%
\pgfpathcurveto{\pgfqpoint{2.031610in}{4.275767in}}{\pgfqpoint{2.026436in}{4.288258in}}{\pgfqpoint{2.017227in}{4.297466in}}%
\pgfpathcurveto{\pgfqpoint{2.008019in}{4.306674in}}{\pgfqpoint{1.995528in}{4.311848in}}{\pgfqpoint{1.982505in}{4.311848in}}%
\pgfpathcurveto{\pgfqpoint{1.969482in}{4.311848in}}{\pgfqpoint{1.956991in}{4.306674in}}{\pgfqpoint{1.947783in}{4.297466in}}%
\pgfpathcurveto{\pgfqpoint{1.938574in}{4.288258in}}{\pgfqpoint{1.933400in}{4.275767in}}{\pgfqpoint{1.933400in}{4.262744in}}%
\pgfpathcurveto{\pgfqpoint{1.933400in}{4.249721in}}{\pgfqpoint{1.938574in}{4.237230in}}{\pgfqpoint{1.947783in}{4.228022in}}%
\pgfpathcurveto{\pgfqpoint{1.956991in}{4.218813in}}{\pgfqpoint{1.969482in}{4.213639in}}{\pgfqpoint{1.982505in}{4.213639in}}%
\pgfpathlineto{\pgfqpoint{1.982505in}{4.213639in}}%
\pgfpathclose%
\pgfusepath{stroke,fill}%
\end{pgfscope}%
\begin{pgfscope}%
\pgfpathrectangle{\pgfqpoint{0.786164in}{0.768110in}}{\pgfqpoint{8.851069in}{7.081890in}}%
\pgfusepath{clip}%
\pgfsetbuttcap%
\pgfsetroundjoin%
\definecolor{currentfill}{rgb}{0.265145,0.232956,0.516599}%
\pgfsetfillcolor{currentfill}%
\pgfsetfillopacity{0.700000}%
\pgfsetlinewidth{0.501875pt}%
\definecolor{currentstroke}{rgb}{1.000000,1.000000,1.000000}%
\pgfsetstrokecolor{currentstroke}%
\pgfsetstrokeopacity{0.700000}%
\pgfsetdash{}{0pt}%
\pgfpathmoveto{\pgfqpoint{2.034276in}{4.192297in}}%
\pgfpathcurveto{\pgfqpoint{2.047298in}{4.192297in}}{\pgfqpoint{2.059789in}{4.197471in}}{\pgfqpoint{2.068998in}{4.206679in}}%
\pgfpathcurveto{\pgfqpoint{2.078206in}{4.215888in}}{\pgfqpoint{2.083380in}{4.228379in}}{\pgfqpoint{2.083380in}{4.241402in}}%
\pgfpathcurveto{\pgfqpoint{2.083380in}{4.254424in}}{\pgfqpoint{2.078206in}{4.266915in}}{\pgfqpoint{2.068998in}{4.276124in}}%
\pgfpathcurveto{\pgfqpoint{2.059789in}{4.285332in}}{\pgfqpoint{2.047298in}{4.290506in}}{\pgfqpoint{2.034276in}{4.290506in}}%
\pgfpathcurveto{\pgfqpoint{2.021253in}{4.290506in}}{\pgfqpoint{2.008762in}{4.285332in}}{\pgfqpoint{1.999553in}{4.276124in}}%
\pgfpathcurveto{\pgfqpoint{1.990345in}{4.266915in}}{\pgfqpoint{1.985171in}{4.254424in}}{\pgfqpoint{1.985171in}{4.241402in}}%
\pgfpathcurveto{\pgfqpoint{1.985171in}{4.228379in}}{\pgfqpoint{1.990345in}{4.215888in}}{\pgfqpoint{1.999553in}{4.206679in}}%
\pgfpathcurveto{\pgfqpoint{2.008762in}{4.197471in}}{\pgfqpoint{2.021253in}{4.192297in}}{\pgfqpoint{2.034276in}{4.192297in}}%
\pgfpathlineto{\pgfqpoint{2.034276in}{4.192297in}}%
\pgfpathclose%
\pgfusepath{stroke,fill}%
\end{pgfscope}%
\begin{pgfscope}%
\pgfpathrectangle{\pgfqpoint{0.786164in}{0.768110in}}{\pgfqpoint{8.851069in}{7.081890in}}%
\pgfusepath{clip}%
\pgfsetbuttcap%
\pgfsetroundjoin%
\definecolor{currentfill}{rgb}{0.271828,0.209303,0.504434}%
\pgfsetfillcolor{currentfill}%
\pgfsetfillopacity{0.700000}%
\pgfsetlinewidth{0.501875pt}%
\definecolor{currentstroke}{rgb}{1.000000,1.000000,1.000000}%
\pgfsetstrokecolor{currentstroke}%
\pgfsetstrokeopacity{0.700000}%
\pgfsetdash{}{0pt}%
\pgfpathmoveto{\pgfqpoint{2.094349in}{4.213639in}}%
\pgfpathcurveto{\pgfqpoint{2.107372in}{4.213639in}}{\pgfqpoint{2.119863in}{4.218813in}}{\pgfqpoint{2.129071in}{4.228022in}}%
\pgfpathcurveto{\pgfqpoint{2.138280in}{4.237230in}}{\pgfqpoint{2.143454in}{4.249721in}}{\pgfqpoint{2.143454in}{4.262744in}}%
\pgfpathcurveto{\pgfqpoint{2.143454in}{4.275767in}}{\pgfqpoint{2.138280in}{4.288258in}}{\pgfqpoint{2.129071in}{4.297466in}}%
\pgfpathcurveto{\pgfqpoint{2.119863in}{4.306674in}}{\pgfqpoint{2.107372in}{4.311848in}}{\pgfqpoint{2.094349in}{4.311848in}}%
\pgfpathcurveto{\pgfqpoint{2.081326in}{4.311848in}}{\pgfqpoint{2.068835in}{4.306674in}}{\pgfqpoint{2.059627in}{4.297466in}}%
\pgfpathcurveto{\pgfqpoint{2.050419in}{4.288258in}}{\pgfqpoint{2.045245in}{4.275767in}}{\pgfqpoint{2.045245in}{4.262744in}}%
\pgfpathcurveto{\pgfqpoint{2.045245in}{4.249721in}}{\pgfqpoint{2.050419in}{4.237230in}}{\pgfqpoint{2.059627in}{4.228022in}}%
\pgfpathcurveto{\pgfqpoint{2.068835in}{4.218813in}}{\pgfqpoint{2.081326in}{4.213639in}}{\pgfqpoint{2.094349in}{4.213639in}}%
\pgfpathlineto{\pgfqpoint{2.094349in}{4.213639in}}%
\pgfpathclose%
\pgfusepath{stroke,fill}%
\end{pgfscope}%
\begin{pgfscope}%
\pgfpathrectangle{\pgfqpoint{0.786164in}{0.768110in}}{\pgfqpoint{8.851069in}{7.081890in}}%
\pgfusepath{clip}%
\pgfsetbuttcap%
\pgfsetroundjoin%
\definecolor{currentfill}{rgb}{0.270595,0.214069,0.507052}%
\pgfsetfillcolor{currentfill}%
\pgfsetfillopacity{0.700000}%
\pgfsetlinewidth{0.501875pt}%
\definecolor{currentstroke}{rgb}{1.000000,1.000000,1.000000}%
\pgfsetstrokecolor{currentstroke}%
\pgfsetstrokeopacity{0.700000}%
\pgfsetdash{}{0pt}%
\pgfpathmoveto{\pgfqpoint{2.215839in}{4.149613in}}%
\pgfpathcurveto{\pgfqpoint{2.228862in}{4.149613in}}{\pgfqpoint{2.241353in}{4.154787in}}{\pgfqpoint{2.250561in}{4.163995in}}%
\pgfpathcurveto{\pgfqpoint{2.259770in}{4.173204in}}{\pgfqpoint{2.264944in}{4.185695in}}{\pgfqpoint{2.264944in}{4.198717in}}%
\pgfpathcurveto{\pgfqpoint{2.264944in}{4.211740in}}{\pgfqpoint{2.259770in}{4.224231in}}{\pgfqpoint{2.250561in}{4.233440in}}%
\pgfpathcurveto{\pgfqpoint{2.241353in}{4.242648in}}{\pgfqpoint{2.228862in}{4.247822in}}{\pgfqpoint{2.215839in}{4.247822in}}%
\pgfpathcurveto{\pgfqpoint{2.202817in}{4.247822in}}{\pgfqpoint{2.190325in}{4.242648in}}{\pgfqpoint{2.181117in}{4.233440in}}%
\pgfpathcurveto{\pgfqpoint{2.171909in}{4.224231in}}{\pgfqpoint{2.166735in}{4.211740in}}{\pgfqpoint{2.166735in}{4.198717in}}%
\pgfpathcurveto{\pgfqpoint{2.166735in}{4.185695in}}{\pgfqpoint{2.171909in}{4.173204in}}{\pgfqpoint{2.181117in}{4.163995in}}%
\pgfpathcurveto{\pgfqpoint{2.190325in}{4.154787in}}{\pgfqpoint{2.202817in}{4.149613in}}{\pgfqpoint{2.215839in}{4.149613in}}%
\pgfpathlineto{\pgfqpoint{2.215839in}{4.149613in}}%
\pgfpathclose%
\pgfusepath{stroke,fill}%
\end{pgfscope}%
\begin{pgfscope}%
\pgfpathrectangle{\pgfqpoint{0.786164in}{0.768110in}}{\pgfqpoint{8.851069in}{7.081890in}}%
\pgfusepath{clip}%
\pgfsetbuttcap%
\pgfsetroundjoin%
\definecolor{currentfill}{rgb}{0.265145,0.232956,0.516599}%
\pgfsetfillcolor{currentfill}%
\pgfsetfillopacity{0.700000}%
\pgfsetlinewidth{0.501875pt}%
\definecolor{currentstroke}{rgb}{1.000000,1.000000,1.000000}%
\pgfsetstrokecolor{currentstroke}%
\pgfsetstrokeopacity{0.700000}%
\pgfsetdash{}{0pt}%
\pgfpathmoveto{\pgfqpoint{2.285070in}{4.277666in}}%
\pgfpathcurveto{\pgfqpoint{2.298093in}{4.277666in}}{\pgfqpoint{2.310584in}{4.282840in}}{\pgfqpoint{2.319792in}{4.292048in}}%
\pgfpathcurveto{\pgfqpoint{2.329001in}{4.301257in}}{\pgfqpoint{2.334175in}{4.313748in}}{\pgfqpoint{2.334175in}{4.326770in}}%
\pgfpathcurveto{\pgfqpoint{2.334175in}{4.339793in}}{\pgfqpoint{2.329001in}{4.352284in}}{\pgfqpoint{2.319792in}{4.361493in}}%
\pgfpathcurveto{\pgfqpoint{2.310584in}{4.370701in}}{\pgfqpoint{2.298093in}{4.375875in}}{\pgfqpoint{2.285070in}{4.375875in}}%
\pgfpathcurveto{\pgfqpoint{2.272048in}{4.375875in}}{\pgfqpoint{2.259556in}{4.370701in}}{\pgfqpoint{2.250348in}{4.361493in}}%
\pgfpathcurveto{\pgfqpoint{2.241140in}{4.352284in}}{\pgfqpoint{2.235966in}{4.339793in}}{\pgfqpoint{2.235966in}{4.326770in}}%
\pgfpathcurveto{\pgfqpoint{2.235966in}{4.313748in}}{\pgfqpoint{2.241140in}{4.301257in}}{\pgfqpoint{2.250348in}{4.292048in}}%
\pgfpathcurveto{\pgfqpoint{2.259556in}{4.282840in}}{\pgfqpoint{2.272048in}{4.277666in}}{\pgfqpoint{2.285070in}{4.277666in}}%
\pgfpathlineto{\pgfqpoint{2.285070in}{4.277666in}}%
\pgfpathclose%
\pgfusepath{stroke,fill}%
\end{pgfscope}%
\begin{pgfscope}%
\pgfpathrectangle{\pgfqpoint{0.786164in}{0.768110in}}{\pgfqpoint{8.851069in}{7.081890in}}%
\pgfusepath{clip}%
\pgfsetbuttcap%
\pgfsetroundjoin%
\definecolor{currentfill}{rgb}{0.260571,0.246922,0.522828}%
\pgfsetfillcolor{currentfill}%
\pgfsetfillopacity{0.700000}%
\pgfsetlinewidth{0.501875pt}%
\definecolor{currentstroke}{rgb}{1.000000,1.000000,1.000000}%
\pgfsetstrokecolor{currentstroke}%
\pgfsetstrokeopacity{0.700000}%
\pgfsetdash{}{0pt}%
\pgfpathmoveto{\pgfqpoint{2.298013in}{4.299008in}}%
\pgfpathcurveto{\pgfqpoint{2.311036in}{4.299008in}}{\pgfqpoint{2.323527in}{4.304182in}}{\pgfqpoint{2.332735in}{4.313390in}}%
\pgfpathcurveto{\pgfqpoint{2.341944in}{4.322599in}}{\pgfqpoint{2.347118in}{4.335090in}}{\pgfqpoint{2.347118in}{4.348113in}}%
\pgfpathcurveto{\pgfqpoint{2.347118in}{4.361135in}}{\pgfqpoint{2.341944in}{4.373626in}}{\pgfqpoint{2.332735in}{4.382835in}}%
\pgfpathcurveto{\pgfqpoint{2.323527in}{4.392043in}}{\pgfqpoint{2.311036in}{4.397217in}}{\pgfqpoint{2.298013in}{4.397217in}}%
\pgfpathcurveto{\pgfqpoint{2.284990in}{4.397217in}}{\pgfqpoint{2.272499in}{4.392043in}}{\pgfqpoint{2.263291in}{4.382835in}}%
\pgfpathcurveto{\pgfqpoint{2.254082in}{4.373626in}}{\pgfqpoint{2.248908in}{4.361135in}}{\pgfqpoint{2.248908in}{4.348113in}}%
\pgfpathcurveto{\pgfqpoint{2.248908in}{4.335090in}}{\pgfqpoint{2.254082in}{4.322599in}}{\pgfqpoint{2.263291in}{4.313390in}}%
\pgfpathcurveto{\pgfqpoint{2.272499in}{4.304182in}}{\pgfqpoint{2.284990in}{4.299008in}}{\pgfqpoint{2.298013in}{4.299008in}}%
\pgfpathlineto{\pgfqpoint{2.298013in}{4.299008in}}%
\pgfpathclose%
\pgfusepath{stroke,fill}%
\end{pgfscope}%
\begin{pgfscope}%
\pgfpathrectangle{\pgfqpoint{0.786164in}{0.768110in}}{\pgfqpoint{8.851069in}{7.081890in}}%
\pgfusepath{clip}%
\pgfsetbuttcap%
\pgfsetroundjoin%
\definecolor{currentfill}{rgb}{0.262138,0.242286,0.520837}%
\pgfsetfillcolor{currentfill}%
\pgfsetfillopacity{0.700000}%
\pgfsetlinewidth{0.501875pt}%
\definecolor{currentstroke}{rgb}{1.000000,1.000000,1.000000}%
\pgfsetstrokecolor{currentstroke}%
\pgfsetstrokeopacity{0.700000}%
\pgfsetdash{}{0pt}%
\pgfpathmoveto{\pgfqpoint{2.460529in}{4.341692in}}%
\pgfpathcurveto{\pgfqpoint{2.473551in}{4.341692in}}{\pgfqpoint{2.486043in}{4.346866in}}{\pgfqpoint{2.495251in}{4.356075in}}%
\pgfpathcurveto{\pgfqpoint{2.504459in}{4.365283in}}{\pgfqpoint{2.509633in}{4.377774in}}{\pgfqpoint{2.509633in}{4.390797in}}%
\pgfpathcurveto{\pgfqpoint{2.509633in}{4.403820in}}{\pgfqpoint{2.504459in}{4.416311in}}{\pgfqpoint{2.495251in}{4.425519in}}%
\pgfpathcurveto{\pgfqpoint{2.486043in}{4.434728in}}{\pgfqpoint{2.473551in}{4.439901in}}{\pgfqpoint{2.460529in}{4.439901in}}%
\pgfpathcurveto{\pgfqpoint{2.447506in}{4.439901in}}{\pgfqpoint{2.435015in}{4.434728in}}{\pgfqpoint{2.425807in}{4.425519in}}%
\pgfpathcurveto{\pgfqpoint{2.416598in}{4.416311in}}{\pgfqpoint{2.411424in}{4.403820in}}{\pgfqpoint{2.411424in}{4.390797in}}%
\pgfpathcurveto{\pgfqpoint{2.411424in}{4.377774in}}{\pgfqpoint{2.416598in}{4.365283in}}{\pgfqpoint{2.425807in}{4.356075in}}%
\pgfpathcurveto{\pgfqpoint{2.435015in}{4.346866in}}{\pgfqpoint{2.447506in}{4.341692in}}{\pgfqpoint{2.460529in}{4.341692in}}%
\pgfpathlineto{\pgfqpoint{2.460529in}{4.341692in}}%
\pgfpathclose%
\pgfusepath{stroke,fill}%
\end{pgfscope}%
\begin{pgfscope}%
\pgfpathrectangle{\pgfqpoint{0.786164in}{0.768110in}}{\pgfqpoint{8.851069in}{7.081890in}}%
\pgfusepath{clip}%
\pgfsetbuttcap%
\pgfsetroundjoin%
\definecolor{currentfill}{rgb}{0.263663,0.237631,0.518762}%
\pgfsetfillcolor{currentfill}%
\pgfsetfillopacity{0.700000}%
\pgfsetlinewidth{0.501875pt}%
\definecolor{currentstroke}{rgb}{1.000000,1.000000,1.000000}%
\pgfsetstrokecolor{currentstroke}%
\pgfsetstrokeopacity{0.700000}%
\pgfsetdash{}{0pt}%
\pgfpathmoveto{\pgfqpoint{2.512055in}{4.299008in}}%
\pgfpathcurveto{\pgfqpoint{2.525078in}{4.299008in}}{\pgfqpoint{2.537569in}{4.304182in}}{\pgfqpoint{2.546777in}{4.313390in}}%
\pgfpathcurveto{\pgfqpoint{2.555986in}{4.322599in}}{\pgfqpoint{2.561160in}{4.335090in}}{\pgfqpoint{2.561160in}{4.348113in}}%
\pgfpathcurveto{\pgfqpoint{2.561160in}{4.361135in}}{\pgfqpoint{2.555986in}{4.373626in}}{\pgfqpoint{2.546777in}{4.382835in}}%
\pgfpathcurveto{\pgfqpoint{2.537569in}{4.392043in}}{\pgfqpoint{2.525078in}{4.397217in}}{\pgfqpoint{2.512055in}{4.397217in}}%
\pgfpathcurveto{\pgfqpoint{2.499032in}{4.397217in}}{\pgfqpoint{2.486541in}{4.392043in}}{\pgfqpoint{2.477333in}{4.382835in}}%
\pgfpathcurveto{\pgfqpoint{2.468125in}{4.373626in}}{\pgfqpoint{2.462951in}{4.361135in}}{\pgfqpoint{2.462951in}{4.348113in}}%
\pgfpathcurveto{\pgfqpoint{2.462951in}{4.335090in}}{\pgfqpoint{2.468125in}{4.322599in}}{\pgfqpoint{2.477333in}{4.313390in}}%
\pgfpathcurveto{\pgfqpoint{2.486541in}{4.304182in}}{\pgfqpoint{2.499032in}{4.299008in}}{\pgfqpoint{2.512055in}{4.299008in}}%
\pgfpathlineto{\pgfqpoint{2.512055in}{4.299008in}}%
\pgfpathclose%
\pgfusepath{stroke,fill}%
\end{pgfscope}%
\begin{pgfscope}%
\pgfpathrectangle{\pgfqpoint{0.786164in}{0.768110in}}{\pgfqpoint{8.851069in}{7.081890in}}%
\pgfusepath{clip}%
\pgfsetbuttcap%
\pgfsetroundjoin%
\definecolor{currentfill}{rgb}{0.265145,0.232956,0.516599}%
\pgfsetfillcolor{currentfill}%
\pgfsetfillopacity{0.700000}%
\pgfsetlinewidth{0.501875pt}%
\definecolor{currentstroke}{rgb}{1.000000,1.000000,1.000000}%
\pgfsetstrokecolor{currentstroke}%
\pgfsetstrokeopacity{0.700000}%
\pgfsetdash{}{0pt}%
\pgfpathmoveto{\pgfqpoint{2.638673in}{4.149613in}}%
\pgfpathcurveto{\pgfqpoint{2.651696in}{4.149613in}}{\pgfqpoint{2.664187in}{4.154787in}}{\pgfqpoint{2.673396in}{4.163995in}}%
\pgfpathcurveto{\pgfqpoint{2.682604in}{4.173204in}}{\pgfqpoint{2.687778in}{4.185695in}}{\pgfqpoint{2.687778in}{4.198717in}}%
\pgfpathcurveto{\pgfqpoint{2.687778in}{4.211740in}}{\pgfqpoint{2.682604in}{4.224231in}}{\pgfqpoint{2.673396in}{4.233440in}}%
\pgfpathcurveto{\pgfqpoint{2.664187in}{4.242648in}}{\pgfqpoint{2.651696in}{4.247822in}}{\pgfqpoint{2.638673in}{4.247822in}}%
\pgfpathcurveto{\pgfqpoint{2.625651in}{4.247822in}}{\pgfqpoint{2.613160in}{4.242648in}}{\pgfqpoint{2.603951in}{4.233440in}}%
\pgfpathcurveto{\pgfqpoint{2.594743in}{4.224231in}}{\pgfqpoint{2.589569in}{4.211740in}}{\pgfqpoint{2.589569in}{4.198717in}}%
\pgfpathcurveto{\pgfqpoint{2.589569in}{4.185695in}}{\pgfqpoint{2.594743in}{4.173204in}}{\pgfqpoint{2.603951in}{4.163995in}}%
\pgfpathcurveto{\pgfqpoint{2.613160in}{4.154787in}}{\pgfqpoint{2.625651in}{4.149613in}}{\pgfqpoint{2.638673in}{4.149613in}}%
\pgfpathlineto{\pgfqpoint{2.638673in}{4.149613in}}%
\pgfpathclose%
\pgfusepath{stroke,fill}%
\end{pgfscope}%
\begin{pgfscope}%
\pgfpathrectangle{\pgfqpoint{0.786164in}{0.768110in}}{\pgfqpoint{8.851069in}{7.081890in}}%
\pgfusepath{clip}%
\pgfsetbuttcap%
\pgfsetroundjoin%
\definecolor{currentfill}{rgb}{0.275191,0.194905,0.496005}%
\pgfsetfillcolor{currentfill}%
\pgfsetfillopacity{0.700000}%
\pgfsetlinewidth{0.501875pt}%
\definecolor{currentstroke}{rgb}{1.000000,1.000000,1.000000}%
\pgfsetstrokecolor{currentstroke}%
\pgfsetstrokeopacity{0.700000}%
\pgfsetdash{}{0pt}%
\pgfpathmoveto{\pgfqpoint{3.149176in}{3.978875in}}%
\pgfpathcurveto{\pgfqpoint{3.162199in}{3.978875in}}{\pgfqpoint{3.174690in}{3.984049in}}{\pgfqpoint{3.183898in}{3.993258in}}%
\pgfpathcurveto{\pgfqpoint{3.193107in}{4.002466in}}{\pgfqpoint{3.198280in}{4.014957in}}{\pgfqpoint{3.198280in}{4.027980in}}%
\pgfpathcurveto{\pgfqpoint{3.198280in}{4.041003in}}{\pgfqpoint{3.193107in}{4.053494in}}{\pgfqpoint{3.183898in}{4.062702in}}%
\pgfpathcurveto{\pgfqpoint{3.174690in}{4.071911in}}{\pgfqpoint{3.162199in}{4.077085in}}{\pgfqpoint{3.149176in}{4.077085in}}%
\pgfpathcurveto{\pgfqpoint{3.136153in}{4.077085in}}{\pgfqpoint{3.123662in}{4.071911in}}{\pgfqpoint{3.114454in}{4.062702in}}%
\pgfpathcurveto{\pgfqpoint{3.105245in}{4.053494in}}{\pgfqpoint{3.100071in}{4.041003in}}{\pgfqpoint{3.100071in}{4.027980in}}%
\pgfpathcurveto{\pgfqpoint{3.100071in}{4.014957in}}{\pgfqpoint{3.105245in}{4.002466in}}{\pgfqpoint{3.114454in}{3.993258in}}%
\pgfpathcurveto{\pgfqpoint{3.123662in}{3.984049in}}{\pgfqpoint{3.136153in}{3.978875in}}{\pgfqpoint{3.149176in}{3.978875in}}%
\pgfpathlineto{\pgfqpoint{3.149176in}{3.978875in}}%
\pgfpathclose%
\pgfusepath{stroke,fill}%
\end{pgfscope}%
\begin{pgfscope}%
\pgfpathrectangle{\pgfqpoint{0.786164in}{0.768110in}}{\pgfqpoint{8.851069in}{7.081890in}}%
\pgfusepath{clip}%
\pgfsetbuttcap%
\pgfsetroundjoin%
\definecolor{currentfill}{rgb}{0.273006,0.204520,0.501721}%
\pgfsetfillcolor{currentfill}%
\pgfsetfillopacity{0.700000}%
\pgfsetlinewidth{0.501875pt}%
\definecolor{currentstroke}{rgb}{1.000000,1.000000,1.000000}%
\pgfsetstrokecolor{currentstroke}%
\pgfsetstrokeopacity{0.700000}%
\pgfsetdash{}{0pt}%
\pgfpathmoveto{\pgfqpoint{2.687147in}{4.064244in}}%
\pgfpathcurveto{\pgfqpoint{2.700170in}{4.064244in}}{\pgfqpoint{2.712661in}{4.069418in}}{\pgfqpoint{2.721870in}{4.078626in}}%
\pgfpathcurveto{\pgfqpoint{2.731078in}{4.087835in}}{\pgfqpoint{2.736252in}{4.100326in}}{\pgfqpoint{2.736252in}{4.113349in}}%
\pgfpathcurveto{\pgfqpoint{2.736252in}{4.126371in}}{\pgfqpoint{2.731078in}{4.138862in}}{\pgfqpoint{2.721870in}{4.148071in}}%
\pgfpathcurveto{\pgfqpoint{2.712661in}{4.157279in}}{\pgfqpoint{2.700170in}{4.162453in}}{\pgfqpoint{2.687147in}{4.162453in}}%
\pgfpathcurveto{\pgfqpoint{2.674125in}{4.162453in}}{\pgfqpoint{2.661634in}{4.157279in}}{\pgfqpoint{2.652425in}{4.148071in}}%
\pgfpathcurveto{\pgfqpoint{2.643217in}{4.138862in}}{\pgfqpoint{2.638043in}{4.126371in}}{\pgfqpoint{2.638043in}{4.113349in}}%
\pgfpathcurveto{\pgfqpoint{2.638043in}{4.100326in}}{\pgfqpoint{2.643217in}{4.087835in}}{\pgfqpoint{2.652425in}{4.078626in}}%
\pgfpathcurveto{\pgfqpoint{2.661634in}{4.069418in}}{\pgfqpoint{2.674125in}{4.064244in}}{\pgfqpoint{2.687147in}{4.064244in}}%
\pgfpathlineto{\pgfqpoint{2.687147in}{4.064244in}}%
\pgfpathclose%
\pgfusepath{stroke,fill}%
\end{pgfscope}%
\begin{pgfscope}%
\pgfpathrectangle{\pgfqpoint{0.786164in}{0.768110in}}{\pgfqpoint{8.851069in}{7.081890in}}%
\pgfusepath{clip}%
\pgfsetbuttcap%
\pgfsetroundjoin%
\definecolor{currentfill}{rgb}{0.273006,0.204520,0.501721}%
\pgfsetfillcolor{currentfill}%
\pgfsetfillopacity{0.700000}%
\pgfsetlinewidth{0.501875pt}%
\definecolor{currentstroke}{rgb}{1.000000,1.000000,1.000000}%
\pgfsetstrokecolor{currentstroke}%
\pgfsetstrokeopacity{0.700000}%
\pgfsetdash{}{0pt}%
\pgfpathmoveto{\pgfqpoint{2.776403in}{3.978875in}}%
\pgfpathcurveto{\pgfqpoint{2.789426in}{3.978875in}}{\pgfqpoint{2.801917in}{3.984049in}}{\pgfqpoint{2.811125in}{3.993258in}}%
\pgfpathcurveto{\pgfqpoint{2.820334in}{4.002466in}}{\pgfqpoint{2.825507in}{4.014957in}}{\pgfqpoint{2.825507in}{4.027980in}}%
\pgfpathcurveto{\pgfqpoint{2.825507in}{4.041003in}}{\pgfqpoint{2.820334in}{4.053494in}}{\pgfqpoint{2.811125in}{4.062702in}}%
\pgfpathcurveto{\pgfqpoint{2.801917in}{4.071911in}}{\pgfqpoint{2.789426in}{4.077085in}}{\pgfqpoint{2.776403in}{4.077085in}}%
\pgfpathcurveto{\pgfqpoint{2.763380in}{4.077085in}}{\pgfqpoint{2.750889in}{4.071911in}}{\pgfqpoint{2.741681in}{4.062702in}}%
\pgfpathcurveto{\pgfqpoint{2.732472in}{4.053494in}}{\pgfqpoint{2.727298in}{4.041003in}}{\pgfqpoint{2.727298in}{4.027980in}}%
\pgfpathcurveto{\pgfqpoint{2.727298in}{4.014957in}}{\pgfqpoint{2.732472in}{4.002466in}}{\pgfqpoint{2.741681in}{3.993258in}}%
\pgfpathcurveto{\pgfqpoint{2.750889in}{3.984049in}}{\pgfqpoint{2.763380in}{3.978875in}}{\pgfqpoint{2.776403in}{3.978875in}}%
\pgfpathlineto{\pgfqpoint{2.776403in}{3.978875in}}%
\pgfpathclose%
\pgfusepath{stroke,fill}%
\end{pgfscope}%
\begin{pgfscope}%
\pgfpathrectangle{\pgfqpoint{0.786164in}{0.768110in}}{\pgfqpoint{8.851069in}{7.081890in}}%
\pgfusepath{clip}%
\pgfsetbuttcap%
\pgfsetroundjoin%
\definecolor{currentfill}{rgb}{0.260571,0.246922,0.522828}%
\pgfsetfillcolor{currentfill}%
\pgfsetfillopacity{0.700000}%
\pgfsetlinewidth{0.501875pt}%
\definecolor{currentstroke}{rgb}{1.000000,1.000000,1.000000}%
\pgfsetstrokecolor{currentstroke}%
\pgfsetstrokeopacity{0.700000}%
\pgfsetdash{}{0pt}%
\pgfpathmoveto{\pgfqpoint{1.947218in}{3.295926in}}%
\pgfpathcurveto{\pgfqpoint{1.960241in}{3.295926in}}{\pgfqpoint{1.972732in}{3.301100in}}{\pgfqpoint{1.981940in}{3.310308in}}%
\pgfpathcurveto{\pgfqpoint{1.991149in}{3.319517in}}{\pgfqpoint{1.996323in}{3.332008in}}{\pgfqpoint{1.996323in}{3.345030in}}%
\pgfpathcurveto{\pgfqpoint{1.996323in}{3.358053in}}{\pgfqpoint{1.991149in}{3.370544in}}{\pgfqpoint{1.981940in}{3.379753in}}%
\pgfpathcurveto{\pgfqpoint{1.972732in}{3.388961in}}{\pgfqpoint{1.960241in}{3.394135in}}{\pgfqpoint{1.947218in}{3.394135in}}%
\pgfpathcurveto{\pgfqpoint{1.934195in}{3.394135in}}{\pgfqpoint{1.921704in}{3.388961in}}{\pgfqpoint{1.912496in}{3.379753in}}%
\pgfpathcurveto{\pgfqpoint{1.903287in}{3.370544in}}{\pgfqpoint{1.898113in}{3.358053in}}{\pgfqpoint{1.898113in}{3.345030in}}%
\pgfpathcurveto{\pgfqpoint{1.898113in}{3.332008in}}{\pgfqpoint{1.903287in}{3.319517in}}{\pgfqpoint{1.912496in}{3.310308in}}%
\pgfpathcurveto{\pgfqpoint{1.921704in}{3.301100in}}{\pgfqpoint{1.934195in}{3.295926in}}{\pgfqpoint{1.947218in}{3.295926in}}%
\pgfpathlineto{\pgfqpoint{1.947218in}{3.295926in}}%
\pgfpathclose%
\pgfusepath{stroke,fill}%
\end{pgfscope}%
\begin{pgfscope}%
\pgfpathrectangle{\pgfqpoint{0.786164in}{0.768110in}}{\pgfqpoint{8.851069in}{7.081890in}}%
\pgfusepath{clip}%
\pgfsetbuttcap%
\pgfsetroundjoin%
\definecolor{currentfill}{rgb}{0.258965,0.251537,0.524736}%
\pgfsetfillcolor{currentfill}%
\pgfsetfillopacity{0.700000}%
\pgfsetlinewidth{0.501875pt}%
\definecolor{currentstroke}{rgb}{1.000000,1.000000,1.000000}%
\pgfsetstrokecolor{currentstroke}%
\pgfsetstrokeopacity{0.700000}%
\pgfsetdash{}{0pt}%
\pgfpathmoveto{\pgfqpoint{2.097768in}{3.274584in}}%
\pgfpathcurveto{\pgfqpoint{2.110791in}{3.274584in}}{\pgfqpoint{2.123282in}{3.279758in}}{\pgfqpoint{2.132490in}{3.288966in}}%
\pgfpathcurveto{\pgfqpoint{2.141699in}{3.298175in}}{\pgfqpoint{2.146873in}{3.310666in}}{\pgfqpoint{2.146873in}{3.323688in}}%
\pgfpathcurveto{\pgfqpoint{2.146873in}{3.336711in}}{\pgfqpoint{2.141699in}{3.349202in}}{\pgfqpoint{2.132490in}{3.358411in}}%
\pgfpathcurveto{\pgfqpoint{2.123282in}{3.367619in}}{\pgfqpoint{2.110791in}{3.372793in}}{\pgfqpoint{2.097768in}{3.372793in}}%
\pgfpathcurveto{\pgfqpoint{2.084745in}{3.372793in}}{\pgfqpoint{2.072254in}{3.367619in}}{\pgfqpoint{2.063046in}{3.358411in}}%
\pgfpathcurveto{\pgfqpoint{2.053837in}{3.349202in}}{\pgfqpoint{2.048663in}{3.336711in}}{\pgfqpoint{2.048663in}{3.323688in}}%
\pgfpathcurveto{\pgfqpoint{2.048663in}{3.310666in}}{\pgfqpoint{2.053837in}{3.298175in}}{\pgfqpoint{2.063046in}{3.288966in}}%
\pgfpathcurveto{\pgfqpoint{2.072254in}{3.279758in}}{\pgfqpoint{2.084745in}{3.274584in}}{\pgfqpoint{2.097768in}{3.274584in}}%
\pgfpathlineto{\pgfqpoint{2.097768in}{3.274584in}}%
\pgfpathclose%
\pgfusepath{stroke,fill}%
\end{pgfscope}%
\begin{pgfscope}%
\pgfpathrectangle{\pgfqpoint{0.786164in}{0.768110in}}{\pgfqpoint{8.851069in}{7.081890in}}%
\pgfusepath{clip}%
\pgfsetbuttcap%
\pgfsetroundjoin%
\definecolor{currentfill}{rgb}{0.260571,0.246922,0.522828}%
\pgfsetfillcolor{currentfill}%
\pgfsetfillopacity{0.700000}%
\pgfsetlinewidth{0.501875pt}%
\definecolor{currentstroke}{rgb}{1.000000,1.000000,1.000000}%
\pgfsetstrokecolor{currentstroke}%
\pgfsetstrokeopacity{0.700000}%
\pgfsetdash{}{0pt}%
\pgfpathmoveto{\pgfqpoint{2.192884in}{3.231899in}}%
\pgfpathcurveto{\pgfqpoint{2.205907in}{3.231899in}}{\pgfqpoint{2.218398in}{3.237073in}}{\pgfqpoint{2.227607in}{3.246282in}}%
\pgfpathcurveto{\pgfqpoint{2.236815in}{3.255490in}}{\pgfqpoint{2.241989in}{3.267981in}}{\pgfqpoint{2.241989in}{3.281004in}}%
\pgfpathcurveto{\pgfqpoint{2.241989in}{3.294027in}}{\pgfqpoint{2.236815in}{3.306518in}}{\pgfqpoint{2.227607in}{3.315726in}}%
\pgfpathcurveto{\pgfqpoint{2.218398in}{3.324935in}}{\pgfqpoint{2.205907in}{3.330109in}}{\pgfqpoint{2.192884in}{3.330109in}}%
\pgfpathcurveto{\pgfqpoint{2.179862in}{3.330109in}}{\pgfqpoint{2.167371in}{3.324935in}}{\pgfqpoint{2.158162in}{3.315726in}}%
\pgfpathcurveto{\pgfqpoint{2.148954in}{3.306518in}}{\pgfqpoint{2.143780in}{3.294027in}}{\pgfqpoint{2.143780in}{3.281004in}}%
\pgfpathcurveto{\pgfqpoint{2.143780in}{3.267981in}}{\pgfqpoint{2.148954in}{3.255490in}}{\pgfqpoint{2.158162in}{3.246282in}}%
\pgfpathcurveto{\pgfqpoint{2.167371in}{3.237073in}}{\pgfqpoint{2.179862in}{3.231899in}}{\pgfqpoint{2.192884in}{3.231899in}}%
\pgfpathlineto{\pgfqpoint{2.192884in}{3.231899in}}%
\pgfpathclose%
\pgfusepath{stroke,fill}%
\end{pgfscope}%
\begin{pgfscope}%
\pgfpathrectangle{\pgfqpoint{0.786164in}{0.768110in}}{\pgfqpoint{8.851069in}{7.081890in}}%
\pgfusepath{clip}%
\pgfsetbuttcap%
\pgfsetroundjoin%
\definecolor{currentfill}{rgb}{0.260571,0.246922,0.522828}%
\pgfsetfillcolor{currentfill}%
\pgfsetfillopacity{0.700000}%
\pgfsetlinewidth{0.501875pt}%
\definecolor{currentstroke}{rgb}{1.000000,1.000000,1.000000}%
\pgfsetstrokecolor{currentstroke}%
\pgfsetstrokeopacity{0.700000}%
\pgfsetdash{}{0pt}%
\pgfpathmoveto{\pgfqpoint{2.373471in}{3.295926in}}%
\pgfpathcurveto{\pgfqpoint{2.386494in}{3.295926in}}{\pgfqpoint{2.398985in}{3.301100in}}{\pgfqpoint{2.408193in}{3.310308in}}%
\pgfpathcurveto{\pgfqpoint{2.417402in}{3.319517in}}{\pgfqpoint{2.422576in}{3.332008in}}{\pgfqpoint{2.422576in}{3.345030in}}%
\pgfpathcurveto{\pgfqpoint{2.422576in}{3.358053in}}{\pgfqpoint{2.417402in}{3.370544in}}{\pgfqpoint{2.408193in}{3.379753in}}%
\pgfpathcurveto{\pgfqpoint{2.398985in}{3.388961in}}{\pgfqpoint{2.386494in}{3.394135in}}{\pgfqpoint{2.373471in}{3.394135in}}%
\pgfpathcurveto{\pgfqpoint{2.360448in}{3.394135in}}{\pgfqpoint{2.347957in}{3.388961in}}{\pgfqpoint{2.338749in}{3.379753in}}%
\pgfpathcurveto{\pgfqpoint{2.329540in}{3.370544in}}{\pgfqpoint{2.324366in}{3.358053in}}{\pgfqpoint{2.324366in}{3.345030in}}%
\pgfpathcurveto{\pgfqpoint{2.324366in}{3.332008in}}{\pgfqpoint{2.329540in}{3.319517in}}{\pgfqpoint{2.338749in}{3.310308in}}%
\pgfpathcurveto{\pgfqpoint{2.347957in}{3.301100in}}{\pgfqpoint{2.360448in}{3.295926in}}{\pgfqpoint{2.373471in}{3.295926in}}%
\pgfpathlineto{\pgfqpoint{2.373471in}{3.295926in}}%
\pgfpathclose%
\pgfusepath{stroke,fill}%
\end{pgfscope}%
\begin{pgfscope}%
\pgfpathrectangle{\pgfqpoint{0.786164in}{0.768110in}}{\pgfqpoint{8.851069in}{7.081890in}}%
\pgfusepath{clip}%
\pgfsetbuttcap%
\pgfsetroundjoin%
\definecolor{currentfill}{rgb}{0.263663,0.237631,0.518762}%
\pgfsetfillcolor{currentfill}%
\pgfsetfillopacity{0.700000}%
\pgfsetlinewidth{0.501875pt}%
\definecolor{currentstroke}{rgb}{1.000000,1.000000,1.000000}%
\pgfsetstrokecolor{currentstroke}%
\pgfsetstrokeopacity{0.700000}%
\pgfsetdash{}{0pt}%
\pgfpathmoveto{\pgfqpoint{2.579210in}{3.167873in}}%
\pgfpathcurveto{\pgfqpoint{2.592233in}{3.167873in}}{\pgfqpoint{2.604724in}{3.173047in}}{\pgfqpoint{2.613933in}{3.182255in}}%
\pgfpathcurveto{\pgfqpoint{2.623141in}{3.191464in}}{\pgfqpoint{2.628315in}{3.203955in}}{\pgfqpoint{2.628315in}{3.216977in}}%
\pgfpathcurveto{\pgfqpoint{2.628315in}{3.230000in}}{\pgfqpoint{2.623141in}{3.242491in}}{\pgfqpoint{2.613933in}{3.251700in}}%
\pgfpathcurveto{\pgfqpoint{2.604724in}{3.260908in}}{\pgfqpoint{2.592233in}{3.266082in}}{\pgfqpoint{2.579210in}{3.266082in}}%
\pgfpathcurveto{\pgfqpoint{2.566188in}{3.266082in}}{\pgfqpoint{2.553697in}{3.260908in}}{\pgfqpoint{2.544488in}{3.251700in}}%
\pgfpathcurveto{\pgfqpoint{2.535280in}{3.242491in}}{\pgfqpoint{2.530106in}{3.230000in}}{\pgfqpoint{2.530106in}{3.216977in}}%
\pgfpathcurveto{\pgfqpoint{2.530106in}{3.203955in}}{\pgfqpoint{2.535280in}{3.191464in}}{\pgfqpoint{2.544488in}{3.182255in}}%
\pgfpathcurveto{\pgfqpoint{2.553697in}{3.173047in}}{\pgfqpoint{2.566188in}{3.167873in}}{\pgfqpoint{2.579210in}{3.167873in}}%
\pgfpathlineto{\pgfqpoint{2.579210in}{3.167873in}}%
\pgfpathclose%
\pgfusepath{stroke,fill}%
\end{pgfscope}%
\begin{pgfscope}%
\pgfpathrectangle{\pgfqpoint{0.786164in}{0.768110in}}{\pgfqpoint{8.851069in}{7.081890in}}%
\pgfusepath{clip}%
\pgfsetbuttcap%
\pgfsetroundjoin%
\definecolor{currentfill}{rgb}{0.273006,0.204520,0.501721}%
\pgfsetfillcolor{currentfill}%
\pgfsetfillopacity{0.700000}%
\pgfsetlinewidth{0.501875pt}%
\definecolor{currentstroke}{rgb}{1.000000,1.000000,1.000000}%
\pgfsetstrokecolor{currentstroke}%
\pgfsetstrokeopacity{0.700000}%
\pgfsetdash{}{0pt}%
\pgfpathmoveto{\pgfqpoint{2.735865in}{2.933109in}}%
\pgfpathcurveto{\pgfqpoint{2.748888in}{2.933109in}}{\pgfqpoint{2.761379in}{2.938283in}}{\pgfqpoint{2.770588in}{2.947491in}}%
\pgfpathcurveto{\pgfqpoint{2.779796in}{2.956700in}}{\pgfqpoint{2.784970in}{2.969191in}}{\pgfqpoint{2.784970in}{2.982214in}}%
\pgfpathcurveto{\pgfqpoint{2.784970in}{2.995236in}}{\pgfqpoint{2.779796in}{3.007727in}}{\pgfqpoint{2.770588in}{3.016936in}}%
\pgfpathcurveto{\pgfqpoint{2.761379in}{3.026144in}}{\pgfqpoint{2.748888in}{3.031318in}}{\pgfqpoint{2.735865in}{3.031318in}}%
\pgfpathcurveto{\pgfqpoint{2.722843in}{3.031318in}}{\pgfqpoint{2.710352in}{3.026144in}}{\pgfqpoint{2.701143in}{3.016936in}}%
\pgfpathcurveto{\pgfqpoint{2.691935in}{3.007727in}}{\pgfqpoint{2.686761in}{2.995236in}}{\pgfqpoint{2.686761in}{2.982214in}}%
\pgfpathcurveto{\pgfqpoint{2.686761in}{2.969191in}}{\pgfqpoint{2.691935in}{2.956700in}}{\pgfqpoint{2.701143in}{2.947491in}}%
\pgfpathcurveto{\pgfqpoint{2.710352in}{2.938283in}}{\pgfqpoint{2.722843in}{2.933109in}}{\pgfqpoint{2.735865in}{2.933109in}}%
\pgfpathlineto{\pgfqpoint{2.735865in}{2.933109in}}%
\pgfpathclose%
\pgfusepath{stroke,fill}%
\end{pgfscope}%
\begin{pgfscope}%
\pgfpathrectangle{\pgfqpoint{0.786164in}{0.768110in}}{\pgfqpoint{8.851069in}{7.081890in}}%
\pgfusepath{clip}%
\pgfsetbuttcap%
\pgfsetroundjoin%
\definecolor{currentfill}{rgb}{0.270595,0.214069,0.507052}%
\pgfsetfillcolor{currentfill}%
\pgfsetfillopacity{0.700000}%
\pgfsetlinewidth{0.501875pt}%
\definecolor{currentstroke}{rgb}{1.000000,1.000000,1.000000}%
\pgfsetstrokecolor{currentstroke}%
\pgfsetstrokeopacity{0.700000}%
\pgfsetdash{}{0pt}%
\pgfpathmoveto{\pgfqpoint{2.766146in}{2.954451in}}%
\pgfpathcurveto{\pgfqpoint{2.779169in}{2.954451in}}{\pgfqpoint{2.791660in}{2.959625in}}{\pgfqpoint{2.800869in}{2.968834in}}%
\pgfpathcurveto{\pgfqpoint{2.810077in}{2.978042in}}{\pgfqpoint{2.815251in}{2.990533in}}{\pgfqpoint{2.815251in}{3.003556in}}%
\pgfpathcurveto{\pgfqpoint{2.815251in}{3.016578in}}{\pgfqpoint{2.810077in}{3.029070in}}{\pgfqpoint{2.800869in}{3.038278in}}%
\pgfpathcurveto{\pgfqpoint{2.791660in}{3.047486in}}{\pgfqpoint{2.779169in}{3.052660in}}{\pgfqpoint{2.766146in}{3.052660in}}%
\pgfpathcurveto{\pgfqpoint{2.753124in}{3.052660in}}{\pgfqpoint{2.740633in}{3.047486in}}{\pgfqpoint{2.731424in}{3.038278in}}%
\pgfpathcurveto{\pgfqpoint{2.722216in}{3.029070in}}{\pgfqpoint{2.717042in}{3.016578in}}{\pgfqpoint{2.717042in}{3.003556in}}%
\pgfpathcurveto{\pgfqpoint{2.717042in}{2.990533in}}{\pgfqpoint{2.722216in}{2.978042in}}{\pgfqpoint{2.731424in}{2.968834in}}%
\pgfpathcurveto{\pgfqpoint{2.740633in}{2.959625in}}{\pgfqpoint{2.753124in}{2.954451in}}{\pgfqpoint{2.766146in}{2.954451in}}%
\pgfpathlineto{\pgfqpoint{2.766146in}{2.954451in}}%
\pgfpathclose%
\pgfusepath{stroke,fill}%
\end{pgfscope}%
\begin{pgfscope}%
\pgfpathrectangle{\pgfqpoint{0.786164in}{0.768110in}}{\pgfqpoint{8.851069in}{7.081890in}}%
\pgfusepath{clip}%
\pgfsetbuttcap%
\pgfsetroundjoin%
\definecolor{currentfill}{rgb}{0.274128,0.199721,0.498911}%
\pgfsetfillcolor{currentfill}%
\pgfsetfillopacity{0.700000}%
\pgfsetlinewidth{0.501875pt}%
\definecolor{currentstroke}{rgb}{1.000000,1.000000,1.000000}%
\pgfsetstrokecolor{currentstroke}%
\pgfsetstrokeopacity{0.700000}%
\pgfsetdash{}{0pt}%
\pgfpathmoveto{\pgfqpoint{2.748808in}{2.933109in}}%
\pgfpathcurveto{\pgfqpoint{2.761831in}{2.933109in}}{\pgfqpoint{2.774322in}{2.938283in}}{\pgfqpoint{2.783530in}{2.947491in}}%
\pgfpathcurveto{\pgfqpoint{2.792739in}{2.956700in}}{\pgfqpoint{2.797913in}{2.969191in}}{\pgfqpoint{2.797913in}{2.982214in}}%
\pgfpathcurveto{\pgfqpoint{2.797913in}{2.995236in}}{\pgfqpoint{2.792739in}{3.007727in}}{\pgfqpoint{2.783530in}{3.016936in}}%
\pgfpathcurveto{\pgfqpoint{2.774322in}{3.026144in}}{\pgfqpoint{2.761831in}{3.031318in}}{\pgfqpoint{2.748808in}{3.031318in}}%
\pgfpathcurveto{\pgfqpoint{2.735785in}{3.031318in}}{\pgfqpoint{2.723294in}{3.026144in}}{\pgfqpoint{2.714086in}{3.016936in}}%
\pgfpathcurveto{\pgfqpoint{2.704877in}{3.007727in}}{\pgfqpoint{2.699703in}{2.995236in}}{\pgfqpoint{2.699703in}{2.982214in}}%
\pgfpathcurveto{\pgfqpoint{2.699703in}{2.969191in}}{\pgfqpoint{2.704877in}{2.956700in}}{\pgfqpoint{2.714086in}{2.947491in}}%
\pgfpathcurveto{\pgfqpoint{2.723294in}{2.938283in}}{\pgfqpoint{2.735785in}{2.933109in}}{\pgfqpoint{2.748808in}{2.933109in}}%
\pgfpathlineto{\pgfqpoint{2.748808in}{2.933109in}}%
\pgfpathclose%
\pgfusepath{stroke,fill}%
\end{pgfscope}%
\begin{pgfscope}%
\pgfpathrectangle{\pgfqpoint{0.786164in}{0.768110in}}{\pgfqpoint{8.851069in}{7.081890in}}%
\pgfusepath{clip}%
\pgfsetbuttcap%
\pgfsetroundjoin%
\definecolor{currentfill}{rgb}{0.277134,0.185228,0.489898}%
\pgfsetfillcolor{currentfill}%
\pgfsetfillopacity{0.700000}%
\pgfsetlinewidth{0.501875pt}%
\definecolor{currentstroke}{rgb}{1.000000,1.000000,1.000000}%
\pgfsetstrokecolor{currentstroke}%
\pgfsetstrokeopacity{0.700000}%
\pgfsetdash{}{0pt}%
\pgfpathmoveto{\pgfqpoint{3.061386in}{2.890425in}}%
\pgfpathcurveto{\pgfqpoint{3.074408in}{2.890425in}}{\pgfqpoint{3.086899in}{2.895599in}}{\pgfqpoint{3.096108in}{2.904807in}}%
\pgfpathcurveto{\pgfqpoint{3.105316in}{2.914015in}}{\pgfqpoint{3.110490in}{2.926507in}}{\pgfqpoint{3.110490in}{2.939529in}}%
\pgfpathcurveto{\pgfqpoint{3.110490in}{2.952552in}}{\pgfqpoint{3.105316in}{2.965043in}}{\pgfqpoint{3.096108in}{2.974251in}}%
\pgfpathcurveto{\pgfqpoint{3.086899in}{2.983460in}}{\pgfqpoint{3.074408in}{2.988634in}}{\pgfqpoint{3.061386in}{2.988634in}}%
\pgfpathcurveto{\pgfqpoint{3.048363in}{2.988634in}}{\pgfqpoint{3.035872in}{2.983460in}}{\pgfqpoint{3.026663in}{2.974251in}}%
\pgfpathcurveto{\pgfqpoint{3.017455in}{2.965043in}}{\pgfqpoint{3.012281in}{2.952552in}}{\pgfqpoint{3.012281in}{2.939529in}}%
\pgfpathcurveto{\pgfqpoint{3.012281in}{2.926507in}}{\pgfqpoint{3.017455in}{2.914015in}}{\pgfqpoint{3.026663in}{2.904807in}}%
\pgfpathcurveto{\pgfqpoint{3.035872in}{2.895599in}}{\pgfqpoint{3.048363in}{2.890425in}}{\pgfqpoint{3.061386in}{2.890425in}}%
\pgfpathlineto{\pgfqpoint{3.061386in}{2.890425in}}%
\pgfpathclose%
\pgfusepath{stroke,fill}%
\end{pgfscope}%
\begin{pgfscope}%
\pgfpathrectangle{\pgfqpoint{0.786164in}{0.768110in}}{\pgfqpoint{8.851069in}{7.081890in}}%
\pgfusepath{clip}%
\pgfsetbuttcap%
\pgfsetroundjoin%
\definecolor{currentfill}{rgb}{0.279574,0.170599,0.479997}%
\pgfsetfillcolor{currentfill}%
\pgfsetfillopacity{0.700000}%
\pgfsetlinewidth{0.501875pt}%
\definecolor{currentstroke}{rgb}{1.000000,1.000000,1.000000}%
\pgfsetstrokecolor{currentstroke}%
\pgfsetstrokeopacity{0.700000}%
\pgfsetdash{}{0pt}%
\pgfpathmoveto{\pgfqpoint{3.220116in}{2.677003in}}%
\pgfpathcurveto{\pgfqpoint{3.233139in}{2.677003in}}{\pgfqpoint{3.245630in}{2.682177in}}{\pgfqpoint{3.254838in}{2.691385in}}%
\pgfpathcurveto{\pgfqpoint{3.264047in}{2.700594in}}{\pgfqpoint{3.269221in}{2.713085in}}{\pgfqpoint{3.269221in}{2.726108in}}%
\pgfpathcurveto{\pgfqpoint{3.269221in}{2.739130in}}{\pgfqpoint{3.264047in}{2.751621in}}{\pgfqpoint{3.254838in}{2.760830in}}%
\pgfpathcurveto{\pgfqpoint{3.245630in}{2.770038in}}{\pgfqpoint{3.233139in}{2.775212in}}{\pgfqpoint{3.220116in}{2.775212in}}%
\pgfpathcurveto{\pgfqpoint{3.207094in}{2.775212in}}{\pgfqpoint{3.194602in}{2.770038in}}{\pgfqpoint{3.185394in}{2.760830in}}%
\pgfpathcurveto{\pgfqpoint{3.176186in}{2.751621in}}{\pgfqpoint{3.171012in}{2.739130in}}{\pgfqpoint{3.171012in}{2.726108in}}%
\pgfpathcurveto{\pgfqpoint{3.171012in}{2.713085in}}{\pgfqpoint{3.176186in}{2.700594in}}{\pgfqpoint{3.185394in}{2.691385in}}%
\pgfpathcurveto{\pgfqpoint{3.194602in}{2.682177in}}{\pgfqpoint{3.207094in}{2.677003in}}{\pgfqpoint{3.220116in}{2.677003in}}%
\pgfpathlineto{\pgfqpoint{3.220116in}{2.677003in}}%
\pgfpathclose%
\pgfusepath{stroke,fill}%
\end{pgfscope}%
\begin{pgfscope}%
\pgfpathrectangle{\pgfqpoint{0.786164in}{0.768110in}}{\pgfqpoint{8.851069in}{7.081890in}}%
\pgfusepath{clip}%
\pgfsetbuttcap%
\pgfsetroundjoin%
\definecolor{currentfill}{rgb}{0.282290,0.145912,0.461510}%
\pgfsetfillcolor{currentfill}%
\pgfsetfillopacity{0.700000}%
\pgfsetlinewidth{0.501875pt}%
\definecolor{currentstroke}{rgb}{1.000000,1.000000,1.000000}%
\pgfsetstrokecolor{currentstroke}%
\pgfsetstrokeopacity{0.700000}%
\pgfsetdash{}{0pt}%
\pgfpathmoveto{\pgfqpoint{3.261753in}{2.591634in}}%
\pgfpathcurveto{\pgfqpoint{3.274775in}{2.591634in}}{\pgfqpoint{3.287266in}{2.596808in}}{\pgfqpoint{3.296475in}{2.606017in}}%
\pgfpathcurveto{\pgfqpoint{3.305683in}{2.615225in}}{\pgfqpoint{3.310857in}{2.627716in}}{\pgfqpoint{3.310857in}{2.640739in}}%
\pgfpathcurveto{\pgfqpoint{3.310857in}{2.653762in}}{\pgfqpoint{3.305683in}{2.666253in}}{\pgfqpoint{3.296475in}{2.675461in}}%
\pgfpathcurveto{\pgfqpoint{3.287266in}{2.684670in}}{\pgfqpoint{3.274775in}{2.689844in}}{\pgfqpoint{3.261753in}{2.689844in}}%
\pgfpathcurveto{\pgfqpoint{3.248730in}{2.689844in}}{\pgfqpoint{3.236239in}{2.684670in}}{\pgfqpoint{3.227030in}{2.675461in}}%
\pgfpathcurveto{\pgfqpoint{3.217822in}{2.666253in}}{\pgfqpoint{3.212648in}{2.653762in}}{\pgfqpoint{3.212648in}{2.640739in}}%
\pgfpathcurveto{\pgfqpoint{3.212648in}{2.627716in}}{\pgfqpoint{3.217822in}{2.615225in}}{\pgfqpoint{3.227030in}{2.606017in}}%
\pgfpathcurveto{\pgfqpoint{3.236239in}{2.596808in}}{\pgfqpoint{3.248730in}{2.591634in}}{\pgfqpoint{3.261753in}{2.591634in}}%
\pgfpathlineto{\pgfqpoint{3.261753in}{2.591634in}}%
\pgfpathclose%
\pgfusepath{stroke,fill}%
\end{pgfscope}%
\begin{pgfscope}%
\pgfpathrectangle{\pgfqpoint{0.786164in}{0.768110in}}{\pgfqpoint{8.851069in}{7.081890in}}%
\pgfusepath{clip}%
\pgfsetbuttcap%
\pgfsetroundjoin%
\definecolor{currentfill}{rgb}{0.280868,0.160771,0.472899}%
\pgfsetfillcolor{currentfill}%
\pgfsetfillopacity{0.700000}%
\pgfsetlinewidth{0.501875pt}%
\definecolor{currentstroke}{rgb}{1.000000,1.000000,1.000000}%
\pgfsetstrokecolor{currentstroke}%
\pgfsetstrokeopacity{0.700000}%
\pgfsetdash{}{0pt}%
\pgfpathmoveto{\pgfqpoint{3.315843in}{2.634319in}}%
\pgfpathcurveto{\pgfqpoint{3.328866in}{2.634319in}}{\pgfqpoint{3.341357in}{2.639493in}}{\pgfqpoint{3.350565in}{2.648701in}}%
\pgfpathcurveto{\pgfqpoint{3.359774in}{2.657909in}}{\pgfqpoint{3.364948in}{2.670401in}}{\pgfqpoint{3.364948in}{2.683423in}}%
\pgfpathcurveto{\pgfqpoint{3.364948in}{2.696446in}}{\pgfqpoint{3.359774in}{2.708937in}}{\pgfqpoint{3.350565in}{2.718145in}}%
\pgfpathcurveto{\pgfqpoint{3.341357in}{2.727354in}}{\pgfqpoint{3.328866in}{2.732528in}}{\pgfqpoint{3.315843in}{2.732528in}}%
\pgfpathcurveto{\pgfqpoint{3.302820in}{2.732528in}}{\pgfqpoint{3.290329in}{2.727354in}}{\pgfqpoint{3.281121in}{2.718145in}}%
\pgfpathcurveto{\pgfqpoint{3.271912in}{2.708937in}}{\pgfqpoint{3.266738in}{2.696446in}}{\pgfqpoint{3.266738in}{2.683423in}}%
\pgfpathcurveto{\pgfqpoint{3.266738in}{2.670401in}}{\pgfqpoint{3.271912in}{2.657909in}}{\pgfqpoint{3.281121in}{2.648701in}}%
\pgfpathcurveto{\pgfqpoint{3.290329in}{2.639493in}}{\pgfqpoint{3.302820in}{2.634319in}}{\pgfqpoint{3.315843in}{2.634319in}}%
\pgfpathlineto{\pgfqpoint{3.315843in}{2.634319in}}%
\pgfpathclose%
\pgfusepath{stroke,fill}%
\end{pgfscope}%
\begin{pgfscope}%
\pgfpathrectangle{\pgfqpoint{0.786164in}{0.768110in}}{\pgfqpoint{8.851069in}{7.081890in}}%
\pgfusepath{clip}%
\pgfsetbuttcap%
\pgfsetroundjoin%
\definecolor{currentfill}{rgb}{0.280868,0.160771,0.472899}%
\pgfsetfillcolor{currentfill}%
\pgfsetfillopacity{0.700000}%
\pgfsetlinewidth{0.501875pt}%
\definecolor{currentstroke}{rgb}{1.000000,1.000000,1.000000}%
\pgfsetstrokecolor{currentstroke}%
\pgfsetstrokeopacity{0.700000}%
\pgfsetdash{}{0pt}%
\pgfpathmoveto{\pgfqpoint{3.302412in}{2.612976in}}%
\pgfpathcurveto{\pgfqpoint{3.315435in}{2.612976in}}{\pgfqpoint{3.327926in}{2.618150in}}{\pgfqpoint{3.337134in}{2.627359in}}%
\pgfpathcurveto{\pgfqpoint{3.346343in}{2.636567in}}{\pgfqpoint{3.351517in}{2.649058in}}{\pgfqpoint{3.351517in}{2.662081in}}%
\pgfpathcurveto{\pgfqpoint{3.351517in}{2.675104in}}{\pgfqpoint{3.346343in}{2.687595in}}{\pgfqpoint{3.337134in}{2.696803in}}%
\pgfpathcurveto{\pgfqpoint{3.327926in}{2.706012in}}{\pgfqpoint{3.315435in}{2.711186in}}{\pgfqpoint{3.302412in}{2.711186in}}%
\pgfpathcurveto{\pgfqpoint{3.289389in}{2.711186in}}{\pgfqpoint{3.276898in}{2.706012in}}{\pgfqpoint{3.267690in}{2.696803in}}%
\pgfpathcurveto{\pgfqpoint{3.258481in}{2.687595in}}{\pgfqpoint{3.253307in}{2.675104in}}{\pgfqpoint{3.253307in}{2.662081in}}%
\pgfpathcurveto{\pgfqpoint{3.253307in}{2.649058in}}{\pgfqpoint{3.258481in}{2.636567in}}{\pgfqpoint{3.267690in}{2.627359in}}%
\pgfpathcurveto{\pgfqpoint{3.276898in}{2.618150in}}{\pgfqpoint{3.289389in}{2.612976in}}{\pgfqpoint{3.302412in}{2.612976in}}%
\pgfpathlineto{\pgfqpoint{3.302412in}{2.612976in}}%
\pgfpathclose%
\pgfusepath{stroke,fill}%
\end{pgfscope}%
\begin{pgfscope}%
\pgfpathrectangle{\pgfqpoint{0.786164in}{0.768110in}}{\pgfqpoint{8.851069in}{7.081890in}}%
\pgfusepath{clip}%
\pgfsetbuttcap%
\pgfsetroundjoin%
\definecolor{currentfill}{rgb}{0.280868,0.160771,0.472899}%
\pgfsetfillcolor{currentfill}%
\pgfsetfillopacity{0.700000}%
\pgfsetlinewidth{0.501875pt}%
\definecolor{currentstroke}{rgb}{1.000000,1.000000,1.000000}%
\pgfsetstrokecolor{currentstroke}%
\pgfsetstrokeopacity{0.700000}%
\pgfsetdash{}{0pt}%
\pgfpathmoveto{\pgfqpoint{3.406442in}{2.677003in}}%
\pgfpathcurveto{\pgfqpoint{3.419464in}{2.677003in}}{\pgfqpoint{3.431955in}{2.682177in}}{\pgfqpoint{3.441164in}{2.691385in}}%
\pgfpathcurveto{\pgfqpoint{3.450372in}{2.700594in}}{\pgfqpoint{3.455546in}{2.713085in}}{\pgfqpoint{3.455546in}{2.726108in}}%
\pgfpathcurveto{\pgfqpoint{3.455546in}{2.739130in}}{\pgfqpoint{3.450372in}{2.751621in}}{\pgfqpoint{3.441164in}{2.760830in}}%
\pgfpathcurveto{\pgfqpoint{3.431955in}{2.770038in}}{\pgfqpoint{3.419464in}{2.775212in}}{\pgfqpoint{3.406442in}{2.775212in}}%
\pgfpathcurveto{\pgfqpoint{3.393419in}{2.775212in}}{\pgfqpoint{3.380928in}{2.770038in}}{\pgfqpoint{3.371719in}{2.760830in}}%
\pgfpathcurveto{\pgfqpoint{3.362511in}{2.751621in}}{\pgfqpoint{3.357337in}{2.739130in}}{\pgfqpoint{3.357337in}{2.726108in}}%
\pgfpathcurveto{\pgfqpoint{3.357337in}{2.713085in}}{\pgfqpoint{3.362511in}{2.700594in}}{\pgfqpoint{3.371719in}{2.691385in}}%
\pgfpathcurveto{\pgfqpoint{3.380928in}{2.682177in}}{\pgfqpoint{3.393419in}{2.677003in}}{\pgfqpoint{3.406442in}{2.677003in}}%
\pgfpathlineto{\pgfqpoint{3.406442in}{2.677003in}}%
\pgfpathclose%
\pgfusepath{stroke,fill}%
\end{pgfscope}%
\begin{pgfscope}%
\pgfpathrectangle{\pgfqpoint{0.786164in}{0.768110in}}{\pgfqpoint{8.851069in}{7.081890in}}%
\pgfusepath{clip}%
\pgfsetbuttcap%
\pgfsetroundjoin%
\definecolor{currentfill}{rgb}{0.280868,0.160771,0.472899}%
\pgfsetfillcolor{currentfill}%
\pgfsetfillopacity{0.700000}%
\pgfsetlinewidth{0.501875pt}%
\definecolor{currentstroke}{rgb}{1.000000,1.000000,1.000000}%
\pgfsetstrokecolor{currentstroke}%
\pgfsetstrokeopacity{0.700000}%
\pgfsetdash{}{0pt}%
\pgfpathmoveto{\pgfqpoint{3.348932in}{2.591634in}}%
\pgfpathcurveto{\pgfqpoint{3.361955in}{2.591634in}}{\pgfqpoint{3.374446in}{2.596808in}}{\pgfqpoint{3.383655in}{2.606017in}}%
\pgfpathcurveto{\pgfqpoint{3.392863in}{2.615225in}}{\pgfqpoint{3.398037in}{2.627716in}}{\pgfqpoint{3.398037in}{2.640739in}}%
\pgfpathcurveto{\pgfqpoint{3.398037in}{2.653762in}}{\pgfqpoint{3.392863in}{2.666253in}}{\pgfqpoint{3.383655in}{2.675461in}}%
\pgfpathcurveto{\pgfqpoint{3.374446in}{2.684670in}}{\pgfqpoint{3.361955in}{2.689844in}}{\pgfqpoint{3.348932in}{2.689844in}}%
\pgfpathcurveto{\pgfqpoint{3.335910in}{2.689844in}}{\pgfqpoint{3.323419in}{2.684670in}}{\pgfqpoint{3.314210in}{2.675461in}}%
\pgfpathcurveto{\pgfqpoint{3.305002in}{2.666253in}}{\pgfqpoint{3.299828in}{2.653762in}}{\pgfqpoint{3.299828in}{2.640739in}}%
\pgfpathcurveto{\pgfqpoint{3.299828in}{2.627716in}}{\pgfqpoint{3.305002in}{2.615225in}}{\pgfqpoint{3.314210in}{2.606017in}}%
\pgfpathcurveto{\pgfqpoint{3.323419in}{2.596808in}}{\pgfqpoint{3.335910in}{2.591634in}}{\pgfqpoint{3.348932in}{2.591634in}}%
\pgfpathlineto{\pgfqpoint{3.348932in}{2.591634in}}%
\pgfpathclose%
\pgfusepath{stroke,fill}%
\end{pgfscope}%
\begin{pgfscope}%
\pgfpathrectangle{\pgfqpoint{0.786164in}{0.768110in}}{\pgfqpoint{8.851069in}{7.081890in}}%
\pgfusepath{clip}%
\pgfsetbuttcap%
\pgfsetroundjoin%
\definecolor{currentfill}{rgb}{0.280868,0.160771,0.472899}%
\pgfsetfillcolor{currentfill}%
\pgfsetfillopacity{0.700000}%
\pgfsetlinewidth{0.501875pt}%
\definecolor{currentstroke}{rgb}{1.000000,1.000000,1.000000}%
\pgfsetstrokecolor{currentstroke}%
\pgfsetstrokeopacity{0.700000}%
\pgfsetdash{}{0pt}%
\pgfpathmoveto{\pgfqpoint{3.395941in}{2.591634in}}%
\pgfpathcurveto{\pgfqpoint{3.408964in}{2.591634in}}{\pgfqpoint{3.421455in}{2.596808in}}{\pgfqpoint{3.430663in}{2.606017in}}%
\pgfpathcurveto{\pgfqpoint{3.439872in}{2.615225in}}{\pgfqpoint{3.445046in}{2.627716in}}{\pgfqpoint{3.445046in}{2.640739in}}%
\pgfpathcurveto{\pgfqpoint{3.445046in}{2.653762in}}{\pgfqpoint{3.439872in}{2.666253in}}{\pgfqpoint{3.430663in}{2.675461in}}%
\pgfpathcurveto{\pgfqpoint{3.421455in}{2.684670in}}{\pgfqpoint{3.408964in}{2.689844in}}{\pgfqpoint{3.395941in}{2.689844in}}%
\pgfpathcurveto{\pgfqpoint{3.382918in}{2.689844in}}{\pgfqpoint{3.370427in}{2.684670in}}{\pgfqpoint{3.361219in}{2.675461in}}%
\pgfpathcurveto{\pgfqpoint{3.352010in}{2.666253in}}{\pgfqpoint{3.346836in}{2.653762in}}{\pgfqpoint{3.346836in}{2.640739in}}%
\pgfpathcurveto{\pgfqpoint{3.346836in}{2.627716in}}{\pgfqpoint{3.352010in}{2.615225in}}{\pgfqpoint{3.361219in}{2.606017in}}%
\pgfpathcurveto{\pgfqpoint{3.370427in}{2.596808in}}{\pgfqpoint{3.382918in}{2.591634in}}{\pgfqpoint{3.395941in}{2.591634in}}%
\pgfpathlineto{\pgfqpoint{3.395941in}{2.591634in}}%
\pgfpathclose%
\pgfusepath{stroke,fill}%
\end{pgfscope}%
\begin{pgfscope}%
\pgfpathrectangle{\pgfqpoint{0.786164in}{0.768110in}}{\pgfqpoint{8.851069in}{7.081890in}}%
\pgfusepath{clip}%
\pgfsetbuttcap%
\pgfsetroundjoin%
\definecolor{currentfill}{rgb}{0.283187,0.125848,0.444960}%
\pgfsetfillcolor{currentfill}%
\pgfsetfillopacity{0.700000}%
\pgfsetlinewidth{0.501875pt}%
\definecolor{currentstroke}{rgb}{1.000000,1.000000,1.000000}%
\pgfsetstrokecolor{currentstroke}%
\pgfsetstrokeopacity{0.700000}%
\pgfsetdash{}{0pt}%
\pgfpathmoveto{\pgfqpoint{3.661876in}{2.463581in}}%
\pgfpathcurveto{\pgfqpoint{3.674899in}{2.463581in}}{\pgfqpoint{3.687390in}{2.468755in}}{\pgfqpoint{3.696598in}{2.477964in}}%
\pgfpathcurveto{\pgfqpoint{3.705807in}{2.487172in}}{\pgfqpoint{3.710981in}{2.499663in}}{\pgfqpoint{3.710981in}{2.512686in}}%
\pgfpathcurveto{\pgfqpoint{3.710981in}{2.525709in}}{\pgfqpoint{3.705807in}{2.538200in}}{\pgfqpoint{3.696598in}{2.547408in}}%
\pgfpathcurveto{\pgfqpoint{3.687390in}{2.556617in}}{\pgfqpoint{3.674899in}{2.561790in}}{\pgfqpoint{3.661876in}{2.561790in}}%
\pgfpathcurveto{\pgfqpoint{3.648853in}{2.561790in}}{\pgfqpoint{3.636362in}{2.556617in}}{\pgfqpoint{3.627154in}{2.547408in}}%
\pgfpathcurveto{\pgfqpoint{3.617945in}{2.538200in}}{\pgfqpoint{3.612771in}{2.525709in}}{\pgfqpoint{3.612771in}{2.512686in}}%
\pgfpathcurveto{\pgfqpoint{3.612771in}{2.499663in}}{\pgfqpoint{3.617945in}{2.487172in}}{\pgfqpoint{3.627154in}{2.477964in}}%
\pgfpathcurveto{\pgfqpoint{3.636362in}{2.468755in}}{\pgfqpoint{3.648853in}{2.463581in}}{\pgfqpoint{3.661876in}{2.463581in}}%
\pgfpathlineto{\pgfqpoint{3.661876in}{2.463581in}}%
\pgfpathclose%
\pgfusepath{stroke,fill}%
\end{pgfscope}%
\begin{pgfscope}%
\pgfpathrectangle{\pgfqpoint{0.786164in}{0.768110in}}{\pgfqpoint{8.851069in}{7.081890in}}%
\pgfusepath{clip}%
\pgfsetbuttcap%
\pgfsetroundjoin%
\definecolor{currentfill}{rgb}{0.280868,0.160771,0.472899}%
\pgfsetfillcolor{currentfill}%
\pgfsetfillopacity{0.700000}%
\pgfsetlinewidth{0.501875pt}%
\definecolor{currentstroke}{rgb}{1.000000,1.000000,1.000000}%
\pgfsetstrokecolor{currentstroke}%
\pgfsetstrokeopacity{0.700000}%
\pgfsetdash{}{0pt}%
\pgfpathmoveto{\pgfqpoint{3.481656in}{2.612976in}}%
\pgfpathcurveto{\pgfqpoint{3.494678in}{2.612976in}}{\pgfqpoint{3.507169in}{2.618150in}}{\pgfqpoint{3.516378in}{2.627359in}}%
\pgfpathcurveto{\pgfqpoint{3.525586in}{2.636567in}}{\pgfqpoint{3.530760in}{2.649058in}}{\pgfqpoint{3.530760in}{2.662081in}}%
\pgfpathcurveto{\pgfqpoint{3.530760in}{2.675104in}}{\pgfqpoint{3.525586in}{2.687595in}}{\pgfqpoint{3.516378in}{2.696803in}}%
\pgfpathcurveto{\pgfqpoint{3.507169in}{2.706012in}}{\pgfqpoint{3.494678in}{2.711186in}}{\pgfqpoint{3.481656in}{2.711186in}}%
\pgfpathcurveto{\pgfqpoint{3.468633in}{2.711186in}}{\pgfqpoint{3.456142in}{2.706012in}}{\pgfqpoint{3.446933in}{2.696803in}}%
\pgfpathcurveto{\pgfqpoint{3.437725in}{2.687595in}}{\pgfqpoint{3.432551in}{2.675104in}}{\pgfqpoint{3.432551in}{2.662081in}}%
\pgfpathcurveto{\pgfqpoint{3.432551in}{2.649058in}}{\pgfqpoint{3.437725in}{2.636567in}}{\pgfqpoint{3.446933in}{2.627359in}}%
\pgfpathcurveto{\pgfqpoint{3.456142in}{2.618150in}}{\pgfqpoint{3.468633in}{2.612976in}}{\pgfqpoint{3.481656in}{2.612976in}}%
\pgfpathlineto{\pgfqpoint{3.481656in}{2.612976in}}%
\pgfpathclose%
\pgfusepath{stroke,fill}%
\end{pgfscope}%
\begin{pgfscope}%
\pgfpathrectangle{\pgfqpoint{0.786164in}{0.768110in}}{\pgfqpoint{8.851069in}{7.081890in}}%
\pgfusepath{clip}%
\pgfsetbuttcap%
\pgfsetroundjoin%
\definecolor{currentfill}{rgb}{0.282290,0.145912,0.461510}%
\pgfsetfillcolor{currentfill}%
\pgfsetfillopacity{0.700000}%
\pgfsetlinewidth{0.501875pt}%
\definecolor{currentstroke}{rgb}{1.000000,1.000000,1.000000}%
\pgfsetstrokecolor{currentstroke}%
\pgfsetstrokeopacity{0.700000}%
\pgfsetdash{}{0pt}%
\pgfpathmoveto{\pgfqpoint{3.511937in}{2.612976in}}%
\pgfpathcurveto{\pgfqpoint{3.524959in}{2.612976in}}{\pgfqpoint{3.537450in}{2.618150in}}{\pgfqpoint{3.546659in}{2.627359in}}%
\pgfpathcurveto{\pgfqpoint{3.555867in}{2.636567in}}{\pgfqpoint{3.561041in}{2.649058in}}{\pgfqpoint{3.561041in}{2.662081in}}%
\pgfpathcurveto{\pgfqpoint{3.561041in}{2.675104in}}{\pgfqpoint{3.555867in}{2.687595in}}{\pgfqpoint{3.546659in}{2.696803in}}%
\pgfpathcurveto{\pgfqpoint{3.537450in}{2.706012in}}{\pgfqpoint{3.524959in}{2.711186in}}{\pgfqpoint{3.511937in}{2.711186in}}%
\pgfpathcurveto{\pgfqpoint{3.498914in}{2.711186in}}{\pgfqpoint{3.486423in}{2.706012in}}{\pgfqpoint{3.477214in}{2.696803in}}%
\pgfpathcurveto{\pgfqpoint{3.468006in}{2.687595in}}{\pgfqpoint{3.462832in}{2.675104in}}{\pgfqpoint{3.462832in}{2.662081in}}%
\pgfpathcurveto{\pgfqpoint{3.462832in}{2.649058in}}{\pgfqpoint{3.468006in}{2.636567in}}{\pgfqpoint{3.477214in}{2.627359in}}%
\pgfpathcurveto{\pgfqpoint{3.486423in}{2.618150in}}{\pgfqpoint{3.498914in}{2.612976in}}{\pgfqpoint{3.511937in}{2.612976in}}%
\pgfpathlineto{\pgfqpoint{3.511937in}{2.612976in}}%
\pgfpathclose%
\pgfusepath{stroke,fill}%
\end{pgfscope}%
\begin{pgfscope}%
\pgfpathrectangle{\pgfqpoint{0.786164in}{0.768110in}}{\pgfqpoint{8.851069in}{7.081890in}}%
\pgfusepath{clip}%
\pgfsetbuttcap%
\pgfsetroundjoin%
\definecolor{currentfill}{rgb}{0.279574,0.170599,0.479997}%
\pgfsetfillcolor{currentfill}%
\pgfsetfillopacity{0.700000}%
\pgfsetlinewidth{0.501875pt}%
\definecolor{currentstroke}{rgb}{1.000000,1.000000,1.000000}%
\pgfsetstrokecolor{currentstroke}%
\pgfsetstrokeopacity{0.700000}%
\pgfsetdash{}{0pt}%
\pgfpathmoveto{\pgfqpoint{3.278725in}{2.228817in}}%
\pgfpathcurveto{\pgfqpoint{3.291747in}{2.228817in}}{\pgfqpoint{3.304238in}{2.233991in}}{\pgfqpoint{3.313447in}{2.243200in}}%
\pgfpathcurveto{\pgfqpoint{3.322655in}{2.252408in}}{\pgfqpoint{3.327829in}{2.264899in}}{\pgfqpoint{3.327829in}{2.277922in}}%
\pgfpathcurveto{\pgfqpoint{3.327829in}{2.290945in}}{\pgfqpoint{3.322655in}{2.303436in}}{\pgfqpoint{3.313447in}{2.312644in}}%
\pgfpathcurveto{\pgfqpoint{3.304238in}{2.321853in}}{\pgfqpoint{3.291747in}{2.327027in}}{\pgfqpoint{3.278725in}{2.327027in}}%
\pgfpathcurveto{\pgfqpoint{3.265702in}{2.327027in}}{\pgfqpoint{3.253211in}{2.321853in}}{\pgfqpoint{3.244002in}{2.312644in}}%
\pgfpathcurveto{\pgfqpoint{3.234794in}{2.303436in}}{\pgfqpoint{3.229620in}{2.290945in}}{\pgfqpoint{3.229620in}{2.277922in}}%
\pgfpathcurveto{\pgfqpoint{3.229620in}{2.264899in}}{\pgfqpoint{3.234794in}{2.252408in}}{\pgfqpoint{3.244002in}{2.243200in}}%
\pgfpathcurveto{\pgfqpoint{3.253211in}{2.233991in}}{\pgfqpoint{3.265702in}{2.228817in}}{\pgfqpoint{3.278725in}{2.228817in}}%
\pgfpathlineto{\pgfqpoint{3.278725in}{2.228817in}}%
\pgfpathclose%
\pgfusepath{stroke,fill}%
\end{pgfscope}%
\begin{pgfscope}%
\pgfpathrectangle{\pgfqpoint{0.786164in}{0.768110in}}{\pgfqpoint{8.851069in}{7.081890in}}%
\pgfusepath{clip}%
\pgfsetbuttcap%
\pgfsetroundjoin%
\definecolor{currentfill}{rgb}{0.281412,0.155834,0.469201}%
\pgfsetfillcolor{currentfill}%
\pgfsetfillopacity{0.700000}%
\pgfsetlinewidth{0.501875pt}%
\definecolor{currentstroke}{rgb}{1.000000,1.000000,1.000000}%
\pgfsetstrokecolor{currentstroke}%
\pgfsetstrokeopacity{0.700000}%
\pgfsetdash{}{0pt}%
\pgfpathmoveto{\pgfqpoint{3.223413in}{2.356870in}}%
\pgfpathcurveto{\pgfqpoint{3.236436in}{2.356870in}}{\pgfqpoint{3.248927in}{2.362044in}}{\pgfqpoint{3.258135in}{2.371253in}}%
\pgfpathcurveto{\pgfqpoint{3.267344in}{2.380461in}}{\pgfqpoint{3.272518in}{2.392952in}}{\pgfqpoint{3.272518in}{2.405975in}}%
\pgfpathcurveto{\pgfqpoint{3.272518in}{2.418998in}}{\pgfqpoint{3.267344in}{2.431489in}}{\pgfqpoint{3.258135in}{2.440697in}}%
\pgfpathcurveto{\pgfqpoint{3.248927in}{2.449906in}}{\pgfqpoint{3.236436in}{2.455080in}}{\pgfqpoint{3.223413in}{2.455080in}}%
\pgfpathcurveto{\pgfqpoint{3.210390in}{2.455080in}}{\pgfqpoint{3.197899in}{2.449906in}}{\pgfqpoint{3.188691in}{2.440697in}}%
\pgfpathcurveto{\pgfqpoint{3.179482in}{2.431489in}}{\pgfqpoint{3.174308in}{2.418998in}}{\pgfqpoint{3.174308in}{2.405975in}}%
\pgfpathcurveto{\pgfqpoint{3.174308in}{2.392952in}}{\pgfqpoint{3.179482in}{2.380461in}}{\pgfqpoint{3.188691in}{2.371253in}}%
\pgfpathcurveto{\pgfqpoint{3.197899in}{2.362044in}}{\pgfqpoint{3.210390in}{2.356870in}}{\pgfqpoint{3.223413in}{2.356870in}}%
\pgfpathlineto{\pgfqpoint{3.223413in}{2.356870in}}%
\pgfpathclose%
\pgfusepath{stroke,fill}%
\end{pgfscope}%
\begin{pgfscope}%
\pgfpathrectangle{\pgfqpoint{0.786164in}{0.768110in}}{\pgfqpoint{8.851069in}{7.081890in}}%
\pgfusepath{clip}%
\pgfsetbuttcap%
\pgfsetroundjoin%
\definecolor{currentfill}{rgb}{0.281412,0.155834,0.469201}%
\pgfsetfillcolor{currentfill}%
\pgfsetfillopacity{0.700000}%
\pgfsetlinewidth{0.501875pt}%
\definecolor{currentstroke}{rgb}{1.000000,1.000000,1.000000}%
\pgfsetstrokecolor{currentstroke}%
\pgfsetstrokeopacity{0.700000}%
\pgfsetdash{}{0pt}%
\pgfpathmoveto{\pgfqpoint{3.237943in}{2.420897in}}%
\pgfpathcurveto{\pgfqpoint{3.250966in}{2.420897in}}{\pgfqpoint{3.263457in}{2.426071in}}{\pgfqpoint{3.272665in}{2.435279in}}%
\pgfpathcurveto{\pgfqpoint{3.281874in}{2.444488in}}{\pgfqpoint{3.287048in}{2.456979in}}{\pgfqpoint{3.287048in}{2.470002in}}%
\pgfpathcurveto{\pgfqpoint{3.287048in}{2.483024in}}{\pgfqpoint{3.281874in}{2.495515in}}{\pgfqpoint{3.272665in}{2.504724in}}%
\pgfpathcurveto{\pgfqpoint{3.263457in}{2.513932in}}{\pgfqpoint{3.250966in}{2.519106in}}{\pgfqpoint{3.237943in}{2.519106in}}%
\pgfpathcurveto{\pgfqpoint{3.224920in}{2.519106in}}{\pgfqpoint{3.212429in}{2.513932in}}{\pgfqpoint{3.203221in}{2.504724in}}%
\pgfpathcurveto{\pgfqpoint{3.194012in}{2.495515in}}{\pgfqpoint{3.188838in}{2.483024in}}{\pgfqpoint{3.188838in}{2.470002in}}%
\pgfpathcurveto{\pgfqpoint{3.188838in}{2.456979in}}{\pgfqpoint{3.194012in}{2.444488in}}{\pgfqpoint{3.203221in}{2.435279in}}%
\pgfpathcurveto{\pgfqpoint{3.212429in}{2.426071in}}{\pgfqpoint{3.224920in}{2.420897in}}{\pgfqpoint{3.237943in}{2.420897in}}%
\pgfpathlineto{\pgfqpoint{3.237943in}{2.420897in}}%
\pgfpathclose%
\pgfusepath{stroke,fill}%
\end{pgfscope}%
\begin{pgfscope}%
\pgfpathrectangle{\pgfqpoint{0.786164in}{0.768110in}}{\pgfqpoint{8.851069in}{7.081890in}}%
\pgfusepath{clip}%
\pgfsetbuttcap%
\pgfsetroundjoin%
\definecolor{currentfill}{rgb}{0.279574,0.170599,0.479997}%
\pgfsetfillcolor{currentfill}%
\pgfsetfillopacity{0.700000}%
\pgfsetlinewidth{0.501875pt}%
\definecolor{currentstroke}{rgb}{1.000000,1.000000,1.000000}%
\pgfsetstrokecolor{currentstroke}%
\pgfsetstrokeopacity{0.700000}%
\pgfsetdash{}{0pt}%
\pgfpathmoveto{\pgfqpoint{3.188492in}{2.463581in}}%
\pgfpathcurveto{\pgfqpoint{3.201515in}{2.463581in}}{\pgfqpoint{3.214006in}{2.468755in}}{\pgfqpoint{3.223214in}{2.477964in}}%
\pgfpathcurveto{\pgfqpoint{3.232423in}{2.487172in}}{\pgfqpoint{3.237597in}{2.499663in}}{\pgfqpoint{3.237597in}{2.512686in}}%
\pgfpathcurveto{\pgfqpoint{3.237597in}{2.525709in}}{\pgfqpoint{3.232423in}{2.538200in}}{\pgfqpoint{3.223214in}{2.547408in}}%
\pgfpathcurveto{\pgfqpoint{3.214006in}{2.556617in}}{\pgfqpoint{3.201515in}{2.561790in}}{\pgfqpoint{3.188492in}{2.561790in}}%
\pgfpathcurveto{\pgfqpoint{3.175470in}{2.561790in}}{\pgfqpoint{3.162978in}{2.556617in}}{\pgfqpoint{3.153770in}{2.547408in}}%
\pgfpathcurveto{\pgfqpoint{3.144562in}{2.538200in}}{\pgfqpoint{3.139388in}{2.525709in}}{\pgfqpoint{3.139388in}{2.512686in}}%
\pgfpathcurveto{\pgfqpoint{3.139388in}{2.499663in}}{\pgfqpoint{3.144562in}{2.487172in}}{\pgfqpoint{3.153770in}{2.477964in}}%
\pgfpathcurveto{\pgfqpoint{3.162978in}{2.468755in}}{\pgfqpoint{3.175470in}{2.463581in}}{\pgfqpoint{3.188492in}{2.463581in}}%
\pgfpathlineto{\pgfqpoint{3.188492in}{2.463581in}}%
\pgfpathclose%
\pgfusepath{stroke,fill}%
\end{pgfscope}%
\begin{pgfscope}%
\pgfpathrectangle{\pgfqpoint{0.786164in}{0.768110in}}{\pgfqpoint{8.851069in}{7.081890in}}%
\pgfusepath{clip}%
\pgfsetbuttcap%
\pgfsetroundjoin%
\definecolor{currentfill}{rgb}{0.278012,0.180367,0.486697}%
\pgfsetfillcolor{currentfill}%
\pgfsetfillopacity{0.700000}%
\pgfsetlinewidth{0.501875pt}%
\definecolor{currentstroke}{rgb}{1.000000,1.000000,1.000000}%
\pgfsetstrokecolor{currentstroke}%
\pgfsetstrokeopacity{0.700000}%
\pgfsetdash{}{0pt}%
\pgfpathmoveto{\pgfqpoint{3.352351in}{2.356870in}}%
\pgfpathcurveto{\pgfqpoint{3.365374in}{2.356870in}}{\pgfqpoint{3.377865in}{2.362044in}}{\pgfqpoint{3.387073in}{2.371253in}}%
\pgfpathcurveto{\pgfqpoint{3.396282in}{2.380461in}}{\pgfqpoint{3.401456in}{2.392952in}}{\pgfqpoint{3.401456in}{2.405975in}}%
\pgfpathcurveto{\pgfqpoint{3.401456in}{2.418998in}}{\pgfqpoint{3.396282in}{2.431489in}}{\pgfqpoint{3.387073in}{2.440697in}}%
\pgfpathcurveto{\pgfqpoint{3.377865in}{2.449906in}}{\pgfqpoint{3.365374in}{2.455080in}}{\pgfqpoint{3.352351in}{2.455080in}}%
\pgfpathcurveto{\pgfqpoint{3.339328in}{2.455080in}}{\pgfqpoint{3.326837in}{2.449906in}}{\pgfqpoint{3.317629in}{2.440697in}}%
\pgfpathcurveto{\pgfqpoint{3.308420in}{2.431489in}}{\pgfqpoint{3.303247in}{2.418998in}}{\pgfqpoint{3.303247in}{2.405975in}}%
\pgfpathcurveto{\pgfqpoint{3.303247in}{2.392952in}}{\pgfqpoint{3.308420in}{2.380461in}}{\pgfqpoint{3.317629in}{2.371253in}}%
\pgfpathcurveto{\pgfqpoint{3.326837in}{2.362044in}}{\pgfqpoint{3.339328in}{2.356870in}}{\pgfqpoint{3.352351in}{2.356870in}}%
\pgfpathlineto{\pgfqpoint{3.352351in}{2.356870in}}%
\pgfpathclose%
\pgfusepath{stroke,fill}%
\end{pgfscope}%
\begin{pgfscope}%
\pgfpathrectangle{\pgfqpoint{0.786164in}{0.768110in}}{\pgfqpoint{8.851069in}{7.081890in}}%
\pgfusepath{clip}%
\pgfsetbuttcap%
\pgfsetroundjoin%
\definecolor{currentfill}{rgb}{0.280255,0.165693,0.476498}%
\pgfsetfillcolor{currentfill}%
\pgfsetfillopacity{0.700000}%
\pgfsetlinewidth{0.501875pt}%
\definecolor{currentstroke}{rgb}{1.000000,1.000000,1.000000}%
\pgfsetstrokecolor{currentstroke}%
\pgfsetstrokeopacity{0.700000}%
\pgfsetdash{}{0pt}%
\pgfpathmoveto{\pgfqpoint{3.593378in}{2.036738in}}%
\pgfpathcurveto{\pgfqpoint{3.606400in}{2.036738in}}{\pgfqpoint{3.618891in}{2.041912in}}{\pgfqpoint{3.628100in}{2.051120in}}%
\pgfpathcurveto{\pgfqpoint{3.637308in}{2.060329in}}{\pgfqpoint{3.642482in}{2.072820in}}{\pgfqpoint{3.642482in}{2.085842in}}%
\pgfpathcurveto{\pgfqpoint{3.642482in}{2.098865in}}{\pgfqpoint{3.637308in}{2.111356in}}{\pgfqpoint{3.628100in}{2.120565in}}%
\pgfpathcurveto{\pgfqpoint{3.618891in}{2.129773in}}{\pgfqpoint{3.606400in}{2.134947in}}{\pgfqpoint{3.593378in}{2.134947in}}%
\pgfpathcurveto{\pgfqpoint{3.580355in}{2.134947in}}{\pgfqpoint{3.567864in}{2.129773in}}{\pgfqpoint{3.558655in}{2.120565in}}%
\pgfpathcurveto{\pgfqpoint{3.549447in}{2.111356in}}{\pgfqpoint{3.544273in}{2.098865in}}{\pgfqpoint{3.544273in}{2.085842in}}%
\pgfpathcurveto{\pgfqpoint{3.544273in}{2.072820in}}{\pgfqpoint{3.549447in}{2.060329in}}{\pgfqpoint{3.558655in}{2.051120in}}%
\pgfpathcurveto{\pgfqpoint{3.567864in}{2.041912in}}{\pgfqpoint{3.580355in}{2.036738in}}{\pgfqpoint{3.593378in}{2.036738in}}%
\pgfpathlineto{\pgfqpoint{3.593378in}{2.036738in}}%
\pgfpathclose%
\pgfusepath{stroke,fill}%
\end{pgfscope}%
\begin{pgfscope}%
\pgfpathrectangle{\pgfqpoint{0.786164in}{0.768110in}}{\pgfqpoint{8.851069in}{7.081890in}}%
\pgfusepath{clip}%
\pgfsetbuttcap%
\pgfsetroundjoin%
\definecolor{currentfill}{rgb}{0.281924,0.089666,0.412415}%
\pgfsetfillcolor{currentfill}%
\pgfsetfillopacity{0.700000}%
\pgfsetlinewidth{0.501875pt}%
\definecolor{currentstroke}{rgb}{1.000000,1.000000,1.000000}%
\pgfsetstrokecolor{currentstroke}%
\pgfsetstrokeopacity{0.700000}%
\pgfsetdash{}{0pt}%
\pgfpathmoveto{\pgfqpoint{3.573964in}{1.908685in}}%
\pgfpathcurveto{\pgfqpoint{3.586986in}{1.908685in}}{\pgfqpoint{3.599477in}{1.913859in}}{\pgfqpoint{3.608686in}{1.923067in}}%
\pgfpathcurveto{\pgfqpoint{3.617894in}{1.932276in}}{\pgfqpoint{3.623068in}{1.944767in}}{\pgfqpoint{3.623068in}{1.957789in}}%
\pgfpathcurveto{\pgfqpoint{3.623068in}{1.970812in}}{\pgfqpoint{3.617894in}{1.983303in}}{\pgfqpoint{3.608686in}{1.992512in}}%
\pgfpathcurveto{\pgfqpoint{3.599477in}{2.001720in}}{\pgfqpoint{3.586986in}{2.006894in}}{\pgfqpoint{3.573964in}{2.006894in}}%
\pgfpathcurveto{\pgfqpoint{3.560941in}{2.006894in}}{\pgfqpoint{3.548450in}{2.001720in}}{\pgfqpoint{3.539241in}{1.992512in}}%
\pgfpathcurveto{\pgfqpoint{3.530033in}{1.983303in}}{\pgfqpoint{3.524859in}{1.970812in}}{\pgfqpoint{3.524859in}{1.957789in}}%
\pgfpathcurveto{\pgfqpoint{3.524859in}{1.944767in}}{\pgfqpoint{3.530033in}{1.932276in}}{\pgfqpoint{3.539241in}{1.923067in}}%
\pgfpathcurveto{\pgfqpoint{3.548450in}{1.913859in}}{\pgfqpoint{3.560941in}{1.908685in}}{\pgfqpoint{3.573964in}{1.908685in}}%
\pgfpathlineto{\pgfqpoint{3.573964in}{1.908685in}}%
\pgfpathclose%
\pgfusepath{stroke,fill}%
\end{pgfscope}%
\begin{pgfscope}%
\pgfpathrectangle{\pgfqpoint{0.786164in}{0.768110in}}{\pgfqpoint{8.851069in}{7.081890in}}%
\pgfusepath{clip}%
\pgfsetbuttcap%
\pgfsetroundjoin%
\definecolor{currentfill}{rgb}{0.281446,0.084320,0.407414}%
\pgfsetfillcolor{currentfill}%
\pgfsetfillopacity{0.700000}%
\pgfsetlinewidth{0.501875pt}%
\definecolor{currentstroke}{rgb}{1.000000,1.000000,1.000000}%
\pgfsetstrokecolor{currentstroke}%
\pgfsetstrokeopacity{0.700000}%
\pgfsetdash{}{0pt}%
\pgfpathmoveto{\pgfqpoint{3.611082in}{1.930027in}}%
\pgfpathcurveto{\pgfqpoint{3.624105in}{1.930027in}}{\pgfqpoint{3.636596in}{1.935201in}}{\pgfqpoint{3.645804in}{1.944409in}}%
\pgfpathcurveto{\pgfqpoint{3.655013in}{1.953618in}}{\pgfqpoint{3.660187in}{1.966109in}}{\pgfqpoint{3.660187in}{1.979132in}}%
\pgfpathcurveto{\pgfqpoint{3.660187in}{1.992154in}}{\pgfqpoint{3.655013in}{2.004645in}}{\pgfqpoint{3.645804in}{2.013854in}}%
\pgfpathcurveto{\pgfqpoint{3.636596in}{2.023062in}}{\pgfqpoint{3.624105in}{2.028236in}}{\pgfqpoint{3.611082in}{2.028236in}}%
\pgfpathcurveto{\pgfqpoint{3.598060in}{2.028236in}}{\pgfqpoint{3.585568in}{2.023062in}}{\pgfqpoint{3.576360in}{2.013854in}}%
\pgfpathcurveto{\pgfqpoint{3.567152in}{2.004645in}}{\pgfqpoint{3.561978in}{1.992154in}}{\pgfqpoint{3.561978in}{1.979132in}}%
\pgfpathcurveto{\pgfqpoint{3.561978in}{1.966109in}}{\pgfqpoint{3.567152in}{1.953618in}}{\pgfqpoint{3.576360in}{1.944409in}}%
\pgfpathcurveto{\pgfqpoint{3.585568in}{1.935201in}}{\pgfqpoint{3.598060in}{1.930027in}}{\pgfqpoint{3.611082in}{1.930027in}}%
\pgfpathlineto{\pgfqpoint{3.611082in}{1.930027in}}%
\pgfpathclose%
\pgfusepath{stroke,fill}%
\end{pgfscope}%
\begin{pgfscope}%
\pgfpathrectangle{\pgfqpoint{0.786164in}{0.768110in}}{\pgfqpoint{8.851069in}{7.081890in}}%
\pgfusepath{clip}%
\pgfsetbuttcap%
\pgfsetroundjoin%
\definecolor{currentfill}{rgb}{0.282327,0.094955,0.417331}%
\pgfsetfillcolor{currentfill}%
\pgfsetfillopacity{0.700000}%
\pgfsetlinewidth{0.501875pt}%
\definecolor{currentstroke}{rgb}{1.000000,1.000000,1.000000}%
\pgfsetstrokecolor{currentstroke}%
\pgfsetstrokeopacity{0.700000}%
\pgfsetdash{}{0pt}%
\pgfpathmoveto{\pgfqpoint{3.793500in}{1.972711in}}%
\pgfpathcurveto{\pgfqpoint{3.806523in}{1.972711in}}{\pgfqpoint{3.819014in}{1.977885in}}{\pgfqpoint{3.828223in}{1.987094in}}%
\pgfpathcurveto{\pgfqpoint{3.837431in}{1.996302in}}{\pgfqpoint{3.842605in}{2.008793in}}{\pgfqpoint{3.842605in}{2.021816in}}%
\pgfpathcurveto{\pgfqpoint{3.842605in}{2.034839in}}{\pgfqpoint{3.837431in}{2.047330in}}{\pgfqpoint{3.828223in}{2.056538in}}%
\pgfpathcurveto{\pgfqpoint{3.819014in}{2.065747in}}{\pgfqpoint{3.806523in}{2.070921in}}{\pgfqpoint{3.793500in}{2.070921in}}%
\pgfpathcurveto{\pgfqpoint{3.780478in}{2.070921in}}{\pgfqpoint{3.767987in}{2.065747in}}{\pgfqpoint{3.758778in}{2.056538in}}%
\pgfpathcurveto{\pgfqpoint{3.749570in}{2.047330in}}{\pgfqpoint{3.744396in}{2.034839in}}{\pgfqpoint{3.744396in}{2.021816in}}%
\pgfpathcurveto{\pgfqpoint{3.744396in}{2.008793in}}{\pgfqpoint{3.749570in}{1.996302in}}{\pgfqpoint{3.758778in}{1.987094in}}%
\pgfpathcurveto{\pgfqpoint{3.767987in}{1.977885in}}{\pgfqpoint{3.780478in}{1.972711in}}{\pgfqpoint{3.793500in}{1.972711in}}%
\pgfpathlineto{\pgfqpoint{3.793500in}{1.972711in}}%
\pgfpathclose%
\pgfusepath{stroke,fill}%
\end{pgfscope}%
\begin{pgfscope}%
\pgfpathrectangle{\pgfqpoint{0.786164in}{0.768110in}}{\pgfqpoint{8.851069in}{7.081890in}}%
\pgfusepath{clip}%
\pgfsetbuttcap%
\pgfsetroundjoin%
\definecolor{currentfill}{rgb}{0.281924,0.089666,0.412415}%
\pgfsetfillcolor{currentfill}%
\pgfsetfillopacity{0.700000}%
\pgfsetlinewidth{0.501875pt}%
\definecolor{currentstroke}{rgb}{1.000000,1.000000,1.000000}%
\pgfsetstrokecolor{currentstroke}%
\pgfsetstrokeopacity{0.700000}%
\pgfsetdash{}{0pt}%
\pgfpathmoveto{\pgfqpoint{3.946370in}{1.951369in}}%
\pgfpathcurveto{\pgfqpoint{3.959393in}{1.951369in}}{\pgfqpoint{3.971884in}{1.956543in}}{\pgfqpoint{3.981093in}{1.965752in}}%
\pgfpathcurveto{\pgfqpoint{3.990301in}{1.974960in}}{\pgfqpoint{3.995475in}{1.987451in}}{\pgfqpoint{3.995475in}{2.000474in}}%
\pgfpathcurveto{\pgfqpoint{3.995475in}{2.013496in}}{\pgfqpoint{3.990301in}{2.025988in}}{\pgfqpoint{3.981093in}{2.035196in}}%
\pgfpathcurveto{\pgfqpoint{3.971884in}{2.044404in}}{\pgfqpoint{3.959393in}{2.049578in}}{\pgfqpoint{3.946370in}{2.049578in}}%
\pgfpathcurveto{\pgfqpoint{3.933348in}{2.049578in}}{\pgfqpoint{3.920857in}{2.044404in}}{\pgfqpoint{3.911648in}{2.035196in}}%
\pgfpathcurveto{\pgfqpoint{3.902440in}{2.025988in}}{\pgfqpoint{3.897266in}{2.013496in}}{\pgfqpoint{3.897266in}{2.000474in}}%
\pgfpathcurveto{\pgfqpoint{3.897266in}{1.987451in}}{\pgfqpoint{3.902440in}{1.974960in}}{\pgfqpoint{3.911648in}{1.965752in}}%
\pgfpathcurveto{\pgfqpoint{3.920857in}{1.956543in}}{\pgfqpoint{3.933348in}{1.951369in}}{\pgfqpoint{3.946370in}{1.951369in}}%
\pgfpathlineto{\pgfqpoint{3.946370in}{1.951369in}}%
\pgfpathclose%
\pgfusepath{stroke,fill}%
\end{pgfscope}%
\begin{pgfscope}%
\pgfpathrectangle{\pgfqpoint{0.786164in}{0.768110in}}{\pgfqpoint{8.851069in}{7.081890in}}%
\pgfusepath{clip}%
\pgfsetbuttcap%
\pgfsetroundjoin%
\definecolor{currentfill}{rgb}{0.281924,0.089666,0.412415}%
\pgfsetfillcolor{currentfill}%
\pgfsetfillopacity{0.700000}%
\pgfsetlinewidth{0.501875pt}%
\definecolor{currentstroke}{rgb}{1.000000,1.000000,1.000000}%
\pgfsetstrokecolor{currentstroke}%
\pgfsetstrokeopacity{0.700000}%
\pgfsetdash{}{0pt}%
\pgfpathmoveto{\pgfqpoint{4.056627in}{1.994053in}}%
\pgfpathcurveto{\pgfqpoint{4.069650in}{1.994053in}}{\pgfqpoint{4.082141in}{1.999227in}}{\pgfqpoint{4.091349in}{2.008436in}}%
\pgfpathcurveto{\pgfqpoint{4.100558in}{2.017644in}}{\pgfqpoint{4.105732in}{2.030135in}}{\pgfqpoint{4.105732in}{2.043158in}}%
\pgfpathcurveto{\pgfqpoint{4.105732in}{2.056181in}}{\pgfqpoint{4.100558in}{2.068672in}}{\pgfqpoint{4.091349in}{2.077880in}}%
\pgfpathcurveto{\pgfqpoint{4.082141in}{2.087089in}}{\pgfqpoint{4.069650in}{2.092263in}}{\pgfqpoint{4.056627in}{2.092263in}}%
\pgfpathcurveto{\pgfqpoint{4.043604in}{2.092263in}}{\pgfqpoint{4.031113in}{2.087089in}}{\pgfqpoint{4.021905in}{2.077880in}}%
\pgfpathcurveto{\pgfqpoint{4.012696in}{2.068672in}}{\pgfqpoint{4.007522in}{2.056181in}}{\pgfqpoint{4.007522in}{2.043158in}}%
\pgfpathcurveto{\pgfqpoint{4.007522in}{2.030135in}}{\pgfqpoint{4.012696in}{2.017644in}}{\pgfqpoint{4.021905in}{2.008436in}}%
\pgfpathcurveto{\pgfqpoint{4.031113in}{1.999227in}}{\pgfqpoint{4.043604in}{1.994053in}}{\pgfqpoint{4.056627in}{1.994053in}}%
\pgfpathlineto{\pgfqpoint{4.056627in}{1.994053in}}%
\pgfpathclose%
\pgfusepath{stroke,fill}%
\end{pgfscope}%
\begin{pgfscope}%
\pgfpathrectangle{\pgfqpoint{0.786164in}{0.768110in}}{\pgfqpoint{8.851069in}{7.081890in}}%
\pgfusepath{clip}%
\pgfsetbuttcap%
\pgfsetroundjoin%
\definecolor{currentfill}{rgb}{0.281924,0.089666,0.412415}%
\pgfsetfillcolor{currentfill}%
\pgfsetfillopacity{0.700000}%
\pgfsetlinewidth{0.501875pt}%
\definecolor{currentstroke}{rgb}{1.000000,1.000000,1.000000}%
\pgfsetstrokecolor{currentstroke}%
\pgfsetstrokeopacity{0.700000}%
\pgfsetdash{}{0pt}%
\pgfpathmoveto{\pgfqpoint{4.319876in}{2.100764in}}%
\pgfpathcurveto{\pgfqpoint{4.332899in}{2.100764in}}{\pgfqpoint{4.345390in}{2.105938in}}{\pgfqpoint{4.354598in}{2.115147in}}%
\pgfpathcurveto{\pgfqpoint{4.363807in}{2.124355in}}{\pgfqpoint{4.368981in}{2.136846in}}{\pgfqpoint{4.368981in}{2.149869in}}%
\pgfpathcurveto{\pgfqpoint{4.368981in}{2.162892in}}{\pgfqpoint{4.363807in}{2.175383in}}{\pgfqpoint{4.354598in}{2.184591in}}%
\pgfpathcurveto{\pgfqpoint{4.345390in}{2.193800in}}{\pgfqpoint{4.332899in}{2.198974in}}{\pgfqpoint{4.319876in}{2.198974in}}%
\pgfpathcurveto{\pgfqpoint{4.306853in}{2.198974in}}{\pgfqpoint{4.294362in}{2.193800in}}{\pgfqpoint{4.285154in}{2.184591in}}%
\pgfpathcurveto{\pgfqpoint{4.275945in}{2.175383in}}{\pgfqpoint{4.270771in}{2.162892in}}{\pgfqpoint{4.270771in}{2.149869in}}%
\pgfpathcurveto{\pgfqpoint{4.270771in}{2.136846in}}{\pgfqpoint{4.275945in}{2.124355in}}{\pgfqpoint{4.285154in}{2.115147in}}%
\pgfpathcurveto{\pgfqpoint{4.294362in}{2.105938in}}{\pgfqpoint{4.306853in}{2.100764in}}{\pgfqpoint{4.319876in}{2.100764in}}%
\pgfpathlineto{\pgfqpoint{4.319876in}{2.100764in}}%
\pgfpathclose%
\pgfusepath{stroke,fill}%
\end{pgfscope}%
\begin{pgfscope}%
\pgfpathrectangle{\pgfqpoint{0.786164in}{0.768110in}}{\pgfqpoint{8.851069in}{7.081890in}}%
\pgfusepath{clip}%
\pgfsetbuttcap%
\pgfsetroundjoin%
\definecolor{currentfill}{rgb}{0.283091,0.110553,0.431554}%
\pgfsetfillcolor{currentfill}%
\pgfsetfillopacity{0.700000}%
\pgfsetlinewidth{0.501875pt}%
\definecolor{currentstroke}{rgb}{1.000000,1.000000,1.000000}%
\pgfsetstrokecolor{currentstroke}%
\pgfsetstrokeopacity{0.700000}%
\pgfsetdash{}{0pt}%
\pgfpathmoveto{\pgfqpoint{4.303270in}{2.164791in}}%
\pgfpathcurveto{\pgfqpoint{4.316293in}{2.164791in}}{\pgfqpoint{4.328784in}{2.169965in}}{\pgfqpoint{4.337992in}{2.179173in}}%
\pgfpathcurveto{\pgfqpoint{4.347201in}{2.188382in}}{\pgfqpoint{4.352375in}{2.200873in}}{\pgfqpoint{4.352375in}{2.213895in}}%
\pgfpathcurveto{\pgfqpoint{4.352375in}{2.226918in}}{\pgfqpoint{4.347201in}{2.239409in}}{\pgfqpoint{4.337992in}{2.248618in}}%
\pgfpathcurveto{\pgfqpoint{4.328784in}{2.257826in}}{\pgfqpoint{4.316293in}{2.263000in}}{\pgfqpoint{4.303270in}{2.263000in}}%
\pgfpathcurveto{\pgfqpoint{4.290248in}{2.263000in}}{\pgfqpoint{4.277756in}{2.257826in}}{\pgfqpoint{4.268548in}{2.248618in}}%
\pgfpathcurveto{\pgfqpoint{4.259340in}{2.239409in}}{\pgfqpoint{4.254166in}{2.226918in}}{\pgfqpoint{4.254166in}{2.213895in}}%
\pgfpathcurveto{\pgfqpoint{4.254166in}{2.200873in}}{\pgfqpoint{4.259340in}{2.188382in}}{\pgfqpoint{4.268548in}{2.179173in}}%
\pgfpathcurveto{\pgfqpoint{4.277756in}{2.169965in}}{\pgfqpoint{4.290248in}{2.164791in}}{\pgfqpoint{4.303270in}{2.164791in}}%
\pgfpathlineto{\pgfqpoint{4.303270in}{2.164791in}}%
\pgfpathclose%
\pgfusepath{stroke,fill}%
\end{pgfscope}%
\begin{pgfscope}%
\pgfpathrectangle{\pgfqpoint{0.786164in}{0.768110in}}{\pgfqpoint{8.851069in}{7.081890in}}%
\pgfusepath{clip}%
\pgfsetbuttcap%
\pgfsetroundjoin%
\definecolor{currentfill}{rgb}{0.283229,0.120777,0.440584}%
\pgfsetfillcolor{currentfill}%
\pgfsetfillopacity{0.700000}%
\pgfsetlinewidth{0.501875pt}%
\definecolor{currentstroke}{rgb}{1.000000,1.000000,1.000000}%
\pgfsetstrokecolor{currentstroke}%
\pgfsetstrokeopacity{0.700000}%
\pgfsetdash{}{0pt}%
\pgfpathmoveto{\pgfqpoint{4.355285in}{2.250159in}}%
\pgfpathcurveto{\pgfqpoint{4.368308in}{2.250159in}}{\pgfqpoint{4.380799in}{2.255333in}}{\pgfqpoint{4.390007in}{2.264542in}}%
\pgfpathcurveto{\pgfqpoint{4.399216in}{2.273750in}}{\pgfqpoint{4.404390in}{2.286241in}}{\pgfqpoint{4.404390in}{2.299264in}}%
\pgfpathcurveto{\pgfqpoint{4.404390in}{2.312287in}}{\pgfqpoint{4.399216in}{2.324778in}}{\pgfqpoint{4.390007in}{2.333986in}}%
\pgfpathcurveto{\pgfqpoint{4.380799in}{2.343195in}}{\pgfqpoint{4.368308in}{2.348369in}}{\pgfqpoint{4.355285in}{2.348369in}}%
\pgfpathcurveto{\pgfqpoint{4.342262in}{2.348369in}}{\pgfqpoint{4.329771in}{2.343195in}}{\pgfqpoint{4.320563in}{2.333986in}}%
\pgfpathcurveto{\pgfqpoint{4.311354in}{2.324778in}}{\pgfqpoint{4.306180in}{2.312287in}}{\pgfqpoint{4.306180in}{2.299264in}}%
\pgfpathcurveto{\pgfqpoint{4.306180in}{2.286241in}}{\pgfqpoint{4.311354in}{2.273750in}}{\pgfqpoint{4.320563in}{2.264542in}}%
\pgfpathcurveto{\pgfqpoint{4.329771in}{2.255333in}}{\pgfqpoint{4.342262in}{2.250159in}}{\pgfqpoint{4.355285in}{2.250159in}}%
\pgfpathlineto{\pgfqpoint{4.355285in}{2.250159in}}%
\pgfpathclose%
\pgfusepath{stroke,fill}%
\end{pgfscope}%
\begin{pgfscope}%
\pgfpathrectangle{\pgfqpoint{0.786164in}{0.768110in}}{\pgfqpoint{8.851069in}{7.081890in}}%
\pgfusepath{clip}%
\pgfsetbuttcap%
\pgfsetroundjoin%
\definecolor{currentfill}{rgb}{0.282884,0.135920,0.453427}%
\pgfsetfillcolor{currentfill}%
\pgfsetfillopacity{0.700000}%
\pgfsetlinewidth{0.501875pt}%
\definecolor{currentstroke}{rgb}{1.000000,1.000000,1.000000}%
\pgfsetstrokecolor{currentstroke}%
\pgfsetstrokeopacity{0.700000}%
\pgfsetdash{}{0pt}%
\pgfpathmoveto{\pgfqpoint{4.191304in}{2.292844in}}%
\pgfpathcurveto{\pgfqpoint{4.204327in}{2.292844in}}{\pgfqpoint{4.216818in}{2.298018in}}{\pgfqpoint{4.226026in}{2.307226in}}%
\pgfpathcurveto{\pgfqpoint{4.235235in}{2.316435in}}{\pgfqpoint{4.240409in}{2.328926in}}{\pgfqpoint{4.240409in}{2.341948in}}%
\pgfpathcurveto{\pgfqpoint{4.240409in}{2.354971in}}{\pgfqpoint{4.235235in}{2.367462in}}{\pgfqpoint{4.226026in}{2.376671in}}%
\pgfpathcurveto{\pgfqpoint{4.216818in}{2.385879in}}{\pgfqpoint{4.204327in}{2.391053in}}{\pgfqpoint{4.191304in}{2.391053in}}%
\pgfpathcurveto{\pgfqpoint{4.178281in}{2.391053in}}{\pgfqpoint{4.165790in}{2.385879in}}{\pgfqpoint{4.156582in}{2.376671in}}%
\pgfpathcurveto{\pgfqpoint{4.147373in}{2.367462in}}{\pgfqpoint{4.142199in}{2.354971in}}{\pgfqpoint{4.142199in}{2.341948in}}%
\pgfpathcurveto{\pgfqpoint{4.142199in}{2.328926in}}{\pgfqpoint{4.147373in}{2.316435in}}{\pgfqpoint{4.156582in}{2.307226in}}%
\pgfpathcurveto{\pgfqpoint{4.165790in}{2.298018in}}{\pgfqpoint{4.178281in}{2.292844in}}{\pgfqpoint{4.191304in}{2.292844in}}%
\pgfpathlineto{\pgfqpoint{4.191304in}{2.292844in}}%
\pgfpathclose%
\pgfusepath{stroke,fill}%
\end{pgfscope}%
\begin{pgfscope}%
\pgfpathrectangle{\pgfqpoint{0.786164in}{0.768110in}}{\pgfqpoint{8.851069in}{7.081890in}}%
\pgfusepath{clip}%
\pgfsetbuttcap%
\pgfsetroundjoin%
\definecolor{currentfill}{rgb}{0.283072,0.130895,0.449241}%
\pgfsetfillcolor{currentfill}%
\pgfsetfillopacity{0.700000}%
\pgfsetlinewidth{0.501875pt}%
\definecolor{currentstroke}{rgb}{1.000000,1.000000,1.000000}%
\pgfsetstrokecolor{currentstroke}%
\pgfsetstrokeopacity{0.700000}%
\pgfsetdash{}{0pt}%
\pgfpathmoveto{\pgfqpoint{4.286420in}{2.292844in}}%
\pgfpathcurveto{\pgfqpoint{4.299443in}{2.292844in}}{\pgfqpoint{4.311934in}{2.298018in}}{\pgfqpoint{4.321143in}{2.307226in}}%
\pgfpathcurveto{\pgfqpoint{4.330351in}{2.316435in}}{\pgfqpoint{4.335525in}{2.328926in}}{\pgfqpoint{4.335525in}{2.341948in}}%
\pgfpathcurveto{\pgfqpoint{4.335525in}{2.354971in}}{\pgfqpoint{4.330351in}{2.367462in}}{\pgfqpoint{4.321143in}{2.376671in}}%
\pgfpathcurveto{\pgfqpoint{4.311934in}{2.385879in}}{\pgfqpoint{4.299443in}{2.391053in}}{\pgfqpoint{4.286420in}{2.391053in}}%
\pgfpathcurveto{\pgfqpoint{4.273398in}{2.391053in}}{\pgfqpoint{4.260907in}{2.385879in}}{\pgfqpoint{4.251698in}{2.376671in}}%
\pgfpathcurveto{\pgfqpoint{4.242490in}{2.367462in}}{\pgfqpoint{4.237316in}{2.354971in}}{\pgfqpoint{4.237316in}{2.341948in}}%
\pgfpathcurveto{\pgfqpoint{4.237316in}{2.328926in}}{\pgfqpoint{4.242490in}{2.316435in}}{\pgfqpoint{4.251698in}{2.307226in}}%
\pgfpathcurveto{\pgfqpoint{4.260907in}{2.298018in}}{\pgfqpoint{4.273398in}{2.292844in}}{\pgfqpoint{4.286420in}{2.292844in}}%
\pgfpathlineto{\pgfqpoint{4.286420in}{2.292844in}}%
\pgfpathclose%
\pgfusepath{stroke,fill}%
\end{pgfscope}%
\begin{pgfscope}%
\pgfpathrectangle{\pgfqpoint{0.786164in}{0.768110in}}{\pgfqpoint{8.851069in}{7.081890in}}%
\pgfusepath{clip}%
\pgfsetbuttcap%
\pgfsetroundjoin%
\definecolor{currentfill}{rgb}{0.283187,0.125848,0.444960}%
\pgfsetfillcolor{currentfill}%
\pgfsetfillopacity{0.700000}%
\pgfsetlinewidth{0.501875pt}%
\definecolor{currentstroke}{rgb}{1.000000,1.000000,1.000000}%
\pgfsetstrokecolor{currentstroke}%
\pgfsetstrokeopacity{0.700000}%
\pgfsetdash{}{0pt}%
\pgfpathmoveto{\pgfqpoint{4.445029in}{2.271502in}}%
\pgfpathcurveto{\pgfqpoint{4.458052in}{2.271502in}}{\pgfqpoint{4.470543in}{2.276676in}}{\pgfqpoint{4.479751in}{2.285884in}}%
\pgfpathcurveto{\pgfqpoint{4.488960in}{2.295093in}}{\pgfqpoint{4.494134in}{2.307584in}}{\pgfqpoint{4.494134in}{2.320606in}}%
\pgfpathcurveto{\pgfqpoint{4.494134in}{2.333629in}}{\pgfqpoint{4.488960in}{2.346120in}}{\pgfqpoint{4.479751in}{2.355329in}}%
\pgfpathcurveto{\pgfqpoint{4.470543in}{2.364537in}}{\pgfqpoint{4.458052in}{2.369711in}}{\pgfqpoint{4.445029in}{2.369711in}}%
\pgfpathcurveto{\pgfqpoint{4.432006in}{2.369711in}}{\pgfqpoint{4.419515in}{2.364537in}}{\pgfqpoint{4.410307in}{2.355329in}}%
\pgfpathcurveto{\pgfqpoint{4.401098in}{2.346120in}}{\pgfqpoint{4.395924in}{2.333629in}}{\pgfqpoint{4.395924in}{2.320606in}}%
\pgfpathcurveto{\pgfqpoint{4.395924in}{2.307584in}}{\pgfqpoint{4.401098in}{2.295093in}}{\pgfqpoint{4.410307in}{2.285884in}}%
\pgfpathcurveto{\pgfqpoint{4.419515in}{2.276676in}}{\pgfqpoint{4.432006in}{2.271502in}}{\pgfqpoint{4.445029in}{2.271502in}}%
\pgfpathlineto{\pgfqpoint{4.445029in}{2.271502in}}%
\pgfpathclose%
\pgfusepath{stroke,fill}%
\end{pgfscope}%
\begin{pgfscope}%
\pgfpathrectangle{\pgfqpoint{0.786164in}{0.768110in}}{\pgfqpoint{8.851069in}{7.081890in}}%
\pgfusepath{clip}%
\pgfsetbuttcap%
\pgfsetroundjoin%
\definecolor{currentfill}{rgb}{0.281887,0.150881,0.465405}%
\pgfsetfillcolor{currentfill}%
\pgfsetfillopacity{0.700000}%
\pgfsetlinewidth{0.501875pt}%
\definecolor{currentstroke}{rgb}{1.000000,1.000000,1.000000}%
\pgfsetstrokecolor{currentstroke}%
\pgfsetstrokeopacity{0.700000}%
\pgfsetdash{}{0pt}%
\pgfpathmoveto{\pgfqpoint{4.615115in}{2.314186in}}%
\pgfpathcurveto{\pgfqpoint{4.628138in}{2.314186in}}{\pgfqpoint{4.640629in}{2.319360in}}{\pgfqpoint{4.649837in}{2.328568in}}%
\pgfpathcurveto{\pgfqpoint{4.659046in}{2.337777in}}{\pgfqpoint{4.664220in}{2.350268in}}{\pgfqpoint{4.664220in}{2.363291in}}%
\pgfpathcurveto{\pgfqpoint{4.664220in}{2.376313in}}{\pgfqpoint{4.659046in}{2.388804in}}{\pgfqpoint{4.649837in}{2.398013in}}%
\pgfpathcurveto{\pgfqpoint{4.640629in}{2.407221in}}{\pgfqpoint{4.628138in}{2.412395in}}{\pgfqpoint{4.615115in}{2.412395in}}%
\pgfpathcurveto{\pgfqpoint{4.602092in}{2.412395in}}{\pgfqpoint{4.589601in}{2.407221in}}{\pgfqpoint{4.580393in}{2.398013in}}%
\pgfpathcurveto{\pgfqpoint{4.571184in}{2.388804in}}{\pgfqpoint{4.566010in}{2.376313in}}{\pgfqpoint{4.566010in}{2.363291in}}%
\pgfpathcurveto{\pgfqpoint{4.566010in}{2.350268in}}{\pgfqpoint{4.571184in}{2.337777in}}{\pgfqpoint{4.580393in}{2.328568in}}%
\pgfpathcurveto{\pgfqpoint{4.589601in}{2.319360in}}{\pgfqpoint{4.602092in}{2.314186in}}{\pgfqpoint{4.615115in}{2.314186in}}%
\pgfpathlineto{\pgfqpoint{4.615115in}{2.314186in}}%
\pgfpathclose%
\pgfusepath{stroke,fill}%
\end{pgfscope}%
\begin{pgfscope}%
\pgfpathrectangle{\pgfqpoint{0.786164in}{0.768110in}}{\pgfqpoint{8.851069in}{7.081890in}}%
\pgfusepath{clip}%
\pgfsetbuttcap%
\pgfsetroundjoin%
\definecolor{currentfill}{rgb}{0.283187,0.125848,0.444960}%
\pgfsetfillcolor{currentfill}%
\pgfsetfillopacity{0.700000}%
\pgfsetlinewidth{0.501875pt}%
\definecolor{currentstroke}{rgb}{1.000000,1.000000,1.000000}%
\pgfsetstrokecolor{currentstroke}%
\pgfsetstrokeopacity{0.700000}%
\pgfsetdash{}{0pt}%
\pgfpathmoveto{\pgfqpoint{4.790085in}{2.164791in}}%
\pgfpathcurveto{\pgfqpoint{4.803108in}{2.164791in}}{\pgfqpoint{4.815599in}{2.169965in}}{\pgfqpoint{4.824807in}{2.179173in}}%
\pgfpathcurveto{\pgfqpoint{4.834016in}{2.188382in}}{\pgfqpoint{4.839190in}{2.200873in}}{\pgfqpoint{4.839190in}{2.213895in}}%
\pgfpathcurveto{\pgfqpoint{4.839190in}{2.226918in}}{\pgfqpoint{4.834016in}{2.239409in}}{\pgfqpoint{4.824807in}{2.248618in}}%
\pgfpathcurveto{\pgfqpoint{4.815599in}{2.257826in}}{\pgfqpoint{4.803108in}{2.263000in}}{\pgfqpoint{4.790085in}{2.263000in}}%
\pgfpathcurveto{\pgfqpoint{4.777062in}{2.263000in}}{\pgfqpoint{4.764571in}{2.257826in}}{\pgfqpoint{4.755363in}{2.248618in}}%
\pgfpathcurveto{\pgfqpoint{4.746154in}{2.239409in}}{\pgfqpoint{4.740980in}{2.226918in}}{\pgfqpoint{4.740980in}{2.213895in}}%
\pgfpathcurveto{\pgfqpoint{4.740980in}{2.200873in}}{\pgfqpoint{4.746154in}{2.188382in}}{\pgfqpoint{4.755363in}{2.179173in}}%
\pgfpathcurveto{\pgfqpoint{4.764571in}{2.169965in}}{\pgfqpoint{4.777062in}{2.164791in}}{\pgfqpoint{4.790085in}{2.164791in}}%
\pgfpathlineto{\pgfqpoint{4.790085in}{2.164791in}}%
\pgfpathclose%
\pgfusepath{stroke,fill}%
\end{pgfscope}%
\begin{pgfscope}%
\pgfpathrectangle{\pgfqpoint{0.786164in}{0.768110in}}{\pgfqpoint{8.851069in}{7.081890in}}%
\pgfusepath{clip}%
\pgfsetbuttcap%
\pgfsetroundjoin%
\definecolor{currentfill}{rgb}{0.974417,0.903590,0.130215}%
\pgfsetfillcolor{currentfill}%
\pgfsetfillopacity{0.700000}%
\pgfsetlinewidth{0.501875pt}%
\definecolor{currentstroke}{rgb}{1.000000,1.000000,1.000000}%
\pgfsetstrokecolor{currentstroke}%
\pgfsetstrokeopacity{0.700000}%
\pgfsetdash{}{0pt}%
\pgfpathmoveto{\pgfqpoint{1.285799in}{7.436307in}}%
\pgfpathcurveto{\pgfqpoint{1.298822in}{7.436307in}}{\pgfqpoint{1.311313in}{7.441481in}}{\pgfqpoint{1.320522in}{7.450689in}}%
\pgfpathcurveto{\pgfqpoint{1.329730in}{7.459898in}}{\pgfqpoint{1.334904in}{7.472389in}}{\pgfqpoint{1.334904in}{7.485412in}}%
\pgfpathcurveto{\pgfqpoint{1.334904in}{7.498434in}}{\pgfqpoint{1.329730in}{7.510925in}}{\pgfqpoint{1.320522in}{7.520134in}}%
\pgfpathcurveto{\pgfqpoint{1.311313in}{7.529342in}}{\pgfqpoint{1.298822in}{7.534516in}}{\pgfqpoint{1.285799in}{7.534516in}}%
\pgfpathcurveto{\pgfqpoint{1.272777in}{7.534516in}}{\pgfqpoint{1.260286in}{7.529342in}}{\pgfqpoint{1.251077in}{7.520134in}}%
\pgfpathcurveto{\pgfqpoint{1.241869in}{7.510925in}}{\pgfqpoint{1.236695in}{7.498434in}}{\pgfqpoint{1.236695in}{7.485412in}}%
\pgfpathcurveto{\pgfqpoint{1.236695in}{7.472389in}}{\pgfqpoint{1.241869in}{7.459898in}}{\pgfqpoint{1.251077in}{7.450689in}}%
\pgfpathcurveto{\pgfqpoint{1.260286in}{7.441481in}}{\pgfqpoint{1.272777in}{7.436307in}}{\pgfqpoint{1.285799in}{7.436307in}}%
\pgfpathlineto{\pgfqpoint{1.285799in}{7.436307in}}%
\pgfpathclose%
\pgfusepath{stroke,fill}%
\end{pgfscope}%
\begin{pgfscope}%
\pgfpathrectangle{\pgfqpoint{0.786164in}{0.768110in}}{\pgfqpoint{8.851069in}{7.081890in}}%
\pgfusepath{clip}%
\pgfsetbuttcap%
\pgfsetroundjoin%
\definecolor{currentfill}{rgb}{0.993248,0.906157,0.143936}%
\pgfsetfillcolor{currentfill}%
\pgfsetfillopacity{0.700000}%
\pgfsetlinewidth{0.501875pt}%
\definecolor{currentstroke}{rgb}{1.000000,1.000000,1.000000}%
\pgfsetstrokecolor{currentstroke}%
\pgfsetstrokeopacity{0.700000}%
\pgfsetdash{}{0pt}%
\pgfpathmoveto{\pgfqpoint{1.347216in}{7.478991in}}%
\pgfpathcurveto{\pgfqpoint{1.360239in}{7.478991in}}{\pgfqpoint{1.372730in}{7.484165in}}{\pgfqpoint{1.381938in}{7.493374in}}%
\pgfpathcurveto{\pgfqpoint{1.391147in}{7.502582in}}{\pgfqpoint{1.396321in}{7.515073in}}{\pgfqpoint{1.396321in}{7.528096in}}%
\pgfpathcurveto{\pgfqpoint{1.396321in}{7.541119in}}{\pgfqpoint{1.391147in}{7.553610in}}{\pgfqpoint{1.381938in}{7.562818in}}%
\pgfpathcurveto{\pgfqpoint{1.372730in}{7.572027in}}{\pgfqpoint{1.360239in}{7.577201in}}{\pgfqpoint{1.347216in}{7.577201in}}%
\pgfpathcurveto{\pgfqpoint{1.334193in}{7.577201in}}{\pgfqpoint{1.321702in}{7.572027in}}{\pgfqpoint{1.312494in}{7.562818in}}%
\pgfpathcurveto{\pgfqpoint{1.303285in}{7.553610in}}{\pgfqpoint{1.298111in}{7.541119in}}{\pgfqpoint{1.298111in}{7.528096in}}%
\pgfpathcurveto{\pgfqpoint{1.298111in}{7.515073in}}{\pgfqpoint{1.303285in}{7.502582in}}{\pgfqpoint{1.312494in}{7.493374in}}%
\pgfpathcurveto{\pgfqpoint{1.321702in}{7.484165in}}{\pgfqpoint{1.334193in}{7.478991in}}{\pgfqpoint{1.347216in}{7.478991in}}%
\pgfpathlineto{\pgfqpoint{1.347216in}{7.478991in}}%
\pgfpathclose%
\pgfusepath{stroke,fill}%
\end{pgfscope}%
\begin{pgfscope}%
\pgfpathrectangle{\pgfqpoint{0.786164in}{0.768110in}}{\pgfqpoint{8.851069in}{7.081890in}}%
\pgfusepath{clip}%
\pgfsetbuttcap%
\pgfsetroundjoin%
\definecolor{currentfill}{rgb}{0.906311,0.894855,0.098125}%
\pgfsetfillcolor{currentfill}%
\pgfsetfillopacity{0.700000}%
\pgfsetlinewidth{0.501875pt}%
\definecolor{currentstroke}{rgb}{1.000000,1.000000,1.000000}%
\pgfsetstrokecolor{currentstroke}%
\pgfsetstrokeopacity{0.700000}%
\pgfsetdash{}{0pt}%
\pgfpathmoveto{\pgfqpoint{1.355397in}{7.308254in}}%
\pgfpathcurveto{\pgfqpoint{1.368419in}{7.308254in}}{\pgfqpoint{1.380910in}{7.313428in}}{\pgfqpoint{1.390119in}{7.322636in}}%
\pgfpathcurveto{\pgfqpoint{1.399327in}{7.331845in}}{\pgfqpoint{1.404501in}{7.344336in}}{\pgfqpoint{1.404501in}{7.357359in}}%
\pgfpathcurveto{\pgfqpoint{1.404501in}{7.370381in}}{\pgfqpoint{1.399327in}{7.382872in}}{\pgfqpoint{1.390119in}{7.392081in}}%
\pgfpathcurveto{\pgfqpoint{1.380910in}{7.401289in}}{\pgfqpoint{1.368419in}{7.406463in}}{\pgfqpoint{1.355397in}{7.406463in}}%
\pgfpathcurveto{\pgfqpoint{1.342374in}{7.406463in}}{\pgfqpoint{1.329883in}{7.401289in}}{\pgfqpoint{1.320674in}{7.392081in}}%
\pgfpathcurveto{\pgfqpoint{1.311466in}{7.382872in}}{\pgfqpoint{1.306292in}{7.370381in}}{\pgfqpoint{1.306292in}{7.357359in}}%
\pgfpathcurveto{\pgfqpoint{1.306292in}{7.344336in}}{\pgfqpoint{1.311466in}{7.331845in}}{\pgfqpoint{1.320674in}{7.322636in}}%
\pgfpathcurveto{\pgfqpoint{1.329883in}{7.313428in}}{\pgfqpoint{1.342374in}{7.308254in}}{\pgfqpoint{1.355397in}{7.308254in}}%
\pgfpathlineto{\pgfqpoint{1.355397in}{7.308254in}}%
\pgfpathclose%
\pgfusepath{stroke,fill}%
\end{pgfscope}%
\begin{pgfscope}%
\pgfpathrectangle{\pgfqpoint{0.786164in}{0.768110in}}{\pgfqpoint{8.851069in}{7.081890in}}%
\pgfusepath{clip}%
\pgfsetbuttcap%
\pgfsetroundjoin%
\definecolor{currentfill}{rgb}{0.814576,0.883393,0.110347}%
\pgfsetfillcolor{currentfill}%
\pgfsetfillopacity{0.700000}%
\pgfsetlinewidth{0.501875pt}%
\definecolor{currentstroke}{rgb}{1.000000,1.000000,1.000000}%
\pgfsetstrokecolor{currentstroke}%
\pgfsetstrokeopacity{0.700000}%
\pgfsetdash{}{0pt}%
\pgfpathmoveto{\pgfqpoint{1.508755in}{7.030806in}}%
\pgfpathcurveto{\pgfqpoint{1.521778in}{7.030806in}}{\pgfqpoint{1.534269in}{7.035980in}}{\pgfqpoint{1.543477in}{7.045188in}}%
\pgfpathcurveto{\pgfqpoint{1.552686in}{7.054397in}}{\pgfqpoint{1.557860in}{7.066888in}}{\pgfqpoint{1.557860in}{7.079910in}}%
\pgfpathcurveto{\pgfqpoint{1.557860in}{7.092933in}}{\pgfqpoint{1.552686in}{7.105424in}}{\pgfqpoint{1.543477in}{7.114633in}}%
\pgfpathcurveto{\pgfqpoint{1.534269in}{7.123841in}}{\pgfqpoint{1.521778in}{7.129015in}}{\pgfqpoint{1.508755in}{7.129015in}}%
\pgfpathcurveto{\pgfqpoint{1.495732in}{7.129015in}}{\pgfqpoint{1.483241in}{7.123841in}}{\pgfqpoint{1.474033in}{7.114633in}}%
\pgfpathcurveto{\pgfqpoint{1.464824in}{7.105424in}}{\pgfqpoint{1.459650in}{7.092933in}}{\pgfqpoint{1.459650in}{7.079910in}}%
\pgfpathcurveto{\pgfqpoint{1.459650in}{7.066888in}}{\pgfqpoint{1.464824in}{7.054397in}}{\pgfqpoint{1.474033in}{7.045188in}}%
\pgfpathcurveto{\pgfqpoint{1.483241in}{7.035980in}}{\pgfqpoint{1.495732in}{7.030806in}}{\pgfqpoint{1.508755in}{7.030806in}}%
\pgfpathlineto{\pgfqpoint{1.508755in}{7.030806in}}%
\pgfpathclose%
\pgfusepath{stroke,fill}%
\end{pgfscope}%
\begin{pgfscope}%
\pgfpathrectangle{\pgfqpoint{0.786164in}{0.768110in}}{\pgfqpoint{8.851069in}{7.081890in}}%
\pgfusepath{clip}%
\pgfsetbuttcap%
\pgfsetroundjoin%
\definecolor{currentfill}{rgb}{0.762373,0.876424,0.137064}%
\pgfsetfillcolor{currentfill}%
\pgfsetfillopacity{0.700000}%
\pgfsetlinewidth{0.501875pt}%
\definecolor{currentstroke}{rgb}{1.000000,1.000000,1.000000}%
\pgfsetstrokecolor{currentstroke}%
\pgfsetstrokeopacity{0.700000}%
\pgfsetdash{}{0pt}%
\pgfpathmoveto{\pgfqpoint{1.519011in}{6.881411in}}%
\pgfpathcurveto{\pgfqpoint{1.532034in}{6.881411in}}{\pgfqpoint{1.544525in}{6.886584in}}{\pgfqpoint{1.553734in}{6.895793in}}%
\pgfpathcurveto{\pgfqpoint{1.562942in}{6.905001in}}{\pgfqpoint{1.568116in}{6.917492in}}{\pgfqpoint{1.568116in}{6.930515in}}%
\pgfpathcurveto{\pgfqpoint{1.568116in}{6.943538in}}{\pgfqpoint{1.562942in}{6.956029in}}{\pgfqpoint{1.553734in}{6.965237in}}%
\pgfpathcurveto{\pgfqpoint{1.544525in}{6.974446in}}{\pgfqpoint{1.532034in}{6.979620in}}{\pgfqpoint{1.519011in}{6.979620in}}%
\pgfpathcurveto{\pgfqpoint{1.505989in}{6.979620in}}{\pgfqpoint{1.493498in}{6.974446in}}{\pgfqpoint{1.484289in}{6.965237in}}%
\pgfpathcurveto{\pgfqpoint{1.475081in}{6.956029in}}{\pgfqpoint{1.469907in}{6.943538in}}{\pgfqpoint{1.469907in}{6.930515in}}%
\pgfpathcurveto{\pgfqpoint{1.469907in}{6.917492in}}{\pgfqpoint{1.475081in}{6.905001in}}{\pgfqpoint{1.484289in}{6.895793in}}%
\pgfpathcurveto{\pgfqpoint{1.493498in}{6.886584in}}{\pgfqpoint{1.505989in}{6.881411in}}{\pgfqpoint{1.519011in}{6.881411in}}%
\pgfpathlineto{\pgfqpoint{1.519011in}{6.881411in}}%
\pgfpathclose%
\pgfusepath{stroke,fill}%
\end{pgfscope}%
\begin{pgfscope}%
\pgfpathrectangle{\pgfqpoint{0.786164in}{0.768110in}}{\pgfqpoint{8.851069in}{7.081890in}}%
\pgfusepath{clip}%
\pgfsetbuttcap%
\pgfsetroundjoin%
\definecolor{currentfill}{rgb}{0.545524,0.838039,0.275626}%
\pgfsetfillcolor{currentfill}%
\pgfsetfillopacity{0.700000}%
\pgfsetlinewidth{0.501875pt}%
\definecolor{currentstroke}{rgb}{1.000000,1.000000,1.000000}%
\pgfsetstrokecolor{currentstroke}%
\pgfsetstrokeopacity{0.700000}%
\pgfsetdash{}{0pt}%
\pgfpathmoveto{\pgfqpoint{1.533663in}{6.518594in}}%
\pgfpathcurveto{\pgfqpoint{1.546686in}{6.518594in}}{\pgfqpoint{1.559177in}{6.523768in}}{\pgfqpoint{1.568386in}{6.532976in}}%
\pgfpathcurveto{\pgfqpoint{1.577594in}{6.542184in}}{\pgfqpoint{1.582768in}{6.554676in}}{\pgfqpoint{1.582768in}{6.567698in}}%
\pgfpathcurveto{\pgfqpoint{1.582768in}{6.580721in}}{\pgfqpoint{1.577594in}{6.593212in}}{\pgfqpoint{1.568386in}{6.602420in}}%
\pgfpathcurveto{\pgfqpoint{1.559177in}{6.611629in}}{\pgfqpoint{1.546686in}{6.616803in}}{\pgfqpoint{1.533663in}{6.616803in}}%
\pgfpathcurveto{\pgfqpoint{1.520641in}{6.616803in}}{\pgfqpoint{1.508150in}{6.611629in}}{\pgfqpoint{1.498941in}{6.602420in}}%
\pgfpathcurveto{\pgfqpoint{1.489733in}{6.593212in}}{\pgfqpoint{1.484559in}{6.580721in}}{\pgfqpoint{1.484559in}{6.567698in}}%
\pgfpathcurveto{\pgfqpoint{1.484559in}{6.554676in}}{\pgfqpoint{1.489733in}{6.542184in}}{\pgfqpoint{1.498941in}{6.532976in}}%
\pgfpathcurveto{\pgfqpoint{1.508150in}{6.523768in}}{\pgfqpoint{1.520641in}{6.518594in}}{\pgfqpoint{1.533663in}{6.518594in}}%
\pgfpathlineto{\pgfqpoint{1.533663in}{6.518594in}}%
\pgfpathclose%
\pgfusepath{stroke,fill}%
\end{pgfscope}%
\begin{pgfscope}%
\pgfpathrectangle{\pgfqpoint{0.786164in}{0.768110in}}{\pgfqpoint{8.851069in}{7.081890in}}%
\pgfusepath{clip}%
\pgfsetbuttcap%
\pgfsetroundjoin%
\definecolor{currentfill}{rgb}{0.657642,0.860219,0.203082}%
\pgfsetfillcolor{currentfill}%
\pgfsetfillopacity{0.700000}%
\pgfsetlinewidth{0.501875pt}%
\definecolor{currentstroke}{rgb}{1.000000,1.000000,1.000000}%
\pgfsetstrokecolor{currentstroke}%
\pgfsetstrokeopacity{0.700000}%
\pgfsetdash{}{0pt}%
\pgfpathmoveto{\pgfqpoint{1.524140in}{6.753357in}}%
\pgfpathcurveto{\pgfqpoint{1.537162in}{6.753357in}}{\pgfqpoint{1.549653in}{6.758531in}}{\pgfqpoint{1.558862in}{6.767740in}}%
\pgfpathcurveto{\pgfqpoint{1.568070in}{6.776948in}}{\pgfqpoint{1.573244in}{6.789439in}}{\pgfqpoint{1.573244in}{6.802462in}}%
\pgfpathcurveto{\pgfqpoint{1.573244in}{6.815485in}}{\pgfqpoint{1.568070in}{6.827976in}}{\pgfqpoint{1.558862in}{6.837184in}}%
\pgfpathcurveto{\pgfqpoint{1.549653in}{6.846393in}}{\pgfqpoint{1.537162in}{6.851567in}}{\pgfqpoint{1.524140in}{6.851567in}}%
\pgfpathcurveto{\pgfqpoint{1.511117in}{6.851567in}}{\pgfqpoint{1.498626in}{6.846393in}}{\pgfqpoint{1.489417in}{6.837184in}}%
\pgfpathcurveto{\pgfqpoint{1.480209in}{6.827976in}}{\pgfqpoint{1.475035in}{6.815485in}}{\pgfqpoint{1.475035in}{6.802462in}}%
\pgfpathcurveto{\pgfqpoint{1.475035in}{6.789439in}}{\pgfqpoint{1.480209in}{6.776948in}}{\pgfqpoint{1.489417in}{6.767740in}}%
\pgfpathcurveto{\pgfqpoint{1.498626in}{6.758531in}}{\pgfqpoint{1.511117in}{6.753357in}}{\pgfqpoint{1.524140in}{6.753357in}}%
\pgfpathlineto{\pgfqpoint{1.524140in}{6.753357in}}%
\pgfpathclose%
\pgfusepath{stroke,fill}%
\end{pgfscope}%
\begin{pgfscope}%
\pgfpathrectangle{\pgfqpoint{0.786164in}{0.768110in}}{\pgfqpoint{8.851069in}{7.081890in}}%
\pgfusepath{clip}%
\pgfsetbuttcap%
\pgfsetroundjoin%
\definecolor{currentfill}{rgb}{0.555484,0.840254,0.269281}%
\pgfsetfillcolor{currentfill}%
\pgfsetfillopacity{0.700000}%
\pgfsetlinewidth{0.501875pt}%
\definecolor{currentstroke}{rgb}{1.000000,1.000000,1.000000}%
\pgfsetstrokecolor{currentstroke}%
\pgfsetstrokeopacity{0.700000}%
\pgfsetdash{}{0pt}%
\pgfpathmoveto{\pgfqpoint{1.524628in}{6.518594in}}%
\pgfpathcurveto{\pgfqpoint{1.537651in}{6.518594in}}{\pgfqpoint{1.550142in}{6.523768in}}{\pgfqpoint{1.559350in}{6.532976in}}%
\pgfpathcurveto{\pgfqpoint{1.568559in}{6.542184in}}{\pgfqpoint{1.573733in}{6.554676in}}{\pgfqpoint{1.573733in}{6.567698in}}%
\pgfpathcurveto{\pgfqpoint{1.573733in}{6.580721in}}{\pgfqpoint{1.568559in}{6.593212in}}{\pgfqpoint{1.559350in}{6.602420in}}%
\pgfpathcurveto{\pgfqpoint{1.550142in}{6.611629in}}{\pgfqpoint{1.537651in}{6.616803in}}{\pgfqpoint{1.524628in}{6.616803in}}%
\pgfpathcurveto{\pgfqpoint{1.511605in}{6.616803in}}{\pgfqpoint{1.499114in}{6.611629in}}{\pgfqpoint{1.489906in}{6.602420in}}%
\pgfpathcurveto{\pgfqpoint{1.480697in}{6.593212in}}{\pgfqpoint{1.475523in}{6.580721in}}{\pgfqpoint{1.475523in}{6.567698in}}%
\pgfpathcurveto{\pgfqpoint{1.475523in}{6.554676in}}{\pgfqpoint{1.480697in}{6.542184in}}{\pgfqpoint{1.489906in}{6.532976in}}%
\pgfpathcurveto{\pgfqpoint{1.499114in}{6.523768in}}{\pgfqpoint{1.511605in}{6.518594in}}{\pgfqpoint{1.524628in}{6.518594in}}%
\pgfpathlineto{\pgfqpoint{1.524628in}{6.518594in}}%
\pgfpathclose%
\pgfusepath{stroke,fill}%
\end{pgfscope}%
\begin{pgfscope}%
\pgfpathrectangle{\pgfqpoint{0.786164in}{0.768110in}}{\pgfqpoint{8.851069in}{7.081890in}}%
\pgfusepath{clip}%
\pgfsetbuttcap%
\pgfsetroundjoin%
\definecolor{currentfill}{rgb}{0.430983,0.808473,0.346476}%
\pgfsetfillcolor{currentfill}%
\pgfsetfillopacity{0.700000}%
\pgfsetlinewidth{0.501875pt}%
\definecolor{currentstroke}{rgb}{1.000000,1.000000,1.000000}%
\pgfsetstrokecolor{currentstroke}%
\pgfsetstrokeopacity{0.700000}%
\pgfsetdash{}{0pt}%
\pgfpathmoveto{\pgfqpoint{1.556008in}{6.241145in}}%
\pgfpathcurveto{\pgfqpoint{1.569031in}{6.241145in}}{\pgfqpoint{1.581522in}{6.246319in}}{\pgfqpoint{1.590730in}{6.255528in}}%
\pgfpathcurveto{\pgfqpoint{1.599939in}{6.264736in}}{\pgfqpoint{1.605112in}{6.277227in}}{\pgfqpoint{1.605112in}{6.290250in}}%
\pgfpathcurveto{\pgfqpoint{1.605112in}{6.303273in}}{\pgfqpoint{1.599939in}{6.315764in}}{\pgfqpoint{1.590730in}{6.324972in}}%
\pgfpathcurveto{\pgfqpoint{1.581522in}{6.334181in}}{\pgfqpoint{1.569031in}{6.339355in}}{\pgfqpoint{1.556008in}{6.339355in}}%
\pgfpathcurveto{\pgfqpoint{1.542985in}{6.339355in}}{\pgfqpoint{1.530494in}{6.334181in}}{\pgfqpoint{1.521286in}{6.324972in}}%
\pgfpathcurveto{\pgfqpoint{1.512077in}{6.315764in}}{\pgfqpoint{1.506903in}{6.303273in}}{\pgfqpoint{1.506903in}{6.290250in}}%
\pgfpathcurveto{\pgfqpoint{1.506903in}{6.277227in}}{\pgfqpoint{1.512077in}{6.264736in}}{\pgfqpoint{1.521286in}{6.255528in}}%
\pgfpathcurveto{\pgfqpoint{1.530494in}{6.246319in}}{\pgfqpoint{1.542985in}{6.241145in}}{\pgfqpoint{1.556008in}{6.241145in}}%
\pgfpathlineto{\pgfqpoint{1.556008in}{6.241145in}}%
\pgfpathclose%
\pgfusepath{stroke,fill}%
\end{pgfscope}%
\begin{pgfscope}%
\pgfpathrectangle{\pgfqpoint{0.786164in}{0.768110in}}{\pgfqpoint{8.851069in}{7.081890in}}%
\pgfusepath{clip}%
\pgfsetbuttcap%
\pgfsetroundjoin%
\definecolor{currentfill}{rgb}{0.335885,0.777018,0.402049}%
\pgfsetfillcolor{currentfill}%
\pgfsetfillopacity{0.700000}%
\pgfsetlinewidth{0.501875pt}%
\definecolor{currentstroke}{rgb}{1.000000,1.000000,1.000000}%
\pgfsetstrokecolor{currentstroke}%
\pgfsetstrokeopacity{0.700000}%
\pgfsetdash{}{0pt}%
\pgfpathmoveto{\pgfqpoint{1.602650in}{5.856986in}}%
\pgfpathcurveto{\pgfqpoint{1.615673in}{5.856986in}}{\pgfqpoint{1.628164in}{5.862160in}}{\pgfqpoint{1.637372in}{5.871369in}}%
\pgfpathcurveto{\pgfqpoint{1.646581in}{5.880577in}}{\pgfqpoint{1.651755in}{5.893068in}}{\pgfqpoint{1.651755in}{5.906091in}}%
\pgfpathcurveto{\pgfqpoint{1.651755in}{5.919114in}}{\pgfqpoint{1.646581in}{5.931605in}}{\pgfqpoint{1.637372in}{5.940813in}}%
\pgfpathcurveto{\pgfqpoint{1.628164in}{5.950022in}}{\pgfqpoint{1.615673in}{5.955196in}}{\pgfqpoint{1.602650in}{5.955196in}}%
\pgfpathcurveto{\pgfqpoint{1.589628in}{5.955196in}}{\pgfqpoint{1.577136in}{5.950022in}}{\pgfqpoint{1.567928in}{5.940813in}}%
\pgfpathcurveto{\pgfqpoint{1.558720in}{5.931605in}}{\pgfqpoint{1.553546in}{5.919114in}}{\pgfqpoint{1.553546in}{5.906091in}}%
\pgfpathcurveto{\pgfqpoint{1.553546in}{5.893068in}}{\pgfqpoint{1.558720in}{5.880577in}}{\pgfqpoint{1.567928in}{5.871369in}}%
\pgfpathcurveto{\pgfqpoint{1.577136in}{5.862160in}}{\pgfqpoint{1.589628in}{5.856986in}}{\pgfqpoint{1.602650in}{5.856986in}}%
\pgfpathlineto{\pgfqpoint{1.602650in}{5.856986in}}%
\pgfpathclose%
\pgfusepath{stroke,fill}%
\end{pgfscope}%
\begin{pgfscope}%
\pgfpathrectangle{\pgfqpoint{0.786164in}{0.768110in}}{\pgfqpoint{8.851069in}{7.081890in}}%
\pgfusepath{clip}%
\pgfsetbuttcap%
\pgfsetroundjoin%
\definecolor{currentfill}{rgb}{0.246070,0.738910,0.452024}%
\pgfsetfillcolor{currentfill}%
\pgfsetfillopacity{0.700000}%
\pgfsetlinewidth{0.501875pt}%
\definecolor{currentstroke}{rgb}{1.000000,1.000000,1.000000}%
\pgfsetstrokecolor{currentstroke}%
\pgfsetstrokeopacity{0.700000}%
\pgfsetdash{}{0pt}%
\pgfpathmoveto{\pgfqpoint{1.721943in}{5.622222in}}%
\pgfpathcurveto{\pgfqpoint{1.734965in}{5.622222in}}{\pgfqpoint{1.747456in}{5.627396in}}{\pgfqpoint{1.756665in}{5.636605in}}%
\pgfpathcurveto{\pgfqpoint{1.765873in}{5.645813in}}{\pgfqpoint{1.771047in}{5.658304in}}{\pgfqpoint{1.771047in}{5.671327in}}%
\pgfpathcurveto{\pgfqpoint{1.771047in}{5.684350in}}{\pgfqpoint{1.765873in}{5.696841in}}{\pgfqpoint{1.756665in}{5.706049in}}%
\pgfpathcurveto{\pgfqpoint{1.747456in}{5.715258in}}{\pgfqpoint{1.734965in}{5.720432in}}{\pgfqpoint{1.721943in}{5.720432in}}%
\pgfpathcurveto{\pgfqpoint{1.708920in}{5.720432in}}{\pgfqpoint{1.696429in}{5.715258in}}{\pgfqpoint{1.687220in}{5.706049in}}%
\pgfpathcurveto{\pgfqpoint{1.678012in}{5.696841in}}{\pgfqpoint{1.672838in}{5.684350in}}{\pgfqpoint{1.672838in}{5.671327in}}%
\pgfpathcurveto{\pgfqpoint{1.672838in}{5.658304in}}{\pgfqpoint{1.678012in}{5.645813in}}{\pgfqpoint{1.687220in}{5.636605in}}%
\pgfpathcurveto{\pgfqpoint{1.696429in}{5.627396in}}{\pgfqpoint{1.708920in}{5.622222in}}{\pgfqpoint{1.721943in}{5.622222in}}%
\pgfpathlineto{\pgfqpoint{1.721943in}{5.622222in}}%
\pgfpathclose%
\pgfusepath{stroke,fill}%
\end{pgfscope}%
\begin{pgfscope}%
\pgfpathrectangle{\pgfqpoint{0.786164in}{0.768110in}}{\pgfqpoint{8.851069in}{7.081890in}}%
\pgfusepath{clip}%
\pgfsetbuttcap%
\pgfsetroundjoin%
\definecolor{currentfill}{rgb}{0.202219,0.715272,0.476084}%
\pgfsetfillcolor{currentfill}%
\pgfsetfillopacity{0.700000}%
\pgfsetlinewidth{0.501875pt}%
\definecolor{currentstroke}{rgb}{1.000000,1.000000,1.000000}%
\pgfsetstrokecolor{currentstroke}%
\pgfsetstrokeopacity{0.700000}%
\pgfsetdash{}{0pt}%
\pgfpathmoveto{\pgfqpoint{1.784946in}{5.430143in}}%
\pgfpathcurveto{\pgfqpoint{1.797969in}{5.430143in}}{\pgfqpoint{1.810460in}{5.435317in}}{\pgfqpoint{1.819669in}{5.444525in}}%
\pgfpathcurveto{\pgfqpoint{1.828877in}{5.453734in}}{\pgfqpoint{1.834051in}{5.466225in}}{\pgfqpoint{1.834051in}{5.479248in}}%
\pgfpathcurveto{\pgfqpoint{1.834051in}{5.492270in}}{\pgfqpoint{1.828877in}{5.504761in}}{\pgfqpoint{1.819669in}{5.513970in}}%
\pgfpathcurveto{\pgfqpoint{1.810460in}{5.523178in}}{\pgfqpoint{1.797969in}{5.528352in}}{\pgfqpoint{1.784946in}{5.528352in}}%
\pgfpathcurveto{\pgfqpoint{1.771924in}{5.528352in}}{\pgfqpoint{1.759433in}{5.523178in}}{\pgfqpoint{1.750224in}{5.513970in}}%
\pgfpathcurveto{\pgfqpoint{1.741016in}{5.504761in}}{\pgfqpoint{1.735842in}{5.492270in}}{\pgfqpoint{1.735842in}{5.479248in}}%
\pgfpathcurveto{\pgfqpoint{1.735842in}{5.466225in}}{\pgfqpoint{1.741016in}{5.453734in}}{\pgfqpoint{1.750224in}{5.444525in}}%
\pgfpathcurveto{\pgfqpoint{1.759433in}{5.435317in}}{\pgfqpoint{1.771924in}{5.430143in}}{\pgfqpoint{1.784946in}{5.430143in}}%
\pgfpathlineto{\pgfqpoint{1.784946in}{5.430143in}}%
\pgfpathclose%
\pgfusepath{stroke,fill}%
\end{pgfscope}%
\begin{pgfscope}%
\pgfpathrectangle{\pgfqpoint{0.786164in}{0.768110in}}{\pgfqpoint{8.851069in}{7.081890in}}%
\pgfusepath{clip}%
\pgfsetbuttcap%
\pgfsetroundjoin%
\definecolor{currentfill}{rgb}{0.180653,0.701402,0.488189}%
\pgfsetfillcolor{currentfill}%
\pgfsetfillopacity{0.700000}%
\pgfsetlinewidth{0.501875pt}%
\definecolor{currentstroke}{rgb}{1.000000,1.000000,1.000000}%
\pgfsetstrokecolor{currentstroke}%
\pgfsetstrokeopacity{0.700000}%
\pgfsetdash{}{0pt}%
\pgfpathmoveto{\pgfqpoint{1.830978in}{5.259406in}}%
\pgfpathcurveto{\pgfqpoint{1.844001in}{5.259406in}}{\pgfqpoint{1.856492in}{5.264580in}}{\pgfqpoint{1.865701in}{5.273788in}}%
\pgfpathcurveto{\pgfqpoint{1.874909in}{5.282996in}}{\pgfqpoint{1.880083in}{5.295487in}}{\pgfqpoint{1.880083in}{5.308510in}}%
\pgfpathcurveto{\pgfqpoint{1.880083in}{5.321533in}}{\pgfqpoint{1.874909in}{5.334024in}}{\pgfqpoint{1.865701in}{5.343232in}}%
\pgfpathcurveto{\pgfqpoint{1.856492in}{5.352441in}}{\pgfqpoint{1.844001in}{5.357615in}}{\pgfqpoint{1.830978in}{5.357615in}}%
\pgfpathcurveto{\pgfqpoint{1.817956in}{5.357615in}}{\pgfqpoint{1.805465in}{5.352441in}}{\pgfqpoint{1.796256in}{5.343232in}}%
\pgfpathcurveto{\pgfqpoint{1.787048in}{5.334024in}}{\pgfqpoint{1.781874in}{5.321533in}}{\pgfqpoint{1.781874in}{5.308510in}}%
\pgfpathcurveto{\pgfqpoint{1.781874in}{5.295487in}}{\pgfqpoint{1.787048in}{5.282996in}}{\pgfqpoint{1.796256in}{5.273788in}}%
\pgfpathcurveto{\pgfqpoint{1.805465in}{5.264580in}}{\pgfqpoint{1.817956in}{5.259406in}}{\pgfqpoint{1.830978in}{5.259406in}}%
\pgfpathlineto{\pgfqpoint{1.830978in}{5.259406in}}%
\pgfpathclose%
\pgfusepath{stroke,fill}%
\end{pgfscope}%
\begin{pgfscope}%
\pgfpathrectangle{\pgfqpoint{0.786164in}{0.768110in}}{\pgfqpoint{8.851069in}{7.081890in}}%
\pgfusepath{clip}%
\pgfsetbuttcap%
\pgfsetroundjoin%
\definecolor{currentfill}{rgb}{0.191090,0.708366,0.482284}%
\pgfsetfillcolor{currentfill}%
\pgfsetfillopacity{0.700000}%
\pgfsetlinewidth{0.501875pt}%
\definecolor{currentstroke}{rgb}{1.000000,1.000000,1.000000}%
\pgfsetstrokecolor{currentstroke}%
\pgfsetstrokeopacity{0.700000}%
\pgfsetdash{}{0pt}%
\pgfpathmoveto{\pgfqpoint{1.932322in}{5.344774in}}%
\pgfpathcurveto{\pgfqpoint{1.945344in}{5.344774in}}{\pgfqpoint{1.957836in}{5.349948in}}{\pgfqpoint{1.967044in}{5.359157in}}%
\pgfpathcurveto{\pgfqpoint{1.976252in}{5.368365in}}{\pgfqpoint{1.981426in}{5.380856in}}{\pgfqpoint{1.981426in}{5.393879in}}%
\pgfpathcurveto{\pgfqpoint{1.981426in}{5.406902in}}{\pgfqpoint{1.976252in}{5.419393in}}{\pgfqpoint{1.967044in}{5.428601in}}%
\pgfpathcurveto{\pgfqpoint{1.957836in}{5.437810in}}{\pgfqpoint{1.945344in}{5.442984in}}{\pgfqpoint{1.932322in}{5.442984in}}%
\pgfpathcurveto{\pgfqpoint{1.919299in}{5.442984in}}{\pgfqpoint{1.906808in}{5.437810in}}{\pgfqpoint{1.897600in}{5.428601in}}%
\pgfpathcurveto{\pgfqpoint{1.888391in}{5.419393in}}{\pgfqpoint{1.883217in}{5.406902in}}{\pgfqpoint{1.883217in}{5.393879in}}%
\pgfpathcurveto{\pgfqpoint{1.883217in}{5.380856in}}{\pgfqpoint{1.888391in}{5.368365in}}{\pgfqpoint{1.897600in}{5.359157in}}%
\pgfpathcurveto{\pgfqpoint{1.906808in}{5.349948in}}{\pgfqpoint{1.919299in}{5.344774in}}{\pgfqpoint{1.932322in}{5.344774in}}%
\pgfpathlineto{\pgfqpoint{1.932322in}{5.344774in}}%
\pgfpathclose%
\pgfusepath{stroke,fill}%
\end{pgfscope}%
\begin{pgfscope}%
\pgfpathrectangle{\pgfqpoint{0.786164in}{0.768110in}}{\pgfqpoint{8.851069in}{7.081890in}}%
\pgfusepath{clip}%
\pgfsetbuttcap%
\pgfsetroundjoin%
\definecolor{currentfill}{rgb}{0.214000,0.722114,0.469588}%
\pgfsetfillcolor{currentfill}%
\pgfsetfillopacity{0.700000}%
\pgfsetlinewidth{0.501875pt}%
\definecolor{currentstroke}{rgb}{1.000000,1.000000,1.000000}%
\pgfsetstrokecolor{currentstroke}%
\pgfsetstrokeopacity{0.700000}%
\pgfsetdash{}{0pt}%
\pgfpathmoveto{\pgfqpoint{2.267854in}{5.451485in}}%
\pgfpathcurveto{\pgfqpoint{2.280877in}{5.451485in}}{\pgfqpoint{2.293368in}{5.456659in}}{\pgfqpoint{2.302576in}{5.465868in}}%
\pgfpathcurveto{\pgfqpoint{2.311785in}{5.475076in}}{\pgfqpoint{2.316959in}{5.487567in}}{\pgfqpoint{2.316959in}{5.500590in}}%
\pgfpathcurveto{\pgfqpoint{2.316959in}{5.513612in}}{\pgfqpoint{2.311785in}{5.526104in}}{\pgfqpoint{2.302576in}{5.535312in}}%
\pgfpathcurveto{\pgfqpoint{2.293368in}{5.544520in}}{\pgfqpoint{2.280877in}{5.549694in}}{\pgfqpoint{2.267854in}{5.549694in}}%
\pgfpathcurveto{\pgfqpoint{2.254831in}{5.549694in}}{\pgfqpoint{2.242340in}{5.544520in}}{\pgfqpoint{2.233132in}{5.535312in}}%
\pgfpathcurveto{\pgfqpoint{2.223923in}{5.526104in}}{\pgfqpoint{2.218749in}{5.513612in}}{\pgfqpoint{2.218749in}{5.500590in}}%
\pgfpathcurveto{\pgfqpoint{2.218749in}{5.487567in}}{\pgfqpoint{2.223923in}{5.475076in}}{\pgfqpoint{2.233132in}{5.465868in}}%
\pgfpathcurveto{\pgfqpoint{2.242340in}{5.456659in}}{\pgfqpoint{2.254831in}{5.451485in}}{\pgfqpoint{2.267854in}{5.451485in}}%
\pgfpathlineto{\pgfqpoint{2.267854in}{5.451485in}}%
\pgfpathclose%
\pgfusepath{stroke,fill}%
\end{pgfscope}%
\begin{pgfscope}%
\pgfpathrectangle{\pgfqpoint{0.786164in}{0.768110in}}{\pgfqpoint{8.851069in}{7.081890in}}%
\pgfusepath{clip}%
\pgfsetbuttcap%
\pgfsetroundjoin%
\definecolor{currentfill}{rgb}{0.202219,0.715272,0.476084}%
\pgfsetfillcolor{currentfill}%
\pgfsetfillopacity{0.700000}%
\pgfsetlinewidth{0.501875pt}%
\definecolor{currentstroke}{rgb}{1.000000,1.000000,1.000000}%
\pgfsetstrokecolor{currentstroke}%
\pgfsetstrokeopacity{0.700000}%
\pgfsetdash{}{0pt}%
\pgfpathmoveto{\pgfqpoint{2.036351in}{5.366116in}}%
\pgfpathcurveto{\pgfqpoint{2.049374in}{5.366116in}}{\pgfqpoint{2.061865in}{5.371290in}}{\pgfqpoint{2.071074in}{5.380499in}}%
\pgfpathcurveto{\pgfqpoint{2.080282in}{5.389707in}}{\pgfqpoint{2.085456in}{5.402198in}}{\pgfqpoint{2.085456in}{5.415221in}}%
\pgfpathcurveto{\pgfqpoint{2.085456in}{5.428244in}}{\pgfqpoint{2.080282in}{5.440735in}}{\pgfqpoint{2.071074in}{5.449943in}}%
\pgfpathcurveto{\pgfqpoint{2.061865in}{5.459152in}}{\pgfqpoint{2.049374in}{5.464326in}}{\pgfqpoint{2.036351in}{5.464326in}}%
\pgfpathcurveto{\pgfqpoint{2.023329in}{5.464326in}}{\pgfqpoint{2.010838in}{5.459152in}}{\pgfqpoint{2.001629in}{5.449943in}}%
\pgfpathcurveto{\pgfqpoint{1.992421in}{5.440735in}}{\pgfqpoint{1.987247in}{5.428244in}}{\pgfqpoint{1.987247in}{5.415221in}}%
\pgfpathcurveto{\pgfqpoint{1.987247in}{5.402198in}}{\pgfqpoint{1.992421in}{5.389707in}}{\pgfqpoint{2.001629in}{5.380499in}}%
\pgfpathcurveto{\pgfqpoint{2.010838in}{5.371290in}}{\pgfqpoint{2.023329in}{5.366116in}}{\pgfqpoint{2.036351in}{5.366116in}}%
\pgfpathlineto{\pgfqpoint{2.036351in}{5.366116in}}%
\pgfpathclose%
\pgfusepath{stroke,fill}%
\end{pgfscope}%
\begin{pgfscope}%
\pgfpathrectangle{\pgfqpoint{0.786164in}{0.768110in}}{\pgfqpoint{8.851069in}{7.081890in}}%
\pgfusepath{clip}%
\pgfsetbuttcap%
\pgfsetroundjoin%
\definecolor{currentfill}{rgb}{0.119512,0.607464,0.540218}%
\pgfsetfillcolor{currentfill}%
\pgfsetfillopacity{0.700000}%
\pgfsetlinewidth{0.501875pt}%
\definecolor{currentstroke}{rgb}{1.000000,1.000000,1.000000}%
\pgfsetstrokecolor{currentstroke}%
\pgfsetstrokeopacity{0.700000}%
\pgfsetdash{}{0pt}%
\pgfpathmoveto{\pgfqpoint{2.604485in}{4.640483in}}%
\pgfpathcurveto{\pgfqpoint{2.617508in}{4.640483in}}{\pgfqpoint{2.629999in}{4.645657in}}{\pgfqpoint{2.639208in}{4.654865in}}%
\pgfpathcurveto{\pgfqpoint{2.648416in}{4.664073in}}{\pgfqpoint{2.653590in}{4.676565in}}{\pgfqpoint{2.653590in}{4.689587in}}%
\pgfpathcurveto{\pgfqpoint{2.653590in}{4.702610in}}{\pgfqpoint{2.648416in}{4.715101in}}{\pgfqpoint{2.639208in}{4.724309in}}%
\pgfpathcurveto{\pgfqpoint{2.629999in}{4.733518in}}{\pgfqpoint{2.617508in}{4.738692in}}{\pgfqpoint{2.604485in}{4.738692in}}%
\pgfpathcurveto{\pgfqpoint{2.591463in}{4.738692in}}{\pgfqpoint{2.578972in}{4.733518in}}{\pgfqpoint{2.569763in}{4.724309in}}%
\pgfpathcurveto{\pgfqpoint{2.560555in}{4.715101in}}{\pgfqpoint{2.555381in}{4.702610in}}{\pgfqpoint{2.555381in}{4.689587in}}%
\pgfpathcurveto{\pgfqpoint{2.555381in}{4.676565in}}{\pgfqpoint{2.560555in}{4.664073in}}{\pgfqpoint{2.569763in}{4.654865in}}%
\pgfpathcurveto{\pgfqpoint{2.578972in}{4.645657in}}{\pgfqpoint{2.591463in}{4.640483in}}{\pgfqpoint{2.604485in}{4.640483in}}%
\pgfpathlineto{\pgfqpoint{2.604485in}{4.640483in}}%
\pgfpathclose%
\pgfusepath{stroke,fill}%
\end{pgfscope}%
\begin{pgfscope}%
\pgfpathrectangle{\pgfqpoint{0.786164in}{0.768110in}}{\pgfqpoint{8.851069in}{7.081890in}}%
\pgfusepath{clip}%
\pgfsetbuttcap%
\pgfsetroundjoin%
\definecolor{currentfill}{rgb}{0.124780,0.640461,0.527068}%
\pgfsetfillcolor{currentfill}%
\pgfsetfillopacity{0.700000}%
\pgfsetlinewidth{0.501875pt}%
\definecolor{currentstroke}{rgb}{1.000000,1.000000,1.000000}%
\pgfsetstrokecolor{currentstroke}%
\pgfsetstrokeopacity{0.700000}%
\pgfsetdash{}{0pt}%
\pgfpathmoveto{\pgfqpoint{2.608270in}{4.747193in}}%
\pgfpathcurveto{\pgfqpoint{2.621293in}{4.747193in}}{\pgfqpoint{2.633784in}{4.752367in}}{\pgfqpoint{2.642993in}{4.761576in}}%
\pgfpathcurveto{\pgfqpoint{2.652201in}{4.770784in}}{\pgfqpoint{2.657375in}{4.783275in}}{\pgfqpoint{2.657375in}{4.796298in}}%
\pgfpathcurveto{\pgfqpoint{2.657375in}{4.809321in}}{\pgfqpoint{2.652201in}{4.821812in}}{\pgfqpoint{2.642993in}{4.831020in}}%
\pgfpathcurveto{\pgfqpoint{2.633784in}{4.840229in}}{\pgfqpoint{2.621293in}{4.845403in}}{\pgfqpoint{2.608270in}{4.845403in}}%
\pgfpathcurveto{\pgfqpoint{2.595248in}{4.845403in}}{\pgfqpoint{2.582757in}{4.840229in}}{\pgfqpoint{2.573548in}{4.831020in}}%
\pgfpathcurveto{\pgfqpoint{2.564340in}{4.821812in}}{\pgfqpoint{2.559166in}{4.809321in}}{\pgfqpoint{2.559166in}{4.796298in}}%
\pgfpathcurveto{\pgfqpoint{2.559166in}{4.783275in}}{\pgfqpoint{2.564340in}{4.770784in}}{\pgfqpoint{2.573548in}{4.761576in}}%
\pgfpathcurveto{\pgfqpoint{2.582757in}{4.752367in}}{\pgfqpoint{2.595248in}{4.747193in}}{\pgfqpoint{2.608270in}{4.747193in}}%
\pgfpathlineto{\pgfqpoint{2.608270in}{4.747193in}}%
\pgfpathclose%
\pgfusepath{stroke,fill}%
\end{pgfscope}%
\begin{pgfscope}%
\pgfpathrectangle{\pgfqpoint{0.786164in}{0.768110in}}{\pgfqpoint{8.851069in}{7.081890in}}%
\pgfusepath{clip}%
\pgfsetbuttcap%
\pgfsetroundjoin%
\definecolor{currentfill}{rgb}{0.128729,0.563265,0.551229}%
\pgfsetfillcolor{currentfill}%
\pgfsetfillopacity{0.700000}%
\pgfsetlinewidth{0.501875pt}%
\definecolor{currentstroke}{rgb}{1.000000,1.000000,1.000000}%
\pgfsetstrokecolor{currentstroke}%
\pgfsetstrokeopacity{0.700000}%
\pgfsetdash{}{0pt}%
\pgfpathmoveto{\pgfqpoint{2.928906in}{4.277666in}}%
\pgfpathcurveto{\pgfqpoint{2.941929in}{4.277666in}}{\pgfqpoint{2.954420in}{4.282840in}}{\pgfqpoint{2.963629in}{4.292048in}}%
\pgfpathcurveto{\pgfqpoint{2.972837in}{4.301257in}}{\pgfqpoint{2.978011in}{4.313748in}}{\pgfqpoint{2.978011in}{4.326770in}}%
\pgfpathcurveto{\pgfqpoint{2.978011in}{4.339793in}}{\pgfqpoint{2.972837in}{4.352284in}}{\pgfqpoint{2.963629in}{4.361493in}}%
\pgfpathcurveto{\pgfqpoint{2.954420in}{4.370701in}}{\pgfqpoint{2.941929in}{4.375875in}}{\pgfqpoint{2.928906in}{4.375875in}}%
\pgfpathcurveto{\pgfqpoint{2.915884in}{4.375875in}}{\pgfqpoint{2.903393in}{4.370701in}}{\pgfqpoint{2.894184in}{4.361493in}}%
\pgfpathcurveto{\pgfqpoint{2.884976in}{4.352284in}}{\pgfqpoint{2.879802in}{4.339793in}}{\pgfqpoint{2.879802in}{4.326770in}}%
\pgfpathcurveto{\pgfqpoint{2.879802in}{4.313748in}}{\pgfqpoint{2.884976in}{4.301257in}}{\pgfqpoint{2.894184in}{4.292048in}}%
\pgfpathcurveto{\pgfqpoint{2.903393in}{4.282840in}}{\pgfqpoint{2.915884in}{4.277666in}}{\pgfqpoint{2.928906in}{4.277666in}}%
\pgfpathlineto{\pgfqpoint{2.928906in}{4.277666in}}%
\pgfpathclose%
\pgfusepath{stroke,fill}%
\end{pgfscope}%
\begin{pgfscope}%
\pgfpathrectangle{\pgfqpoint{0.786164in}{0.768110in}}{\pgfqpoint{8.851069in}{7.081890in}}%
\pgfusepath{clip}%
\pgfsetbuttcap%
\pgfsetroundjoin%
\definecolor{currentfill}{rgb}{0.281446,0.084320,0.407414}%
\pgfsetfillcolor{currentfill}%
\pgfsetfillopacity{0.700000}%
\pgfsetlinewidth{0.501875pt}%
\definecolor{currentstroke}{rgb}{1.000000,1.000000,1.000000}%
\pgfsetstrokecolor{currentstroke}%
\pgfsetstrokeopacity{0.700000}%
\pgfsetdash{}{0pt}%
\pgfpathmoveto{\pgfqpoint{5.180196in}{1.609894in}}%
\pgfpathcurveto{\pgfqpoint{5.193219in}{1.609894in}}{\pgfqpoint{5.205710in}{1.615068in}}{\pgfqpoint{5.214919in}{1.624277in}}%
\pgfpathcurveto{\pgfqpoint{5.224127in}{1.633485in}}{\pgfqpoint{5.229301in}{1.645976in}}{\pgfqpoint{5.229301in}{1.658999in}}%
\pgfpathcurveto{\pgfqpoint{5.229301in}{1.672022in}}{\pgfqpoint{5.224127in}{1.684513in}}{\pgfqpoint{5.214919in}{1.693721in}}%
\pgfpathcurveto{\pgfqpoint{5.205710in}{1.702930in}}{\pgfqpoint{5.193219in}{1.708104in}}{\pgfqpoint{5.180196in}{1.708104in}}%
\pgfpathcurveto{\pgfqpoint{5.167174in}{1.708104in}}{\pgfqpoint{5.154683in}{1.702930in}}{\pgfqpoint{5.145474in}{1.693721in}}%
\pgfpathcurveto{\pgfqpoint{5.136266in}{1.684513in}}{\pgfqpoint{5.131092in}{1.672022in}}{\pgfqpoint{5.131092in}{1.658999in}}%
\pgfpathcurveto{\pgfqpoint{5.131092in}{1.645976in}}{\pgfqpoint{5.136266in}{1.633485in}}{\pgfqpoint{5.145474in}{1.624277in}}%
\pgfpathcurveto{\pgfqpoint{5.154683in}{1.615068in}}{\pgfqpoint{5.167174in}{1.609894in}}{\pgfqpoint{5.180196in}{1.609894in}}%
\pgfpathlineto{\pgfqpoint{5.180196in}{1.609894in}}%
\pgfpathclose%
\pgfusepath{stroke,fill}%
\end{pgfscope}%
\begin{pgfscope}%
\pgfpathrectangle{\pgfqpoint{0.786164in}{0.768110in}}{\pgfqpoint{8.851069in}{7.081890in}}%
\pgfusepath{clip}%
\pgfsetbuttcap%
\pgfsetroundjoin%
\definecolor{currentfill}{rgb}{0.282327,0.094955,0.417331}%
\pgfsetfillcolor{currentfill}%
\pgfsetfillopacity{0.700000}%
\pgfsetlinewidth{0.501875pt}%
\definecolor{currentstroke}{rgb}{1.000000,1.000000,1.000000}%
\pgfsetstrokecolor{currentstroke}%
\pgfsetstrokeopacity{0.700000}%
\pgfsetdash{}{0pt}%
\pgfpathmoveto{\pgfqpoint{5.115483in}{1.695263in}}%
\pgfpathcurveto{\pgfqpoint{5.128506in}{1.695263in}}{\pgfqpoint{5.140997in}{1.700437in}}{\pgfqpoint{5.150205in}{1.709645in}}%
\pgfpathcurveto{\pgfqpoint{5.159414in}{1.718854in}}{\pgfqpoint{5.164588in}{1.731345in}}{\pgfqpoint{5.164588in}{1.744368in}}%
\pgfpathcurveto{\pgfqpoint{5.164588in}{1.757390in}}{\pgfqpoint{5.159414in}{1.769881in}}{\pgfqpoint{5.150205in}{1.779090in}}%
\pgfpathcurveto{\pgfqpoint{5.140997in}{1.788298in}}{\pgfqpoint{5.128506in}{1.793472in}}{\pgfqpoint{5.115483in}{1.793472in}}%
\pgfpathcurveto{\pgfqpoint{5.102460in}{1.793472in}}{\pgfqpoint{5.089969in}{1.788298in}}{\pgfqpoint{5.080761in}{1.779090in}}%
\pgfpathcurveto{\pgfqpoint{5.071552in}{1.769881in}}{\pgfqpoint{5.066378in}{1.757390in}}{\pgfqpoint{5.066378in}{1.744368in}}%
\pgfpathcurveto{\pgfqpoint{5.066378in}{1.731345in}}{\pgfqpoint{5.071552in}{1.718854in}}{\pgfqpoint{5.080761in}{1.709645in}}%
\pgfpathcurveto{\pgfqpoint{5.089969in}{1.700437in}}{\pgfqpoint{5.102460in}{1.695263in}}{\pgfqpoint{5.115483in}{1.695263in}}%
\pgfpathlineto{\pgfqpoint{5.115483in}{1.695263in}}%
\pgfpathclose%
\pgfusepath{stroke,fill}%
\end{pgfscope}%
\begin{pgfscope}%
\pgfpathrectangle{\pgfqpoint{0.786164in}{0.768110in}}{\pgfqpoint{8.851069in}{7.081890in}}%
\pgfusepath{clip}%
\pgfsetbuttcap%
\pgfsetroundjoin%
\definecolor{currentfill}{rgb}{0.283091,0.110553,0.431554}%
\pgfsetfillcolor{currentfill}%
\pgfsetfillopacity{0.700000}%
\pgfsetlinewidth{0.501875pt}%
\definecolor{currentstroke}{rgb}{1.000000,1.000000,1.000000}%
\pgfsetstrokecolor{currentstroke}%
\pgfsetstrokeopacity{0.700000}%
\pgfsetdash{}{0pt}%
\pgfpathmoveto{\pgfqpoint{4.978364in}{1.673921in}}%
\pgfpathcurveto{\pgfqpoint{4.991387in}{1.673921in}}{\pgfqpoint{5.003878in}{1.679095in}}{\pgfqpoint{5.013086in}{1.688303in}}%
\pgfpathcurveto{\pgfqpoint{5.022295in}{1.697512in}}{\pgfqpoint{5.027469in}{1.710003in}}{\pgfqpoint{5.027469in}{1.723026in}}%
\pgfpathcurveto{\pgfqpoint{5.027469in}{1.736048in}}{\pgfqpoint{5.022295in}{1.748539in}}{\pgfqpoint{5.013086in}{1.757748in}}%
\pgfpathcurveto{\pgfqpoint{5.003878in}{1.766956in}}{\pgfqpoint{4.991387in}{1.772130in}}{\pgfqpoint{4.978364in}{1.772130in}}%
\pgfpathcurveto{\pgfqpoint{4.965341in}{1.772130in}}{\pgfqpoint{4.952850in}{1.766956in}}{\pgfqpoint{4.943642in}{1.757748in}}%
\pgfpathcurveto{\pgfqpoint{4.934434in}{1.748539in}}{\pgfqpoint{4.929260in}{1.736048in}}{\pgfqpoint{4.929260in}{1.723026in}}%
\pgfpathcurveto{\pgfqpoint{4.929260in}{1.710003in}}{\pgfqpoint{4.934434in}{1.697512in}}{\pgfqpoint{4.943642in}{1.688303in}}%
\pgfpathcurveto{\pgfqpoint{4.952850in}{1.679095in}}{\pgfqpoint{4.965341in}{1.673921in}}{\pgfqpoint{4.978364in}{1.673921in}}%
\pgfpathlineto{\pgfqpoint{4.978364in}{1.673921in}}%
\pgfpathclose%
\pgfusepath{stroke,fill}%
\end{pgfscope}%
\begin{pgfscope}%
\pgfpathrectangle{\pgfqpoint{0.786164in}{0.768110in}}{\pgfqpoint{8.851069in}{7.081890in}}%
\pgfusepath{clip}%
\pgfsetbuttcap%
\pgfsetroundjoin%
\definecolor{currentfill}{rgb}{0.283229,0.120777,0.440584}%
\pgfsetfillcolor{currentfill}%
\pgfsetfillopacity{0.700000}%
\pgfsetlinewidth{0.501875pt}%
\definecolor{currentstroke}{rgb}{1.000000,1.000000,1.000000}%
\pgfsetstrokecolor{currentstroke}%
\pgfsetstrokeopacity{0.700000}%
\pgfsetdash{}{0pt}%
\pgfpathmoveto{\pgfqpoint{4.792039in}{1.759290in}}%
\pgfpathcurveto{\pgfqpoint{4.805061in}{1.759290in}}{\pgfqpoint{4.817553in}{1.764464in}}{\pgfqpoint{4.826761in}{1.773672in}}%
\pgfpathcurveto{\pgfqpoint{4.835969in}{1.782880in}}{\pgfqpoint{4.841143in}{1.795372in}}{\pgfqpoint{4.841143in}{1.808394in}}%
\pgfpathcurveto{\pgfqpoint{4.841143in}{1.821417in}}{\pgfqpoint{4.835969in}{1.833908in}}{\pgfqpoint{4.826761in}{1.843116in}}%
\pgfpathcurveto{\pgfqpoint{4.817553in}{1.852325in}}{\pgfqpoint{4.805061in}{1.857499in}}{\pgfqpoint{4.792039in}{1.857499in}}%
\pgfpathcurveto{\pgfqpoint{4.779016in}{1.857499in}}{\pgfqpoint{4.766525in}{1.852325in}}{\pgfqpoint{4.757317in}{1.843116in}}%
\pgfpathcurveto{\pgfqpoint{4.748108in}{1.833908in}}{\pgfqpoint{4.742934in}{1.821417in}}{\pgfqpoint{4.742934in}{1.808394in}}%
\pgfpathcurveto{\pgfqpoint{4.742934in}{1.795372in}}{\pgfqpoint{4.748108in}{1.782880in}}{\pgfqpoint{4.757317in}{1.773672in}}%
\pgfpathcurveto{\pgfqpoint{4.766525in}{1.764464in}}{\pgfqpoint{4.779016in}{1.759290in}}{\pgfqpoint{4.792039in}{1.759290in}}%
\pgfpathlineto{\pgfqpoint{4.792039in}{1.759290in}}%
\pgfpathclose%
\pgfusepath{stroke,fill}%
\end{pgfscope}%
\begin{pgfscope}%
\pgfpathrectangle{\pgfqpoint{0.786164in}{0.768110in}}{\pgfqpoint{8.851069in}{7.081890in}}%
\pgfusepath{clip}%
\pgfsetbuttcap%
\pgfsetroundjoin%
\definecolor{currentfill}{rgb}{0.283091,0.110553,0.431554}%
\pgfsetfillcolor{currentfill}%
\pgfsetfillopacity{0.700000}%
\pgfsetlinewidth{0.501875pt}%
\definecolor{currentstroke}{rgb}{1.000000,1.000000,1.000000}%
\pgfsetstrokecolor{currentstroke}%
\pgfsetstrokeopacity{0.700000}%
\pgfsetdash{}{0pt}%
\pgfpathmoveto{\pgfqpoint{4.815970in}{1.695263in}}%
\pgfpathcurveto{\pgfqpoint{4.828993in}{1.695263in}}{\pgfqpoint{4.841484in}{1.700437in}}{\pgfqpoint{4.850693in}{1.709645in}}%
\pgfpathcurveto{\pgfqpoint{4.859901in}{1.718854in}}{\pgfqpoint{4.865075in}{1.731345in}}{\pgfqpoint{4.865075in}{1.744368in}}%
\pgfpathcurveto{\pgfqpoint{4.865075in}{1.757390in}}{\pgfqpoint{4.859901in}{1.769881in}}{\pgfqpoint{4.850693in}{1.779090in}}%
\pgfpathcurveto{\pgfqpoint{4.841484in}{1.788298in}}{\pgfqpoint{4.828993in}{1.793472in}}{\pgfqpoint{4.815970in}{1.793472in}}%
\pgfpathcurveto{\pgfqpoint{4.802948in}{1.793472in}}{\pgfqpoint{4.790457in}{1.788298in}}{\pgfqpoint{4.781248in}{1.779090in}}%
\pgfpathcurveto{\pgfqpoint{4.772040in}{1.769881in}}{\pgfqpoint{4.766866in}{1.757390in}}{\pgfqpoint{4.766866in}{1.744368in}}%
\pgfpathcurveto{\pgfqpoint{4.766866in}{1.731345in}}{\pgfqpoint{4.772040in}{1.718854in}}{\pgfqpoint{4.781248in}{1.709645in}}%
\pgfpathcurveto{\pgfqpoint{4.790457in}{1.700437in}}{\pgfqpoint{4.802948in}{1.695263in}}{\pgfqpoint{4.815970in}{1.695263in}}%
\pgfpathlineto{\pgfqpoint{4.815970in}{1.695263in}}%
\pgfpathclose%
\pgfusepath{stroke,fill}%
\end{pgfscope}%
\begin{pgfscope}%
\pgfpathrectangle{\pgfqpoint{0.786164in}{0.768110in}}{\pgfqpoint{8.851069in}{7.081890in}}%
\pgfusepath{clip}%
\pgfsetbuttcap%
\pgfsetroundjoin%
\definecolor{currentfill}{rgb}{0.282910,0.105393,0.426902}%
\pgfsetfillcolor{currentfill}%
\pgfsetfillopacity{0.700000}%
\pgfsetlinewidth{0.501875pt}%
\definecolor{currentstroke}{rgb}{1.000000,1.000000,1.000000}%
\pgfsetstrokecolor{currentstroke}%
\pgfsetstrokeopacity{0.700000}%
\pgfsetdash{}{0pt}%
\pgfpathmoveto{\pgfqpoint{5.366156in}{1.930027in}}%
\pgfpathcurveto{\pgfqpoint{5.379178in}{1.930027in}}{\pgfqpoint{5.391669in}{1.935201in}}{\pgfqpoint{5.400878in}{1.944409in}}%
\pgfpathcurveto{\pgfqpoint{5.410086in}{1.953618in}}{\pgfqpoint{5.415260in}{1.966109in}}{\pgfqpoint{5.415260in}{1.979132in}}%
\pgfpathcurveto{\pgfqpoint{5.415260in}{1.992154in}}{\pgfqpoint{5.410086in}{2.004645in}}{\pgfqpoint{5.400878in}{2.013854in}}%
\pgfpathcurveto{\pgfqpoint{5.391669in}{2.023062in}}{\pgfqpoint{5.379178in}{2.028236in}}{\pgfqpoint{5.366156in}{2.028236in}}%
\pgfpathcurveto{\pgfqpoint{5.353133in}{2.028236in}}{\pgfqpoint{5.340642in}{2.023062in}}{\pgfqpoint{5.331433in}{2.013854in}}%
\pgfpathcurveto{\pgfqpoint{5.322225in}{2.004645in}}{\pgfqpoint{5.317051in}{1.992154in}}{\pgfqpoint{5.317051in}{1.979132in}}%
\pgfpathcurveto{\pgfqpoint{5.317051in}{1.966109in}}{\pgfqpoint{5.322225in}{1.953618in}}{\pgfqpoint{5.331433in}{1.944409in}}%
\pgfpathcurveto{\pgfqpoint{5.340642in}{1.935201in}}{\pgfqpoint{5.353133in}{1.930027in}}{\pgfqpoint{5.366156in}{1.930027in}}%
\pgfpathlineto{\pgfqpoint{5.366156in}{1.930027in}}%
\pgfpathclose%
\pgfusepath{stroke,fill}%
\end{pgfscope}%
\begin{pgfscope}%
\pgfpathrectangle{\pgfqpoint{0.786164in}{0.768110in}}{\pgfqpoint{8.851069in}{7.081890in}}%
\pgfusepath{clip}%
\pgfsetbuttcap%
\pgfsetroundjoin%
\definecolor{currentfill}{rgb}{0.283229,0.120777,0.440584}%
\pgfsetfillcolor{currentfill}%
\pgfsetfillopacity{0.700000}%
\pgfsetlinewidth{0.501875pt}%
\definecolor{currentstroke}{rgb}{1.000000,1.000000,1.000000}%
\pgfsetstrokecolor{currentstroke}%
\pgfsetstrokeopacity{0.700000}%
\pgfsetdash{}{0pt}%
\pgfpathmoveto{\pgfqpoint{4.884835in}{2.228817in}}%
\pgfpathcurveto{\pgfqpoint{4.897858in}{2.228817in}}{\pgfqpoint{4.910349in}{2.233991in}}{\pgfqpoint{4.919557in}{2.243200in}}%
\pgfpathcurveto{\pgfqpoint{4.928766in}{2.252408in}}{\pgfqpoint{4.933940in}{2.264899in}}{\pgfqpoint{4.933940in}{2.277922in}}%
\pgfpathcurveto{\pgfqpoint{4.933940in}{2.290945in}}{\pgfqpoint{4.928766in}{2.303436in}}{\pgfqpoint{4.919557in}{2.312644in}}%
\pgfpathcurveto{\pgfqpoint{4.910349in}{2.321853in}}{\pgfqpoint{4.897858in}{2.327027in}}{\pgfqpoint{4.884835in}{2.327027in}}%
\pgfpathcurveto{\pgfqpoint{4.871812in}{2.327027in}}{\pgfqpoint{4.859321in}{2.321853in}}{\pgfqpoint{4.850113in}{2.312644in}}%
\pgfpathcurveto{\pgfqpoint{4.840904in}{2.303436in}}{\pgfqpoint{4.835731in}{2.290945in}}{\pgfqpoint{4.835731in}{2.277922in}}%
\pgfpathcurveto{\pgfqpoint{4.835731in}{2.264899in}}{\pgfqpoint{4.840904in}{2.252408in}}{\pgfqpoint{4.850113in}{2.243200in}}%
\pgfpathcurveto{\pgfqpoint{4.859321in}{2.233991in}}{\pgfqpoint{4.871812in}{2.228817in}}{\pgfqpoint{4.884835in}{2.228817in}}%
\pgfpathlineto{\pgfqpoint{4.884835in}{2.228817in}}%
\pgfpathclose%
\pgfusepath{stroke,fill}%
\end{pgfscope}%
\begin{pgfscope}%
\pgfpathrectangle{\pgfqpoint{0.786164in}{0.768110in}}{\pgfqpoint{8.851069in}{7.081890in}}%
\pgfusepath{clip}%
\pgfsetbuttcap%
\pgfsetroundjoin%
\definecolor{currentfill}{rgb}{0.282910,0.105393,0.426902}%
\pgfsetfillcolor{currentfill}%
\pgfsetfillopacity{0.700000}%
\pgfsetlinewidth{0.501875pt}%
\definecolor{currentstroke}{rgb}{1.000000,1.000000,1.000000}%
\pgfsetstrokecolor{currentstroke}%
\pgfsetstrokeopacity{0.700000}%
\pgfsetdash{}{0pt}%
\pgfpathmoveto{\pgfqpoint{5.263713in}{2.143449in}}%
\pgfpathcurveto{\pgfqpoint{5.276736in}{2.143449in}}{\pgfqpoint{5.289227in}{2.148623in}}{\pgfqpoint{5.298435in}{2.157831in}}%
\pgfpathcurveto{\pgfqpoint{5.307644in}{2.167040in}}{\pgfqpoint{5.312818in}{2.179531in}}{\pgfqpoint{5.312818in}{2.192553in}}%
\pgfpathcurveto{\pgfqpoint{5.312818in}{2.205576in}}{\pgfqpoint{5.307644in}{2.218067in}}{\pgfqpoint{5.298435in}{2.227276in}}%
\pgfpathcurveto{\pgfqpoint{5.289227in}{2.236484in}}{\pgfqpoint{5.276736in}{2.241658in}}{\pgfqpoint{5.263713in}{2.241658in}}%
\pgfpathcurveto{\pgfqpoint{5.250690in}{2.241658in}}{\pgfqpoint{5.238199in}{2.236484in}}{\pgfqpoint{5.228991in}{2.227276in}}%
\pgfpathcurveto{\pgfqpoint{5.219783in}{2.218067in}}{\pgfqpoint{5.214609in}{2.205576in}}{\pgfqpoint{5.214609in}{2.192553in}}%
\pgfpathcurveto{\pgfqpoint{5.214609in}{2.179531in}}{\pgfqpoint{5.219783in}{2.167040in}}{\pgfqpoint{5.228991in}{2.157831in}}%
\pgfpathcurveto{\pgfqpoint{5.238199in}{2.148623in}}{\pgfqpoint{5.250690in}{2.143449in}}{\pgfqpoint{5.263713in}{2.143449in}}%
\pgfpathlineto{\pgfqpoint{5.263713in}{2.143449in}}%
\pgfpathclose%
\pgfusepath{stroke,fill}%
\end{pgfscope}%
\begin{pgfscope}%
\pgfpathrectangle{\pgfqpoint{0.786164in}{0.768110in}}{\pgfqpoint{8.851069in}{7.081890in}}%
\pgfusepath{clip}%
\pgfsetbuttcap%
\pgfsetroundjoin%
\definecolor{currentfill}{rgb}{0.283229,0.120777,0.440584}%
\pgfsetfillcolor{currentfill}%
\pgfsetfillopacity{0.700000}%
\pgfsetlinewidth{0.501875pt}%
\definecolor{currentstroke}{rgb}{1.000000,1.000000,1.000000}%
\pgfsetstrokecolor{currentstroke}%
\pgfsetstrokeopacity{0.700000}%
\pgfsetdash{}{0pt}%
\pgfpathmoveto{\pgfqpoint{5.536119in}{1.994053in}}%
\pgfpathcurveto{\pgfqpoint{5.549142in}{1.994053in}}{\pgfqpoint{5.561633in}{1.999227in}}{\pgfqpoint{5.570842in}{2.008436in}}%
\pgfpathcurveto{\pgfqpoint{5.580050in}{2.017644in}}{\pgfqpoint{5.585224in}{2.030135in}}{\pgfqpoint{5.585224in}{2.043158in}}%
\pgfpathcurveto{\pgfqpoint{5.585224in}{2.056181in}}{\pgfqpoint{5.580050in}{2.068672in}}{\pgfqpoint{5.570842in}{2.077880in}}%
\pgfpathcurveto{\pgfqpoint{5.561633in}{2.087089in}}{\pgfqpoint{5.549142in}{2.092263in}}{\pgfqpoint{5.536119in}{2.092263in}}%
\pgfpathcurveto{\pgfqpoint{5.523097in}{2.092263in}}{\pgfqpoint{5.510606in}{2.087089in}}{\pgfqpoint{5.501397in}{2.077880in}}%
\pgfpathcurveto{\pgfqpoint{5.492189in}{2.068672in}}{\pgfqpoint{5.487015in}{2.056181in}}{\pgfqpoint{5.487015in}{2.043158in}}%
\pgfpathcurveto{\pgfqpoint{5.487015in}{2.030135in}}{\pgfqpoint{5.492189in}{2.017644in}}{\pgfqpoint{5.501397in}{2.008436in}}%
\pgfpathcurveto{\pgfqpoint{5.510606in}{1.999227in}}{\pgfqpoint{5.523097in}{1.994053in}}{\pgfqpoint{5.536119in}{1.994053in}}%
\pgfpathlineto{\pgfqpoint{5.536119in}{1.994053in}}%
\pgfpathclose%
\pgfusepath{stroke,fill}%
\end{pgfscope}%
\begin{pgfscope}%
\pgfpathrectangle{\pgfqpoint{0.786164in}{0.768110in}}{\pgfqpoint{8.851069in}{7.081890in}}%
\pgfusepath{clip}%
\pgfsetbuttcap%
\pgfsetroundjoin%
\definecolor{currentfill}{rgb}{0.283229,0.120777,0.440584}%
\pgfsetfillcolor{currentfill}%
\pgfsetfillopacity{0.700000}%
\pgfsetlinewidth{0.501875pt}%
\definecolor{currentstroke}{rgb}{1.000000,1.000000,1.000000}%
\pgfsetstrokecolor{currentstroke}%
\pgfsetstrokeopacity{0.700000}%
\pgfsetdash{}{0pt}%
\pgfpathmoveto{\pgfqpoint{5.698269in}{2.122106in}}%
\pgfpathcurveto{\pgfqpoint{5.711292in}{2.122106in}}{\pgfqpoint{5.723783in}{2.127280in}}{\pgfqpoint{5.732991in}{2.136489in}}%
\pgfpathcurveto{\pgfqpoint{5.742200in}{2.145697in}}{\pgfqpoint{5.747374in}{2.158188in}}{\pgfqpoint{5.747374in}{2.171211in}}%
\pgfpathcurveto{\pgfqpoint{5.747374in}{2.184234in}}{\pgfqpoint{5.742200in}{2.196725in}}{\pgfqpoint{5.732991in}{2.205933in}}%
\pgfpathcurveto{\pgfqpoint{5.723783in}{2.215142in}}{\pgfqpoint{5.711292in}{2.220316in}}{\pgfqpoint{5.698269in}{2.220316in}}%
\pgfpathcurveto{\pgfqpoint{5.685246in}{2.220316in}}{\pgfqpoint{5.672755in}{2.215142in}}{\pgfqpoint{5.663547in}{2.205933in}}%
\pgfpathcurveto{\pgfqpoint{5.654338in}{2.196725in}}{\pgfqpoint{5.649164in}{2.184234in}}{\pgfqpoint{5.649164in}{2.171211in}}%
\pgfpathcurveto{\pgfqpoint{5.649164in}{2.158188in}}{\pgfqpoint{5.654338in}{2.145697in}}{\pgfqpoint{5.663547in}{2.136489in}}%
\pgfpathcurveto{\pgfqpoint{5.672755in}{2.127280in}}{\pgfqpoint{5.685246in}{2.122106in}}{\pgfqpoint{5.698269in}{2.122106in}}%
\pgfpathlineto{\pgfqpoint{5.698269in}{2.122106in}}%
\pgfpathclose%
\pgfusepath{stroke,fill}%
\end{pgfscope}%
\begin{pgfscope}%
\pgfpathrectangle{\pgfqpoint{0.786164in}{0.768110in}}{\pgfqpoint{8.851069in}{7.081890in}}%
\pgfusepath{clip}%
\pgfsetbuttcap%
\pgfsetroundjoin%
\definecolor{currentfill}{rgb}{0.283229,0.120777,0.440584}%
\pgfsetfillcolor{currentfill}%
\pgfsetfillopacity{0.700000}%
\pgfsetlinewidth{0.501875pt}%
\definecolor{currentstroke}{rgb}{1.000000,1.000000,1.000000}%
\pgfsetstrokecolor{currentstroke}%
\pgfsetstrokeopacity{0.700000}%
\pgfsetdash{}{0pt}%
\pgfpathmoveto{\pgfqpoint{5.892653in}{2.548950in}}%
\pgfpathcurveto{\pgfqpoint{5.905676in}{2.548950in}}{\pgfqpoint{5.918167in}{2.554124in}}{\pgfqpoint{5.927375in}{2.563332in}}%
\pgfpathcurveto{\pgfqpoint{5.936584in}{2.572541in}}{\pgfqpoint{5.941758in}{2.585032in}}{\pgfqpoint{5.941758in}{2.598055in}}%
\pgfpathcurveto{\pgfqpoint{5.941758in}{2.611077in}}{\pgfqpoint{5.936584in}{2.623568in}}{\pgfqpoint{5.927375in}{2.632777in}}%
\pgfpathcurveto{\pgfqpoint{5.918167in}{2.641985in}}{\pgfqpoint{5.905676in}{2.647159in}}{\pgfqpoint{5.892653in}{2.647159in}}%
\pgfpathcurveto{\pgfqpoint{5.879630in}{2.647159in}}{\pgfqpoint{5.867139in}{2.641985in}}{\pgfqpoint{5.857931in}{2.632777in}}%
\pgfpathcurveto{\pgfqpoint{5.848722in}{2.623568in}}{\pgfqpoint{5.843548in}{2.611077in}}{\pgfqpoint{5.843548in}{2.598055in}}%
\pgfpathcurveto{\pgfqpoint{5.843548in}{2.585032in}}{\pgfqpoint{5.848722in}{2.572541in}}{\pgfqpoint{5.857931in}{2.563332in}}%
\pgfpathcurveto{\pgfqpoint{5.867139in}{2.554124in}}{\pgfqpoint{5.879630in}{2.548950in}}{\pgfqpoint{5.892653in}{2.548950in}}%
\pgfpathlineto{\pgfqpoint{5.892653in}{2.548950in}}%
\pgfpathclose%
\pgfusepath{stroke,fill}%
\end{pgfscope}%
\begin{pgfscope}%
\pgfpathrectangle{\pgfqpoint{0.786164in}{0.768110in}}{\pgfqpoint{8.851069in}{7.081890in}}%
\pgfusepath{clip}%
\pgfsetbuttcap%
\pgfsetroundjoin%
\definecolor{currentfill}{rgb}{0.283197,0.115680,0.436115}%
\pgfsetfillcolor{currentfill}%
\pgfsetfillopacity{0.700000}%
\pgfsetlinewidth{0.501875pt}%
\definecolor{currentstroke}{rgb}{1.000000,1.000000,1.000000}%
\pgfsetstrokecolor{currentstroke}%
\pgfsetstrokeopacity{0.700000}%
\pgfsetdash{}{0pt}%
\pgfpathmoveto{\pgfqpoint{5.759441in}{2.420897in}}%
\pgfpathcurveto{\pgfqpoint{5.772464in}{2.420897in}}{\pgfqpoint{5.784955in}{2.426071in}}{\pgfqpoint{5.794164in}{2.435279in}}%
\pgfpathcurveto{\pgfqpoint{5.803372in}{2.444488in}}{\pgfqpoint{5.808546in}{2.456979in}}{\pgfqpoint{5.808546in}{2.470002in}}%
\pgfpathcurveto{\pgfqpoint{5.808546in}{2.483024in}}{\pgfqpoint{5.803372in}{2.495515in}}{\pgfqpoint{5.794164in}{2.504724in}}%
\pgfpathcurveto{\pgfqpoint{5.784955in}{2.513932in}}{\pgfqpoint{5.772464in}{2.519106in}}{\pgfqpoint{5.759441in}{2.519106in}}%
\pgfpathcurveto{\pgfqpoint{5.746419in}{2.519106in}}{\pgfqpoint{5.733928in}{2.513932in}}{\pgfqpoint{5.724719in}{2.504724in}}%
\pgfpathcurveto{\pgfqpoint{5.715511in}{2.495515in}}{\pgfqpoint{5.710337in}{2.483024in}}{\pgfqpoint{5.710337in}{2.470002in}}%
\pgfpathcurveto{\pgfqpoint{5.710337in}{2.456979in}}{\pgfqpoint{5.715511in}{2.444488in}}{\pgfqpoint{5.724719in}{2.435279in}}%
\pgfpathcurveto{\pgfqpoint{5.733928in}{2.426071in}}{\pgfqpoint{5.746419in}{2.420897in}}{\pgfqpoint{5.759441in}{2.420897in}}%
\pgfpathlineto{\pgfqpoint{5.759441in}{2.420897in}}%
\pgfpathclose%
\pgfusepath{stroke,fill}%
\end{pgfscope}%
\begin{pgfscope}%
\pgfpathrectangle{\pgfqpoint{0.786164in}{0.768110in}}{\pgfqpoint{8.851069in}{7.081890in}}%
\pgfusepath{clip}%
\pgfsetbuttcap%
\pgfsetroundjoin%
\definecolor{currentfill}{rgb}{0.283229,0.120777,0.440584}%
\pgfsetfillcolor{currentfill}%
\pgfsetfillopacity{0.700000}%
\pgfsetlinewidth{0.501875pt}%
\definecolor{currentstroke}{rgb}{1.000000,1.000000,1.000000}%
\pgfsetstrokecolor{currentstroke}%
\pgfsetstrokeopacity{0.700000}%
\pgfsetdash{}{0pt}%
\pgfpathmoveto{\pgfqpoint{5.710601in}{2.228817in}}%
\pgfpathcurveto{\pgfqpoint{5.723624in}{2.228817in}}{\pgfqpoint{5.736115in}{2.233991in}}{\pgfqpoint{5.745323in}{2.243200in}}%
\pgfpathcurveto{\pgfqpoint{5.754532in}{2.252408in}}{\pgfqpoint{5.759706in}{2.264899in}}{\pgfqpoint{5.759706in}{2.277922in}}%
\pgfpathcurveto{\pgfqpoint{5.759706in}{2.290945in}}{\pgfqpoint{5.754532in}{2.303436in}}{\pgfqpoint{5.745323in}{2.312644in}}%
\pgfpathcurveto{\pgfqpoint{5.736115in}{2.321853in}}{\pgfqpoint{5.723624in}{2.327027in}}{\pgfqpoint{5.710601in}{2.327027in}}%
\pgfpathcurveto{\pgfqpoint{5.697578in}{2.327027in}}{\pgfqpoint{5.685087in}{2.321853in}}{\pgfqpoint{5.675879in}{2.312644in}}%
\pgfpathcurveto{\pgfqpoint{5.666671in}{2.303436in}}{\pgfqpoint{5.661497in}{2.290945in}}{\pgfqpoint{5.661497in}{2.277922in}}%
\pgfpathcurveto{\pgfqpoint{5.661497in}{2.264899in}}{\pgfqpoint{5.666671in}{2.252408in}}{\pgfqpoint{5.675879in}{2.243200in}}%
\pgfpathcurveto{\pgfqpoint{5.685087in}{2.233991in}}{\pgfqpoint{5.697578in}{2.228817in}}{\pgfqpoint{5.710601in}{2.228817in}}%
\pgfpathlineto{\pgfqpoint{5.710601in}{2.228817in}}%
\pgfpathclose%
\pgfusepath{stroke,fill}%
\end{pgfscope}%
\begin{pgfscope}%
\pgfpathrectangle{\pgfqpoint{0.786164in}{0.768110in}}{\pgfqpoint{8.851069in}{7.081890in}}%
\pgfusepath{clip}%
\pgfsetbuttcap%
\pgfsetroundjoin%
\definecolor{currentfill}{rgb}{0.282884,0.135920,0.453427}%
\pgfsetfillcolor{currentfill}%
\pgfsetfillopacity{0.700000}%
\pgfsetlinewidth{0.501875pt}%
\definecolor{currentstroke}{rgb}{1.000000,1.000000,1.000000}%
\pgfsetstrokecolor{currentstroke}%
\pgfsetstrokeopacity{0.700000}%
\pgfsetdash{}{0pt}%
\pgfpathmoveto{\pgfqpoint{5.938929in}{2.100764in}}%
\pgfpathcurveto{\pgfqpoint{5.951952in}{2.100764in}}{\pgfqpoint{5.964443in}{2.105938in}}{\pgfqpoint{5.973651in}{2.115147in}}%
\pgfpathcurveto{\pgfqpoint{5.982860in}{2.124355in}}{\pgfqpoint{5.988034in}{2.136846in}}{\pgfqpoint{5.988034in}{2.149869in}}%
\pgfpathcurveto{\pgfqpoint{5.988034in}{2.162892in}}{\pgfqpoint{5.982860in}{2.175383in}}{\pgfqpoint{5.973651in}{2.184591in}}%
\pgfpathcurveto{\pgfqpoint{5.964443in}{2.193800in}}{\pgfqpoint{5.951952in}{2.198974in}}{\pgfqpoint{5.938929in}{2.198974in}}%
\pgfpathcurveto{\pgfqpoint{5.925907in}{2.198974in}}{\pgfqpoint{5.913415in}{2.193800in}}{\pgfqpoint{5.904207in}{2.184591in}}%
\pgfpathcurveto{\pgfqpoint{5.894999in}{2.175383in}}{\pgfqpoint{5.889825in}{2.162892in}}{\pgfqpoint{5.889825in}{2.149869in}}%
\pgfpathcurveto{\pgfqpoint{5.889825in}{2.136846in}}{\pgfqpoint{5.894999in}{2.124355in}}{\pgfqpoint{5.904207in}{2.115147in}}%
\pgfpathcurveto{\pgfqpoint{5.913415in}{2.105938in}}{\pgfqpoint{5.925907in}{2.100764in}}{\pgfqpoint{5.938929in}{2.100764in}}%
\pgfpathlineto{\pgfqpoint{5.938929in}{2.100764in}}%
\pgfpathclose%
\pgfusepath{stroke,fill}%
\end{pgfscope}%
\begin{pgfscope}%
\pgfpathrectangle{\pgfqpoint{0.786164in}{0.768110in}}{\pgfqpoint{8.851069in}{7.081890in}}%
\pgfusepath{clip}%
\pgfsetbuttcap%
\pgfsetroundjoin%
\definecolor{currentfill}{rgb}{0.281412,0.155834,0.469201}%
\pgfsetfillcolor{currentfill}%
\pgfsetfillopacity{0.700000}%
\pgfsetlinewidth{0.501875pt}%
\definecolor{currentstroke}{rgb}{1.000000,1.000000,1.000000}%
\pgfsetstrokecolor{currentstroke}%
\pgfsetstrokeopacity{0.700000}%
\pgfsetdash{}{0pt}%
\pgfpathmoveto{\pgfqpoint{6.062373in}{2.442239in}}%
\pgfpathcurveto{\pgfqpoint{6.075396in}{2.442239in}}{\pgfqpoint{6.087887in}{2.447413in}}{\pgfqpoint{6.097095in}{2.456621in}}%
\pgfpathcurveto{\pgfqpoint{6.106304in}{2.465830in}}{\pgfqpoint{6.111478in}{2.478321in}}{\pgfqpoint{6.111478in}{2.491344in}}%
\pgfpathcurveto{\pgfqpoint{6.111478in}{2.504366in}}{\pgfqpoint{6.106304in}{2.516857in}}{\pgfqpoint{6.097095in}{2.526066in}}%
\pgfpathcurveto{\pgfqpoint{6.087887in}{2.535274in}}{\pgfqpoint{6.075396in}{2.540448in}}{\pgfqpoint{6.062373in}{2.540448in}}%
\pgfpathcurveto{\pgfqpoint{6.049350in}{2.540448in}}{\pgfqpoint{6.036859in}{2.535274in}}{\pgfqpoint{6.027651in}{2.526066in}}%
\pgfpathcurveto{\pgfqpoint{6.018442in}{2.516857in}}{\pgfqpoint{6.013268in}{2.504366in}}{\pgfqpoint{6.013268in}{2.491344in}}%
\pgfpathcurveto{\pgfqpoint{6.013268in}{2.478321in}}{\pgfqpoint{6.018442in}{2.465830in}}{\pgfqpoint{6.027651in}{2.456621in}}%
\pgfpathcurveto{\pgfqpoint{6.036859in}{2.447413in}}{\pgfqpoint{6.049350in}{2.442239in}}{\pgfqpoint{6.062373in}{2.442239in}}%
\pgfpathlineto{\pgfqpoint{6.062373in}{2.442239in}}%
\pgfpathclose%
\pgfusepath{stroke,fill}%
\end{pgfscope}%
\begin{pgfscope}%
\pgfpathrectangle{\pgfqpoint{0.786164in}{0.768110in}}{\pgfqpoint{8.851069in}{7.081890in}}%
\pgfusepath{clip}%
\pgfsetbuttcap%
\pgfsetroundjoin%
\definecolor{currentfill}{rgb}{0.281887,0.150881,0.465405}%
\pgfsetfillcolor{currentfill}%
\pgfsetfillopacity{0.700000}%
\pgfsetlinewidth{0.501875pt}%
\definecolor{currentstroke}{rgb}{1.000000,1.000000,1.000000}%
\pgfsetstrokecolor{currentstroke}%
\pgfsetstrokeopacity{0.700000}%
\pgfsetdash{}{0pt}%
\pgfpathmoveto{\pgfqpoint{6.173484in}{2.271502in}}%
\pgfpathcurveto{\pgfqpoint{6.186507in}{2.271502in}}{\pgfqpoint{6.198998in}{2.276676in}}{\pgfqpoint{6.208207in}{2.285884in}}%
\pgfpathcurveto{\pgfqpoint{6.217415in}{2.295093in}}{\pgfqpoint{6.222589in}{2.307584in}}{\pgfqpoint{6.222589in}{2.320606in}}%
\pgfpathcurveto{\pgfqpoint{6.222589in}{2.333629in}}{\pgfqpoint{6.217415in}{2.346120in}}{\pgfqpoint{6.208207in}{2.355329in}}%
\pgfpathcurveto{\pgfqpoint{6.198998in}{2.364537in}}{\pgfqpoint{6.186507in}{2.369711in}}{\pgfqpoint{6.173484in}{2.369711in}}%
\pgfpathcurveto{\pgfqpoint{6.160462in}{2.369711in}}{\pgfqpoint{6.147971in}{2.364537in}}{\pgfqpoint{6.138762in}{2.355329in}}%
\pgfpathcurveto{\pgfqpoint{6.129554in}{2.346120in}}{\pgfqpoint{6.124380in}{2.333629in}}{\pgfqpoint{6.124380in}{2.320606in}}%
\pgfpathcurveto{\pgfqpoint{6.124380in}{2.307584in}}{\pgfqpoint{6.129554in}{2.295093in}}{\pgfqpoint{6.138762in}{2.285884in}}%
\pgfpathcurveto{\pgfqpoint{6.147971in}{2.276676in}}{\pgfqpoint{6.160462in}{2.271502in}}{\pgfqpoint{6.173484in}{2.271502in}}%
\pgfpathlineto{\pgfqpoint{6.173484in}{2.271502in}}%
\pgfpathclose%
\pgfusepath{stroke,fill}%
\end{pgfscope}%
\begin{pgfscope}%
\pgfpathrectangle{\pgfqpoint{0.786164in}{0.768110in}}{\pgfqpoint{8.851069in}{7.081890in}}%
\pgfusepath{clip}%
\pgfsetbuttcap%
\pgfsetroundjoin%
\definecolor{currentfill}{rgb}{0.283072,0.130895,0.449241}%
\pgfsetfillcolor{currentfill}%
\pgfsetfillopacity{0.700000}%
\pgfsetlinewidth{0.501875pt}%
\definecolor{currentstroke}{rgb}{1.000000,1.000000,1.000000}%
\pgfsetstrokecolor{currentstroke}%
\pgfsetstrokeopacity{0.700000}%
\pgfsetdash{}{0pt}%
\pgfpathmoveto{\pgfqpoint{6.320371in}{2.463581in}}%
\pgfpathcurveto{\pgfqpoint{6.333394in}{2.463581in}}{\pgfqpoint{6.345885in}{2.468755in}}{\pgfqpoint{6.355094in}{2.477964in}}%
\pgfpathcurveto{\pgfqpoint{6.364302in}{2.487172in}}{\pgfqpoint{6.369476in}{2.499663in}}{\pgfqpoint{6.369476in}{2.512686in}}%
\pgfpathcurveto{\pgfqpoint{6.369476in}{2.525709in}}{\pgfqpoint{6.364302in}{2.538200in}}{\pgfqpoint{6.355094in}{2.547408in}}%
\pgfpathcurveto{\pgfqpoint{6.345885in}{2.556617in}}{\pgfqpoint{6.333394in}{2.561790in}}{\pgfqpoint{6.320371in}{2.561790in}}%
\pgfpathcurveto{\pgfqpoint{6.307349in}{2.561790in}}{\pgfqpoint{6.294858in}{2.556617in}}{\pgfqpoint{6.285649in}{2.547408in}}%
\pgfpathcurveto{\pgfqpoint{6.276441in}{2.538200in}}{\pgfqpoint{6.271267in}{2.525709in}}{\pgfqpoint{6.271267in}{2.512686in}}%
\pgfpathcurveto{\pgfqpoint{6.271267in}{2.499663in}}{\pgfqpoint{6.276441in}{2.487172in}}{\pgfqpoint{6.285649in}{2.477964in}}%
\pgfpathcurveto{\pgfqpoint{6.294858in}{2.468755in}}{\pgfqpoint{6.307349in}{2.463581in}}{\pgfqpoint{6.320371in}{2.463581in}}%
\pgfpathlineto{\pgfqpoint{6.320371in}{2.463581in}}%
\pgfpathclose%
\pgfusepath{stroke,fill}%
\end{pgfscope}%
\begin{pgfscope}%
\pgfpathrectangle{\pgfqpoint{0.786164in}{0.768110in}}{\pgfqpoint{8.851069in}{7.081890in}}%
\pgfusepath{clip}%
\pgfsetbuttcap%
\pgfsetroundjoin%
\definecolor{currentfill}{rgb}{0.281887,0.150881,0.465405}%
\pgfsetfillcolor{currentfill}%
\pgfsetfillopacity{0.700000}%
\pgfsetlinewidth{0.501875pt}%
\definecolor{currentstroke}{rgb}{1.000000,1.000000,1.000000}%
\pgfsetstrokecolor{currentstroke}%
\pgfsetstrokeopacity{0.700000}%
\pgfsetdash{}{0pt}%
\pgfpathmoveto{\pgfqpoint{6.316220in}{2.677003in}}%
\pgfpathcurveto{\pgfqpoint{6.329243in}{2.677003in}}{\pgfqpoint{6.341734in}{2.682177in}}{\pgfqpoint{6.350942in}{2.691385in}}%
\pgfpathcurveto{\pgfqpoint{6.360151in}{2.700594in}}{\pgfqpoint{6.365325in}{2.713085in}}{\pgfqpoint{6.365325in}{2.726108in}}%
\pgfpathcurveto{\pgfqpoint{6.365325in}{2.739130in}}{\pgfqpoint{6.360151in}{2.751621in}}{\pgfqpoint{6.350942in}{2.760830in}}%
\pgfpathcurveto{\pgfqpoint{6.341734in}{2.770038in}}{\pgfqpoint{6.329243in}{2.775212in}}{\pgfqpoint{6.316220in}{2.775212in}}%
\pgfpathcurveto{\pgfqpoint{6.303197in}{2.775212in}}{\pgfqpoint{6.290706in}{2.770038in}}{\pgfqpoint{6.281498in}{2.760830in}}%
\pgfpathcurveto{\pgfqpoint{6.272289in}{2.751621in}}{\pgfqpoint{6.267115in}{2.739130in}}{\pgfqpoint{6.267115in}{2.726108in}}%
\pgfpathcurveto{\pgfqpoint{6.267115in}{2.713085in}}{\pgfqpoint{6.272289in}{2.700594in}}{\pgfqpoint{6.281498in}{2.691385in}}%
\pgfpathcurveto{\pgfqpoint{6.290706in}{2.682177in}}{\pgfqpoint{6.303197in}{2.677003in}}{\pgfqpoint{6.316220in}{2.677003in}}%
\pgfpathlineto{\pgfqpoint{6.316220in}{2.677003in}}%
\pgfpathclose%
\pgfusepath{stroke,fill}%
\end{pgfscope}%
\begin{pgfscope}%
\pgfpathrectangle{\pgfqpoint{0.786164in}{0.768110in}}{\pgfqpoint{8.851069in}{7.081890in}}%
\pgfusepath{clip}%
\pgfsetbuttcap%
\pgfsetroundjoin%
\definecolor{currentfill}{rgb}{0.282884,0.135920,0.453427}%
\pgfsetfillcolor{currentfill}%
\pgfsetfillopacity{0.700000}%
\pgfsetlinewidth{0.501875pt}%
\definecolor{currentstroke}{rgb}{1.000000,1.000000,1.000000}%
\pgfsetstrokecolor{currentstroke}%
\pgfsetstrokeopacity{0.700000}%
\pgfsetdash{}{0pt}%
\pgfpathmoveto{\pgfqpoint{6.464938in}{2.954451in}}%
\pgfpathcurveto{\pgfqpoint{6.477961in}{2.954451in}}{\pgfqpoint{6.490452in}{2.959625in}}{\pgfqpoint{6.499661in}{2.968834in}}%
\pgfpathcurveto{\pgfqpoint{6.508869in}{2.978042in}}{\pgfqpoint{6.514043in}{2.990533in}}{\pgfqpoint{6.514043in}{3.003556in}}%
\pgfpathcurveto{\pgfqpoint{6.514043in}{3.016578in}}{\pgfqpoint{6.508869in}{3.029070in}}{\pgfqpoint{6.499661in}{3.038278in}}%
\pgfpathcurveto{\pgfqpoint{6.490452in}{3.047486in}}{\pgfqpoint{6.477961in}{3.052660in}}{\pgfqpoint{6.464938in}{3.052660in}}%
\pgfpathcurveto{\pgfqpoint{6.451916in}{3.052660in}}{\pgfqpoint{6.439425in}{3.047486in}}{\pgfqpoint{6.430216in}{3.038278in}}%
\pgfpathcurveto{\pgfqpoint{6.421008in}{3.029070in}}{\pgfqpoint{6.415834in}{3.016578in}}{\pgfqpoint{6.415834in}{3.003556in}}%
\pgfpathcurveto{\pgfqpoint{6.415834in}{2.990533in}}{\pgfqpoint{6.421008in}{2.978042in}}{\pgfqpoint{6.430216in}{2.968834in}}%
\pgfpathcurveto{\pgfqpoint{6.439425in}{2.959625in}}{\pgfqpoint{6.451916in}{2.954451in}}{\pgfqpoint{6.464938in}{2.954451in}}%
\pgfpathlineto{\pgfqpoint{6.464938in}{2.954451in}}%
\pgfpathclose%
\pgfusepath{stroke,fill}%
\end{pgfscope}%
\begin{pgfscope}%
\pgfpathrectangle{\pgfqpoint{0.786164in}{0.768110in}}{\pgfqpoint{8.851069in}{7.081890in}}%
\pgfusepath{clip}%
\pgfsetbuttcap%
\pgfsetroundjoin%
\definecolor{currentfill}{rgb}{0.282623,0.140926,0.457517}%
\pgfsetfillcolor{currentfill}%
\pgfsetfillopacity{0.700000}%
\pgfsetlinewidth{0.501875pt}%
\definecolor{currentstroke}{rgb}{1.000000,1.000000,1.000000}%
\pgfsetstrokecolor{currentstroke}%
\pgfsetstrokeopacity{0.700000}%
\pgfsetdash{}{0pt}%
\pgfpathmoveto{\pgfqpoint{1.188485in}{3.018478in}}%
\pgfpathcurveto{\pgfqpoint{1.201508in}{3.018478in}}{\pgfqpoint{1.213999in}{3.023652in}}{\pgfqpoint{1.223207in}{3.032860in}}%
\pgfpathcurveto{\pgfqpoint{1.232416in}{3.042068in}}{\pgfqpoint{1.237590in}{3.054560in}}{\pgfqpoint{1.237590in}{3.067582in}}%
\pgfpathcurveto{\pgfqpoint{1.237590in}{3.080605in}}{\pgfqpoint{1.232416in}{3.093096in}}{\pgfqpoint{1.223207in}{3.102305in}}%
\pgfpathcurveto{\pgfqpoint{1.213999in}{3.111513in}}{\pgfqpoint{1.201508in}{3.116687in}}{\pgfqpoint{1.188485in}{3.116687in}}%
\pgfpathcurveto{\pgfqpoint{1.175462in}{3.116687in}}{\pgfqpoint{1.162971in}{3.111513in}}{\pgfqpoint{1.153763in}{3.102305in}}%
\pgfpathcurveto{\pgfqpoint{1.144555in}{3.093096in}}{\pgfqpoint{1.139381in}{3.080605in}}{\pgfqpoint{1.139381in}{3.067582in}}%
\pgfpathcurveto{\pgfqpoint{1.139381in}{3.054560in}}{\pgfqpoint{1.144555in}{3.042068in}}{\pgfqpoint{1.153763in}{3.032860in}}%
\pgfpathcurveto{\pgfqpoint{1.162971in}{3.023652in}}{\pgfqpoint{1.175462in}{3.018478in}}{\pgfqpoint{1.188485in}{3.018478in}}%
\pgfpathlineto{\pgfqpoint{1.188485in}{3.018478in}}%
\pgfpathclose%
\pgfusepath{stroke,fill}%
\end{pgfscope}%
\begin{pgfscope}%
\pgfpathrectangle{\pgfqpoint{0.786164in}{0.768110in}}{\pgfqpoint{8.851069in}{7.081890in}}%
\pgfusepath{clip}%
\pgfsetbuttcap%
\pgfsetroundjoin%
\definecolor{currentfill}{rgb}{0.282884,0.135920,0.453427}%
\pgfsetfillcolor{currentfill}%
\pgfsetfillopacity{0.700000}%
\pgfsetlinewidth{0.501875pt}%
\definecolor{currentstroke}{rgb}{1.000000,1.000000,1.000000}%
\pgfsetstrokecolor{currentstroke}%
\pgfsetstrokeopacity{0.700000}%
\pgfsetdash{}{0pt}%
\pgfpathmoveto{\pgfqpoint{1.191049in}{2.911767in}}%
\pgfpathcurveto{\pgfqpoint{1.204072in}{2.911767in}}{\pgfqpoint{1.216563in}{2.916941in}}{\pgfqpoint{1.225772in}{2.926149in}}%
\pgfpathcurveto{\pgfqpoint{1.234980in}{2.935358in}}{\pgfqpoint{1.240154in}{2.947849in}}{\pgfqpoint{1.240154in}{2.960871in}}%
\pgfpathcurveto{\pgfqpoint{1.240154in}{2.973894in}}{\pgfqpoint{1.234980in}{2.986385in}}{\pgfqpoint{1.225772in}{2.995594in}}%
\pgfpathcurveto{\pgfqpoint{1.216563in}{3.004802in}}{\pgfqpoint{1.204072in}{3.009976in}}{\pgfqpoint{1.191049in}{3.009976in}}%
\pgfpathcurveto{\pgfqpoint{1.178027in}{3.009976in}}{\pgfqpoint{1.165536in}{3.004802in}}{\pgfqpoint{1.156327in}{2.995594in}}%
\pgfpathcurveto{\pgfqpoint{1.147119in}{2.986385in}}{\pgfqpoint{1.141945in}{2.973894in}}{\pgfqpoint{1.141945in}{2.960871in}}%
\pgfpathcurveto{\pgfqpoint{1.141945in}{2.947849in}}{\pgfqpoint{1.147119in}{2.935358in}}{\pgfqpoint{1.156327in}{2.926149in}}%
\pgfpathcurveto{\pgfqpoint{1.165536in}{2.916941in}}{\pgfqpoint{1.178027in}{2.911767in}}{\pgfqpoint{1.191049in}{2.911767in}}%
\pgfpathlineto{\pgfqpoint{1.191049in}{2.911767in}}%
\pgfpathclose%
\pgfusepath{stroke,fill}%
\end{pgfscope}%
\begin{pgfscope}%
\pgfpathrectangle{\pgfqpoint{0.786164in}{0.768110in}}{\pgfqpoint{8.851069in}{7.081890in}}%
\pgfusepath{clip}%
\pgfsetbuttcap%
\pgfsetroundjoin%
\definecolor{currentfill}{rgb}{0.282884,0.135920,0.453427}%
\pgfsetfillcolor{currentfill}%
\pgfsetfillopacity{0.700000}%
\pgfsetlinewidth{0.501875pt}%
\definecolor{currentstroke}{rgb}{1.000000,1.000000,1.000000}%
\pgfsetstrokecolor{currentstroke}%
\pgfsetstrokeopacity{0.700000}%
\pgfsetdash{}{0pt}%
\pgfpathmoveto{\pgfqpoint{1.194224in}{2.911767in}}%
\pgfpathcurveto{\pgfqpoint{1.207247in}{2.911767in}}{\pgfqpoint{1.219738in}{2.916941in}}{\pgfqpoint{1.228946in}{2.926149in}}%
\pgfpathcurveto{\pgfqpoint{1.238155in}{2.935358in}}{\pgfqpoint{1.243329in}{2.947849in}}{\pgfqpoint{1.243329in}{2.960871in}}%
\pgfpathcurveto{\pgfqpoint{1.243329in}{2.973894in}}{\pgfqpoint{1.238155in}{2.986385in}}{\pgfqpoint{1.228946in}{2.995594in}}%
\pgfpathcurveto{\pgfqpoint{1.219738in}{3.004802in}}{\pgfqpoint{1.207247in}{3.009976in}}{\pgfqpoint{1.194224in}{3.009976in}}%
\pgfpathcurveto{\pgfqpoint{1.181201in}{3.009976in}}{\pgfqpoint{1.168710in}{3.004802in}}{\pgfqpoint{1.159502in}{2.995594in}}%
\pgfpathcurveto{\pgfqpoint{1.150293in}{2.986385in}}{\pgfqpoint{1.145119in}{2.973894in}}{\pgfqpoint{1.145119in}{2.960871in}}%
\pgfpathcurveto{\pgfqpoint{1.145119in}{2.947849in}}{\pgfqpoint{1.150293in}{2.935358in}}{\pgfqpoint{1.159502in}{2.926149in}}%
\pgfpathcurveto{\pgfqpoint{1.168710in}{2.916941in}}{\pgfqpoint{1.181201in}{2.911767in}}{\pgfqpoint{1.194224in}{2.911767in}}%
\pgfpathlineto{\pgfqpoint{1.194224in}{2.911767in}}%
\pgfpathclose%
\pgfusepath{stroke,fill}%
\end{pgfscope}%
\begin{pgfscope}%
\pgfpathrectangle{\pgfqpoint{0.786164in}{0.768110in}}{\pgfqpoint{8.851069in}{7.081890in}}%
\pgfusepath{clip}%
\pgfsetbuttcap%
\pgfsetroundjoin%
\definecolor{currentfill}{rgb}{0.282290,0.145912,0.461510}%
\pgfsetfillcolor{currentfill}%
\pgfsetfillopacity{0.700000}%
\pgfsetlinewidth{0.501875pt}%
\definecolor{currentstroke}{rgb}{1.000000,1.000000,1.000000}%
\pgfsetstrokecolor{currentstroke}%
\pgfsetstrokeopacity{0.700000}%
\pgfsetdash{}{0pt}%
\pgfpathmoveto{\pgfqpoint{1.197643in}{2.975793in}}%
\pgfpathcurveto{\pgfqpoint{1.210665in}{2.975793in}}{\pgfqpoint{1.223157in}{2.980967in}}{\pgfqpoint{1.232365in}{2.990176in}}%
\pgfpathcurveto{\pgfqpoint{1.241573in}{2.999384in}}{\pgfqpoint{1.246747in}{3.011875in}}{\pgfqpoint{1.246747in}{3.024898in}}%
\pgfpathcurveto{\pgfqpoint{1.246747in}{3.037921in}}{\pgfqpoint{1.241573in}{3.050412in}}{\pgfqpoint{1.232365in}{3.059620in}}%
\pgfpathcurveto{\pgfqpoint{1.223157in}{3.068829in}}{\pgfqpoint{1.210665in}{3.074003in}}{\pgfqpoint{1.197643in}{3.074003in}}%
\pgfpathcurveto{\pgfqpoint{1.184620in}{3.074003in}}{\pgfqpoint{1.172129in}{3.068829in}}{\pgfqpoint{1.162921in}{3.059620in}}%
\pgfpathcurveto{\pgfqpoint{1.153712in}{3.050412in}}{\pgfqpoint{1.148538in}{3.037921in}}{\pgfqpoint{1.148538in}{3.024898in}}%
\pgfpathcurveto{\pgfqpoint{1.148538in}{3.011875in}}{\pgfqpoint{1.153712in}{2.999384in}}{\pgfqpoint{1.162921in}{2.990176in}}%
\pgfpathcurveto{\pgfqpoint{1.172129in}{2.980967in}}{\pgfqpoint{1.184620in}{2.975793in}}{\pgfqpoint{1.197643in}{2.975793in}}%
\pgfpathlineto{\pgfqpoint{1.197643in}{2.975793in}}%
\pgfpathclose%
\pgfusepath{stroke,fill}%
\end{pgfscope}%
\begin{pgfscope}%
\pgfpathrectangle{\pgfqpoint{0.786164in}{0.768110in}}{\pgfqpoint{8.851069in}{7.081890in}}%
\pgfusepath{clip}%
\pgfsetbuttcap%
\pgfsetroundjoin%
\definecolor{currentfill}{rgb}{0.282290,0.145912,0.461510}%
\pgfsetfillcolor{currentfill}%
\pgfsetfillopacity{0.700000}%
\pgfsetlinewidth{0.501875pt}%
\definecolor{currentstroke}{rgb}{1.000000,1.000000,1.000000}%
\pgfsetstrokecolor{currentstroke}%
\pgfsetstrokeopacity{0.700000}%
\pgfsetdash{}{0pt}%
\pgfpathmoveto{\pgfqpoint{1.199841in}{2.933109in}}%
\pgfpathcurveto{\pgfqpoint{1.212863in}{2.933109in}}{\pgfqpoint{1.225354in}{2.938283in}}{\pgfqpoint{1.234563in}{2.947491in}}%
\pgfpathcurveto{\pgfqpoint{1.243771in}{2.956700in}}{\pgfqpoint{1.248945in}{2.969191in}}{\pgfqpoint{1.248945in}{2.982214in}}%
\pgfpathcurveto{\pgfqpoint{1.248945in}{2.995236in}}{\pgfqpoint{1.243771in}{3.007727in}}{\pgfqpoint{1.234563in}{3.016936in}}%
\pgfpathcurveto{\pgfqpoint{1.225354in}{3.026144in}}{\pgfqpoint{1.212863in}{3.031318in}}{\pgfqpoint{1.199841in}{3.031318in}}%
\pgfpathcurveto{\pgfqpoint{1.186818in}{3.031318in}}{\pgfqpoint{1.174327in}{3.026144in}}{\pgfqpoint{1.165118in}{3.016936in}}%
\pgfpathcurveto{\pgfqpoint{1.155910in}{3.007727in}}{\pgfqpoint{1.150736in}{2.995236in}}{\pgfqpoint{1.150736in}{2.982214in}}%
\pgfpathcurveto{\pgfqpoint{1.150736in}{2.969191in}}{\pgfqpoint{1.155910in}{2.956700in}}{\pgfqpoint{1.165118in}{2.947491in}}%
\pgfpathcurveto{\pgfqpoint{1.174327in}{2.938283in}}{\pgfqpoint{1.186818in}{2.933109in}}{\pgfqpoint{1.199841in}{2.933109in}}%
\pgfpathlineto{\pgfqpoint{1.199841in}{2.933109in}}%
\pgfpathclose%
\pgfusepath{stroke,fill}%
\end{pgfscope}%
\begin{pgfscope}%
\pgfpathrectangle{\pgfqpoint{0.786164in}{0.768110in}}{\pgfqpoint{8.851069in}{7.081890in}}%
\pgfusepath{clip}%
\pgfsetbuttcap%
\pgfsetroundjoin%
\definecolor{currentfill}{rgb}{0.283197,0.115680,0.436115}%
\pgfsetfillcolor{currentfill}%
\pgfsetfillopacity{0.700000}%
\pgfsetlinewidth{0.501875pt}%
\definecolor{currentstroke}{rgb}{1.000000,1.000000,1.000000}%
\pgfsetstrokecolor{currentstroke}%
\pgfsetstrokeopacity{0.700000}%
\pgfsetdash{}{0pt}%
\pgfpathmoveto{\pgfqpoint{1.203015in}{2.826398in}}%
\pgfpathcurveto{\pgfqpoint{1.216038in}{2.826398in}}{\pgfqpoint{1.228529in}{2.831572in}}{\pgfqpoint{1.237737in}{2.840781in}}%
\pgfpathcurveto{\pgfqpoint{1.246946in}{2.849989in}}{\pgfqpoint{1.252120in}{2.862480in}}{\pgfqpoint{1.252120in}{2.875503in}}%
\pgfpathcurveto{\pgfqpoint{1.252120in}{2.888525in}}{\pgfqpoint{1.246946in}{2.901017in}}{\pgfqpoint{1.237737in}{2.910225in}}%
\pgfpathcurveto{\pgfqpoint{1.228529in}{2.919433in}}{\pgfqpoint{1.216038in}{2.924607in}}{\pgfqpoint{1.203015in}{2.924607in}}%
\pgfpathcurveto{\pgfqpoint{1.189992in}{2.924607in}}{\pgfqpoint{1.177501in}{2.919433in}}{\pgfqpoint{1.168293in}{2.910225in}}%
\pgfpathcurveto{\pgfqpoint{1.159085in}{2.901017in}}{\pgfqpoint{1.153911in}{2.888525in}}{\pgfqpoint{1.153911in}{2.875503in}}%
\pgfpathcurveto{\pgfqpoint{1.153911in}{2.862480in}}{\pgfqpoint{1.159085in}{2.849989in}}{\pgfqpoint{1.168293in}{2.840781in}}%
\pgfpathcurveto{\pgfqpoint{1.177501in}{2.831572in}}{\pgfqpoint{1.189992in}{2.826398in}}{\pgfqpoint{1.203015in}{2.826398in}}%
\pgfpathlineto{\pgfqpoint{1.203015in}{2.826398in}}%
\pgfpathclose%
\pgfusepath{stroke,fill}%
\end{pgfscope}%
\begin{pgfscope}%
\pgfpathrectangle{\pgfqpoint{0.786164in}{0.768110in}}{\pgfqpoint{8.851069in}{7.081890in}}%
\pgfusepath{clip}%
\pgfsetbuttcap%
\pgfsetroundjoin%
\definecolor{currentfill}{rgb}{0.283072,0.130895,0.449241}%
\pgfsetfillcolor{currentfill}%
\pgfsetfillopacity{0.700000}%
\pgfsetlinewidth{0.501875pt}%
\definecolor{currentstroke}{rgb}{1.000000,1.000000,1.000000}%
\pgfsetstrokecolor{currentstroke}%
\pgfsetstrokeopacity{0.700000}%
\pgfsetdash{}{0pt}%
\pgfpathmoveto{\pgfqpoint{1.295567in}{2.869082in}}%
\pgfpathcurveto{\pgfqpoint{1.308590in}{2.869082in}}{\pgfqpoint{1.321081in}{2.874256in}}{\pgfqpoint{1.330290in}{2.883465in}}%
\pgfpathcurveto{\pgfqpoint{1.339498in}{2.892673in}}{\pgfqpoint{1.344672in}{2.905164in}}{\pgfqpoint{1.344672in}{2.918187in}}%
\pgfpathcurveto{\pgfqpoint{1.344672in}{2.931210in}}{\pgfqpoint{1.339498in}{2.943701in}}{\pgfqpoint{1.330290in}{2.952909in}}%
\pgfpathcurveto{\pgfqpoint{1.321081in}{2.962118in}}{\pgfqpoint{1.308590in}{2.967292in}}{\pgfqpoint{1.295567in}{2.967292in}}%
\pgfpathcurveto{\pgfqpoint{1.282545in}{2.967292in}}{\pgfqpoint{1.270054in}{2.962118in}}{\pgfqpoint{1.260845in}{2.952909in}}%
\pgfpathcurveto{\pgfqpoint{1.251637in}{2.943701in}}{\pgfqpoint{1.246463in}{2.931210in}}{\pgfqpoint{1.246463in}{2.918187in}}%
\pgfpathcurveto{\pgfqpoint{1.246463in}{2.905164in}}{\pgfqpoint{1.251637in}{2.892673in}}{\pgfqpoint{1.260845in}{2.883465in}}%
\pgfpathcurveto{\pgfqpoint{1.270054in}{2.874256in}}{\pgfqpoint{1.282545in}{2.869082in}}{\pgfqpoint{1.295567in}{2.869082in}}%
\pgfpathlineto{\pgfqpoint{1.295567in}{2.869082in}}%
\pgfpathclose%
\pgfusepath{stroke,fill}%
\end{pgfscope}%
\begin{pgfscope}%
\pgfpathrectangle{\pgfqpoint{0.786164in}{0.768110in}}{\pgfqpoint{8.851069in}{7.081890in}}%
\pgfusepath{clip}%
\pgfsetbuttcap%
\pgfsetroundjoin%
\definecolor{currentfill}{rgb}{0.283187,0.125848,0.444960}%
\pgfsetfillcolor{currentfill}%
\pgfsetfillopacity{0.700000}%
\pgfsetlinewidth{0.501875pt}%
\definecolor{currentstroke}{rgb}{1.000000,1.000000,1.000000}%
\pgfsetstrokecolor{currentstroke}%
\pgfsetstrokeopacity{0.700000}%
\pgfsetdash{}{0pt}%
\pgfpathmoveto{\pgfqpoint{1.401917in}{2.869082in}}%
\pgfpathcurveto{\pgfqpoint{1.414940in}{2.869082in}}{\pgfqpoint{1.427431in}{2.874256in}}{\pgfqpoint{1.436639in}{2.883465in}}%
\pgfpathcurveto{\pgfqpoint{1.445848in}{2.892673in}}{\pgfqpoint{1.451022in}{2.905164in}}{\pgfqpoint{1.451022in}{2.918187in}}%
\pgfpathcurveto{\pgfqpoint{1.451022in}{2.931210in}}{\pgfqpoint{1.445848in}{2.943701in}}{\pgfqpoint{1.436639in}{2.952909in}}%
\pgfpathcurveto{\pgfqpoint{1.427431in}{2.962118in}}{\pgfqpoint{1.414940in}{2.967292in}}{\pgfqpoint{1.401917in}{2.967292in}}%
\pgfpathcurveto{\pgfqpoint{1.388894in}{2.967292in}}{\pgfqpoint{1.376403in}{2.962118in}}{\pgfqpoint{1.367195in}{2.952909in}}%
\pgfpathcurveto{\pgfqpoint{1.357986in}{2.943701in}}{\pgfqpoint{1.352812in}{2.931210in}}{\pgfqpoint{1.352812in}{2.918187in}}%
\pgfpathcurveto{\pgfqpoint{1.352812in}{2.905164in}}{\pgfqpoint{1.357986in}{2.892673in}}{\pgfqpoint{1.367195in}{2.883465in}}%
\pgfpathcurveto{\pgfqpoint{1.376403in}{2.874256in}}{\pgfqpoint{1.388894in}{2.869082in}}{\pgfqpoint{1.401917in}{2.869082in}}%
\pgfpathlineto{\pgfqpoint{1.401917in}{2.869082in}}%
\pgfpathclose%
\pgfusepath{stroke,fill}%
\end{pgfscope}%
\begin{pgfscope}%
\pgfpathrectangle{\pgfqpoint{0.786164in}{0.768110in}}{\pgfqpoint{8.851069in}{7.081890in}}%
\pgfusepath{clip}%
\pgfsetbuttcap%
\pgfsetroundjoin%
\definecolor{currentfill}{rgb}{0.282623,0.140926,0.457517}%
\pgfsetfillcolor{currentfill}%
\pgfsetfillopacity{0.700000}%
\pgfsetlinewidth{0.501875pt}%
\definecolor{currentstroke}{rgb}{1.000000,1.000000,1.000000}%
\pgfsetstrokecolor{currentstroke}%
\pgfsetstrokeopacity{0.700000}%
\pgfsetdash{}{0pt}%
\pgfpathmoveto{\pgfqpoint{1.525483in}{2.933109in}}%
\pgfpathcurveto{\pgfqpoint{1.538505in}{2.933109in}}{\pgfqpoint{1.550997in}{2.938283in}}{\pgfqpoint{1.560205in}{2.947491in}}%
\pgfpathcurveto{\pgfqpoint{1.569413in}{2.956700in}}{\pgfqpoint{1.574587in}{2.969191in}}{\pgfqpoint{1.574587in}{2.982214in}}%
\pgfpathcurveto{\pgfqpoint{1.574587in}{2.995236in}}{\pgfqpoint{1.569413in}{3.007727in}}{\pgfqpoint{1.560205in}{3.016936in}}%
\pgfpathcurveto{\pgfqpoint{1.550997in}{3.026144in}}{\pgfqpoint{1.538505in}{3.031318in}}{\pgfqpoint{1.525483in}{3.031318in}}%
\pgfpathcurveto{\pgfqpoint{1.512460in}{3.031318in}}{\pgfqpoint{1.499969in}{3.026144in}}{\pgfqpoint{1.490760in}{3.016936in}}%
\pgfpathcurveto{\pgfqpoint{1.481552in}{3.007727in}}{\pgfqpoint{1.476378in}{2.995236in}}{\pgfqpoint{1.476378in}{2.982214in}}%
\pgfpathcurveto{\pgfqpoint{1.476378in}{2.969191in}}{\pgfqpoint{1.481552in}{2.956700in}}{\pgfqpoint{1.490760in}{2.947491in}}%
\pgfpathcurveto{\pgfqpoint{1.499969in}{2.938283in}}{\pgfqpoint{1.512460in}{2.933109in}}{\pgfqpoint{1.525483in}{2.933109in}}%
\pgfpathlineto{\pgfqpoint{1.525483in}{2.933109in}}%
\pgfpathclose%
\pgfusepath{stroke,fill}%
\end{pgfscope}%
\begin{pgfscope}%
\pgfpathrectangle{\pgfqpoint{0.786164in}{0.768110in}}{\pgfqpoint{8.851069in}{7.081890in}}%
\pgfusepath{clip}%
\pgfsetbuttcap%
\pgfsetroundjoin%
\definecolor{currentfill}{rgb}{0.282656,0.100196,0.422160}%
\pgfsetfillcolor{currentfill}%
\pgfsetfillopacity{0.700000}%
\pgfsetlinewidth{0.501875pt}%
\definecolor{currentstroke}{rgb}{1.000000,1.000000,1.000000}%
\pgfsetstrokecolor{currentstroke}%
\pgfsetstrokeopacity{0.700000}%
\pgfsetdash{}{0pt}%
\pgfpathmoveto{\pgfqpoint{1.635129in}{2.762372in}}%
\pgfpathcurveto{\pgfqpoint{1.648152in}{2.762372in}}{\pgfqpoint{1.660643in}{2.767546in}}{\pgfqpoint{1.669851in}{2.776754in}}%
\pgfpathcurveto{\pgfqpoint{1.679060in}{2.785962in}}{\pgfqpoint{1.684234in}{2.798454in}}{\pgfqpoint{1.684234in}{2.811476in}}%
\pgfpathcurveto{\pgfqpoint{1.684234in}{2.824499in}}{\pgfqpoint{1.679060in}{2.836990in}}{\pgfqpoint{1.669851in}{2.846198in}}%
\pgfpathcurveto{\pgfqpoint{1.660643in}{2.855407in}}{\pgfqpoint{1.648152in}{2.860581in}}{\pgfqpoint{1.635129in}{2.860581in}}%
\pgfpathcurveto{\pgfqpoint{1.622106in}{2.860581in}}{\pgfqpoint{1.609615in}{2.855407in}}{\pgfqpoint{1.600407in}{2.846198in}}%
\pgfpathcurveto{\pgfqpoint{1.591198in}{2.836990in}}{\pgfqpoint{1.586024in}{2.824499in}}{\pgfqpoint{1.586024in}{2.811476in}}%
\pgfpathcurveto{\pgfqpoint{1.586024in}{2.798454in}}{\pgfqpoint{1.591198in}{2.785962in}}{\pgfqpoint{1.600407in}{2.776754in}}%
\pgfpathcurveto{\pgfqpoint{1.609615in}{2.767546in}}{\pgfqpoint{1.622106in}{2.762372in}}{\pgfqpoint{1.635129in}{2.762372in}}%
\pgfpathlineto{\pgfqpoint{1.635129in}{2.762372in}}%
\pgfpathclose%
\pgfusepath{stroke,fill}%
\end{pgfscope}%
\begin{pgfscope}%
\pgfpathrectangle{\pgfqpoint{0.786164in}{0.768110in}}{\pgfqpoint{8.851069in}{7.081890in}}%
\pgfusepath{clip}%
\pgfsetbuttcap%
\pgfsetroundjoin%
\definecolor{currentfill}{rgb}{0.282656,0.100196,0.422160}%
\pgfsetfillcolor{currentfill}%
\pgfsetfillopacity{0.700000}%
\pgfsetlinewidth{0.501875pt}%
\definecolor{currentstroke}{rgb}{1.000000,1.000000,1.000000}%
\pgfsetstrokecolor{currentstroke}%
\pgfsetstrokeopacity{0.700000}%
\pgfsetdash{}{0pt}%
\pgfpathmoveto{\pgfqpoint{1.755276in}{2.741029in}}%
\pgfpathcurveto{\pgfqpoint{1.768299in}{2.741029in}}{\pgfqpoint{1.780790in}{2.746203in}}{\pgfqpoint{1.789998in}{2.755412in}}%
\pgfpathcurveto{\pgfqpoint{1.799207in}{2.764620in}}{\pgfqpoint{1.804381in}{2.777111in}}{\pgfqpoint{1.804381in}{2.790134in}}%
\pgfpathcurveto{\pgfqpoint{1.804381in}{2.803157in}}{\pgfqpoint{1.799207in}{2.815648in}}{\pgfqpoint{1.789998in}{2.824856in}}%
\pgfpathcurveto{\pgfqpoint{1.780790in}{2.834065in}}{\pgfqpoint{1.768299in}{2.839239in}}{\pgfqpoint{1.755276in}{2.839239in}}%
\pgfpathcurveto{\pgfqpoint{1.742253in}{2.839239in}}{\pgfqpoint{1.729762in}{2.834065in}}{\pgfqpoint{1.720554in}{2.824856in}}%
\pgfpathcurveto{\pgfqpoint{1.711345in}{2.815648in}}{\pgfqpoint{1.706171in}{2.803157in}}{\pgfqpoint{1.706171in}{2.790134in}}%
\pgfpathcurveto{\pgfqpoint{1.706171in}{2.777111in}}{\pgfqpoint{1.711345in}{2.764620in}}{\pgfqpoint{1.720554in}{2.755412in}}%
\pgfpathcurveto{\pgfqpoint{1.729762in}{2.746203in}}{\pgfqpoint{1.742253in}{2.741029in}}{\pgfqpoint{1.755276in}{2.741029in}}%
\pgfpathlineto{\pgfqpoint{1.755276in}{2.741029in}}%
\pgfpathclose%
\pgfusepath{stroke,fill}%
\end{pgfscope}%
\begin{pgfscope}%
\pgfpathrectangle{\pgfqpoint{0.786164in}{0.768110in}}{\pgfqpoint{8.851069in}{7.081890in}}%
\pgfusepath{clip}%
\pgfsetbuttcap%
\pgfsetroundjoin%
\definecolor{currentfill}{rgb}{0.276022,0.044167,0.370164}%
\pgfsetfillcolor{currentfill}%
\pgfsetfillopacity{0.700000}%
\pgfsetlinewidth{0.501875pt}%
\definecolor{currentstroke}{rgb}{1.000000,1.000000,1.000000}%
\pgfsetstrokecolor{currentstroke}%
\pgfsetstrokeopacity{0.700000}%
\pgfsetdash{}{0pt}%
\pgfpathmoveto{\pgfqpoint{1.801064in}{2.378213in}}%
\pgfpathcurveto{\pgfqpoint{1.814086in}{2.378213in}}{\pgfqpoint{1.826577in}{2.383386in}}{\pgfqpoint{1.835786in}{2.392595in}}%
\pgfpathcurveto{\pgfqpoint{1.844994in}{2.401803in}}{\pgfqpoint{1.850168in}{2.414294in}}{\pgfqpoint{1.850168in}{2.427317in}}%
\pgfpathcurveto{\pgfqpoint{1.850168in}{2.440340in}}{\pgfqpoint{1.844994in}{2.452831in}}{\pgfqpoint{1.835786in}{2.462039in}}%
\pgfpathcurveto{\pgfqpoint{1.826577in}{2.471248in}}{\pgfqpoint{1.814086in}{2.476422in}}{\pgfqpoint{1.801064in}{2.476422in}}%
\pgfpathcurveto{\pgfqpoint{1.788041in}{2.476422in}}{\pgfqpoint{1.775550in}{2.471248in}}{\pgfqpoint{1.766341in}{2.462039in}}%
\pgfpathcurveto{\pgfqpoint{1.757133in}{2.452831in}}{\pgfqpoint{1.751959in}{2.440340in}}{\pgfqpoint{1.751959in}{2.427317in}}%
\pgfpathcurveto{\pgfqpoint{1.751959in}{2.414294in}}{\pgfqpoint{1.757133in}{2.401803in}}{\pgfqpoint{1.766341in}{2.392595in}}%
\pgfpathcurveto{\pgfqpoint{1.775550in}{2.383386in}}{\pgfqpoint{1.788041in}{2.378213in}}{\pgfqpoint{1.801064in}{2.378213in}}%
\pgfpathlineto{\pgfqpoint{1.801064in}{2.378213in}}%
\pgfpathclose%
\pgfusepath{stroke,fill}%
\end{pgfscope}%
\begin{pgfscope}%
\pgfpathrectangle{\pgfqpoint{0.786164in}{0.768110in}}{\pgfqpoint{8.851069in}{7.081890in}}%
\pgfusepath{clip}%
\pgfsetbuttcap%
\pgfsetroundjoin%
\definecolor{currentfill}{rgb}{0.271305,0.019942,0.347269}%
\pgfsetfillcolor{currentfill}%
\pgfsetfillopacity{0.700000}%
\pgfsetlinewidth{0.501875pt}%
\definecolor{currentstroke}{rgb}{1.000000,1.000000,1.000000}%
\pgfsetstrokecolor{currentstroke}%
\pgfsetstrokeopacity{0.700000}%
\pgfsetdash{}{0pt}%
\pgfpathmoveto{\pgfqpoint{1.934031in}{2.228817in}}%
\pgfpathcurveto{\pgfqpoint{1.947054in}{2.228817in}}{\pgfqpoint{1.959545in}{2.233991in}}{\pgfqpoint{1.968753in}{2.243200in}}%
\pgfpathcurveto{\pgfqpoint{1.977962in}{2.252408in}}{\pgfqpoint{1.983136in}{2.264899in}}{\pgfqpoint{1.983136in}{2.277922in}}%
\pgfpathcurveto{\pgfqpoint{1.983136in}{2.290945in}}{\pgfqpoint{1.977962in}{2.303436in}}{\pgfqpoint{1.968753in}{2.312644in}}%
\pgfpathcurveto{\pgfqpoint{1.959545in}{2.321853in}}{\pgfqpoint{1.947054in}{2.327027in}}{\pgfqpoint{1.934031in}{2.327027in}}%
\pgfpathcurveto{\pgfqpoint{1.921008in}{2.327027in}}{\pgfqpoint{1.908517in}{2.321853in}}{\pgfqpoint{1.899309in}{2.312644in}}%
\pgfpathcurveto{\pgfqpoint{1.890100in}{2.303436in}}{\pgfqpoint{1.884927in}{2.290945in}}{\pgfqpoint{1.884927in}{2.277922in}}%
\pgfpathcurveto{\pgfqpoint{1.884927in}{2.264899in}}{\pgfqpoint{1.890100in}{2.252408in}}{\pgfqpoint{1.899309in}{2.243200in}}%
\pgfpathcurveto{\pgfqpoint{1.908517in}{2.233991in}}{\pgfqpoint{1.921008in}{2.228817in}}{\pgfqpoint{1.934031in}{2.228817in}}%
\pgfpathlineto{\pgfqpoint{1.934031in}{2.228817in}}%
\pgfpathclose%
\pgfusepath{stroke,fill}%
\end{pgfscope}%
\begin{pgfscope}%
\pgfpathrectangle{\pgfqpoint{0.786164in}{0.768110in}}{\pgfqpoint{8.851069in}{7.081890in}}%
\pgfusepath{clip}%
\pgfsetbuttcap%
\pgfsetroundjoin%
\definecolor{currentfill}{rgb}{0.277018,0.050344,0.375715}%
\pgfsetfillcolor{currentfill}%
\pgfsetfillopacity{0.700000}%
\pgfsetlinewidth{0.501875pt}%
\definecolor{currentstroke}{rgb}{1.000000,1.000000,1.000000}%
\pgfsetstrokecolor{currentstroke}%
\pgfsetstrokeopacity{0.700000}%
\pgfsetdash{}{0pt}%
\pgfpathmoveto{\pgfqpoint{2.057475in}{2.292844in}}%
\pgfpathcurveto{\pgfqpoint{2.070498in}{2.292844in}}{\pgfqpoint{2.082989in}{2.298018in}}{\pgfqpoint{2.092197in}{2.307226in}}%
\pgfpathcurveto{\pgfqpoint{2.101405in}{2.316435in}}{\pgfqpoint{2.106579in}{2.328926in}}{\pgfqpoint{2.106579in}{2.341948in}}%
\pgfpathcurveto{\pgfqpoint{2.106579in}{2.354971in}}{\pgfqpoint{2.101405in}{2.367462in}}{\pgfqpoint{2.092197in}{2.376671in}}%
\pgfpathcurveto{\pgfqpoint{2.082989in}{2.385879in}}{\pgfqpoint{2.070498in}{2.391053in}}{\pgfqpoint{2.057475in}{2.391053in}}%
\pgfpathcurveto{\pgfqpoint{2.044452in}{2.391053in}}{\pgfqpoint{2.031961in}{2.385879in}}{\pgfqpoint{2.022753in}{2.376671in}}%
\pgfpathcurveto{\pgfqpoint{2.013544in}{2.367462in}}{\pgfqpoint{2.008370in}{2.354971in}}{\pgfqpoint{2.008370in}{2.341948in}}%
\pgfpathcurveto{\pgfqpoint{2.008370in}{2.328926in}}{\pgfqpoint{2.013544in}{2.316435in}}{\pgfqpoint{2.022753in}{2.307226in}}%
\pgfpathcurveto{\pgfqpoint{2.031961in}{2.298018in}}{\pgfqpoint{2.044452in}{2.292844in}}{\pgfqpoint{2.057475in}{2.292844in}}%
\pgfpathlineto{\pgfqpoint{2.057475in}{2.292844in}}%
\pgfpathclose%
\pgfusepath{stroke,fill}%
\end{pgfscope}%
\begin{pgfscope}%
\pgfpathrectangle{\pgfqpoint{0.786164in}{0.768110in}}{\pgfqpoint{8.851069in}{7.081890in}}%
\pgfusepath{clip}%
\pgfsetbuttcap%
\pgfsetroundjoin%
\definecolor{currentfill}{rgb}{0.276022,0.044167,0.370164}%
\pgfsetfillcolor{currentfill}%
\pgfsetfillopacity{0.700000}%
\pgfsetlinewidth{0.501875pt}%
\definecolor{currentstroke}{rgb}{1.000000,1.000000,1.000000}%
\pgfsetstrokecolor{currentstroke}%
\pgfsetstrokeopacity{0.700000}%
\pgfsetdash{}{0pt}%
\pgfpathmoveto{\pgfqpoint{2.142335in}{2.271502in}}%
\pgfpathcurveto{\pgfqpoint{2.155357in}{2.271502in}}{\pgfqpoint{2.167848in}{2.276676in}}{\pgfqpoint{2.177057in}{2.285884in}}%
\pgfpathcurveto{\pgfqpoint{2.186265in}{2.295093in}}{\pgfqpoint{2.191439in}{2.307584in}}{\pgfqpoint{2.191439in}{2.320606in}}%
\pgfpathcurveto{\pgfqpoint{2.191439in}{2.333629in}}{\pgfqpoint{2.186265in}{2.346120in}}{\pgfqpoint{2.177057in}{2.355329in}}%
\pgfpathcurveto{\pgfqpoint{2.167848in}{2.364537in}}{\pgfqpoint{2.155357in}{2.369711in}}{\pgfqpoint{2.142335in}{2.369711in}}%
\pgfpathcurveto{\pgfqpoint{2.129312in}{2.369711in}}{\pgfqpoint{2.116821in}{2.364537in}}{\pgfqpoint{2.107612in}{2.355329in}}%
\pgfpathcurveto{\pgfqpoint{2.098404in}{2.346120in}}{\pgfqpoint{2.093230in}{2.333629in}}{\pgfqpoint{2.093230in}{2.320606in}}%
\pgfpathcurveto{\pgfqpoint{2.093230in}{2.307584in}}{\pgfqpoint{2.098404in}{2.295093in}}{\pgfqpoint{2.107612in}{2.285884in}}%
\pgfpathcurveto{\pgfqpoint{2.116821in}{2.276676in}}{\pgfqpoint{2.129312in}{2.271502in}}{\pgfqpoint{2.142335in}{2.271502in}}%
\pgfpathlineto{\pgfqpoint{2.142335in}{2.271502in}}%
\pgfpathclose%
\pgfusepath{stroke,fill}%
\end{pgfscope}%
\begin{pgfscope}%
\pgfpathrectangle{\pgfqpoint{0.786164in}{0.768110in}}{\pgfqpoint{8.851069in}{7.081890in}}%
\pgfusepath{clip}%
\pgfsetbuttcap%
\pgfsetroundjoin%
\definecolor{currentfill}{rgb}{0.277018,0.050344,0.375715}%
\pgfsetfillcolor{currentfill}%
\pgfsetfillopacity{0.700000}%
\pgfsetlinewidth{0.501875pt}%
\definecolor{currentstroke}{rgb}{1.000000,1.000000,1.000000}%
\pgfsetstrokecolor{currentstroke}%
\pgfsetstrokeopacity{0.700000}%
\pgfsetdash{}{0pt}%
\pgfpathmoveto{\pgfqpoint{2.180918in}{2.292844in}}%
\pgfpathcurveto{\pgfqpoint{2.193941in}{2.292844in}}{\pgfqpoint{2.206432in}{2.298018in}}{\pgfqpoint{2.215641in}{2.307226in}}%
\pgfpathcurveto{\pgfqpoint{2.224849in}{2.316435in}}{\pgfqpoint{2.230023in}{2.328926in}}{\pgfqpoint{2.230023in}{2.341948in}}%
\pgfpathcurveto{\pgfqpoint{2.230023in}{2.354971in}}{\pgfqpoint{2.224849in}{2.367462in}}{\pgfqpoint{2.215641in}{2.376671in}}%
\pgfpathcurveto{\pgfqpoint{2.206432in}{2.385879in}}{\pgfqpoint{2.193941in}{2.391053in}}{\pgfqpoint{2.180918in}{2.391053in}}%
\pgfpathcurveto{\pgfqpoint{2.167896in}{2.391053in}}{\pgfqpoint{2.155405in}{2.385879in}}{\pgfqpoint{2.146196in}{2.376671in}}%
\pgfpathcurveto{\pgfqpoint{2.136988in}{2.367462in}}{\pgfqpoint{2.131814in}{2.354971in}}{\pgfqpoint{2.131814in}{2.341948in}}%
\pgfpathcurveto{\pgfqpoint{2.131814in}{2.328926in}}{\pgfqpoint{2.136988in}{2.316435in}}{\pgfqpoint{2.146196in}{2.307226in}}%
\pgfpathcurveto{\pgfqpoint{2.155405in}{2.298018in}}{\pgfqpoint{2.167896in}{2.292844in}}{\pgfqpoint{2.180918in}{2.292844in}}%
\pgfpathlineto{\pgfqpoint{2.180918in}{2.292844in}}%
\pgfpathclose%
\pgfusepath{stroke,fill}%
\end{pgfscope}%
\begin{pgfscope}%
\pgfpathrectangle{\pgfqpoint{0.786164in}{0.768110in}}{\pgfqpoint{8.851069in}{7.081890in}}%
\pgfusepath{clip}%
\pgfsetbuttcap%
\pgfsetroundjoin%
\definecolor{currentfill}{rgb}{0.267004,0.004874,0.329415}%
\pgfsetfillcolor{currentfill}%
\pgfsetfillopacity{0.700000}%
\pgfsetlinewidth{0.501875pt}%
\definecolor{currentstroke}{rgb}{1.000000,1.000000,1.000000}%
\pgfsetstrokecolor{currentstroke}%
\pgfsetstrokeopacity{0.700000}%
\pgfsetdash{}{0pt}%
\pgfpathmoveto{\pgfqpoint{2.484216in}{2.122106in}}%
\pgfpathcurveto{\pgfqpoint{2.497239in}{2.122106in}}{\pgfqpoint{2.509730in}{2.127280in}}{\pgfqpoint{2.518938in}{2.136489in}}%
\pgfpathcurveto{\pgfqpoint{2.528147in}{2.145697in}}{\pgfqpoint{2.533321in}{2.158188in}}{\pgfqpoint{2.533321in}{2.171211in}}%
\pgfpathcurveto{\pgfqpoint{2.533321in}{2.184234in}}{\pgfqpoint{2.528147in}{2.196725in}}{\pgfqpoint{2.518938in}{2.205933in}}%
\pgfpathcurveto{\pgfqpoint{2.509730in}{2.215142in}}{\pgfqpoint{2.497239in}{2.220316in}}{\pgfqpoint{2.484216in}{2.220316in}}%
\pgfpathcurveto{\pgfqpoint{2.471194in}{2.220316in}}{\pgfqpoint{2.458702in}{2.215142in}}{\pgfqpoint{2.449494in}{2.205933in}}%
\pgfpathcurveto{\pgfqpoint{2.440286in}{2.196725in}}{\pgfqpoint{2.435112in}{2.184234in}}{\pgfqpoint{2.435112in}{2.171211in}}%
\pgfpathcurveto{\pgfqpoint{2.435112in}{2.158188in}}{\pgfqpoint{2.440286in}{2.145697in}}{\pgfqpoint{2.449494in}{2.136489in}}%
\pgfpathcurveto{\pgfqpoint{2.458702in}{2.127280in}}{\pgfqpoint{2.471194in}{2.122106in}}{\pgfqpoint{2.484216in}{2.122106in}}%
\pgfpathlineto{\pgfqpoint{2.484216in}{2.122106in}}%
\pgfpathclose%
\pgfusepath{stroke,fill}%
\end{pgfscope}%
\begin{pgfscope}%
\pgfpathrectangle{\pgfqpoint{0.786164in}{0.768110in}}{\pgfqpoint{8.851069in}{7.081890in}}%
\pgfusepath{clip}%
\pgfsetbuttcap%
\pgfsetroundjoin%
\definecolor{currentfill}{rgb}{0.268510,0.009605,0.335427}%
\pgfsetfillcolor{currentfill}%
\pgfsetfillopacity{0.700000}%
\pgfsetlinewidth{0.501875pt}%
\definecolor{currentstroke}{rgb}{1.000000,1.000000,1.000000}%
\pgfsetstrokecolor{currentstroke}%
\pgfsetstrokeopacity{0.700000}%
\pgfsetdash{}{0pt}%
\pgfpathmoveto{\pgfqpoint{2.723289in}{2.143449in}}%
\pgfpathcurveto{\pgfqpoint{2.736312in}{2.143449in}}{\pgfqpoint{2.748803in}{2.148623in}}{\pgfqpoint{2.758011in}{2.157831in}}%
\pgfpathcurveto{\pgfqpoint{2.767220in}{2.167040in}}{\pgfqpoint{2.772394in}{2.179531in}}{\pgfqpoint{2.772394in}{2.192553in}}%
\pgfpathcurveto{\pgfqpoint{2.772394in}{2.205576in}}{\pgfqpoint{2.767220in}{2.218067in}}{\pgfqpoint{2.758011in}{2.227276in}}%
\pgfpathcurveto{\pgfqpoint{2.748803in}{2.236484in}}{\pgfqpoint{2.736312in}{2.241658in}}{\pgfqpoint{2.723289in}{2.241658in}}%
\pgfpathcurveto{\pgfqpoint{2.710266in}{2.241658in}}{\pgfqpoint{2.697775in}{2.236484in}}{\pgfqpoint{2.688567in}{2.227276in}}%
\pgfpathcurveto{\pgfqpoint{2.679358in}{2.218067in}}{\pgfqpoint{2.674184in}{2.205576in}}{\pgfqpoint{2.674184in}{2.192553in}}%
\pgfpathcurveto{\pgfqpoint{2.674184in}{2.179531in}}{\pgfqpoint{2.679358in}{2.167040in}}{\pgfqpoint{2.688567in}{2.157831in}}%
\pgfpathcurveto{\pgfqpoint{2.697775in}{2.148623in}}{\pgfqpoint{2.710266in}{2.143449in}}{\pgfqpoint{2.723289in}{2.143449in}}%
\pgfpathlineto{\pgfqpoint{2.723289in}{2.143449in}}%
\pgfpathclose%
\pgfusepath{stroke,fill}%
\end{pgfscope}%
\begin{pgfscope}%
\pgfpathrectangle{\pgfqpoint{0.786164in}{0.768110in}}{\pgfqpoint{8.851069in}{7.081890in}}%
\pgfusepath{clip}%
\pgfsetbuttcap%
\pgfsetroundjoin%
\definecolor{currentfill}{rgb}{0.274952,0.037752,0.364543}%
\pgfsetfillcolor{currentfill}%
\pgfsetfillopacity{0.700000}%
\pgfsetlinewidth{0.501875pt}%
\definecolor{currentstroke}{rgb}{1.000000,1.000000,1.000000}%
\pgfsetstrokecolor{currentstroke}%
\pgfsetstrokeopacity{0.700000}%
\pgfsetdash{}{0pt}%
\pgfpathmoveto{\pgfqpoint{2.812667in}{2.250159in}}%
\pgfpathcurveto{\pgfqpoint{2.825689in}{2.250159in}}{\pgfqpoint{2.838181in}{2.255333in}}{\pgfqpoint{2.847389in}{2.264542in}}%
\pgfpathcurveto{\pgfqpoint{2.856597in}{2.273750in}}{\pgfqpoint{2.861771in}{2.286241in}}{\pgfqpoint{2.861771in}{2.299264in}}%
\pgfpathcurveto{\pgfqpoint{2.861771in}{2.312287in}}{\pgfqpoint{2.856597in}{2.324778in}}{\pgfqpoint{2.847389in}{2.333986in}}%
\pgfpathcurveto{\pgfqpoint{2.838181in}{2.343195in}}{\pgfqpoint{2.825689in}{2.348369in}}{\pgfqpoint{2.812667in}{2.348369in}}%
\pgfpathcurveto{\pgfqpoint{2.799644in}{2.348369in}}{\pgfqpoint{2.787153in}{2.343195in}}{\pgfqpoint{2.777944in}{2.333986in}}%
\pgfpathcurveto{\pgfqpoint{2.768736in}{2.324778in}}{\pgfqpoint{2.763562in}{2.312287in}}{\pgfqpoint{2.763562in}{2.299264in}}%
\pgfpathcurveto{\pgfqpoint{2.763562in}{2.286241in}}{\pgfqpoint{2.768736in}{2.273750in}}{\pgfqpoint{2.777944in}{2.264542in}}%
\pgfpathcurveto{\pgfqpoint{2.787153in}{2.255333in}}{\pgfqpoint{2.799644in}{2.250159in}}{\pgfqpoint{2.812667in}{2.250159in}}%
\pgfpathlineto{\pgfqpoint{2.812667in}{2.250159in}}%
\pgfpathclose%
\pgfusepath{stroke,fill}%
\end{pgfscope}%
\begin{pgfscope}%
\pgfpathrectangle{\pgfqpoint{0.786164in}{0.768110in}}{\pgfqpoint{8.851069in}{7.081890in}}%
\pgfusepath{clip}%
\pgfsetbuttcap%
\pgfsetroundjoin%
\definecolor{currentfill}{rgb}{0.190631,0.407061,0.556089}%
\pgfsetfillcolor{currentfill}%
\pgfsetfillopacity{0.700000}%
\pgfsetlinewidth{0.501875pt}%
\definecolor{currentstroke}{rgb}{1.000000,1.000000,1.000000}%
\pgfsetstrokecolor{currentstroke}%
\pgfsetstrokeopacity{0.700000}%
\pgfsetdash{}{0pt}%
\pgfpathmoveto{\pgfqpoint{1.423895in}{4.299008in}}%
\pgfpathcurveto{\pgfqpoint{1.436918in}{4.299008in}}{\pgfqpoint{1.449409in}{4.304182in}}{\pgfqpoint{1.458617in}{4.313390in}}%
\pgfpathcurveto{\pgfqpoint{1.467826in}{4.322599in}}{\pgfqpoint{1.473000in}{4.335090in}}{\pgfqpoint{1.473000in}{4.348113in}}%
\pgfpathcurveto{\pgfqpoint{1.473000in}{4.361135in}}{\pgfqpoint{1.467826in}{4.373626in}}{\pgfqpoint{1.458617in}{4.382835in}}%
\pgfpathcurveto{\pgfqpoint{1.449409in}{4.392043in}}{\pgfqpoint{1.436918in}{4.397217in}}{\pgfqpoint{1.423895in}{4.397217in}}%
\pgfpathcurveto{\pgfqpoint{1.410872in}{4.397217in}}{\pgfqpoint{1.398381in}{4.392043in}}{\pgfqpoint{1.389173in}{4.382835in}}%
\pgfpathcurveto{\pgfqpoint{1.379964in}{4.373626in}}{\pgfqpoint{1.374790in}{4.361135in}}{\pgfqpoint{1.374790in}{4.348113in}}%
\pgfpathcurveto{\pgfqpoint{1.374790in}{4.335090in}}{\pgfqpoint{1.379964in}{4.322599in}}{\pgfqpoint{1.389173in}{4.313390in}}%
\pgfpathcurveto{\pgfqpoint{1.398381in}{4.304182in}}{\pgfqpoint{1.410872in}{4.299008in}}{\pgfqpoint{1.423895in}{4.299008in}}%
\pgfpathlineto{\pgfqpoint{1.423895in}{4.299008in}}%
\pgfpathclose%
\pgfusepath{stroke,fill}%
\end{pgfscope}%
\begin{pgfscope}%
\pgfpathrectangle{\pgfqpoint{0.786164in}{0.768110in}}{\pgfqpoint{8.851069in}{7.081890in}}%
\pgfusepath{clip}%
\pgfsetbuttcap%
\pgfsetroundjoin%
\definecolor{currentfill}{rgb}{0.194100,0.399323,0.555565}%
\pgfsetfillcolor{currentfill}%
\pgfsetfillopacity{0.700000}%
\pgfsetlinewidth{0.501875pt}%
\definecolor{currentstroke}{rgb}{1.000000,1.000000,1.000000}%
\pgfsetstrokecolor{currentstroke}%
\pgfsetstrokeopacity{0.700000}%
\pgfsetdash{}{0pt}%
\pgfpathmoveto{\pgfqpoint{1.478596in}{4.213639in}}%
\pgfpathcurveto{\pgfqpoint{1.491619in}{4.213639in}}{\pgfqpoint{1.504110in}{4.218813in}}{\pgfqpoint{1.513318in}{4.228022in}}%
\pgfpathcurveto{\pgfqpoint{1.522527in}{4.237230in}}{\pgfqpoint{1.527701in}{4.249721in}}{\pgfqpoint{1.527701in}{4.262744in}}%
\pgfpathcurveto{\pgfqpoint{1.527701in}{4.275767in}}{\pgfqpoint{1.522527in}{4.288258in}}{\pgfqpoint{1.513318in}{4.297466in}}%
\pgfpathcurveto{\pgfqpoint{1.504110in}{4.306674in}}{\pgfqpoint{1.491619in}{4.311848in}}{\pgfqpoint{1.478596in}{4.311848in}}%
\pgfpathcurveto{\pgfqpoint{1.465573in}{4.311848in}}{\pgfqpoint{1.453082in}{4.306674in}}{\pgfqpoint{1.443874in}{4.297466in}}%
\pgfpathcurveto{\pgfqpoint{1.434665in}{4.288258in}}{\pgfqpoint{1.429491in}{4.275767in}}{\pgfqpoint{1.429491in}{4.262744in}}%
\pgfpathcurveto{\pgfqpoint{1.429491in}{4.249721in}}{\pgfqpoint{1.434665in}{4.237230in}}{\pgfqpoint{1.443874in}{4.228022in}}%
\pgfpathcurveto{\pgfqpoint{1.453082in}{4.218813in}}{\pgfqpoint{1.465573in}{4.213639in}}{\pgfqpoint{1.478596in}{4.213639in}}%
\pgfpathlineto{\pgfqpoint{1.478596in}{4.213639in}}%
\pgfpathclose%
\pgfusepath{stroke,fill}%
\end{pgfscope}%
\begin{pgfscope}%
\pgfpathrectangle{\pgfqpoint{0.786164in}{0.768110in}}{\pgfqpoint{8.851069in}{7.081890in}}%
\pgfusepath{clip}%
\pgfsetbuttcap%
\pgfsetroundjoin%
\definecolor{currentfill}{rgb}{0.197636,0.391528,0.554969}%
\pgfsetfillcolor{currentfill}%
\pgfsetfillopacity{0.700000}%
\pgfsetlinewidth{0.501875pt}%
\definecolor{currentstroke}{rgb}{1.000000,1.000000,1.000000}%
\pgfsetstrokecolor{currentstroke}%
\pgfsetstrokeopacity{0.700000}%
\pgfsetdash{}{0pt}%
\pgfpathmoveto{\pgfqpoint{1.515226in}{4.149613in}}%
\pgfpathcurveto{\pgfqpoint{1.528249in}{4.149613in}}{\pgfqpoint{1.540740in}{4.154787in}}{\pgfqpoint{1.549948in}{4.163995in}}%
\pgfpathcurveto{\pgfqpoint{1.559157in}{4.173204in}}{\pgfqpoint{1.564331in}{4.185695in}}{\pgfqpoint{1.564331in}{4.198717in}}%
\pgfpathcurveto{\pgfqpoint{1.564331in}{4.211740in}}{\pgfqpoint{1.559157in}{4.224231in}}{\pgfqpoint{1.549948in}{4.233440in}}%
\pgfpathcurveto{\pgfqpoint{1.540740in}{4.242648in}}{\pgfqpoint{1.528249in}{4.247822in}}{\pgfqpoint{1.515226in}{4.247822in}}%
\pgfpathcurveto{\pgfqpoint{1.502204in}{4.247822in}}{\pgfqpoint{1.489712in}{4.242648in}}{\pgfqpoint{1.480504in}{4.233440in}}%
\pgfpathcurveto{\pgfqpoint{1.471296in}{4.224231in}}{\pgfqpoint{1.466122in}{4.211740in}}{\pgfqpoint{1.466122in}{4.198717in}}%
\pgfpathcurveto{\pgfqpoint{1.466122in}{4.185695in}}{\pgfqpoint{1.471296in}{4.173204in}}{\pgfqpoint{1.480504in}{4.163995in}}%
\pgfpathcurveto{\pgfqpoint{1.489712in}{4.154787in}}{\pgfqpoint{1.502204in}{4.149613in}}{\pgfqpoint{1.515226in}{4.149613in}}%
\pgfpathlineto{\pgfqpoint{1.515226in}{4.149613in}}%
\pgfpathclose%
\pgfusepath{stroke,fill}%
\end{pgfscope}%
\begin{pgfscope}%
\pgfpathrectangle{\pgfqpoint{0.786164in}{0.768110in}}{\pgfqpoint{8.851069in}{7.081890in}}%
\pgfusepath{clip}%
\pgfsetbuttcap%
\pgfsetroundjoin%
\definecolor{currentfill}{rgb}{0.197636,0.391528,0.554969}%
\pgfsetfillcolor{currentfill}%
\pgfsetfillopacity{0.700000}%
\pgfsetlinewidth{0.501875pt}%
\definecolor{currentstroke}{rgb}{1.000000,1.000000,1.000000}%
\pgfsetstrokecolor{currentstroke}%
\pgfsetstrokeopacity{0.700000}%
\pgfsetdash{}{0pt}%
\pgfpathmoveto{\pgfqpoint{1.578719in}{4.128271in}}%
\pgfpathcurveto{\pgfqpoint{1.591741in}{4.128271in}}{\pgfqpoint{1.604232in}{4.133444in}}{\pgfqpoint{1.613441in}{4.142653in}}%
\pgfpathcurveto{\pgfqpoint{1.622649in}{4.151861in}}{\pgfqpoint{1.627823in}{4.164352in}}{\pgfqpoint{1.627823in}{4.177375in}}%
\pgfpathcurveto{\pgfqpoint{1.627823in}{4.190398in}}{\pgfqpoint{1.622649in}{4.202889in}}{\pgfqpoint{1.613441in}{4.212097in}}%
\pgfpathcurveto{\pgfqpoint{1.604232in}{4.221306in}}{\pgfqpoint{1.591741in}{4.226480in}}{\pgfqpoint{1.578719in}{4.226480in}}%
\pgfpathcurveto{\pgfqpoint{1.565696in}{4.226480in}}{\pgfqpoint{1.553205in}{4.221306in}}{\pgfqpoint{1.543996in}{4.212097in}}%
\pgfpathcurveto{\pgfqpoint{1.534788in}{4.202889in}}{\pgfqpoint{1.529614in}{4.190398in}}{\pgfqpoint{1.529614in}{4.177375in}}%
\pgfpathcurveto{\pgfqpoint{1.529614in}{4.164352in}}{\pgfqpoint{1.534788in}{4.151861in}}{\pgfqpoint{1.543996in}{4.142653in}}%
\pgfpathcurveto{\pgfqpoint{1.553205in}{4.133444in}}{\pgfqpoint{1.565696in}{4.128271in}}{\pgfqpoint{1.578719in}{4.128271in}}%
\pgfpathlineto{\pgfqpoint{1.578719in}{4.128271in}}%
\pgfpathclose%
\pgfusepath{stroke,fill}%
\end{pgfscope}%
\begin{pgfscope}%
\pgfpathrectangle{\pgfqpoint{0.786164in}{0.768110in}}{\pgfqpoint{8.851069in}{7.081890in}}%
\pgfusepath{clip}%
\pgfsetbuttcap%
\pgfsetroundjoin%
\definecolor{currentfill}{rgb}{0.197636,0.391528,0.554969}%
\pgfsetfillcolor{currentfill}%
\pgfsetfillopacity{0.700000}%
\pgfsetlinewidth{0.501875pt}%
\definecolor{currentstroke}{rgb}{1.000000,1.000000,1.000000}%
\pgfsetstrokecolor{currentstroke}%
\pgfsetstrokeopacity{0.700000}%
\pgfsetdash{}{0pt}%
\pgfpathmoveto{\pgfqpoint{1.615105in}{4.106928in}}%
\pgfpathcurveto{\pgfqpoint{1.628127in}{4.106928in}}{\pgfqpoint{1.640618in}{4.112102in}}{\pgfqpoint{1.649827in}{4.121311in}}%
\pgfpathcurveto{\pgfqpoint{1.659035in}{4.130519in}}{\pgfqpoint{1.664209in}{4.143010in}}{\pgfqpoint{1.664209in}{4.156033in}}%
\pgfpathcurveto{\pgfqpoint{1.664209in}{4.169056in}}{\pgfqpoint{1.659035in}{4.181547in}}{\pgfqpoint{1.649827in}{4.190755in}}%
\pgfpathcurveto{\pgfqpoint{1.640618in}{4.199964in}}{\pgfqpoint{1.628127in}{4.205138in}}{\pgfqpoint{1.615105in}{4.205138in}}%
\pgfpathcurveto{\pgfqpoint{1.602082in}{4.205138in}}{\pgfqpoint{1.589591in}{4.199964in}}{\pgfqpoint{1.580382in}{4.190755in}}%
\pgfpathcurveto{\pgfqpoint{1.571174in}{4.181547in}}{\pgfqpoint{1.566000in}{4.169056in}}{\pgfqpoint{1.566000in}{4.156033in}}%
\pgfpathcurveto{\pgfqpoint{1.566000in}{4.143010in}}{\pgfqpoint{1.571174in}{4.130519in}}{\pgfqpoint{1.580382in}{4.121311in}}%
\pgfpathcurveto{\pgfqpoint{1.589591in}{4.112102in}}{\pgfqpoint{1.602082in}{4.106928in}}{\pgfqpoint{1.615105in}{4.106928in}}%
\pgfpathlineto{\pgfqpoint{1.615105in}{4.106928in}}%
\pgfpathclose%
\pgfusepath{stroke,fill}%
\end{pgfscope}%
\begin{pgfscope}%
\pgfpathrectangle{\pgfqpoint{0.786164in}{0.768110in}}{\pgfqpoint{8.851069in}{7.081890in}}%
\pgfusepath{clip}%
\pgfsetbuttcap%
\pgfsetroundjoin%
\definecolor{currentfill}{rgb}{0.204903,0.375746,0.553533}%
\pgfsetfillcolor{currentfill}%
\pgfsetfillopacity{0.700000}%
\pgfsetlinewidth{0.501875pt}%
\definecolor{currentstroke}{rgb}{1.000000,1.000000,1.000000}%
\pgfsetstrokecolor{currentstroke}%
\pgfsetstrokeopacity{0.700000}%
\pgfsetdash{}{0pt}%
\pgfpathmoveto{\pgfqpoint{1.696912in}{4.000217in}}%
\pgfpathcurveto{\pgfqpoint{1.709935in}{4.000217in}}{\pgfqpoint{1.722426in}{4.005391in}}{\pgfqpoint{1.731634in}{4.014600in}}%
\pgfpathcurveto{\pgfqpoint{1.740843in}{4.023808in}}{\pgfqpoint{1.746017in}{4.036299in}}{\pgfqpoint{1.746017in}{4.049322in}}%
\pgfpathcurveto{\pgfqpoint{1.746017in}{4.062345in}}{\pgfqpoint{1.740843in}{4.074836in}}{\pgfqpoint{1.731634in}{4.084044in}}%
\pgfpathcurveto{\pgfqpoint{1.722426in}{4.093253in}}{\pgfqpoint{1.709935in}{4.098427in}}{\pgfqpoint{1.696912in}{4.098427in}}%
\pgfpathcurveto{\pgfqpoint{1.683889in}{4.098427in}}{\pgfqpoint{1.671398in}{4.093253in}}{\pgfqpoint{1.662190in}{4.084044in}}%
\pgfpathcurveto{\pgfqpoint{1.652981in}{4.074836in}}{\pgfqpoint{1.647807in}{4.062345in}}{\pgfqpoint{1.647807in}{4.049322in}}%
\pgfpathcurveto{\pgfqpoint{1.647807in}{4.036299in}}{\pgfqpoint{1.652981in}{4.023808in}}{\pgfqpoint{1.662190in}{4.014600in}}%
\pgfpathcurveto{\pgfqpoint{1.671398in}{4.005391in}}{\pgfqpoint{1.683889in}{4.000217in}}{\pgfqpoint{1.696912in}{4.000217in}}%
\pgfpathlineto{\pgfqpoint{1.696912in}{4.000217in}}%
\pgfpathclose%
\pgfusepath{stroke,fill}%
\end{pgfscope}%
\begin{pgfscope}%
\pgfpathrectangle{\pgfqpoint{0.786164in}{0.768110in}}{\pgfqpoint{8.851069in}{7.081890in}}%
\pgfusepath{clip}%
\pgfsetbuttcap%
\pgfsetroundjoin%
\definecolor{currentfill}{rgb}{0.194100,0.399323,0.555565}%
\pgfsetfillcolor{currentfill}%
\pgfsetfillopacity{0.700000}%
\pgfsetlinewidth{0.501875pt}%
\definecolor{currentstroke}{rgb}{1.000000,1.000000,1.000000}%
\pgfsetstrokecolor{currentstroke}%
\pgfsetstrokeopacity{0.700000}%
\pgfsetdash{}{0pt}%
\pgfpathmoveto{\pgfqpoint{1.654299in}{4.149613in}}%
\pgfpathcurveto{\pgfqpoint{1.667322in}{4.149613in}}{\pgfqpoint{1.679813in}{4.154787in}}{\pgfqpoint{1.689021in}{4.163995in}}%
\pgfpathcurveto{\pgfqpoint{1.698229in}{4.173204in}}{\pgfqpoint{1.703403in}{4.185695in}}{\pgfqpoint{1.703403in}{4.198717in}}%
\pgfpathcurveto{\pgfqpoint{1.703403in}{4.211740in}}{\pgfqpoint{1.698229in}{4.224231in}}{\pgfqpoint{1.689021in}{4.233440in}}%
\pgfpathcurveto{\pgfqpoint{1.679813in}{4.242648in}}{\pgfqpoint{1.667322in}{4.247822in}}{\pgfqpoint{1.654299in}{4.247822in}}%
\pgfpathcurveto{\pgfqpoint{1.641276in}{4.247822in}}{\pgfqpoint{1.628785in}{4.242648in}}{\pgfqpoint{1.619577in}{4.233440in}}%
\pgfpathcurveto{\pgfqpoint{1.610368in}{4.224231in}}{\pgfqpoint{1.605194in}{4.211740in}}{\pgfqpoint{1.605194in}{4.198717in}}%
\pgfpathcurveto{\pgfqpoint{1.605194in}{4.185695in}}{\pgfqpoint{1.610368in}{4.173204in}}{\pgfqpoint{1.619577in}{4.163995in}}%
\pgfpathcurveto{\pgfqpoint{1.628785in}{4.154787in}}{\pgfqpoint{1.641276in}{4.149613in}}{\pgfqpoint{1.654299in}{4.149613in}}%
\pgfpathlineto{\pgfqpoint{1.654299in}{4.149613in}}%
\pgfpathclose%
\pgfusepath{stroke,fill}%
\end{pgfscope}%
\begin{pgfscope}%
\pgfpathrectangle{\pgfqpoint{0.786164in}{0.768110in}}{\pgfqpoint{8.851069in}{7.081890in}}%
\pgfusepath{clip}%
\pgfsetbuttcap%
\pgfsetroundjoin%
\definecolor{currentfill}{rgb}{0.210503,0.363727,0.552206}%
\pgfsetfillcolor{currentfill}%
\pgfsetfillopacity{0.700000}%
\pgfsetlinewidth{0.501875pt}%
\definecolor{currentstroke}{rgb}{1.000000,1.000000,1.000000}%
\pgfsetstrokecolor{currentstroke}%
\pgfsetstrokeopacity{0.700000}%
\pgfsetdash{}{0pt}%
\pgfpathmoveto{\pgfqpoint{1.728414in}{3.957533in}}%
\pgfpathcurveto{\pgfqpoint{1.741437in}{3.957533in}}{\pgfqpoint{1.753928in}{3.962707in}}{\pgfqpoint{1.763136in}{3.971916in}}%
\pgfpathcurveto{\pgfqpoint{1.772344in}{3.981124in}}{\pgfqpoint{1.777518in}{3.993615in}}{\pgfqpoint{1.777518in}{4.006638in}}%
\pgfpathcurveto{\pgfqpoint{1.777518in}{4.019660in}}{\pgfqpoint{1.772344in}{4.032152in}}{\pgfqpoint{1.763136in}{4.041360in}}%
\pgfpathcurveto{\pgfqpoint{1.753928in}{4.050568in}}{\pgfqpoint{1.741437in}{4.055742in}}{\pgfqpoint{1.728414in}{4.055742in}}%
\pgfpathcurveto{\pgfqpoint{1.715391in}{4.055742in}}{\pgfqpoint{1.702900in}{4.050568in}}{\pgfqpoint{1.693692in}{4.041360in}}%
\pgfpathcurveto{\pgfqpoint{1.684483in}{4.032152in}}{\pgfqpoint{1.679309in}{4.019660in}}{\pgfqpoint{1.679309in}{4.006638in}}%
\pgfpathcurveto{\pgfqpoint{1.679309in}{3.993615in}}{\pgfqpoint{1.684483in}{3.981124in}}{\pgfqpoint{1.693692in}{3.971916in}}%
\pgfpathcurveto{\pgfqpoint{1.702900in}{3.962707in}}{\pgfqpoint{1.715391in}{3.957533in}}{\pgfqpoint{1.728414in}{3.957533in}}%
\pgfpathlineto{\pgfqpoint{1.728414in}{3.957533in}}%
\pgfpathclose%
\pgfusepath{stroke,fill}%
\end{pgfscope}%
\begin{pgfscope}%
\pgfpathrectangle{\pgfqpoint{0.786164in}{0.768110in}}{\pgfqpoint{8.851069in}{7.081890in}}%
\pgfusepath{clip}%
\pgfsetbuttcap%
\pgfsetroundjoin%
\definecolor{currentfill}{rgb}{0.212395,0.359683,0.551710}%
\pgfsetfillcolor{currentfill}%
\pgfsetfillopacity{0.700000}%
\pgfsetlinewidth{0.501875pt}%
\definecolor{currentstroke}{rgb}{1.000000,1.000000,1.000000}%
\pgfsetstrokecolor{currentstroke}%
\pgfsetstrokeopacity{0.700000}%
\pgfsetdash{}{0pt}%
\pgfpathmoveto{\pgfqpoint{1.744897in}{3.872164in}}%
\pgfpathcurveto{\pgfqpoint{1.757920in}{3.872164in}}{\pgfqpoint{1.770411in}{3.877338in}}{\pgfqpoint{1.779620in}{3.886547in}}%
\pgfpathcurveto{\pgfqpoint{1.788828in}{3.895755in}}{\pgfqpoint{1.794002in}{3.908246in}}{\pgfqpoint{1.794002in}{3.921269in}}%
\pgfpathcurveto{\pgfqpoint{1.794002in}{3.934292in}}{\pgfqpoint{1.788828in}{3.946783in}}{\pgfqpoint{1.779620in}{3.955991in}}%
\pgfpathcurveto{\pgfqpoint{1.770411in}{3.965200in}}{\pgfqpoint{1.757920in}{3.970374in}}{\pgfqpoint{1.744897in}{3.970374in}}%
\pgfpathcurveto{\pgfqpoint{1.731875in}{3.970374in}}{\pgfqpoint{1.719384in}{3.965200in}}{\pgfqpoint{1.710175in}{3.955991in}}%
\pgfpathcurveto{\pgfqpoint{1.700967in}{3.946783in}}{\pgfqpoint{1.695793in}{3.934292in}}{\pgfqpoint{1.695793in}{3.921269in}}%
\pgfpathcurveto{\pgfqpoint{1.695793in}{3.908246in}}{\pgfqpoint{1.700967in}{3.895755in}}{\pgfqpoint{1.710175in}{3.886547in}}%
\pgfpathcurveto{\pgfqpoint{1.719384in}{3.877338in}}{\pgfqpoint{1.731875in}{3.872164in}}{\pgfqpoint{1.744897in}{3.872164in}}%
\pgfpathlineto{\pgfqpoint{1.744897in}{3.872164in}}%
\pgfpathclose%
\pgfusepath{stroke,fill}%
\end{pgfscope}%
\begin{pgfscope}%
\pgfpathrectangle{\pgfqpoint{0.786164in}{0.768110in}}{\pgfqpoint{8.851069in}{7.081890in}}%
\pgfusepath{clip}%
\pgfsetbuttcap%
\pgfsetroundjoin%
\definecolor{currentfill}{rgb}{0.214298,0.355619,0.551184}%
\pgfsetfillcolor{currentfill}%
\pgfsetfillopacity{0.700000}%
\pgfsetlinewidth{0.501875pt}%
\definecolor{currentstroke}{rgb}{1.000000,1.000000,1.000000}%
\pgfsetstrokecolor{currentstroke}%
\pgfsetstrokeopacity{0.700000}%
\pgfsetdash{}{0pt}%
\pgfpathmoveto{\pgfqpoint{1.748805in}{3.850822in}}%
\pgfpathcurveto{\pgfqpoint{1.761827in}{3.850822in}}{\pgfqpoint{1.774318in}{3.855996in}}{\pgfqpoint{1.783527in}{3.865205in}}%
\pgfpathcurveto{\pgfqpoint{1.792735in}{3.874413in}}{\pgfqpoint{1.797909in}{3.886904in}}{\pgfqpoint{1.797909in}{3.899927in}}%
\pgfpathcurveto{\pgfqpoint{1.797909in}{3.912950in}}{\pgfqpoint{1.792735in}{3.925441in}}{\pgfqpoint{1.783527in}{3.934649in}}%
\pgfpathcurveto{\pgfqpoint{1.774318in}{3.943858in}}{\pgfqpoint{1.761827in}{3.949032in}}{\pgfqpoint{1.748805in}{3.949032in}}%
\pgfpathcurveto{\pgfqpoint{1.735782in}{3.949032in}}{\pgfqpoint{1.723291in}{3.943858in}}{\pgfqpoint{1.714082in}{3.934649in}}%
\pgfpathcurveto{\pgfqpoint{1.704874in}{3.925441in}}{\pgfqpoint{1.699700in}{3.912950in}}{\pgfqpoint{1.699700in}{3.899927in}}%
\pgfpathcurveto{\pgfqpoint{1.699700in}{3.886904in}}{\pgfqpoint{1.704874in}{3.874413in}}{\pgfqpoint{1.714082in}{3.865205in}}%
\pgfpathcurveto{\pgfqpoint{1.723291in}{3.855996in}}{\pgfqpoint{1.735782in}{3.850822in}}{\pgfqpoint{1.748805in}{3.850822in}}%
\pgfpathlineto{\pgfqpoint{1.748805in}{3.850822in}}%
\pgfpathclose%
\pgfusepath{stroke,fill}%
\end{pgfscope}%
\begin{pgfscope}%
\pgfpathrectangle{\pgfqpoint{0.786164in}{0.768110in}}{\pgfqpoint{8.851069in}{7.081890in}}%
\pgfusepath{clip}%
\pgfsetbuttcap%
\pgfsetroundjoin%
\definecolor{currentfill}{rgb}{0.227802,0.326594,0.546532}%
\pgfsetfillcolor{currentfill}%
\pgfsetfillopacity{0.700000}%
\pgfsetlinewidth{0.501875pt}%
\definecolor{currentstroke}{rgb}{1.000000,1.000000,1.000000}%
\pgfsetstrokecolor{currentstroke}%
\pgfsetstrokeopacity{0.700000}%
\pgfsetdash{}{0pt}%
\pgfpathmoveto{\pgfqpoint{1.837205in}{3.765454in}}%
\pgfpathcurveto{\pgfqpoint{1.850228in}{3.765454in}}{\pgfqpoint{1.862719in}{3.770628in}}{\pgfqpoint{1.871928in}{3.779836in}}%
\pgfpathcurveto{\pgfqpoint{1.881136in}{3.789044in}}{\pgfqpoint{1.886310in}{3.801536in}}{\pgfqpoint{1.886310in}{3.814558in}}%
\pgfpathcurveto{\pgfqpoint{1.886310in}{3.827581in}}{\pgfqpoint{1.881136in}{3.840072in}}{\pgfqpoint{1.871928in}{3.849280in}}%
\pgfpathcurveto{\pgfqpoint{1.862719in}{3.858489in}}{\pgfqpoint{1.850228in}{3.863663in}}{\pgfqpoint{1.837205in}{3.863663in}}%
\pgfpathcurveto{\pgfqpoint{1.824183in}{3.863663in}}{\pgfqpoint{1.811692in}{3.858489in}}{\pgfqpoint{1.802483in}{3.849280in}}%
\pgfpathcurveto{\pgfqpoint{1.793275in}{3.840072in}}{\pgfqpoint{1.788101in}{3.827581in}}{\pgfqpoint{1.788101in}{3.814558in}}%
\pgfpathcurveto{\pgfqpoint{1.788101in}{3.801536in}}{\pgfqpoint{1.793275in}{3.789044in}}{\pgfqpoint{1.802483in}{3.779836in}}%
\pgfpathcurveto{\pgfqpoint{1.811692in}{3.770628in}}{\pgfqpoint{1.824183in}{3.765454in}}{\pgfqpoint{1.837205in}{3.765454in}}%
\pgfpathlineto{\pgfqpoint{1.837205in}{3.765454in}}%
\pgfpathclose%
\pgfusepath{stroke,fill}%
\end{pgfscope}%
\begin{pgfscope}%
\pgfpathrectangle{\pgfqpoint{0.786164in}{0.768110in}}{\pgfqpoint{8.851069in}{7.081890in}}%
\pgfusepath{clip}%
\pgfsetbuttcap%
\pgfsetroundjoin%
\definecolor{currentfill}{rgb}{0.239346,0.300855,0.540844}%
\pgfsetfillcolor{currentfill}%
\pgfsetfillopacity{0.700000}%
\pgfsetlinewidth{0.501875pt}%
\definecolor{currentstroke}{rgb}{1.000000,1.000000,1.000000}%
\pgfsetstrokecolor{currentstroke}%
\pgfsetstrokeopacity{0.700000}%
\pgfsetdash{}{0pt}%
\pgfpathmoveto{\pgfqpoint{1.873713in}{3.850822in}}%
\pgfpathcurveto{\pgfqpoint{1.886736in}{3.850822in}}{\pgfqpoint{1.899227in}{3.855996in}}{\pgfqpoint{1.908436in}{3.865205in}}%
\pgfpathcurveto{\pgfqpoint{1.917644in}{3.874413in}}{\pgfqpoint{1.922818in}{3.886904in}}{\pgfqpoint{1.922818in}{3.899927in}}%
\pgfpathcurveto{\pgfqpoint{1.922818in}{3.912950in}}{\pgfqpoint{1.917644in}{3.925441in}}{\pgfqpoint{1.908436in}{3.934649in}}%
\pgfpathcurveto{\pgfqpoint{1.899227in}{3.943858in}}{\pgfqpoint{1.886736in}{3.949032in}}{\pgfqpoint{1.873713in}{3.949032in}}%
\pgfpathcurveto{\pgfqpoint{1.860691in}{3.949032in}}{\pgfqpoint{1.848200in}{3.943858in}}{\pgfqpoint{1.838991in}{3.934649in}}%
\pgfpathcurveto{\pgfqpoint{1.829783in}{3.925441in}}{\pgfqpoint{1.824609in}{3.912950in}}{\pgfqpoint{1.824609in}{3.899927in}}%
\pgfpathcurveto{\pgfqpoint{1.824609in}{3.886904in}}{\pgfqpoint{1.829783in}{3.874413in}}{\pgfqpoint{1.838991in}{3.865205in}}%
\pgfpathcurveto{\pgfqpoint{1.848200in}{3.855996in}}{\pgfqpoint{1.860691in}{3.850822in}}{\pgfqpoint{1.873713in}{3.850822in}}%
\pgfpathlineto{\pgfqpoint{1.873713in}{3.850822in}}%
\pgfpathclose%
\pgfusepath{stroke,fill}%
\end{pgfscope}%
\begin{pgfscope}%
\pgfpathrectangle{\pgfqpoint{0.786164in}{0.768110in}}{\pgfqpoint{8.851069in}{7.081890in}}%
\pgfusepath{clip}%
\pgfsetbuttcap%
\pgfsetroundjoin%
\definecolor{currentfill}{rgb}{0.227802,0.326594,0.546532}%
\pgfsetfillcolor{currentfill}%
\pgfsetfillopacity{0.700000}%
\pgfsetlinewidth{0.501875pt}%
\definecolor{currentstroke}{rgb}{1.000000,1.000000,1.000000}%
\pgfsetstrokecolor{currentstroke}%
\pgfsetstrokeopacity{0.700000}%
\pgfsetdash{}{0pt}%
\pgfpathmoveto{\pgfqpoint{1.889831in}{3.850822in}}%
\pgfpathcurveto{\pgfqpoint{1.902853in}{3.850822in}}{\pgfqpoint{1.915345in}{3.855996in}}{\pgfqpoint{1.924553in}{3.865205in}}%
\pgfpathcurveto{\pgfqpoint{1.933761in}{3.874413in}}{\pgfqpoint{1.938935in}{3.886904in}}{\pgfqpoint{1.938935in}{3.899927in}}%
\pgfpathcurveto{\pgfqpoint{1.938935in}{3.912950in}}{\pgfqpoint{1.933761in}{3.925441in}}{\pgfqpoint{1.924553in}{3.934649in}}%
\pgfpathcurveto{\pgfqpoint{1.915345in}{3.943858in}}{\pgfqpoint{1.902853in}{3.949032in}}{\pgfqpoint{1.889831in}{3.949032in}}%
\pgfpathcurveto{\pgfqpoint{1.876808in}{3.949032in}}{\pgfqpoint{1.864317in}{3.943858in}}{\pgfqpoint{1.855109in}{3.934649in}}%
\pgfpathcurveto{\pgfqpoint{1.845900in}{3.925441in}}{\pgfqpoint{1.840726in}{3.912950in}}{\pgfqpoint{1.840726in}{3.899927in}}%
\pgfpathcurveto{\pgfqpoint{1.840726in}{3.886904in}}{\pgfqpoint{1.845900in}{3.874413in}}{\pgfqpoint{1.855109in}{3.865205in}}%
\pgfpathcurveto{\pgfqpoint{1.864317in}{3.855996in}}{\pgfqpoint{1.876808in}{3.850822in}}{\pgfqpoint{1.889831in}{3.850822in}}%
\pgfpathlineto{\pgfqpoint{1.889831in}{3.850822in}}%
\pgfpathclose%
\pgfusepath{stroke,fill}%
\end{pgfscope}%
\begin{pgfscope}%
\pgfpathrectangle{\pgfqpoint{0.786164in}{0.768110in}}{\pgfqpoint{8.851069in}{7.081890in}}%
\pgfusepath{clip}%
\pgfsetbuttcap%
\pgfsetroundjoin%
\definecolor{currentfill}{rgb}{0.233603,0.313828,0.543914}%
\pgfsetfillcolor{currentfill}%
\pgfsetfillopacity{0.700000}%
\pgfsetlinewidth{0.501875pt}%
\definecolor{currentstroke}{rgb}{1.000000,1.000000,1.000000}%
\pgfsetstrokecolor{currentstroke}%
\pgfsetstrokeopacity{0.700000}%
\pgfsetdash{}{0pt}%
\pgfpathmoveto{\pgfqpoint{1.970539in}{3.786796in}}%
\pgfpathcurveto{\pgfqpoint{1.983562in}{3.786796in}}{\pgfqpoint{1.996053in}{3.791970in}}{\pgfqpoint{2.005261in}{3.801178in}}%
\pgfpathcurveto{\pgfqpoint{2.014470in}{3.810387in}}{\pgfqpoint{2.019644in}{3.822878in}}{\pgfqpoint{2.019644in}{3.835900in}}%
\pgfpathcurveto{\pgfqpoint{2.019644in}{3.848923in}}{\pgfqpoint{2.014470in}{3.861414in}}{\pgfqpoint{2.005261in}{3.870623in}}%
\pgfpathcurveto{\pgfqpoint{1.996053in}{3.879831in}}{\pgfqpoint{1.983562in}{3.885005in}}{\pgfqpoint{1.970539in}{3.885005in}}%
\pgfpathcurveto{\pgfqpoint{1.957517in}{3.885005in}}{\pgfqpoint{1.945025in}{3.879831in}}{\pgfqpoint{1.935817in}{3.870623in}}%
\pgfpathcurveto{\pgfqpoint{1.926609in}{3.861414in}}{\pgfqpoint{1.921435in}{3.848923in}}{\pgfqpoint{1.921435in}{3.835900in}}%
\pgfpathcurveto{\pgfqpoint{1.921435in}{3.822878in}}{\pgfqpoint{1.926609in}{3.810387in}}{\pgfqpoint{1.935817in}{3.801178in}}%
\pgfpathcurveto{\pgfqpoint{1.945025in}{3.791970in}}{\pgfqpoint{1.957517in}{3.786796in}}{\pgfqpoint{1.970539in}{3.786796in}}%
\pgfpathlineto{\pgfqpoint{1.970539in}{3.786796in}}%
\pgfpathclose%
\pgfusepath{stroke,fill}%
\end{pgfscope}%
\begin{pgfscope}%
\pgfpathrectangle{\pgfqpoint{0.786164in}{0.768110in}}{\pgfqpoint{8.851069in}{7.081890in}}%
\pgfusepath{clip}%
\pgfsetbuttcap%
\pgfsetroundjoin%
\definecolor{currentfill}{rgb}{0.244972,0.287675,0.537260}%
\pgfsetfillcolor{currentfill}%
\pgfsetfillopacity{0.700000}%
\pgfsetlinewidth{0.501875pt}%
\definecolor{currentstroke}{rgb}{1.000000,1.000000,1.000000}%
\pgfsetstrokecolor{currentstroke}%
\pgfsetstrokeopacity{0.700000}%
\pgfsetdash{}{0pt}%
\pgfpathmoveto{\pgfqpoint{2.078842in}{3.701427in}}%
\pgfpathcurveto{\pgfqpoint{2.091865in}{3.701427in}}{\pgfqpoint{2.104356in}{3.706601in}}{\pgfqpoint{2.113565in}{3.715810in}}%
\pgfpathcurveto{\pgfqpoint{2.122773in}{3.725018in}}{\pgfqpoint{2.127947in}{3.737509in}}{\pgfqpoint{2.127947in}{3.750532in}}%
\pgfpathcurveto{\pgfqpoint{2.127947in}{3.763554in}}{\pgfqpoint{2.122773in}{3.776046in}}{\pgfqpoint{2.113565in}{3.785254in}}%
\pgfpathcurveto{\pgfqpoint{2.104356in}{3.794462in}}{\pgfqpoint{2.091865in}{3.799636in}}{\pgfqpoint{2.078842in}{3.799636in}}%
\pgfpathcurveto{\pgfqpoint{2.065820in}{3.799636in}}{\pgfqpoint{2.053329in}{3.794462in}}{\pgfqpoint{2.044120in}{3.785254in}}%
\pgfpathcurveto{\pgfqpoint{2.034912in}{3.776046in}}{\pgfqpoint{2.029738in}{3.763554in}}{\pgfqpoint{2.029738in}{3.750532in}}%
\pgfpathcurveto{\pgfqpoint{2.029738in}{3.737509in}}{\pgfqpoint{2.034912in}{3.725018in}}{\pgfqpoint{2.044120in}{3.715810in}}%
\pgfpathcurveto{\pgfqpoint{2.053329in}{3.706601in}}{\pgfqpoint{2.065820in}{3.701427in}}{\pgfqpoint{2.078842in}{3.701427in}}%
\pgfpathlineto{\pgfqpoint{2.078842in}{3.701427in}}%
\pgfpathclose%
\pgfusepath{stroke,fill}%
\end{pgfscope}%
\begin{pgfscope}%
\pgfpathrectangle{\pgfqpoint{0.786164in}{0.768110in}}{\pgfqpoint{8.851069in}{7.081890in}}%
\pgfusepath{clip}%
\pgfsetbuttcap%
\pgfsetroundjoin%
\definecolor{currentfill}{rgb}{0.227802,0.326594,0.546532}%
\pgfsetfillcolor{currentfill}%
\pgfsetfillopacity{0.700000}%
\pgfsetlinewidth{0.501875pt}%
\definecolor{currentstroke}{rgb}{1.000000,1.000000,1.000000}%
\pgfsetstrokecolor{currentstroke}%
\pgfsetstrokeopacity{0.700000}%
\pgfsetdash{}{0pt}%
\pgfpathmoveto{\pgfqpoint{2.261016in}{3.616058in}}%
\pgfpathcurveto{\pgfqpoint{2.274039in}{3.616058in}}{\pgfqpoint{2.286530in}{3.621232in}}{\pgfqpoint{2.295739in}{3.630441in}}%
\pgfpathcurveto{\pgfqpoint{2.304947in}{3.639649in}}{\pgfqpoint{2.310121in}{3.652140in}}{\pgfqpoint{2.310121in}{3.665163in}}%
\pgfpathcurveto{\pgfqpoint{2.310121in}{3.678186in}}{\pgfqpoint{2.304947in}{3.690677in}}{\pgfqpoint{2.295739in}{3.699885in}}%
\pgfpathcurveto{\pgfqpoint{2.286530in}{3.709094in}}{\pgfqpoint{2.274039in}{3.714268in}}{\pgfqpoint{2.261016in}{3.714268in}}%
\pgfpathcurveto{\pgfqpoint{2.247994in}{3.714268in}}{\pgfqpoint{2.235503in}{3.709094in}}{\pgfqpoint{2.226294in}{3.699885in}}%
\pgfpathcurveto{\pgfqpoint{2.217086in}{3.690677in}}{\pgfqpoint{2.211912in}{3.678186in}}{\pgfqpoint{2.211912in}{3.665163in}}%
\pgfpathcurveto{\pgfqpoint{2.211912in}{3.652140in}}{\pgfqpoint{2.217086in}{3.639649in}}{\pgfqpoint{2.226294in}{3.630441in}}%
\pgfpathcurveto{\pgfqpoint{2.235503in}{3.621232in}}{\pgfqpoint{2.247994in}{3.616058in}}{\pgfqpoint{2.261016in}{3.616058in}}%
\pgfpathlineto{\pgfqpoint{2.261016in}{3.616058in}}%
\pgfpathclose%
\pgfusepath{stroke,fill}%
\end{pgfscope}%
\begin{pgfscope}%
\pgfpathrectangle{\pgfqpoint{0.786164in}{0.768110in}}{\pgfqpoint{8.851069in}{7.081890in}}%
\pgfusepath{clip}%
\pgfsetbuttcap%
\pgfsetroundjoin%
\definecolor{currentfill}{rgb}{0.246811,0.283237,0.535941}%
\pgfsetfillcolor{currentfill}%
\pgfsetfillopacity{0.700000}%
\pgfsetlinewidth{0.501875pt}%
\definecolor{currentstroke}{rgb}{1.000000,1.000000,1.000000}%
\pgfsetstrokecolor{currentstroke}%
\pgfsetstrokeopacity{0.700000}%
\pgfsetdash{}{0pt}%
\pgfpathmoveto{\pgfqpoint{2.885317in}{3.338610in}}%
\pgfpathcurveto{\pgfqpoint{2.898339in}{3.338610in}}{\pgfqpoint{2.910830in}{3.343784in}}{\pgfqpoint{2.920039in}{3.352993in}}%
\pgfpathcurveto{\pgfqpoint{2.929247in}{3.362201in}}{\pgfqpoint{2.934421in}{3.374692in}}{\pgfqpoint{2.934421in}{3.387715in}}%
\pgfpathcurveto{\pgfqpoint{2.934421in}{3.400738in}}{\pgfqpoint{2.929247in}{3.413229in}}{\pgfqpoint{2.920039in}{3.422437in}}%
\pgfpathcurveto{\pgfqpoint{2.910830in}{3.431645in}}{\pgfqpoint{2.898339in}{3.436819in}}{\pgfqpoint{2.885317in}{3.436819in}}%
\pgfpathcurveto{\pgfqpoint{2.872294in}{3.436819in}}{\pgfqpoint{2.859803in}{3.431645in}}{\pgfqpoint{2.850594in}{3.422437in}}%
\pgfpathcurveto{\pgfqpoint{2.841386in}{3.413229in}}{\pgfqpoint{2.836212in}{3.400738in}}{\pgfqpoint{2.836212in}{3.387715in}}%
\pgfpathcurveto{\pgfqpoint{2.836212in}{3.374692in}}{\pgfqpoint{2.841386in}{3.362201in}}{\pgfqpoint{2.850594in}{3.352993in}}%
\pgfpathcurveto{\pgfqpoint{2.859803in}{3.343784in}}{\pgfqpoint{2.872294in}{3.338610in}}{\pgfqpoint{2.885317in}{3.338610in}}%
\pgfpathlineto{\pgfqpoint{2.885317in}{3.338610in}}%
\pgfpathclose%
\pgfusepath{stroke,fill}%
\end{pgfscope}%
\begin{pgfscope}%
\pgfpathrectangle{\pgfqpoint{0.786164in}{0.768110in}}{\pgfqpoint{8.851069in}{7.081890in}}%
\pgfusepath{clip}%
\pgfsetbuttcap%
\pgfsetroundjoin%
\definecolor{currentfill}{rgb}{0.241237,0.296485,0.539709}%
\pgfsetfillcolor{currentfill}%
\pgfsetfillopacity{0.700000}%
\pgfsetlinewidth{0.501875pt}%
\definecolor{currentstroke}{rgb}{1.000000,1.000000,1.000000}%
\pgfsetstrokecolor{currentstroke}%
\pgfsetstrokeopacity{0.700000}%
\pgfsetdash{}{0pt}%
\pgfpathmoveto{\pgfqpoint{2.761873in}{3.359952in}}%
\pgfpathcurveto{\pgfqpoint{2.774896in}{3.359952in}}{\pgfqpoint{2.787387in}{3.365126in}}{\pgfqpoint{2.796595in}{3.374335in}}%
\pgfpathcurveto{\pgfqpoint{2.805804in}{3.383543in}}{\pgfqpoint{2.810978in}{3.396034in}}{\pgfqpoint{2.810978in}{3.409057in}}%
\pgfpathcurveto{\pgfqpoint{2.810978in}{3.422080in}}{\pgfqpoint{2.805804in}{3.434571in}}{\pgfqpoint{2.796595in}{3.443779in}}%
\pgfpathcurveto{\pgfqpoint{2.787387in}{3.452988in}}{\pgfqpoint{2.774896in}{3.458162in}}{\pgfqpoint{2.761873in}{3.458162in}}%
\pgfpathcurveto{\pgfqpoint{2.748850in}{3.458162in}}{\pgfqpoint{2.736359in}{3.452988in}}{\pgfqpoint{2.727151in}{3.443779in}}%
\pgfpathcurveto{\pgfqpoint{2.717942in}{3.434571in}}{\pgfqpoint{2.712768in}{3.422080in}}{\pgfqpoint{2.712768in}{3.409057in}}%
\pgfpathcurveto{\pgfqpoint{2.712768in}{3.396034in}}{\pgfqpoint{2.717942in}{3.383543in}}{\pgfqpoint{2.727151in}{3.374335in}}%
\pgfpathcurveto{\pgfqpoint{2.736359in}{3.365126in}}{\pgfqpoint{2.748850in}{3.359952in}}{\pgfqpoint{2.761873in}{3.359952in}}%
\pgfpathlineto{\pgfqpoint{2.761873in}{3.359952in}}%
\pgfpathclose%
\pgfusepath{stroke,fill}%
\end{pgfscope}%
\begin{pgfscope}%
\pgfpathrectangle{\pgfqpoint{0.786164in}{0.768110in}}{\pgfqpoint{8.851069in}{7.081890in}}%
\pgfusepath{clip}%
\pgfsetbuttcap%
\pgfsetroundjoin%
\definecolor{currentfill}{rgb}{0.255645,0.260703,0.528312}%
\pgfsetfillcolor{currentfill}%
\pgfsetfillopacity{0.700000}%
\pgfsetlinewidth{0.501875pt}%
\definecolor{currentstroke}{rgb}{1.000000,1.000000,1.000000}%
\pgfsetstrokecolor{currentstroke}%
\pgfsetstrokeopacity{0.700000}%
\pgfsetdash{}{0pt}%
\pgfpathmoveto{\pgfqpoint{3.004120in}{3.210557in}}%
\pgfpathcurveto{\pgfqpoint{3.017143in}{3.210557in}}{\pgfqpoint{3.029634in}{3.215731in}}{\pgfqpoint{3.038843in}{3.224940in}}%
\pgfpathcurveto{\pgfqpoint{3.048051in}{3.234148in}}{\pgfqpoint{3.053225in}{3.246639in}}{\pgfqpoint{3.053225in}{3.259662in}}%
\pgfpathcurveto{\pgfqpoint{3.053225in}{3.272685in}}{\pgfqpoint{3.048051in}{3.285176in}}{\pgfqpoint{3.038843in}{3.294384in}}%
\pgfpathcurveto{\pgfqpoint{3.029634in}{3.303592in}}{\pgfqpoint{3.017143in}{3.308766in}}{\pgfqpoint{3.004120in}{3.308766in}}%
\pgfpathcurveto{\pgfqpoint{2.991098in}{3.308766in}}{\pgfqpoint{2.978607in}{3.303592in}}{\pgfqpoint{2.969398in}{3.294384in}}%
\pgfpathcurveto{\pgfqpoint{2.960190in}{3.285176in}}{\pgfqpoint{2.955016in}{3.272685in}}{\pgfqpoint{2.955016in}{3.259662in}}%
\pgfpathcurveto{\pgfqpoint{2.955016in}{3.246639in}}{\pgfqpoint{2.960190in}{3.234148in}}{\pgfqpoint{2.969398in}{3.224940in}}%
\pgfpathcurveto{\pgfqpoint{2.978607in}{3.215731in}}{\pgfqpoint{2.991098in}{3.210557in}}{\pgfqpoint{3.004120in}{3.210557in}}%
\pgfpathlineto{\pgfqpoint{3.004120in}{3.210557in}}%
\pgfpathclose%
\pgfusepath{stroke,fill}%
\end{pgfscope}%
\begin{pgfscope}%
\pgfpathrectangle{\pgfqpoint{0.786164in}{0.768110in}}{\pgfqpoint{8.851069in}{7.081890in}}%
\pgfusepath{clip}%
\pgfsetbuttcap%
\pgfsetroundjoin%
\definecolor{currentfill}{rgb}{0.282884,0.135920,0.453427}%
\pgfsetfillcolor{currentfill}%
\pgfsetfillopacity{0.700000}%
\pgfsetlinewidth{0.501875pt}%
\definecolor{currentstroke}{rgb}{1.000000,1.000000,1.000000}%
\pgfsetstrokecolor{currentstroke}%
\pgfsetstrokeopacity{0.700000}%
\pgfsetdash{}{0pt}%
\pgfpathmoveto{\pgfqpoint{2.016327in}{3.146531in}}%
\pgfpathcurveto{\pgfqpoint{2.029350in}{3.146531in}}{\pgfqpoint{2.041841in}{3.151705in}}{\pgfqpoint{2.051049in}{3.160913in}}%
\pgfpathcurveto{\pgfqpoint{2.060258in}{3.170122in}}{\pgfqpoint{2.065432in}{3.182613in}}{\pgfqpoint{2.065432in}{3.195635in}}%
\pgfpathcurveto{\pgfqpoint{2.065432in}{3.208658in}}{\pgfqpoint{2.060258in}{3.221149in}}{\pgfqpoint{2.051049in}{3.230358in}}%
\pgfpathcurveto{\pgfqpoint{2.041841in}{3.239566in}}{\pgfqpoint{2.029350in}{3.244740in}}{\pgfqpoint{2.016327in}{3.244740in}}%
\pgfpathcurveto{\pgfqpoint{2.003304in}{3.244740in}}{\pgfqpoint{1.990813in}{3.239566in}}{\pgfqpoint{1.981605in}{3.230358in}}%
\pgfpathcurveto{\pgfqpoint{1.972396in}{3.221149in}}{\pgfqpoint{1.967222in}{3.208658in}}{\pgfqpoint{1.967222in}{3.195635in}}%
\pgfpathcurveto{\pgfqpoint{1.967222in}{3.182613in}}{\pgfqpoint{1.972396in}{3.170122in}}{\pgfqpoint{1.981605in}{3.160913in}}%
\pgfpathcurveto{\pgfqpoint{1.990813in}{3.151705in}}{\pgfqpoint{2.003304in}{3.146531in}}{\pgfqpoint{2.016327in}{3.146531in}}%
\pgfpathlineto{\pgfqpoint{2.016327in}{3.146531in}}%
\pgfpathclose%
\pgfusepath{stroke,fill}%
\end{pgfscope}%
\begin{pgfscope}%
\pgfpathrectangle{\pgfqpoint{0.786164in}{0.768110in}}{\pgfqpoint{8.851069in}{7.081890in}}%
\pgfusepath{clip}%
\pgfsetbuttcap%
\pgfsetroundjoin%
\definecolor{currentfill}{rgb}{0.282884,0.135920,0.453427}%
\pgfsetfillcolor{currentfill}%
\pgfsetfillopacity{0.700000}%
\pgfsetlinewidth{0.501875pt}%
\definecolor{currentstroke}{rgb}{1.000000,1.000000,1.000000}%
\pgfsetstrokecolor{currentstroke}%
\pgfsetstrokeopacity{0.700000}%
\pgfsetdash{}{0pt}%
\pgfpathmoveto{\pgfqpoint{2.014495in}{3.146531in}}%
\pgfpathcurveto{\pgfqpoint{2.027518in}{3.146531in}}{\pgfqpoint{2.040009in}{3.151705in}}{\pgfqpoint{2.049218in}{3.160913in}}%
\pgfpathcurveto{\pgfqpoint{2.058426in}{3.170122in}}{\pgfqpoint{2.063600in}{3.182613in}}{\pgfqpoint{2.063600in}{3.195635in}}%
\pgfpathcurveto{\pgfqpoint{2.063600in}{3.208658in}}{\pgfqpoint{2.058426in}{3.221149in}}{\pgfqpoint{2.049218in}{3.230358in}}%
\pgfpathcurveto{\pgfqpoint{2.040009in}{3.239566in}}{\pgfqpoint{2.027518in}{3.244740in}}{\pgfqpoint{2.014495in}{3.244740in}}%
\pgfpathcurveto{\pgfqpoint{2.001473in}{3.244740in}}{\pgfqpoint{1.988982in}{3.239566in}}{\pgfqpoint{1.979773in}{3.230358in}}%
\pgfpathcurveto{\pgfqpoint{1.970565in}{3.221149in}}{\pgfqpoint{1.965391in}{3.208658in}}{\pgfqpoint{1.965391in}{3.195635in}}%
\pgfpathcurveto{\pgfqpoint{1.965391in}{3.182613in}}{\pgfqpoint{1.970565in}{3.170122in}}{\pgfqpoint{1.979773in}{3.160913in}}%
\pgfpathcurveto{\pgfqpoint{1.988982in}{3.151705in}}{\pgfqpoint{2.001473in}{3.146531in}}{\pgfqpoint{2.014495in}{3.146531in}}%
\pgfpathlineto{\pgfqpoint{2.014495in}{3.146531in}}%
\pgfpathclose%
\pgfusepath{stroke,fill}%
\end{pgfscope}%
\begin{pgfscope}%
\pgfpathrectangle{\pgfqpoint{0.786164in}{0.768110in}}{\pgfqpoint{8.851069in}{7.081890in}}%
\pgfusepath{clip}%
\pgfsetbuttcap%
\pgfsetroundjoin%
\definecolor{currentfill}{rgb}{0.281412,0.155834,0.469201}%
\pgfsetfillcolor{currentfill}%
\pgfsetfillopacity{0.700000}%
\pgfsetlinewidth{0.501875pt}%
\definecolor{currentstroke}{rgb}{1.000000,1.000000,1.000000}%
\pgfsetstrokecolor{currentstroke}%
\pgfsetstrokeopacity{0.700000}%
\pgfsetdash{}{0pt}%
\pgfpathmoveto{\pgfqpoint{2.013519in}{3.274584in}}%
\pgfpathcurveto{\pgfqpoint{2.026541in}{3.274584in}}{\pgfqpoint{2.039032in}{3.279758in}}{\pgfqpoint{2.048241in}{3.288966in}}%
\pgfpathcurveto{\pgfqpoint{2.057449in}{3.298175in}}{\pgfqpoint{2.062623in}{3.310666in}}{\pgfqpoint{2.062623in}{3.323688in}}%
\pgfpathcurveto{\pgfqpoint{2.062623in}{3.336711in}}{\pgfqpoint{2.057449in}{3.349202in}}{\pgfqpoint{2.048241in}{3.358411in}}%
\pgfpathcurveto{\pgfqpoint{2.039032in}{3.367619in}}{\pgfqpoint{2.026541in}{3.372793in}}{\pgfqpoint{2.013519in}{3.372793in}}%
\pgfpathcurveto{\pgfqpoint{2.000496in}{3.372793in}}{\pgfqpoint{1.988005in}{3.367619in}}{\pgfqpoint{1.978796in}{3.358411in}}%
\pgfpathcurveto{\pgfqpoint{1.969588in}{3.349202in}}{\pgfqpoint{1.964414in}{3.336711in}}{\pgfqpoint{1.964414in}{3.323688in}}%
\pgfpathcurveto{\pgfqpoint{1.964414in}{3.310666in}}{\pgfqpoint{1.969588in}{3.298175in}}{\pgfqpoint{1.978796in}{3.288966in}}%
\pgfpathcurveto{\pgfqpoint{1.988005in}{3.279758in}}{\pgfqpoint{2.000496in}{3.274584in}}{\pgfqpoint{2.013519in}{3.274584in}}%
\pgfpathlineto{\pgfqpoint{2.013519in}{3.274584in}}%
\pgfpathclose%
\pgfusepath{stroke,fill}%
\end{pgfscope}%
\begin{pgfscope}%
\pgfpathrectangle{\pgfqpoint{0.786164in}{0.768110in}}{\pgfqpoint{8.851069in}{7.081890in}}%
\pgfusepath{clip}%
\pgfsetbuttcap%
\pgfsetroundjoin%
\definecolor{currentfill}{rgb}{0.281412,0.155834,0.469201}%
\pgfsetfillcolor{currentfill}%
\pgfsetfillopacity{0.700000}%
\pgfsetlinewidth{0.501875pt}%
\definecolor{currentstroke}{rgb}{1.000000,1.000000,1.000000}%
\pgfsetstrokecolor{currentstroke}%
\pgfsetstrokeopacity{0.700000}%
\pgfsetdash{}{0pt}%
\pgfpathmoveto{\pgfqpoint{2.018891in}{3.295926in}}%
\pgfpathcurveto{\pgfqpoint{2.031914in}{3.295926in}}{\pgfqpoint{2.044405in}{3.301100in}}{\pgfqpoint{2.053613in}{3.310308in}}%
\pgfpathcurveto{\pgfqpoint{2.062822in}{3.319517in}}{\pgfqpoint{2.067996in}{3.332008in}}{\pgfqpoint{2.067996in}{3.345030in}}%
\pgfpathcurveto{\pgfqpoint{2.067996in}{3.358053in}}{\pgfqpoint{2.062822in}{3.370544in}}{\pgfqpoint{2.053613in}{3.379753in}}%
\pgfpathcurveto{\pgfqpoint{2.044405in}{3.388961in}}{\pgfqpoint{2.031914in}{3.394135in}}{\pgfqpoint{2.018891in}{3.394135in}}%
\pgfpathcurveto{\pgfqpoint{2.005868in}{3.394135in}}{\pgfqpoint{1.993377in}{3.388961in}}{\pgfqpoint{1.984169in}{3.379753in}}%
\pgfpathcurveto{\pgfqpoint{1.974960in}{3.370544in}}{\pgfqpoint{1.969786in}{3.358053in}}{\pgfqpoint{1.969786in}{3.345030in}}%
\pgfpathcurveto{\pgfqpoint{1.969786in}{3.332008in}}{\pgfqpoint{1.974960in}{3.319517in}}{\pgfqpoint{1.984169in}{3.310308in}}%
\pgfpathcurveto{\pgfqpoint{1.993377in}{3.301100in}}{\pgfqpoint{2.005868in}{3.295926in}}{\pgfqpoint{2.018891in}{3.295926in}}%
\pgfpathlineto{\pgfqpoint{2.018891in}{3.295926in}}%
\pgfpathclose%
\pgfusepath{stroke,fill}%
\end{pgfscope}%
\begin{pgfscope}%
\pgfpathrectangle{\pgfqpoint{0.786164in}{0.768110in}}{\pgfqpoint{8.851069in}{7.081890in}}%
\pgfusepath{clip}%
\pgfsetbuttcap%
\pgfsetroundjoin%
\definecolor{currentfill}{rgb}{0.280868,0.160771,0.472899}%
\pgfsetfillcolor{currentfill}%
\pgfsetfillopacity{0.700000}%
\pgfsetlinewidth{0.501875pt}%
\definecolor{currentstroke}{rgb}{1.000000,1.000000,1.000000}%
\pgfsetstrokecolor{currentstroke}%
\pgfsetstrokeopacity{0.700000}%
\pgfsetdash{}{0pt}%
\pgfpathmoveto{\pgfqpoint{2.114496in}{3.274584in}}%
\pgfpathcurveto{\pgfqpoint{2.127518in}{3.274584in}}{\pgfqpoint{2.140010in}{3.279758in}}{\pgfqpoint{2.149218in}{3.288966in}}%
\pgfpathcurveto{\pgfqpoint{2.158426in}{3.298175in}}{\pgfqpoint{2.163600in}{3.310666in}}{\pgfqpoint{2.163600in}{3.323688in}}%
\pgfpathcurveto{\pgfqpoint{2.163600in}{3.336711in}}{\pgfqpoint{2.158426in}{3.349202in}}{\pgfqpoint{2.149218in}{3.358411in}}%
\pgfpathcurveto{\pgfqpoint{2.140010in}{3.367619in}}{\pgfqpoint{2.127518in}{3.372793in}}{\pgfqpoint{2.114496in}{3.372793in}}%
\pgfpathcurveto{\pgfqpoint{2.101473in}{3.372793in}}{\pgfqpoint{2.088982in}{3.367619in}}{\pgfqpoint{2.079774in}{3.358411in}}%
\pgfpathcurveto{\pgfqpoint{2.070565in}{3.349202in}}{\pgfqpoint{2.065391in}{3.336711in}}{\pgfqpoint{2.065391in}{3.323688in}}%
\pgfpathcurveto{\pgfqpoint{2.065391in}{3.310666in}}{\pgfqpoint{2.070565in}{3.298175in}}{\pgfqpoint{2.079774in}{3.288966in}}%
\pgfpathcurveto{\pgfqpoint{2.088982in}{3.279758in}}{\pgfqpoint{2.101473in}{3.274584in}}{\pgfqpoint{2.114496in}{3.274584in}}%
\pgfpathlineto{\pgfqpoint{2.114496in}{3.274584in}}%
\pgfpathclose%
\pgfusepath{stroke,fill}%
\end{pgfscope}%
\begin{pgfscope}%
\pgfpathrectangle{\pgfqpoint{0.786164in}{0.768110in}}{\pgfqpoint{8.851069in}{7.081890in}}%
\pgfusepath{clip}%
\pgfsetbuttcap%
\pgfsetroundjoin%
\definecolor{currentfill}{rgb}{0.282290,0.145912,0.461510}%
\pgfsetfillcolor{currentfill}%
\pgfsetfillopacity{0.700000}%
\pgfsetlinewidth{0.501875pt}%
\definecolor{currentstroke}{rgb}{1.000000,1.000000,1.000000}%
\pgfsetstrokecolor{currentstroke}%
\pgfsetstrokeopacity{0.700000}%
\pgfsetdash{}{0pt}%
\pgfpathmoveto{\pgfqpoint{2.235375in}{3.167873in}}%
\pgfpathcurveto{\pgfqpoint{2.248398in}{3.167873in}}{\pgfqpoint{2.260889in}{3.173047in}}{\pgfqpoint{2.270098in}{3.182255in}}%
\pgfpathcurveto{\pgfqpoint{2.279306in}{3.191464in}}{\pgfqpoint{2.284480in}{3.203955in}}{\pgfqpoint{2.284480in}{3.216977in}}%
\pgfpathcurveto{\pgfqpoint{2.284480in}{3.230000in}}{\pgfqpoint{2.279306in}{3.242491in}}{\pgfqpoint{2.270098in}{3.251700in}}%
\pgfpathcurveto{\pgfqpoint{2.260889in}{3.260908in}}{\pgfqpoint{2.248398in}{3.266082in}}{\pgfqpoint{2.235375in}{3.266082in}}%
\pgfpathcurveto{\pgfqpoint{2.222353in}{3.266082in}}{\pgfqpoint{2.209862in}{3.260908in}}{\pgfqpoint{2.200653in}{3.251700in}}%
\pgfpathcurveto{\pgfqpoint{2.191445in}{3.242491in}}{\pgfqpoint{2.186271in}{3.230000in}}{\pgfqpoint{2.186271in}{3.216977in}}%
\pgfpathcurveto{\pgfqpoint{2.186271in}{3.203955in}}{\pgfqpoint{2.191445in}{3.191464in}}{\pgfqpoint{2.200653in}{3.182255in}}%
\pgfpathcurveto{\pgfqpoint{2.209862in}{3.173047in}}{\pgfqpoint{2.222353in}{3.167873in}}{\pgfqpoint{2.235375in}{3.167873in}}%
\pgfpathlineto{\pgfqpoint{2.235375in}{3.167873in}}%
\pgfpathclose%
\pgfusepath{stroke,fill}%
\end{pgfscope}%
\begin{pgfscope}%
\pgfpathrectangle{\pgfqpoint{0.786164in}{0.768110in}}{\pgfqpoint{8.851069in}{7.081890in}}%
\pgfusepath{clip}%
\pgfsetbuttcap%
\pgfsetroundjoin%
\definecolor{currentfill}{rgb}{0.279574,0.170599,0.479997}%
\pgfsetfillcolor{currentfill}%
\pgfsetfillopacity{0.700000}%
\pgfsetlinewidth{0.501875pt}%
\definecolor{currentstroke}{rgb}{1.000000,1.000000,1.000000}%
\pgfsetstrokecolor{currentstroke}%
\pgfsetstrokeopacity{0.700000}%
\pgfsetdash{}{0pt}%
\pgfpathmoveto{\pgfqpoint{2.309246in}{3.274584in}}%
\pgfpathcurveto{\pgfqpoint{2.322269in}{3.274584in}}{\pgfqpoint{2.334760in}{3.279758in}}{\pgfqpoint{2.343968in}{3.288966in}}%
\pgfpathcurveto{\pgfqpoint{2.353177in}{3.298175in}}{\pgfqpoint{2.358351in}{3.310666in}}{\pgfqpoint{2.358351in}{3.323688in}}%
\pgfpathcurveto{\pgfqpoint{2.358351in}{3.336711in}}{\pgfqpoint{2.353177in}{3.349202in}}{\pgfqpoint{2.343968in}{3.358411in}}%
\pgfpathcurveto{\pgfqpoint{2.334760in}{3.367619in}}{\pgfqpoint{2.322269in}{3.372793in}}{\pgfqpoint{2.309246in}{3.372793in}}%
\pgfpathcurveto{\pgfqpoint{2.296223in}{3.372793in}}{\pgfqpoint{2.283732in}{3.367619in}}{\pgfqpoint{2.274524in}{3.358411in}}%
\pgfpathcurveto{\pgfqpoint{2.265315in}{3.349202in}}{\pgfqpoint{2.260142in}{3.336711in}}{\pgfqpoint{2.260142in}{3.323688in}}%
\pgfpathcurveto{\pgfqpoint{2.260142in}{3.310666in}}{\pgfqpoint{2.265315in}{3.298175in}}{\pgfqpoint{2.274524in}{3.288966in}}%
\pgfpathcurveto{\pgfqpoint{2.283732in}{3.279758in}}{\pgfqpoint{2.296223in}{3.274584in}}{\pgfqpoint{2.309246in}{3.274584in}}%
\pgfpathlineto{\pgfqpoint{2.309246in}{3.274584in}}%
\pgfpathclose%
\pgfusepath{stroke,fill}%
\end{pgfscope}%
\begin{pgfscope}%
\pgfpathrectangle{\pgfqpoint{0.786164in}{0.768110in}}{\pgfqpoint{8.851069in}{7.081890in}}%
\pgfusepath{clip}%
\pgfsetbuttcap%
\pgfsetroundjoin%
\definecolor{currentfill}{rgb}{0.279574,0.170599,0.479997}%
\pgfsetfillcolor{currentfill}%
\pgfsetfillopacity{0.700000}%
\pgfsetlinewidth{0.501875pt}%
\definecolor{currentstroke}{rgb}{1.000000,1.000000,1.000000}%
\pgfsetstrokecolor{currentstroke}%
\pgfsetstrokeopacity{0.700000}%
\pgfsetdash{}{0pt}%
\pgfpathmoveto{\pgfqpoint{2.438184in}{3.231899in}}%
\pgfpathcurveto{\pgfqpoint{2.451207in}{3.231899in}}{\pgfqpoint{2.463698in}{3.237073in}}{\pgfqpoint{2.472907in}{3.246282in}}%
\pgfpathcurveto{\pgfqpoint{2.482115in}{3.255490in}}{\pgfqpoint{2.487289in}{3.267981in}}{\pgfqpoint{2.487289in}{3.281004in}}%
\pgfpathcurveto{\pgfqpoint{2.487289in}{3.294027in}}{\pgfqpoint{2.482115in}{3.306518in}}{\pgfqpoint{2.472907in}{3.315726in}}%
\pgfpathcurveto{\pgfqpoint{2.463698in}{3.324935in}}{\pgfqpoint{2.451207in}{3.330109in}}{\pgfqpoint{2.438184in}{3.330109in}}%
\pgfpathcurveto{\pgfqpoint{2.425162in}{3.330109in}}{\pgfqpoint{2.412671in}{3.324935in}}{\pgfqpoint{2.403462in}{3.315726in}}%
\pgfpathcurveto{\pgfqpoint{2.394254in}{3.306518in}}{\pgfqpoint{2.389080in}{3.294027in}}{\pgfqpoint{2.389080in}{3.281004in}}%
\pgfpathcurveto{\pgfqpoint{2.389080in}{3.267981in}}{\pgfqpoint{2.394254in}{3.255490in}}{\pgfqpoint{2.403462in}{3.246282in}}%
\pgfpathcurveto{\pgfqpoint{2.412671in}{3.237073in}}{\pgfqpoint{2.425162in}{3.231899in}}{\pgfqpoint{2.438184in}{3.231899in}}%
\pgfpathlineto{\pgfqpoint{2.438184in}{3.231899in}}%
\pgfpathclose%
\pgfusepath{stroke,fill}%
\end{pgfscope}%
\begin{pgfscope}%
\pgfpathrectangle{\pgfqpoint{0.786164in}{0.768110in}}{\pgfqpoint{8.851069in}{7.081890in}}%
\pgfusepath{clip}%
\pgfsetbuttcap%
\pgfsetroundjoin%
\definecolor{currentfill}{rgb}{0.280868,0.160771,0.472899}%
\pgfsetfillcolor{currentfill}%
\pgfsetfillopacity{0.700000}%
\pgfsetlinewidth{0.501875pt}%
\definecolor{currentstroke}{rgb}{1.000000,1.000000,1.000000}%
\pgfsetstrokecolor{currentstroke}%
\pgfsetstrokeopacity{0.700000}%
\pgfsetdash{}{0pt}%
\pgfpathmoveto{\pgfqpoint{2.513642in}{3.189215in}}%
\pgfpathcurveto{\pgfqpoint{2.526665in}{3.189215in}}{\pgfqpoint{2.539156in}{3.194389in}}{\pgfqpoint{2.548365in}{3.203597in}}%
\pgfpathcurveto{\pgfqpoint{2.557573in}{3.212806in}}{\pgfqpoint{2.562747in}{3.225297in}}{\pgfqpoint{2.562747in}{3.238320in}}%
\pgfpathcurveto{\pgfqpoint{2.562747in}{3.251342in}}{\pgfqpoint{2.557573in}{3.263833in}}{\pgfqpoint{2.548365in}{3.273042in}}%
\pgfpathcurveto{\pgfqpoint{2.539156in}{3.282250in}}{\pgfqpoint{2.526665in}{3.287424in}}{\pgfqpoint{2.513642in}{3.287424in}}%
\pgfpathcurveto{\pgfqpoint{2.500620in}{3.287424in}}{\pgfqpoint{2.488129in}{3.282250in}}{\pgfqpoint{2.478920in}{3.273042in}}%
\pgfpathcurveto{\pgfqpoint{2.469712in}{3.263833in}}{\pgfqpoint{2.464538in}{3.251342in}}{\pgfqpoint{2.464538in}{3.238320in}}%
\pgfpathcurveto{\pgfqpoint{2.464538in}{3.225297in}}{\pgfqpoint{2.469712in}{3.212806in}}{\pgfqpoint{2.478920in}{3.203597in}}%
\pgfpathcurveto{\pgfqpoint{2.488129in}{3.194389in}}{\pgfqpoint{2.500620in}{3.189215in}}{\pgfqpoint{2.513642in}{3.189215in}}%
\pgfpathlineto{\pgfqpoint{2.513642in}{3.189215in}}%
\pgfpathclose%
\pgfusepath{stroke,fill}%
\end{pgfscope}%
\begin{pgfscope}%
\pgfpathrectangle{\pgfqpoint{0.786164in}{0.768110in}}{\pgfqpoint{8.851069in}{7.081890in}}%
\pgfusepath{clip}%
\pgfsetbuttcap%
\pgfsetroundjoin%
\definecolor{currentfill}{rgb}{0.280868,0.160771,0.472899}%
\pgfsetfillcolor{currentfill}%
\pgfsetfillopacity{0.700000}%
\pgfsetlinewidth{0.501875pt}%
\definecolor{currentstroke}{rgb}{1.000000,1.000000,1.000000}%
\pgfsetstrokecolor{currentstroke}%
\pgfsetstrokeopacity{0.700000}%
\pgfsetdash{}{0pt}%
\pgfpathmoveto{\pgfqpoint{2.574326in}{3.146531in}}%
\pgfpathcurveto{\pgfqpoint{2.587349in}{3.146531in}}{\pgfqpoint{2.599840in}{3.151705in}}{\pgfqpoint{2.609049in}{3.160913in}}%
\pgfpathcurveto{\pgfqpoint{2.618257in}{3.170122in}}{\pgfqpoint{2.623431in}{3.182613in}}{\pgfqpoint{2.623431in}{3.195635in}}%
\pgfpathcurveto{\pgfqpoint{2.623431in}{3.208658in}}{\pgfqpoint{2.618257in}{3.221149in}}{\pgfqpoint{2.609049in}{3.230358in}}%
\pgfpathcurveto{\pgfqpoint{2.599840in}{3.239566in}}{\pgfqpoint{2.587349in}{3.244740in}}{\pgfqpoint{2.574326in}{3.244740in}}%
\pgfpathcurveto{\pgfqpoint{2.561304in}{3.244740in}}{\pgfqpoint{2.548813in}{3.239566in}}{\pgfqpoint{2.539604in}{3.230358in}}%
\pgfpathcurveto{\pgfqpoint{2.530396in}{3.221149in}}{\pgfqpoint{2.525222in}{3.208658in}}{\pgfqpoint{2.525222in}{3.195635in}}%
\pgfpathcurveto{\pgfqpoint{2.525222in}{3.182613in}}{\pgfqpoint{2.530396in}{3.170122in}}{\pgfqpoint{2.539604in}{3.160913in}}%
\pgfpathcurveto{\pgfqpoint{2.548813in}{3.151705in}}{\pgfqpoint{2.561304in}{3.146531in}}{\pgfqpoint{2.574326in}{3.146531in}}%
\pgfpathlineto{\pgfqpoint{2.574326in}{3.146531in}}%
\pgfpathclose%
\pgfusepath{stroke,fill}%
\end{pgfscope}%
\begin{pgfscope}%
\pgfpathrectangle{\pgfqpoint{0.786164in}{0.768110in}}{\pgfqpoint{8.851069in}{7.081890in}}%
\pgfusepath{clip}%
\pgfsetbuttcap%
\pgfsetroundjoin%
\definecolor{currentfill}{rgb}{0.282290,0.145912,0.461510}%
\pgfsetfillcolor{currentfill}%
\pgfsetfillopacity{0.700000}%
\pgfsetlinewidth{0.501875pt}%
\definecolor{currentstroke}{rgb}{1.000000,1.000000,1.000000}%
\pgfsetstrokecolor{currentstroke}%
\pgfsetstrokeopacity{0.700000}%
\pgfsetdash{}{0pt}%
\pgfpathmoveto{\pgfqpoint{2.593008in}{3.125188in}}%
\pgfpathcurveto{\pgfqpoint{2.606031in}{3.125188in}}{\pgfqpoint{2.618522in}{3.130362in}}{\pgfqpoint{2.627730in}{3.139571in}}%
\pgfpathcurveto{\pgfqpoint{2.636938in}{3.148779in}}{\pgfqpoint{2.642112in}{3.161270in}}{\pgfqpoint{2.642112in}{3.174293in}}%
\pgfpathcurveto{\pgfqpoint{2.642112in}{3.187316in}}{\pgfqpoint{2.636938in}{3.199807in}}{\pgfqpoint{2.627730in}{3.209015in}}%
\pgfpathcurveto{\pgfqpoint{2.618522in}{3.218224in}}{\pgfqpoint{2.606031in}{3.223398in}}{\pgfqpoint{2.593008in}{3.223398in}}%
\pgfpathcurveto{\pgfqpoint{2.579985in}{3.223398in}}{\pgfqpoint{2.567494in}{3.218224in}}{\pgfqpoint{2.558286in}{3.209015in}}%
\pgfpathcurveto{\pgfqpoint{2.549077in}{3.199807in}}{\pgfqpoint{2.543903in}{3.187316in}}{\pgfqpoint{2.543903in}{3.174293in}}%
\pgfpathcurveto{\pgfqpoint{2.543903in}{3.161270in}}{\pgfqpoint{2.549077in}{3.148779in}}{\pgfqpoint{2.558286in}{3.139571in}}%
\pgfpathcurveto{\pgfqpoint{2.567494in}{3.130362in}}{\pgfqpoint{2.579985in}{3.125188in}}{\pgfqpoint{2.593008in}{3.125188in}}%
\pgfpathlineto{\pgfqpoint{2.593008in}{3.125188in}}%
\pgfpathclose%
\pgfusepath{stroke,fill}%
\end{pgfscope}%
\begin{pgfscope}%
\pgfpathrectangle{\pgfqpoint{0.786164in}{0.768110in}}{\pgfqpoint{8.851069in}{7.081890in}}%
\pgfusepath{clip}%
\pgfsetbuttcap%
\pgfsetroundjoin%
\definecolor{currentfill}{rgb}{0.281887,0.150881,0.465405}%
\pgfsetfillcolor{currentfill}%
\pgfsetfillopacity{0.700000}%
\pgfsetlinewidth{0.501875pt}%
\definecolor{currentstroke}{rgb}{1.000000,1.000000,1.000000}%
\pgfsetstrokecolor{currentstroke}%
\pgfsetstrokeopacity{0.700000}%
\pgfsetdash{}{0pt}%
\pgfpathmoveto{\pgfqpoint{2.626708in}{3.146531in}}%
\pgfpathcurveto{\pgfqpoint{2.639730in}{3.146531in}}{\pgfqpoint{2.652221in}{3.151705in}}{\pgfqpoint{2.661430in}{3.160913in}}%
\pgfpathcurveto{\pgfqpoint{2.670638in}{3.170122in}}{\pgfqpoint{2.675812in}{3.182613in}}{\pgfqpoint{2.675812in}{3.195635in}}%
\pgfpathcurveto{\pgfqpoint{2.675812in}{3.208658in}}{\pgfqpoint{2.670638in}{3.221149in}}{\pgfqpoint{2.661430in}{3.230358in}}%
\pgfpathcurveto{\pgfqpoint{2.652221in}{3.239566in}}{\pgfqpoint{2.639730in}{3.244740in}}{\pgfqpoint{2.626708in}{3.244740in}}%
\pgfpathcurveto{\pgfqpoint{2.613685in}{3.244740in}}{\pgfqpoint{2.601194in}{3.239566in}}{\pgfqpoint{2.591985in}{3.230358in}}%
\pgfpathcurveto{\pgfqpoint{2.582777in}{3.221149in}}{\pgfqpoint{2.577603in}{3.208658in}}{\pgfqpoint{2.577603in}{3.195635in}}%
\pgfpathcurveto{\pgfqpoint{2.577603in}{3.182613in}}{\pgfqpoint{2.582777in}{3.170122in}}{\pgfqpoint{2.591985in}{3.160913in}}%
\pgfpathcurveto{\pgfqpoint{2.601194in}{3.151705in}}{\pgfqpoint{2.613685in}{3.146531in}}{\pgfqpoint{2.626708in}{3.146531in}}%
\pgfpathlineto{\pgfqpoint{2.626708in}{3.146531in}}%
\pgfpathclose%
\pgfusepath{stroke,fill}%
\end{pgfscope}%
\begin{pgfscope}%
\pgfpathrectangle{\pgfqpoint{0.786164in}{0.768110in}}{\pgfqpoint{8.851069in}{7.081890in}}%
\pgfusepath{clip}%
\pgfsetbuttcap%
\pgfsetroundjoin%
\definecolor{currentfill}{rgb}{0.280255,0.165693,0.476498}%
\pgfsetfillcolor{currentfill}%
\pgfsetfillopacity{0.700000}%
\pgfsetlinewidth{0.501875pt}%
\definecolor{currentstroke}{rgb}{1.000000,1.000000,1.000000}%
\pgfsetstrokecolor{currentstroke}%
\pgfsetstrokeopacity{0.700000}%
\pgfsetdash{}{0pt}%
\pgfpathmoveto{\pgfqpoint{2.567489in}{3.189215in}}%
\pgfpathcurveto{\pgfqpoint{2.580512in}{3.189215in}}{\pgfqpoint{2.593003in}{3.194389in}}{\pgfqpoint{2.602211in}{3.203597in}}%
\pgfpathcurveto{\pgfqpoint{2.611419in}{3.212806in}}{\pgfqpoint{2.616593in}{3.225297in}}{\pgfqpoint{2.616593in}{3.238320in}}%
\pgfpathcurveto{\pgfqpoint{2.616593in}{3.251342in}}{\pgfqpoint{2.611419in}{3.263833in}}{\pgfqpoint{2.602211in}{3.273042in}}%
\pgfpathcurveto{\pgfqpoint{2.593003in}{3.282250in}}{\pgfqpoint{2.580512in}{3.287424in}}{\pgfqpoint{2.567489in}{3.287424in}}%
\pgfpathcurveto{\pgfqpoint{2.554466in}{3.287424in}}{\pgfqpoint{2.541975in}{3.282250in}}{\pgfqpoint{2.532767in}{3.273042in}}%
\pgfpathcurveto{\pgfqpoint{2.523558in}{3.263833in}}{\pgfqpoint{2.518384in}{3.251342in}}{\pgfqpoint{2.518384in}{3.238320in}}%
\pgfpathcurveto{\pgfqpoint{2.518384in}{3.225297in}}{\pgfqpoint{2.523558in}{3.212806in}}{\pgfqpoint{2.532767in}{3.203597in}}%
\pgfpathcurveto{\pgfqpoint{2.541975in}{3.194389in}}{\pgfqpoint{2.554466in}{3.189215in}}{\pgfqpoint{2.567489in}{3.189215in}}%
\pgfpathlineto{\pgfqpoint{2.567489in}{3.189215in}}%
\pgfpathclose%
\pgfusepath{stroke,fill}%
\end{pgfscope}%
\begin{pgfscope}%
\pgfpathrectangle{\pgfqpoint{0.786164in}{0.768110in}}{\pgfqpoint{8.851069in}{7.081890in}}%
\pgfusepath{clip}%
\pgfsetbuttcap%
\pgfsetroundjoin%
\definecolor{currentfill}{rgb}{0.278012,0.180367,0.486697}%
\pgfsetfillcolor{currentfill}%
\pgfsetfillopacity{0.700000}%
\pgfsetlinewidth{0.501875pt}%
\definecolor{currentstroke}{rgb}{1.000000,1.000000,1.000000}%
\pgfsetstrokecolor{currentstroke}%
\pgfsetstrokeopacity{0.700000}%
\pgfsetdash{}{0pt}%
\pgfpathmoveto{\pgfqpoint{2.526341in}{3.253242in}}%
\pgfpathcurveto{\pgfqpoint{2.539364in}{3.253242in}}{\pgfqpoint{2.551855in}{3.258415in}}{\pgfqpoint{2.561063in}{3.267624in}}%
\pgfpathcurveto{\pgfqpoint{2.570272in}{3.276832in}}{\pgfqpoint{2.575446in}{3.289323in}}{\pgfqpoint{2.575446in}{3.302346in}}%
\pgfpathcurveto{\pgfqpoint{2.575446in}{3.315369in}}{\pgfqpoint{2.570272in}{3.327860in}}{\pgfqpoint{2.561063in}{3.337068in}}%
\pgfpathcurveto{\pgfqpoint{2.551855in}{3.346277in}}{\pgfqpoint{2.539364in}{3.351451in}}{\pgfqpoint{2.526341in}{3.351451in}}%
\pgfpathcurveto{\pgfqpoint{2.513318in}{3.351451in}}{\pgfqpoint{2.500827in}{3.346277in}}{\pgfqpoint{2.491619in}{3.337068in}}%
\pgfpathcurveto{\pgfqpoint{2.482410in}{3.327860in}}{\pgfqpoint{2.477236in}{3.315369in}}{\pgfqpoint{2.477236in}{3.302346in}}%
\pgfpathcurveto{\pgfqpoint{2.477236in}{3.289323in}}{\pgfqpoint{2.482410in}{3.276832in}}{\pgfqpoint{2.491619in}{3.267624in}}%
\pgfpathcurveto{\pgfqpoint{2.500827in}{3.258415in}}{\pgfqpoint{2.513318in}{3.253242in}}{\pgfqpoint{2.526341in}{3.253242in}}%
\pgfpathlineto{\pgfqpoint{2.526341in}{3.253242in}}%
\pgfpathclose%
\pgfusepath{stroke,fill}%
\end{pgfscope}%
\begin{pgfscope}%
\pgfpathrectangle{\pgfqpoint{0.786164in}{0.768110in}}{\pgfqpoint{8.851069in}{7.081890in}}%
\pgfusepath{clip}%
\pgfsetbuttcap%
\pgfsetroundjoin%
\definecolor{currentfill}{rgb}{0.274128,0.199721,0.498911}%
\pgfsetfillcolor{currentfill}%
\pgfsetfillopacity{0.700000}%
\pgfsetlinewidth{0.501875pt}%
\definecolor{currentstroke}{rgb}{1.000000,1.000000,1.000000}%
\pgfsetstrokecolor{currentstroke}%
\pgfsetstrokeopacity{0.700000}%
\pgfsetdash{}{0pt}%
\pgfpathmoveto{\pgfqpoint{2.999725in}{3.274584in}}%
\pgfpathcurveto{\pgfqpoint{3.012747in}{3.274584in}}{\pgfqpoint{3.025239in}{3.279758in}}{\pgfqpoint{3.034447in}{3.288966in}}%
\pgfpathcurveto{\pgfqpoint{3.043655in}{3.298175in}}{\pgfqpoint{3.048829in}{3.310666in}}{\pgfqpoint{3.048829in}{3.323688in}}%
\pgfpathcurveto{\pgfqpoint{3.048829in}{3.336711in}}{\pgfqpoint{3.043655in}{3.349202in}}{\pgfqpoint{3.034447in}{3.358411in}}%
\pgfpathcurveto{\pgfqpoint{3.025239in}{3.367619in}}{\pgfqpoint{3.012747in}{3.372793in}}{\pgfqpoint{2.999725in}{3.372793in}}%
\pgfpathcurveto{\pgfqpoint{2.986702in}{3.372793in}}{\pgfqpoint{2.974211in}{3.367619in}}{\pgfqpoint{2.965003in}{3.358411in}}%
\pgfpathcurveto{\pgfqpoint{2.955794in}{3.349202in}}{\pgfqpoint{2.950620in}{3.336711in}}{\pgfqpoint{2.950620in}{3.323688in}}%
\pgfpathcurveto{\pgfqpoint{2.950620in}{3.310666in}}{\pgfqpoint{2.955794in}{3.298175in}}{\pgfqpoint{2.965003in}{3.288966in}}%
\pgfpathcurveto{\pgfqpoint{2.974211in}{3.279758in}}{\pgfqpoint{2.986702in}{3.274584in}}{\pgfqpoint{2.999725in}{3.274584in}}%
\pgfpathlineto{\pgfqpoint{2.999725in}{3.274584in}}%
\pgfpathclose%
\pgfusepath{stroke,fill}%
\end{pgfscope}%
\begin{pgfscope}%
\pgfpathrectangle{\pgfqpoint{0.786164in}{0.768110in}}{\pgfqpoint{8.851069in}{7.081890in}}%
\pgfusepath{clip}%
\pgfsetbuttcap%
\pgfsetroundjoin%
\definecolor{currentfill}{rgb}{0.277134,0.185228,0.489898}%
\pgfsetfillcolor{currentfill}%
\pgfsetfillopacity{0.700000}%
\pgfsetlinewidth{0.501875pt}%
\definecolor{currentstroke}{rgb}{1.000000,1.000000,1.000000}%
\pgfsetstrokecolor{currentstroke}%
\pgfsetstrokeopacity{0.700000}%
\pgfsetdash{}{0pt}%
\pgfpathmoveto{\pgfqpoint{3.053571in}{3.253242in}}%
\pgfpathcurveto{\pgfqpoint{3.066594in}{3.253242in}}{\pgfqpoint{3.079085in}{3.258415in}}{\pgfqpoint{3.088293in}{3.267624in}}%
\pgfpathcurveto{\pgfqpoint{3.097502in}{3.276832in}}{\pgfqpoint{3.102676in}{3.289323in}}{\pgfqpoint{3.102676in}{3.302346in}}%
\pgfpathcurveto{\pgfqpoint{3.102676in}{3.315369in}}{\pgfqpoint{3.097502in}{3.327860in}}{\pgfqpoint{3.088293in}{3.337068in}}%
\pgfpathcurveto{\pgfqpoint{3.079085in}{3.346277in}}{\pgfqpoint{3.066594in}{3.351451in}}{\pgfqpoint{3.053571in}{3.351451in}}%
\pgfpathcurveto{\pgfqpoint{3.040548in}{3.351451in}}{\pgfqpoint{3.028057in}{3.346277in}}{\pgfqpoint{3.018849in}{3.337068in}}%
\pgfpathcurveto{\pgfqpoint{3.009640in}{3.327860in}}{\pgfqpoint{3.004466in}{3.315369in}}{\pgfqpoint{3.004466in}{3.302346in}}%
\pgfpathcurveto{\pgfqpoint{3.004466in}{3.289323in}}{\pgfqpoint{3.009640in}{3.276832in}}{\pgfqpoint{3.018849in}{3.267624in}}%
\pgfpathcurveto{\pgfqpoint{3.028057in}{3.258415in}}{\pgfqpoint{3.040548in}{3.253242in}}{\pgfqpoint{3.053571in}{3.253242in}}%
\pgfpathlineto{\pgfqpoint{3.053571in}{3.253242in}}%
\pgfpathclose%
\pgfusepath{stroke,fill}%
\end{pgfscope}%
\begin{pgfscope}%
\pgfpathrectangle{\pgfqpoint{0.786164in}{0.768110in}}{\pgfqpoint{8.851069in}{7.081890in}}%
\pgfusepath{clip}%
\pgfsetbuttcap%
\pgfsetroundjoin%
\definecolor{currentfill}{rgb}{0.278826,0.175490,0.483397}%
\pgfsetfillcolor{currentfill}%
\pgfsetfillopacity{0.700000}%
\pgfsetlinewidth{0.501875pt}%
\definecolor{currentstroke}{rgb}{1.000000,1.000000,1.000000}%
\pgfsetstrokecolor{currentstroke}%
\pgfsetstrokeopacity{0.700000}%
\pgfsetdash{}{0pt}%
\pgfpathmoveto{\pgfqpoint{3.142094in}{3.146531in}}%
\pgfpathcurveto{\pgfqpoint{3.155117in}{3.146531in}}{\pgfqpoint{3.167608in}{3.151705in}}{\pgfqpoint{3.176816in}{3.160913in}}%
\pgfpathcurveto{\pgfqpoint{3.186025in}{3.170122in}}{\pgfqpoint{3.191199in}{3.182613in}}{\pgfqpoint{3.191199in}{3.195635in}}%
\pgfpathcurveto{\pgfqpoint{3.191199in}{3.208658in}}{\pgfqpoint{3.186025in}{3.221149in}}{\pgfqpoint{3.176816in}{3.230358in}}%
\pgfpathcurveto{\pgfqpoint{3.167608in}{3.239566in}}{\pgfqpoint{3.155117in}{3.244740in}}{\pgfqpoint{3.142094in}{3.244740in}}%
\pgfpathcurveto{\pgfqpoint{3.129071in}{3.244740in}}{\pgfqpoint{3.116580in}{3.239566in}}{\pgfqpoint{3.107372in}{3.230358in}}%
\pgfpathcurveto{\pgfqpoint{3.098163in}{3.221149in}}{\pgfqpoint{3.092989in}{3.208658in}}{\pgfqpoint{3.092989in}{3.195635in}}%
\pgfpathcurveto{\pgfqpoint{3.092989in}{3.182613in}}{\pgfqpoint{3.098163in}{3.170122in}}{\pgfqpoint{3.107372in}{3.160913in}}%
\pgfpathcurveto{\pgfqpoint{3.116580in}{3.151705in}}{\pgfqpoint{3.129071in}{3.146531in}}{\pgfqpoint{3.142094in}{3.146531in}}%
\pgfpathlineto{\pgfqpoint{3.142094in}{3.146531in}}%
\pgfpathclose%
\pgfusepath{stroke,fill}%
\end{pgfscope}%
\begin{pgfscope}%
\pgfpathrectangle{\pgfqpoint{0.786164in}{0.768110in}}{\pgfqpoint{8.851069in}{7.081890in}}%
\pgfusepath{clip}%
\pgfsetbuttcap%
\pgfsetroundjoin%
\definecolor{currentfill}{rgb}{0.271828,0.209303,0.504434}%
\pgfsetfillcolor{currentfill}%
\pgfsetfillopacity{0.700000}%
\pgfsetlinewidth{0.501875pt}%
\definecolor{currentstroke}{rgb}{1.000000,1.000000,1.000000}%
\pgfsetstrokecolor{currentstroke}%
\pgfsetstrokeopacity{0.700000}%
\pgfsetdash{}{0pt}%
\pgfpathmoveto{\pgfqpoint{3.082387in}{3.359952in}}%
\pgfpathcurveto{\pgfqpoint{3.095410in}{3.359952in}}{\pgfqpoint{3.107901in}{3.365126in}}{\pgfqpoint{3.117109in}{3.374335in}}%
\pgfpathcurveto{\pgfqpoint{3.126318in}{3.383543in}}{\pgfqpoint{3.131491in}{3.396034in}}{\pgfqpoint{3.131491in}{3.409057in}}%
\pgfpathcurveto{\pgfqpoint{3.131491in}{3.422080in}}{\pgfqpoint{3.126318in}{3.434571in}}{\pgfqpoint{3.117109in}{3.443779in}}%
\pgfpathcurveto{\pgfqpoint{3.107901in}{3.452988in}}{\pgfqpoint{3.095410in}{3.458162in}}{\pgfqpoint{3.082387in}{3.458162in}}%
\pgfpathcurveto{\pgfqpoint{3.069364in}{3.458162in}}{\pgfqpoint{3.056873in}{3.452988in}}{\pgfqpoint{3.047665in}{3.443779in}}%
\pgfpathcurveto{\pgfqpoint{3.038456in}{3.434571in}}{\pgfqpoint{3.033282in}{3.422080in}}{\pgfqpoint{3.033282in}{3.409057in}}%
\pgfpathcurveto{\pgfqpoint{3.033282in}{3.396034in}}{\pgfqpoint{3.038456in}{3.383543in}}{\pgfqpoint{3.047665in}{3.374335in}}%
\pgfpathcurveto{\pgfqpoint{3.056873in}{3.365126in}}{\pgfqpoint{3.069364in}{3.359952in}}{\pgfqpoint{3.082387in}{3.359952in}}%
\pgfpathlineto{\pgfqpoint{3.082387in}{3.359952in}}%
\pgfpathclose%
\pgfusepath{stroke,fill}%
\end{pgfscope}%
\begin{pgfscope}%
\pgfpathrectangle{\pgfqpoint{0.786164in}{0.768110in}}{\pgfqpoint{8.851069in}{7.081890in}}%
\pgfusepath{clip}%
\pgfsetbuttcap%
\pgfsetroundjoin%
\definecolor{currentfill}{rgb}{0.276194,0.190074,0.493001}%
\pgfsetfillcolor{currentfill}%
\pgfsetfillopacity{0.700000}%
\pgfsetlinewidth{0.501875pt}%
\definecolor{currentstroke}{rgb}{1.000000,1.000000,1.000000}%
\pgfsetstrokecolor{currentstroke}%
\pgfsetstrokeopacity{0.700000}%
\pgfsetdash{}{0pt}%
\pgfpathmoveto{\pgfqpoint{3.236966in}{3.189215in}}%
\pgfpathcurveto{\pgfqpoint{3.249989in}{3.189215in}}{\pgfqpoint{3.262480in}{3.194389in}}{\pgfqpoint{3.271688in}{3.203597in}}%
\pgfpathcurveto{\pgfqpoint{3.280897in}{3.212806in}}{\pgfqpoint{3.286071in}{3.225297in}}{\pgfqpoint{3.286071in}{3.238320in}}%
\pgfpathcurveto{\pgfqpoint{3.286071in}{3.251342in}}{\pgfqpoint{3.280897in}{3.263833in}}{\pgfqpoint{3.271688in}{3.273042in}}%
\pgfpathcurveto{\pgfqpoint{3.262480in}{3.282250in}}{\pgfqpoint{3.249989in}{3.287424in}}{\pgfqpoint{3.236966in}{3.287424in}}%
\pgfpathcurveto{\pgfqpoint{3.223943in}{3.287424in}}{\pgfqpoint{3.211452in}{3.282250in}}{\pgfqpoint{3.202244in}{3.273042in}}%
\pgfpathcurveto{\pgfqpoint{3.193035in}{3.263833in}}{\pgfqpoint{3.187861in}{3.251342in}}{\pgfqpoint{3.187861in}{3.238320in}}%
\pgfpathcurveto{\pgfqpoint{3.187861in}{3.225297in}}{\pgfqpoint{3.193035in}{3.212806in}}{\pgfqpoint{3.202244in}{3.203597in}}%
\pgfpathcurveto{\pgfqpoint{3.211452in}{3.194389in}}{\pgfqpoint{3.223943in}{3.189215in}}{\pgfqpoint{3.236966in}{3.189215in}}%
\pgfpathlineto{\pgfqpoint{3.236966in}{3.189215in}}%
\pgfpathclose%
\pgfusepath{stroke,fill}%
\end{pgfscope}%
\begin{pgfscope}%
\pgfpathrectangle{\pgfqpoint{0.786164in}{0.768110in}}{\pgfqpoint{8.851069in}{7.081890in}}%
\pgfusepath{clip}%
\pgfsetbuttcap%
\pgfsetroundjoin%
\definecolor{currentfill}{rgb}{0.282623,0.140926,0.457517}%
\pgfsetfillcolor{currentfill}%
\pgfsetfillopacity{0.700000}%
\pgfsetlinewidth{0.501875pt}%
\definecolor{currentstroke}{rgb}{1.000000,1.000000,1.000000}%
\pgfsetstrokecolor{currentstroke}%
\pgfsetstrokeopacity{0.700000}%
\pgfsetdash{}{0pt}%
\pgfpathmoveto{\pgfqpoint{3.520972in}{2.847740in}}%
\pgfpathcurveto{\pgfqpoint{3.533995in}{2.847740in}}{\pgfqpoint{3.546486in}{2.852914in}}{\pgfqpoint{3.555694in}{2.862123in}}%
\pgfpathcurveto{\pgfqpoint{3.564903in}{2.871331in}}{\pgfqpoint{3.570077in}{2.883822in}}{\pgfqpoint{3.570077in}{2.896845in}}%
\pgfpathcurveto{\pgfqpoint{3.570077in}{2.909868in}}{\pgfqpoint{3.564903in}{2.922359in}}{\pgfqpoint{3.555694in}{2.931567in}}%
\pgfpathcurveto{\pgfqpoint{3.546486in}{2.940776in}}{\pgfqpoint{3.533995in}{2.945950in}}{\pgfqpoint{3.520972in}{2.945950in}}%
\pgfpathcurveto{\pgfqpoint{3.507949in}{2.945950in}}{\pgfqpoint{3.495458in}{2.940776in}}{\pgfqpoint{3.486250in}{2.931567in}}%
\pgfpathcurveto{\pgfqpoint{3.477041in}{2.922359in}}{\pgfqpoint{3.471867in}{2.909868in}}{\pgfqpoint{3.471867in}{2.896845in}}%
\pgfpathcurveto{\pgfqpoint{3.471867in}{2.883822in}}{\pgfqpoint{3.477041in}{2.871331in}}{\pgfqpoint{3.486250in}{2.862123in}}%
\pgfpathcurveto{\pgfqpoint{3.495458in}{2.852914in}}{\pgfqpoint{3.507949in}{2.847740in}}{\pgfqpoint{3.520972in}{2.847740in}}%
\pgfpathlineto{\pgfqpoint{3.520972in}{2.847740in}}%
\pgfpathclose%
\pgfusepath{stroke,fill}%
\end{pgfscope}%
\begin{pgfscope}%
\pgfpathrectangle{\pgfqpoint{0.786164in}{0.768110in}}{\pgfqpoint{8.851069in}{7.081890in}}%
\pgfusepath{clip}%
\pgfsetbuttcap%
\pgfsetroundjoin%
\definecolor{currentfill}{rgb}{0.281887,0.150881,0.465405}%
\pgfsetfillcolor{currentfill}%
\pgfsetfillopacity{0.700000}%
\pgfsetlinewidth{0.501875pt}%
\definecolor{currentstroke}{rgb}{1.000000,1.000000,1.000000}%
\pgfsetstrokecolor{currentstroke}%
\pgfsetstrokeopacity{0.700000}%
\pgfsetdash{}{0pt}%
\pgfpathmoveto{\pgfqpoint{3.559800in}{3.039820in}}%
\pgfpathcurveto{\pgfqpoint{3.572823in}{3.039820in}}{\pgfqpoint{3.585314in}{3.044994in}}{\pgfqpoint{3.594522in}{3.054202in}}%
\pgfpathcurveto{\pgfqpoint{3.603731in}{3.063411in}}{\pgfqpoint{3.608905in}{3.075902in}}{\pgfqpoint{3.608905in}{3.088924in}}%
\pgfpathcurveto{\pgfqpoint{3.608905in}{3.101947in}}{\pgfqpoint{3.603731in}{3.114438in}}{\pgfqpoint{3.594522in}{3.123647in}}%
\pgfpathcurveto{\pgfqpoint{3.585314in}{3.132855in}}{\pgfqpoint{3.572823in}{3.138029in}}{\pgfqpoint{3.559800in}{3.138029in}}%
\pgfpathcurveto{\pgfqpoint{3.546777in}{3.138029in}}{\pgfqpoint{3.534286in}{3.132855in}}{\pgfqpoint{3.525078in}{3.123647in}}%
\pgfpathcurveto{\pgfqpoint{3.515869in}{3.114438in}}{\pgfqpoint{3.510695in}{3.101947in}}{\pgfqpoint{3.510695in}{3.088924in}}%
\pgfpathcurveto{\pgfqpoint{3.510695in}{3.075902in}}{\pgfqpoint{3.515869in}{3.063411in}}{\pgfqpoint{3.525078in}{3.054202in}}%
\pgfpathcurveto{\pgfqpoint{3.534286in}{3.044994in}}{\pgfqpoint{3.546777in}{3.039820in}}{\pgfqpoint{3.559800in}{3.039820in}}%
\pgfpathlineto{\pgfqpoint{3.559800in}{3.039820in}}%
\pgfpathclose%
\pgfusepath{stroke,fill}%
\end{pgfscope}%
\begin{pgfscope}%
\pgfpathrectangle{\pgfqpoint{0.786164in}{0.768110in}}{\pgfqpoint{8.851069in}{7.081890in}}%
\pgfusepath{clip}%
\pgfsetbuttcap%
\pgfsetroundjoin%
\definecolor{currentfill}{rgb}{0.282884,0.135920,0.453427}%
\pgfsetfillcolor{currentfill}%
\pgfsetfillopacity{0.700000}%
\pgfsetlinewidth{0.501875pt}%
\definecolor{currentstroke}{rgb}{1.000000,1.000000,1.000000}%
\pgfsetstrokecolor{currentstroke}%
\pgfsetstrokeopacity{0.700000}%
\pgfsetdash{}{0pt}%
\pgfpathmoveto{\pgfqpoint{3.714746in}{2.847740in}}%
\pgfpathcurveto{\pgfqpoint{3.727768in}{2.847740in}}{\pgfqpoint{3.740259in}{2.852914in}}{\pgfqpoint{3.749468in}{2.862123in}}%
\pgfpathcurveto{\pgfqpoint{3.758676in}{2.871331in}}{\pgfqpoint{3.763850in}{2.883822in}}{\pgfqpoint{3.763850in}{2.896845in}}%
\pgfpathcurveto{\pgfqpoint{3.763850in}{2.909868in}}{\pgfqpoint{3.758676in}{2.922359in}}{\pgfqpoint{3.749468in}{2.931567in}}%
\pgfpathcurveto{\pgfqpoint{3.740259in}{2.940776in}}{\pgfqpoint{3.727768in}{2.945950in}}{\pgfqpoint{3.714746in}{2.945950in}}%
\pgfpathcurveto{\pgfqpoint{3.701723in}{2.945950in}}{\pgfqpoint{3.689232in}{2.940776in}}{\pgfqpoint{3.680023in}{2.931567in}}%
\pgfpathcurveto{\pgfqpoint{3.670815in}{2.922359in}}{\pgfqpoint{3.665641in}{2.909868in}}{\pgfqpoint{3.665641in}{2.896845in}}%
\pgfpathcurveto{\pgfqpoint{3.665641in}{2.883822in}}{\pgfqpoint{3.670815in}{2.871331in}}{\pgfqpoint{3.680023in}{2.862123in}}%
\pgfpathcurveto{\pgfqpoint{3.689232in}{2.852914in}}{\pgfqpoint{3.701723in}{2.847740in}}{\pgfqpoint{3.714746in}{2.847740in}}%
\pgfpathlineto{\pgfqpoint{3.714746in}{2.847740in}}%
\pgfpathclose%
\pgfusepath{stroke,fill}%
\end{pgfscope}%
\begin{pgfscope}%
\pgfpathrectangle{\pgfqpoint{0.786164in}{0.768110in}}{\pgfqpoint{8.851069in}{7.081890in}}%
\pgfusepath{clip}%
\pgfsetbuttcap%
\pgfsetroundjoin%
\definecolor{currentfill}{rgb}{0.283072,0.130895,0.449241}%
\pgfsetfillcolor{currentfill}%
\pgfsetfillopacity{0.700000}%
\pgfsetlinewidth{0.501875pt}%
\definecolor{currentstroke}{rgb}{1.000000,1.000000,1.000000}%
\pgfsetstrokecolor{currentstroke}%
\pgfsetstrokeopacity{0.700000}%
\pgfsetdash{}{0pt}%
\pgfpathmoveto{\pgfqpoint{3.850888in}{2.762372in}}%
\pgfpathcurveto{\pgfqpoint{3.863910in}{2.762372in}}{\pgfqpoint{3.876401in}{2.767546in}}{\pgfqpoint{3.885610in}{2.776754in}}%
\pgfpathcurveto{\pgfqpoint{3.894818in}{2.785962in}}{\pgfqpoint{3.899992in}{2.798454in}}{\pgfqpoint{3.899992in}{2.811476in}}%
\pgfpathcurveto{\pgfqpoint{3.899992in}{2.824499in}}{\pgfqpoint{3.894818in}{2.836990in}}{\pgfqpoint{3.885610in}{2.846198in}}%
\pgfpathcurveto{\pgfqpoint{3.876401in}{2.855407in}}{\pgfqpoint{3.863910in}{2.860581in}}{\pgfqpoint{3.850888in}{2.860581in}}%
\pgfpathcurveto{\pgfqpoint{3.837865in}{2.860581in}}{\pgfqpoint{3.825374in}{2.855407in}}{\pgfqpoint{3.816165in}{2.846198in}}%
\pgfpathcurveto{\pgfqpoint{3.806957in}{2.836990in}}{\pgfqpoint{3.801783in}{2.824499in}}{\pgfqpoint{3.801783in}{2.811476in}}%
\pgfpathcurveto{\pgfqpoint{3.801783in}{2.798454in}}{\pgfqpoint{3.806957in}{2.785962in}}{\pgfqpoint{3.816165in}{2.776754in}}%
\pgfpathcurveto{\pgfqpoint{3.825374in}{2.767546in}}{\pgfqpoint{3.837865in}{2.762372in}}{\pgfqpoint{3.850888in}{2.762372in}}%
\pgfpathlineto{\pgfqpoint{3.850888in}{2.762372in}}%
\pgfpathclose%
\pgfusepath{stroke,fill}%
\end{pgfscope}%
\begin{pgfscope}%
\pgfpathrectangle{\pgfqpoint{0.786164in}{0.768110in}}{\pgfqpoint{8.851069in}{7.081890in}}%
\pgfusepath{clip}%
\pgfsetbuttcap%
\pgfsetroundjoin%
\definecolor{currentfill}{rgb}{0.283072,0.130895,0.449241}%
\pgfsetfillcolor{currentfill}%
\pgfsetfillopacity{0.700000}%
\pgfsetlinewidth{0.501875pt}%
\definecolor{currentstroke}{rgb}{1.000000,1.000000,1.000000}%
\pgfsetstrokecolor{currentstroke}%
\pgfsetstrokeopacity{0.700000}%
\pgfsetdash{}{0pt}%
\pgfpathmoveto{\pgfqpoint{3.975674in}{2.591634in}}%
\pgfpathcurveto{\pgfqpoint{3.988697in}{2.591634in}}{\pgfqpoint{4.001188in}{2.596808in}}{\pgfqpoint{4.010397in}{2.606017in}}%
\pgfpathcurveto{\pgfqpoint{4.019605in}{2.615225in}}{\pgfqpoint{4.024779in}{2.627716in}}{\pgfqpoint{4.024779in}{2.640739in}}%
\pgfpathcurveto{\pgfqpoint{4.024779in}{2.653762in}}{\pgfqpoint{4.019605in}{2.666253in}}{\pgfqpoint{4.010397in}{2.675461in}}%
\pgfpathcurveto{\pgfqpoint{4.001188in}{2.684670in}}{\pgfqpoint{3.988697in}{2.689844in}}{\pgfqpoint{3.975674in}{2.689844in}}%
\pgfpathcurveto{\pgfqpoint{3.962652in}{2.689844in}}{\pgfqpoint{3.950161in}{2.684670in}}{\pgfqpoint{3.940952in}{2.675461in}}%
\pgfpathcurveto{\pgfqpoint{3.931744in}{2.666253in}}{\pgfqpoint{3.926570in}{2.653762in}}{\pgfqpoint{3.926570in}{2.640739in}}%
\pgfpathcurveto{\pgfqpoint{3.926570in}{2.627716in}}{\pgfqpoint{3.931744in}{2.615225in}}{\pgfqpoint{3.940952in}{2.606017in}}%
\pgfpathcurveto{\pgfqpoint{3.950161in}{2.596808in}}{\pgfqpoint{3.962652in}{2.591634in}}{\pgfqpoint{3.975674in}{2.591634in}}%
\pgfpathlineto{\pgfqpoint{3.975674in}{2.591634in}}%
\pgfpathclose%
\pgfusepath{stroke,fill}%
\end{pgfscope}%
\begin{pgfscope}%
\pgfpathrectangle{\pgfqpoint{0.786164in}{0.768110in}}{\pgfqpoint{8.851069in}{7.081890in}}%
\pgfusepath{clip}%
\pgfsetbuttcap%
\pgfsetroundjoin%
\definecolor{currentfill}{rgb}{0.283072,0.130895,0.449241}%
\pgfsetfillcolor{currentfill}%
\pgfsetfillopacity{0.700000}%
\pgfsetlinewidth{0.501875pt}%
\definecolor{currentstroke}{rgb}{1.000000,1.000000,1.000000}%
\pgfsetstrokecolor{currentstroke}%
\pgfsetstrokeopacity{0.700000}%
\pgfsetdash{}{0pt}%
\pgfpathmoveto{\pgfqpoint{4.155895in}{2.506266in}}%
\pgfpathcurveto{\pgfqpoint{4.168918in}{2.506266in}}{\pgfqpoint{4.181409in}{2.511440in}}{\pgfqpoint{4.190617in}{2.520648in}}%
\pgfpathcurveto{\pgfqpoint{4.199826in}{2.529856in}}{\pgfqpoint{4.205000in}{2.542347in}}{\pgfqpoint{4.205000in}{2.555370in}}%
\pgfpathcurveto{\pgfqpoint{4.205000in}{2.568393in}}{\pgfqpoint{4.199826in}{2.580884in}}{\pgfqpoint{4.190617in}{2.590092in}}%
\pgfpathcurveto{\pgfqpoint{4.181409in}{2.599301in}}{\pgfqpoint{4.168918in}{2.604475in}}{\pgfqpoint{4.155895in}{2.604475in}}%
\pgfpathcurveto{\pgfqpoint{4.142872in}{2.604475in}}{\pgfqpoint{4.130381in}{2.599301in}}{\pgfqpoint{4.121173in}{2.590092in}}%
\pgfpathcurveto{\pgfqpoint{4.111964in}{2.580884in}}{\pgfqpoint{4.106790in}{2.568393in}}{\pgfqpoint{4.106790in}{2.555370in}}%
\pgfpathcurveto{\pgfqpoint{4.106790in}{2.542347in}}{\pgfqpoint{4.111964in}{2.529856in}}{\pgfqpoint{4.121173in}{2.520648in}}%
\pgfpathcurveto{\pgfqpoint{4.130381in}{2.511440in}}{\pgfqpoint{4.142872in}{2.506266in}}{\pgfqpoint{4.155895in}{2.506266in}}%
\pgfpathlineto{\pgfqpoint{4.155895in}{2.506266in}}%
\pgfpathclose%
\pgfusepath{stroke,fill}%
\end{pgfscope}%
\begin{pgfscope}%
\pgfpathrectangle{\pgfqpoint{0.786164in}{0.768110in}}{\pgfqpoint{8.851069in}{7.081890in}}%
\pgfusepath{clip}%
\pgfsetbuttcap%
\pgfsetroundjoin%
\definecolor{currentfill}{rgb}{0.283197,0.115680,0.436115}%
\pgfsetfillcolor{currentfill}%
\pgfsetfillopacity{0.700000}%
\pgfsetlinewidth{0.501875pt}%
\definecolor{currentstroke}{rgb}{1.000000,1.000000,1.000000}%
\pgfsetstrokecolor{currentstroke}%
\pgfsetstrokeopacity{0.700000}%
\pgfsetdash{}{0pt}%
\pgfpathmoveto{\pgfqpoint{4.124759in}{2.506266in}}%
\pgfpathcurveto{\pgfqpoint{4.137782in}{2.506266in}}{\pgfqpoint{4.150273in}{2.511440in}}{\pgfqpoint{4.159481in}{2.520648in}}%
\pgfpathcurveto{\pgfqpoint{4.168690in}{2.529856in}}{\pgfqpoint{4.173864in}{2.542347in}}{\pgfqpoint{4.173864in}{2.555370in}}%
\pgfpathcurveto{\pgfqpoint{4.173864in}{2.568393in}}{\pgfqpoint{4.168690in}{2.580884in}}{\pgfqpoint{4.159481in}{2.590092in}}%
\pgfpathcurveto{\pgfqpoint{4.150273in}{2.599301in}}{\pgfqpoint{4.137782in}{2.604475in}}{\pgfqpoint{4.124759in}{2.604475in}}%
\pgfpathcurveto{\pgfqpoint{4.111737in}{2.604475in}}{\pgfqpoint{4.099245in}{2.599301in}}{\pgfqpoint{4.090037in}{2.590092in}}%
\pgfpathcurveto{\pgfqpoint{4.080829in}{2.580884in}}{\pgfqpoint{4.075655in}{2.568393in}}{\pgfqpoint{4.075655in}{2.555370in}}%
\pgfpathcurveto{\pgfqpoint{4.075655in}{2.542347in}}{\pgfqpoint{4.080829in}{2.529856in}}{\pgfqpoint{4.090037in}{2.520648in}}%
\pgfpathcurveto{\pgfqpoint{4.099245in}{2.511440in}}{\pgfqpoint{4.111737in}{2.506266in}}{\pgfqpoint{4.124759in}{2.506266in}}%
\pgfpathlineto{\pgfqpoint{4.124759in}{2.506266in}}%
\pgfpathclose%
\pgfusepath{stroke,fill}%
\end{pgfscope}%
\begin{pgfscope}%
\pgfpathrectangle{\pgfqpoint{0.786164in}{0.768110in}}{\pgfqpoint{8.851069in}{7.081890in}}%
\pgfusepath{clip}%
\pgfsetbuttcap%
\pgfsetroundjoin%
\definecolor{currentfill}{rgb}{0.282910,0.105393,0.426902}%
\pgfsetfillcolor{currentfill}%
\pgfsetfillopacity{0.700000}%
\pgfsetlinewidth{0.501875pt}%
\definecolor{currentstroke}{rgb}{1.000000,1.000000,1.000000}%
\pgfsetstrokecolor{currentstroke}%
\pgfsetstrokeopacity{0.700000}%
\pgfsetdash{}{0pt}%
\pgfpathmoveto{\pgfqpoint{4.180071in}{2.548950in}}%
\pgfpathcurveto{\pgfqpoint{4.193093in}{2.548950in}}{\pgfqpoint{4.205585in}{2.554124in}}{\pgfqpoint{4.214793in}{2.563332in}}%
\pgfpathcurveto{\pgfqpoint{4.224001in}{2.572541in}}{\pgfqpoint{4.229175in}{2.585032in}}{\pgfqpoint{4.229175in}{2.598055in}}%
\pgfpathcurveto{\pgfqpoint{4.229175in}{2.611077in}}{\pgfqpoint{4.224001in}{2.623568in}}{\pgfqpoint{4.214793in}{2.632777in}}%
\pgfpathcurveto{\pgfqpoint{4.205585in}{2.641985in}}{\pgfqpoint{4.193093in}{2.647159in}}{\pgfqpoint{4.180071in}{2.647159in}}%
\pgfpathcurveto{\pgfqpoint{4.167048in}{2.647159in}}{\pgfqpoint{4.154557in}{2.641985in}}{\pgfqpoint{4.145349in}{2.632777in}}%
\pgfpathcurveto{\pgfqpoint{4.136140in}{2.623568in}}{\pgfqpoint{4.130966in}{2.611077in}}{\pgfqpoint{4.130966in}{2.598055in}}%
\pgfpathcurveto{\pgfqpoint{4.130966in}{2.585032in}}{\pgfqpoint{4.136140in}{2.572541in}}{\pgfqpoint{4.145349in}{2.563332in}}%
\pgfpathcurveto{\pgfqpoint{4.154557in}{2.554124in}}{\pgfqpoint{4.167048in}{2.548950in}}{\pgfqpoint{4.180071in}{2.548950in}}%
\pgfpathlineto{\pgfqpoint{4.180071in}{2.548950in}}%
\pgfpathclose%
\pgfusepath{stroke,fill}%
\end{pgfscope}%
\begin{pgfscope}%
\pgfpathrectangle{\pgfqpoint{0.786164in}{0.768110in}}{\pgfqpoint{8.851069in}{7.081890in}}%
\pgfusepath{clip}%
\pgfsetbuttcap%
\pgfsetroundjoin%
\definecolor{currentfill}{rgb}{0.282327,0.094955,0.417331}%
\pgfsetfillcolor{currentfill}%
\pgfsetfillopacity{0.700000}%
\pgfsetlinewidth{0.501875pt}%
\definecolor{currentstroke}{rgb}{1.000000,1.000000,1.000000}%
\pgfsetstrokecolor{currentstroke}%
\pgfsetstrokeopacity{0.700000}%
\pgfsetdash{}{0pt}%
\pgfpathmoveto{\pgfqpoint{4.176530in}{2.548950in}}%
\pgfpathcurveto{\pgfqpoint{4.189553in}{2.548950in}}{\pgfqpoint{4.202044in}{2.554124in}}{\pgfqpoint{4.211252in}{2.563332in}}%
\pgfpathcurveto{\pgfqpoint{4.220461in}{2.572541in}}{\pgfqpoint{4.225635in}{2.585032in}}{\pgfqpoint{4.225635in}{2.598055in}}%
\pgfpathcurveto{\pgfqpoint{4.225635in}{2.611077in}}{\pgfqpoint{4.220461in}{2.623568in}}{\pgfqpoint{4.211252in}{2.632777in}}%
\pgfpathcurveto{\pgfqpoint{4.202044in}{2.641985in}}{\pgfqpoint{4.189553in}{2.647159in}}{\pgfqpoint{4.176530in}{2.647159in}}%
\pgfpathcurveto{\pgfqpoint{4.163507in}{2.647159in}}{\pgfqpoint{4.151016in}{2.641985in}}{\pgfqpoint{4.141808in}{2.632777in}}%
\pgfpathcurveto{\pgfqpoint{4.132599in}{2.623568in}}{\pgfqpoint{4.127425in}{2.611077in}}{\pgfqpoint{4.127425in}{2.598055in}}%
\pgfpathcurveto{\pgfqpoint{4.127425in}{2.585032in}}{\pgfqpoint{4.132599in}{2.572541in}}{\pgfqpoint{4.141808in}{2.563332in}}%
\pgfpathcurveto{\pgfqpoint{4.151016in}{2.554124in}}{\pgfqpoint{4.163507in}{2.548950in}}{\pgfqpoint{4.176530in}{2.548950in}}%
\pgfpathlineto{\pgfqpoint{4.176530in}{2.548950in}}%
\pgfpathclose%
\pgfusepath{stroke,fill}%
\end{pgfscope}%
\begin{pgfscope}%
\pgfpathrectangle{\pgfqpoint{0.786164in}{0.768110in}}{\pgfqpoint{8.851069in}{7.081890in}}%
\pgfusepath{clip}%
\pgfsetbuttcap%
\pgfsetroundjoin%
\definecolor{currentfill}{rgb}{0.282327,0.094955,0.417331}%
\pgfsetfillcolor{currentfill}%
\pgfsetfillopacity{0.700000}%
\pgfsetlinewidth{0.501875pt}%
\definecolor{currentstroke}{rgb}{1.000000,1.000000,1.000000}%
\pgfsetstrokecolor{currentstroke}%
\pgfsetstrokeopacity{0.700000}%
\pgfsetdash{}{0pt}%
\pgfpathmoveto{\pgfqpoint{4.313893in}{2.548950in}}%
\pgfpathcurveto{\pgfqpoint{4.326916in}{2.548950in}}{\pgfqpoint{4.339407in}{2.554124in}}{\pgfqpoint{4.348615in}{2.563332in}}%
\pgfpathcurveto{\pgfqpoint{4.357824in}{2.572541in}}{\pgfqpoint{4.362998in}{2.585032in}}{\pgfqpoint{4.362998in}{2.598055in}}%
\pgfpathcurveto{\pgfqpoint{4.362998in}{2.611077in}}{\pgfqpoint{4.357824in}{2.623568in}}{\pgfqpoint{4.348615in}{2.632777in}}%
\pgfpathcurveto{\pgfqpoint{4.339407in}{2.641985in}}{\pgfqpoint{4.326916in}{2.647159in}}{\pgfqpoint{4.313893in}{2.647159in}}%
\pgfpathcurveto{\pgfqpoint{4.300870in}{2.647159in}}{\pgfqpoint{4.288379in}{2.641985in}}{\pgfqpoint{4.279171in}{2.632777in}}%
\pgfpathcurveto{\pgfqpoint{4.269962in}{2.623568in}}{\pgfqpoint{4.264788in}{2.611077in}}{\pgfqpoint{4.264788in}{2.598055in}}%
\pgfpathcurveto{\pgfqpoint{4.264788in}{2.585032in}}{\pgfqpoint{4.269962in}{2.572541in}}{\pgfqpoint{4.279171in}{2.563332in}}%
\pgfpathcurveto{\pgfqpoint{4.288379in}{2.554124in}}{\pgfqpoint{4.300870in}{2.548950in}}{\pgfqpoint{4.313893in}{2.548950in}}%
\pgfpathlineto{\pgfqpoint{4.313893in}{2.548950in}}%
\pgfpathclose%
\pgfusepath{stroke,fill}%
\end{pgfscope}%
\begin{pgfscope}%
\pgfpathrectangle{\pgfqpoint{0.786164in}{0.768110in}}{\pgfqpoint{8.851069in}{7.081890in}}%
\pgfusepath{clip}%
\pgfsetbuttcap%
\pgfsetroundjoin%
\definecolor{currentfill}{rgb}{0.281924,0.089666,0.412415}%
\pgfsetfillcolor{currentfill}%
\pgfsetfillopacity{0.700000}%
\pgfsetlinewidth{0.501875pt}%
\definecolor{currentstroke}{rgb}{1.000000,1.000000,1.000000}%
\pgfsetstrokecolor{currentstroke}%
\pgfsetstrokeopacity{0.700000}%
\pgfsetdash{}{0pt}%
\pgfpathmoveto{\pgfqpoint{4.778974in}{2.442239in}}%
\pgfpathcurveto{\pgfqpoint{4.791997in}{2.442239in}}{\pgfqpoint{4.804488in}{2.447413in}}{\pgfqpoint{4.813696in}{2.456621in}}%
\pgfpathcurveto{\pgfqpoint{4.822905in}{2.465830in}}{\pgfqpoint{4.828079in}{2.478321in}}{\pgfqpoint{4.828079in}{2.491344in}}%
\pgfpathcurveto{\pgfqpoint{4.828079in}{2.504366in}}{\pgfqpoint{4.822905in}{2.516857in}}{\pgfqpoint{4.813696in}{2.526066in}}%
\pgfpathcurveto{\pgfqpoint{4.804488in}{2.535274in}}{\pgfqpoint{4.791997in}{2.540448in}}{\pgfqpoint{4.778974in}{2.540448in}}%
\pgfpathcurveto{\pgfqpoint{4.765951in}{2.540448in}}{\pgfqpoint{4.753460in}{2.535274in}}{\pgfqpoint{4.744252in}{2.526066in}}%
\pgfpathcurveto{\pgfqpoint{4.735043in}{2.516857in}}{\pgfqpoint{4.729869in}{2.504366in}}{\pgfqpoint{4.729869in}{2.491344in}}%
\pgfpathcurveto{\pgfqpoint{4.729869in}{2.478321in}}{\pgfqpoint{4.735043in}{2.465830in}}{\pgfqpoint{4.744252in}{2.456621in}}%
\pgfpathcurveto{\pgfqpoint{4.753460in}{2.447413in}}{\pgfqpoint{4.765951in}{2.442239in}}{\pgfqpoint{4.778974in}{2.442239in}}%
\pgfpathlineto{\pgfqpoint{4.778974in}{2.442239in}}%
\pgfpathclose%
\pgfusepath{stroke,fill}%
\end{pgfscope}%
\begin{pgfscope}%
\pgfpathrectangle{\pgfqpoint{0.786164in}{0.768110in}}{\pgfqpoint{8.851069in}{7.081890in}}%
\pgfusepath{clip}%
\pgfsetbuttcap%
\pgfsetroundjoin%
\definecolor{currentfill}{rgb}{0.282910,0.105393,0.426902}%
\pgfsetfillcolor{currentfill}%
\pgfsetfillopacity{0.700000}%
\pgfsetlinewidth{0.501875pt}%
\definecolor{currentstroke}{rgb}{1.000000,1.000000,1.000000}%
\pgfsetstrokecolor{currentstroke}%
\pgfsetstrokeopacity{0.700000}%
\pgfsetdash{}{0pt}%
\pgfpathmoveto{\pgfqpoint{4.901807in}{2.570292in}}%
\pgfpathcurveto{\pgfqpoint{4.914830in}{2.570292in}}{\pgfqpoint{4.927321in}{2.575466in}}{\pgfqpoint{4.936529in}{2.584674in}}%
\pgfpathcurveto{\pgfqpoint{4.945738in}{2.593883in}}{\pgfqpoint{4.950912in}{2.606374in}}{\pgfqpoint{4.950912in}{2.619397in}}%
\pgfpathcurveto{\pgfqpoint{4.950912in}{2.632419in}}{\pgfqpoint{4.945738in}{2.644910in}}{\pgfqpoint{4.936529in}{2.654119in}}%
\pgfpathcurveto{\pgfqpoint{4.927321in}{2.663327in}}{\pgfqpoint{4.914830in}{2.668501in}}{\pgfqpoint{4.901807in}{2.668501in}}%
\pgfpathcurveto{\pgfqpoint{4.888784in}{2.668501in}}{\pgfqpoint{4.876293in}{2.663327in}}{\pgfqpoint{4.867085in}{2.654119in}}%
\pgfpathcurveto{\pgfqpoint{4.857876in}{2.644910in}}{\pgfqpoint{4.852702in}{2.632419in}}{\pgfqpoint{4.852702in}{2.619397in}}%
\pgfpathcurveto{\pgfqpoint{4.852702in}{2.606374in}}{\pgfqpoint{4.857876in}{2.593883in}}{\pgfqpoint{4.867085in}{2.584674in}}%
\pgfpathcurveto{\pgfqpoint{4.876293in}{2.575466in}}{\pgfqpoint{4.888784in}{2.570292in}}{\pgfqpoint{4.901807in}{2.570292in}}%
\pgfpathlineto{\pgfqpoint{4.901807in}{2.570292in}}%
\pgfpathclose%
\pgfusepath{stroke,fill}%
\end{pgfscope}%
\begin{pgfscope}%
\pgfpathrectangle{\pgfqpoint{0.786164in}{0.768110in}}{\pgfqpoint{8.851069in}{7.081890in}}%
\pgfusepath{clip}%
\pgfsetbuttcap%
\pgfsetroundjoin%
\definecolor{currentfill}{rgb}{0.283091,0.110553,0.431554}%
\pgfsetfillcolor{currentfill}%
\pgfsetfillopacity{0.700000}%
\pgfsetlinewidth{0.501875pt}%
\definecolor{currentstroke}{rgb}{1.000000,1.000000,1.000000}%
\pgfsetstrokecolor{currentstroke}%
\pgfsetstrokeopacity{0.700000}%
\pgfsetdash{}{0pt}%
\pgfpathmoveto{\pgfqpoint{4.944542in}{2.634319in}}%
\pgfpathcurveto{\pgfqpoint{4.957565in}{2.634319in}}{\pgfqpoint{4.970056in}{2.639493in}}{\pgfqpoint{4.979265in}{2.648701in}}%
\pgfpathcurveto{\pgfqpoint{4.988473in}{2.657909in}}{\pgfqpoint{4.993647in}{2.670401in}}{\pgfqpoint{4.993647in}{2.683423in}}%
\pgfpathcurveto{\pgfqpoint{4.993647in}{2.696446in}}{\pgfqpoint{4.988473in}{2.708937in}}{\pgfqpoint{4.979265in}{2.718145in}}%
\pgfpathcurveto{\pgfqpoint{4.970056in}{2.727354in}}{\pgfqpoint{4.957565in}{2.732528in}}{\pgfqpoint{4.944542in}{2.732528in}}%
\pgfpathcurveto{\pgfqpoint{4.931520in}{2.732528in}}{\pgfqpoint{4.919029in}{2.727354in}}{\pgfqpoint{4.909820in}{2.718145in}}%
\pgfpathcurveto{\pgfqpoint{4.900612in}{2.708937in}}{\pgfqpoint{4.895438in}{2.696446in}}{\pgfqpoint{4.895438in}{2.683423in}}%
\pgfpathcurveto{\pgfqpoint{4.895438in}{2.670401in}}{\pgfqpoint{4.900612in}{2.657909in}}{\pgfqpoint{4.909820in}{2.648701in}}%
\pgfpathcurveto{\pgfqpoint{4.919029in}{2.639493in}}{\pgfqpoint{4.931520in}{2.634319in}}{\pgfqpoint{4.944542in}{2.634319in}}%
\pgfpathlineto{\pgfqpoint{4.944542in}{2.634319in}}%
\pgfpathclose%
\pgfusepath{stroke,fill}%
\end{pgfscope}%
\begin{pgfscope}%
\pgfpathrectangle{\pgfqpoint{0.786164in}{0.768110in}}{\pgfqpoint{8.851069in}{7.081890in}}%
\pgfusepath{clip}%
\pgfsetbuttcap%
\pgfsetroundjoin%
\definecolor{currentfill}{rgb}{0.283187,0.125848,0.444960}%
\pgfsetfillcolor{currentfill}%
\pgfsetfillopacity{0.700000}%
\pgfsetlinewidth{0.501875pt}%
\definecolor{currentstroke}{rgb}{1.000000,1.000000,1.000000}%
\pgfsetstrokecolor{currentstroke}%
\pgfsetstrokeopacity{0.700000}%
\pgfsetdash{}{0pt}%
\pgfpathmoveto{\pgfqpoint{4.913651in}{3.189215in}}%
\pgfpathcurveto{\pgfqpoint{4.926674in}{3.189215in}}{\pgfqpoint{4.939165in}{3.194389in}}{\pgfqpoint{4.948373in}{3.203597in}}%
\pgfpathcurveto{\pgfqpoint{4.957582in}{3.212806in}}{\pgfqpoint{4.962756in}{3.225297in}}{\pgfqpoint{4.962756in}{3.238320in}}%
\pgfpathcurveto{\pgfqpoint{4.962756in}{3.251342in}}{\pgfqpoint{4.957582in}{3.263833in}}{\pgfqpoint{4.948373in}{3.273042in}}%
\pgfpathcurveto{\pgfqpoint{4.939165in}{3.282250in}}{\pgfqpoint{4.926674in}{3.287424in}}{\pgfqpoint{4.913651in}{3.287424in}}%
\pgfpathcurveto{\pgfqpoint{4.900628in}{3.287424in}}{\pgfqpoint{4.888137in}{3.282250in}}{\pgfqpoint{4.878929in}{3.273042in}}%
\pgfpathcurveto{\pgfqpoint{4.869720in}{3.263833in}}{\pgfqpoint{4.864546in}{3.251342in}}{\pgfqpoint{4.864546in}{3.238320in}}%
\pgfpathcurveto{\pgfqpoint{4.864546in}{3.225297in}}{\pgfqpoint{4.869720in}{3.212806in}}{\pgfqpoint{4.878929in}{3.203597in}}%
\pgfpathcurveto{\pgfqpoint{4.888137in}{3.194389in}}{\pgfqpoint{4.900628in}{3.189215in}}{\pgfqpoint{4.913651in}{3.189215in}}%
\pgfpathlineto{\pgfqpoint{4.913651in}{3.189215in}}%
\pgfpathclose%
\pgfusepath{stroke,fill}%
\end{pgfscope}%
\begin{pgfscope}%
\pgfpathrectangle{\pgfqpoint{0.786164in}{0.768110in}}{\pgfqpoint{8.851069in}{7.081890in}}%
\pgfusepath{clip}%
\pgfsetbuttcap%
\pgfsetroundjoin%
\definecolor{currentfill}{rgb}{0.283187,0.125848,0.444960}%
\pgfsetfillcolor{currentfill}%
\pgfsetfillopacity{0.700000}%
\pgfsetlinewidth{0.501875pt}%
\definecolor{currentstroke}{rgb}{1.000000,1.000000,1.000000}%
\pgfsetstrokecolor{currentstroke}%
\pgfsetstrokeopacity{0.700000}%
\pgfsetdash{}{0pt}%
\pgfpathmoveto{\pgfqpoint{4.863834in}{2.591634in}}%
\pgfpathcurveto{\pgfqpoint{4.876857in}{2.591634in}}{\pgfqpoint{4.889348in}{2.596808in}}{\pgfqpoint{4.898556in}{2.606017in}}%
\pgfpathcurveto{\pgfqpoint{4.907765in}{2.615225in}}{\pgfqpoint{4.912939in}{2.627716in}}{\pgfqpoint{4.912939in}{2.640739in}}%
\pgfpathcurveto{\pgfqpoint{4.912939in}{2.653762in}}{\pgfqpoint{4.907765in}{2.666253in}}{\pgfqpoint{4.898556in}{2.675461in}}%
\pgfpathcurveto{\pgfqpoint{4.889348in}{2.684670in}}{\pgfqpoint{4.876857in}{2.689844in}}{\pgfqpoint{4.863834in}{2.689844in}}%
\pgfpathcurveto{\pgfqpoint{4.850811in}{2.689844in}}{\pgfqpoint{4.838320in}{2.684670in}}{\pgfqpoint{4.829112in}{2.675461in}}%
\pgfpathcurveto{\pgfqpoint{4.819903in}{2.666253in}}{\pgfqpoint{4.814729in}{2.653762in}}{\pgfqpoint{4.814729in}{2.640739in}}%
\pgfpathcurveto{\pgfqpoint{4.814729in}{2.627716in}}{\pgfqpoint{4.819903in}{2.615225in}}{\pgfqpoint{4.829112in}{2.606017in}}%
\pgfpathcurveto{\pgfqpoint{4.838320in}{2.596808in}}{\pgfqpoint{4.850811in}{2.591634in}}{\pgfqpoint{4.863834in}{2.591634in}}%
\pgfpathlineto{\pgfqpoint{4.863834in}{2.591634in}}%
\pgfpathclose%
\pgfusepath{stroke,fill}%
\end{pgfscope}%
\begin{pgfscope}%
\pgfpathrectangle{\pgfqpoint{0.786164in}{0.768110in}}{\pgfqpoint{8.851069in}{7.081890in}}%
\pgfusepath{clip}%
\pgfsetbuttcap%
\pgfsetroundjoin%
\definecolor{currentfill}{rgb}{0.283197,0.115680,0.436115}%
\pgfsetfillcolor{currentfill}%
\pgfsetfillopacity{0.700000}%
\pgfsetlinewidth{0.501875pt}%
\definecolor{currentstroke}{rgb}{1.000000,1.000000,1.000000}%
\pgfsetstrokecolor{currentstroke}%
\pgfsetstrokeopacity{0.700000}%
\pgfsetdash{}{0pt}%
\pgfpathmoveto{\pgfqpoint{4.915116in}{2.506266in}}%
\pgfpathcurveto{\pgfqpoint{4.928139in}{2.506266in}}{\pgfqpoint{4.940630in}{2.511440in}}{\pgfqpoint{4.949838in}{2.520648in}}%
\pgfpathcurveto{\pgfqpoint{4.959047in}{2.529856in}}{\pgfqpoint{4.964221in}{2.542347in}}{\pgfqpoint{4.964221in}{2.555370in}}%
\pgfpathcurveto{\pgfqpoint{4.964221in}{2.568393in}}{\pgfqpoint{4.959047in}{2.580884in}}{\pgfqpoint{4.949838in}{2.590092in}}%
\pgfpathcurveto{\pgfqpoint{4.940630in}{2.599301in}}{\pgfqpoint{4.928139in}{2.604475in}}{\pgfqpoint{4.915116in}{2.604475in}}%
\pgfpathcurveto{\pgfqpoint{4.902093in}{2.604475in}}{\pgfqpoint{4.889602in}{2.599301in}}{\pgfqpoint{4.880394in}{2.590092in}}%
\pgfpathcurveto{\pgfqpoint{4.871185in}{2.580884in}}{\pgfqpoint{4.866011in}{2.568393in}}{\pgfqpoint{4.866011in}{2.555370in}}%
\pgfpathcurveto{\pgfqpoint{4.866011in}{2.542347in}}{\pgfqpoint{4.871185in}{2.529856in}}{\pgfqpoint{4.880394in}{2.520648in}}%
\pgfpathcurveto{\pgfqpoint{4.889602in}{2.511440in}}{\pgfqpoint{4.902093in}{2.506266in}}{\pgfqpoint{4.915116in}{2.506266in}}%
\pgfpathlineto{\pgfqpoint{4.915116in}{2.506266in}}%
\pgfpathclose%
\pgfusepath{stroke,fill}%
\end{pgfscope}%
\begin{pgfscope}%
\pgfpathrectangle{\pgfqpoint{0.786164in}{0.768110in}}{\pgfqpoint{8.851069in}{7.081890in}}%
\pgfusepath{clip}%
\pgfsetbuttcap%
\pgfsetroundjoin%
\definecolor{currentfill}{rgb}{0.280894,0.078907,0.402329}%
\pgfsetfillcolor{currentfill}%
\pgfsetfillopacity{0.700000}%
\pgfsetlinewidth{0.501875pt}%
\definecolor{currentstroke}{rgb}{1.000000,1.000000,1.000000}%
\pgfsetstrokecolor{currentstroke}%
\pgfsetstrokeopacity{0.700000}%
\pgfsetdash{}{0pt}%
\pgfpathmoveto{\pgfqpoint{5.325252in}{2.314186in}}%
\pgfpathcurveto{\pgfqpoint{5.338275in}{2.314186in}}{\pgfqpoint{5.350766in}{2.319360in}}{\pgfqpoint{5.359974in}{2.328568in}}%
\pgfpathcurveto{\pgfqpoint{5.369183in}{2.337777in}}{\pgfqpoint{5.374356in}{2.350268in}}{\pgfqpoint{5.374356in}{2.363291in}}%
\pgfpathcurveto{\pgfqpoint{5.374356in}{2.376313in}}{\pgfqpoint{5.369183in}{2.388804in}}{\pgfqpoint{5.359974in}{2.398013in}}%
\pgfpathcurveto{\pgfqpoint{5.350766in}{2.407221in}}{\pgfqpoint{5.338275in}{2.412395in}}{\pgfqpoint{5.325252in}{2.412395in}}%
\pgfpathcurveto{\pgfqpoint{5.312229in}{2.412395in}}{\pgfqpoint{5.299738in}{2.407221in}}{\pgfqpoint{5.290530in}{2.398013in}}%
\pgfpathcurveto{\pgfqpoint{5.281321in}{2.388804in}}{\pgfqpoint{5.276147in}{2.376313in}}{\pgfqpoint{5.276147in}{2.363291in}}%
\pgfpathcurveto{\pgfqpoint{5.276147in}{2.350268in}}{\pgfqpoint{5.281321in}{2.337777in}}{\pgfqpoint{5.290530in}{2.328568in}}%
\pgfpathcurveto{\pgfqpoint{5.299738in}{2.319360in}}{\pgfqpoint{5.312229in}{2.314186in}}{\pgfqpoint{5.325252in}{2.314186in}}%
\pgfpathlineto{\pgfqpoint{5.325252in}{2.314186in}}%
\pgfpathclose%
\pgfusepath{stroke,fill}%
\end{pgfscope}%
\begin{pgfscope}%
\pgfpathrectangle{\pgfqpoint{0.786164in}{0.768110in}}{\pgfqpoint{8.851069in}{7.081890in}}%
\pgfusepath{clip}%
\pgfsetbuttcap%
\pgfsetroundjoin%
\definecolor{currentfill}{rgb}{0.280267,0.073417,0.397163}%
\pgfsetfillcolor{currentfill}%
\pgfsetfillopacity{0.700000}%
\pgfsetlinewidth{0.501875pt}%
\definecolor{currentstroke}{rgb}{1.000000,1.000000,1.000000}%
\pgfsetstrokecolor{currentstroke}%
\pgfsetstrokeopacity{0.700000}%
\pgfsetdash{}{0pt}%
\pgfpathmoveto{\pgfqpoint{5.325252in}{2.250159in}}%
\pgfpathcurveto{\pgfqpoint{5.338275in}{2.250159in}}{\pgfqpoint{5.350766in}{2.255333in}}{\pgfqpoint{5.359974in}{2.264542in}}%
\pgfpathcurveto{\pgfqpoint{5.369183in}{2.273750in}}{\pgfqpoint{5.374356in}{2.286241in}}{\pgfqpoint{5.374356in}{2.299264in}}%
\pgfpathcurveto{\pgfqpoint{5.374356in}{2.312287in}}{\pgfqpoint{5.369183in}{2.324778in}}{\pgfqpoint{5.359974in}{2.333986in}}%
\pgfpathcurveto{\pgfqpoint{5.350766in}{2.343195in}}{\pgfqpoint{5.338275in}{2.348369in}}{\pgfqpoint{5.325252in}{2.348369in}}%
\pgfpathcurveto{\pgfqpoint{5.312229in}{2.348369in}}{\pgfqpoint{5.299738in}{2.343195in}}{\pgfqpoint{5.290530in}{2.333986in}}%
\pgfpathcurveto{\pgfqpoint{5.281321in}{2.324778in}}{\pgfqpoint{5.276147in}{2.312287in}}{\pgfqpoint{5.276147in}{2.299264in}}%
\pgfpathcurveto{\pgfqpoint{5.276147in}{2.286241in}}{\pgfqpoint{5.281321in}{2.273750in}}{\pgfqpoint{5.290530in}{2.264542in}}%
\pgfpathcurveto{\pgfqpoint{5.299738in}{2.255333in}}{\pgfqpoint{5.312229in}{2.250159in}}{\pgfqpoint{5.325252in}{2.250159in}}%
\pgfpathlineto{\pgfqpoint{5.325252in}{2.250159in}}%
\pgfpathclose%
\pgfusepath{stroke,fill}%
\end{pgfscope}%
\begin{pgfscope}%
\pgfpathrectangle{\pgfqpoint{0.786164in}{0.768110in}}{\pgfqpoint{8.851069in}{7.081890in}}%
\pgfusepath{clip}%
\pgfsetbuttcap%
\pgfsetroundjoin%
\definecolor{currentfill}{rgb}{0.282327,0.094955,0.417331}%
\pgfsetfillcolor{currentfill}%
\pgfsetfillopacity{0.700000}%
\pgfsetlinewidth{0.501875pt}%
\definecolor{currentstroke}{rgb}{1.000000,1.000000,1.000000}%
\pgfsetstrokecolor{currentstroke}%
\pgfsetstrokeopacity{0.700000}%
\pgfsetdash{}{0pt}%
\pgfpathmoveto{\pgfqpoint{5.410112in}{2.314186in}}%
\pgfpathcurveto{\pgfqpoint{5.423134in}{2.314186in}}{\pgfqpoint{5.435626in}{2.319360in}}{\pgfqpoint{5.444834in}{2.328568in}}%
\pgfpathcurveto{\pgfqpoint{5.454042in}{2.337777in}}{\pgfqpoint{5.459216in}{2.350268in}}{\pgfqpoint{5.459216in}{2.363291in}}%
\pgfpathcurveto{\pgfqpoint{5.459216in}{2.376313in}}{\pgfqpoint{5.454042in}{2.388804in}}{\pgfqpoint{5.444834in}{2.398013in}}%
\pgfpathcurveto{\pgfqpoint{5.435626in}{2.407221in}}{\pgfqpoint{5.423134in}{2.412395in}}{\pgfqpoint{5.410112in}{2.412395in}}%
\pgfpathcurveto{\pgfqpoint{5.397089in}{2.412395in}}{\pgfqpoint{5.384598in}{2.407221in}}{\pgfqpoint{5.375390in}{2.398013in}}%
\pgfpathcurveto{\pgfqpoint{5.366181in}{2.388804in}}{\pgfqpoint{5.361007in}{2.376313in}}{\pgfqpoint{5.361007in}{2.363291in}}%
\pgfpathcurveto{\pgfqpoint{5.361007in}{2.350268in}}{\pgfqpoint{5.366181in}{2.337777in}}{\pgfqpoint{5.375390in}{2.328568in}}%
\pgfpathcurveto{\pgfqpoint{5.384598in}{2.319360in}}{\pgfqpoint{5.397089in}{2.314186in}}{\pgfqpoint{5.410112in}{2.314186in}}%
\pgfpathlineto{\pgfqpoint{5.410112in}{2.314186in}}%
\pgfpathclose%
\pgfusepath{stroke,fill}%
\end{pgfscope}%
\begin{pgfscope}%
\pgfpathrectangle{\pgfqpoint{0.786164in}{0.768110in}}{\pgfqpoint{8.851069in}{7.081890in}}%
\pgfusepath{clip}%
\pgfsetbuttcap%
\pgfsetroundjoin%
\definecolor{currentfill}{rgb}{0.277018,0.050344,0.375715}%
\pgfsetfillcolor{currentfill}%
\pgfsetfillopacity{0.700000}%
\pgfsetlinewidth{0.501875pt}%
\definecolor{currentstroke}{rgb}{1.000000,1.000000,1.000000}%
\pgfsetstrokecolor{currentstroke}%
\pgfsetstrokeopacity{0.700000}%
\pgfsetdash{}{0pt}%
\pgfpathmoveto{\pgfqpoint{3.228663in}{2.399555in}}%
\pgfpathcurveto{\pgfqpoint{3.241686in}{2.399555in}}{\pgfqpoint{3.254177in}{2.404729in}}{\pgfqpoint{3.263386in}{2.413937in}}%
\pgfpathcurveto{\pgfqpoint{3.272594in}{2.423146in}}{\pgfqpoint{3.277768in}{2.435637in}}{\pgfqpoint{3.277768in}{2.448659in}}%
\pgfpathcurveto{\pgfqpoint{3.277768in}{2.461682in}}{\pgfqpoint{3.272594in}{2.474173in}}{\pgfqpoint{3.263386in}{2.483382in}}%
\pgfpathcurveto{\pgfqpoint{3.254177in}{2.492590in}}{\pgfqpoint{3.241686in}{2.497764in}}{\pgfqpoint{3.228663in}{2.497764in}}%
\pgfpathcurveto{\pgfqpoint{3.215641in}{2.497764in}}{\pgfqpoint{3.203150in}{2.492590in}}{\pgfqpoint{3.193941in}{2.483382in}}%
\pgfpathcurveto{\pgfqpoint{3.184733in}{2.474173in}}{\pgfqpoint{3.179559in}{2.461682in}}{\pgfqpoint{3.179559in}{2.448659in}}%
\pgfpathcurveto{\pgfqpoint{3.179559in}{2.435637in}}{\pgfqpoint{3.184733in}{2.423146in}}{\pgfqpoint{3.193941in}{2.413937in}}%
\pgfpathcurveto{\pgfqpoint{3.203150in}{2.404729in}}{\pgfqpoint{3.215641in}{2.399555in}}{\pgfqpoint{3.228663in}{2.399555in}}%
\pgfpathlineto{\pgfqpoint{3.228663in}{2.399555in}}%
\pgfpathclose%
\pgfusepath{stroke,fill}%
\end{pgfscope}%
\begin{pgfscope}%
\pgfpathrectangle{\pgfqpoint{0.786164in}{0.768110in}}{\pgfqpoint{8.851069in}{7.081890in}}%
\pgfusepath{clip}%
\pgfsetbuttcap%
\pgfsetroundjoin%
\definecolor{currentfill}{rgb}{0.276022,0.044167,0.370164}%
\pgfsetfillcolor{currentfill}%
\pgfsetfillopacity{0.700000}%
\pgfsetlinewidth{0.501875pt}%
\definecolor{currentstroke}{rgb}{1.000000,1.000000,1.000000}%
\pgfsetstrokecolor{currentstroke}%
\pgfsetstrokeopacity{0.700000}%
\pgfsetdash{}{0pt}%
\pgfpathmoveto{\pgfqpoint{3.321460in}{2.378213in}}%
\pgfpathcurveto{\pgfqpoint{3.334482in}{2.378213in}}{\pgfqpoint{3.346973in}{2.383386in}}{\pgfqpoint{3.356182in}{2.392595in}}%
\pgfpathcurveto{\pgfqpoint{3.365390in}{2.401803in}}{\pgfqpoint{3.370564in}{2.414294in}}{\pgfqpoint{3.370564in}{2.427317in}}%
\pgfpathcurveto{\pgfqpoint{3.370564in}{2.440340in}}{\pgfqpoint{3.365390in}{2.452831in}}{\pgfqpoint{3.356182in}{2.462039in}}%
\pgfpathcurveto{\pgfqpoint{3.346973in}{2.471248in}}{\pgfqpoint{3.334482in}{2.476422in}}{\pgfqpoint{3.321460in}{2.476422in}}%
\pgfpathcurveto{\pgfqpoint{3.308437in}{2.476422in}}{\pgfqpoint{3.295946in}{2.471248in}}{\pgfqpoint{3.286737in}{2.462039in}}%
\pgfpathcurveto{\pgfqpoint{3.277529in}{2.452831in}}{\pgfqpoint{3.272355in}{2.440340in}}{\pgfqpoint{3.272355in}{2.427317in}}%
\pgfpathcurveto{\pgfqpoint{3.272355in}{2.414294in}}{\pgfqpoint{3.277529in}{2.401803in}}{\pgfqpoint{3.286737in}{2.392595in}}%
\pgfpathcurveto{\pgfqpoint{3.295946in}{2.383386in}}{\pgfqpoint{3.308437in}{2.378213in}}{\pgfqpoint{3.321460in}{2.378213in}}%
\pgfpathlineto{\pgfqpoint{3.321460in}{2.378213in}}%
\pgfpathclose%
\pgfusepath{stroke,fill}%
\end{pgfscope}%
\begin{pgfscope}%
\pgfpathrectangle{\pgfqpoint{0.786164in}{0.768110in}}{\pgfqpoint{8.851069in}{7.081890in}}%
\pgfusepath{clip}%
\pgfsetbuttcap%
\pgfsetroundjoin%
\definecolor{currentfill}{rgb}{0.277941,0.056324,0.381191}%
\pgfsetfillcolor{currentfill}%
\pgfsetfillopacity{0.700000}%
\pgfsetlinewidth{0.501875pt}%
\definecolor{currentstroke}{rgb}{1.000000,1.000000,1.000000}%
\pgfsetstrokecolor{currentstroke}%
\pgfsetstrokeopacity{0.700000}%
\pgfsetdash{}{0pt}%
\pgfpathmoveto{\pgfqpoint{3.263462in}{2.420897in}}%
\pgfpathcurveto{\pgfqpoint{3.276485in}{2.420897in}}{\pgfqpoint{3.288976in}{2.426071in}}{\pgfqpoint{3.298184in}{2.435279in}}%
\pgfpathcurveto{\pgfqpoint{3.307393in}{2.444488in}}{\pgfqpoint{3.312567in}{2.456979in}}{\pgfqpoint{3.312567in}{2.470002in}}%
\pgfpathcurveto{\pgfqpoint{3.312567in}{2.483024in}}{\pgfqpoint{3.307393in}{2.495515in}}{\pgfqpoint{3.298184in}{2.504724in}}%
\pgfpathcurveto{\pgfqpoint{3.288976in}{2.513932in}}{\pgfqpoint{3.276485in}{2.519106in}}{\pgfqpoint{3.263462in}{2.519106in}}%
\pgfpathcurveto{\pgfqpoint{3.250439in}{2.519106in}}{\pgfqpoint{3.237948in}{2.513932in}}{\pgfqpoint{3.228740in}{2.504724in}}%
\pgfpathcurveto{\pgfqpoint{3.219531in}{2.495515in}}{\pgfqpoint{3.214357in}{2.483024in}}{\pgfqpoint{3.214357in}{2.470002in}}%
\pgfpathcurveto{\pgfqpoint{3.214357in}{2.456979in}}{\pgfqpoint{3.219531in}{2.444488in}}{\pgfqpoint{3.228740in}{2.435279in}}%
\pgfpathcurveto{\pgfqpoint{3.237948in}{2.426071in}}{\pgfqpoint{3.250439in}{2.420897in}}{\pgfqpoint{3.263462in}{2.420897in}}%
\pgfpathlineto{\pgfqpoint{3.263462in}{2.420897in}}%
\pgfpathclose%
\pgfusepath{stroke,fill}%
\end{pgfscope}%
\begin{pgfscope}%
\pgfpathrectangle{\pgfqpoint{0.786164in}{0.768110in}}{\pgfqpoint{8.851069in}{7.081890in}}%
\pgfusepath{clip}%
\pgfsetbuttcap%
\pgfsetroundjoin%
\definecolor{currentfill}{rgb}{0.278791,0.062145,0.386592}%
\pgfsetfillcolor{currentfill}%
\pgfsetfillopacity{0.700000}%
\pgfsetlinewidth{0.501875pt}%
\definecolor{currentstroke}{rgb}{1.000000,1.000000,1.000000}%
\pgfsetstrokecolor{currentstroke}%
\pgfsetstrokeopacity{0.700000}%
\pgfsetdash{}{0pt}%
\pgfpathmoveto{\pgfqpoint{3.397650in}{2.442239in}}%
\pgfpathcurveto{\pgfqpoint{3.410673in}{2.442239in}}{\pgfqpoint{3.423164in}{2.447413in}}{\pgfqpoint{3.432373in}{2.456621in}}%
\pgfpathcurveto{\pgfqpoint{3.441581in}{2.465830in}}{\pgfqpoint{3.446755in}{2.478321in}}{\pgfqpoint{3.446755in}{2.491344in}}%
\pgfpathcurveto{\pgfqpoint{3.446755in}{2.504366in}}{\pgfqpoint{3.441581in}{2.516857in}}{\pgfqpoint{3.432373in}{2.526066in}}%
\pgfpathcurveto{\pgfqpoint{3.423164in}{2.535274in}}{\pgfqpoint{3.410673in}{2.540448in}}{\pgfqpoint{3.397650in}{2.540448in}}%
\pgfpathcurveto{\pgfqpoint{3.384628in}{2.540448in}}{\pgfqpoint{3.372137in}{2.535274in}}{\pgfqpoint{3.362928in}{2.526066in}}%
\pgfpathcurveto{\pgfqpoint{3.353720in}{2.516857in}}{\pgfqpoint{3.348546in}{2.504366in}}{\pgfqpoint{3.348546in}{2.491344in}}%
\pgfpathcurveto{\pgfqpoint{3.348546in}{2.478321in}}{\pgfqpoint{3.353720in}{2.465830in}}{\pgfqpoint{3.362928in}{2.456621in}}%
\pgfpathcurveto{\pgfqpoint{3.372137in}{2.447413in}}{\pgfqpoint{3.384628in}{2.442239in}}{\pgfqpoint{3.397650in}{2.442239in}}%
\pgfpathlineto{\pgfqpoint{3.397650in}{2.442239in}}%
\pgfpathclose%
\pgfusepath{stroke,fill}%
\end{pgfscope}%
\begin{pgfscope}%
\pgfpathrectangle{\pgfqpoint{0.786164in}{0.768110in}}{\pgfqpoint{8.851069in}{7.081890in}}%
\pgfusepath{clip}%
\pgfsetbuttcap%
\pgfsetroundjoin%
\definecolor{currentfill}{rgb}{0.279566,0.067836,0.391917}%
\pgfsetfillcolor{currentfill}%
\pgfsetfillopacity{0.700000}%
\pgfsetlinewidth{0.501875pt}%
\definecolor{currentstroke}{rgb}{1.000000,1.000000,1.000000}%
\pgfsetstrokecolor{currentstroke}%
\pgfsetstrokeopacity{0.700000}%
\pgfsetdash{}{0pt}%
\pgfpathmoveto{\pgfqpoint{3.642950in}{2.420897in}}%
\pgfpathcurveto{\pgfqpoint{3.655973in}{2.420897in}}{\pgfqpoint{3.668464in}{2.426071in}}{\pgfqpoint{3.677673in}{2.435279in}}%
\pgfpathcurveto{\pgfqpoint{3.686881in}{2.444488in}}{\pgfqpoint{3.692055in}{2.456979in}}{\pgfqpoint{3.692055in}{2.470002in}}%
\pgfpathcurveto{\pgfqpoint{3.692055in}{2.483024in}}{\pgfqpoint{3.686881in}{2.495515in}}{\pgfqpoint{3.677673in}{2.504724in}}%
\pgfpathcurveto{\pgfqpoint{3.668464in}{2.513932in}}{\pgfqpoint{3.655973in}{2.519106in}}{\pgfqpoint{3.642950in}{2.519106in}}%
\pgfpathcurveto{\pgfqpoint{3.629928in}{2.519106in}}{\pgfqpoint{3.617437in}{2.513932in}}{\pgfqpoint{3.608228in}{2.504724in}}%
\pgfpathcurveto{\pgfqpoint{3.599020in}{2.495515in}}{\pgfqpoint{3.593846in}{2.483024in}}{\pgfqpoint{3.593846in}{2.470002in}}%
\pgfpathcurveto{\pgfqpoint{3.593846in}{2.456979in}}{\pgfqpoint{3.599020in}{2.444488in}}{\pgfqpoint{3.608228in}{2.435279in}}%
\pgfpathcurveto{\pgfqpoint{3.617437in}{2.426071in}}{\pgfqpoint{3.629928in}{2.420897in}}{\pgfqpoint{3.642950in}{2.420897in}}%
\pgfpathlineto{\pgfqpoint{3.642950in}{2.420897in}}%
\pgfpathclose%
\pgfusepath{stroke,fill}%
\end{pgfscope}%
\begin{pgfscope}%
\pgfpathrectangle{\pgfqpoint{0.786164in}{0.768110in}}{\pgfqpoint{8.851069in}{7.081890in}}%
\pgfusepath{clip}%
\pgfsetbuttcap%
\pgfsetroundjoin%
\definecolor{currentfill}{rgb}{0.272594,0.025563,0.353093}%
\pgfsetfillcolor{currentfill}%
\pgfsetfillopacity{0.700000}%
\pgfsetlinewidth{0.501875pt}%
\definecolor{currentstroke}{rgb}{1.000000,1.000000,1.000000}%
\pgfsetstrokecolor{currentstroke}%
\pgfsetstrokeopacity{0.700000}%
\pgfsetdash{}{0pt}%
\pgfpathmoveto{\pgfqpoint{3.881413in}{2.250159in}}%
\pgfpathcurveto{\pgfqpoint{3.894436in}{2.250159in}}{\pgfqpoint{3.906927in}{2.255333in}}{\pgfqpoint{3.916135in}{2.264542in}}%
\pgfpathcurveto{\pgfqpoint{3.925344in}{2.273750in}}{\pgfqpoint{3.930517in}{2.286241in}}{\pgfqpoint{3.930517in}{2.299264in}}%
\pgfpathcurveto{\pgfqpoint{3.930517in}{2.312287in}}{\pgfqpoint{3.925344in}{2.324778in}}{\pgfqpoint{3.916135in}{2.333986in}}%
\pgfpathcurveto{\pgfqpoint{3.906927in}{2.343195in}}{\pgfqpoint{3.894436in}{2.348369in}}{\pgfqpoint{3.881413in}{2.348369in}}%
\pgfpathcurveto{\pgfqpoint{3.868390in}{2.348369in}}{\pgfqpoint{3.855899in}{2.343195in}}{\pgfqpoint{3.846691in}{2.333986in}}%
\pgfpathcurveto{\pgfqpoint{3.837482in}{2.324778in}}{\pgfqpoint{3.832308in}{2.312287in}}{\pgfqpoint{3.832308in}{2.299264in}}%
\pgfpathcurveto{\pgfqpoint{3.832308in}{2.286241in}}{\pgfqpoint{3.837482in}{2.273750in}}{\pgfqpoint{3.846691in}{2.264542in}}%
\pgfpathcurveto{\pgfqpoint{3.855899in}{2.255333in}}{\pgfqpoint{3.868390in}{2.250159in}}{\pgfqpoint{3.881413in}{2.250159in}}%
\pgfpathlineto{\pgfqpoint{3.881413in}{2.250159in}}%
\pgfpathclose%
\pgfusepath{stroke,fill}%
\end{pgfscope}%
\begin{pgfscope}%
\pgfpathrectangle{\pgfqpoint{0.786164in}{0.768110in}}{\pgfqpoint{8.851069in}{7.081890in}}%
\pgfusepath{clip}%
\pgfsetbuttcap%
\pgfsetroundjoin%
\definecolor{currentfill}{rgb}{0.273809,0.031497,0.358853}%
\pgfsetfillcolor{currentfill}%
\pgfsetfillopacity{0.700000}%
\pgfsetlinewidth{0.501875pt}%
\definecolor{currentstroke}{rgb}{1.000000,1.000000,1.000000}%
\pgfsetstrokecolor{currentstroke}%
\pgfsetstrokeopacity{0.700000}%
\pgfsetdash{}{0pt}%
\pgfpathmoveto{\pgfqpoint{3.964075in}{2.143449in}}%
\pgfpathcurveto{\pgfqpoint{3.977098in}{2.143449in}}{\pgfqpoint{3.989589in}{2.148623in}}{\pgfqpoint{3.998797in}{2.157831in}}%
\pgfpathcurveto{\pgfqpoint{4.008006in}{2.167040in}}{\pgfqpoint{4.013180in}{2.179531in}}{\pgfqpoint{4.013180in}{2.192553in}}%
\pgfpathcurveto{\pgfqpoint{4.013180in}{2.205576in}}{\pgfqpoint{4.008006in}{2.218067in}}{\pgfqpoint{3.998797in}{2.227276in}}%
\pgfpathcurveto{\pgfqpoint{3.989589in}{2.236484in}}{\pgfqpoint{3.977098in}{2.241658in}}{\pgfqpoint{3.964075in}{2.241658in}}%
\pgfpathcurveto{\pgfqpoint{3.951052in}{2.241658in}}{\pgfqpoint{3.938561in}{2.236484in}}{\pgfqpoint{3.929353in}{2.227276in}}%
\pgfpathcurveto{\pgfqpoint{3.920144in}{2.218067in}}{\pgfqpoint{3.914970in}{2.205576in}}{\pgfqpoint{3.914970in}{2.192553in}}%
\pgfpathcurveto{\pgfqpoint{3.914970in}{2.179531in}}{\pgfqpoint{3.920144in}{2.167040in}}{\pgfqpoint{3.929353in}{2.157831in}}%
\pgfpathcurveto{\pgfqpoint{3.938561in}{2.148623in}}{\pgfqpoint{3.951052in}{2.143449in}}{\pgfqpoint{3.964075in}{2.143449in}}%
\pgfpathlineto{\pgfqpoint{3.964075in}{2.143449in}}%
\pgfpathclose%
\pgfusepath{stroke,fill}%
\end{pgfscope}%
\begin{pgfscope}%
\pgfpathrectangle{\pgfqpoint{0.786164in}{0.768110in}}{\pgfqpoint{8.851069in}{7.081890in}}%
\pgfusepath{clip}%
\pgfsetbuttcap%
\pgfsetroundjoin%
\definecolor{currentfill}{rgb}{0.274952,0.037752,0.364543}%
\pgfsetfillcolor{currentfill}%
\pgfsetfillopacity{0.700000}%
\pgfsetlinewidth{0.501875pt}%
\definecolor{currentstroke}{rgb}{1.000000,1.000000,1.000000}%
\pgfsetstrokecolor{currentstroke}%
\pgfsetstrokeopacity{0.700000}%
\pgfsetdash{}{0pt}%
\pgfpathmoveto{\pgfqpoint{3.830863in}{2.207475in}}%
\pgfpathcurveto{\pgfqpoint{3.843886in}{2.207475in}}{\pgfqpoint{3.856377in}{2.212649in}}{\pgfqpoint{3.865585in}{2.221858in}}%
\pgfpathcurveto{\pgfqpoint{3.874794in}{2.231066in}}{\pgfqpoint{3.879968in}{2.243557in}}{\pgfqpoint{3.879968in}{2.256580in}}%
\pgfpathcurveto{\pgfqpoint{3.879968in}{2.269602in}}{\pgfqpoint{3.874794in}{2.282094in}}{\pgfqpoint{3.865585in}{2.291302in}}%
\pgfpathcurveto{\pgfqpoint{3.856377in}{2.300510in}}{\pgfqpoint{3.843886in}{2.305684in}}{\pgfqpoint{3.830863in}{2.305684in}}%
\pgfpathcurveto{\pgfqpoint{3.817841in}{2.305684in}}{\pgfqpoint{3.805349in}{2.300510in}}{\pgfqpoint{3.796141in}{2.291302in}}%
\pgfpathcurveto{\pgfqpoint{3.786933in}{2.282094in}}{\pgfqpoint{3.781759in}{2.269602in}}{\pgfqpoint{3.781759in}{2.256580in}}%
\pgfpathcurveto{\pgfqpoint{3.781759in}{2.243557in}}{\pgfqpoint{3.786933in}{2.231066in}}{\pgfqpoint{3.796141in}{2.221858in}}%
\pgfpathcurveto{\pgfqpoint{3.805349in}{2.212649in}}{\pgfqpoint{3.817841in}{2.207475in}}{\pgfqpoint{3.830863in}{2.207475in}}%
\pgfpathlineto{\pgfqpoint{3.830863in}{2.207475in}}%
\pgfpathclose%
\pgfusepath{stroke,fill}%
\end{pgfscope}%
\begin{pgfscope}%
\pgfpathrectangle{\pgfqpoint{0.786164in}{0.768110in}}{\pgfqpoint{8.851069in}{7.081890in}}%
\pgfusepath{clip}%
\pgfsetbuttcap%
\pgfsetroundjoin%
\definecolor{currentfill}{rgb}{0.274952,0.037752,0.364543}%
\pgfsetfillcolor{currentfill}%
\pgfsetfillopacity{0.700000}%
\pgfsetlinewidth{0.501875pt}%
\definecolor{currentstroke}{rgb}{1.000000,1.000000,1.000000}%
\pgfsetstrokecolor{currentstroke}%
\pgfsetstrokeopacity{0.700000}%
\pgfsetdash{}{0pt}%
\pgfpathmoveto{\pgfqpoint{3.962976in}{2.164791in}}%
\pgfpathcurveto{\pgfqpoint{3.975999in}{2.164791in}}{\pgfqpoint{3.988490in}{2.169965in}}{\pgfqpoint{3.997698in}{2.179173in}}%
\pgfpathcurveto{\pgfqpoint{4.006907in}{2.188382in}}{\pgfqpoint{4.012081in}{2.200873in}}{\pgfqpoint{4.012081in}{2.213895in}}%
\pgfpathcurveto{\pgfqpoint{4.012081in}{2.226918in}}{\pgfqpoint{4.006907in}{2.239409in}}{\pgfqpoint{3.997698in}{2.248618in}}%
\pgfpathcurveto{\pgfqpoint{3.988490in}{2.257826in}}{\pgfqpoint{3.975999in}{2.263000in}}{\pgfqpoint{3.962976in}{2.263000in}}%
\pgfpathcurveto{\pgfqpoint{3.949953in}{2.263000in}}{\pgfqpoint{3.937462in}{2.257826in}}{\pgfqpoint{3.928254in}{2.248618in}}%
\pgfpathcurveto{\pgfqpoint{3.919045in}{2.239409in}}{\pgfqpoint{3.913871in}{2.226918in}}{\pgfqpoint{3.913871in}{2.213895in}}%
\pgfpathcurveto{\pgfqpoint{3.913871in}{2.200873in}}{\pgfqpoint{3.919045in}{2.188382in}}{\pgfqpoint{3.928254in}{2.179173in}}%
\pgfpathcurveto{\pgfqpoint{3.937462in}{2.169965in}}{\pgfqpoint{3.949953in}{2.164791in}}{\pgfqpoint{3.962976in}{2.164791in}}%
\pgfpathlineto{\pgfqpoint{3.962976in}{2.164791in}}%
\pgfpathclose%
\pgfusepath{stroke,fill}%
\end{pgfscope}%
\begin{pgfscope}%
\pgfpathrectangle{\pgfqpoint{0.786164in}{0.768110in}}{\pgfqpoint{8.851069in}{7.081890in}}%
\pgfusepath{clip}%
\pgfsetbuttcap%
\pgfsetroundjoin%
\definecolor{currentfill}{rgb}{0.269944,0.014625,0.341379}%
\pgfsetfillcolor{currentfill}%
\pgfsetfillopacity{0.700000}%
\pgfsetlinewidth{0.501875pt}%
\definecolor{currentstroke}{rgb}{1.000000,1.000000,1.000000}%
\pgfsetstrokecolor{currentstroke}%
\pgfsetstrokeopacity{0.700000}%
\pgfsetdash{}{0pt}%
\pgfpathmoveto{\pgfqpoint{4.092525in}{2.036738in}}%
\pgfpathcurveto{\pgfqpoint{4.105547in}{2.036738in}}{\pgfqpoint{4.118038in}{2.041912in}}{\pgfqpoint{4.127247in}{2.051120in}}%
\pgfpathcurveto{\pgfqpoint{4.136455in}{2.060329in}}{\pgfqpoint{4.141629in}{2.072820in}}{\pgfqpoint{4.141629in}{2.085842in}}%
\pgfpathcurveto{\pgfqpoint{4.141629in}{2.098865in}}{\pgfqpoint{4.136455in}{2.111356in}}{\pgfqpoint{4.127247in}{2.120565in}}%
\pgfpathcurveto{\pgfqpoint{4.118038in}{2.129773in}}{\pgfqpoint{4.105547in}{2.134947in}}{\pgfqpoint{4.092525in}{2.134947in}}%
\pgfpathcurveto{\pgfqpoint{4.079502in}{2.134947in}}{\pgfqpoint{4.067011in}{2.129773in}}{\pgfqpoint{4.057802in}{2.120565in}}%
\pgfpathcurveto{\pgfqpoint{4.048594in}{2.111356in}}{\pgfqpoint{4.043420in}{2.098865in}}{\pgfqpoint{4.043420in}{2.085842in}}%
\pgfpathcurveto{\pgfqpoint{4.043420in}{2.072820in}}{\pgfqpoint{4.048594in}{2.060329in}}{\pgfqpoint{4.057802in}{2.051120in}}%
\pgfpathcurveto{\pgfqpoint{4.067011in}{2.041912in}}{\pgfqpoint{4.079502in}{2.036738in}}{\pgfqpoint{4.092525in}{2.036738in}}%
\pgfpathlineto{\pgfqpoint{4.092525in}{2.036738in}}%
\pgfpathclose%
\pgfusepath{stroke,fill}%
\end{pgfscope}%
\begin{pgfscope}%
\pgfpathrectangle{\pgfqpoint{0.786164in}{0.768110in}}{\pgfqpoint{8.851069in}{7.081890in}}%
\pgfusepath{clip}%
\pgfsetbuttcap%
\pgfsetroundjoin%
\definecolor{currentfill}{rgb}{0.269944,0.014625,0.341379}%
\pgfsetfillcolor{currentfill}%
\pgfsetfillopacity{0.700000}%
\pgfsetlinewidth{0.501875pt}%
\definecolor{currentstroke}{rgb}{1.000000,1.000000,1.000000}%
\pgfsetstrokecolor{currentstroke}%
\pgfsetstrokeopacity{0.700000}%
\pgfsetdash{}{0pt}%
\pgfpathmoveto{\pgfqpoint{4.209619in}{1.930027in}}%
\pgfpathcurveto{\pgfqpoint{4.222642in}{1.930027in}}{\pgfqpoint{4.235133in}{1.935201in}}{\pgfqpoint{4.244341in}{1.944409in}}%
\pgfpathcurveto{\pgfqpoint{4.253550in}{1.953618in}}{\pgfqpoint{4.258724in}{1.966109in}}{\pgfqpoint{4.258724in}{1.979132in}}%
\pgfpathcurveto{\pgfqpoint{4.258724in}{1.992154in}}{\pgfqpoint{4.253550in}{2.004645in}}{\pgfqpoint{4.244341in}{2.013854in}}%
\pgfpathcurveto{\pgfqpoint{4.235133in}{2.023062in}}{\pgfqpoint{4.222642in}{2.028236in}}{\pgfqpoint{4.209619in}{2.028236in}}%
\pgfpathcurveto{\pgfqpoint{4.196596in}{2.028236in}}{\pgfqpoint{4.184105in}{2.023062in}}{\pgfqpoint{4.174897in}{2.013854in}}%
\pgfpathcurveto{\pgfqpoint{4.165688in}{2.004645in}}{\pgfqpoint{4.160514in}{1.992154in}}{\pgfqpoint{4.160514in}{1.979132in}}%
\pgfpathcurveto{\pgfqpoint{4.160514in}{1.966109in}}{\pgfqpoint{4.165688in}{1.953618in}}{\pgfqpoint{4.174897in}{1.944409in}}%
\pgfpathcurveto{\pgfqpoint{4.184105in}{1.935201in}}{\pgfqpoint{4.196596in}{1.930027in}}{\pgfqpoint{4.209619in}{1.930027in}}%
\pgfpathlineto{\pgfqpoint{4.209619in}{1.930027in}}%
\pgfpathclose%
\pgfusepath{stroke,fill}%
\end{pgfscope}%
\begin{pgfscope}%
\pgfpathrectangle{\pgfqpoint{0.786164in}{0.768110in}}{\pgfqpoint{8.851069in}{7.081890in}}%
\pgfusepath{clip}%
\pgfsetbuttcap%
\pgfsetroundjoin%
\definecolor{currentfill}{rgb}{0.271305,0.019942,0.347269}%
\pgfsetfillcolor{currentfill}%
\pgfsetfillopacity{0.700000}%
\pgfsetlinewidth{0.501875pt}%
\definecolor{currentstroke}{rgb}{1.000000,1.000000,1.000000}%
\pgfsetstrokecolor{currentstroke}%
\pgfsetstrokeopacity{0.700000}%
\pgfsetdash{}{0pt}%
\pgfpathmoveto{\pgfqpoint{4.202293in}{1.930027in}}%
\pgfpathcurveto{\pgfqpoint{4.215316in}{1.930027in}}{\pgfqpoint{4.227807in}{1.935201in}}{\pgfqpoint{4.237015in}{1.944409in}}%
\pgfpathcurveto{\pgfqpoint{4.246224in}{1.953618in}}{\pgfqpoint{4.251398in}{1.966109in}}{\pgfqpoint{4.251398in}{1.979132in}}%
\pgfpathcurveto{\pgfqpoint{4.251398in}{1.992154in}}{\pgfqpoint{4.246224in}{2.004645in}}{\pgfqpoint{4.237015in}{2.013854in}}%
\pgfpathcurveto{\pgfqpoint{4.227807in}{2.023062in}}{\pgfqpoint{4.215316in}{2.028236in}}{\pgfqpoint{4.202293in}{2.028236in}}%
\pgfpathcurveto{\pgfqpoint{4.189270in}{2.028236in}}{\pgfqpoint{4.176779in}{2.023062in}}{\pgfqpoint{4.167571in}{2.013854in}}%
\pgfpathcurveto{\pgfqpoint{4.158362in}{2.004645in}}{\pgfqpoint{4.153188in}{1.992154in}}{\pgfqpoint{4.153188in}{1.979132in}}%
\pgfpathcurveto{\pgfqpoint{4.153188in}{1.966109in}}{\pgfqpoint{4.158362in}{1.953618in}}{\pgfqpoint{4.167571in}{1.944409in}}%
\pgfpathcurveto{\pgfqpoint{4.176779in}{1.935201in}}{\pgfqpoint{4.189270in}{1.930027in}}{\pgfqpoint{4.202293in}{1.930027in}}%
\pgfpathlineto{\pgfqpoint{4.202293in}{1.930027in}}%
\pgfpathclose%
\pgfusepath{stroke,fill}%
\end{pgfscope}%
\begin{pgfscope}%
\pgfpathrectangle{\pgfqpoint{0.786164in}{0.768110in}}{\pgfqpoint{8.851069in}{7.081890in}}%
\pgfusepath{clip}%
\pgfsetbuttcap%
\pgfsetroundjoin%
\definecolor{currentfill}{rgb}{0.271305,0.019942,0.347269}%
\pgfsetfillcolor{currentfill}%
\pgfsetfillopacity{0.700000}%
\pgfsetlinewidth{0.501875pt}%
\definecolor{currentstroke}{rgb}{1.000000,1.000000,1.000000}%
\pgfsetstrokecolor{currentstroke}%
\pgfsetstrokeopacity{0.700000}%
\pgfsetdash{}{0pt}%
\pgfpathmoveto{\pgfqpoint{4.232452in}{1.887343in}}%
\pgfpathcurveto{\pgfqpoint{4.245475in}{1.887343in}}{\pgfqpoint{4.257966in}{1.892517in}}{\pgfqpoint{4.267174in}{1.901725in}}%
\pgfpathcurveto{\pgfqpoint{4.276383in}{1.910933in}}{\pgfqpoint{4.281557in}{1.923425in}}{\pgfqpoint{4.281557in}{1.936447in}}%
\pgfpathcurveto{\pgfqpoint{4.281557in}{1.949470in}}{\pgfqpoint{4.276383in}{1.961961in}}{\pgfqpoint{4.267174in}{1.971169in}}%
\pgfpathcurveto{\pgfqpoint{4.257966in}{1.980378in}}{\pgfqpoint{4.245475in}{1.985552in}}{\pgfqpoint{4.232452in}{1.985552in}}%
\pgfpathcurveto{\pgfqpoint{4.219429in}{1.985552in}}{\pgfqpoint{4.206938in}{1.980378in}}{\pgfqpoint{4.197730in}{1.971169in}}%
\pgfpathcurveto{\pgfqpoint{4.188521in}{1.961961in}}{\pgfqpoint{4.183347in}{1.949470in}}{\pgfqpoint{4.183347in}{1.936447in}}%
\pgfpathcurveto{\pgfqpoint{4.183347in}{1.923425in}}{\pgfqpoint{4.188521in}{1.910933in}}{\pgfqpoint{4.197730in}{1.901725in}}%
\pgfpathcurveto{\pgfqpoint{4.206938in}{1.892517in}}{\pgfqpoint{4.219429in}{1.887343in}}{\pgfqpoint{4.232452in}{1.887343in}}%
\pgfpathlineto{\pgfqpoint{4.232452in}{1.887343in}}%
\pgfpathclose%
\pgfusepath{stroke,fill}%
\end{pgfscope}%
\begin{pgfscope}%
\pgfpathrectangle{\pgfqpoint{0.786164in}{0.768110in}}{\pgfqpoint{8.851069in}{7.081890in}}%
\pgfusepath{clip}%
\pgfsetbuttcap%
\pgfsetroundjoin%
\definecolor{currentfill}{rgb}{0.274952,0.037752,0.364543}%
\pgfsetfillcolor{currentfill}%
\pgfsetfillopacity{0.700000}%
\pgfsetlinewidth{0.501875pt}%
\definecolor{currentstroke}{rgb}{1.000000,1.000000,1.000000}%
\pgfsetstrokecolor{currentstroke}%
\pgfsetstrokeopacity{0.700000}%
\pgfsetdash{}{0pt}%
\pgfpathmoveto{\pgfqpoint{4.161878in}{1.951369in}}%
\pgfpathcurveto{\pgfqpoint{4.174901in}{1.951369in}}{\pgfqpoint{4.187392in}{1.956543in}}{\pgfqpoint{4.196600in}{1.965752in}}%
\pgfpathcurveto{\pgfqpoint{4.205808in}{1.974960in}}{\pgfqpoint{4.210982in}{1.987451in}}{\pgfqpoint{4.210982in}{2.000474in}}%
\pgfpathcurveto{\pgfqpoint{4.210982in}{2.013496in}}{\pgfqpoint{4.205808in}{2.025988in}}{\pgfqpoint{4.196600in}{2.035196in}}%
\pgfpathcurveto{\pgfqpoint{4.187392in}{2.044404in}}{\pgfqpoint{4.174901in}{2.049578in}}{\pgfqpoint{4.161878in}{2.049578in}}%
\pgfpathcurveto{\pgfqpoint{4.148855in}{2.049578in}}{\pgfqpoint{4.136364in}{2.044404in}}{\pgfqpoint{4.127156in}{2.035196in}}%
\pgfpathcurveto{\pgfqpoint{4.117947in}{2.025988in}}{\pgfqpoint{4.112773in}{2.013496in}}{\pgfqpoint{4.112773in}{2.000474in}}%
\pgfpathcurveto{\pgfqpoint{4.112773in}{1.987451in}}{\pgfqpoint{4.117947in}{1.974960in}}{\pgfqpoint{4.127156in}{1.965752in}}%
\pgfpathcurveto{\pgfqpoint{4.136364in}{1.956543in}}{\pgfqpoint{4.148855in}{1.951369in}}{\pgfqpoint{4.161878in}{1.951369in}}%
\pgfpathlineto{\pgfqpoint{4.161878in}{1.951369in}}%
\pgfpathclose%
\pgfusepath{stroke,fill}%
\end{pgfscope}%
\begin{pgfscope}%
\pgfpathrectangle{\pgfqpoint{0.786164in}{0.768110in}}{\pgfqpoint{8.851069in}{7.081890in}}%
\pgfusepath{clip}%
\pgfsetbuttcap%
\pgfsetroundjoin%
\definecolor{currentfill}{rgb}{0.274952,0.037752,0.364543}%
\pgfsetfillcolor{currentfill}%
\pgfsetfillopacity{0.700000}%
\pgfsetlinewidth{0.501875pt}%
\definecolor{currentstroke}{rgb}{1.000000,1.000000,1.000000}%
\pgfsetstrokecolor{currentstroke}%
\pgfsetstrokeopacity{0.700000}%
\pgfsetdash{}{0pt}%
\pgfpathmoveto{\pgfqpoint{4.091182in}{1.972711in}}%
\pgfpathcurveto{\pgfqpoint{4.104204in}{1.972711in}}{\pgfqpoint{4.116695in}{1.977885in}}{\pgfqpoint{4.125904in}{1.987094in}}%
\pgfpathcurveto{\pgfqpoint{4.135112in}{1.996302in}}{\pgfqpoint{4.140286in}{2.008793in}}{\pgfqpoint{4.140286in}{2.021816in}}%
\pgfpathcurveto{\pgfqpoint{4.140286in}{2.034839in}}{\pgfqpoint{4.135112in}{2.047330in}}{\pgfqpoint{4.125904in}{2.056538in}}%
\pgfpathcurveto{\pgfqpoint{4.116695in}{2.065747in}}{\pgfqpoint{4.104204in}{2.070921in}}{\pgfqpoint{4.091182in}{2.070921in}}%
\pgfpathcurveto{\pgfqpoint{4.078159in}{2.070921in}}{\pgfqpoint{4.065668in}{2.065747in}}{\pgfqpoint{4.056459in}{2.056538in}}%
\pgfpathcurveto{\pgfqpoint{4.047251in}{2.047330in}}{\pgfqpoint{4.042077in}{2.034839in}}{\pgfqpoint{4.042077in}{2.021816in}}%
\pgfpathcurveto{\pgfqpoint{4.042077in}{2.008793in}}{\pgfqpoint{4.047251in}{1.996302in}}{\pgfqpoint{4.056459in}{1.987094in}}%
\pgfpathcurveto{\pgfqpoint{4.065668in}{1.977885in}}{\pgfqpoint{4.078159in}{1.972711in}}{\pgfqpoint{4.091182in}{1.972711in}}%
\pgfpathlineto{\pgfqpoint{4.091182in}{1.972711in}}%
\pgfpathclose%
\pgfusepath{stroke,fill}%
\end{pgfscope}%
\begin{pgfscope}%
\pgfpathrectangle{\pgfqpoint{0.786164in}{0.768110in}}{\pgfqpoint{8.851069in}{7.081890in}}%
\pgfusepath{clip}%
\pgfsetbuttcap%
\pgfsetroundjoin%
\definecolor{currentfill}{rgb}{0.274952,0.037752,0.364543}%
\pgfsetfillcolor{currentfill}%
\pgfsetfillopacity{0.700000}%
\pgfsetlinewidth{0.501875pt}%
\definecolor{currentstroke}{rgb}{1.000000,1.000000,1.000000}%
\pgfsetstrokecolor{currentstroke}%
\pgfsetstrokeopacity{0.700000}%
\pgfsetdash{}{0pt}%
\pgfpathmoveto{\pgfqpoint{4.141853in}{1.951369in}}%
\pgfpathcurveto{\pgfqpoint{4.154876in}{1.951369in}}{\pgfqpoint{4.167367in}{1.956543in}}{\pgfqpoint{4.176576in}{1.965752in}}%
\pgfpathcurveto{\pgfqpoint{4.185784in}{1.974960in}}{\pgfqpoint{4.190958in}{1.987451in}}{\pgfqpoint{4.190958in}{2.000474in}}%
\pgfpathcurveto{\pgfqpoint{4.190958in}{2.013496in}}{\pgfqpoint{4.185784in}{2.025988in}}{\pgfqpoint{4.176576in}{2.035196in}}%
\pgfpathcurveto{\pgfqpoint{4.167367in}{2.044404in}}{\pgfqpoint{4.154876in}{2.049578in}}{\pgfqpoint{4.141853in}{2.049578in}}%
\pgfpathcurveto{\pgfqpoint{4.128831in}{2.049578in}}{\pgfqpoint{4.116340in}{2.044404in}}{\pgfqpoint{4.107131in}{2.035196in}}%
\pgfpathcurveto{\pgfqpoint{4.097923in}{2.025988in}}{\pgfqpoint{4.092749in}{2.013496in}}{\pgfqpoint{4.092749in}{2.000474in}}%
\pgfpathcurveto{\pgfqpoint{4.092749in}{1.987451in}}{\pgfqpoint{4.097923in}{1.974960in}}{\pgfqpoint{4.107131in}{1.965752in}}%
\pgfpathcurveto{\pgfqpoint{4.116340in}{1.956543in}}{\pgfqpoint{4.128831in}{1.951369in}}{\pgfqpoint{4.141853in}{1.951369in}}%
\pgfpathlineto{\pgfqpoint{4.141853in}{1.951369in}}%
\pgfpathclose%
\pgfusepath{stroke,fill}%
\end{pgfscope}%
\begin{pgfscope}%
\pgfpathrectangle{\pgfqpoint{0.786164in}{0.768110in}}{\pgfqpoint{8.851069in}{7.081890in}}%
\pgfusepath{clip}%
\pgfsetbuttcap%
\pgfsetroundjoin%
\definecolor{currentfill}{rgb}{0.272594,0.025563,0.353093}%
\pgfsetfillcolor{currentfill}%
\pgfsetfillopacity{0.700000}%
\pgfsetlinewidth{0.501875pt}%
\definecolor{currentstroke}{rgb}{1.000000,1.000000,1.000000}%
\pgfsetstrokecolor{currentstroke}%
\pgfsetstrokeopacity{0.700000}%
\pgfsetdash{}{0pt}%
\pgfpathmoveto{\pgfqpoint{4.164808in}{1.866000in}}%
\pgfpathcurveto{\pgfqpoint{4.177831in}{1.866000in}}{\pgfqpoint{4.190322in}{1.871174in}}{\pgfqpoint{4.199530in}{1.880383in}}%
\pgfpathcurveto{\pgfqpoint{4.208739in}{1.889591in}}{\pgfqpoint{4.213913in}{1.902082in}}{\pgfqpoint{4.213913in}{1.915105in}}%
\pgfpathcurveto{\pgfqpoint{4.213913in}{1.928128in}}{\pgfqpoint{4.208739in}{1.940619in}}{\pgfqpoint{4.199530in}{1.949827in}}%
\pgfpathcurveto{\pgfqpoint{4.190322in}{1.959036in}}{\pgfqpoint{4.177831in}{1.964210in}}{\pgfqpoint{4.164808in}{1.964210in}}%
\pgfpathcurveto{\pgfqpoint{4.151786in}{1.964210in}}{\pgfqpoint{4.139294in}{1.959036in}}{\pgfqpoint{4.130086in}{1.949827in}}%
\pgfpathcurveto{\pgfqpoint{4.120878in}{1.940619in}}{\pgfqpoint{4.115704in}{1.928128in}}{\pgfqpoint{4.115704in}{1.915105in}}%
\pgfpathcurveto{\pgfqpoint{4.115704in}{1.902082in}}{\pgfqpoint{4.120878in}{1.889591in}}{\pgfqpoint{4.130086in}{1.880383in}}%
\pgfpathcurveto{\pgfqpoint{4.139294in}{1.871174in}}{\pgfqpoint{4.151786in}{1.866000in}}{\pgfqpoint{4.164808in}{1.866000in}}%
\pgfpathlineto{\pgfqpoint{4.164808in}{1.866000in}}%
\pgfpathclose%
\pgfusepath{stroke,fill}%
\end{pgfscope}%
\begin{pgfscope}%
\pgfpathrectangle{\pgfqpoint{0.786164in}{0.768110in}}{\pgfqpoint{8.851069in}{7.081890in}}%
\pgfusepath{clip}%
\pgfsetbuttcap%
\pgfsetroundjoin%
\definecolor{currentfill}{rgb}{0.277018,0.050344,0.375715}%
\pgfsetfillcolor{currentfill}%
\pgfsetfillopacity{0.700000}%
\pgfsetlinewidth{0.501875pt}%
\definecolor{currentstroke}{rgb}{1.000000,1.000000,1.000000}%
\pgfsetstrokecolor{currentstroke}%
\pgfsetstrokeopacity{0.700000}%
\pgfsetdash{}{0pt}%
\pgfpathmoveto{\pgfqpoint{4.090693in}{1.930027in}}%
\pgfpathcurveto{\pgfqpoint{4.103716in}{1.930027in}}{\pgfqpoint{4.116207in}{1.935201in}}{\pgfqpoint{4.125415in}{1.944409in}}%
\pgfpathcurveto{\pgfqpoint{4.134624in}{1.953618in}}{\pgfqpoint{4.139798in}{1.966109in}}{\pgfqpoint{4.139798in}{1.979132in}}%
\pgfpathcurveto{\pgfqpoint{4.139798in}{1.992154in}}{\pgfqpoint{4.134624in}{2.004645in}}{\pgfqpoint{4.125415in}{2.013854in}}%
\pgfpathcurveto{\pgfqpoint{4.116207in}{2.023062in}}{\pgfqpoint{4.103716in}{2.028236in}}{\pgfqpoint{4.090693in}{2.028236in}}%
\pgfpathcurveto{\pgfqpoint{4.077670in}{2.028236in}}{\pgfqpoint{4.065179in}{2.023062in}}{\pgfqpoint{4.055971in}{2.013854in}}%
\pgfpathcurveto{\pgfqpoint{4.046763in}{2.004645in}}{\pgfqpoint{4.041589in}{1.992154in}}{\pgfqpoint{4.041589in}{1.979132in}}%
\pgfpathcurveto{\pgfqpoint{4.041589in}{1.966109in}}{\pgfqpoint{4.046763in}{1.953618in}}{\pgfqpoint{4.055971in}{1.944409in}}%
\pgfpathcurveto{\pgfqpoint{4.065179in}{1.935201in}}{\pgfqpoint{4.077670in}{1.930027in}}{\pgfqpoint{4.090693in}{1.930027in}}%
\pgfpathlineto{\pgfqpoint{4.090693in}{1.930027in}}%
\pgfpathclose%
\pgfusepath{stroke,fill}%
\end{pgfscope}%
\begin{pgfscope}%
\pgfpathrectangle{\pgfqpoint{0.786164in}{0.768110in}}{\pgfqpoint{8.851069in}{7.081890in}}%
\pgfusepath{clip}%
\pgfsetbuttcap%
\pgfsetroundjoin%
\definecolor{currentfill}{rgb}{0.273809,0.031497,0.358853}%
\pgfsetfillcolor{currentfill}%
\pgfsetfillopacity{0.700000}%
\pgfsetlinewidth{0.501875pt}%
\definecolor{currentstroke}{rgb}{1.000000,1.000000,1.000000}%
\pgfsetstrokecolor{currentstroke}%
\pgfsetstrokeopacity{0.700000}%
\pgfsetdash{}{0pt}%
\pgfpathmoveto{\pgfqpoint{4.123538in}{1.887343in}}%
\pgfpathcurveto{\pgfqpoint{4.136561in}{1.887343in}}{\pgfqpoint{4.149052in}{1.892517in}}{\pgfqpoint{4.158260in}{1.901725in}}%
\pgfpathcurveto{\pgfqpoint{4.167469in}{1.910933in}}{\pgfqpoint{4.172643in}{1.923425in}}{\pgfqpoint{4.172643in}{1.936447in}}%
\pgfpathcurveto{\pgfqpoint{4.172643in}{1.949470in}}{\pgfqpoint{4.167469in}{1.961961in}}{\pgfqpoint{4.158260in}{1.971169in}}%
\pgfpathcurveto{\pgfqpoint{4.149052in}{1.980378in}}{\pgfqpoint{4.136561in}{1.985552in}}{\pgfqpoint{4.123538in}{1.985552in}}%
\pgfpathcurveto{\pgfqpoint{4.110516in}{1.985552in}}{\pgfqpoint{4.098024in}{1.980378in}}{\pgfqpoint{4.088816in}{1.971169in}}%
\pgfpathcurveto{\pgfqpoint{4.079608in}{1.961961in}}{\pgfqpoint{4.074434in}{1.949470in}}{\pgfqpoint{4.074434in}{1.936447in}}%
\pgfpathcurveto{\pgfqpoint{4.074434in}{1.923425in}}{\pgfqpoint{4.079608in}{1.910933in}}{\pgfqpoint{4.088816in}{1.901725in}}%
\pgfpathcurveto{\pgfqpoint{4.098024in}{1.892517in}}{\pgfqpoint{4.110516in}{1.887343in}}{\pgfqpoint{4.123538in}{1.887343in}}%
\pgfpathlineto{\pgfqpoint{4.123538in}{1.887343in}}%
\pgfpathclose%
\pgfusepath{stroke,fill}%
\end{pgfscope}%
\begin{pgfscope}%
\pgfpathrectangle{\pgfqpoint{0.786164in}{0.768110in}}{\pgfqpoint{8.851069in}{7.081890in}}%
\pgfusepath{clip}%
\pgfsetbuttcap%
\pgfsetroundjoin%
\definecolor{currentfill}{rgb}{0.136408,0.541173,0.554483}%
\pgfsetfillcolor{currentfill}%
\pgfsetfillopacity{0.700000}%
\pgfsetlinewidth{0.501875pt}%
\definecolor{currentstroke}{rgb}{1.000000,1.000000,1.000000}%
\pgfsetstrokecolor{currentstroke}%
\pgfsetstrokeopacity{0.700000}%
\pgfsetdash{}{0pt}%
\pgfpathmoveto{\pgfqpoint{5.867989in}{1.695263in}}%
\pgfpathcurveto{\pgfqpoint{5.881011in}{1.695263in}}{\pgfqpoint{5.893503in}{1.700437in}}{\pgfqpoint{5.902711in}{1.709645in}}%
\pgfpathcurveto{\pgfqpoint{5.911919in}{1.718854in}}{\pgfqpoint{5.917093in}{1.731345in}}{\pgfqpoint{5.917093in}{1.744368in}}%
\pgfpathcurveto{\pgfqpoint{5.917093in}{1.757390in}}{\pgfqpoint{5.911919in}{1.769881in}}{\pgfqpoint{5.902711in}{1.779090in}}%
\pgfpathcurveto{\pgfqpoint{5.893503in}{1.788298in}}{\pgfqpoint{5.881011in}{1.793472in}}{\pgfqpoint{5.867989in}{1.793472in}}%
\pgfpathcurveto{\pgfqpoint{5.854966in}{1.793472in}}{\pgfqpoint{5.842475in}{1.788298in}}{\pgfqpoint{5.833267in}{1.779090in}}%
\pgfpathcurveto{\pgfqpoint{5.824058in}{1.769881in}}{\pgfqpoint{5.818884in}{1.757390in}}{\pgfqpoint{5.818884in}{1.744368in}}%
\pgfpathcurveto{\pgfqpoint{5.818884in}{1.731345in}}{\pgfqpoint{5.824058in}{1.718854in}}{\pgfqpoint{5.833267in}{1.709645in}}%
\pgfpathcurveto{\pgfqpoint{5.842475in}{1.700437in}}{\pgfqpoint{5.854966in}{1.695263in}}{\pgfqpoint{5.867989in}{1.695263in}}%
\pgfpathlineto{\pgfqpoint{5.867989in}{1.695263in}}%
\pgfpathclose%
\pgfusepath{stroke,fill}%
\end{pgfscope}%
\begin{pgfscope}%
\pgfpathrectangle{\pgfqpoint{0.786164in}{0.768110in}}{\pgfqpoint{8.851069in}{7.081890in}}%
\pgfusepath{clip}%
\pgfsetbuttcap%
\pgfsetroundjoin%
\definecolor{currentfill}{rgb}{0.144759,0.519093,0.556572}%
\pgfsetfillcolor{currentfill}%
\pgfsetfillopacity{0.700000}%
\pgfsetlinewidth{0.501875pt}%
\definecolor{currentstroke}{rgb}{1.000000,1.000000,1.000000}%
\pgfsetstrokecolor{currentstroke}%
\pgfsetstrokeopacity{0.700000}%
\pgfsetdash{}{0pt}%
\pgfpathmoveto{\pgfqpoint{6.057855in}{1.588552in}}%
\pgfpathcurveto{\pgfqpoint{6.070878in}{1.588552in}}{\pgfqpoint{6.083369in}{1.593726in}}{\pgfqpoint{6.092577in}{1.602935in}}%
\pgfpathcurveto{\pgfqpoint{6.101786in}{1.612143in}}{\pgfqpoint{6.106960in}{1.624634in}}{\pgfqpoint{6.106960in}{1.637657in}}%
\pgfpathcurveto{\pgfqpoint{6.106960in}{1.650680in}}{\pgfqpoint{6.101786in}{1.663171in}}{\pgfqpoint{6.092577in}{1.672379in}}%
\pgfpathcurveto{\pgfqpoint{6.083369in}{1.681588in}}{\pgfqpoint{6.070878in}{1.686761in}}{\pgfqpoint{6.057855in}{1.686761in}}%
\pgfpathcurveto{\pgfqpoint{6.044832in}{1.686761in}}{\pgfqpoint{6.032341in}{1.681588in}}{\pgfqpoint{6.023133in}{1.672379in}}%
\pgfpathcurveto{\pgfqpoint{6.013924in}{1.663171in}}{\pgfqpoint{6.008751in}{1.650680in}}{\pgfqpoint{6.008751in}{1.637657in}}%
\pgfpathcurveto{\pgfqpoint{6.008751in}{1.624634in}}{\pgfqpoint{6.013924in}{1.612143in}}{\pgfqpoint{6.023133in}{1.602935in}}%
\pgfpathcurveto{\pgfqpoint{6.032341in}{1.593726in}}{\pgfqpoint{6.044832in}{1.588552in}}{\pgfqpoint{6.057855in}{1.588552in}}%
\pgfpathlineto{\pgfqpoint{6.057855in}{1.588552in}}%
\pgfpathclose%
\pgfusepath{stroke,fill}%
\end{pgfscope}%
\begin{pgfscope}%
\pgfpathrectangle{\pgfqpoint{0.786164in}{0.768110in}}{\pgfqpoint{8.851069in}{7.081890in}}%
\pgfusepath{clip}%
\pgfsetbuttcap%
\pgfsetroundjoin%
\definecolor{currentfill}{rgb}{0.151918,0.500685,0.557587}%
\pgfsetfillcolor{currentfill}%
\pgfsetfillopacity{0.700000}%
\pgfsetlinewidth{0.501875pt}%
\definecolor{currentstroke}{rgb}{1.000000,1.000000,1.000000}%
\pgfsetstrokecolor{currentstroke}%
\pgfsetstrokeopacity{0.700000}%
\pgfsetdash{}{0pt}%
\pgfpathmoveto{\pgfqpoint{6.271775in}{1.481841in}}%
\pgfpathcurveto{\pgfqpoint{6.284798in}{1.481841in}}{\pgfqpoint{6.297289in}{1.487015in}}{\pgfqpoint{6.306498in}{1.496224in}}%
\pgfpathcurveto{\pgfqpoint{6.315706in}{1.505432in}}{\pgfqpoint{6.320880in}{1.517923in}}{\pgfqpoint{6.320880in}{1.530946in}}%
\pgfpathcurveto{\pgfqpoint{6.320880in}{1.543969in}}{\pgfqpoint{6.315706in}{1.556460in}}{\pgfqpoint{6.306498in}{1.565668in}}%
\pgfpathcurveto{\pgfqpoint{6.297289in}{1.574877in}}{\pgfqpoint{6.284798in}{1.580051in}}{\pgfqpoint{6.271775in}{1.580051in}}%
\pgfpathcurveto{\pgfqpoint{6.258753in}{1.580051in}}{\pgfqpoint{6.246262in}{1.574877in}}{\pgfqpoint{6.237053in}{1.565668in}}%
\pgfpathcurveto{\pgfqpoint{6.227845in}{1.556460in}}{\pgfqpoint{6.222671in}{1.543969in}}{\pgfqpoint{6.222671in}{1.530946in}}%
\pgfpathcurveto{\pgfqpoint{6.222671in}{1.517923in}}{\pgfqpoint{6.227845in}{1.505432in}}{\pgfqpoint{6.237053in}{1.496224in}}%
\pgfpathcurveto{\pgfqpoint{6.246262in}{1.487015in}}{\pgfqpoint{6.258753in}{1.481841in}}{\pgfqpoint{6.271775in}{1.481841in}}%
\pgfpathlineto{\pgfqpoint{6.271775in}{1.481841in}}%
\pgfpathclose%
\pgfusepath{stroke,fill}%
\end{pgfscope}%
\begin{pgfscope}%
\pgfpathrectangle{\pgfqpoint{0.786164in}{0.768110in}}{\pgfqpoint{8.851069in}{7.081890in}}%
\pgfusepath{clip}%
\pgfsetbuttcap%
\pgfsetroundjoin%
\definecolor{currentfill}{rgb}{0.154815,0.493313,0.557840}%
\pgfsetfillcolor{currentfill}%
\pgfsetfillopacity{0.700000}%
\pgfsetlinewidth{0.501875pt}%
\definecolor{currentstroke}{rgb}{1.000000,1.000000,1.000000}%
\pgfsetstrokecolor{currentstroke}%
\pgfsetstrokeopacity{0.700000}%
\pgfsetdash{}{0pt}%
\pgfpathmoveto{\pgfqpoint{6.453705in}{1.460499in}}%
\pgfpathcurveto{\pgfqpoint{6.466728in}{1.460499in}}{\pgfqpoint{6.479219in}{1.465673in}}{\pgfqpoint{6.488427in}{1.474882in}}%
\pgfpathcurveto{\pgfqpoint{6.497636in}{1.484090in}}{\pgfqpoint{6.502810in}{1.496581in}}{\pgfqpoint{6.502810in}{1.509604in}}%
\pgfpathcurveto{\pgfqpoint{6.502810in}{1.522627in}}{\pgfqpoint{6.497636in}{1.535118in}}{\pgfqpoint{6.488427in}{1.544326in}}%
\pgfpathcurveto{\pgfqpoint{6.479219in}{1.553534in}}{\pgfqpoint{6.466728in}{1.558708in}}{\pgfqpoint{6.453705in}{1.558708in}}%
\pgfpathcurveto{\pgfqpoint{6.440682in}{1.558708in}}{\pgfqpoint{6.428191in}{1.553534in}}{\pgfqpoint{6.418983in}{1.544326in}}%
\pgfpathcurveto{\pgfqpoint{6.409774in}{1.535118in}}{\pgfqpoint{6.404600in}{1.522627in}}{\pgfqpoint{6.404600in}{1.509604in}}%
\pgfpathcurveto{\pgfqpoint{6.404600in}{1.496581in}}{\pgfqpoint{6.409774in}{1.484090in}}{\pgfqpoint{6.418983in}{1.474882in}}%
\pgfpathcurveto{\pgfqpoint{6.428191in}{1.465673in}}{\pgfqpoint{6.440682in}{1.460499in}}{\pgfqpoint{6.453705in}{1.460499in}}%
\pgfpathlineto{\pgfqpoint{6.453705in}{1.460499in}}%
\pgfpathclose%
\pgfusepath{stroke,fill}%
\end{pgfscope}%
\begin{pgfscope}%
\pgfpathrectangle{\pgfqpoint{0.786164in}{0.768110in}}{\pgfqpoint{8.851069in}{7.081890in}}%
\pgfusepath{clip}%
\pgfsetbuttcap%
\pgfsetroundjoin%
\definecolor{currentfill}{rgb}{0.157729,0.485932,0.558013}%
\pgfsetfillcolor{currentfill}%
\pgfsetfillopacity{0.700000}%
\pgfsetlinewidth{0.501875pt}%
\definecolor{currentstroke}{rgb}{1.000000,1.000000,1.000000}%
\pgfsetstrokecolor{currentstroke}%
\pgfsetstrokeopacity{0.700000}%
\pgfsetdash{}{0pt}%
\pgfpathmoveto{\pgfqpoint{6.538931in}{1.417815in}}%
\pgfpathcurveto{\pgfqpoint{6.551954in}{1.417815in}}{\pgfqpoint{6.564445in}{1.422989in}}{\pgfqpoint{6.573654in}{1.432197in}}%
\pgfpathcurveto{\pgfqpoint{6.582862in}{1.441406in}}{\pgfqpoint{6.588036in}{1.453897in}}{\pgfqpoint{6.588036in}{1.466919in}}%
\pgfpathcurveto{\pgfqpoint{6.588036in}{1.479942in}}{\pgfqpoint{6.582862in}{1.492433in}}{\pgfqpoint{6.573654in}{1.501642in}}%
\pgfpathcurveto{\pgfqpoint{6.564445in}{1.510850in}}{\pgfqpoint{6.551954in}{1.516024in}}{\pgfqpoint{6.538931in}{1.516024in}}%
\pgfpathcurveto{\pgfqpoint{6.525909in}{1.516024in}}{\pgfqpoint{6.513418in}{1.510850in}}{\pgfqpoint{6.504209in}{1.501642in}}%
\pgfpathcurveto{\pgfqpoint{6.495001in}{1.492433in}}{\pgfqpoint{6.489827in}{1.479942in}}{\pgfqpoint{6.489827in}{1.466919in}}%
\pgfpathcurveto{\pgfqpoint{6.489827in}{1.453897in}}{\pgfqpoint{6.495001in}{1.441406in}}{\pgfqpoint{6.504209in}{1.432197in}}%
\pgfpathcurveto{\pgfqpoint{6.513418in}{1.422989in}}{\pgfqpoint{6.525909in}{1.417815in}}{\pgfqpoint{6.538931in}{1.417815in}}%
\pgfpathlineto{\pgfqpoint{6.538931in}{1.417815in}}%
\pgfpathclose%
\pgfusepath{stroke,fill}%
\end{pgfscope}%
\begin{pgfscope}%
\pgfpathrectangle{\pgfqpoint{0.786164in}{0.768110in}}{\pgfqpoint{8.851069in}{7.081890in}}%
\pgfusepath{clip}%
\pgfsetbuttcap%
\pgfsetroundjoin%
\definecolor{currentfill}{rgb}{0.179019,0.433756,0.557430}%
\pgfsetfillcolor{currentfill}%
\pgfsetfillopacity{0.700000}%
\pgfsetlinewidth{0.501875pt}%
\definecolor{currentstroke}{rgb}{1.000000,1.000000,1.000000}%
\pgfsetstrokecolor{currentstroke}%
\pgfsetstrokeopacity{0.700000}%
\pgfsetdash{}{0pt}%
\pgfpathmoveto{\pgfqpoint{6.917687in}{1.332446in}}%
\pgfpathcurveto{\pgfqpoint{6.930710in}{1.332446in}}{\pgfqpoint{6.943201in}{1.337620in}}{\pgfqpoint{6.952409in}{1.346829in}}%
\pgfpathcurveto{\pgfqpoint{6.961618in}{1.356037in}}{\pgfqpoint{6.966792in}{1.368528in}}{\pgfqpoint{6.966792in}{1.381551in}}%
\pgfpathcurveto{\pgfqpoint{6.966792in}{1.394574in}}{\pgfqpoint{6.961618in}{1.407065in}}{\pgfqpoint{6.952409in}{1.416273in}}%
\pgfpathcurveto{\pgfqpoint{6.943201in}{1.425481in}}{\pgfqpoint{6.930710in}{1.430655in}}{\pgfqpoint{6.917687in}{1.430655in}}%
\pgfpathcurveto{\pgfqpoint{6.904665in}{1.430655in}}{\pgfqpoint{6.892173in}{1.425481in}}{\pgfqpoint{6.882965in}{1.416273in}}%
\pgfpathcurveto{\pgfqpoint{6.873757in}{1.407065in}}{\pgfqpoint{6.868583in}{1.394574in}}{\pgfqpoint{6.868583in}{1.381551in}}%
\pgfpathcurveto{\pgfqpoint{6.868583in}{1.368528in}}{\pgfqpoint{6.873757in}{1.356037in}}{\pgfqpoint{6.882965in}{1.346829in}}%
\pgfpathcurveto{\pgfqpoint{6.892173in}{1.337620in}}{\pgfqpoint{6.904665in}{1.332446in}}{\pgfqpoint{6.917687in}{1.332446in}}%
\pgfpathlineto{\pgfqpoint{6.917687in}{1.332446in}}%
\pgfpathclose%
\pgfusepath{stroke,fill}%
\end{pgfscope}%
\begin{pgfscope}%
\pgfpathrectangle{\pgfqpoint{0.786164in}{0.768110in}}{\pgfqpoint{8.851069in}{7.081890in}}%
\pgfusepath{clip}%
\pgfsetbuttcap%
\pgfsetroundjoin%
\definecolor{currentfill}{rgb}{0.156270,0.489624,0.557936}%
\pgfsetfillcolor{currentfill}%
\pgfsetfillopacity{0.700000}%
\pgfsetlinewidth{0.501875pt}%
\definecolor{currentstroke}{rgb}{1.000000,1.000000,1.000000}%
\pgfsetstrokecolor{currentstroke}%
\pgfsetstrokeopacity{0.700000}%
\pgfsetdash{}{0pt}%
\pgfpathmoveto{\pgfqpoint{6.804744in}{1.396473in}}%
\pgfpathcurveto{\pgfqpoint{6.817767in}{1.396473in}}{\pgfqpoint{6.830258in}{1.401647in}}{\pgfqpoint{6.839466in}{1.410855in}}%
\pgfpathcurveto{\pgfqpoint{6.848675in}{1.420064in}}{\pgfqpoint{6.853849in}{1.432555in}}{\pgfqpoint{6.853849in}{1.445577in}}%
\pgfpathcurveto{\pgfqpoint{6.853849in}{1.458600in}}{\pgfqpoint{6.848675in}{1.471091in}}{\pgfqpoint{6.839466in}{1.480300in}}%
\pgfpathcurveto{\pgfqpoint{6.830258in}{1.489508in}}{\pgfqpoint{6.817767in}{1.494682in}}{\pgfqpoint{6.804744in}{1.494682in}}%
\pgfpathcurveto{\pgfqpoint{6.791722in}{1.494682in}}{\pgfqpoint{6.779230in}{1.489508in}}{\pgfqpoint{6.770022in}{1.480300in}}%
\pgfpathcurveto{\pgfqpoint{6.760814in}{1.471091in}}{\pgfqpoint{6.755640in}{1.458600in}}{\pgfqpoint{6.755640in}{1.445577in}}%
\pgfpathcurveto{\pgfqpoint{6.755640in}{1.432555in}}{\pgfqpoint{6.760814in}{1.420064in}}{\pgfqpoint{6.770022in}{1.410855in}}%
\pgfpathcurveto{\pgfqpoint{6.779230in}{1.401647in}}{\pgfqpoint{6.791722in}{1.396473in}}{\pgfqpoint{6.804744in}{1.396473in}}%
\pgfpathlineto{\pgfqpoint{6.804744in}{1.396473in}}%
\pgfpathclose%
\pgfusepath{stroke,fill}%
\end{pgfscope}%
\begin{pgfscope}%
\pgfpathrectangle{\pgfqpoint{0.786164in}{0.768110in}}{\pgfqpoint{8.851069in}{7.081890in}}%
\pgfusepath{clip}%
\pgfsetbuttcap%
\pgfsetroundjoin%
\definecolor{currentfill}{rgb}{0.160665,0.478540,0.558115}%
\pgfsetfillcolor{currentfill}%
\pgfsetfillopacity{0.700000}%
\pgfsetlinewidth{0.501875pt}%
\definecolor{currentstroke}{rgb}{1.000000,1.000000,1.000000}%
\pgfsetstrokecolor{currentstroke}%
\pgfsetstrokeopacity{0.700000}%
\pgfsetdash{}{0pt}%
\pgfpathmoveto{\pgfqpoint{6.991924in}{1.332446in}}%
\pgfpathcurveto{\pgfqpoint{7.004947in}{1.332446in}}{\pgfqpoint{7.017438in}{1.337620in}}{\pgfqpoint{7.026647in}{1.346829in}}%
\pgfpathcurveto{\pgfqpoint{7.035855in}{1.356037in}}{\pgfqpoint{7.041029in}{1.368528in}}{\pgfqpoint{7.041029in}{1.381551in}}%
\pgfpathcurveto{\pgfqpoint{7.041029in}{1.394574in}}{\pgfqpoint{7.035855in}{1.407065in}}{\pgfqpoint{7.026647in}{1.416273in}}%
\pgfpathcurveto{\pgfqpoint{7.017438in}{1.425481in}}{\pgfqpoint{7.004947in}{1.430655in}}{\pgfqpoint{6.991924in}{1.430655in}}%
\pgfpathcurveto{\pgfqpoint{6.978902in}{1.430655in}}{\pgfqpoint{6.966411in}{1.425481in}}{\pgfqpoint{6.957202in}{1.416273in}}%
\pgfpathcurveto{\pgfqpoint{6.947994in}{1.407065in}}{\pgfqpoint{6.942820in}{1.394574in}}{\pgfqpoint{6.942820in}{1.381551in}}%
\pgfpathcurveto{\pgfqpoint{6.942820in}{1.368528in}}{\pgfqpoint{6.947994in}{1.356037in}}{\pgfqpoint{6.957202in}{1.346829in}}%
\pgfpathcurveto{\pgfqpoint{6.966411in}{1.337620in}}{\pgfqpoint{6.978902in}{1.332446in}}{\pgfqpoint{6.991924in}{1.332446in}}%
\pgfpathlineto{\pgfqpoint{6.991924in}{1.332446in}}%
\pgfpathclose%
\pgfusepath{stroke,fill}%
\end{pgfscope}%
\begin{pgfscope}%
\pgfpathrectangle{\pgfqpoint{0.786164in}{0.768110in}}{\pgfqpoint{8.851069in}{7.081890in}}%
\pgfusepath{clip}%
\pgfsetbuttcap%
\pgfsetroundjoin%
\definecolor{currentfill}{rgb}{0.163625,0.471133,0.558148}%
\pgfsetfillcolor{currentfill}%
\pgfsetfillopacity{0.700000}%
\pgfsetlinewidth{0.501875pt}%
\definecolor{currentstroke}{rgb}{1.000000,1.000000,1.000000}%
\pgfsetstrokecolor{currentstroke}%
\pgfsetstrokeopacity{0.700000}%
\pgfsetdash{}{0pt}%
\pgfpathmoveto{\pgfqpoint{7.208164in}{1.268420in}}%
\pgfpathcurveto{\pgfqpoint{7.221187in}{1.268420in}}{\pgfqpoint{7.233678in}{1.273594in}}{\pgfqpoint{7.242887in}{1.282802in}}%
\pgfpathcurveto{\pgfqpoint{7.252095in}{1.292011in}}{\pgfqpoint{7.257269in}{1.304502in}}{\pgfqpoint{7.257269in}{1.317524in}}%
\pgfpathcurveto{\pgfqpoint{7.257269in}{1.330547in}}{\pgfqpoint{7.252095in}{1.343038in}}{\pgfqpoint{7.242887in}{1.352247in}}%
\pgfpathcurveto{\pgfqpoint{7.233678in}{1.361455in}}{\pgfqpoint{7.221187in}{1.366629in}}{\pgfqpoint{7.208164in}{1.366629in}}%
\pgfpathcurveto{\pgfqpoint{7.195142in}{1.366629in}}{\pgfqpoint{7.182651in}{1.361455in}}{\pgfqpoint{7.173442in}{1.352247in}}%
\pgfpathcurveto{\pgfqpoint{7.164234in}{1.343038in}}{\pgfqpoint{7.159060in}{1.330547in}}{\pgfqpoint{7.159060in}{1.317524in}}%
\pgfpathcurveto{\pgfqpoint{7.159060in}{1.304502in}}{\pgfqpoint{7.164234in}{1.292011in}}{\pgfqpoint{7.173442in}{1.282802in}}%
\pgfpathcurveto{\pgfqpoint{7.182651in}{1.273594in}}{\pgfqpoint{7.195142in}{1.268420in}}{\pgfqpoint{7.208164in}{1.268420in}}%
\pgfpathlineto{\pgfqpoint{7.208164in}{1.268420in}}%
\pgfpathclose%
\pgfusepath{stroke,fill}%
\end{pgfscope}%
\begin{pgfscope}%
\pgfpathrectangle{\pgfqpoint{0.786164in}{0.768110in}}{\pgfqpoint{8.851069in}{7.081890in}}%
\pgfusepath{clip}%
\pgfsetbuttcap%
\pgfsetroundjoin%
\definecolor{currentfill}{rgb}{0.169646,0.456262,0.558030}%
\pgfsetfillcolor{currentfill}%
\pgfsetfillopacity{0.700000}%
\pgfsetlinewidth{0.501875pt}%
\definecolor{currentstroke}{rgb}{1.000000,1.000000,1.000000}%
\pgfsetstrokecolor{currentstroke}%
\pgfsetstrokeopacity{0.700000}%
\pgfsetdash{}{0pt}%
\pgfpathmoveto{\pgfqpoint{7.299740in}{1.204393in}}%
\pgfpathcurveto{\pgfqpoint{7.312763in}{1.204393in}}{\pgfqpoint{7.325254in}{1.209567in}}{\pgfqpoint{7.334462in}{1.218776in}}%
\pgfpathcurveto{\pgfqpoint{7.343671in}{1.227984in}}{\pgfqpoint{7.348844in}{1.240475in}}{\pgfqpoint{7.348844in}{1.253498in}}%
\pgfpathcurveto{\pgfqpoint{7.348844in}{1.266520in}}{\pgfqpoint{7.343671in}{1.279012in}}{\pgfqpoint{7.334462in}{1.288220in}}%
\pgfpathcurveto{\pgfqpoint{7.325254in}{1.297428in}}{\pgfqpoint{7.312763in}{1.302602in}}{\pgfqpoint{7.299740in}{1.302602in}}%
\pgfpathcurveto{\pgfqpoint{7.286717in}{1.302602in}}{\pgfqpoint{7.274226in}{1.297428in}}{\pgfqpoint{7.265018in}{1.288220in}}%
\pgfpathcurveto{\pgfqpoint{7.255809in}{1.279012in}}{\pgfqpoint{7.250635in}{1.266520in}}{\pgfqpoint{7.250635in}{1.253498in}}%
\pgfpathcurveto{\pgfqpoint{7.250635in}{1.240475in}}{\pgfqpoint{7.255809in}{1.227984in}}{\pgfqpoint{7.265018in}{1.218776in}}%
\pgfpathcurveto{\pgfqpoint{7.274226in}{1.209567in}}{\pgfqpoint{7.286717in}{1.204393in}}{\pgfqpoint{7.299740in}{1.204393in}}%
\pgfpathlineto{\pgfqpoint{7.299740in}{1.204393in}}%
\pgfpathclose%
\pgfusepath{stroke,fill}%
\end{pgfscope}%
\begin{pgfscope}%
\pgfpathrectangle{\pgfqpoint{0.786164in}{0.768110in}}{\pgfqpoint{8.851069in}{7.081890in}}%
\pgfusepath{clip}%
\pgfsetbuttcap%
\pgfsetroundjoin%
\definecolor{currentfill}{rgb}{0.174274,0.445044,0.557792}%
\pgfsetfillcolor{currentfill}%
\pgfsetfillopacity{0.700000}%
\pgfsetlinewidth{0.501875pt}%
\definecolor{currentstroke}{rgb}{1.000000,1.000000,1.000000}%
\pgfsetstrokecolor{currentstroke}%
\pgfsetstrokeopacity{0.700000}%
\pgfsetdash{}{0pt}%
\pgfpathmoveto{\pgfqpoint{7.421596in}{1.204393in}}%
\pgfpathcurveto{\pgfqpoint{7.434619in}{1.204393in}}{\pgfqpoint{7.447110in}{1.209567in}}{\pgfqpoint{7.456318in}{1.218776in}}%
\pgfpathcurveto{\pgfqpoint{7.465527in}{1.227984in}}{\pgfqpoint{7.470701in}{1.240475in}}{\pgfqpoint{7.470701in}{1.253498in}}%
\pgfpathcurveto{\pgfqpoint{7.470701in}{1.266520in}}{\pgfqpoint{7.465527in}{1.279012in}}{\pgfqpoint{7.456318in}{1.288220in}}%
\pgfpathcurveto{\pgfqpoint{7.447110in}{1.297428in}}{\pgfqpoint{7.434619in}{1.302602in}}{\pgfqpoint{7.421596in}{1.302602in}}%
\pgfpathcurveto{\pgfqpoint{7.408573in}{1.302602in}}{\pgfqpoint{7.396082in}{1.297428in}}{\pgfqpoint{7.386874in}{1.288220in}}%
\pgfpathcurveto{\pgfqpoint{7.377666in}{1.279012in}}{\pgfqpoint{7.372492in}{1.266520in}}{\pgfqpoint{7.372492in}{1.253498in}}%
\pgfpathcurveto{\pgfqpoint{7.372492in}{1.240475in}}{\pgfqpoint{7.377666in}{1.227984in}}{\pgfqpoint{7.386874in}{1.218776in}}%
\pgfpathcurveto{\pgfqpoint{7.396082in}{1.209567in}}{\pgfqpoint{7.408573in}{1.204393in}}{\pgfqpoint{7.421596in}{1.204393in}}%
\pgfpathlineto{\pgfqpoint{7.421596in}{1.204393in}}%
\pgfpathclose%
\pgfusepath{stroke,fill}%
\end{pgfscope}%
\begin{pgfscope}%
\pgfpathrectangle{\pgfqpoint{0.786164in}{0.768110in}}{\pgfqpoint{8.851069in}{7.081890in}}%
\pgfusepath{clip}%
\pgfsetbuttcap%
\pgfsetroundjoin%
\definecolor{currentfill}{rgb}{0.185556,0.418570,0.556753}%
\pgfsetfillcolor{currentfill}%
\pgfsetfillopacity{0.700000}%
\pgfsetlinewidth{0.501875pt}%
\definecolor{currentstroke}{rgb}{1.000000,1.000000,1.000000}%
\pgfsetstrokecolor{currentstroke}%
\pgfsetstrokeopacity{0.700000}%
\pgfsetdash{}{0pt}%
\pgfpathmoveto{\pgfqpoint{7.552122in}{1.247077in}}%
\pgfpathcurveto{\pgfqpoint{7.565144in}{1.247077in}}{\pgfqpoint{7.577635in}{1.252251in}}{\pgfqpoint{7.586844in}{1.261460in}}%
\pgfpathcurveto{\pgfqpoint{7.596052in}{1.270668in}}{\pgfqpoint{7.601226in}{1.283159in}}{\pgfqpoint{7.601226in}{1.296182in}}%
\pgfpathcurveto{\pgfqpoint{7.601226in}{1.309205in}}{\pgfqpoint{7.596052in}{1.321696in}}{\pgfqpoint{7.586844in}{1.330904in}}%
\pgfpathcurveto{\pgfqpoint{7.577635in}{1.340113in}}{\pgfqpoint{7.565144in}{1.345287in}}{\pgfqpoint{7.552122in}{1.345287in}}%
\pgfpathcurveto{\pgfqpoint{7.539099in}{1.345287in}}{\pgfqpoint{7.526608in}{1.340113in}}{\pgfqpoint{7.517399in}{1.330904in}}%
\pgfpathcurveto{\pgfqpoint{7.508191in}{1.321696in}}{\pgfqpoint{7.503017in}{1.309205in}}{\pgfqpoint{7.503017in}{1.296182in}}%
\pgfpathcurveto{\pgfqpoint{7.503017in}{1.283159in}}{\pgfqpoint{7.508191in}{1.270668in}}{\pgfqpoint{7.517399in}{1.261460in}}%
\pgfpathcurveto{\pgfqpoint{7.526608in}{1.252251in}}{\pgfqpoint{7.539099in}{1.247077in}}{\pgfqpoint{7.552122in}{1.247077in}}%
\pgfpathlineto{\pgfqpoint{7.552122in}{1.247077in}}%
\pgfpathclose%
\pgfusepath{stroke,fill}%
\end{pgfscope}%
\begin{pgfscope}%
\pgfpathrectangle{\pgfqpoint{0.786164in}{0.768110in}}{\pgfqpoint{8.851069in}{7.081890in}}%
\pgfusepath{clip}%
\pgfsetbuttcap%
\pgfsetroundjoin%
\definecolor{currentfill}{rgb}{0.182256,0.426184,0.557120}%
\pgfsetfillcolor{currentfill}%
\pgfsetfillopacity{0.700000}%
\pgfsetlinewidth{0.501875pt}%
\definecolor{currentstroke}{rgb}{1.000000,1.000000,1.000000}%
\pgfsetstrokecolor{currentstroke}%
\pgfsetstrokeopacity{0.700000}%
\pgfsetdash{}{0pt}%
\pgfpathmoveto{\pgfqpoint{7.598154in}{1.289762in}}%
\pgfpathcurveto{\pgfqpoint{7.611176in}{1.289762in}}{\pgfqpoint{7.623667in}{1.294936in}}{\pgfqpoint{7.632876in}{1.304144in}}%
\pgfpathcurveto{\pgfqpoint{7.642084in}{1.313353in}}{\pgfqpoint{7.647258in}{1.325844in}}{\pgfqpoint{7.647258in}{1.338866in}}%
\pgfpathcurveto{\pgfqpoint{7.647258in}{1.351889in}}{\pgfqpoint{7.642084in}{1.364380in}}{\pgfqpoint{7.632876in}{1.373589in}}%
\pgfpathcurveto{\pgfqpoint{7.623667in}{1.382797in}}{\pgfqpoint{7.611176in}{1.387971in}}{\pgfqpoint{7.598154in}{1.387971in}}%
\pgfpathcurveto{\pgfqpoint{7.585131in}{1.387971in}}{\pgfqpoint{7.572640in}{1.382797in}}{\pgfqpoint{7.563431in}{1.373589in}}%
\pgfpathcurveto{\pgfqpoint{7.554223in}{1.364380in}}{\pgfqpoint{7.549049in}{1.351889in}}{\pgfqpoint{7.549049in}{1.338866in}}%
\pgfpathcurveto{\pgfqpoint{7.549049in}{1.325844in}}{\pgfqpoint{7.554223in}{1.313353in}}{\pgfqpoint{7.563431in}{1.304144in}}%
\pgfpathcurveto{\pgfqpoint{7.572640in}{1.294936in}}{\pgfqpoint{7.585131in}{1.289762in}}{\pgfqpoint{7.598154in}{1.289762in}}%
\pgfpathlineto{\pgfqpoint{7.598154in}{1.289762in}}%
\pgfpathclose%
\pgfusepath{stroke,fill}%
\end{pgfscope}%
\begin{pgfscope}%
\pgfpathrectangle{\pgfqpoint{0.786164in}{0.768110in}}{\pgfqpoint{8.851069in}{7.081890in}}%
\pgfusepath{clip}%
\pgfsetbuttcap%
\pgfsetroundjoin%
\definecolor{currentfill}{rgb}{0.179019,0.433756,0.557430}%
\pgfsetfillcolor{currentfill}%
\pgfsetfillopacity{0.700000}%
\pgfsetlinewidth{0.501875pt}%
\definecolor{currentstroke}{rgb}{1.000000,1.000000,1.000000}%
\pgfsetstrokecolor{currentstroke}%
\pgfsetstrokeopacity{0.700000}%
\pgfsetdash{}{0pt}%
\pgfpathmoveto{\pgfqpoint{7.694979in}{1.396473in}}%
\pgfpathcurveto{\pgfqpoint{7.708002in}{1.396473in}}{\pgfqpoint{7.720493in}{1.401647in}}{\pgfqpoint{7.729702in}{1.410855in}}%
\pgfpathcurveto{\pgfqpoint{7.738910in}{1.420064in}}{\pgfqpoint{7.744084in}{1.432555in}}{\pgfqpoint{7.744084in}{1.445577in}}%
\pgfpathcurveto{\pgfqpoint{7.744084in}{1.458600in}}{\pgfqpoint{7.738910in}{1.471091in}}{\pgfqpoint{7.729702in}{1.480300in}}%
\pgfpathcurveto{\pgfqpoint{7.720493in}{1.489508in}}{\pgfqpoint{7.708002in}{1.494682in}}{\pgfqpoint{7.694979in}{1.494682in}}%
\pgfpathcurveto{\pgfqpoint{7.681957in}{1.494682in}}{\pgfqpoint{7.669466in}{1.489508in}}{\pgfqpoint{7.660257in}{1.480300in}}%
\pgfpathcurveto{\pgfqpoint{7.651049in}{1.471091in}}{\pgfqpoint{7.645875in}{1.458600in}}{\pgfqpoint{7.645875in}{1.445577in}}%
\pgfpathcurveto{\pgfqpoint{7.645875in}{1.432555in}}{\pgfqpoint{7.651049in}{1.420064in}}{\pgfqpoint{7.660257in}{1.410855in}}%
\pgfpathcurveto{\pgfqpoint{7.669466in}{1.401647in}}{\pgfqpoint{7.681957in}{1.396473in}}{\pgfqpoint{7.694979in}{1.396473in}}%
\pgfpathlineto{\pgfqpoint{7.694979in}{1.396473in}}%
\pgfpathclose%
\pgfusepath{stroke,fill}%
\end{pgfscope}%
\begin{pgfscope}%
\pgfpathrectangle{\pgfqpoint{0.786164in}{0.768110in}}{\pgfqpoint{8.851069in}{7.081890in}}%
\pgfusepath{clip}%
\pgfsetbuttcap%
\pgfsetroundjoin%
\definecolor{currentfill}{rgb}{0.179019,0.433756,0.557430}%
\pgfsetfillcolor{currentfill}%
\pgfsetfillopacity{0.700000}%
\pgfsetlinewidth{0.501875pt}%
\definecolor{currentstroke}{rgb}{1.000000,1.000000,1.000000}%
\pgfsetstrokecolor{currentstroke}%
\pgfsetstrokeopacity{0.700000}%
\pgfsetdash{}{0pt}%
\pgfpathmoveto{\pgfqpoint{7.759204in}{1.524526in}}%
\pgfpathcurveto{\pgfqpoint{7.772227in}{1.524526in}}{\pgfqpoint{7.784718in}{1.529700in}}{\pgfqpoint{7.793926in}{1.538908in}}%
\pgfpathcurveto{\pgfqpoint{7.803135in}{1.548117in}}{\pgfqpoint{7.808309in}{1.560608in}}{\pgfqpoint{7.808309in}{1.573630in}}%
\pgfpathcurveto{\pgfqpoint{7.808309in}{1.586653in}}{\pgfqpoint{7.803135in}{1.599144in}}{\pgfqpoint{7.793926in}{1.608353in}}%
\pgfpathcurveto{\pgfqpoint{7.784718in}{1.617561in}}{\pgfqpoint{7.772227in}{1.622735in}}{\pgfqpoint{7.759204in}{1.622735in}}%
\pgfpathcurveto{\pgfqpoint{7.746182in}{1.622735in}}{\pgfqpoint{7.733690in}{1.617561in}}{\pgfqpoint{7.724482in}{1.608353in}}%
\pgfpathcurveto{\pgfqpoint{7.715274in}{1.599144in}}{\pgfqpoint{7.710100in}{1.586653in}}{\pgfqpoint{7.710100in}{1.573630in}}%
\pgfpathcurveto{\pgfqpoint{7.710100in}{1.560608in}}{\pgfqpoint{7.715274in}{1.548117in}}{\pgfqpoint{7.724482in}{1.538908in}}%
\pgfpathcurveto{\pgfqpoint{7.733690in}{1.529700in}}{\pgfqpoint{7.746182in}{1.524526in}}{\pgfqpoint{7.759204in}{1.524526in}}%
\pgfpathlineto{\pgfqpoint{7.759204in}{1.524526in}}%
\pgfpathclose%
\pgfusepath{stroke,fill}%
\end{pgfscope}%
\begin{pgfscope}%
\pgfpathrectangle{\pgfqpoint{0.786164in}{0.768110in}}{\pgfqpoint{8.851069in}{7.081890in}}%
\pgfusepath{clip}%
\pgfsetbuttcap%
\pgfsetroundjoin%
\definecolor{currentfill}{rgb}{0.187231,0.414746,0.556547}%
\pgfsetfillcolor{currentfill}%
\pgfsetfillopacity{0.700000}%
\pgfsetlinewidth{0.501875pt}%
\definecolor{currentstroke}{rgb}{1.000000,1.000000,1.000000}%
\pgfsetstrokecolor{currentstroke}%
\pgfsetstrokeopacity{0.700000}%
\pgfsetdash{}{0pt}%
\pgfpathmoveto{\pgfqpoint{7.987410in}{1.439157in}}%
\pgfpathcurveto{\pgfqpoint{8.000433in}{1.439157in}}{\pgfqpoint{8.012924in}{1.444331in}}{\pgfqpoint{8.022132in}{1.453539in}}%
\pgfpathcurveto{\pgfqpoint{8.031341in}{1.462748in}}{\pgfqpoint{8.036515in}{1.475239in}}{\pgfqpoint{8.036515in}{1.488262in}}%
\pgfpathcurveto{\pgfqpoint{8.036515in}{1.501284in}}{\pgfqpoint{8.031341in}{1.513775in}}{\pgfqpoint{8.022132in}{1.522984in}}%
\pgfpathcurveto{\pgfqpoint{8.012924in}{1.532192in}}{\pgfqpoint{8.000433in}{1.537366in}}{\pgfqpoint{7.987410in}{1.537366in}}%
\pgfpathcurveto{\pgfqpoint{7.974387in}{1.537366in}}{\pgfqpoint{7.961896in}{1.532192in}}{\pgfqpoint{7.952688in}{1.522984in}}%
\pgfpathcurveto{\pgfqpoint{7.943479in}{1.513775in}}{\pgfqpoint{7.938306in}{1.501284in}}{\pgfqpoint{7.938306in}{1.488262in}}%
\pgfpathcurveto{\pgfqpoint{7.938306in}{1.475239in}}{\pgfqpoint{7.943479in}{1.462748in}}{\pgfqpoint{7.952688in}{1.453539in}}%
\pgfpathcurveto{\pgfqpoint{7.961896in}{1.444331in}}{\pgfqpoint{7.974387in}{1.439157in}}{\pgfqpoint{7.987410in}{1.439157in}}%
\pgfpathlineto{\pgfqpoint{7.987410in}{1.439157in}}%
\pgfpathclose%
\pgfusepath{stroke,fill}%
\end{pgfscope}%
\begin{pgfscope}%
\pgfpathrectangle{\pgfqpoint{0.786164in}{0.768110in}}{\pgfqpoint{8.851069in}{7.081890in}}%
\pgfusepath{clip}%
\pgfsetbuttcap%
\pgfsetroundjoin%
\definecolor{currentfill}{rgb}{0.210503,0.363727,0.552206}%
\pgfsetfillcolor{currentfill}%
\pgfsetfillopacity{0.700000}%
\pgfsetlinewidth{0.501875pt}%
\definecolor{currentstroke}{rgb}{1.000000,1.000000,1.000000}%
\pgfsetstrokecolor{currentstroke}%
\pgfsetstrokeopacity{0.700000}%
\pgfsetdash{}{0pt}%
\pgfpathmoveto{\pgfqpoint{8.517204in}{1.268420in}}%
\pgfpathcurveto{\pgfqpoint{8.530227in}{1.268420in}}{\pgfqpoint{8.542718in}{1.273594in}}{\pgfqpoint{8.551927in}{1.282802in}}%
\pgfpathcurveto{\pgfqpoint{8.561135in}{1.292011in}}{\pgfqpoint{8.566309in}{1.304502in}}{\pgfqpoint{8.566309in}{1.317524in}}%
\pgfpathcurveto{\pgfqpoint{8.566309in}{1.330547in}}{\pgfqpoint{8.561135in}{1.343038in}}{\pgfqpoint{8.551927in}{1.352247in}}%
\pgfpathcurveto{\pgfqpoint{8.542718in}{1.361455in}}{\pgfqpoint{8.530227in}{1.366629in}}{\pgfqpoint{8.517204in}{1.366629in}}%
\pgfpathcurveto{\pgfqpoint{8.504182in}{1.366629in}}{\pgfqpoint{8.491691in}{1.361455in}}{\pgfqpoint{8.482482in}{1.352247in}}%
\pgfpathcurveto{\pgfqpoint{8.473274in}{1.343038in}}{\pgfqpoint{8.468100in}{1.330547in}}{\pgfqpoint{8.468100in}{1.317524in}}%
\pgfpathcurveto{\pgfqpoint{8.468100in}{1.304502in}}{\pgfqpoint{8.473274in}{1.292011in}}{\pgfqpoint{8.482482in}{1.282802in}}%
\pgfpathcurveto{\pgfqpoint{8.491691in}{1.273594in}}{\pgfqpoint{8.504182in}{1.268420in}}{\pgfqpoint{8.517204in}{1.268420in}}%
\pgfpathlineto{\pgfqpoint{8.517204in}{1.268420in}}%
\pgfpathclose%
\pgfusepath{stroke,fill}%
\end{pgfscope}%
\begin{pgfscope}%
\pgfpathrectangle{\pgfqpoint{0.786164in}{0.768110in}}{\pgfqpoint{8.851069in}{7.081890in}}%
\pgfusepath{clip}%
\pgfsetbuttcap%
\pgfsetroundjoin%
\definecolor{currentfill}{rgb}{0.201239,0.383670,0.554294}%
\pgfsetfillcolor{currentfill}%
\pgfsetfillopacity{0.700000}%
\pgfsetlinewidth{0.501875pt}%
\definecolor{currentstroke}{rgb}{1.000000,1.000000,1.000000}%
\pgfsetstrokecolor{currentstroke}%
\pgfsetstrokeopacity{0.700000}%
\pgfsetdash{}{0pt}%
\pgfpathmoveto{\pgfqpoint{8.830026in}{1.289762in}}%
\pgfpathcurveto{\pgfqpoint{8.843049in}{1.289762in}}{\pgfqpoint{8.855540in}{1.294936in}}{\pgfqpoint{8.864748in}{1.304144in}}%
\pgfpathcurveto{\pgfqpoint{8.873957in}{1.313353in}}{\pgfqpoint{8.879131in}{1.325844in}}{\pgfqpoint{8.879131in}{1.338866in}}%
\pgfpathcurveto{\pgfqpoint{8.879131in}{1.351889in}}{\pgfqpoint{8.873957in}{1.364380in}}{\pgfqpoint{8.864748in}{1.373589in}}%
\pgfpathcurveto{\pgfqpoint{8.855540in}{1.382797in}}{\pgfqpoint{8.843049in}{1.387971in}}{\pgfqpoint{8.830026in}{1.387971in}}%
\pgfpathcurveto{\pgfqpoint{8.817003in}{1.387971in}}{\pgfqpoint{8.804512in}{1.382797in}}{\pgfqpoint{8.795304in}{1.373589in}}%
\pgfpathcurveto{\pgfqpoint{8.786095in}{1.364380in}}{\pgfqpoint{8.780921in}{1.351889in}}{\pgfqpoint{8.780921in}{1.338866in}}%
\pgfpathcurveto{\pgfqpoint{8.780921in}{1.325844in}}{\pgfqpoint{8.786095in}{1.313353in}}{\pgfqpoint{8.795304in}{1.304144in}}%
\pgfpathcurveto{\pgfqpoint{8.804512in}{1.294936in}}{\pgfqpoint{8.817003in}{1.289762in}}{\pgfqpoint{8.830026in}{1.289762in}}%
\pgfpathlineto{\pgfqpoint{8.830026in}{1.289762in}}%
\pgfpathclose%
\pgfusepath{stroke,fill}%
\end{pgfscope}%
\begin{pgfscope}%
\pgfpathrectangle{\pgfqpoint{0.786164in}{0.768110in}}{\pgfqpoint{8.851069in}{7.081890in}}%
\pgfusepath{clip}%
\pgfsetbuttcap%
\pgfsetroundjoin%
\definecolor{currentfill}{rgb}{0.208623,0.367752,0.552675}%
\pgfsetfillcolor{currentfill}%
\pgfsetfillopacity{0.700000}%
\pgfsetlinewidth{0.501875pt}%
\definecolor{currentstroke}{rgb}{1.000000,1.000000,1.000000}%
\pgfsetstrokecolor{currentstroke}%
\pgfsetstrokeopacity{0.700000}%
\pgfsetdash{}{0pt}%
\pgfpathmoveto{\pgfqpoint{9.234911in}{1.311104in}}%
\pgfpathcurveto{\pgfqpoint{9.247934in}{1.311104in}}{\pgfqpoint{9.260425in}{1.316278in}}{\pgfqpoint{9.269634in}{1.325486in}}%
\pgfpathcurveto{\pgfqpoint{9.278842in}{1.334695in}}{\pgfqpoint{9.284016in}{1.347186in}}{\pgfqpoint{9.284016in}{1.360209in}}%
\pgfpathcurveto{\pgfqpoint{9.284016in}{1.373231in}}{\pgfqpoint{9.278842in}{1.385722in}}{\pgfqpoint{9.269634in}{1.394931in}}%
\pgfpathcurveto{\pgfqpoint{9.260425in}{1.404139in}}{\pgfqpoint{9.247934in}{1.409313in}}{\pgfqpoint{9.234911in}{1.409313in}}%
\pgfpathcurveto{\pgfqpoint{9.221889in}{1.409313in}}{\pgfqpoint{9.209398in}{1.404139in}}{\pgfqpoint{9.200189in}{1.394931in}}%
\pgfpathcurveto{\pgfqpoint{9.190981in}{1.385722in}}{\pgfqpoint{9.185807in}{1.373231in}}{\pgfqpoint{9.185807in}{1.360209in}}%
\pgfpathcurveto{\pgfqpoint{9.185807in}{1.347186in}}{\pgfqpoint{9.190981in}{1.334695in}}{\pgfqpoint{9.200189in}{1.325486in}}%
\pgfpathcurveto{\pgfqpoint{9.209398in}{1.316278in}}{\pgfqpoint{9.221889in}{1.311104in}}{\pgfqpoint{9.234911in}{1.311104in}}%
\pgfpathlineto{\pgfqpoint{9.234911in}{1.311104in}}%
\pgfpathclose%
\pgfusepath{stroke,fill}%
\end{pgfscope}%
\begin{pgfscope}%
\pgfpathrectangle{\pgfqpoint{0.786164in}{0.768110in}}{\pgfqpoint{8.851069in}{7.081890in}}%
\pgfusepath{clip}%
\pgfsetbuttcap%
\pgfsetroundjoin%
\definecolor{currentfill}{rgb}{0.239346,0.300855,0.540844}%
\pgfsetfillcolor{currentfill}%
\pgfsetfillopacity{0.700000}%
\pgfsetlinewidth{0.501875pt}%
\definecolor{currentstroke}{rgb}{1.000000,1.000000,1.000000}%
\pgfsetstrokecolor{currentstroke}%
\pgfsetstrokeopacity{0.700000}%
\pgfsetdash{}{0pt}%
\pgfpathmoveto{\pgfqpoint{3.422315in}{2.591634in}}%
\pgfpathcurveto{\pgfqpoint{3.435337in}{2.591634in}}{\pgfqpoint{3.447829in}{2.596808in}}{\pgfqpoint{3.457037in}{2.606017in}}%
\pgfpathcurveto{\pgfqpoint{3.466245in}{2.615225in}}{\pgfqpoint{3.471419in}{2.627716in}}{\pgfqpoint{3.471419in}{2.640739in}}%
\pgfpathcurveto{\pgfqpoint{3.471419in}{2.653762in}}{\pgfqpoint{3.466245in}{2.666253in}}{\pgfqpoint{3.457037in}{2.675461in}}%
\pgfpathcurveto{\pgfqpoint{3.447829in}{2.684670in}}{\pgfqpoint{3.435337in}{2.689844in}}{\pgfqpoint{3.422315in}{2.689844in}}%
\pgfpathcurveto{\pgfqpoint{3.409292in}{2.689844in}}{\pgfqpoint{3.396801in}{2.684670in}}{\pgfqpoint{3.387593in}{2.675461in}}%
\pgfpathcurveto{\pgfqpoint{3.378384in}{2.666253in}}{\pgfqpoint{3.373210in}{2.653762in}}{\pgfqpoint{3.373210in}{2.640739in}}%
\pgfpathcurveto{\pgfqpoint{3.373210in}{2.627716in}}{\pgfqpoint{3.378384in}{2.615225in}}{\pgfqpoint{3.387593in}{2.606017in}}%
\pgfpathcurveto{\pgfqpoint{3.396801in}{2.596808in}}{\pgfqpoint{3.409292in}{2.591634in}}{\pgfqpoint{3.422315in}{2.591634in}}%
\pgfpathlineto{\pgfqpoint{3.422315in}{2.591634in}}%
\pgfpathclose%
\pgfusepath{stroke,fill}%
\end{pgfscope}%
\begin{pgfscope}%
\pgfpathrectangle{\pgfqpoint{0.786164in}{0.768110in}}{\pgfqpoint{8.851069in}{7.081890in}}%
\pgfusepath{clip}%
\pgfsetbuttcap%
\pgfsetroundjoin%
\definecolor{currentfill}{rgb}{0.231674,0.318106,0.544834}%
\pgfsetfillcolor{currentfill}%
\pgfsetfillopacity{0.700000}%
\pgfsetlinewidth{0.501875pt}%
\definecolor{currentstroke}{rgb}{1.000000,1.000000,1.000000}%
\pgfsetstrokecolor{currentstroke}%
\pgfsetstrokeopacity{0.700000}%
\pgfsetdash{}{0pt}%
\pgfpathmoveto{\pgfqpoint{3.594721in}{2.612976in}}%
\pgfpathcurveto{\pgfqpoint{3.607743in}{2.612976in}}{\pgfqpoint{3.620235in}{2.618150in}}{\pgfqpoint{3.629443in}{2.627359in}}%
\pgfpathcurveto{\pgfqpoint{3.638651in}{2.636567in}}{\pgfqpoint{3.643825in}{2.649058in}}{\pgfqpoint{3.643825in}{2.662081in}}%
\pgfpathcurveto{\pgfqpoint{3.643825in}{2.675104in}}{\pgfqpoint{3.638651in}{2.687595in}}{\pgfqpoint{3.629443in}{2.696803in}}%
\pgfpathcurveto{\pgfqpoint{3.620235in}{2.706012in}}{\pgfqpoint{3.607743in}{2.711186in}}{\pgfqpoint{3.594721in}{2.711186in}}%
\pgfpathcurveto{\pgfqpoint{3.581698in}{2.711186in}}{\pgfqpoint{3.569207in}{2.706012in}}{\pgfqpoint{3.559999in}{2.696803in}}%
\pgfpathcurveto{\pgfqpoint{3.550790in}{2.687595in}}{\pgfqpoint{3.545616in}{2.675104in}}{\pgfqpoint{3.545616in}{2.662081in}}%
\pgfpathcurveto{\pgfqpoint{3.545616in}{2.649058in}}{\pgfqpoint{3.550790in}{2.636567in}}{\pgfqpoint{3.559999in}{2.627359in}}%
\pgfpathcurveto{\pgfqpoint{3.569207in}{2.618150in}}{\pgfqpoint{3.581698in}{2.612976in}}{\pgfqpoint{3.594721in}{2.612976in}}%
\pgfpathlineto{\pgfqpoint{3.594721in}{2.612976in}}%
\pgfpathclose%
\pgfusepath{stroke,fill}%
\end{pgfscope}%
\begin{pgfscope}%
\pgfpathrectangle{\pgfqpoint{0.786164in}{0.768110in}}{\pgfqpoint{8.851069in}{7.081890in}}%
\pgfusepath{clip}%
\pgfsetbuttcap%
\pgfsetroundjoin%
\definecolor{currentfill}{rgb}{0.233603,0.313828,0.543914}%
\pgfsetfillcolor{currentfill}%
\pgfsetfillopacity{0.700000}%
\pgfsetlinewidth{0.501875pt}%
\definecolor{currentstroke}{rgb}{1.000000,1.000000,1.000000}%
\pgfsetstrokecolor{currentstroke}%
\pgfsetstrokeopacity{0.700000}%
\pgfsetdash{}{0pt}%
\pgfpathmoveto{\pgfqpoint{3.424635in}{2.634319in}}%
\pgfpathcurveto{\pgfqpoint{3.437657in}{2.634319in}}{\pgfqpoint{3.450148in}{2.639493in}}{\pgfqpoint{3.459357in}{2.648701in}}%
\pgfpathcurveto{\pgfqpoint{3.468565in}{2.657909in}}{\pgfqpoint{3.473739in}{2.670401in}}{\pgfqpoint{3.473739in}{2.683423in}}%
\pgfpathcurveto{\pgfqpoint{3.473739in}{2.696446in}}{\pgfqpoint{3.468565in}{2.708937in}}{\pgfqpoint{3.459357in}{2.718145in}}%
\pgfpathcurveto{\pgfqpoint{3.450148in}{2.727354in}}{\pgfqpoint{3.437657in}{2.732528in}}{\pgfqpoint{3.424635in}{2.732528in}}%
\pgfpathcurveto{\pgfqpoint{3.411612in}{2.732528in}}{\pgfqpoint{3.399121in}{2.727354in}}{\pgfqpoint{3.389912in}{2.718145in}}%
\pgfpathcurveto{\pgfqpoint{3.380704in}{2.708937in}}{\pgfqpoint{3.375530in}{2.696446in}}{\pgfqpoint{3.375530in}{2.683423in}}%
\pgfpathcurveto{\pgfqpoint{3.375530in}{2.670401in}}{\pgfqpoint{3.380704in}{2.657909in}}{\pgfqpoint{3.389912in}{2.648701in}}%
\pgfpathcurveto{\pgfqpoint{3.399121in}{2.639493in}}{\pgfqpoint{3.411612in}{2.634319in}}{\pgfqpoint{3.424635in}{2.634319in}}%
\pgfpathlineto{\pgfqpoint{3.424635in}{2.634319in}}%
\pgfpathclose%
\pgfusepath{stroke,fill}%
\end{pgfscope}%
\begin{pgfscope}%
\pgfpathrectangle{\pgfqpoint{0.786164in}{0.768110in}}{\pgfqpoint{8.851069in}{7.081890in}}%
\pgfusepath{clip}%
\pgfsetbuttcap%
\pgfsetroundjoin%
\definecolor{currentfill}{rgb}{0.231674,0.318106,0.544834}%
\pgfsetfillcolor{currentfill}%
\pgfsetfillopacity{0.700000}%
\pgfsetlinewidth{0.501875pt}%
\definecolor{currentstroke}{rgb}{1.000000,1.000000,1.000000}%
\pgfsetstrokecolor{currentstroke}%
\pgfsetstrokeopacity{0.700000}%
\pgfsetdash{}{0pt}%
\pgfpathmoveto{\pgfqpoint{3.578359in}{2.612976in}}%
\pgfpathcurveto{\pgfqpoint{3.591382in}{2.612976in}}{\pgfqpoint{3.603873in}{2.618150in}}{\pgfqpoint{3.613081in}{2.627359in}}%
\pgfpathcurveto{\pgfqpoint{3.622290in}{2.636567in}}{\pgfqpoint{3.627464in}{2.649058in}}{\pgfqpoint{3.627464in}{2.662081in}}%
\pgfpathcurveto{\pgfqpoint{3.627464in}{2.675104in}}{\pgfqpoint{3.622290in}{2.687595in}}{\pgfqpoint{3.613081in}{2.696803in}}%
\pgfpathcurveto{\pgfqpoint{3.603873in}{2.706012in}}{\pgfqpoint{3.591382in}{2.711186in}}{\pgfqpoint{3.578359in}{2.711186in}}%
\pgfpathcurveto{\pgfqpoint{3.565337in}{2.711186in}}{\pgfqpoint{3.552845in}{2.706012in}}{\pgfqpoint{3.543637in}{2.696803in}}%
\pgfpathcurveto{\pgfqpoint{3.534429in}{2.687595in}}{\pgfqpoint{3.529255in}{2.675104in}}{\pgfqpoint{3.529255in}{2.662081in}}%
\pgfpathcurveto{\pgfqpoint{3.529255in}{2.649058in}}{\pgfqpoint{3.534429in}{2.636567in}}{\pgfqpoint{3.543637in}{2.627359in}}%
\pgfpathcurveto{\pgfqpoint{3.552845in}{2.618150in}}{\pgfqpoint{3.565337in}{2.612976in}}{\pgfqpoint{3.578359in}{2.612976in}}%
\pgfpathlineto{\pgfqpoint{3.578359in}{2.612976in}}%
\pgfpathclose%
\pgfusepath{stroke,fill}%
\end{pgfscope}%
\begin{pgfscope}%
\pgfpathrectangle{\pgfqpoint{0.786164in}{0.768110in}}{\pgfqpoint{8.851069in}{7.081890in}}%
\pgfusepath{clip}%
\pgfsetbuttcap%
\pgfsetroundjoin%
\definecolor{currentfill}{rgb}{0.220057,0.343307,0.549413}%
\pgfsetfillcolor{currentfill}%
\pgfsetfillopacity{0.700000}%
\pgfsetlinewidth{0.501875pt}%
\definecolor{currentstroke}{rgb}{1.000000,1.000000,1.000000}%
\pgfsetstrokecolor{currentstroke}%
\pgfsetstrokeopacity{0.700000}%
\pgfsetdash{}{0pt}%
\pgfpathmoveto{\pgfqpoint{3.452718in}{2.719687in}}%
\pgfpathcurveto{\pgfqpoint{3.465741in}{2.719687in}}{\pgfqpoint{3.478232in}{2.724861in}}{\pgfqpoint{3.487440in}{2.734070in}}%
\pgfpathcurveto{\pgfqpoint{3.496648in}{2.743278in}}{\pgfqpoint{3.501822in}{2.755769in}}{\pgfqpoint{3.501822in}{2.768792in}}%
\pgfpathcurveto{\pgfqpoint{3.501822in}{2.781815in}}{\pgfqpoint{3.496648in}{2.794306in}}{\pgfqpoint{3.487440in}{2.803514in}}%
\pgfpathcurveto{\pgfqpoint{3.478232in}{2.812723in}}{\pgfqpoint{3.465741in}{2.817897in}}{\pgfqpoint{3.452718in}{2.817897in}}%
\pgfpathcurveto{\pgfqpoint{3.439695in}{2.817897in}}{\pgfqpoint{3.427204in}{2.812723in}}{\pgfqpoint{3.417996in}{2.803514in}}%
\pgfpathcurveto{\pgfqpoint{3.408787in}{2.794306in}}{\pgfqpoint{3.403613in}{2.781815in}}{\pgfqpoint{3.403613in}{2.768792in}}%
\pgfpathcurveto{\pgfqpoint{3.403613in}{2.755769in}}{\pgfqpoint{3.408787in}{2.743278in}}{\pgfqpoint{3.417996in}{2.734070in}}%
\pgfpathcurveto{\pgfqpoint{3.427204in}{2.724861in}}{\pgfqpoint{3.439695in}{2.719687in}}{\pgfqpoint{3.452718in}{2.719687in}}%
\pgfpathlineto{\pgfqpoint{3.452718in}{2.719687in}}%
\pgfpathclose%
\pgfusepath{stroke,fill}%
\end{pgfscope}%
\begin{pgfscope}%
\pgfpathrectangle{\pgfqpoint{0.786164in}{0.768110in}}{\pgfqpoint{8.851069in}{7.081890in}}%
\pgfusepath{clip}%
\pgfsetbuttcap%
\pgfsetroundjoin%
\definecolor{currentfill}{rgb}{0.246811,0.283237,0.535941}%
\pgfsetfillcolor{currentfill}%
\pgfsetfillopacity{0.700000}%
\pgfsetlinewidth{0.501875pt}%
\definecolor{currentstroke}{rgb}{1.000000,1.000000,1.000000}%
\pgfsetstrokecolor{currentstroke}%
\pgfsetstrokeopacity{0.700000}%
\pgfsetdash{}{0pt}%
\pgfpathmoveto{\pgfqpoint{3.711449in}{2.484923in}}%
\pgfpathcurveto{\pgfqpoint{3.724472in}{2.484923in}}{\pgfqpoint{3.736963in}{2.490097in}}{\pgfqpoint{3.746171in}{2.499306in}}%
\pgfpathcurveto{\pgfqpoint{3.755380in}{2.508514in}}{\pgfqpoint{3.760554in}{2.521005in}}{\pgfqpoint{3.760554in}{2.534028in}}%
\pgfpathcurveto{\pgfqpoint{3.760554in}{2.547051in}}{\pgfqpoint{3.755380in}{2.559542in}}{\pgfqpoint{3.746171in}{2.568750in}}%
\pgfpathcurveto{\pgfqpoint{3.736963in}{2.577959in}}{\pgfqpoint{3.724472in}{2.583133in}}{\pgfqpoint{3.711449in}{2.583133in}}%
\pgfpathcurveto{\pgfqpoint{3.698426in}{2.583133in}}{\pgfqpoint{3.685935in}{2.577959in}}{\pgfqpoint{3.676727in}{2.568750in}}%
\pgfpathcurveto{\pgfqpoint{3.667518in}{2.559542in}}{\pgfqpoint{3.662344in}{2.547051in}}{\pgfqpoint{3.662344in}{2.534028in}}%
\pgfpathcurveto{\pgfqpoint{3.662344in}{2.521005in}}{\pgfqpoint{3.667518in}{2.508514in}}{\pgfqpoint{3.676727in}{2.499306in}}%
\pgfpathcurveto{\pgfqpoint{3.685935in}{2.490097in}}{\pgfqpoint{3.698426in}{2.484923in}}{\pgfqpoint{3.711449in}{2.484923in}}%
\pgfpathlineto{\pgfqpoint{3.711449in}{2.484923in}}%
\pgfpathclose%
\pgfusepath{stroke,fill}%
\end{pgfscope}%
\begin{pgfscope}%
\pgfpathrectangle{\pgfqpoint{0.786164in}{0.768110in}}{\pgfqpoint{8.851069in}{7.081890in}}%
\pgfusepath{clip}%
\pgfsetbuttcap%
\pgfsetroundjoin%
\definecolor{currentfill}{rgb}{0.243113,0.292092,0.538516}%
\pgfsetfillcolor{currentfill}%
\pgfsetfillopacity{0.700000}%
\pgfsetlinewidth{0.501875pt}%
\definecolor{currentstroke}{rgb}{1.000000,1.000000,1.000000}%
\pgfsetstrokecolor{currentstroke}%
\pgfsetstrokeopacity{0.700000}%
\pgfsetdash{}{0pt}%
\pgfpathmoveto{\pgfqpoint{3.750033in}{2.506266in}}%
\pgfpathcurveto{\pgfqpoint{3.763055in}{2.506266in}}{\pgfqpoint{3.775546in}{2.511440in}}{\pgfqpoint{3.784755in}{2.520648in}}%
\pgfpathcurveto{\pgfqpoint{3.793963in}{2.529856in}}{\pgfqpoint{3.799137in}{2.542347in}}{\pgfqpoint{3.799137in}{2.555370in}}%
\pgfpathcurveto{\pgfqpoint{3.799137in}{2.568393in}}{\pgfqpoint{3.793963in}{2.580884in}}{\pgfqpoint{3.784755in}{2.590092in}}%
\pgfpathcurveto{\pgfqpoint{3.775546in}{2.599301in}}{\pgfqpoint{3.763055in}{2.604475in}}{\pgfqpoint{3.750033in}{2.604475in}}%
\pgfpathcurveto{\pgfqpoint{3.737010in}{2.604475in}}{\pgfqpoint{3.724519in}{2.599301in}}{\pgfqpoint{3.715310in}{2.590092in}}%
\pgfpathcurveto{\pgfqpoint{3.706102in}{2.580884in}}{\pgfqpoint{3.700928in}{2.568393in}}{\pgfqpoint{3.700928in}{2.555370in}}%
\pgfpathcurveto{\pgfqpoint{3.700928in}{2.542347in}}{\pgfqpoint{3.706102in}{2.529856in}}{\pgfqpoint{3.715310in}{2.520648in}}%
\pgfpathcurveto{\pgfqpoint{3.724519in}{2.511440in}}{\pgfqpoint{3.737010in}{2.506266in}}{\pgfqpoint{3.750033in}{2.506266in}}%
\pgfpathlineto{\pgfqpoint{3.750033in}{2.506266in}}%
\pgfpathclose%
\pgfusepath{stroke,fill}%
\end{pgfscope}%
\begin{pgfscope}%
\pgfpathrectangle{\pgfqpoint{0.786164in}{0.768110in}}{\pgfqpoint{8.851069in}{7.081890in}}%
\pgfusepath{clip}%
\pgfsetbuttcap%
\pgfsetroundjoin%
\definecolor{currentfill}{rgb}{0.241237,0.296485,0.539709}%
\pgfsetfillcolor{currentfill}%
\pgfsetfillopacity{0.700000}%
\pgfsetlinewidth{0.501875pt}%
\definecolor{currentstroke}{rgb}{1.000000,1.000000,1.000000}%
\pgfsetstrokecolor{currentstroke}%
\pgfsetstrokeopacity{0.700000}%
\pgfsetdash{}{0pt}%
\pgfpathmoveto{\pgfqpoint{3.732450in}{2.506266in}}%
\pgfpathcurveto{\pgfqpoint{3.745473in}{2.506266in}}{\pgfqpoint{3.757964in}{2.511440in}}{\pgfqpoint{3.767172in}{2.520648in}}%
\pgfpathcurveto{\pgfqpoint{3.776381in}{2.529856in}}{\pgfqpoint{3.781555in}{2.542347in}}{\pgfqpoint{3.781555in}{2.555370in}}%
\pgfpathcurveto{\pgfqpoint{3.781555in}{2.568393in}}{\pgfqpoint{3.776381in}{2.580884in}}{\pgfqpoint{3.767172in}{2.590092in}}%
\pgfpathcurveto{\pgfqpoint{3.757964in}{2.599301in}}{\pgfqpoint{3.745473in}{2.604475in}}{\pgfqpoint{3.732450in}{2.604475in}}%
\pgfpathcurveto{\pgfqpoint{3.719427in}{2.604475in}}{\pgfqpoint{3.706936in}{2.599301in}}{\pgfqpoint{3.697728in}{2.590092in}}%
\pgfpathcurveto{\pgfqpoint{3.688520in}{2.580884in}}{\pgfqpoint{3.683346in}{2.568393in}}{\pgfqpoint{3.683346in}{2.555370in}}%
\pgfpathcurveto{\pgfqpoint{3.683346in}{2.542347in}}{\pgfqpoint{3.688520in}{2.529856in}}{\pgfqpoint{3.697728in}{2.520648in}}%
\pgfpathcurveto{\pgfqpoint{3.706936in}{2.511440in}}{\pgfqpoint{3.719427in}{2.506266in}}{\pgfqpoint{3.732450in}{2.506266in}}%
\pgfpathlineto{\pgfqpoint{3.732450in}{2.506266in}}%
\pgfpathclose%
\pgfusepath{stroke,fill}%
\end{pgfscope}%
\begin{pgfscope}%
\pgfpathrectangle{\pgfqpoint{0.786164in}{0.768110in}}{\pgfqpoint{8.851069in}{7.081890in}}%
\pgfusepath{clip}%
\pgfsetbuttcap%
\pgfsetroundjoin%
\definecolor{currentfill}{rgb}{0.248629,0.278775,0.534556}%
\pgfsetfillcolor{currentfill}%
\pgfsetfillopacity{0.700000}%
\pgfsetlinewidth{0.501875pt}%
\definecolor{currentstroke}{rgb}{1.000000,1.000000,1.000000}%
\pgfsetstrokecolor{currentstroke}%
\pgfsetstrokeopacity{0.700000}%
\pgfsetdash{}{0pt}%
\pgfpathmoveto{\pgfqpoint{3.807420in}{2.442239in}}%
\pgfpathcurveto{\pgfqpoint{3.820443in}{2.442239in}}{\pgfqpoint{3.832934in}{2.447413in}}{\pgfqpoint{3.842142in}{2.456621in}}%
\pgfpathcurveto{\pgfqpoint{3.851351in}{2.465830in}}{\pgfqpoint{3.856525in}{2.478321in}}{\pgfqpoint{3.856525in}{2.491344in}}%
\pgfpathcurveto{\pgfqpoint{3.856525in}{2.504366in}}{\pgfqpoint{3.851351in}{2.516857in}}{\pgfqpoint{3.842142in}{2.526066in}}%
\pgfpathcurveto{\pgfqpoint{3.832934in}{2.535274in}}{\pgfqpoint{3.820443in}{2.540448in}}{\pgfqpoint{3.807420in}{2.540448in}}%
\pgfpathcurveto{\pgfqpoint{3.794397in}{2.540448in}}{\pgfqpoint{3.781906in}{2.535274in}}{\pgfqpoint{3.772698in}{2.526066in}}%
\pgfpathcurveto{\pgfqpoint{3.763489in}{2.516857in}}{\pgfqpoint{3.758315in}{2.504366in}}{\pgfqpoint{3.758315in}{2.491344in}}%
\pgfpathcurveto{\pgfqpoint{3.758315in}{2.478321in}}{\pgfqpoint{3.763489in}{2.465830in}}{\pgfqpoint{3.772698in}{2.456621in}}%
\pgfpathcurveto{\pgfqpoint{3.781906in}{2.447413in}}{\pgfqpoint{3.794397in}{2.442239in}}{\pgfqpoint{3.807420in}{2.442239in}}%
\pgfpathlineto{\pgfqpoint{3.807420in}{2.442239in}}%
\pgfpathclose%
\pgfusepath{stroke,fill}%
\end{pgfscope}%
\begin{pgfscope}%
\pgfpathrectangle{\pgfqpoint{0.786164in}{0.768110in}}{\pgfqpoint{8.851069in}{7.081890in}}%
\pgfusepath{clip}%
\pgfsetbuttcap%
\pgfsetroundjoin%
\definecolor{currentfill}{rgb}{0.252194,0.269783,0.531579}%
\pgfsetfillcolor{currentfill}%
\pgfsetfillopacity{0.700000}%
\pgfsetlinewidth{0.501875pt}%
\definecolor{currentstroke}{rgb}{1.000000,1.000000,1.000000}%
\pgfsetstrokecolor{currentstroke}%
\pgfsetstrokeopacity{0.700000}%
\pgfsetdash{}{0pt}%
\pgfpathmoveto{\pgfqpoint{4.003880in}{2.548950in}}%
\pgfpathcurveto{\pgfqpoint{4.016902in}{2.548950in}}{\pgfqpoint{4.029393in}{2.554124in}}{\pgfqpoint{4.038602in}{2.563332in}}%
\pgfpathcurveto{\pgfqpoint{4.047810in}{2.572541in}}{\pgfqpoint{4.052984in}{2.585032in}}{\pgfqpoint{4.052984in}{2.598055in}}%
\pgfpathcurveto{\pgfqpoint{4.052984in}{2.611077in}}{\pgfqpoint{4.047810in}{2.623568in}}{\pgfqpoint{4.038602in}{2.632777in}}%
\pgfpathcurveto{\pgfqpoint{4.029393in}{2.641985in}}{\pgfqpoint{4.016902in}{2.647159in}}{\pgfqpoint{4.003880in}{2.647159in}}%
\pgfpathcurveto{\pgfqpoint{3.990857in}{2.647159in}}{\pgfqpoint{3.978366in}{2.641985in}}{\pgfqpoint{3.969157in}{2.632777in}}%
\pgfpathcurveto{\pgfqpoint{3.959949in}{2.623568in}}{\pgfqpoint{3.954775in}{2.611077in}}{\pgfqpoint{3.954775in}{2.598055in}}%
\pgfpathcurveto{\pgfqpoint{3.954775in}{2.585032in}}{\pgfqpoint{3.959949in}{2.572541in}}{\pgfqpoint{3.969157in}{2.563332in}}%
\pgfpathcurveto{\pgfqpoint{3.978366in}{2.554124in}}{\pgfqpoint{3.990857in}{2.548950in}}{\pgfqpoint{4.003880in}{2.548950in}}%
\pgfpathlineto{\pgfqpoint{4.003880in}{2.548950in}}%
\pgfpathclose%
\pgfusepath{stroke,fill}%
\end{pgfscope}%
\begin{pgfscope}%
\pgfpathrectangle{\pgfqpoint{0.786164in}{0.768110in}}{\pgfqpoint{8.851069in}{7.081890in}}%
\pgfusepath{clip}%
\pgfsetbuttcap%
\pgfsetroundjoin%
\definecolor{currentfill}{rgb}{0.258965,0.251537,0.524736}%
\pgfsetfillcolor{currentfill}%
\pgfsetfillopacity{0.700000}%
\pgfsetlinewidth{0.501875pt}%
\definecolor{currentstroke}{rgb}{1.000000,1.000000,1.000000}%
\pgfsetstrokecolor{currentstroke}%
\pgfsetstrokeopacity{0.700000}%
\pgfsetdash{}{0pt}%
\pgfpathmoveto{\pgfqpoint{3.918287in}{2.975793in}}%
\pgfpathcurveto{\pgfqpoint{3.931310in}{2.975793in}}{\pgfqpoint{3.943801in}{2.980967in}}{\pgfqpoint{3.953009in}{2.990176in}}%
\pgfpathcurveto{\pgfqpoint{3.962218in}{2.999384in}}{\pgfqpoint{3.967392in}{3.011875in}}{\pgfqpoint{3.967392in}{3.024898in}}%
\pgfpathcurveto{\pgfqpoint{3.967392in}{3.037921in}}{\pgfqpoint{3.962218in}{3.050412in}}{\pgfqpoint{3.953009in}{3.059620in}}%
\pgfpathcurveto{\pgfqpoint{3.943801in}{3.068829in}}{\pgfqpoint{3.931310in}{3.074003in}}{\pgfqpoint{3.918287in}{3.074003in}}%
\pgfpathcurveto{\pgfqpoint{3.905265in}{3.074003in}}{\pgfqpoint{3.892773in}{3.068829in}}{\pgfqpoint{3.883565in}{3.059620in}}%
\pgfpathcurveto{\pgfqpoint{3.874357in}{3.050412in}}{\pgfqpoint{3.869183in}{3.037921in}}{\pgfqpoint{3.869183in}{3.024898in}}%
\pgfpathcurveto{\pgfqpoint{3.869183in}{3.011875in}}{\pgfqpoint{3.874357in}{2.999384in}}{\pgfqpoint{3.883565in}{2.990176in}}%
\pgfpathcurveto{\pgfqpoint{3.892773in}{2.980967in}}{\pgfqpoint{3.905265in}{2.975793in}}{\pgfqpoint{3.918287in}{2.975793in}}%
\pgfpathlineto{\pgfqpoint{3.918287in}{2.975793in}}%
\pgfpathclose%
\pgfusepath{stroke,fill}%
\end{pgfscope}%
\begin{pgfscope}%
\pgfpathrectangle{\pgfqpoint{0.786164in}{0.768110in}}{\pgfqpoint{8.851069in}{7.081890in}}%
\pgfusepath{clip}%
\pgfsetbuttcap%
\pgfsetroundjoin%
\definecolor{currentfill}{rgb}{0.260571,0.246922,0.522828}%
\pgfsetfillcolor{currentfill}%
\pgfsetfillopacity{0.700000}%
\pgfsetlinewidth{0.501875pt}%
\definecolor{currentstroke}{rgb}{1.000000,1.000000,1.000000}%
\pgfsetstrokecolor{currentstroke}%
\pgfsetstrokeopacity{0.700000}%
\pgfsetdash{}{0pt}%
\pgfpathmoveto{\pgfqpoint{3.969569in}{2.997135in}}%
\pgfpathcurveto{\pgfqpoint{3.982592in}{2.997135in}}{\pgfqpoint{3.995083in}{3.002309in}}{\pgfqpoint{4.004292in}{3.011518in}}%
\pgfpathcurveto{\pgfqpoint{4.013500in}{3.020726in}}{\pgfqpoint{4.018674in}{3.033217in}}{\pgfqpoint{4.018674in}{3.046240in}}%
\pgfpathcurveto{\pgfqpoint{4.018674in}{3.059263in}}{\pgfqpoint{4.013500in}{3.071754in}}{\pgfqpoint{4.004292in}{3.080962in}}%
\pgfpathcurveto{\pgfqpoint{3.995083in}{3.090171in}}{\pgfqpoint{3.982592in}{3.095345in}}{\pgfqpoint{3.969569in}{3.095345in}}%
\pgfpathcurveto{\pgfqpoint{3.956547in}{3.095345in}}{\pgfqpoint{3.944056in}{3.090171in}}{\pgfqpoint{3.934847in}{3.080962in}}%
\pgfpathcurveto{\pgfqpoint{3.925639in}{3.071754in}}{\pgfqpoint{3.920465in}{3.059263in}}{\pgfqpoint{3.920465in}{3.046240in}}%
\pgfpathcurveto{\pgfqpoint{3.920465in}{3.033217in}}{\pgfqpoint{3.925639in}{3.020726in}}{\pgfqpoint{3.934847in}{3.011518in}}%
\pgfpathcurveto{\pgfqpoint{3.944056in}{3.002309in}}{\pgfqpoint{3.956547in}{2.997135in}}{\pgfqpoint{3.969569in}{2.997135in}}%
\pgfpathlineto{\pgfqpoint{3.969569in}{2.997135in}}%
\pgfpathclose%
\pgfusepath{stroke,fill}%
\end{pgfscope}%
\begin{pgfscope}%
\pgfpathrectangle{\pgfqpoint{0.786164in}{0.768110in}}{\pgfqpoint{8.851069in}{7.081890in}}%
\pgfusepath{clip}%
\pgfsetbuttcap%
\pgfsetroundjoin%
\definecolor{currentfill}{rgb}{0.255645,0.260703,0.528312}%
\pgfsetfillcolor{currentfill}%
\pgfsetfillopacity{0.700000}%
\pgfsetlinewidth{0.501875pt}%
\definecolor{currentstroke}{rgb}{1.000000,1.000000,1.000000}%
\pgfsetstrokecolor{currentstroke}%
\pgfsetstrokeopacity{0.700000}%
\pgfsetdash{}{0pt}%
\pgfpathmoveto{\pgfqpoint{3.859191in}{3.103846in}}%
\pgfpathcurveto{\pgfqpoint{3.872213in}{3.103846in}}{\pgfqpoint{3.884704in}{3.109020in}}{\pgfqpoint{3.893913in}{3.118229in}}%
\pgfpathcurveto{\pgfqpoint{3.903121in}{3.127437in}}{\pgfqpoint{3.908295in}{3.139928in}}{\pgfqpoint{3.908295in}{3.152951in}}%
\pgfpathcurveto{\pgfqpoint{3.908295in}{3.165974in}}{\pgfqpoint{3.903121in}{3.178465in}}{\pgfqpoint{3.893913in}{3.187673in}}%
\pgfpathcurveto{\pgfqpoint{3.884704in}{3.196882in}}{\pgfqpoint{3.872213in}{3.202056in}}{\pgfqpoint{3.859191in}{3.202056in}}%
\pgfpathcurveto{\pgfqpoint{3.846168in}{3.202056in}}{\pgfqpoint{3.833677in}{3.196882in}}{\pgfqpoint{3.824468in}{3.187673in}}%
\pgfpathcurveto{\pgfqpoint{3.815260in}{3.178465in}}{\pgfqpoint{3.810086in}{3.165974in}}{\pgfqpoint{3.810086in}{3.152951in}}%
\pgfpathcurveto{\pgfqpoint{3.810086in}{3.139928in}}{\pgfqpoint{3.815260in}{3.127437in}}{\pgfqpoint{3.824468in}{3.118229in}}%
\pgfpathcurveto{\pgfqpoint{3.833677in}{3.109020in}}{\pgfqpoint{3.846168in}{3.103846in}}{\pgfqpoint{3.859191in}{3.103846in}}%
\pgfpathlineto{\pgfqpoint{3.859191in}{3.103846in}}%
\pgfpathclose%
\pgfusepath{stroke,fill}%
\end{pgfscope}%
\begin{pgfscope}%
\pgfpathrectangle{\pgfqpoint{0.786164in}{0.768110in}}{\pgfqpoint{8.851069in}{7.081890in}}%
\pgfusepath{clip}%
\pgfsetbuttcap%
\pgfsetroundjoin%
\definecolor{currentfill}{rgb}{0.252194,0.269783,0.531579}%
\pgfsetfillcolor{currentfill}%
\pgfsetfillopacity{0.700000}%
\pgfsetlinewidth{0.501875pt}%
\definecolor{currentstroke}{rgb}{1.000000,1.000000,1.000000}%
\pgfsetstrokecolor{currentstroke}%
\pgfsetstrokeopacity{0.700000}%
\pgfsetdash{}{0pt}%
\pgfpathmoveto{\pgfqpoint{3.820485in}{3.103846in}}%
\pgfpathcurveto{\pgfqpoint{3.833507in}{3.103846in}}{\pgfqpoint{3.845998in}{3.109020in}}{\pgfqpoint{3.855207in}{3.118229in}}%
\pgfpathcurveto{\pgfqpoint{3.864415in}{3.127437in}}{\pgfqpoint{3.869589in}{3.139928in}}{\pgfqpoint{3.869589in}{3.152951in}}%
\pgfpathcurveto{\pgfqpoint{3.869589in}{3.165974in}}{\pgfqpoint{3.864415in}{3.178465in}}{\pgfqpoint{3.855207in}{3.187673in}}%
\pgfpathcurveto{\pgfqpoint{3.845998in}{3.196882in}}{\pgfqpoint{3.833507in}{3.202056in}}{\pgfqpoint{3.820485in}{3.202056in}}%
\pgfpathcurveto{\pgfqpoint{3.807462in}{3.202056in}}{\pgfqpoint{3.794971in}{3.196882in}}{\pgfqpoint{3.785762in}{3.187673in}}%
\pgfpathcurveto{\pgfqpoint{3.776554in}{3.178465in}}{\pgfqpoint{3.771380in}{3.165974in}}{\pgfqpoint{3.771380in}{3.152951in}}%
\pgfpathcurveto{\pgfqpoint{3.771380in}{3.139928in}}{\pgfqpoint{3.776554in}{3.127437in}}{\pgfqpoint{3.785762in}{3.118229in}}%
\pgfpathcurveto{\pgfqpoint{3.794971in}{3.109020in}}{\pgfqpoint{3.807462in}{3.103846in}}{\pgfqpoint{3.820485in}{3.103846in}}%
\pgfpathlineto{\pgfqpoint{3.820485in}{3.103846in}}%
\pgfpathclose%
\pgfusepath{stroke,fill}%
\end{pgfscope}%
\begin{pgfscope}%
\pgfpathrectangle{\pgfqpoint{0.786164in}{0.768110in}}{\pgfqpoint{8.851069in}{7.081890in}}%
\pgfusepath{clip}%
\pgfsetbuttcap%
\pgfsetroundjoin%
\definecolor{currentfill}{rgb}{0.253935,0.265254,0.529983}%
\pgfsetfillcolor{currentfill}%
\pgfsetfillopacity{0.700000}%
\pgfsetlinewidth{0.501875pt}%
\definecolor{currentstroke}{rgb}{1.000000,1.000000,1.000000}%
\pgfsetstrokecolor{currentstroke}%
\pgfsetstrokeopacity{0.700000}%
\pgfsetdash{}{0pt}%
\pgfpathmoveto{\pgfqpoint{3.786297in}{3.082504in}}%
\pgfpathcurveto{\pgfqpoint{3.799319in}{3.082504in}}{\pgfqpoint{3.811810in}{3.087678in}}{\pgfqpoint{3.821019in}{3.096887in}}%
\pgfpathcurveto{\pgfqpoint{3.830227in}{3.106095in}}{\pgfqpoint{3.835401in}{3.118586in}}{\pgfqpoint{3.835401in}{3.131609in}}%
\pgfpathcurveto{\pgfqpoint{3.835401in}{3.144631in}}{\pgfqpoint{3.830227in}{3.157123in}}{\pgfqpoint{3.821019in}{3.166331in}}%
\pgfpathcurveto{\pgfqpoint{3.811810in}{3.175539in}}{\pgfqpoint{3.799319in}{3.180713in}}{\pgfqpoint{3.786297in}{3.180713in}}%
\pgfpathcurveto{\pgfqpoint{3.773274in}{3.180713in}}{\pgfqpoint{3.760783in}{3.175539in}}{\pgfqpoint{3.751574in}{3.166331in}}%
\pgfpathcurveto{\pgfqpoint{3.742366in}{3.157123in}}{\pgfqpoint{3.737192in}{3.144631in}}{\pgfqpoint{3.737192in}{3.131609in}}%
\pgfpathcurveto{\pgfqpoint{3.737192in}{3.118586in}}{\pgfqpoint{3.742366in}{3.106095in}}{\pgfqpoint{3.751574in}{3.096887in}}%
\pgfpathcurveto{\pgfqpoint{3.760783in}{3.087678in}}{\pgfqpoint{3.773274in}{3.082504in}}{\pgfqpoint{3.786297in}{3.082504in}}%
\pgfpathlineto{\pgfqpoint{3.786297in}{3.082504in}}%
\pgfpathclose%
\pgfusepath{stroke,fill}%
\end{pgfscope}%
\begin{pgfscope}%
\pgfpathrectangle{\pgfqpoint{0.786164in}{0.768110in}}{\pgfqpoint{8.851069in}{7.081890in}}%
\pgfusepath{clip}%
\pgfsetbuttcap%
\pgfsetroundjoin%
\definecolor{currentfill}{rgb}{0.258965,0.251537,0.524736}%
\pgfsetfillcolor{currentfill}%
\pgfsetfillopacity{0.700000}%
\pgfsetlinewidth{0.501875pt}%
\definecolor{currentstroke}{rgb}{1.000000,1.000000,1.000000}%
\pgfsetstrokecolor{currentstroke}%
\pgfsetstrokeopacity{0.700000}%
\pgfsetdash{}{0pt}%
\pgfpathmoveto{\pgfqpoint{3.858336in}{2.869082in}}%
\pgfpathcurveto{\pgfqpoint{3.871359in}{2.869082in}}{\pgfqpoint{3.883850in}{2.874256in}}{\pgfqpoint{3.893058in}{2.883465in}}%
\pgfpathcurveto{\pgfqpoint{3.902266in}{2.892673in}}{\pgfqpoint{3.907440in}{2.905164in}}{\pgfqpoint{3.907440in}{2.918187in}}%
\pgfpathcurveto{\pgfqpoint{3.907440in}{2.931210in}}{\pgfqpoint{3.902266in}{2.943701in}}{\pgfqpoint{3.893058in}{2.952909in}}%
\pgfpathcurveto{\pgfqpoint{3.883850in}{2.962118in}}{\pgfqpoint{3.871359in}{2.967292in}}{\pgfqpoint{3.858336in}{2.967292in}}%
\pgfpathcurveto{\pgfqpoint{3.845313in}{2.967292in}}{\pgfqpoint{3.832822in}{2.962118in}}{\pgfqpoint{3.823614in}{2.952909in}}%
\pgfpathcurveto{\pgfqpoint{3.814405in}{2.943701in}}{\pgfqpoint{3.809231in}{2.931210in}}{\pgfqpoint{3.809231in}{2.918187in}}%
\pgfpathcurveto{\pgfqpoint{3.809231in}{2.905164in}}{\pgfqpoint{3.814405in}{2.892673in}}{\pgfqpoint{3.823614in}{2.883465in}}%
\pgfpathcurveto{\pgfqpoint{3.832822in}{2.874256in}}{\pgfqpoint{3.845313in}{2.869082in}}{\pgfqpoint{3.858336in}{2.869082in}}%
\pgfpathlineto{\pgfqpoint{3.858336in}{2.869082in}}%
\pgfpathclose%
\pgfusepath{stroke,fill}%
\end{pgfscope}%
\begin{pgfscope}%
\pgfpathrectangle{\pgfqpoint{0.786164in}{0.768110in}}{\pgfqpoint{8.851069in}{7.081890in}}%
\pgfusepath{clip}%
\pgfsetbuttcap%
\pgfsetroundjoin%
\definecolor{currentfill}{rgb}{0.266580,0.228262,0.514349}%
\pgfsetfillcolor{currentfill}%
\pgfsetfillopacity{0.700000}%
\pgfsetlinewidth{0.501875pt}%
\definecolor{currentstroke}{rgb}{1.000000,1.000000,1.000000}%
\pgfsetstrokecolor{currentstroke}%
\pgfsetstrokeopacity{0.700000}%
\pgfsetdash{}{0pt}%
\pgfpathmoveto{\pgfqpoint{4.228545in}{2.762372in}}%
\pgfpathcurveto{\pgfqpoint{4.241567in}{2.762372in}}{\pgfqpoint{4.254058in}{2.767546in}}{\pgfqpoint{4.263267in}{2.776754in}}%
\pgfpathcurveto{\pgfqpoint{4.272475in}{2.785962in}}{\pgfqpoint{4.277649in}{2.798454in}}{\pgfqpoint{4.277649in}{2.811476in}}%
\pgfpathcurveto{\pgfqpoint{4.277649in}{2.824499in}}{\pgfqpoint{4.272475in}{2.836990in}}{\pgfqpoint{4.263267in}{2.846198in}}%
\pgfpathcurveto{\pgfqpoint{4.254058in}{2.855407in}}{\pgfqpoint{4.241567in}{2.860581in}}{\pgfqpoint{4.228545in}{2.860581in}}%
\pgfpathcurveto{\pgfqpoint{4.215522in}{2.860581in}}{\pgfqpoint{4.203031in}{2.855407in}}{\pgfqpoint{4.193822in}{2.846198in}}%
\pgfpathcurveto{\pgfqpoint{4.184614in}{2.836990in}}{\pgfqpoint{4.179440in}{2.824499in}}{\pgfqpoint{4.179440in}{2.811476in}}%
\pgfpathcurveto{\pgfqpoint{4.179440in}{2.798454in}}{\pgfqpoint{4.184614in}{2.785962in}}{\pgfqpoint{4.193822in}{2.776754in}}%
\pgfpathcurveto{\pgfqpoint{4.203031in}{2.767546in}}{\pgfqpoint{4.215522in}{2.762372in}}{\pgfqpoint{4.228545in}{2.762372in}}%
\pgfpathlineto{\pgfqpoint{4.228545in}{2.762372in}}%
\pgfpathclose%
\pgfusepath{stroke,fill}%
\end{pgfscope}%
\begin{pgfscope}%
\pgfpathrectangle{\pgfqpoint{0.786164in}{0.768110in}}{\pgfqpoint{8.851069in}{7.081890in}}%
\pgfusepath{clip}%
\pgfsetbuttcap%
\pgfsetroundjoin%
\definecolor{currentfill}{rgb}{0.263663,0.237631,0.518762}%
\pgfsetfillcolor{currentfill}%
\pgfsetfillopacity{0.700000}%
\pgfsetlinewidth{0.501875pt}%
\definecolor{currentstroke}{rgb}{1.000000,1.000000,1.000000}%
\pgfsetstrokecolor{currentstroke}%
\pgfsetstrokeopacity{0.700000}%
\pgfsetdash{}{0pt}%
\pgfpathmoveto{\pgfqpoint{4.228545in}{2.783714in}}%
\pgfpathcurveto{\pgfqpoint{4.241567in}{2.783714in}}{\pgfqpoint{4.254058in}{2.788888in}}{\pgfqpoint{4.263267in}{2.798096in}}%
\pgfpathcurveto{\pgfqpoint{4.272475in}{2.807305in}}{\pgfqpoint{4.277649in}{2.819796in}}{\pgfqpoint{4.277649in}{2.832818in}}%
\pgfpathcurveto{\pgfqpoint{4.277649in}{2.845841in}}{\pgfqpoint{4.272475in}{2.858332in}}{\pgfqpoint{4.263267in}{2.867541in}}%
\pgfpathcurveto{\pgfqpoint{4.254058in}{2.876749in}}{\pgfqpoint{4.241567in}{2.881923in}}{\pgfqpoint{4.228545in}{2.881923in}}%
\pgfpathcurveto{\pgfqpoint{4.215522in}{2.881923in}}{\pgfqpoint{4.203031in}{2.876749in}}{\pgfqpoint{4.193822in}{2.867541in}}%
\pgfpathcurveto{\pgfqpoint{4.184614in}{2.858332in}}{\pgfqpoint{4.179440in}{2.845841in}}{\pgfqpoint{4.179440in}{2.832818in}}%
\pgfpathcurveto{\pgfqpoint{4.179440in}{2.819796in}}{\pgfqpoint{4.184614in}{2.807305in}}{\pgfqpoint{4.193822in}{2.798096in}}%
\pgfpathcurveto{\pgfqpoint{4.203031in}{2.788888in}}{\pgfqpoint{4.215522in}{2.783714in}}{\pgfqpoint{4.228545in}{2.783714in}}%
\pgfpathlineto{\pgfqpoint{4.228545in}{2.783714in}}%
\pgfpathclose%
\pgfusepath{stroke,fill}%
\end{pgfscope}%
\begin{pgfscope}%
\pgfpathrectangle{\pgfqpoint{0.786164in}{0.768110in}}{\pgfqpoint{8.851069in}{7.081890in}}%
\pgfusepath{clip}%
\pgfsetbuttcap%
\pgfsetroundjoin%
\definecolor{currentfill}{rgb}{0.266580,0.228262,0.514349}%
\pgfsetfillcolor{currentfill}%
\pgfsetfillopacity{0.700000}%
\pgfsetlinewidth{0.501875pt}%
\definecolor{currentstroke}{rgb}{1.000000,1.000000,1.000000}%
\pgfsetstrokecolor{currentstroke}%
\pgfsetstrokeopacity{0.700000}%
\pgfsetdash{}{0pt}%
\pgfpathmoveto{\pgfqpoint{4.228789in}{2.741029in}}%
\pgfpathcurveto{\pgfqpoint{4.241812in}{2.741029in}}{\pgfqpoint{4.254303in}{2.746203in}}{\pgfqpoint{4.263511in}{2.755412in}}%
\pgfpathcurveto{\pgfqpoint{4.272720in}{2.764620in}}{\pgfqpoint{4.277894in}{2.777111in}}{\pgfqpoint{4.277894in}{2.790134in}}%
\pgfpathcurveto{\pgfqpoint{4.277894in}{2.803157in}}{\pgfqpoint{4.272720in}{2.815648in}}{\pgfqpoint{4.263511in}{2.824856in}}%
\pgfpathcurveto{\pgfqpoint{4.254303in}{2.834065in}}{\pgfqpoint{4.241812in}{2.839239in}}{\pgfqpoint{4.228789in}{2.839239in}}%
\pgfpathcurveto{\pgfqpoint{4.215766in}{2.839239in}}{\pgfqpoint{4.203275in}{2.834065in}}{\pgfqpoint{4.194067in}{2.824856in}}%
\pgfpathcurveto{\pgfqpoint{4.184858in}{2.815648in}}{\pgfqpoint{4.179684in}{2.803157in}}{\pgfqpoint{4.179684in}{2.790134in}}%
\pgfpathcurveto{\pgfqpoint{4.179684in}{2.777111in}}{\pgfqpoint{4.184858in}{2.764620in}}{\pgfqpoint{4.194067in}{2.755412in}}%
\pgfpathcurveto{\pgfqpoint{4.203275in}{2.746203in}}{\pgfqpoint{4.215766in}{2.741029in}}{\pgfqpoint{4.228789in}{2.741029in}}%
\pgfpathlineto{\pgfqpoint{4.228789in}{2.741029in}}%
\pgfpathclose%
\pgfusepath{stroke,fill}%
\end{pgfscope}%
\begin{pgfscope}%
\pgfpathrectangle{\pgfqpoint{0.786164in}{0.768110in}}{\pgfqpoint{8.851069in}{7.081890in}}%
\pgfusepath{clip}%
\pgfsetbuttcap%
\pgfsetroundjoin%
\definecolor{currentfill}{rgb}{0.257322,0.256130,0.526563}%
\pgfsetfillcolor{currentfill}%
\pgfsetfillopacity{0.700000}%
\pgfsetlinewidth{0.501875pt}%
\definecolor{currentstroke}{rgb}{1.000000,1.000000,1.000000}%
\pgfsetstrokecolor{currentstroke}%
\pgfsetstrokeopacity{0.700000}%
\pgfsetdash{}{0pt}%
\pgfpathmoveto{\pgfqpoint{1.956376in}{2.869082in}}%
\pgfpathcurveto{\pgfqpoint{1.969398in}{2.869082in}}{\pgfqpoint{1.981889in}{2.874256in}}{\pgfqpoint{1.991098in}{2.883465in}}%
\pgfpathcurveto{\pgfqpoint{2.000306in}{2.892673in}}{\pgfqpoint{2.005480in}{2.905164in}}{\pgfqpoint{2.005480in}{2.918187in}}%
\pgfpathcurveto{\pgfqpoint{2.005480in}{2.931210in}}{\pgfqpoint{2.000306in}{2.943701in}}{\pgfqpoint{1.991098in}{2.952909in}}%
\pgfpathcurveto{\pgfqpoint{1.981889in}{2.962118in}}{\pgfqpoint{1.969398in}{2.967292in}}{\pgfqpoint{1.956376in}{2.967292in}}%
\pgfpathcurveto{\pgfqpoint{1.943353in}{2.967292in}}{\pgfqpoint{1.930862in}{2.962118in}}{\pgfqpoint{1.921653in}{2.952909in}}%
\pgfpathcurveto{\pgfqpoint{1.912445in}{2.943701in}}{\pgfqpoint{1.907271in}{2.931210in}}{\pgfqpoint{1.907271in}{2.918187in}}%
\pgfpathcurveto{\pgfqpoint{1.907271in}{2.905164in}}{\pgfqpoint{1.912445in}{2.892673in}}{\pgfqpoint{1.921653in}{2.883465in}}%
\pgfpathcurveto{\pgfqpoint{1.930862in}{2.874256in}}{\pgfqpoint{1.943353in}{2.869082in}}{\pgfqpoint{1.956376in}{2.869082in}}%
\pgfpathlineto{\pgfqpoint{1.956376in}{2.869082in}}%
\pgfpathclose%
\pgfusepath{stroke,fill}%
\end{pgfscope}%
\begin{pgfscope}%
\pgfpathrectangle{\pgfqpoint{0.786164in}{0.768110in}}{\pgfqpoint{8.851069in}{7.081890in}}%
\pgfusepath{clip}%
\pgfsetbuttcap%
\pgfsetroundjoin%
\definecolor{currentfill}{rgb}{0.252194,0.269783,0.531579}%
\pgfsetfillcolor{currentfill}%
\pgfsetfillopacity{0.700000}%
\pgfsetlinewidth{0.501875pt}%
\definecolor{currentstroke}{rgb}{1.000000,1.000000,1.000000}%
\pgfsetstrokecolor{currentstroke}%
\pgfsetstrokeopacity{0.700000}%
\pgfsetdash{}{0pt}%
\pgfpathmoveto{\pgfqpoint{1.952590in}{3.039820in}}%
\pgfpathcurveto{\pgfqpoint{1.965613in}{3.039820in}}{\pgfqpoint{1.978104in}{3.044994in}}{\pgfqpoint{1.987313in}{3.054202in}}%
\pgfpathcurveto{\pgfqpoint{1.996521in}{3.063411in}}{\pgfqpoint{2.001695in}{3.075902in}}{\pgfqpoint{2.001695in}{3.088924in}}%
\pgfpathcurveto{\pgfqpoint{2.001695in}{3.101947in}}{\pgfqpoint{1.996521in}{3.114438in}}{\pgfqpoint{1.987313in}{3.123647in}}%
\pgfpathcurveto{\pgfqpoint{1.978104in}{3.132855in}}{\pgfqpoint{1.965613in}{3.138029in}}{\pgfqpoint{1.952590in}{3.138029in}}%
\pgfpathcurveto{\pgfqpoint{1.939568in}{3.138029in}}{\pgfqpoint{1.927077in}{3.132855in}}{\pgfqpoint{1.917868in}{3.123647in}}%
\pgfpathcurveto{\pgfqpoint{1.908660in}{3.114438in}}{\pgfqpoint{1.903486in}{3.101947in}}{\pgfqpoint{1.903486in}{3.088924in}}%
\pgfpathcurveto{\pgfqpoint{1.903486in}{3.075902in}}{\pgfqpoint{1.908660in}{3.063411in}}{\pgfqpoint{1.917868in}{3.054202in}}%
\pgfpathcurveto{\pgfqpoint{1.927077in}{3.044994in}}{\pgfqpoint{1.939568in}{3.039820in}}{\pgfqpoint{1.952590in}{3.039820in}}%
\pgfpathlineto{\pgfqpoint{1.952590in}{3.039820in}}%
\pgfpathclose%
\pgfusepath{stroke,fill}%
\end{pgfscope}%
\begin{pgfscope}%
\pgfpathrectangle{\pgfqpoint{0.786164in}{0.768110in}}{\pgfqpoint{8.851069in}{7.081890in}}%
\pgfusepath{clip}%
\pgfsetbuttcap%
\pgfsetroundjoin%
\definecolor{currentfill}{rgb}{0.253935,0.265254,0.529983}%
\pgfsetfillcolor{currentfill}%
\pgfsetfillopacity{0.700000}%
\pgfsetlinewidth{0.501875pt}%
\definecolor{currentstroke}{rgb}{1.000000,1.000000,1.000000}%
\pgfsetstrokecolor{currentstroke}%
\pgfsetstrokeopacity{0.700000}%
\pgfsetdash{}{0pt}%
\pgfpathmoveto{\pgfqpoint{1.979941in}{2.911767in}}%
\pgfpathcurveto{\pgfqpoint{1.992964in}{2.911767in}}{\pgfqpoint{2.005455in}{2.916941in}}{\pgfqpoint{2.014663in}{2.926149in}}%
\pgfpathcurveto{\pgfqpoint{2.023872in}{2.935358in}}{\pgfqpoint{2.029046in}{2.947849in}}{\pgfqpoint{2.029046in}{2.960871in}}%
\pgfpathcurveto{\pgfqpoint{2.029046in}{2.973894in}}{\pgfqpoint{2.023872in}{2.986385in}}{\pgfqpoint{2.014663in}{2.995594in}}%
\pgfpathcurveto{\pgfqpoint{2.005455in}{3.004802in}}{\pgfqpoint{1.992964in}{3.009976in}}{\pgfqpoint{1.979941in}{3.009976in}}%
\pgfpathcurveto{\pgfqpoint{1.966918in}{3.009976in}}{\pgfqpoint{1.954427in}{3.004802in}}{\pgfqpoint{1.945219in}{2.995594in}}%
\pgfpathcurveto{\pgfqpoint{1.936010in}{2.986385in}}{\pgfqpoint{1.930836in}{2.973894in}}{\pgfqpoint{1.930836in}{2.960871in}}%
\pgfpathcurveto{\pgfqpoint{1.930836in}{2.947849in}}{\pgfqpoint{1.936010in}{2.935358in}}{\pgfqpoint{1.945219in}{2.926149in}}%
\pgfpathcurveto{\pgfqpoint{1.954427in}{2.916941in}}{\pgfqpoint{1.966918in}{2.911767in}}{\pgfqpoint{1.979941in}{2.911767in}}%
\pgfpathlineto{\pgfqpoint{1.979941in}{2.911767in}}%
\pgfpathclose%
\pgfusepath{stroke,fill}%
\end{pgfscope}%
\begin{pgfscope}%
\pgfpathrectangle{\pgfqpoint{0.786164in}{0.768110in}}{\pgfqpoint{8.851069in}{7.081890in}}%
\pgfusepath{clip}%
\pgfsetbuttcap%
\pgfsetroundjoin%
\definecolor{currentfill}{rgb}{0.260571,0.246922,0.522828}%
\pgfsetfillcolor{currentfill}%
\pgfsetfillopacity{0.700000}%
\pgfsetlinewidth{0.501875pt}%
\definecolor{currentstroke}{rgb}{1.000000,1.000000,1.000000}%
\pgfsetstrokecolor{currentstroke}%
\pgfsetstrokeopacity{0.700000}%
\pgfsetdash{}{0pt}%
\pgfpathmoveto{\pgfqpoint{2.124264in}{2.847740in}}%
\pgfpathcurveto{\pgfqpoint{2.137287in}{2.847740in}}{\pgfqpoint{2.149778in}{2.852914in}}{\pgfqpoint{2.158986in}{2.862123in}}%
\pgfpathcurveto{\pgfqpoint{2.168194in}{2.871331in}}{\pgfqpoint{2.173368in}{2.883822in}}{\pgfqpoint{2.173368in}{2.896845in}}%
\pgfpathcurveto{\pgfqpoint{2.173368in}{2.909868in}}{\pgfqpoint{2.168194in}{2.922359in}}{\pgfqpoint{2.158986in}{2.931567in}}%
\pgfpathcurveto{\pgfqpoint{2.149778in}{2.940776in}}{\pgfqpoint{2.137287in}{2.945950in}}{\pgfqpoint{2.124264in}{2.945950in}}%
\pgfpathcurveto{\pgfqpoint{2.111241in}{2.945950in}}{\pgfqpoint{2.098750in}{2.940776in}}{\pgfqpoint{2.089542in}{2.931567in}}%
\pgfpathcurveto{\pgfqpoint{2.080333in}{2.922359in}}{\pgfqpoint{2.075159in}{2.909868in}}{\pgfqpoint{2.075159in}{2.896845in}}%
\pgfpathcurveto{\pgfqpoint{2.075159in}{2.883822in}}{\pgfqpoint{2.080333in}{2.871331in}}{\pgfqpoint{2.089542in}{2.862123in}}%
\pgfpathcurveto{\pgfqpoint{2.098750in}{2.852914in}}{\pgfqpoint{2.111241in}{2.847740in}}{\pgfqpoint{2.124264in}{2.847740in}}%
\pgfpathlineto{\pgfqpoint{2.124264in}{2.847740in}}%
\pgfpathclose%
\pgfusepath{stroke,fill}%
\end{pgfscope}%
\begin{pgfscope}%
\pgfpathrectangle{\pgfqpoint{0.786164in}{0.768110in}}{\pgfqpoint{8.851069in}{7.081890in}}%
\pgfusepath{clip}%
\pgfsetbuttcap%
\pgfsetroundjoin%
\definecolor{currentfill}{rgb}{0.257322,0.256130,0.526563}%
\pgfsetfillcolor{currentfill}%
\pgfsetfillopacity{0.700000}%
\pgfsetlinewidth{0.501875pt}%
\definecolor{currentstroke}{rgb}{1.000000,1.000000,1.000000}%
\pgfsetstrokecolor{currentstroke}%
\pgfsetstrokeopacity{0.700000}%
\pgfsetdash{}{0pt}%
\pgfpathmoveto{\pgfqpoint{2.119013in}{2.911767in}}%
\pgfpathcurveto{\pgfqpoint{2.132036in}{2.911767in}}{\pgfqpoint{2.144527in}{2.916941in}}{\pgfqpoint{2.153736in}{2.926149in}}%
\pgfpathcurveto{\pgfqpoint{2.162944in}{2.935358in}}{\pgfqpoint{2.168118in}{2.947849in}}{\pgfqpoint{2.168118in}{2.960871in}}%
\pgfpathcurveto{\pgfqpoint{2.168118in}{2.973894in}}{\pgfqpoint{2.162944in}{2.986385in}}{\pgfqpoint{2.153736in}{2.995594in}}%
\pgfpathcurveto{\pgfqpoint{2.144527in}{3.004802in}}{\pgfqpoint{2.132036in}{3.009976in}}{\pgfqpoint{2.119013in}{3.009976in}}%
\pgfpathcurveto{\pgfqpoint{2.105991in}{3.009976in}}{\pgfqpoint{2.093500in}{3.004802in}}{\pgfqpoint{2.084291in}{2.995594in}}%
\pgfpathcurveto{\pgfqpoint{2.075083in}{2.986385in}}{\pgfqpoint{2.069909in}{2.973894in}}{\pgfqpoint{2.069909in}{2.960871in}}%
\pgfpathcurveto{\pgfqpoint{2.069909in}{2.947849in}}{\pgfqpoint{2.075083in}{2.935358in}}{\pgfqpoint{2.084291in}{2.926149in}}%
\pgfpathcurveto{\pgfqpoint{2.093500in}{2.916941in}}{\pgfqpoint{2.105991in}{2.911767in}}{\pgfqpoint{2.119013in}{2.911767in}}%
\pgfpathlineto{\pgfqpoint{2.119013in}{2.911767in}}%
\pgfpathclose%
\pgfusepath{stroke,fill}%
\end{pgfscope}%
\begin{pgfscope}%
\pgfpathrectangle{\pgfqpoint{0.786164in}{0.768110in}}{\pgfqpoint{8.851069in}{7.081890in}}%
\pgfusepath{clip}%
\pgfsetbuttcap%
\pgfsetroundjoin%
\definecolor{currentfill}{rgb}{0.269308,0.218818,0.509577}%
\pgfsetfillcolor{currentfill}%
\pgfsetfillopacity{0.700000}%
\pgfsetlinewidth{0.501875pt}%
\definecolor{currentstroke}{rgb}{1.000000,1.000000,1.000000}%
\pgfsetstrokecolor{currentstroke}%
\pgfsetstrokeopacity{0.700000}%
\pgfsetdash{}{0pt}%
\pgfpathmoveto{\pgfqpoint{2.319869in}{2.762372in}}%
\pgfpathcurveto{\pgfqpoint{2.332892in}{2.762372in}}{\pgfqpoint{2.345383in}{2.767546in}}{\pgfqpoint{2.354591in}{2.776754in}}%
\pgfpathcurveto{\pgfqpoint{2.363800in}{2.785962in}}{\pgfqpoint{2.368974in}{2.798454in}}{\pgfqpoint{2.368974in}{2.811476in}}%
\pgfpathcurveto{\pgfqpoint{2.368974in}{2.824499in}}{\pgfqpoint{2.363800in}{2.836990in}}{\pgfqpoint{2.354591in}{2.846198in}}%
\pgfpathcurveto{\pgfqpoint{2.345383in}{2.855407in}}{\pgfqpoint{2.332892in}{2.860581in}}{\pgfqpoint{2.319869in}{2.860581in}}%
\pgfpathcurveto{\pgfqpoint{2.306846in}{2.860581in}}{\pgfqpoint{2.294355in}{2.855407in}}{\pgfqpoint{2.285147in}{2.846198in}}%
\pgfpathcurveto{\pgfqpoint{2.275938in}{2.836990in}}{\pgfqpoint{2.270764in}{2.824499in}}{\pgfqpoint{2.270764in}{2.811476in}}%
\pgfpathcurveto{\pgfqpoint{2.270764in}{2.798454in}}{\pgfqpoint{2.275938in}{2.785962in}}{\pgfqpoint{2.285147in}{2.776754in}}%
\pgfpathcurveto{\pgfqpoint{2.294355in}{2.767546in}}{\pgfqpoint{2.306846in}{2.762372in}}{\pgfqpoint{2.319869in}{2.762372in}}%
\pgfpathlineto{\pgfqpoint{2.319869in}{2.762372in}}%
\pgfpathclose%
\pgfusepath{stroke,fill}%
\end{pgfscope}%
\begin{pgfscope}%
\pgfpathrectangle{\pgfqpoint{0.786164in}{0.768110in}}{\pgfqpoint{8.851069in}{7.081890in}}%
\pgfusepath{clip}%
\pgfsetbuttcap%
\pgfsetroundjoin%
\definecolor{currentfill}{rgb}{0.253935,0.265254,0.529983}%
\pgfsetfillcolor{currentfill}%
\pgfsetfillopacity{0.700000}%
\pgfsetlinewidth{0.501875pt}%
\definecolor{currentstroke}{rgb}{1.000000,1.000000,1.000000}%
\pgfsetstrokecolor{currentstroke}%
\pgfsetstrokeopacity{0.700000}%
\pgfsetdash{}{0pt}%
\pgfpathmoveto{\pgfqpoint{2.287024in}{2.826398in}}%
\pgfpathcurveto{\pgfqpoint{2.300047in}{2.826398in}}{\pgfqpoint{2.312538in}{2.831572in}}{\pgfqpoint{2.321746in}{2.840781in}}%
\pgfpathcurveto{\pgfqpoint{2.330955in}{2.849989in}}{\pgfqpoint{2.336128in}{2.862480in}}{\pgfqpoint{2.336128in}{2.875503in}}%
\pgfpathcurveto{\pgfqpoint{2.336128in}{2.888525in}}{\pgfqpoint{2.330955in}{2.901017in}}{\pgfqpoint{2.321746in}{2.910225in}}%
\pgfpathcurveto{\pgfqpoint{2.312538in}{2.919433in}}{\pgfqpoint{2.300047in}{2.924607in}}{\pgfqpoint{2.287024in}{2.924607in}}%
\pgfpathcurveto{\pgfqpoint{2.274001in}{2.924607in}}{\pgfqpoint{2.261510in}{2.919433in}}{\pgfqpoint{2.252302in}{2.910225in}}%
\pgfpathcurveto{\pgfqpoint{2.243093in}{2.901017in}}{\pgfqpoint{2.237919in}{2.888525in}}{\pgfqpoint{2.237919in}{2.875503in}}%
\pgfpathcurveto{\pgfqpoint{2.237919in}{2.862480in}}{\pgfqpoint{2.243093in}{2.849989in}}{\pgfqpoint{2.252302in}{2.840781in}}%
\pgfpathcurveto{\pgfqpoint{2.261510in}{2.831572in}}{\pgfqpoint{2.274001in}{2.826398in}}{\pgfqpoint{2.287024in}{2.826398in}}%
\pgfpathlineto{\pgfqpoint{2.287024in}{2.826398in}}%
\pgfpathclose%
\pgfusepath{stroke,fill}%
\end{pgfscope}%
\begin{pgfscope}%
\pgfpathrectangle{\pgfqpoint{0.786164in}{0.768110in}}{\pgfqpoint{8.851069in}{7.081890in}}%
\pgfusepath{clip}%
\pgfsetbuttcap%
\pgfsetroundjoin%
\definecolor{currentfill}{rgb}{0.260571,0.246922,0.522828}%
\pgfsetfillcolor{currentfill}%
\pgfsetfillopacity{0.700000}%
\pgfsetlinewidth{0.501875pt}%
\definecolor{currentstroke}{rgb}{1.000000,1.000000,1.000000}%
\pgfsetstrokecolor{currentstroke}%
\pgfsetstrokeopacity{0.700000}%
\pgfsetdash{}{0pt}%
\pgfpathmoveto{\pgfqpoint{2.439527in}{2.762372in}}%
\pgfpathcurveto{\pgfqpoint{2.452550in}{2.762372in}}{\pgfqpoint{2.465041in}{2.767546in}}{\pgfqpoint{2.474250in}{2.776754in}}%
\pgfpathcurveto{\pgfqpoint{2.483458in}{2.785962in}}{\pgfqpoint{2.488632in}{2.798454in}}{\pgfqpoint{2.488632in}{2.811476in}}%
\pgfpathcurveto{\pgfqpoint{2.488632in}{2.824499in}}{\pgfqpoint{2.483458in}{2.836990in}}{\pgfqpoint{2.474250in}{2.846198in}}%
\pgfpathcurveto{\pgfqpoint{2.465041in}{2.855407in}}{\pgfqpoint{2.452550in}{2.860581in}}{\pgfqpoint{2.439527in}{2.860581in}}%
\pgfpathcurveto{\pgfqpoint{2.426505in}{2.860581in}}{\pgfqpoint{2.414014in}{2.855407in}}{\pgfqpoint{2.404805in}{2.846198in}}%
\pgfpathcurveto{\pgfqpoint{2.395597in}{2.836990in}}{\pgfqpoint{2.390423in}{2.824499in}}{\pgfqpoint{2.390423in}{2.811476in}}%
\pgfpathcurveto{\pgfqpoint{2.390423in}{2.798454in}}{\pgfqpoint{2.395597in}{2.785962in}}{\pgfqpoint{2.404805in}{2.776754in}}%
\pgfpathcurveto{\pgfqpoint{2.414014in}{2.767546in}}{\pgfqpoint{2.426505in}{2.762372in}}{\pgfqpoint{2.439527in}{2.762372in}}%
\pgfpathlineto{\pgfqpoint{2.439527in}{2.762372in}}%
\pgfpathclose%
\pgfusepath{stroke,fill}%
\end{pgfscope}%
\begin{pgfscope}%
\pgfpathrectangle{\pgfqpoint{0.786164in}{0.768110in}}{\pgfqpoint{8.851069in}{7.081890in}}%
\pgfusepath{clip}%
\pgfsetbuttcap%
\pgfsetroundjoin%
\definecolor{currentfill}{rgb}{0.263663,0.237631,0.518762}%
\pgfsetfillcolor{currentfill}%
\pgfsetfillopacity{0.700000}%
\pgfsetlinewidth{0.501875pt}%
\definecolor{currentstroke}{rgb}{1.000000,1.000000,1.000000}%
\pgfsetstrokecolor{currentstroke}%
\pgfsetstrokeopacity{0.700000}%
\pgfsetdash{}{0pt}%
\pgfpathmoveto{\pgfqpoint{2.452348in}{2.612976in}}%
\pgfpathcurveto{\pgfqpoint{2.465371in}{2.612976in}}{\pgfqpoint{2.477862in}{2.618150in}}{\pgfqpoint{2.487070in}{2.627359in}}%
\pgfpathcurveto{\pgfqpoint{2.496279in}{2.636567in}}{\pgfqpoint{2.501453in}{2.649058in}}{\pgfqpoint{2.501453in}{2.662081in}}%
\pgfpathcurveto{\pgfqpoint{2.501453in}{2.675104in}}{\pgfqpoint{2.496279in}{2.687595in}}{\pgfqpoint{2.487070in}{2.696803in}}%
\pgfpathcurveto{\pgfqpoint{2.477862in}{2.706012in}}{\pgfqpoint{2.465371in}{2.711186in}}{\pgfqpoint{2.452348in}{2.711186in}}%
\pgfpathcurveto{\pgfqpoint{2.439325in}{2.711186in}}{\pgfqpoint{2.426834in}{2.706012in}}{\pgfqpoint{2.417626in}{2.696803in}}%
\pgfpathcurveto{\pgfqpoint{2.408417in}{2.687595in}}{\pgfqpoint{2.403243in}{2.675104in}}{\pgfqpoint{2.403243in}{2.662081in}}%
\pgfpathcurveto{\pgfqpoint{2.403243in}{2.649058in}}{\pgfqpoint{2.408417in}{2.636567in}}{\pgfqpoint{2.417626in}{2.627359in}}%
\pgfpathcurveto{\pgfqpoint{2.426834in}{2.618150in}}{\pgfqpoint{2.439325in}{2.612976in}}{\pgfqpoint{2.452348in}{2.612976in}}%
\pgfpathlineto{\pgfqpoint{2.452348in}{2.612976in}}%
\pgfpathclose%
\pgfusepath{stroke,fill}%
\end{pgfscope}%
\begin{pgfscope}%
\pgfpathrectangle{\pgfqpoint{0.786164in}{0.768110in}}{\pgfqpoint{8.851069in}{7.081890in}}%
\pgfusepath{clip}%
\pgfsetbuttcap%
\pgfsetroundjoin%
\definecolor{currentfill}{rgb}{0.262138,0.242286,0.520837}%
\pgfsetfillcolor{currentfill}%
\pgfsetfillopacity{0.700000}%
\pgfsetlinewidth{0.501875pt}%
\definecolor{currentstroke}{rgb}{1.000000,1.000000,1.000000}%
\pgfsetstrokecolor{currentstroke}%
\pgfsetstrokeopacity{0.700000}%
\pgfsetdash{}{0pt}%
\pgfpathmoveto{\pgfqpoint{2.413276in}{2.570292in}}%
\pgfpathcurveto{\pgfqpoint{2.426299in}{2.570292in}}{\pgfqpoint{2.438790in}{2.575466in}}{\pgfqpoint{2.447998in}{2.584674in}}%
\pgfpathcurveto{\pgfqpoint{2.457206in}{2.593883in}}{\pgfqpoint{2.462380in}{2.606374in}}{\pgfqpoint{2.462380in}{2.619397in}}%
\pgfpathcurveto{\pgfqpoint{2.462380in}{2.632419in}}{\pgfqpoint{2.457206in}{2.644910in}}{\pgfqpoint{2.447998in}{2.654119in}}%
\pgfpathcurveto{\pgfqpoint{2.438790in}{2.663327in}}{\pgfqpoint{2.426299in}{2.668501in}}{\pgfqpoint{2.413276in}{2.668501in}}%
\pgfpathcurveto{\pgfqpoint{2.400253in}{2.668501in}}{\pgfqpoint{2.387762in}{2.663327in}}{\pgfqpoint{2.378554in}{2.654119in}}%
\pgfpathcurveto{\pgfqpoint{2.369345in}{2.644910in}}{\pgfqpoint{2.364171in}{2.632419in}}{\pgfqpoint{2.364171in}{2.619397in}}%
\pgfpathcurveto{\pgfqpoint{2.364171in}{2.606374in}}{\pgfqpoint{2.369345in}{2.593883in}}{\pgfqpoint{2.378554in}{2.584674in}}%
\pgfpathcurveto{\pgfqpoint{2.387762in}{2.575466in}}{\pgfqpoint{2.400253in}{2.570292in}}{\pgfqpoint{2.413276in}{2.570292in}}%
\pgfpathlineto{\pgfqpoint{2.413276in}{2.570292in}}%
\pgfpathclose%
\pgfusepath{stroke,fill}%
\end{pgfscope}%
\begin{pgfscope}%
\pgfpathrectangle{\pgfqpoint{0.786164in}{0.768110in}}{\pgfqpoint{8.851069in}{7.081890in}}%
\pgfusepath{clip}%
\pgfsetbuttcap%
\pgfsetroundjoin%
\definecolor{currentfill}{rgb}{0.271828,0.209303,0.504434}%
\pgfsetfillcolor{currentfill}%
\pgfsetfillopacity{0.700000}%
\pgfsetlinewidth{0.501875pt}%
\definecolor{currentstroke}{rgb}{1.000000,1.000000,1.000000}%
\pgfsetstrokecolor{currentstroke}%
\pgfsetstrokeopacity{0.700000}%
\pgfsetdash{}{0pt}%
\pgfpathmoveto{\pgfqpoint{2.606195in}{2.570292in}}%
\pgfpathcurveto{\pgfqpoint{2.619217in}{2.570292in}}{\pgfqpoint{2.631708in}{2.575466in}}{\pgfqpoint{2.640917in}{2.584674in}}%
\pgfpathcurveto{\pgfqpoint{2.650125in}{2.593883in}}{\pgfqpoint{2.655299in}{2.606374in}}{\pgfqpoint{2.655299in}{2.619397in}}%
\pgfpathcurveto{\pgfqpoint{2.655299in}{2.632419in}}{\pgfqpoint{2.650125in}{2.644910in}}{\pgfqpoint{2.640917in}{2.654119in}}%
\pgfpathcurveto{\pgfqpoint{2.631708in}{2.663327in}}{\pgfqpoint{2.619217in}{2.668501in}}{\pgfqpoint{2.606195in}{2.668501in}}%
\pgfpathcurveto{\pgfqpoint{2.593172in}{2.668501in}}{\pgfqpoint{2.580681in}{2.663327in}}{\pgfqpoint{2.571472in}{2.654119in}}%
\pgfpathcurveto{\pgfqpoint{2.562264in}{2.644910in}}{\pgfqpoint{2.557090in}{2.632419in}}{\pgfqpoint{2.557090in}{2.619397in}}%
\pgfpathcurveto{\pgfqpoint{2.557090in}{2.606374in}}{\pgfqpoint{2.562264in}{2.593883in}}{\pgfqpoint{2.571472in}{2.584674in}}%
\pgfpathcurveto{\pgfqpoint{2.580681in}{2.575466in}}{\pgfqpoint{2.593172in}{2.570292in}}{\pgfqpoint{2.606195in}{2.570292in}}%
\pgfpathlineto{\pgfqpoint{2.606195in}{2.570292in}}%
\pgfpathclose%
\pgfusepath{stroke,fill}%
\end{pgfscope}%
\begin{pgfscope}%
\pgfpathrectangle{\pgfqpoint{0.786164in}{0.768110in}}{\pgfqpoint{8.851069in}{7.081890in}}%
\pgfusepath{clip}%
\pgfsetbuttcap%
\pgfsetroundjoin%
\definecolor{currentfill}{rgb}{0.267968,0.223549,0.512008}%
\pgfsetfillcolor{currentfill}%
\pgfsetfillopacity{0.700000}%
\pgfsetlinewidth{0.501875pt}%
\definecolor{currentstroke}{rgb}{1.000000,1.000000,1.000000}%
\pgfsetstrokecolor{currentstroke}%
\pgfsetstrokeopacity{0.700000}%
\pgfsetdash{}{0pt}%
\pgfpathmoveto{\pgfqpoint{2.748930in}{2.591634in}}%
\pgfpathcurveto{\pgfqpoint{2.761953in}{2.591634in}}{\pgfqpoint{2.774444in}{2.596808in}}{\pgfqpoint{2.783652in}{2.606017in}}%
\pgfpathcurveto{\pgfqpoint{2.792861in}{2.615225in}}{\pgfqpoint{2.798035in}{2.627716in}}{\pgfqpoint{2.798035in}{2.640739in}}%
\pgfpathcurveto{\pgfqpoint{2.798035in}{2.653762in}}{\pgfqpoint{2.792861in}{2.666253in}}{\pgfqpoint{2.783652in}{2.675461in}}%
\pgfpathcurveto{\pgfqpoint{2.774444in}{2.684670in}}{\pgfqpoint{2.761953in}{2.689844in}}{\pgfqpoint{2.748930in}{2.689844in}}%
\pgfpathcurveto{\pgfqpoint{2.735908in}{2.689844in}}{\pgfqpoint{2.723416in}{2.684670in}}{\pgfqpoint{2.714208in}{2.675461in}}%
\pgfpathcurveto{\pgfqpoint{2.705000in}{2.666253in}}{\pgfqpoint{2.699826in}{2.653762in}}{\pgfqpoint{2.699826in}{2.640739in}}%
\pgfpathcurveto{\pgfqpoint{2.699826in}{2.627716in}}{\pgfqpoint{2.705000in}{2.615225in}}{\pgfqpoint{2.714208in}{2.606017in}}%
\pgfpathcurveto{\pgfqpoint{2.723416in}{2.596808in}}{\pgfqpoint{2.735908in}{2.591634in}}{\pgfqpoint{2.748930in}{2.591634in}}%
\pgfpathlineto{\pgfqpoint{2.748930in}{2.591634in}}%
\pgfpathclose%
\pgfusepath{stroke,fill}%
\end{pgfscope}%
\begin{pgfscope}%
\pgfpathrectangle{\pgfqpoint{0.786164in}{0.768110in}}{\pgfqpoint{8.851069in}{7.081890in}}%
\pgfusepath{clip}%
\pgfsetbuttcap%
\pgfsetroundjoin%
\definecolor{currentfill}{rgb}{0.266580,0.228262,0.514349}%
\pgfsetfillcolor{currentfill}%
\pgfsetfillopacity{0.700000}%
\pgfsetlinewidth{0.501875pt}%
\definecolor{currentstroke}{rgb}{1.000000,1.000000,1.000000}%
\pgfsetstrokecolor{currentstroke}%
\pgfsetstrokeopacity{0.700000}%
\pgfsetdash{}{0pt}%
\pgfpathmoveto{\pgfqpoint{2.644778in}{2.612976in}}%
\pgfpathcurveto{\pgfqpoint{2.657801in}{2.612976in}}{\pgfqpoint{2.670292in}{2.618150in}}{\pgfqpoint{2.679501in}{2.627359in}}%
\pgfpathcurveto{\pgfqpoint{2.688709in}{2.636567in}}{\pgfqpoint{2.693883in}{2.649058in}}{\pgfqpoint{2.693883in}{2.662081in}}%
\pgfpathcurveto{\pgfqpoint{2.693883in}{2.675104in}}{\pgfqpoint{2.688709in}{2.687595in}}{\pgfqpoint{2.679501in}{2.696803in}}%
\pgfpathcurveto{\pgfqpoint{2.670292in}{2.706012in}}{\pgfqpoint{2.657801in}{2.711186in}}{\pgfqpoint{2.644778in}{2.711186in}}%
\pgfpathcurveto{\pgfqpoint{2.631756in}{2.711186in}}{\pgfqpoint{2.619265in}{2.706012in}}{\pgfqpoint{2.610056in}{2.696803in}}%
\pgfpathcurveto{\pgfqpoint{2.600848in}{2.687595in}}{\pgfqpoint{2.595674in}{2.675104in}}{\pgfqpoint{2.595674in}{2.662081in}}%
\pgfpathcurveto{\pgfqpoint{2.595674in}{2.649058in}}{\pgfqpoint{2.600848in}{2.636567in}}{\pgfqpoint{2.610056in}{2.627359in}}%
\pgfpathcurveto{\pgfqpoint{2.619265in}{2.618150in}}{\pgfqpoint{2.631756in}{2.612976in}}{\pgfqpoint{2.644778in}{2.612976in}}%
\pgfpathlineto{\pgfqpoint{2.644778in}{2.612976in}}%
\pgfpathclose%
\pgfusepath{stroke,fill}%
\end{pgfscope}%
\begin{pgfscope}%
\pgfpathrectangle{\pgfqpoint{0.786164in}{0.768110in}}{\pgfqpoint{8.851069in}{7.081890in}}%
\pgfusepath{clip}%
\pgfsetbuttcap%
\pgfsetroundjoin%
\definecolor{currentfill}{rgb}{0.260571,0.246922,0.522828}%
\pgfsetfillcolor{currentfill}%
\pgfsetfillopacity{0.700000}%
\pgfsetlinewidth{0.501875pt}%
\definecolor{currentstroke}{rgb}{1.000000,1.000000,1.000000}%
\pgfsetstrokecolor{currentstroke}%
\pgfsetstrokeopacity{0.700000}%
\pgfsetdash{}{0pt}%
\pgfpathmoveto{\pgfqpoint{2.575914in}{2.655661in}}%
\pgfpathcurveto{\pgfqpoint{2.588936in}{2.655661in}}{\pgfqpoint{2.601428in}{2.660835in}}{\pgfqpoint{2.610636in}{2.670043in}}%
\pgfpathcurveto{\pgfqpoint{2.619844in}{2.679252in}}{\pgfqpoint{2.625018in}{2.691743in}}{\pgfqpoint{2.625018in}{2.704765in}}%
\pgfpathcurveto{\pgfqpoint{2.625018in}{2.717788in}}{\pgfqpoint{2.619844in}{2.730279in}}{\pgfqpoint{2.610636in}{2.739488in}}%
\pgfpathcurveto{\pgfqpoint{2.601428in}{2.748696in}}{\pgfqpoint{2.588936in}{2.753870in}}{\pgfqpoint{2.575914in}{2.753870in}}%
\pgfpathcurveto{\pgfqpoint{2.562891in}{2.753870in}}{\pgfqpoint{2.550400in}{2.748696in}}{\pgfqpoint{2.541192in}{2.739488in}}%
\pgfpathcurveto{\pgfqpoint{2.531983in}{2.730279in}}{\pgfqpoint{2.526809in}{2.717788in}}{\pgfqpoint{2.526809in}{2.704765in}}%
\pgfpathcurveto{\pgfqpoint{2.526809in}{2.691743in}}{\pgfqpoint{2.531983in}{2.679252in}}{\pgfqpoint{2.541192in}{2.670043in}}%
\pgfpathcurveto{\pgfqpoint{2.550400in}{2.660835in}}{\pgfqpoint{2.562891in}{2.655661in}}{\pgfqpoint{2.575914in}{2.655661in}}%
\pgfpathlineto{\pgfqpoint{2.575914in}{2.655661in}}%
\pgfpathclose%
\pgfusepath{stroke,fill}%
\end{pgfscope}%
\begin{pgfscope}%
\pgfpathrectangle{\pgfqpoint{0.786164in}{0.768110in}}{\pgfqpoint{8.851069in}{7.081890in}}%
\pgfusepath{clip}%
\pgfsetbuttcap%
\pgfsetroundjoin%
\definecolor{currentfill}{rgb}{0.263663,0.237631,0.518762}%
\pgfsetfillcolor{currentfill}%
\pgfsetfillopacity{0.700000}%
\pgfsetlinewidth{0.501875pt}%
\definecolor{currentstroke}{rgb}{1.000000,1.000000,1.000000}%
\pgfsetstrokecolor{currentstroke}%
\pgfsetstrokeopacity{0.700000}%
\pgfsetdash{}{0pt}%
\pgfpathmoveto{\pgfqpoint{2.628539in}{2.677003in}}%
\pgfpathcurveto{\pgfqpoint{2.641562in}{2.677003in}}{\pgfqpoint{2.654053in}{2.682177in}}{\pgfqpoint{2.663261in}{2.691385in}}%
\pgfpathcurveto{\pgfqpoint{2.672470in}{2.700594in}}{\pgfqpoint{2.677644in}{2.713085in}}{\pgfqpoint{2.677644in}{2.726108in}}%
\pgfpathcurveto{\pgfqpoint{2.677644in}{2.739130in}}{\pgfqpoint{2.672470in}{2.751621in}}{\pgfqpoint{2.663261in}{2.760830in}}%
\pgfpathcurveto{\pgfqpoint{2.654053in}{2.770038in}}{\pgfqpoint{2.641562in}{2.775212in}}{\pgfqpoint{2.628539in}{2.775212in}}%
\pgfpathcurveto{\pgfqpoint{2.615516in}{2.775212in}}{\pgfqpoint{2.603025in}{2.770038in}}{\pgfqpoint{2.593817in}{2.760830in}}%
\pgfpathcurveto{\pgfqpoint{2.584608in}{2.751621in}}{\pgfqpoint{2.579434in}{2.739130in}}{\pgfqpoint{2.579434in}{2.726108in}}%
\pgfpathcurveto{\pgfqpoint{2.579434in}{2.713085in}}{\pgfqpoint{2.584608in}{2.700594in}}{\pgfqpoint{2.593817in}{2.691385in}}%
\pgfpathcurveto{\pgfqpoint{2.603025in}{2.682177in}}{\pgfqpoint{2.615516in}{2.677003in}}{\pgfqpoint{2.628539in}{2.677003in}}%
\pgfpathlineto{\pgfqpoint{2.628539in}{2.677003in}}%
\pgfpathclose%
\pgfusepath{stroke,fill}%
\end{pgfscope}%
\begin{pgfscope}%
\pgfpathrectangle{\pgfqpoint{0.786164in}{0.768110in}}{\pgfqpoint{8.851069in}{7.081890in}}%
\pgfusepath{clip}%
\pgfsetbuttcap%
\pgfsetroundjoin%
\definecolor{currentfill}{rgb}{0.266580,0.228262,0.514349}%
\pgfsetfillcolor{currentfill}%
\pgfsetfillopacity{0.700000}%
\pgfsetlinewidth{0.501875pt}%
\definecolor{currentstroke}{rgb}{1.000000,1.000000,1.000000}%
\pgfsetstrokecolor{currentstroke}%
\pgfsetstrokeopacity{0.700000}%
\pgfsetdash{}{0pt}%
\pgfpathmoveto{\pgfqpoint{3.238798in}{2.548950in}}%
\pgfpathcurveto{\pgfqpoint{3.251820in}{2.548950in}}{\pgfqpoint{3.264311in}{2.554124in}}{\pgfqpoint{3.273520in}{2.563332in}}%
\pgfpathcurveto{\pgfqpoint{3.282728in}{2.572541in}}{\pgfqpoint{3.287902in}{2.585032in}}{\pgfqpoint{3.287902in}{2.598055in}}%
\pgfpathcurveto{\pgfqpoint{3.287902in}{2.611077in}}{\pgfqpoint{3.282728in}{2.623568in}}{\pgfqpoint{3.273520in}{2.632777in}}%
\pgfpathcurveto{\pgfqpoint{3.264311in}{2.641985in}}{\pgfqpoint{3.251820in}{2.647159in}}{\pgfqpoint{3.238798in}{2.647159in}}%
\pgfpathcurveto{\pgfqpoint{3.225775in}{2.647159in}}{\pgfqpoint{3.213284in}{2.641985in}}{\pgfqpoint{3.204075in}{2.632777in}}%
\pgfpathcurveto{\pgfqpoint{3.194867in}{2.623568in}}{\pgfqpoint{3.189693in}{2.611077in}}{\pgfqpoint{3.189693in}{2.598055in}}%
\pgfpathcurveto{\pgfqpoint{3.189693in}{2.585032in}}{\pgfqpoint{3.194867in}{2.572541in}}{\pgfqpoint{3.204075in}{2.563332in}}%
\pgfpathcurveto{\pgfqpoint{3.213284in}{2.554124in}}{\pgfqpoint{3.225775in}{2.548950in}}{\pgfqpoint{3.238798in}{2.548950in}}%
\pgfpathlineto{\pgfqpoint{3.238798in}{2.548950in}}%
\pgfpathclose%
\pgfusepath{stroke,fill}%
\end{pgfscope}%
\begin{pgfscope}%
\pgfpathrectangle{\pgfqpoint{0.786164in}{0.768110in}}{\pgfqpoint{8.851069in}{7.081890in}}%
\pgfusepath{clip}%
\pgfsetbuttcap%
\pgfsetroundjoin%
\definecolor{currentfill}{rgb}{0.273006,0.204520,0.501721}%
\pgfsetfillcolor{currentfill}%
\pgfsetfillopacity{0.700000}%
\pgfsetlinewidth{0.501875pt}%
\definecolor{currentstroke}{rgb}{1.000000,1.000000,1.000000}%
\pgfsetstrokecolor{currentstroke}%
\pgfsetstrokeopacity{0.700000}%
\pgfsetdash{}{0pt}%
\pgfpathmoveto{\pgfqpoint{3.293865in}{2.356870in}}%
\pgfpathcurveto{\pgfqpoint{3.306888in}{2.356870in}}{\pgfqpoint{3.319379in}{2.362044in}}{\pgfqpoint{3.328587in}{2.371253in}}%
\pgfpathcurveto{\pgfqpoint{3.337796in}{2.380461in}}{\pgfqpoint{3.342970in}{2.392952in}}{\pgfqpoint{3.342970in}{2.405975in}}%
\pgfpathcurveto{\pgfqpoint{3.342970in}{2.418998in}}{\pgfqpoint{3.337796in}{2.431489in}}{\pgfqpoint{3.328587in}{2.440697in}}%
\pgfpathcurveto{\pgfqpoint{3.319379in}{2.449906in}}{\pgfqpoint{3.306888in}{2.455080in}}{\pgfqpoint{3.293865in}{2.455080in}}%
\pgfpathcurveto{\pgfqpoint{3.280842in}{2.455080in}}{\pgfqpoint{3.268351in}{2.449906in}}{\pgfqpoint{3.259143in}{2.440697in}}%
\pgfpathcurveto{\pgfqpoint{3.249934in}{2.431489in}}{\pgfqpoint{3.244760in}{2.418998in}}{\pgfqpoint{3.244760in}{2.405975in}}%
\pgfpathcurveto{\pgfqpoint{3.244760in}{2.392952in}}{\pgfqpoint{3.249934in}{2.380461in}}{\pgfqpoint{3.259143in}{2.371253in}}%
\pgfpathcurveto{\pgfqpoint{3.268351in}{2.362044in}}{\pgfqpoint{3.280842in}{2.356870in}}{\pgfqpoint{3.293865in}{2.356870in}}%
\pgfpathlineto{\pgfqpoint{3.293865in}{2.356870in}}%
\pgfpathclose%
\pgfusepath{stroke,fill}%
\end{pgfscope}%
\begin{pgfscope}%
\pgfpathrectangle{\pgfqpoint{0.786164in}{0.768110in}}{\pgfqpoint{8.851069in}{7.081890in}}%
\pgfusepath{clip}%
\pgfsetbuttcap%
\pgfsetroundjoin%
\definecolor{currentfill}{rgb}{0.263663,0.237631,0.518762}%
\pgfsetfillcolor{currentfill}%
\pgfsetfillopacity{0.700000}%
\pgfsetlinewidth{0.501875pt}%
\definecolor{currentstroke}{rgb}{1.000000,1.000000,1.000000}%
\pgfsetstrokecolor{currentstroke}%
\pgfsetstrokeopacity{0.700000}%
\pgfsetdash{}{0pt}%
\pgfpathmoveto{\pgfqpoint{3.302900in}{2.527608in}}%
\pgfpathcurveto{\pgfqpoint{3.315923in}{2.527608in}}{\pgfqpoint{3.328414in}{2.532782in}}{\pgfqpoint{3.337623in}{2.541990in}}%
\pgfpathcurveto{\pgfqpoint{3.346831in}{2.551199in}}{\pgfqpoint{3.352005in}{2.563690in}}{\pgfqpoint{3.352005in}{2.576712in}}%
\pgfpathcurveto{\pgfqpoint{3.352005in}{2.589735in}}{\pgfqpoint{3.346831in}{2.602226in}}{\pgfqpoint{3.337623in}{2.611435in}}%
\pgfpathcurveto{\pgfqpoint{3.328414in}{2.620643in}}{\pgfqpoint{3.315923in}{2.625817in}}{\pgfqpoint{3.302900in}{2.625817in}}%
\pgfpathcurveto{\pgfqpoint{3.289878in}{2.625817in}}{\pgfqpoint{3.277387in}{2.620643in}}{\pgfqpoint{3.268178in}{2.611435in}}%
\pgfpathcurveto{\pgfqpoint{3.258970in}{2.602226in}}{\pgfqpoint{3.253796in}{2.589735in}}{\pgfqpoint{3.253796in}{2.576712in}}%
\pgfpathcurveto{\pgfqpoint{3.253796in}{2.563690in}}{\pgfqpoint{3.258970in}{2.551199in}}{\pgfqpoint{3.268178in}{2.541990in}}%
\pgfpathcurveto{\pgfqpoint{3.277387in}{2.532782in}}{\pgfqpoint{3.289878in}{2.527608in}}{\pgfqpoint{3.302900in}{2.527608in}}%
\pgfpathlineto{\pgfqpoint{3.302900in}{2.527608in}}%
\pgfpathclose%
\pgfusepath{stroke,fill}%
\end{pgfscope}%
\begin{pgfscope}%
\pgfpathrectangle{\pgfqpoint{0.786164in}{0.768110in}}{\pgfqpoint{8.851069in}{7.081890in}}%
\pgfusepath{clip}%
\pgfsetbuttcap%
\pgfsetroundjoin%
\definecolor{currentfill}{rgb}{0.270595,0.214069,0.507052}%
\pgfsetfillcolor{currentfill}%
\pgfsetfillopacity{0.700000}%
\pgfsetlinewidth{0.501875pt}%
\definecolor{currentstroke}{rgb}{1.000000,1.000000,1.000000}%
\pgfsetstrokecolor{currentstroke}%
\pgfsetstrokeopacity{0.700000}%
\pgfsetdash{}{0pt}%
\pgfpathmoveto{\pgfqpoint{3.312913in}{2.356870in}}%
\pgfpathcurveto{\pgfqpoint{3.325935in}{2.356870in}}{\pgfqpoint{3.338426in}{2.362044in}}{\pgfqpoint{3.347635in}{2.371253in}}%
\pgfpathcurveto{\pgfqpoint{3.356843in}{2.380461in}}{\pgfqpoint{3.362017in}{2.392952in}}{\pgfqpoint{3.362017in}{2.405975in}}%
\pgfpathcurveto{\pgfqpoint{3.362017in}{2.418998in}}{\pgfqpoint{3.356843in}{2.431489in}}{\pgfqpoint{3.347635in}{2.440697in}}%
\pgfpathcurveto{\pgfqpoint{3.338426in}{2.449906in}}{\pgfqpoint{3.325935in}{2.455080in}}{\pgfqpoint{3.312913in}{2.455080in}}%
\pgfpathcurveto{\pgfqpoint{3.299890in}{2.455080in}}{\pgfqpoint{3.287399in}{2.449906in}}{\pgfqpoint{3.278190in}{2.440697in}}%
\pgfpathcurveto{\pgfqpoint{3.268982in}{2.431489in}}{\pgfqpoint{3.263808in}{2.418998in}}{\pgfqpoint{3.263808in}{2.405975in}}%
\pgfpathcurveto{\pgfqpoint{3.263808in}{2.392952in}}{\pgfqpoint{3.268982in}{2.380461in}}{\pgfqpoint{3.278190in}{2.371253in}}%
\pgfpathcurveto{\pgfqpoint{3.287399in}{2.362044in}}{\pgfqpoint{3.299890in}{2.356870in}}{\pgfqpoint{3.312913in}{2.356870in}}%
\pgfpathlineto{\pgfqpoint{3.312913in}{2.356870in}}%
\pgfpathclose%
\pgfusepath{stroke,fill}%
\end{pgfscope}%
\begin{pgfscope}%
\pgfpathrectangle{\pgfqpoint{0.786164in}{0.768110in}}{\pgfqpoint{8.851069in}{7.081890in}}%
\pgfusepath{clip}%
\pgfsetrectcap%
\pgfsetroundjoin%
\pgfsetlinewidth{0.803000pt}%
\definecolor{currentstroke}{rgb}{0.690196,0.690196,0.690196}%
\pgfsetstrokecolor{currentstroke}%
\pgfsetdash{}{0pt}%
\pgfpathmoveto{\pgfqpoint{1.176031in}{0.768110in}}%
\pgfpathlineto{\pgfqpoint{1.176031in}{7.850000in}}%
\pgfusepath{stroke}%
\end{pgfscope}%
\begin{pgfscope}%
\pgfsetbuttcap%
\pgfsetroundjoin%
\definecolor{currentfill}{rgb}{0.000000,0.000000,0.000000}%
\pgfsetfillcolor{currentfill}%
\pgfsetlinewidth{0.803000pt}%
\definecolor{currentstroke}{rgb}{0.000000,0.000000,0.000000}%
\pgfsetstrokecolor{currentstroke}%
\pgfsetdash{}{0pt}%
\pgfsys@defobject{currentmarker}{\pgfqpoint{0.000000in}{-0.048611in}}{\pgfqpoint{0.000000in}{0.000000in}}{%
\pgfpathmoveto{\pgfqpoint{0.000000in}{0.000000in}}%
\pgfpathlineto{\pgfqpoint{0.000000in}{-0.048611in}}%
\pgfusepath{stroke,fill}%
}%
\begin{pgfscope}%
\pgfsys@transformshift{1.176031in}{0.768110in}%
\pgfsys@useobject{currentmarker}{}%
\end{pgfscope}%
\end{pgfscope}%
\begin{pgfscope}%
\definecolor{textcolor}{rgb}{0.000000,0.000000,0.000000}%
\pgfsetstrokecolor{textcolor}%
\pgfsetfillcolor{textcolor}%
\pgftext[x=1.176031in,y=0.670888in,,top]{\color{textcolor}{\sffamily\fontsize{15.000000}{18.000000}\selectfont\catcode`\^=\active\def^{\ifmmode\sp\else\^{}\fi}\catcode`\%=\active\def%{\%}0}}%
\end{pgfscope}%
\begin{pgfscope}%
\pgfpathrectangle{\pgfqpoint{0.786164in}{0.768110in}}{\pgfqpoint{8.851069in}{7.081890in}}%
\pgfusepath{clip}%
\pgfsetrectcap%
\pgfsetroundjoin%
\pgfsetlinewidth{0.803000pt}%
\definecolor{currentstroke}{rgb}{0.690196,0.690196,0.690196}%
\pgfsetstrokecolor{currentstroke}%
\pgfsetdash{}{0pt}%
\pgfpathmoveto{\pgfqpoint{2.397036in}{0.768110in}}%
\pgfpathlineto{\pgfqpoint{2.397036in}{7.850000in}}%
\pgfusepath{stroke}%
\end{pgfscope}%
\begin{pgfscope}%
\pgfsetbuttcap%
\pgfsetroundjoin%
\definecolor{currentfill}{rgb}{0.000000,0.000000,0.000000}%
\pgfsetfillcolor{currentfill}%
\pgfsetlinewidth{0.803000pt}%
\definecolor{currentstroke}{rgb}{0.000000,0.000000,0.000000}%
\pgfsetstrokecolor{currentstroke}%
\pgfsetdash{}{0pt}%
\pgfsys@defobject{currentmarker}{\pgfqpoint{0.000000in}{-0.048611in}}{\pgfqpoint{0.000000in}{0.000000in}}{%
\pgfpathmoveto{\pgfqpoint{0.000000in}{0.000000in}}%
\pgfpathlineto{\pgfqpoint{0.000000in}{-0.048611in}}%
\pgfusepath{stroke,fill}%
}%
\begin{pgfscope}%
\pgfsys@transformshift{2.397036in}{0.768110in}%
\pgfsys@useobject{currentmarker}{}%
\end{pgfscope}%
\end{pgfscope}%
\begin{pgfscope}%
\definecolor{textcolor}{rgb}{0.000000,0.000000,0.000000}%
\pgfsetstrokecolor{textcolor}%
\pgfsetfillcolor{textcolor}%
\pgftext[x=2.397036in,y=0.670888in,,top]{\color{textcolor}{\sffamily\fontsize{15.000000}{18.000000}\selectfont\catcode`\^=\active\def^{\ifmmode\sp\else\^{}\fi}\catcode`\%=\active\def%{\%}10}}%
\end{pgfscope}%
\begin{pgfscope}%
\pgfpathrectangle{\pgfqpoint{0.786164in}{0.768110in}}{\pgfqpoint{8.851069in}{7.081890in}}%
\pgfusepath{clip}%
\pgfsetrectcap%
\pgfsetroundjoin%
\pgfsetlinewidth{0.803000pt}%
\definecolor{currentstroke}{rgb}{0.690196,0.690196,0.690196}%
\pgfsetstrokecolor{currentstroke}%
\pgfsetdash{}{0pt}%
\pgfpathmoveto{\pgfqpoint{3.618042in}{0.768110in}}%
\pgfpathlineto{\pgfqpoint{3.618042in}{7.850000in}}%
\pgfusepath{stroke}%
\end{pgfscope}%
\begin{pgfscope}%
\pgfsetbuttcap%
\pgfsetroundjoin%
\definecolor{currentfill}{rgb}{0.000000,0.000000,0.000000}%
\pgfsetfillcolor{currentfill}%
\pgfsetlinewidth{0.803000pt}%
\definecolor{currentstroke}{rgb}{0.000000,0.000000,0.000000}%
\pgfsetstrokecolor{currentstroke}%
\pgfsetdash{}{0pt}%
\pgfsys@defobject{currentmarker}{\pgfqpoint{0.000000in}{-0.048611in}}{\pgfqpoint{0.000000in}{0.000000in}}{%
\pgfpathmoveto{\pgfqpoint{0.000000in}{0.000000in}}%
\pgfpathlineto{\pgfqpoint{0.000000in}{-0.048611in}}%
\pgfusepath{stroke,fill}%
}%
\begin{pgfscope}%
\pgfsys@transformshift{3.618042in}{0.768110in}%
\pgfsys@useobject{currentmarker}{}%
\end{pgfscope}%
\end{pgfscope}%
\begin{pgfscope}%
\definecolor{textcolor}{rgb}{0.000000,0.000000,0.000000}%
\pgfsetstrokecolor{textcolor}%
\pgfsetfillcolor{textcolor}%
\pgftext[x=3.618042in,y=0.670888in,,top]{\color{textcolor}{\sffamily\fontsize{15.000000}{18.000000}\selectfont\catcode`\^=\active\def^{\ifmmode\sp\else\^{}\fi}\catcode`\%=\active\def%{\%}20}}%
\end{pgfscope}%
\begin{pgfscope}%
\pgfpathrectangle{\pgfqpoint{0.786164in}{0.768110in}}{\pgfqpoint{8.851069in}{7.081890in}}%
\pgfusepath{clip}%
\pgfsetrectcap%
\pgfsetroundjoin%
\pgfsetlinewidth{0.803000pt}%
\definecolor{currentstroke}{rgb}{0.690196,0.690196,0.690196}%
\pgfsetstrokecolor{currentstroke}%
\pgfsetdash{}{0pt}%
\pgfpathmoveto{\pgfqpoint{4.839047in}{0.768110in}}%
\pgfpathlineto{\pgfqpoint{4.839047in}{7.850000in}}%
\pgfusepath{stroke}%
\end{pgfscope}%
\begin{pgfscope}%
\pgfsetbuttcap%
\pgfsetroundjoin%
\definecolor{currentfill}{rgb}{0.000000,0.000000,0.000000}%
\pgfsetfillcolor{currentfill}%
\pgfsetlinewidth{0.803000pt}%
\definecolor{currentstroke}{rgb}{0.000000,0.000000,0.000000}%
\pgfsetstrokecolor{currentstroke}%
\pgfsetdash{}{0pt}%
\pgfsys@defobject{currentmarker}{\pgfqpoint{0.000000in}{-0.048611in}}{\pgfqpoint{0.000000in}{0.000000in}}{%
\pgfpathmoveto{\pgfqpoint{0.000000in}{0.000000in}}%
\pgfpathlineto{\pgfqpoint{0.000000in}{-0.048611in}}%
\pgfusepath{stroke,fill}%
}%
\begin{pgfscope}%
\pgfsys@transformshift{4.839047in}{0.768110in}%
\pgfsys@useobject{currentmarker}{}%
\end{pgfscope}%
\end{pgfscope}%
\begin{pgfscope}%
\definecolor{textcolor}{rgb}{0.000000,0.000000,0.000000}%
\pgfsetstrokecolor{textcolor}%
\pgfsetfillcolor{textcolor}%
\pgftext[x=4.839047in,y=0.670888in,,top]{\color{textcolor}{\sffamily\fontsize{15.000000}{18.000000}\selectfont\catcode`\^=\active\def^{\ifmmode\sp\else\^{}\fi}\catcode`\%=\active\def%{\%}30}}%
\end{pgfscope}%
\begin{pgfscope}%
\pgfpathrectangle{\pgfqpoint{0.786164in}{0.768110in}}{\pgfqpoint{8.851069in}{7.081890in}}%
\pgfusepath{clip}%
\pgfsetrectcap%
\pgfsetroundjoin%
\pgfsetlinewidth{0.803000pt}%
\definecolor{currentstroke}{rgb}{0.690196,0.690196,0.690196}%
\pgfsetstrokecolor{currentstroke}%
\pgfsetdash{}{0pt}%
\pgfpathmoveto{\pgfqpoint{6.060053in}{0.768110in}}%
\pgfpathlineto{\pgfqpoint{6.060053in}{7.850000in}}%
\pgfusepath{stroke}%
\end{pgfscope}%
\begin{pgfscope}%
\pgfsetbuttcap%
\pgfsetroundjoin%
\definecolor{currentfill}{rgb}{0.000000,0.000000,0.000000}%
\pgfsetfillcolor{currentfill}%
\pgfsetlinewidth{0.803000pt}%
\definecolor{currentstroke}{rgb}{0.000000,0.000000,0.000000}%
\pgfsetstrokecolor{currentstroke}%
\pgfsetdash{}{0pt}%
\pgfsys@defobject{currentmarker}{\pgfqpoint{0.000000in}{-0.048611in}}{\pgfqpoint{0.000000in}{0.000000in}}{%
\pgfpathmoveto{\pgfqpoint{0.000000in}{0.000000in}}%
\pgfpathlineto{\pgfqpoint{0.000000in}{-0.048611in}}%
\pgfusepath{stroke,fill}%
}%
\begin{pgfscope}%
\pgfsys@transformshift{6.060053in}{0.768110in}%
\pgfsys@useobject{currentmarker}{}%
\end{pgfscope}%
\end{pgfscope}%
\begin{pgfscope}%
\definecolor{textcolor}{rgb}{0.000000,0.000000,0.000000}%
\pgfsetstrokecolor{textcolor}%
\pgfsetfillcolor{textcolor}%
\pgftext[x=6.060053in,y=0.670888in,,top]{\color{textcolor}{\sffamily\fontsize{15.000000}{18.000000}\selectfont\catcode`\^=\active\def^{\ifmmode\sp\else\^{}\fi}\catcode`\%=\active\def%{\%}40}}%
\end{pgfscope}%
\begin{pgfscope}%
\pgfpathrectangle{\pgfqpoint{0.786164in}{0.768110in}}{\pgfqpoint{8.851069in}{7.081890in}}%
\pgfusepath{clip}%
\pgfsetrectcap%
\pgfsetroundjoin%
\pgfsetlinewidth{0.803000pt}%
\definecolor{currentstroke}{rgb}{0.690196,0.690196,0.690196}%
\pgfsetstrokecolor{currentstroke}%
\pgfsetdash{}{0pt}%
\pgfpathmoveto{\pgfqpoint{7.281058in}{0.768110in}}%
\pgfpathlineto{\pgfqpoint{7.281058in}{7.850000in}}%
\pgfusepath{stroke}%
\end{pgfscope}%
\begin{pgfscope}%
\pgfsetbuttcap%
\pgfsetroundjoin%
\definecolor{currentfill}{rgb}{0.000000,0.000000,0.000000}%
\pgfsetfillcolor{currentfill}%
\pgfsetlinewidth{0.803000pt}%
\definecolor{currentstroke}{rgb}{0.000000,0.000000,0.000000}%
\pgfsetstrokecolor{currentstroke}%
\pgfsetdash{}{0pt}%
\pgfsys@defobject{currentmarker}{\pgfqpoint{0.000000in}{-0.048611in}}{\pgfqpoint{0.000000in}{0.000000in}}{%
\pgfpathmoveto{\pgfqpoint{0.000000in}{0.000000in}}%
\pgfpathlineto{\pgfqpoint{0.000000in}{-0.048611in}}%
\pgfusepath{stroke,fill}%
}%
\begin{pgfscope}%
\pgfsys@transformshift{7.281058in}{0.768110in}%
\pgfsys@useobject{currentmarker}{}%
\end{pgfscope}%
\end{pgfscope}%
\begin{pgfscope}%
\definecolor{textcolor}{rgb}{0.000000,0.000000,0.000000}%
\pgfsetstrokecolor{textcolor}%
\pgfsetfillcolor{textcolor}%
\pgftext[x=7.281058in,y=0.670888in,,top]{\color{textcolor}{\sffamily\fontsize{15.000000}{18.000000}\selectfont\catcode`\^=\active\def^{\ifmmode\sp\else\^{}\fi}\catcode`\%=\active\def%{\%}50}}%
\end{pgfscope}%
\begin{pgfscope}%
\pgfpathrectangle{\pgfqpoint{0.786164in}{0.768110in}}{\pgfqpoint{8.851069in}{7.081890in}}%
\pgfusepath{clip}%
\pgfsetrectcap%
\pgfsetroundjoin%
\pgfsetlinewidth{0.803000pt}%
\definecolor{currentstroke}{rgb}{0.690196,0.690196,0.690196}%
\pgfsetstrokecolor{currentstroke}%
\pgfsetdash{}{0pt}%
\pgfpathmoveto{\pgfqpoint{8.502064in}{0.768110in}}%
\pgfpathlineto{\pgfqpoint{8.502064in}{7.850000in}}%
\pgfusepath{stroke}%
\end{pgfscope}%
\begin{pgfscope}%
\pgfsetbuttcap%
\pgfsetroundjoin%
\definecolor{currentfill}{rgb}{0.000000,0.000000,0.000000}%
\pgfsetfillcolor{currentfill}%
\pgfsetlinewidth{0.803000pt}%
\definecolor{currentstroke}{rgb}{0.000000,0.000000,0.000000}%
\pgfsetstrokecolor{currentstroke}%
\pgfsetdash{}{0pt}%
\pgfsys@defobject{currentmarker}{\pgfqpoint{0.000000in}{-0.048611in}}{\pgfqpoint{0.000000in}{0.000000in}}{%
\pgfpathmoveto{\pgfqpoint{0.000000in}{0.000000in}}%
\pgfpathlineto{\pgfqpoint{0.000000in}{-0.048611in}}%
\pgfusepath{stroke,fill}%
}%
\begin{pgfscope}%
\pgfsys@transformshift{8.502064in}{0.768110in}%
\pgfsys@useobject{currentmarker}{}%
\end{pgfscope}%
\end{pgfscope}%
\begin{pgfscope}%
\definecolor{textcolor}{rgb}{0.000000,0.000000,0.000000}%
\pgfsetstrokecolor{textcolor}%
\pgfsetfillcolor{textcolor}%
\pgftext[x=8.502064in,y=0.670888in,,top]{\color{textcolor}{\sffamily\fontsize{15.000000}{18.000000}\selectfont\catcode`\^=\active\def^{\ifmmode\sp\else\^{}\fi}\catcode`\%=\active\def%{\%}60}}%
\end{pgfscope}%
\begin{pgfscope}%
\definecolor{textcolor}{rgb}{0.000000,0.000000,0.000000}%
\pgfsetstrokecolor{textcolor}%
\pgfsetfillcolor{textcolor}%
\pgftext[x=5.211698in,y=0.437555in,,top]{\color{textcolor}{\sffamily\fontsize{20.000000}{24.000000}\selectfont\catcode`\^=\active\def^{\ifmmode\sp\else\^{}\fi}\catcode`\%=\active\def%{\%}Share of Renewable Energy (%)}}%
\end{pgfscope}%
\begin{pgfscope}%
\pgfpathrectangle{\pgfqpoint{0.786164in}{0.768110in}}{\pgfqpoint{8.851069in}{7.081890in}}%
\pgfusepath{clip}%
\pgfsetrectcap%
\pgfsetroundjoin%
\pgfsetlinewidth{0.803000pt}%
\definecolor{currentstroke}{rgb}{0.690196,0.690196,0.690196}%
\pgfsetstrokecolor{currentstroke}%
\pgfsetdash{}{0pt}%
\pgfpathmoveto{\pgfqpoint{0.786164in}{1.232156in}}%
\pgfpathlineto{\pgfqpoint{9.637233in}{1.232156in}}%
\pgfusepath{stroke}%
\end{pgfscope}%
\begin{pgfscope}%
\pgfsetbuttcap%
\pgfsetroundjoin%
\definecolor{currentfill}{rgb}{0.000000,0.000000,0.000000}%
\pgfsetfillcolor{currentfill}%
\pgfsetlinewidth{0.803000pt}%
\definecolor{currentstroke}{rgb}{0.000000,0.000000,0.000000}%
\pgfsetstrokecolor{currentstroke}%
\pgfsetdash{}{0pt}%
\pgfsys@defobject{currentmarker}{\pgfqpoint{-0.048611in}{0.000000in}}{\pgfqpoint{-0.000000in}{0.000000in}}{%
\pgfpathmoveto{\pgfqpoint{-0.000000in}{0.000000in}}%
\pgfpathlineto{\pgfqpoint{-0.048611in}{0.000000in}}%
\pgfusepath{stroke,fill}%
}%
\begin{pgfscope}%
\pgfsys@transformshift{0.786164in}{1.232156in}%
\pgfsys@useobject{currentmarker}{}%
\end{pgfscope}%
\end{pgfscope}%
\begin{pgfscope}%
\definecolor{textcolor}{rgb}{0.000000,0.000000,0.000000}%
\pgfsetstrokecolor{textcolor}%
\pgfsetfillcolor{textcolor}%
\pgftext[x=0.591026in, y=1.162711in, left, base]{\color{textcolor}{\sffamily\fontsize{15.000000}{18.000000}\selectfont\catcode`\^=\active\def^{\ifmmode\sp\else\^{}\fi}\catcode`\%=\active\def%{\%}0}}%
\end{pgfscope}%
\begin{pgfscope}%
\pgfpathrectangle{\pgfqpoint{0.786164in}{0.768110in}}{\pgfqpoint{8.851069in}{7.081890in}}%
\pgfusepath{clip}%
\pgfsetrectcap%
\pgfsetroundjoin%
\pgfsetlinewidth{0.803000pt}%
\definecolor{currentstroke}{rgb}{0.690196,0.690196,0.690196}%
\pgfsetstrokecolor{currentstroke}%
\pgfsetdash{}{0pt}%
\pgfpathmoveto{\pgfqpoint{0.786164in}{2.299264in}}%
\pgfpathlineto{\pgfqpoint{9.637233in}{2.299264in}}%
\pgfusepath{stroke}%
\end{pgfscope}%
\begin{pgfscope}%
\pgfsetbuttcap%
\pgfsetroundjoin%
\definecolor{currentfill}{rgb}{0.000000,0.000000,0.000000}%
\pgfsetfillcolor{currentfill}%
\pgfsetlinewidth{0.803000pt}%
\definecolor{currentstroke}{rgb}{0.000000,0.000000,0.000000}%
\pgfsetstrokecolor{currentstroke}%
\pgfsetdash{}{0pt}%
\pgfsys@defobject{currentmarker}{\pgfqpoint{-0.048611in}{0.000000in}}{\pgfqpoint{-0.000000in}{0.000000in}}{%
\pgfpathmoveto{\pgfqpoint{-0.000000in}{0.000000in}}%
\pgfpathlineto{\pgfqpoint{-0.048611in}{0.000000in}}%
\pgfusepath{stroke,fill}%
}%
\begin{pgfscope}%
\pgfsys@transformshift{0.786164in}{2.299264in}%
\pgfsys@useobject{currentmarker}{}%
\end{pgfscope}%
\end{pgfscope}%
\begin{pgfscope}%
\definecolor{textcolor}{rgb}{0.000000,0.000000,0.000000}%
\pgfsetstrokecolor{textcolor}%
\pgfsetfillcolor{textcolor}%
\pgftext[x=0.591026in, y=2.229820in, left, base]{\color{textcolor}{\sffamily\fontsize{15.000000}{18.000000}\selectfont\catcode`\^=\active\def^{\ifmmode\sp\else\^{}\fi}\catcode`\%=\active\def%{\%}5}}%
\end{pgfscope}%
\begin{pgfscope}%
\pgfpathrectangle{\pgfqpoint{0.786164in}{0.768110in}}{\pgfqpoint{8.851069in}{7.081890in}}%
\pgfusepath{clip}%
\pgfsetrectcap%
\pgfsetroundjoin%
\pgfsetlinewidth{0.803000pt}%
\definecolor{currentstroke}{rgb}{0.690196,0.690196,0.690196}%
\pgfsetstrokecolor{currentstroke}%
\pgfsetdash{}{0pt}%
\pgfpathmoveto{\pgfqpoint{0.786164in}{3.366373in}}%
\pgfpathlineto{\pgfqpoint{9.637233in}{3.366373in}}%
\pgfusepath{stroke}%
\end{pgfscope}%
\begin{pgfscope}%
\pgfsetbuttcap%
\pgfsetroundjoin%
\definecolor{currentfill}{rgb}{0.000000,0.000000,0.000000}%
\pgfsetfillcolor{currentfill}%
\pgfsetlinewidth{0.803000pt}%
\definecolor{currentstroke}{rgb}{0.000000,0.000000,0.000000}%
\pgfsetstrokecolor{currentstroke}%
\pgfsetdash{}{0pt}%
\pgfsys@defobject{currentmarker}{\pgfqpoint{-0.048611in}{0.000000in}}{\pgfqpoint{-0.000000in}{0.000000in}}{%
\pgfpathmoveto{\pgfqpoint{-0.000000in}{0.000000in}}%
\pgfpathlineto{\pgfqpoint{-0.048611in}{0.000000in}}%
\pgfusepath{stroke,fill}%
}%
\begin{pgfscope}%
\pgfsys@transformshift{0.786164in}{3.366373in}%
\pgfsys@useobject{currentmarker}{}%
\end{pgfscope}%
\end{pgfscope}%
\begin{pgfscope}%
\definecolor{textcolor}{rgb}{0.000000,0.000000,0.000000}%
\pgfsetstrokecolor{textcolor}%
\pgfsetfillcolor{textcolor}%
\pgftext[x=0.493111in, y=3.296928in, left, base]{\color{textcolor}{\sffamily\fontsize{15.000000}{18.000000}\selectfont\catcode`\^=\active\def^{\ifmmode\sp\else\^{}\fi}\catcode`\%=\active\def%{\%}10}}%
\end{pgfscope}%
\begin{pgfscope}%
\pgfpathrectangle{\pgfqpoint{0.786164in}{0.768110in}}{\pgfqpoint{8.851069in}{7.081890in}}%
\pgfusepath{clip}%
\pgfsetrectcap%
\pgfsetroundjoin%
\pgfsetlinewidth{0.803000pt}%
\definecolor{currentstroke}{rgb}{0.690196,0.690196,0.690196}%
\pgfsetstrokecolor{currentstroke}%
\pgfsetdash{}{0pt}%
\pgfpathmoveto{\pgfqpoint{0.786164in}{4.433481in}}%
\pgfpathlineto{\pgfqpoint{9.637233in}{4.433481in}}%
\pgfusepath{stroke}%
\end{pgfscope}%
\begin{pgfscope}%
\pgfsetbuttcap%
\pgfsetroundjoin%
\definecolor{currentfill}{rgb}{0.000000,0.000000,0.000000}%
\pgfsetfillcolor{currentfill}%
\pgfsetlinewidth{0.803000pt}%
\definecolor{currentstroke}{rgb}{0.000000,0.000000,0.000000}%
\pgfsetstrokecolor{currentstroke}%
\pgfsetdash{}{0pt}%
\pgfsys@defobject{currentmarker}{\pgfqpoint{-0.048611in}{0.000000in}}{\pgfqpoint{-0.000000in}{0.000000in}}{%
\pgfpathmoveto{\pgfqpoint{-0.000000in}{0.000000in}}%
\pgfpathlineto{\pgfqpoint{-0.048611in}{0.000000in}}%
\pgfusepath{stroke,fill}%
}%
\begin{pgfscope}%
\pgfsys@transformshift{0.786164in}{4.433481in}%
\pgfsys@useobject{currentmarker}{}%
\end{pgfscope}%
\end{pgfscope}%
\begin{pgfscope}%
\definecolor{textcolor}{rgb}{0.000000,0.000000,0.000000}%
\pgfsetstrokecolor{textcolor}%
\pgfsetfillcolor{textcolor}%
\pgftext[x=0.493111in, y=4.364037in, left, base]{\color{textcolor}{\sffamily\fontsize{15.000000}{18.000000}\selectfont\catcode`\^=\active\def^{\ifmmode\sp\else\^{}\fi}\catcode`\%=\active\def%{\%}15}}%
\end{pgfscope}%
\begin{pgfscope}%
\pgfpathrectangle{\pgfqpoint{0.786164in}{0.768110in}}{\pgfqpoint{8.851069in}{7.081890in}}%
\pgfusepath{clip}%
\pgfsetrectcap%
\pgfsetroundjoin%
\pgfsetlinewidth{0.803000pt}%
\definecolor{currentstroke}{rgb}{0.690196,0.690196,0.690196}%
\pgfsetstrokecolor{currentstroke}%
\pgfsetdash{}{0pt}%
\pgfpathmoveto{\pgfqpoint{0.786164in}{5.500590in}}%
\pgfpathlineto{\pgfqpoint{9.637233in}{5.500590in}}%
\pgfusepath{stroke}%
\end{pgfscope}%
\begin{pgfscope}%
\pgfsetbuttcap%
\pgfsetroundjoin%
\definecolor{currentfill}{rgb}{0.000000,0.000000,0.000000}%
\pgfsetfillcolor{currentfill}%
\pgfsetlinewidth{0.803000pt}%
\definecolor{currentstroke}{rgb}{0.000000,0.000000,0.000000}%
\pgfsetstrokecolor{currentstroke}%
\pgfsetdash{}{0pt}%
\pgfsys@defobject{currentmarker}{\pgfqpoint{-0.048611in}{0.000000in}}{\pgfqpoint{-0.000000in}{0.000000in}}{%
\pgfpathmoveto{\pgfqpoint{-0.000000in}{0.000000in}}%
\pgfpathlineto{\pgfqpoint{-0.048611in}{0.000000in}}%
\pgfusepath{stroke,fill}%
}%
\begin{pgfscope}%
\pgfsys@transformshift{0.786164in}{5.500590in}%
\pgfsys@useobject{currentmarker}{}%
\end{pgfscope}%
\end{pgfscope}%
\begin{pgfscope}%
\definecolor{textcolor}{rgb}{0.000000,0.000000,0.000000}%
\pgfsetstrokecolor{textcolor}%
\pgfsetfillcolor{textcolor}%
\pgftext[x=0.493111in, y=5.431145in, left, base]{\color{textcolor}{\sffamily\fontsize{15.000000}{18.000000}\selectfont\catcode`\^=\active\def^{\ifmmode\sp\else\^{}\fi}\catcode`\%=\active\def%{\%}20}}%
\end{pgfscope}%
\begin{pgfscope}%
\pgfpathrectangle{\pgfqpoint{0.786164in}{0.768110in}}{\pgfqpoint{8.851069in}{7.081890in}}%
\pgfusepath{clip}%
\pgfsetrectcap%
\pgfsetroundjoin%
\pgfsetlinewidth{0.803000pt}%
\definecolor{currentstroke}{rgb}{0.690196,0.690196,0.690196}%
\pgfsetstrokecolor{currentstroke}%
\pgfsetdash{}{0pt}%
\pgfpathmoveto{\pgfqpoint{0.786164in}{6.567698in}}%
\pgfpathlineto{\pgfqpoint{9.637233in}{6.567698in}}%
\pgfusepath{stroke}%
\end{pgfscope}%
\begin{pgfscope}%
\pgfsetbuttcap%
\pgfsetroundjoin%
\definecolor{currentfill}{rgb}{0.000000,0.000000,0.000000}%
\pgfsetfillcolor{currentfill}%
\pgfsetlinewidth{0.803000pt}%
\definecolor{currentstroke}{rgb}{0.000000,0.000000,0.000000}%
\pgfsetstrokecolor{currentstroke}%
\pgfsetdash{}{0pt}%
\pgfsys@defobject{currentmarker}{\pgfqpoint{-0.048611in}{0.000000in}}{\pgfqpoint{-0.000000in}{0.000000in}}{%
\pgfpathmoveto{\pgfqpoint{-0.000000in}{0.000000in}}%
\pgfpathlineto{\pgfqpoint{-0.048611in}{0.000000in}}%
\pgfusepath{stroke,fill}%
}%
\begin{pgfscope}%
\pgfsys@transformshift{0.786164in}{6.567698in}%
\pgfsys@useobject{currentmarker}{}%
\end{pgfscope}%
\end{pgfscope}%
\begin{pgfscope}%
\definecolor{textcolor}{rgb}{0.000000,0.000000,0.000000}%
\pgfsetstrokecolor{textcolor}%
\pgfsetfillcolor{textcolor}%
\pgftext[x=0.493111in, y=6.498254in, left, base]{\color{textcolor}{\sffamily\fontsize{15.000000}{18.000000}\selectfont\catcode`\^=\active\def^{\ifmmode\sp\else\^{}\fi}\catcode`\%=\active\def%{\%}25}}%
\end{pgfscope}%
\begin{pgfscope}%
\pgfpathrectangle{\pgfqpoint{0.786164in}{0.768110in}}{\pgfqpoint{8.851069in}{7.081890in}}%
\pgfusepath{clip}%
\pgfsetrectcap%
\pgfsetroundjoin%
\pgfsetlinewidth{0.803000pt}%
\definecolor{currentstroke}{rgb}{0.690196,0.690196,0.690196}%
\pgfsetstrokecolor{currentstroke}%
\pgfsetdash{}{0pt}%
\pgfpathmoveto{\pgfqpoint{0.786164in}{7.634807in}}%
\pgfpathlineto{\pgfqpoint{9.637233in}{7.634807in}}%
\pgfusepath{stroke}%
\end{pgfscope}%
\begin{pgfscope}%
\pgfsetbuttcap%
\pgfsetroundjoin%
\definecolor{currentfill}{rgb}{0.000000,0.000000,0.000000}%
\pgfsetfillcolor{currentfill}%
\pgfsetlinewidth{0.803000pt}%
\definecolor{currentstroke}{rgb}{0.000000,0.000000,0.000000}%
\pgfsetstrokecolor{currentstroke}%
\pgfsetdash{}{0pt}%
\pgfsys@defobject{currentmarker}{\pgfqpoint{-0.048611in}{0.000000in}}{\pgfqpoint{-0.000000in}{0.000000in}}{%
\pgfpathmoveto{\pgfqpoint{-0.000000in}{0.000000in}}%
\pgfpathlineto{\pgfqpoint{-0.048611in}{0.000000in}}%
\pgfusepath{stroke,fill}%
}%
\begin{pgfscope}%
\pgfsys@transformshift{0.786164in}{7.634807in}%
\pgfsys@useobject{currentmarker}{}%
\end{pgfscope}%
\end{pgfscope}%
\begin{pgfscope}%
\definecolor{textcolor}{rgb}{0.000000,0.000000,0.000000}%
\pgfsetstrokecolor{textcolor}%
\pgfsetfillcolor{textcolor}%
\pgftext[x=0.493111in, y=7.565362in, left, base]{\color{textcolor}{\sffamily\fontsize{15.000000}{18.000000}\selectfont\catcode`\^=\active\def^{\ifmmode\sp\else\^{}\fi}\catcode`\%=\active\def%{\%}30}}%
\end{pgfscope}%
\begin{pgfscope}%
\definecolor{textcolor}{rgb}{0.000000,0.000000,0.000000}%
\pgfsetstrokecolor{textcolor}%
\pgfsetfillcolor{textcolor}%
\pgftext[x=0.437555in,y=4.309055in,,bottom,rotate=90.000000]{\color{textcolor}{\sffamily\fontsize{20.000000}{24.000000}\selectfont\catcode`\^=\active\def^{\ifmmode\sp\else\^{}\fi}\catcode`\%=\active\def%{\%}Greenhouse Gas Emissions (tonnes per capita)}}%
\end{pgfscope}%
\begin{pgfscope}%
\pgfpathrectangle{\pgfqpoint{0.786164in}{0.768110in}}{\pgfqpoint{8.851069in}{7.081890in}}%
\pgfusepath{clip}%
\pgfsetrectcap%
\pgfsetroundjoin%
\pgfsetlinewidth{1.505625pt}%
\definecolor{currentstroke}{rgb}{1.000000,0.000000,0.000000}%
\pgfsetstrokecolor{currentstroke}%
\pgfsetdash{}{0pt}%
\pgfpathmoveto{\pgfqpoint{3.929765in}{2.964626in}}%
\pgfpathlineto{\pgfqpoint{9.234911in}{1.090014in}}%
\pgfpathlineto{\pgfqpoint{1.188485in}{3.933276in}}%
\pgfpathlineto{\pgfqpoint{3.312913in}{3.182595in}}%
\pgfusepath{stroke}%
\end{pgfscope}%
\begin{pgfscope}%
\pgfsetrectcap%
\pgfsetmiterjoin%
\pgfsetlinewidth{0.803000pt}%
\definecolor{currentstroke}{rgb}{0.000000,0.000000,0.000000}%
\pgfsetstrokecolor{currentstroke}%
\pgfsetdash{}{0pt}%
\pgfpathmoveto{\pgfqpoint{0.786164in}{0.768110in}}%
\pgfpathlineto{\pgfqpoint{0.786164in}{7.850000in}}%
\pgfusepath{stroke}%
\end{pgfscope}%
\begin{pgfscope}%
\pgfsetrectcap%
\pgfsetmiterjoin%
\pgfsetlinewidth{0.803000pt}%
\definecolor{currentstroke}{rgb}{0.000000,0.000000,0.000000}%
\pgfsetstrokecolor{currentstroke}%
\pgfsetdash{}{0pt}%
\pgfpathmoveto{\pgfqpoint{9.637233in}{0.768110in}}%
\pgfpathlineto{\pgfqpoint{9.637233in}{7.850000in}}%
\pgfusepath{stroke}%
\end{pgfscope}%
\begin{pgfscope}%
\pgfsetrectcap%
\pgfsetmiterjoin%
\pgfsetlinewidth{0.803000pt}%
\definecolor{currentstroke}{rgb}{0.000000,0.000000,0.000000}%
\pgfsetstrokecolor{currentstroke}%
\pgfsetdash{}{0pt}%
\pgfpathmoveto{\pgfqpoint{0.786164in}{0.768110in}}%
\pgfpathlineto{\pgfqpoint{9.637233in}{0.768110in}}%
\pgfusepath{stroke}%
\end{pgfscope}%
\begin{pgfscope}%
\pgfsetrectcap%
\pgfsetmiterjoin%
\pgfsetlinewidth{0.803000pt}%
\definecolor{currentstroke}{rgb}{0.000000,0.000000,0.000000}%
\pgfsetstrokecolor{currentstroke}%
\pgfsetdash{}{0pt}%
\pgfpathmoveto{\pgfqpoint{0.786164in}{7.850000in}}%
\pgfpathlineto{\pgfqpoint{9.637233in}{7.850000in}}%
\pgfusepath{stroke}%
\end{pgfscope}%
\begin{pgfscope}%
\pgfsetbuttcap%
\pgfsetmiterjoin%
\definecolor{currentfill}{rgb}{1.000000,1.000000,1.000000}%
\pgfsetfillcolor{currentfill}%
\pgfsetfillopacity{0.800000}%
\pgfsetlinewidth{1.003750pt}%
\definecolor{currentstroke}{rgb}{0.800000,0.800000,0.800000}%
\pgfsetstrokecolor{currentstroke}%
\pgfsetstrokeopacity{0.800000}%
\pgfsetdash{}{0pt}%
\pgfpathmoveto{\pgfqpoint{7.498496in}{7.232821in}}%
\pgfpathlineto{\pgfqpoint{9.442788in}{7.232821in}}%
\pgfpathquadraticcurveto{\pgfqpoint{9.498344in}{7.232821in}}{\pgfqpoint{9.498344in}{7.288377in}}%
\pgfpathlineto{\pgfqpoint{9.498344in}{7.655556in}}%
\pgfpathquadraticcurveto{\pgfqpoint{9.498344in}{7.711111in}}{\pgfqpoint{9.442788in}{7.711111in}}%
\pgfpathlineto{\pgfqpoint{7.498496in}{7.711111in}}%
\pgfpathquadraticcurveto{\pgfqpoint{7.442940in}{7.711111in}}{\pgfqpoint{7.442940in}{7.655556in}}%
\pgfpathlineto{\pgfqpoint{7.442940in}{7.288377in}}%
\pgfpathquadraticcurveto{\pgfqpoint{7.442940in}{7.232821in}}{\pgfqpoint{7.498496in}{7.232821in}}%
\pgfpathlineto{\pgfqpoint{7.498496in}{7.232821in}}%
\pgfpathclose%
\pgfusepath{stroke,fill}%
\end{pgfscope}%
\begin{pgfscope}%
\pgfsetrectcap%
\pgfsetroundjoin%
\pgfsetlinewidth{1.505625pt}%
\definecolor{currentstroke}{rgb}{1.000000,0.000000,0.000000}%
\pgfsetstrokecolor{currentstroke}%
\pgfsetdash{}{0pt}%
\pgfpathmoveto{\pgfqpoint{7.554051in}{7.497184in}}%
\pgfpathlineto{\pgfqpoint{7.831829in}{7.497184in}}%
\pgfpathlineto{\pgfqpoint{8.109607in}{7.497184in}}%
\pgfusepath{stroke}%
\end{pgfscope}%
\begin{pgfscope}%
\definecolor{textcolor}{rgb}{0.000000,0.000000,0.000000}%
\pgfsetstrokecolor{textcolor}%
\pgfsetfillcolor{textcolor}%
\pgftext[x=8.331829in,y=7.399962in,left,base]{\color{textcolor}{\sffamily\fontsize{20.000000}{24.000000}\selectfont\catcode`\^=\active\def^{\ifmmode\sp\else\^{}\fi}\catcode`\%=\active\def%{\%}r = -0.54}}%
\end{pgfscope}%
\begin{pgfscope}%
\pgfsetbuttcap%
\pgfsetmiterjoin%
\definecolor{currentfill}{rgb}{1.000000,1.000000,1.000000}%
\pgfsetfillcolor{currentfill}%
\pgfsetlinewidth{0.000000pt}%
\definecolor{currentstroke}{rgb}{0.000000,0.000000,0.000000}%
\pgfsetstrokecolor{currentstroke}%
\pgfsetstrokeopacity{0.000000}%
\pgfsetdash{}{0pt}%
\pgfpathmoveto{\pgfqpoint{10.190425in}{0.768110in}}%
\pgfpathlineto{\pgfqpoint{10.544519in}{0.768110in}}%
\pgfpathlineto{\pgfqpoint{10.544519in}{7.850000in}}%
\pgfpathlineto{\pgfqpoint{10.190425in}{7.850000in}}%
\pgfpathlineto{\pgfqpoint{10.190425in}{0.768110in}}%
\pgfpathclose%
\pgfusepath{fill}%
\end{pgfscope}%
\begin{pgfscope}%
\pgfsys@transformshift{10.190000in}{0.770000in}%
\pgftext[left,bottom]{\includegraphics[interpolate=true,width=0.350000in,height=7.080000in]{plot_renewables_vs_emissions-img0.png}}%
\end{pgfscope}%
\begin{pgfscope}%
\pgfsetbuttcap%
\pgfsetroundjoin%
\definecolor{currentfill}{rgb}{0.000000,0.000000,0.000000}%
\pgfsetfillcolor{currentfill}%
\pgfsetlinewidth{0.803000pt}%
\definecolor{currentstroke}{rgb}{0.000000,0.000000,0.000000}%
\pgfsetstrokecolor{currentstroke}%
\pgfsetdash{}{0pt}%
\pgfsys@defobject{currentmarker}{\pgfqpoint{0.000000in}{0.000000in}}{\pgfqpoint{0.048611in}{0.000000in}}{%
\pgfpathmoveto{\pgfqpoint{0.000000in}{0.000000in}}%
\pgfpathlineto{\pgfqpoint{0.048611in}{0.000000in}}%
\pgfusepath{stroke,fill}%
}%
\begin{pgfscope}%
\pgfsys@transformshift{10.544519in}{1.218265in}%
\pgfsys@useobject{currentmarker}{}%
\end{pgfscope}%
\end{pgfscope}%
\begin{pgfscope}%
\definecolor{textcolor}{rgb}{0.000000,0.000000,0.000000}%
\pgfsetstrokecolor{textcolor}%
\pgfsetfillcolor{textcolor}%
\pgftext[x=10.641741in, y=1.148820in, left, base]{\color{textcolor}{\sffamily\fontsize{15.000000}{18.000000}\selectfont\catcode`\^=\active\def^{\ifmmode\sp\else\^{}\fi}\catcode`\%=\active\def%{\%}2}}%
\end{pgfscope}%
\begin{pgfscope}%
\pgfsetbuttcap%
\pgfsetroundjoin%
\definecolor{currentfill}{rgb}{0.000000,0.000000,0.000000}%
\pgfsetfillcolor{currentfill}%
\pgfsetlinewidth{0.803000pt}%
\definecolor{currentstroke}{rgb}{0.000000,0.000000,0.000000}%
\pgfsetstrokecolor{currentstroke}%
\pgfsetdash{}{0pt}%
\pgfsys@defobject{currentmarker}{\pgfqpoint{0.000000in}{0.000000in}}{\pgfqpoint{0.048611in}{0.000000in}}{%
\pgfpathmoveto{\pgfqpoint{0.000000in}{0.000000in}}%
\pgfpathlineto{\pgfqpoint{0.048611in}{0.000000in}}%
\pgfusepath{stroke,fill}%
}%
\begin{pgfscope}%
\pgfsys@transformshift{10.544519in}{2.022111in}%
\pgfsys@useobject{currentmarker}{}%
\end{pgfscope}%
\end{pgfscope}%
\begin{pgfscope}%
\definecolor{textcolor}{rgb}{0.000000,0.000000,0.000000}%
\pgfsetstrokecolor{textcolor}%
\pgfsetfillcolor{textcolor}%
\pgftext[x=10.641741in, y=1.952667in, left, base]{\color{textcolor}{\sffamily\fontsize{15.000000}{18.000000}\selectfont\catcode`\^=\active\def^{\ifmmode\sp\else\^{}\fi}\catcode`\%=\active\def%{\%}3}}%
\end{pgfscope}%
\begin{pgfscope}%
\pgfsetbuttcap%
\pgfsetroundjoin%
\definecolor{currentfill}{rgb}{0.000000,0.000000,0.000000}%
\pgfsetfillcolor{currentfill}%
\pgfsetlinewidth{0.803000pt}%
\definecolor{currentstroke}{rgb}{0.000000,0.000000,0.000000}%
\pgfsetstrokecolor{currentstroke}%
\pgfsetdash{}{0pt}%
\pgfsys@defobject{currentmarker}{\pgfqpoint{0.000000in}{0.000000in}}{\pgfqpoint{0.048611in}{0.000000in}}{%
\pgfpathmoveto{\pgfqpoint{0.000000in}{0.000000in}}%
\pgfpathlineto{\pgfqpoint{0.048611in}{0.000000in}}%
\pgfusepath{stroke,fill}%
}%
\begin{pgfscope}%
\pgfsys@transformshift{10.544519in}{2.825958in}%
\pgfsys@useobject{currentmarker}{}%
\end{pgfscope}%
\end{pgfscope}%
\begin{pgfscope}%
\definecolor{textcolor}{rgb}{0.000000,0.000000,0.000000}%
\pgfsetstrokecolor{textcolor}%
\pgfsetfillcolor{textcolor}%
\pgftext[x=10.641741in, y=2.756514in, left, base]{\color{textcolor}{\sffamily\fontsize{15.000000}{18.000000}\selectfont\catcode`\^=\active\def^{\ifmmode\sp\else\^{}\fi}\catcode`\%=\active\def%{\%}4}}%
\end{pgfscope}%
\begin{pgfscope}%
\pgfsetbuttcap%
\pgfsetroundjoin%
\definecolor{currentfill}{rgb}{0.000000,0.000000,0.000000}%
\pgfsetfillcolor{currentfill}%
\pgfsetlinewidth{0.803000pt}%
\definecolor{currentstroke}{rgb}{0.000000,0.000000,0.000000}%
\pgfsetstrokecolor{currentstroke}%
\pgfsetdash{}{0pt}%
\pgfsys@defobject{currentmarker}{\pgfqpoint{0.000000in}{0.000000in}}{\pgfqpoint{0.048611in}{0.000000in}}{%
\pgfpathmoveto{\pgfqpoint{0.000000in}{0.000000in}}%
\pgfpathlineto{\pgfqpoint{0.048611in}{0.000000in}}%
\pgfusepath{stroke,fill}%
}%
\begin{pgfscope}%
\pgfsys@transformshift{10.544519in}{3.629805in}%
\pgfsys@useobject{currentmarker}{}%
\end{pgfscope}%
\end{pgfscope}%
\begin{pgfscope}%
\definecolor{textcolor}{rgb}{0.000000,0.000000,0.000000}%
\pgfsetstrokecolor{textcolor}%
\pgfsetfillcolor{textcolor}%
\pgftext[x=10.641741in, y=3.560360in, left, base]{\color{textcolor}{\sffamily\fontsize{15.000000}{18.000000}\selectfont\catcode`\^=\active\def^{\ifmmode\sp\else\^{}\fi}\catcode`\%=\active\def%{\%}5}}%
\end{pgfscope}%
\begin{pgfscope}%
\pgfsetbuttcap%
\pgfsetroundjoin%
\definecolor{currentfill}{rgb}{0.000000,0.000000,0.000000}%
\pgfsetfillcolor{currentfill}%
\pgfsetlinewidth{0.803000pt}%
\definecolor{currentstroke}{rgb}{0.000000,0.000000,0.000000}%
\pgfsetstrokecolor{currentstroke}%
\pgfsetdash{}{0pt}%
\pgfsys@defobject{currentmarker}{\pgfqpoint{0.000000in}{0.000000in}}{\pgfqpoint{0.048611in}{0.000000in}}{%
\pgfpathmoveto{\pgfqpoint{0.000000in}{0.000000in}}%
\pgfpathlineto{\pgfqpoint{0.048611in}{0.000000in}}%
\pgfusepath{stroke,fill}%
}%
\begin{pgfscope}%
\pgfsys@transformshift{10.544519in}{4.433651in}%
\pgfsys@useobject{currentmarker}{}%
\end{pgfscope}%
\end{pgfscope}%
\begin{pgfscope}%
\definecolor{textcolor}{rgb}{0.000000,0.000000,0.000000}%
\pgfsetstrokecolor{textcolor}%
\pgfsetfillcolor{textcolor}%
\pgftext[x=10.641741in, y=4.364207in, left, base]{\color{textcolor}{\sffamily\fontsize{15.000000}{18.000000}\selectfont\catcode`\^=\active\def^{\ifmmode\sp\else\^{}\fi}\catcode`\%=\active\def%{\%}6}}%
\end{pgfscope}%
\begin{pgfscope}%
\pgfsetbuttcap%
\pgfsetroundjoin%
\definecolor{currentfill}{rgb}{0.000000,0.000000,0.000000}%
\pgfsetfillcolor{currentfill}%
\pgfsetlinewidth{0.803000pt}%
\definecolor{currentstroke}{rgb}{0.000000,0.000000,0.000000}%
\pgfsetstrokecolor{currentstroke}%
\pgfsetdash{}{0pt}%
\pgfsys@defobject{currentmarker}{\pgfqpoint{0.000000in}{0.000000in}}{\pgfqpoint{0.048611in}{0.000000in}}{%
\pgfpathmoveto{\pgfqpoint{0.000000in}{0.000000in}}%
\pgfpathlineto{\pgfqpoint{0.048611in}{0.000000in}}%
\pgfusepath{stroke,fill}%
}%
\begin{pgfscope}%
\pgfsys@transformshift{10.544519in}{5.237498in}%
\pgfsys@useobject{currentmarker}{}%
\end{pgfscope}%
\end{pgfscope}%
\begin{pgfscope}%
\definecolor{textcolor}{rgb}{0.000000,0.000000,0.000000}%
\pgfsetstrokecolor{textcolor}%
\pgfsetfillcolor{textcolor}%
\pgftext[x=10.641741in, y=5.168054in, left, base]{\color{textcolor}{\sffamily\fontsize{15.000000}{18.000000}\selectfont\catcode`\^=\active\def^{\ifmmode\sp\else\^{}\fi}\catcode`\%=\active\def%{\%}7}}%
\end{pgfscope}%
\begin{pgfscope}%
\pgfsetbuttcap%
\pgfsetroundjoin%
\definecolor{currentfill}{rgb}{0.000000,0.000000,0.000000}%
\pgfsetfillcolor{currentfill}%
\pgfsetlinewidth{0.803000pt}%
\definecolor{currentstroke}{rgb}{0.000000,0.000000,0.000000}%
\pgfsetstrokecolor{currentstroke}%
\pgfsetdash{}{0pt}%
\pgfsys@defobject{currentmarker}{\pgfqpoint{0.000000in}{0.000000in}}{\pgfqpoint{0.048611in}{0.000000in}}{%
\pgfpathmoveto{\pgfqpoint{0.000000in}{0.000000in}}%
\pgfpathlineto{\pgfqpoint{0.048611in}{0.000000in}}%
\pgfusepath{stroke,fill}%
}%
\begin{pgfscope}%
\pgfsys@transformshift{10.544519in}{6.041345in}%
\pgfsys@useobject{currentmarker}{}%
\end{pgfscope}%
\end{pgfscope}%
\begin{pgfscope}%
\definecolor{textcolor}{rgb}{0.000000,0.000000,0.000000}%
\pgfsetstrokecolor{textcolor}%
\pgfsetfillcolor{textcolor}%
\pgftext[x=10.641741in, y=5.971901in, left, base]{\color{textcolor}{\sffamily\fontsize{15.000000}{18.000000}\selectfont\catcode`\^=\active\def^{\ifmmode\sp\else\^{}\fi}\catcode`\%=\active\def%{\%}8}}%
\end{pgfscope}%
\begin{pgfscope}%
\pgfsetbuttcap%
\pgfsetroundjoin%
\definecolor{currentfill}{rgb}{0.000000,0.000000,0.000000}%
\pgfsetfillcolor{currentfill}%
\pgfsetlinewidth{0.803000pt}%
\definecolor{currentstroke}{rgb}{0.000000,0.000000,0.000000}%
\pgfsetstrokecolor{currentstroke}%
\pgfsetdash{}{0pt}%
\pgfsys@defobject{currentmarker}{\pgfqpoint{0.000000in}{0.000000in}}{\pgfqpoint{0.048611in}{0.000000in}}{%
\pgfpathmoveto{\pgfqpoint{0.000000in}{0.000000in}}%
\pgfpathlineto{\pgfqpoint{0.048611in}{0.000000in}}%
\pgfusepath{stroke,fill}%
}%
\begin{pgfscope}%
\pgfsys@transformshift{10.544519in}{6.845192in}%
\pgfsys@useobject{currentmarker}{}%
\end{pgfscope}%
\end{pgfscope}%
\begin{pgfscope}%
\definecolor{textcolor}{rgb}{0.000000,0.000000,0.000000}%
\pgfsetstrokecolor{textcolor}%
\pgfsetfillcolor{textcolor}%
\pgftext[x=10.641741in, y=6.775747in, left, base]{\color{textcolor}{\sffamily\fontsize{15.000000}{18.000000}\selectfont\catcode`\^=\active\def^{\ifmmode\sp\else\^{}\fi}\catcode`\%=\active\def%{\%}9}}%
\end{pgfscope}%
\begin{pgfscope}%
\pgfsetbuttcap%
\pgfsetroundjoin%
\definecolor{currentfill}{rgb}{0.000000,0.000000,0.000000}%
\pgfsetfillcolor{currentfill}%
\pgfsetlinewidth{0.803000pt}%
\definecolor{currentstroke}{rgb}{0.000000,0.000000,0.000000}%
\pgfsetstrokecolor{currentstroke}%
\pgfsetdash{}{0pt}%
\pgfsys@defobject{currentmarker}{\pgfqpoint{0.000000in}{0.000000in}}{\pgfqpoint{0.048611in}{0.000000in}}{%
\pgfpathmoveto{\pgfqpoint{0.000000in}{0.000000in}}%
\pgfpathlineto{\pgfqpoint{0.048611in}{0.000000in}}%
\pgfusepath{stroke,fill}%
}%
\begin{pgfscope}%
\pgfsys@transformshift{10.544519in}{7.649038in}%
\pgfsys@useobject{currentmarker}{}%
\end{pgfscope}%
\end{pgfscope}%
\begin{pgfscope}%
\definecolor{textcolor}{rgb}{0.000000,0.000000,0.000000}%
\pgfsetstrokecolor{textcolor}%
\pgfsetfillcolor{textcolor}%
\pgftext[x=10.641741in, y=7.579594in, left, base]{\color{textcolor}{\sffamily\fontsize{15.000000}{18.000000}\selectfont\catcode`\^=\active\def^{\ifmmode\sp\else\^{}\fi}\catcode`\%=\active\def%{\%}10}}%
\end{pgfscope}%
\begin{pgfscope}%
\definecolor{textcolor}{rgb}{0.000000,0.000000,0.000000}%
\pgfsetstrokecolor{textcolor}%
\pgfsetfillcolor{textcolor}%
\pgftext[x=10.893128in,y=4.309055in,,top,rotate=90.000000]{\color{textcolor}{\sffamily\fontsize{20.000000}{24.000000}\selectfont\catcode`\^=\active\def^{\ifmmode\sp\else\^{}\fi}\catcode`\%=\active\def%{\%}Energy Consumption (tonnes of oil equivalent per capita)}}%
\end{pgfscope}%
\begin{pgfscope}%
\pgfsetrectcap%
\pgfsetmiterjoin%
\pgfsetlinewidth{0.803000pt}%
\definecolor{currentstroke}{rgb}{0.000000,0.000000,0.000000}%
\pgfsetstrokecolor{currentstroke}%
\pgfsetdash{}{0pt}%
\pgfpathmoveto{\pgfqpoint{10.190425in}{0.768110in}}%
\pgfpathlineto{\pgfqpoint{10.367472in}{0.768110in}}%
\pgfpathlineto{\pgfqpoint{10.544519in}{0.768110in}}%
\pgfpathlineto{\pgfqpoint{10.544519in}{7.850000in}}%
\pgfpathlineto{\pgfqpoint{10.367472in}{7.850000in}}%
\pgfpathlineto{\pgfqpoint{10.190425in}{7.850000in}}%
\pgfpathlineto{\pgfqpoint{10.190425in}{0.768110in}}%
\pgfpathclose%
\pgfusepath{stroke}%
\end{pgfscope}%
\end{pgfpicture}%
\makeatother%
\endgroup%
}
    \caption{Global Correlation between Share of Renewable Energy and Greenhouse Gas Emissions}
    \label{plt:global_share_vs_emissions}
\end{figure}

\subsubsection*{Interpretation of the Findings}
The analysis of the relationship between the share of renewable energy and greenhouse gas emissions indicates a significant negative correlation.
This suggests that increasing the share of renewable energy in the energy mix is generally associated with a reduction in greenhouse gas
emissions at the EU level.

When examining individual countries, the median PCC is even stronger at -.9. This high median value implies that most countries experience
substantial reductions in greenhouse gas emissions as they increase their share of renewable energy.

\section*{Conclusions}
\subsubsection*{How does the amount of energy consumed influence the net greenhouse gas emissions of European countries?}
The analysis strongly indicates, that higher energy consumption leads to higher greenhouse gas emissions.
To what extend this effect can be seen differs between countries, though overall the correlation is very strong.

\subsubsection*{And how is this influenced by the share of renewables in total energy?}
The significant negative correlation between share of renewable energy sources and emissions strongly indicates, that emissions are also strongly
dependent on the burning fossil fuels.
This suggests that most countries see substantial emission reductions as they increase their renewable energy share.

\subsection*{Critical Reflection}
The questions were answered effectively, showing that higher energy consumption tends to increase emissions,
while a higher share of renewables tends to decrease emissions. However, limitations include:
\begin{enumerate}
    \item Data Accuracy: Results depend on the quality of the data used.
    \item National Variations: Further research is needed to explore factors driving national differences.
    \item Causality: The analysis shows correlation, not causation.
    \item Temporal Dynamics: Aggregated data may hide trends over time.
    \item External Influences: Factors like technological advances and global policies were not considered.
\end{enumerate}

\end{document}
